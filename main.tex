\PassOptionsToPackage{german,english}{babel}
\documentclass[output=book,%nobabel,
colorlinks,citecolor=brown
,collection
% draft,
% draftmode
		  ]{./langscibook}

\input{localmetadata.tex}
\usepackage{tabularx} 

\usepackage{./langsci-optional}
\usepackage{./langsci-gb4e}
\usepackage{./langsci-lgr}
\usepackage[perpage,para,symbol*]{footmisc}
\usepackage{nccfoots}

\usepackage[german,english]{babel}

\input{localhyphenation.tex}
\bibliography{localbibliography} 

%%%%%%%%%%%%%%%%%%%%%%%%%%%%%%%%%%%%%%%%%%%%%%%%%%%%
%%%             Frontmatter                      %%%
%%%%%%%%%%%%%%%%%%%%%%%%%%%%%%%%%%%%%%%%%%%%%%%%%%%% 
\begin{document}     
\input{localcommands.tex} 
 

\maketitle                
\frontmatter
\currentpdfbookmark{Contents}{name} % adds a PDF bookmark
\tableofcontents
\mainmatter         
  
%
% Besonders beachtenswert ist nun aber, dass diese Stellungsgewohnheit oft bei Homer und fast noch häufiger bei Herodot (vgl. Stein zu 1, 115, 8) dazu geführt hat, dem οἱ eine dem syntaktischen Zusammenhang widersprechende oder in andrer Hinsicht auffällige Stellung anzuweisen.
%
% 1)	Entschieden dativisches οἱ steht von dem regierenden Worte weit ab und drängt sich mitten in eine am Satzanfang stehende sonstige Wortgruppe ein. Ρ~232 τὸ δέ \spation{οἱ} κλέοϲ \spation{ἔϲϲεται} ὅϲϲον ἐμοί περ. γ~306 τῷ δέ οἱ ὀγδοάτῳ κακὸν \spation{ἤλυθε} δῖοϲ Ὀρέϲτηϲ. — Herodot~1, 75, 10 Θαλῆϲ \spation{οἱ} ὁ Μιλήϲιοϲ \spation{διεβίβαϲε}. 1, 199, 14 ἤ τίϲ \spation{οἱ} ξείνων ἀργύριον ἐμβαλὼν ἐϲ τὰ γούνατα \spation{μιχθῇ} (τίϲ geht dem οἱ voran, weil es selbst ein Enklitikum ist). 2, 108, 4 τούϲ τέ \spation{οἱ} λίθουϲ (folgen 14 Worte) οὗτοι ἦϲαν οἱ \spation{ἑλκύϲαντεϲ}. 4, 45, 19 ὅϲτιϲ \spation{οἱ} ἦν ὁ \spation{θέμενοϲ} (scil. τοὔνομα). 5, 92, β~8 ἐκ δέ \spation{οἱ} ταύτηϲ τῆϲ γυναίκοϲ οὐδ᾽ ἐξ ἄλληϲ παῖδεϲ \spation{ἐγίνοντο}. 6, 63, 2 ἐν δέ \spation{οἱ} χρόνῳ ἐλάϲϲονι ἡ γυνὴ \spation{τίκτει} τούτον. 7, 5, 14 οὗτοϲ μέν \spation{οἱ} ὁ λόγοϲ \spation{ἦν} \spation{τιμωρόϲ}.
%
% \rule{\textwidth}{2pt}
\renewcommand\spationcolor{\color{black}}
%
% Besonders beachtenswert ist nun aber, dass diese Stellungsgewohnheit oft bei Homer und fast noch häufiger bei Herodot (vgl. Stein zu 1, 115, 8) dazu geführt hat, dem οἱ eine dem syntaktischen Zusammenhang widersprechende oder in andrer Hinsicht auffällige Stellung anzuweisen.
%
% 1)	Entschieden dativisches οἱ steht von dem regierenden Worte weit ab und drängt sich mitten in eine am Satzanfang stehende sonstige Wortgruppe ein. Ρ~232 τὸ δέ \spation{οἱ} κλέοϲ \spation{ἔϲϲεται} ὅϲϲον ἐμοί περ. γ~306 τῷ δέ οἱ ὀγδοάτῳ κακὸν \spation{ἤλυθε} δῖοϲ Ὀρέϲτηϲ. — Herodot~1, 75, 10 Θαλῆϲ \spation{οἱ} ὁ Μιλήϲιοϲ \spation{διεβίβαϲε}. 1, 199, 14 ἤ τίϲ \spation{οἱ} ξείνων ἀργύριον ἐμβαλὼν ἐϲ τὰ γούνατα \spation{μιχθῇ} (τίϲ geht dem οἱ voran, weil es selbst ein Enklitikum ist). 2, 108, 4 τούϲ τέ \spation{οἱ} λίθουϲ (folgen 14 Worte) οὗτοι ἦϲαν οἱ \spation{ἑλκύϲαντεϲ}. 4, 45, 19 ὅϲτιϲ \spation{οἱ} ἦν ὁ \spation{θέμενοϲ} (scil. τοὔνομα). 5, 92, β~8 ἐκ δέ \spation{οἱ} ταύτηϲ τῆϲ γυναίκοϲ οὐδ᾽ ἐξ ἄλληϲ παῖδεϲ \spation{ἐγίνοντο}. 6, 63, 2 ἐν δέ \spation{οἱ} χρόνῳ ἐλάϲϲονι ἡ γυνὴ \spation{τίκτει} τούτον. 7, 5, 14 οὗτοϲ μέν \spation{οἱ} ὁ λόγοϲ \spation{ἦν} \spation{τιμωρόϲ}.

\part{}
\includepaper{chapters/01}
\part{}
\include{chapters/02}
\include{chapters/03}

%%%%%%%%%%%%%%%%%%%%%%%%%%%%%%%%%%%%%%%%%%%%%%%%%%%%
%%%             Backmatter                       %%%
%%%%%%%%%%%%%%%%%%%%%%%%%%%%%%%%%%%%%%%%%%%%%%%%%%%%
\sloppy
% \input{localseealso.tex}
\input{backmatter.tex} 
\end{document}

\chapter{Translation}

\section{Greek \emph{min}, \emph{nin} and \emph{hoi}}\il{Greek|(}

\hyperlink{p333}{\emph{[p333]}}\label{s333} Four years ago, Albert \citet{Thumb1887} made the claim that the Greek pro\-nom\-i\-nal\is{pronouns} accusatives\is{accusative} \emph{min} and \emph{nin} (\textsc{3.acc}) arose through merger of \isi{particles} with the old \isi{accusative} of the pronominal\is{pronouns} \emph{i}-stems.\Footnote{1}{For the collections of examples in what follows I owe a lot to the well-known reference works on Greek grammar, as well as to the specialized dictionaries, though I will not always be able to acknowledge my sources of information individually. I could only briefly make use of Monro's \citeyearpar{Monro1891} \emph{Grammar of the Homeric\il{Greek, Homeric} Dialect}, second edition, pages 335--338 of which contain observations on Homeric word order that accord closely with what I present here, and I was not able to use Gehring's \citeyearpar{Gehring1891} \emph{Index Homericus} at all.} In particular, he claimed that Ionic\il{Greek, Ionic} \emph{min} was based on the unification of \emph{im} `me.\textsc{acc}' with a particle\is{particles} \emph{ma}, earlier \emph{sma}, evidenced by Thessalonian\il{Greek, Thessalonian} \emph{ma} and \ili{Sanskrit} \emph{sma} `indeed, certainly'. Thumb's main piece of evidence for this interpretation came from the supposed fact that the position of \emph{min} in Homer is essentially the same as the position of \emph{sma} in the \textit{\d{R}gveda}. Even after the independent use of \emph{sma} as a particle\is{particles} was lost and \emph{min} had completely reached the status of a unitary pronominal\is{pronouns} form, the same rule that had regulated the position of \emph{sma} still held for \emph{min}, and a corresponding sense for positioning accompanied its use. And at any rate this sense was still valid for the composers of the Homeric poems.\is{poetry}

However, if one looks at the material adduced by Thumb without limiting oneself to the perspective he proposes, this positional similarity is largely limited to the fact that \emph{min}, like \emph{sma}, in general rarely occurs directly after nouns or after \isi{adverbs} of nominal origin (to be precise, \emph{min} is much rarer in this position than \emph{sma}). And there are significant deviations from this general banal similarity. Thumb\ia{Thumb, Albert} makes a strange error in not being able to dig up any instances of \ili{Sanskrit} \emph{mā sma}, which under his hypothesis would correspond to the ten instances of \emph{mē min} in Homer: \hyperlink{p334}{\emph{[p334]}} not only does Böhtlingk-Roth \citeyearpar{BoehtlingkRoth1855} give numerous examples (s.v. \emph{mā} 9), including one from the \textit{\d{R}gveda} ((\ref{RV_ex1})), but there is also a well-known rule of \ili{Sanskrit} grammar regarding the meaning and form of preterites\is{preterite} after \emph{mā sma} (\iai{Panini} 3. 3. 176. 6, 4, 74. See \citealp[361, §808 note 4]{Benfey1852}).

\begin{exe}
\ex \gll mā \emph{sma}itādṛg apa gūhaḥ samarye \\
\textsc{prohib} \textsc{emph}-such.like away hide.\textsc{2.sg.pres} clash.\textsc{sg.loc} \\
\trans `Don’t hide away such a thing in the clash.'
(\textit{\d{R}gveda}, 10.27.24b; trans. \citealt[1417]{JamisonBrereton2014})
\label{RV_ex1}
\end{exe}

But in other cases there is a genuine divergence between \emph{min} and \emph{sma}. According to Thumb, \emph{min} is found in Homer after subordinating\is{subordination} \isi{particles} about 60 times (10\% of all examples); \emph{sma} is found only rarely in this environment in the \textit{\d{R}gveda}, and only after \emph{yathā} `so, thus'. And while \emph{sma} is happy to occur after \isi{prepositions}, \emph{min} is never found here.

Admittedly, Thumb wants to explain this deviation with reference to the fact that the Homeric\il{Greek, Homeric} language is not fond of inserting \isi{particles} between \isi{prepositions} and nominals. He even makes the bold claim that with this in mind the deviation comes close to supporting his theory. I freely admit that I do not understand this explanation. When \emph{sma} follows a preposition\is{prepositions} in the \textit{\d{R}gveda}, the preposition\is{prepositions} is either verbal in \isi{tmesis} (including for instance (\ref{RV_ex2}), cf. \citet[1598]{Grassmann1873}) or, if cases of this second kind are attested at all, in `\isi{anastrophe}'.

\begin{exe}
\ex \gll ā \emph{smā} rathaṃ vṛṣapāṇeṣu tiṣṭhasi \\
\textsc{pv} \textsc{emph} chariot.\textsc{sg.acc} bull-drink.\textsc{pl.loc} mount.\textsc{2.sg.pres.} \\
\trans `You mount the chariot to the bullish drinks' 
(\textit{\d{R}gveda}, 1.51.12a; trans. \citealt[1417]{JamisonBrereton2014})
\label{RV_ex2}
\end{exe}% NOTE(George+Self): the p(re)v(erb) is part of the "mount" verb in-situ. 

If \emph{min} shares the usual position of \emph{sma}, then, we should not expect to find it after \isi{prepositions} associated with a case, and when it is absent here we should not excuse this by means of an apparent Homeric\il{Greek, Homeric} aversion to infixation\is{infixes} of particles:\is{particles} we should expect it to occur after independent \isi{prepositions}, and if we find that it is absent here we should recognize this as counterevidence to Thumb's\ia{Thumb, Albert} proposal.

But even if we disregard these differences (as well as others that could be mentioned) between the placement of Homeric\il{Greek, Homeric} \emph{min} and Vedic \emph{sma}, in my view Thumb should have felt obliged to investigate whether the position of \emph{min} in the Homeric clause could not also be explained from a different perspective, without reference to the quality of the preceding word, and whether similar positional regularities to those found with \emph{min} could not also be found with other words (e.g. those that are related in meaning \hyperlink{p335}{\emph{[p335]}} or similar in form) for which no connection with \emph{sma} is conceivable.

In this connection it is worth observing that of the nine `isolated' cases in which \emph{min} follows a nominal adverb,\is{adverbs} five\il{Greek, Homeric|(} (Ε 181, Ζ 173, Λ 479, Ο 160, and δ 500) have it in second position of the clause, and furthermore that all the examples Thumb gives of \emph{min} following verbs, demonstratives or \isi{negation} show the same. In light of this positional rule it also becomes clear why \emph{min} occurs so freely after particles,\is{particles} particularly subordinating\is{subordination} particles,\is{particles} in contrast to \emph{sma}, as well as why it essentially only immediately follows \isi{pronouns} when they serve a clause-linking role and hence appear at the beginning of the clause.

Alternatively, counting from another point of view, the books Ν, Π and Ρ, which together comprise 2,465 verses and so provide a good basis for conclusions about the language of the oldest part of the Iliad, yield instances of \emph{min} in the following positions: 21 times as second word in the clause; 28 times as third or fourth word, but separated from the first word of the clause only by an enclitic\is{enclitics} or an enclitic-like particle\is{particles} such as \emph{de} `but, and' or \emph{gar} `for, since'. In addition, we have \emph{ei kai min} (`if and him.\textsc{cl.masc.acc}'; Ν 58) and \emph{touneka kai min} (`therefore and him.\textsc{cl.masc.acc}'; Ν 432), where \emph{kai} `and' belongs closely with the first word of the clause; \emph{epei ou min} (`because not him.\textsc{cl.masc.acc}'; Ρ 641), for which the tendency of \isi{negation} to precede \isi{enclitics} in the same clause must be taken into account (for the moment, compare \emph{outis} `no one', \emph{oupō} `not yet, not at all', \emph{ou pote} `never', and \emph{ouk an} `not if');\label{ouk1} and finally (\ref{min1}).

\begin{exe}
\ex οὐδ᾽ εἰ μάλα μιν χόλοϲ ἵκοι\\
\gll oud' ei mála \emph{min} khólos híkoi\\
nor if very \textsc{3.acc} anger.\textsc{nom} come-upon.\textsc{3sg.prs.opt}\\
\trans `even if great anger came over them' (Homer, \emph{Iliad} 17.399)
\label{min1}
\end{exe}

We thus have 49 cases that obey the aforementioned rule precisely; three cases that are amenable to specific explanations; and only one genuine exception. (From the other books, \citet[337f.]{Monro1891} gives only \emph{oud' ebalon min} (`but.not strike.\textsc{1sg.aor.ind.act} him'; Γ 368), (\ref{min2}), in which he thinks that \emph{min} should be deleted, and (\ref{min3}).

\begin{exe}
\ex εἴ περ γὰρ φθάμενόϲ {μιν} ἢ οὐτάϲῃ\\
\gll eí per gàr phthámenós \emph{min} ḕ outásēi\\
if \textsc{emph} for arrive.\textsc{ptcp.aor.mid.m.nom.sg} \textsc{3.acc} or wound.\textsc{3sg.aor.sbjv} \\
\trans `though the man be beforehand with her and smite her' (Homer, \emph{Iliad} 21.576)
\label{min2}
\end{exe}

\begin{exe}
\ex ἀλλ᾽ ἐῶμέν {μιν} πρῶτα παρεξελθεῖν πεδίοιο\\
\gll all' eômén \emph{min} prôta parexeltheîn pedíoio\\
but allow.\textsc{1pl.prs.sbjv} \textsc{3.acc} firstly
pass.\textsc{aor.inf} plain.\textsc{gen.sg}\\
\trans `But let us suffer him at the first to pass by us on the plain' (Homer, \emph{Iliad} 10.344)
\label{min3}
\end{exe}

All of this is in verse,\is{poetry} i.e. under conditions that make it more difficult to keep to the usual word order. Particularly remarkable is the well-known, frequently-occurring phrase (\ref{min4a}) in place of (\ref{min4b}); here the pressure to put \emph{min} in second position is clearly enough in effect. Similarly in the common expression in (\ref{min5}), where \emph{min} belongs to \emph{prosēuda} and not to \emph{phōnēsas}. 

\begin{exe}
\ex\begin{xlist}
\ex\label{min4a} τῷ {μιν} ἐειϲάμενοϲ προϲέφη / προϲέφώνεε\\
\gll tôi \emph{min} eeisámenos proséphē/prosephṓnee\\
him.\textsc{dat} \textsc{3.acc}
appear.\textsc{ptcp.aor.mid.m.nom.sg} address.\textsc{3sg.imp}\\
\trans `In his likeness addressed ...' (Homer, \emph{Iliad} 17.326)
\ex\label{min4b} τῷ ἐειϲάμενοϲ προϲέφη μιν
\end{xlist}
\end{exe}

\begin{exe}
\ex καί {μιν} φωνήϲαϲ ἔπεα πτερόεντα προϲηύδα\\
\gll kaí \emph{min} phōnḗsas épea pteróenta prosēúda\\
and \textsc{3.acc} produce-a-sound.\textsc{ptcp.aor.m.nom.sg} word.\textsc{acc.pl} winged.\textsc{n.acc.pl} address.\textsc{3sg.imp}\\
\trans `and addressed him loudly with winged words' (Homer, \emph{Odyssey} 8.407)
\label{min5}
\end{exe}

In addition, observe (\ref{phi347}). \hyperlink{p336}{\emph{[p336]}} Here the pronoun\is{pronouns} that belongs to the subordinate\is{subordination} clause is moved to the main clause, without this being attributable to `prolepsis', as the verb of the main clause would require the \isi{dative}. Only the pressure towards sentence-initial position can explain the position of \emph{min}.

\begin{exe}
\ex χαίρει δέ μιν ὅϲτιϲ ἐθείρῃ\\
\gll khaírei dé \emph{min} hóstis etheírēi\\
rejoice.\textsc{3sg.imp} but \textsc{3.acc} who.\textsc{m.nom.sg} prepare.\textsc{3sg.aor}\\
\trans `and glad is he that prepared it (the field)' (Homer, \emph{Iliad} 21.347)\footnote{\emph{Translator's note}: The modern Perseus edition has \textit{hós tis} rather than \textit{hóstis}.}
\label{phi347}
\end{exe}\il{Greek, Homeric|)}

For the post-Homeric\il{Greek, Classical|(} use of \emph{min}, Herodotus plays the role of primary witness, and, in addition to my sporadic reading across all books, his seventh book provided me with the necessary material. And here I can at least say that the majority of examples show \emph{min} in second or near-second position, including such typical cases as (\ref{HerodotusMin1}) (in which \emph{min} belongs to the \isi{participles}), (\ref{HerodotusMin2}) (in which \emph{min} belongs only to \emph{anēke}), (\ref{HerodotusMin3}) and (\ref{HerodotusMin4}). Cf. also (\ref{KallinosMin}), where I would like to add that the elegiac\is{elegiac poets} poets\is{poetry} up to and including Theognis used \emph{min} 12 times in second position and only once (Theognis 195) in third position.

\begin{exe}
\ex πολλά τε γάρ {μιν} καὶ μεγάλα τὰ ἐπαείροντα καὶ ἐποτρύνοντα ἦν\\
\gll pollá te gár \emph{min} kaì megála tà epaeíronta kaì epotrúnonta ên\\
many.\textsc{n.acc.pl} and for \textsc{3.acc} and great.\textsc{n.acc.pl} the.\textsc{n.acc.pl} choose.\textsc{ptcp.prs.n.acc.pl} and urge.\textsc{ptcp.prs.n.acc.pl} be.\textsc{3pl.imp}\\
\trans `For there were many weighty reasons that impelled and encouraged him to do so' (Herodotus, 1.204.7)
\label{HerodotusMin1}
\end{exe}

\begin{exe}
\ex ὥϲ {μιν} ὅ τε οἶνοϲ ἀνῆκε καὶ ἔμαθε ...\\
\gll hṓs \emph{min} hó te oînos anêke kaì émathe\\
when \textsc{3.acc} the.\textsc{m.nom.sg} and wine.\textsc{nom.sg} let-go.\textsc{3sg.aor} and learn.\textsc{3sg.aor}\\
\trans `after the wine wore off and he recognized (...)' (Herodotus, 1.213.3)
\label{HerodotusMin2}
\end{exe}

\begin{exe}
\ex ἀλλά {μιν} οἱ ἱρεεϲ αὐτοὶ οἱ τοῦ Νείλου ... θάπτουϲι\\
\gll allá \emph{min} hoi hirees autoì hoi toû Neílou tháptousi\\
but \textsc{3.acc} the.\textsc{m.nom.pl} priests.\textsc{nom.pl} themselves.\textsc{m.nom.pl} the.\textsc{m.nom.pl} the.\textsc{m.gen.sg} Nile.\textsc{gen} bury.\textsc{3pl.prs}\\
\trans `But the priests of the Nile themselves buried him' (Herodotus, 2.90.7)
\label{HerodotusMin3}
\end{exe}

\begin{exe}
\ex οἱ γάρ {μιν} Σελινούϲιοι ἐπαναϲτάντεϲ ἀπέκτειναν καταφυγόντα ἐπὶ Διὸϲ ἀγοραίου βωμόν\\
\gll hoi gár \emph{min} Selinoúsioi epanastántes apékteinan kataphugónta epì Diòs agoraíou bōmón\\
the.\textsc{m.nom.pl} for \textsc{3.acc} Selinusian.\textsc{nom.pl} rise.\textsc{ptcp.aor.m.nom.pl} kill.\textsc{3sg.aor} fleeing.\textsc{ptcp.aor.m.acc.sg} upon Zeus.\textsc{gen} market.\textsc{gen.sg} altar.\textsc{acc.sg}\\
\trans `since the people of Selinus rose against him and slew him at the altar of Zeus of the marketplace, to which he had fled for refuge' (Herodotus, 5.46.2)
\label{HerodotusMin4}
\end{exe}

\begin{exe}
\ex ὥϲπερ γάρ {μιν} πύργον ἐν ὀφθαλμοῖϲιν ὁρῶϲιν\\
\gll hṓsper gár \emph{min} púrgon en ophthalmoîsin horôsin\\
thus for \textsc{3.acc} tower.\textsc{acc.sg} in eyes.\textsc{dat.pl} see.\textsc{3pl.prs}\\
\trans `For they see him in their (own) eyes as the tower' (Kallinos, 1.20)
\label{KallinosMin}
\end{exe}\il{Greek, Classical|)}

And it can be shown that this pressure towards initial position for \emph{min} is not based on some etymological relationship by looking at the very similar treatment of the enclitic\is{enclitics} \isi{dative} \emph{hoi} (\textsc{3.dat}), which is very close to the \isi{accusative} \emph{min} (\textsc{3sg.acc}) in meaning and accent, but differs entirely in pronunciation. In the Ν, Π and Ρ books of the Iliad, this \emph{hoi} is found 92 times. Of these, 34 instances are in second position; 53 are in third or fourth position, but separated from the first word of the clause by one or two words which have even greater claim to the clausal second position, such as \emph{de} `but, and' or \emph{te} or \emph{ke} `and'. Only five instances differ: Π 251 \emph{nēōn men hoi} `ships.\textsc{gen.pl} then \textsc{3.dat}' and Ρ 273 \emph{tōi kai hoi} `therefore and \textsc{3.dat}', where \emph{men} and \emph{kai} belong closely to the first word of the clause, and also Ρ 153 \emph{nun d' ou hoi} `now then not \textsc{3.dat}' and Ρ 410 \emph{dē tote g' ou hoi} `exactly then at.least not \textsc{3.dat}', which follow the rule that when \isi{negation} and enclitic\is{enclitics} are adjacent the \isi{negation} must precede.\label{ouk2} This would also explain (\ref{oi1}), \hyperlink{p337}{\emph{[p337]}} if the inseparability of \emph{ei} and \emph{mē} did not already offer a satisfactory explanation. It is therefore justifiable to state that the rule established for \emph{min} also holds for \emph{hoi}.

\begin{exe}
\ex εἰ μή οἱ ἀγάϲϲατο Φοῖβοϲ Ἀπόλλων\\
\gll ei mḗ \emph{hoi} agássato Phoîbos Apóllōn\\
if not \textsc{3.dat} envy.\textsc{3sg.aor} Phoebus.\textsc{nom} Apollo.\textsc{nom}\\
\trans `but that Phoebus Apollo begrudged it him' (Homer, \emph{Iliad} 17.71)
\label{oi1}
\end{exe}

This analogy between \emph{min} and \emph{hoi} is continued in Herodotus. In his writings, \emph{hoi} is found roughly twice as often in second or almost-second position as in other positions. (In the works of the older elegiac\is{elegiac poets} poets,\is{poetry} \emph{hoi} appears only to be found in second position.)

Particularly remarkable, however, is the fact that, often in Homer and almost even more frequently in Herodotus (cf. \citealp[138]{Stein1883} on 1.115.8), this positional tendency has often led to \emph{hoi} being assigned a position that contradicts the syntactic context or is unusual in another respect.\label{oi}

1) Distinctively \isi{dative} \emph{hoi} occurs far from its governing word and intervenes in another group of words at the beginning of the clause: (\ref{oi2})--(\ref{oi10}). (In (\ref{oi5}), \emph{tis} precedes \emph{hoi} because it is itself an enclitic.)\is{enclitics}

\begin{exe}
\ex τὸ δέ {οἱ} κλέοϲ {ἔϲϲεται} ὅϲϲον ἐμοί περ\\
\gll tò dé \emph{hoi} kléos {éssetai} hósson emoí per\\
the.\textsc{n.nom.sg} but \textsc{3.dat} glory.\textsc{nom.sg} be.\textsc{3sg.fut.mid} as-much.\textsc{n.acc.sg} me.\textsc{dat} \textsc{emph}\\
\trans `and his glory shall be even as mine own' (Homer, \emph{Iliad} 17.232)
\label{oi2}
\end{exe}

\begin{exe}
\ex τῷ δέ {οἱ} ὀγδοάτῳ κακὸν {ἤλυθε} δῖοϲ Ὀρέϲτεϲ\\
\gll tôi dé \emph{hoi} ogdoátōi kakòn {ḗluthe} dîos Oréstes\\
the.\textsc{n.dat.sg} but \textsc{3.dat} eighth.\textsc{n.dat.sg} bad.\textsc{acc.sg} come.\textsc{3sg.aor} divine.\textsc{m.nom.sg} Orestes.\textsc{nom}\\
\trans `but in the eighth came as his bane the godly Orestes' (Homer, \emph{Odyssey} 3.307)
\label{oi3}
\end{exe}

\begin{exe}
\ex Θαλῆϲ {οἱ} ὁ Μιλήϲιοϲ {διεβίβαϲε}\\
\gll Thalês \emph{hoi} ho Milḗsios {diebíbase}\\
Thales.\textsc{nom} \textsc{3.dat} the.\textsc{m.nom.sg} of-Miletus.\textsc{m.nom.sg} carry-over.\textsc{3sg.aor}\\
\trans `Thales of Miletus carried them (the army) across' (Herodotus 1.75.3)
\label{oi4}
\end{exe}

\begin{exe}
\ex ἤ τίϲ {οἱ} ξείνων ἀργύριον ἐμβαλὼν ἐϲ τὰ γούνατα {μιχθῇ}\\
\gll ḗ tís \emph{hoi} xeínōn argúrion embalṑn es tà goúnata mikhthêi\\
or some.\textsc{m.nom.sg} \textsc{3.dat} stranger.\textsc{gen.pl} money.\textsc{acc.sg} place.\textsc{ptcp.aor.m.com.sg} in the.\textsc{n.acc.pl} knees.\textsc{n.acc.pl} mix-up.\textsc{3sg.aor.sbjv.pass}\\
\trans `before some stranger has cast money into her lap (and) has united with her' (Herodotus 1.199.3)
\label{oi5}
\end{exe}

\begin{exe}
\ex τούϲ τέ {οἱ} λίθουϲ ... οὗτοι ἦϲαν οἱ ἑλκύϲαντεϲ\\
\gll toús té \emph{hoi} líthous hoûtoi êsan {hoi} helkúsantes\\
the.\textsc{m.acc.pl} and \textsc{3.dat} stone.\textsc{acc.pl} this.\textsc{m.nom.pl} be.\textsc{3pl.imp} the.\textsc{m.nom.pl} drag.\textsc{ptcp.aor.m.nom.pl}\\
\trans `It was these who dragged the ... blocks of stone' (Herodotus 2.108.2)
\label{oi6}
\end{exe}

\begin{exe}
\ex οὔτε ὅϲτιϲ {οἱ} ἦν ὁ {θέμενοϲ} {[}τοὔνομα{]} φαίνεται\\
\gll oúte hóstis \emph{hoi} ên ho {thémenos} toúnoma phaínetai\\
nor who.\textsc{m.nom.sg} \textsc{3.dat} be.\textsc{3sg.imp} the.\textsc{m.nom.sg} put.\textsc{ptcp.aor.mid.m.nom.sg} the=name.\textsc{acc.sg} seem.\textsc{3sg.prs}\\
\trans `nor is it clear who gave her {[}the name{]}' (Herodotus 4.45.4)
\label{oi7}
\end{exe}

\begin{exe}
\ex ἐκ δέ {οἱ} ταύτηϲ τῆϲ γυναίκοϲ οὐδ᾽ ἐξ ἄλληϲ παῖδεϲ {ἐγίνοντο}\\
\gll ek dé \emph{hoi} taútēs tês gunaíkos oud' ex állēs paîdes {egínonto}\\
from but \textsc{3.dat} this.\textsc{f.gen.sg} the.\textsc{gen.sg} woman.\textsc{gen.sg} nor from other.\textsc{f.gen.sg} child.\textsc{nom.pl} be.born.\textsc{3pl.imp}\\
\trans `no sons were born to him by this wife or any other' (Herodotus 5.92B.2)
\label{oi8}
\end{exe}

\begin{exe}
\ex ἐν δέ {οἱ} χρόνῳ ἐλάϲϲονι ἡ γυνὴ {τίκτει} τούτον\\
\gll en dé \emph{hoi} khrónōi elássoni hē gunḕ tíktei toúton\\
in but \textsc{3.dat} time.\textsc{dat.sg} less.\textsc{m.dat.sg} the.\textsc{f.nom.sg} woman.\textsc{nom.sg} birth.\textsc{3sg.prs} this.\textsc{m.acc.sg}\\
\trans `His [new] wife gave birth to him in less time' (Herodotus 6.63.1)
\label{oi9}
\end{exe}

\begin{exe}
\ex οὗτοϲ μέν {οἱ} ὁ λόγοϲ ἦν {τιμωρόϲ}\\
\gll hoûtos mén \emph{hoi} ho lógos ên timōrós\\
this.\textsc{m.nom.sg} indeed \textsc{3.dat} the.\textsc{m.nom.sg} argument.\textsc{nom.sg} be.\textsc{3sg.imp} avenging.\textsc{m.nom.sg}\\
\trans `This argument was for vengeance' (Herodotus 7.5.3)
\label{oi10}
\end{exe}

2) Genitive\is{genitive} or half-genitive\is{genitive} \emph{hoi} is separated from its following noun by other words: (\ref{genOiSep1})--(\ref{genOiSep6}). (In (\ref{genOiSep5}), \citealp[195]{Herwerden1878} writes \emph{hôi} `whom.\textsc{dat}' for \emph{hoi}!)

\begin{exe}
\ex τά {οἵ} ποτε {πατρὶ} φίλα φρονέων πόρε Χείρων\\
\gll tá \emph{hoí} pote {patrì} phíla phronéōn póre Kheírōn\\
the.\textsc{n.acc.pl} \textsc{3.dat} once father.\textsc{dat.sg} dear.\textsc{n.acc.pl} think.\textsc{ptcp.prs.m.nom.sg} give.\textsc{3sg.aor} Cheiron.\textsc{nom}\\
\trans `which Cheiron had once given to his father with kindly thought' (Homer, \emph{Iliad} 4.219)
\label{genOiSep1}
\end{exe}

\begin{exe}
\ex ὅϲτιϲ {οἱ} ἀρὴν {ἑτάροιϲιν} ἀμύναι\\
\gll hóstis \emph{hoi} arḕn hetároisin amúnai\\
who.\textsc{m.nom.sg} \textsc{3.dat} help.\textsc{acc.sg} companion.\textsc{dat.pl} keep-off.\textsc{aor.inf}\\
\trans `who would ward off bane from his comrades' (Homer, \emph{Iliad} 12.333)
\label{genOiSep2}
\end{exe}

\begin{exe}
\ex ἅ {oἱ} θεοὶ οὐρανίωνεϲ {πατρὶ} φίλῳ ἔπορον\\
\gll há \emph{hoi} theoì ouraníōnes {patrì} phílōi époron\\
which.\textsc{n.acc.pl} \textsc{3.dat} God.\textsc{nom.pl} heavenly.\textsc{m.nom.pl} father.\textsc{dat} beloved.\textsc{m.dat.sg} give.\textsc{3pl.aor}\\
\trans `that the heavenly gods had given to his (beloved) father' (Homer, \emph{Iliad} 17.195--196)
\label{genOiSep3}
\end{exe}

\begin{exe}
\ex θεὰ δέ {οἱ} ἔκλυεν {ἀρῆϲ}\\
\gll theà dé \emph{hoi} ékluen arês\\
Goddess.\textsc{nom.sg} but \textsc{3.dat} hear.\textsc{3sg.aor} prayer.\textsc{gen.sg}\\
\trans `and the goddess heard her prayer' (Homer, \emph{Odyssey} 4.767)
\label{genOiSep4}
\end{exe}

\begin{exe}
\ex ὅ οἱ φόνοϲ {υἷι} τέτυκται\\
\gll hó \emph{hoi} phónos {huîi} tétuktai\\
that.\textsc{n.nom.sg} \textsc{3.dat} death.\textsc{nom.sg} son.\textsc{gen.sg} ready.\textsc{3sg.pf}\\
\trans `(nor does she know at all) that death has been made ready for her son' (Homer, \emph{Odyssey} 4.771)
\label{genOiSep5}
\end{exe}

\begin{exe}
\ex μή τί {οἱ} κρεμάμενον τῷ {παιδὶ} ἐμπέϲῃ\\
\gll mḗ tí \emph{hoi} kremámenon tôi paidì empésēi\\
lest some.\textsc{n.nom.sg} \textsc{3.dat} hang.\textsc{ptcp.prs.pass.n.nom.sg} the.\textsc{m.dat.sg} child.\textsc{dat.sg} fall.\textsc{3sg.aor.sbjv}\\
\trans `lest one should fall on his son from where it hung' (Herodotus 1.34.3)
\label{genOiSep6}
\end{exe}

3) Genitive\is{genitive} or half-genitive\is{genitive} \emph{hoi} immediately precedes its noun and attributes, a position that is incomprehensible for an enclitic,\is{enclitics} in and of itself: (\ref{genOiPre1})--(\ref{genOiPre5}). \hyperlink{p338}{\emph{[p338]}} However, this word order is also found in Herodotus without \emph{hoi} in second position, e.g. (\ref{genOiPre6}). But I believe the situation is as follows: because \emph{hoi} in second position occurred so often preceding its governing noun, it became the case that \emph{hoi} could also immediately precede its governing noun in clause-medial position.

\begin{exe}
\ex μή {οἱ} {ἀπειλὰϲ} ἐκτελέϲωϲι θεοί\\
\gll mḗ \emph{hoi} {apeilàs} ektelésōsi theoí\\
lest \textsc{3.dat} boasts.\textsc{acc.pl} fulfil.\textsc{3pl.aor.sbjv} God.\textsc{nom.pl}\\
\trans `lest the gods fulfill for him his boastings' (Homer, \emph{Iliad} 9.244)
\label{genOiPre1}
\end{exe}

\begin{exe}
\ex ὅϲ οἱ παρὰ πατρὶ γέροντι κηρύϲϲων γήραϲκε\\
\gll hós \emph{hoi} {parà} {patrì} {géronti} kērússōn gḗraske\\
who.\textsc{m.nom.sg} \textsc{3.dat} in father.\textsc{dat.sg} old.\textsc{m.dat.sg} herald.\textsc{ptcp.prs.m.nom.sg} grow-old.\textsc{3sg.aor}\\
\trans `who in the house of his old father had grown old in his heraldship' (Homer, \emph{Iliad} 17.324)
\label{genOiPre2}
\end{exe}

\begin{exe}
\ex δεύτερά {οἱ τὸν παῖδα} ἔπεμπε\\
\gll deúterá \emph{hoi} {tòn} {paîda} épempe\\
then \textsc{3.dat} the.\textsc{m.acc.sg} child.\textsc{acc.sg} send.\textsc{3sg.imp}\\
\trans `{[}Cambyses{]} next made the child go out (before) him' (Herodotus 3.14.4)
\label{genOiPre3}
\end{exe}

\begin{exe}
\ex τήν {οἱ ὁ πατὴρ} εἶχε ἀρχήν\\
\gll tḗn \emph{hoi} {ho} {patḕr} eîkhe arkhḗn\\
the.\textsc{f.acc.sg} \textsc{3.dat} the.\textsc{m.nom.sg} father.\textsc{nom.sg} have.\textsc{aor.3sg} power.\textsc{acc.fem}\\
\trans `The father had the power' (Herodotus 3.15.3)
\label{genOiPre4}
\end{exe}

\begin{exe}
\ex καί {οἱ} (\emph{καὶ οἷ}?) {τῷ πατρὶ} ἔφη Σάμιον τοὔνομα τεθῆναι, ὅτι {οἱ ὁ πατὴρ} Ἀρχίηϲ ἐν Σάμῳ ἀριϲτεύϲαϲ ἐτελεύτηϲε\\
\gll kaí~(/kaì) \emph{hoi}~(/hoî) {tôi} {patrì} éphē Sámion toúnoma tethênai, hóti {hoi} {ho} {patḕr} Arkhíēs en Sámōi aristeúsas eteleútēse\\
and \textsc{3.dat} the.\textsc{m.dat.sg} father.\textsc{dat.sg} said.\textsc{3sg.imp} Samius.\textsc{acc} the=name.\textsc{acc.sg} put.\textsc{aor.inf.pass} that him.\textsc{dat} the.\textsc{m.nom.sg} father.\textsc{nom.sg} Archias.\textsc{nom} in Samos.\textsc{dat} be-best.\textsc{ptcp.aor.m.nom.sg} die.\textsc{3sg.aor}\\
\trans `and told me that his father had borne the name Samius because he was the son of that Archias who was killed fighting bravely at Samos' (Herodotus 3.55.2)
\label{genOiPre5}
\end{exe}

\begin{exe}
\ex εἰ βούλοιτό {οἱ τὴν θυγατέρα} ἔχειν γυναῖκα\\
\gll ei boúloitó \emph{hoi} tḕn thugatéra ékhein gunaîka\\
if want.\textsc{3sg.aor.opt.mid} \textsc{3.dat} the.\textsc{f.acc.sg} daughter.\textsc{acc.sg} have.\textsc{prs.inf} woman.\textsc{acc.sg}\\
\trans `If he wanted to take his daughter as a wife' (Herodotus 1.60.2)
\label{genOiPre6}
\end{exe}

4) Genitive\is{genitive} or half-genitive\is{genitive} \emph{hoi} intervenes between the first and second element of its governing expression, also an unusual position for an enclitic\is{enclitics} in itself.\label{nin} a) Between a preposition\is{prepositions} and a following particle\is{particles} and article, as in (\ref{genOiPostPrep}).

\begin{exe}
\ex ἐκ γάρ {οἱ} τῆϲ ὄψιοϲ οἱ τῶν μάγων ὀνειροπόλοι ἐϲήμαινον\\
\gll ek gár \emph{hoi} tês ópsios hoi tôn mágōn oneiropóloi esḗmainon\\
from for \textsc{3.dat} the.\textsc{f.gen.sg} sight.\textsc{gen.sg} the.\textsc{m.nom.pl} the.\textsc{m.gen.pl} magus.\textsc{gen.pl} dream-interpreter.\textsc{nom.pl} declare.\textsc{3pl.aor}\\
\trans `for the interpreters declared that to be the meaning of his dream' (Herodotus 1.108.2)
\label{genOiPostPrep}
\end{exe}

b) Between an article and a following particle\is{particles} and noun: (\ref{genOiPostArt1})--(\ref{genOiPostArt3}) (similar are Ξ 438, Ο 607, Τ 635 and many examples in the Odyssey) as well as (\ref{genOiPostArt4})--(\ref{genOiIoun5}).

\begin{exe}
\ex τὼ δέ {οἱ} ὤμω κυρτώ\\
\gll tṑ dé \emph{hoi} ṓmō kurtṓ\\
the.\textsc{nom.du} but \textsc{3.dat} shoulder.\textsc{nom.du} rounded.\textsc{m.nom.du}\\
\trans `and his two shoulders were rounded' (Homer, \emph{Iliad} 2.217)
\label{genOiPostArt1}
\end{exe}

\begin{exe}
\ex τὼ δέ {οἱ} ὄϲϲε ... χαμαὶ πέϲον\\
\gll tṑ dé \emph{hoi} ósse khamaì péson\\
the.\textsc{nom.du} but \textsc{3.dat} eye.\textsc{nom.du} down fall.\textsc{3sg.aor}\\
\trans `and his two eyeballs fell down' (Homer, \emph{Iliad} 13.616)
\label{genOiPostArt2}
\end{exe}

\begin{exe}
\ex τὼ δέ {οἱ} ὄϲϲε δακρυόφιν πλῆϲθεν\\
\gll tṑ dé \emph{hoi} ósse dakruóphin plêsthen\\
the.\textsc{nom.du} but \textsc{3.dat} eye.\textsc{nom.du} tears.\textsc{n.gen.pl} fill.\textsc{3pl.aor.pass}\\
\trans `and both his eyes were filled with tears' (Homer, \emph{Iliad} 17.695 = 23.396)
\label{genOiPostArt3}
\end{exe}

\begin{exe}
\ex αἱ δέ {οἱ} ἵπποι ἀμφίϲ ὁδοῦ δραμέτην\\
\gll hai dé \emph{hoi} híppoi amphís hodoû dramétēn\\
the.\textsc{f.nom.pl} but \textsc{3.dat} horse.\textsc{nom.pl} on-both-sides road.\textsc{gen.sg} run.\textsc{3du.aor}\\
\trans `and his mares swerved to this side and that of the course' (Homer, \emph{Iliad} 23.392)
\label{genOiPostArt4}
\end{exe}

\begin{exe}
\ex αἱ δέ {οἱ} ἵπποι ὑψόϲ᾽ ἀειρέϲθην\\
\gll hai dé \emph{hoi} híppoi hupsós' aeirésthēn\\
the.\textsc{f.nom.pl} but \textsc{3.dat} horse\textsc{.nom.pl} high leap.\textsc{3du.imp}\\
\trans `and his horses leapt on high' (Homer, \emph{Iliad} 23.500)
\label{genOiPostArt5}
\end{exe}

\begin{exe}
\ex τὸ δέ {οἱ} οὔνομα εἶναι ... Ἰοῦν\\
\gll tò dé \emph{hoi} oúnoma eînai Ioûn\\
the.\textsc{n.acc.sg} but \textsc{3.dat} name.\textsc{acc.sg} be\textsc{.prs.inf} Io.\textsc{acc}\\
\trans `and her name to be Io' (Herodotus 1.1.3)
\label{genOiHerod}
\end{exe}

\begin{exe}
\ex τῶν δέ {οἱ} παίδων τὸν πρεϲβύτερον εἰπεῖν\\
\gll tôn dé \emph{hoi} paídōn tòn presbúteron eipeîn\\
the.\textsc{gen.pl} but \textsc{3.dat} child.\textsc{gen.pl} the.\textsc{m.acc.sg} elder.\textsc{m.acc.sg} say.\textsc{aor.inf}\\
\trans `to name the eldest of his children' (Herodotus 3.3.2)
\label{genOiIoun1}
\end{exe}

\begin{exe}
\ex τόν τέ {οἱ} παῖδα ἐκ τῶν ἀπολλυμένων ϲῴζειν\\
\gll tón té \emph{hoi} paîda ek tôn apolluménōn sṓizein\\
the.\textsc{m.acc.sg} and \textsc{3.dat} child.\textsc{acc.sg} from the.\textsc{m.gen.pl} perish.\textsc{ptcp.prs.pass.m.gen.pl} save.\textsc{prs.inf}\\
\trans `to save then his child from perishing' (Herodotus 3.14.11)
\label{genOiIoun2}
\end{exe}

\begin{exe}
\ex ὁ γάρ {οἱ} ἀϲτράγαλοϲ ἐξεχώρηϲε ἐκ τῶν ἄρθρων\\
\gll ho gár \emph{hoi} astrágalos exekhṓrēse ek tôn árthrōn\\
the.\textsc{m.nom.sg} for \textsc{3.dat} ankle.\textsc{nom.sg} dislocate.\textsc{3sg.aor} from the.\textsc{n.gen.pl} sockets.\textsc{gen.pl}\\
\trans `and then his ankle was dislocated from its sockets' (Herodotus 3.129.2)
\label{genOiIoun3}
\end{exe}

\begin{exe}
\ex τὰ δέ {οἱ} ὅπλα ἔχουϲι Ἀθηναῖοι\\
\gll ta dé \emph{hoi} hopla ekhousi Athēnaioi\\
the.\textsc{n.acc.pl} but \textsc{3.dat} weapons.\textsc{acc.pl} have.\textsc{3pl.prs} Athenian.\textsc{nom.pl}\\
\trans `thus the Athenians have his weapons' (Herodotus 5.95.1)
\label{genOiIoun4}
\end{exe}

\begin{exe}
\ex τὴν δέ {οἱ} πέμπτην τῶν νεῶν κατεῖλον διώκοντεϲ οἱ Φοίνικεϲ\\
\gll tḕn dé \emph{hoi} pémptēn tôn neôn kateîlon diṓkontes hoi Phoínikes\\
the.\textsc{f.acc.sg} but \textsc{3.dat} fifth.\textsc{f.acc.sg} the.\textsc{f.gen.pl} ship.\textsc{gen.pl} take-over.\textsc{3pl.aor} chase.\textsc{ptcp.prs.m.nom.pl} the.\textsc{m.nom.pl} Phoenician.\textsc{nom.pl}\\
\trans `the Phoenicians took over one fifth of his ships by chasing (them)' (Herodotus 6.41.7)
\label{genOiIoun5}
\end{exe}

The Ionic\il{Greek, Ionic} poets\is{poetry} also provide examples, e.g. (\ref{genOiArchi1}) and (\ref{genOiArchi2}).

\begin{exe}
\ex ἡ δέ {οἱ} κόμη ὤμουϲ κατεϲκίαζε καὶ μετάφρενα\\
\gll hē dé \emph{hoi} kómē ṓmous kateskíaze kaì metáphrena\\
the.\textsc{f.nom.sg} but \textsc{3.dat} hair.\textsc{nom.sg} shoulder.\textsc{acc.pl} shadow.\textsc{imp.3sg} and chest.\textsc{acc.pl}\\
\trans `and his hair shadowed his shoulders and his chest' (Archilochus 29.2)
\label{genOiArchi1}
\end{exe}

\begin{exe}
\ex ἡ δέ {οἱ} ϲάθη ... ἐπλήμμυρεν\\
\gll hē dé \emph{hoi} sáthē eplḗmmuren\\
the.\textsc{f.nom.sg} but \textsc{3.dat} penis.\textsc{nom.sg} be-full-of-blood.\textsc{3sg.imp}\\
\trans `and then his penis was erect' (Archilochus 29.2)
\label{genOiArchi2}
\end{exe}

c) Between an article and a noun: (\ref{genOiPreNounHerod1}) and (\ref{genOiPreNounHerod2}).

\begin{exe}
\ex τῶν {οἱ} ϲυλλοχιτέων διεφθαρμένων\\
\gll tôn \emph{hoi} sullokhitéōn diephtharménōn\\
the.\textsc{m.gen.pl} \textsc{3.dat} fellow-men.\textsc{gen.pl} kill.\textsc{ptcp.pf.pass.m.gen.pl}\\
\trans `after all the men of his company had been killed' (Herodotus 1.82.8)
\label{genOiPreNounHerod1}
\end{exe}

\begin{exe}
\ex τῶν {οἱ} ϲιτοφόρων ἡμιόνων μία ἔτεκε\\
\gll tôn \emph{hoi} sitophórōn hēmiónōn mía éteke\\
the.\textsc{f.gen.pl} \textsc{3.dat} wheat-carrying.\textsc{f.gen.pl} mule.\textsc{gen.pl} one.\textsc{f.nom.sg} birth.\textsc{3sg.aor}\\
\trans `one of their donkeys that carried the wheat gave birth' (Herodotus 3.153.1)
\label{genOiPreNounHerod2}
\end{exe}

The non-Ionic\il{Greek, Ionic} post-Homeric poets,\is{poetry} for whom \emph{hoi} was part of the traditional stock of poetic\is{poetry} language, also provide parallels: here I present the examples that I have so far found. Category 1) includes (\ref{oiPindar}) as well as (\ref{oiEuph}) (=\citealp[164]{Meineke1843}).

\begin{exe}
\ex ἄνευ {οἱ} Χαρίτων {τέκεν} γόνον ὑπερφίαλον\\
\gll áneu \emph{hoi} Kharítōn {téken} gónon huperphíalon\\
without \textsc{3.dat} Grace.\textsc{gen.pl} birth.\textsc{3sg.aor} offspring.\textsc{acc.sg} monstrous.\textsc{m.acc.sg}\\
\trans `she bore to him, without the blessing of the Graces, a monstrous offspring' (Pindar, \emph{Pyth.}, 2.42)
\label{oiPindar}
\end{exe}

\begin{exe}
\ex ἀντὶ δέ {οἱ} πλοκαμῖδοϲ ἑκηβόλε καλὸϲ {ἐπείη} ὡχαρνῆθεν ἀεὶ κιϲϲὸϲ ἀεξομένῳ\\
\gll antì dé \emph{hoi} plokamîdos hekēbóle kalòs {epeíē} hōkharnêthen aeì kissòs aexoménōi\\
instead but \textsc{3.dat} braid.\textsc{gen.sg} archer.\textsc{voc.sg} beautiful.\textsc{m.nom.sg} be-upon.\textsc{3sg.prs.opt} the=from-Acharnae always ivy.\textsc{nom.sg} grow.\textsc{ptcp.prs.pass.n.dat.sg}\\
\trans `Instead of his locks, O Archer, may the beautiful ivy of Acharnae be added to the eternal growth.' (Anthologia Graeca 6.279)\footnote{\emph{Translator's note}: `Archer' is an epithet of Apollo.}
\label{oiEuph}
\end{exe}

Category 2) includes (\ref{oiTheo}) from Theocritus (cf. \citealp[256]{Meineke1856} on Theocritus 7.88). Example (\ref{oiSoph}) belongs to either 1) or 2).

\begin{exe}
\ex ἐγὼ δέ {οἱ} ἁ ταχυπειθὴϲ {χειρὸϲ} ἐφαψαμένα\\
\gll egṑ dé \emph{hoi} ha takhupeithḕs {kheiròs} ephapsaména\\
I.\textsc{nom} but \textsc{3.dat} the.\textsc{f.nom.sg} credulous.\textsc{f.nom.sg} hand.\textsc{gen.sg} bind.\textsc{ptcp.aor.mid.f.nom.sg}\\
\trans `then I, being credulous, bound her hands to him' (Theocritus 2.138)
\label{oiTheo}
\end{exe}

\begin{exe}
\ex ἐν γάρ {οἱ} χθονὶ {πηκτὸν τόδ᾽ ἔγχοϲ} περιπετέϲ κατηγορεῖ\\
\gll en gár \emph{hoi} khthonì {pēktòn} {tód'} {énkhos} peripetés katēgoreî\\
in for \textsc{3.dat} ground.\textsc{dat.sg} fixed.\textsc{n.nom.sg} this.\textsc{n.nom.sg} spear.\textsc{nom.sg} surrounded.\textsc{n.nom.sg} convict.\textsc{3sg.prs}\\
\trans `His sword which he planted in the ground and on which he fell convicts him.' (Sophocles, \emph{Ajax} 907)
\label{oiSoph}
\end{exe}

\hyperlink{p339}{\emph{[p339]}} Example (\ref{oiEur}) belongs to 3), and (\ref{oiTrach}) belongs to 4).

\begin{exe}
\ex ἅτε {οἱ} αἵματοϲ ἔϲκεν\\
\gll háte \emph{hoi} haímatos ésken\\
who.\textsc{f.nom.sg} \textsc{3.dat} blood.\textsc{gen.sg} be.\textsc{3sg.imp}\\
\trans `who was of his blood' (Moschus, \emph{Europa} 41)
\label{oiEur}
\end{exe}

\begin{exe}
\ex ἁ δέ {οἱ} φίλα δάμαρ τάλαιναν δυστάλαινα καρδίαν παγκλαυτοϲ αἰὲν ὤλλυτο\\
\gll ha dé \emph{hoi} phíla dámar tálainan dustálaina kardían panklautos aièn ṓlluto\\
she.\textsc{nom} but \textsc{3.dat} dear.\textsc{f.nom.sg} wife.\textsc{nom} suffering.\textsc{f.acc.sg} most-miserable.\textsc{f.nom.sg} heart.\textsc{acc.sg} most-lamentable.\textsc{f.nom.sg} always destroy.\textsc{3sg.imp.pass}\\
\trans `She, his loving wife, miserable, was ever pining in her miserable heart, always weeping' (Sophocles, \emph{Trachiniai} 650)
\label{oiTrach}
\end{exe}

Inscriptions\is{inscriptions|(} in the dialects that employ \emph{hoi} are unrevealing. Among the Doric\il{Greek, Doric} dialects, only Epidauric yields richer results, and these are well known to be relatively late. In No. 3339 and 3340 of Collitz \citep{Prellwitz1889} I can count fourteen instances of \emph{hoi} in second position and eight of \emph{hoi} elsewhere. The few non-Doric\il{Greek, Doric} examples I have to hand all follow the rule: (\ref{Tegea1222}), (\ref{Kypros59}) (cf. \citealp[148]{Meister1889}, \citealp[67f.]{Hoffmann1891}), and by the same author (\ref{Kypros60}).

\begin{exe}
\ex μή {οἱ} ἔϲτω ἴνδικον\\
\gll mḗ \emph{hoi} éstō índikon\\
not \textsc{3.dat} be.\textsc{3sg.imper} unjust.\textsc{n.nom.sg}\\
\trans `let it not be unjust to him' (Inscription 1222.33 Collitz, Tegea)\footnote{\emph{Translator's note}: In this inscription the author seems to be using ἰν- like Attic\il{Greek, Attic} ἀ-.}
\label{Tegea1222}
\end{exe}

\begin{exe}
\ex ἀφ᾽ ὧ ϝοι τὰϲ εὐχωλὰϲ ἐπέτυχε /ἐπέδυκε\\
\gll aph' hô \emph{woi} tàs eukhōlàs epétukhe/epéduke\\
of whom.\textsc{dat} \textsc{3.dat} the.\textsc{f.acc.pl} prayer.\textsc{acc.pl} succeed.\textsc{3sg.aor}\\
\trans `from whom his prayers were granted' (Inscription 59.3 Collitz, Cyprus)
\label{Kypros59}
\end{exe}

\begin{exe}
\ex ἀνοϲίϳα ϝοι γένοιτυ\\
\gll anosíja \emph{woi} génoitu\\
unholy.\textsc{n.nom.pl} \textsc{3.dat} become.\textsc{3sg.aor.opt.mid}\\
\trans `may curses come upon him' (Inscription 60.29 Collitz, Cyprus)
\label{Kypros60}
\end{exe}\is{inscriptions|)}

Despite all of this, however, one might nevertheless find it remarkable that Thumb\ia{Thumb, Albert} could discover this idiosyncratic positional custom, apparently reminiscent of the position of \emph{sma} in the \textit{\d{R}gveda}, and might still be inclined to suspect something of significance behind it. To shed light on this, it seems most appropriate to compare the statistics that \citet{Thumb1887} gives for \emph{min} against the use of \emph{hoi} in ΝΠΡ. Thumb 1a: ``in 68\% of all cases, \emph{min} follows a particle'';\is{particles} \emph{hoi} does so in 66 of 92 cases, i.e. 72\% (33 times after \emph{dé} `but, and', just as \emph{de} also most commonly precedes \emph{min}; after that, in decreasing order of frequency, it is found after \emph{ára} (interrogative),\is{interrogatives} \emph{rha}, \emph{kaí} `and', \emph{gár} `for, since', \emph{oudé} `but not', \emph{te} `and', \emph{éntha} `there/where', \emph{allá} `but', \emph{ḗ} `or, than', \emph{mén} `while, so', \emph{pōs} `in any way', \emph{tákha} `quickly, soon'). Thumb 1b: ``in 10\% of cases, \emph{min} follows a subordinating\is{subordination} conjunction''; \emph{hoi} does so four times (after \emph{hó(t)ti} `that/because', \emph{epeí} `after, since', \emph{óphra} `in order that, as long as, until'), i.e. only in 4\% of cases -- a difference that is made even less meaningful by the fact that Thumb is obliged to note a difference between \emph{min} and \emph{sma} for this category, as \emph{sma} is not keen on this position. Thumb 2: ``\emph{min} never immediately follows \isi{prepositions} (in contrast to \emph{sma}!)''; the same is true for \emph{hoi}. Thumb 3: ``\emph{oú min}, \emph{mḗ min} in 15 of 600 examples'', i.e. 2.5\%; \emph{oú hoi}, \emph{mḗ hoi} in 3 of 92 examples, i.e. 3.25\%. Thumb 4: ``\emph{min} very often occurs after \isi{pronouns}'', apparently about 100 times or 16.67\%; \emph{hoi} is also often found here, in fact 17 times, i.e. 18.5\%. Thumb 5 and 6:\ia{Thumb, Albert} ``\emph{min} follows verbs and nominal words in 3\% of cases''; \emph{hoi} follows \emph{aipú} `steep' in Ν 317 and \emph{haímati} `blood.\textsc{dat}' in Ρ 51, i.e. in 2\% of cases.

Thumb's\ia{Thumb, Albert} observations are thus just as valid for \emph{hoi} as for \emph{min}. \emph{hoi} is found following the same \hyperlink{p340}{\emph{[p340]}} words as \emph{min} and with almost exactly the same frequency as \emph{min}. What Thumb\ia{Thumb, Albert} has demonstrated for \emph{min} is therefore not a property specific to \emph{min} but rather a consequence, common to \emph{min} and \emph{hoi}, of the positional law that assigns to both of them the second position in the clause.

This removes the main point in support of the argument that \emph{min} has its origin in \emph{sm(a)-im}, this argument is almost entirely refuted by the absence of any reflex of the hypothesized earlier initial cluster \emph{sm-}. One would expect occasional instances of \emph{de min} as a trochee or spondee; Thumb\ia{Thumb, Albert} is mute on this point. A further consideration can be adduced. The combination of \emph{sma} and \emph{im} that supposedly gave rise to \emph{min} could be seen as ancient: in this case, the loss of the original function of \emph{sma} in the use of \emph{min} makes sense, but one would expect Greek *\emph{(s)main} corresponding to \ili{Sanskrit} *\emph{smēm}. The other possibility is that this combination arose not long before Homer, in which case the presence of the specifically Greek reduction, i.e. the development \emph{ma in} $\to$ \emph{m' in} $\to$ \emph{min}, makes sense -- but then the complete loss of the function of \emph{(s)ma}, the treatment of \emph{min} exactly like any other normal pronoun,\is{pronouns} is inexplicable, especially since in Thessalonian\il{Greek, Thessalonian} a particle\is{particles} \emph{ma} with the meaning `but' occurs, which can however only debatably be connected with \ili{Sanskrit} \emph{sma}.

Thumb's explanation of Doric\il{Greek, Doric} \emph{nin} as arising from \emph{nu-im} seems to me to be even less successful, since here insurmountable phonetic difficulties seem to stand in its way. In his observation that ``it is safe to assume that at an earlier stage it was possible to pronounce final \emph{u} as a consonant (ṷ) under certain conditions, as in \ili{Sanskrit} (e.g. ((\ref{RV_ex3})), adducing examples such as \emph{pros} from \emph{proti̯}, \emph{ein} from \emph{eni̯}, \emph{hupeir} from \emph{hyperi̯} (= \ili{Sanskrit} \emph{upary} alongside \emph{upari}), Lesbian \emph{perr}- from \emph{peri̯-}, in which \emph{i̯} could stand in for \emph{i} during the period of Indo-European unity, Thumb overlooks the fact that not all final \emph{-i}s and \emph{-u}s can be treated the same.

\begin{exe}
\ex \gll kō nv atra \\
who.\textsc{nom.sg} now here \\
\trans `Now who [has given liberally to you] here {[}, Maruts{]}?' 
(\textit{\d{R}gveda}, 1.165.13a; trans. \citealt[1417]{JamisonBrereton2014})
\label{RV_ex3}
\end{exe}

In the \textit{\d{R}gveda}, \emph{-i} and \emph{-u} only become \emph{-y} and \emph{-v} with any frequency in the word class in which Greek shows \hyperlink{p341}{\emph{[p341]}} reflexes of such a change, namely in the disyllabic \isi{prepositions} such as \emph{abhi}, \emph{prati}, \emph{anu}, \emph{pari}, \emph{adhi}; otherwise, outside the later 10th book and the \emph{Vālakhilyas} this occurs only very sporadically. In monosyllables it is only found in the compound \emph{avyuṣṭāḥ}\footnote{\emph{Translator's note}: The form attested in the text is \emph{avyuṣṭā}, with sandhi.} (`not yet dawned'; \textit{\d{R}gveda}, 2.28.9a) and then in \emph{ny alipsata} (`wiped out'; \textit{\d{R}gveda}, 1.191.3d, i.e. in a song that is known to be late \citep[438, note 4]{Oldenberg1888}. And \emph{nu} in particular (like \emph{u}) avoids this sandhi completely; in fact, it often lengthens, even becoming disyllabic in extreme cases. And even if we could reconstruct Pre-Greek\il{Pre-Greek} \emph{nϝin}, hence Doric\il{Greek, Doric} \emph{nin}, following a final vowel, a postconsonantal \emph{nin} would still be inexplicable; the development \emph{hós nu in}, \emph{hós nw in}, \emph{hós nin} is completely inconceivable.

Furthermore, when \citet[646--647]{Thumb1887} suggests that the position of \emph{nin} in the clause shows no special analogy with that of \ili{Sanskrit} \emph{nu} and Greek \emph{nu}, and excuses this with reference to the young age of the sources that contain \emph{nin} (Pindar and the tragic poets),\is{poetry} it is certainly true that these authors can provide no clean results for \emph{nin} like those from Homer and Herodotus for \emph{min} -- not only on chronological grounds, but also because of the more artificial nature of their word order. But one might well still ask whether certain tendencies can be recognized. And here it can be observed that, in 30 of 47 relevant examples from Aeschylus, \emph{nin} follows the positional law established for \emph{min} and \emph{hoi} -- and, remarkably, in 5 of 7 examples in \emph{The Persians} and \emph{Seven against Thebes}, 21 of 32 in the \emph{Oresteia}, and 2 of 5 in \emph{Prometheus Bound}. The ratio in Sophocles is less favourable: of 81 examples, 47 follow the law for \emph{nin} and 34 do not. The first class includes the cases of \isi{tmesis}, e.g. (\ref{SophTmes1}) and (\ref{SophTmes2}).

\begin{exe}
\ex ϲὺν δέ {νιν} θηρώμεθα\\
\gll sùn dé \emph{nin} thērṓmetha\\
with but \textsc{3.acc} chase.\textsc{1pl.prs}\\
\trans `and at once closed upon our quarry' (Sophocles, \emph{Antigone} 432)
\label{SophTmes1}
\end{exe}

\begin{exe}
\ex κατ᾽ αὖ {νιν} φοινία θεῶν τῶν νερτέρων ἀμᾷ κοπίϲ\\
\gll kat' aû \emph{nin} phoinía theôn tôn nertérōn amâi kopís\\
against again \textsc{3.acc} bloody.\textsc{f.nom.sg} god.\textsc{gen.pl} the.\textsc{m.gen.pl} lower.\textsc{m.gen.pl} reap.\textsc{3sg.prs} knife.\textsc{nom.sg}\\
\trans `that hope, in its turn, the blood-stained knife\footnote{\emph{Translator's note}: Wackernagel here has \textit{kopís} `knife, sword', while the modern Perseus edition has \textit{kónis} `dust'.} of the gods infernal cuts down' (Sophocles, \emph{Antigone} 601)
\label{SophTmes2}
\end{exe}

A sense for the actual position of \emph{nin} is alive elsewhere too: cf. Aristophanes \emph{Acharnians} 775, and in particular (\ref{MedeaNin})--(\ref{BacchNin}) from Euripides; in addition, (\ref{TheoNin1}) and (\ref{TheoNin2}) from Theocritus.

\begin{exe}
\ex ἀλλά {νιν}, ὦ φάοϲ διογενέϲ, κατεῖργε\\
\gll allá \emph{nin}, ô pháos diogenés, kateîrge\\
but \textsc{3.acc} O light.\textsc{voc.sg} Zeus-born.\textsc{f.voc.sg} check.\textsc{2sg.imper}\\
\trans `O light begotten of Zeus, check her' (Euripides, \emph{Medea} 1258)
\label{MedeaNin}
\end{exe}

\begin{exe}
\ex τίϲ δέ {νιν} ναυκληρία ἐκ τῆϲδ᾽ ἀπῆρε χθονόϲ\\
\gll tís dé \emph{nin} nauklēría ek têsd' apêre khthonós\\
what.\textsc{f.nom.sg} but \textsc{3.acc} voyage.\textsc{nom.sg} from this.\textsc{f.gen.sg} lift-off.\textsc{3sg.aor} earth.\textsc{gen.sg}\\
\trans `What ship carried her off from this land?' (Euripides, \emph{Helen} 1519)
\label{HelenaNin}
\end{exe}

\begin{exe}
\ex ὑμεῖϲ δὲ, νεάνιδέϲ, {νιν} ἀγκάλαιϲ ἔπι δέξαϲθε\\
\gll humeîs dè, neánidés, \emph{nin} ankálais épi déxasthe\\
you.\textsc{nom.pl} then young-woman.\textsc{voc.pl} \textsc{3.acc} arm.\textsc{acc.pl} in accept.\textsc{2pl.aor.imper.mid}\\
\trans `You then, young women, take her in your arms' (Euripides, \emph{Iphigenia in Aulis} 615)
\label{IphiNin}
\end{exe}

\begin{exe}
\ex ὧν {νιν} οὕνεκα κτανεῖν Ζῆν᾽ ἐξεκαυχῶντ(ο)\\
\gll hôn \emph{nin} hoúneka ktaneîn Zên' exekaukhônt(o)\\
which.\textsc{n.gen.pl} \textsc{3.acc} for kill.\textsc{aor.inf} Zeus.\textsc{acc} boast.\textsc{3pl.imp}\\
\trans `for which they boasted that Zeus killed her' (Euripides, \emph{Bacchae} 30)
\label{BacchNin}
\end{exe}

\begin{exe}
\ex ἐγὼ δέ {νιν} ὡϲ ἐνόηϲα\\
\gll egṑ dé \emph{nin} hōs enóēsa\\
I.\textsc{nom} but \textsc{3.acc} as understand.\textsc{1sg.aor}\\
\trans `and when I felt him' (Theocritus 2.103)
\label{TheoNin1}
\end{exe}

\begin{exe}
\ex τὰ δέ {νιν} καλὰ κύματα φαίνει\\
\gll tà dé \emph{nin} kalà kúmata phaínei\\
the.\textsc{n.acc.pl} but you good.\textsc{n.acc.pl} wave.\textsc{acc.pl} show.\textsc{3sg.pres}\\
\trans `[She] shows the good waves to you' (Theocritus 6.11)
\label{TheoNin2}
\end{exe}

Finally, \hyperlink{p342}{\emph{[p342]}} the ancient Rhodian inscription\is{inscriptions} (\ref{Rhodian}) recently presented by \citet{Selivanov1891} is very striking: \emph{nin} syntactically belongs with \emph{pēmaínoi}, corresponding exactly to the \emph{min} in example (\ref{phi347}) discussed above.

\begin{exe}
\ex ϲᾶμα τόζ᾽ Ἰδαμενεὺϲ ποίηϲα ἵνα κλέοϲ εἴη· Ζεὺϲ δέ {νιν} ὅϲτιϲ πημαίνοι, λειώλη θείη\\
\gll sâma tóz' Idameneùs poíēsa hína kléos eíē; Zeùs dé \emph{nin} hóstis, pēmaínoi leiṓlē theíē\\
sign.\textsc{acc.sg} this.\textsc{n.acc.sg} Idomeneus.\textsc{nom} make.\textsc{1sg.aor} that fame.\textsc{nom} be.\textsc{3sg.prs.opt} Zeus.\textsc{nom} but \textsc{3.acc} who.\textsc{m.nom.sg} ruin.\textsc{3sg.prs.opt} destructive.\textsc{n.acc.pl} put.\textsc{3sg.aor.opt}\\
\trans `I, Idomeneus, have made this sign that there be fame (to me), and that Zeus, who shall ruin him, may send destruction' (IG XII,1 737)
\label{Rhodian}
\end{exe}

This essential identity in position between \emph{nin} and \emph{min} is another spanner in the works for Thumb's\ia{Thumb, Albert} argumentation. I agree with him on one point: that \emph{m-in} and \emph{n-in} should be segmented and that *\emph{in} is the \isi{accusative} corresponding to Latin \emph{is}, as well as that both the assumption of underlying reduplication *\emph{imim}, *\emph{inin} and the assumption of roots \emph{mi-}, \emph{ni-} preserved in \emph{min} and \emph{nin} are incorrect. In the absence of a better theory, it seems to me to be simplest to derive \emph{n}- and \emph{m}- from sandhi. Given the pairs \emph{autíka-m-in} (from \emph{-kṃm im}) and \emph{autíka mán}, \emph{ára-m-in} and \emph{ára mán}, and \emph{rha-m-in} and \emph{rha mán} (if we can assume a voiced bilabial nasal word-finally in \emph{ára} and \emph{rha}), it would have been possible for \emph{alla min} to emerge alongside \emph{alla man} and for \emph{min} to spread further, little by little; \emph{alla min} : \emph{autíka min} = \emph{mēkéti} : \emph{oukéti}. Similarly, the \emph{n}- of \emph{nin} can be explained as originating in a word-final voiced dental nasal. See \citet[119--125]{Wackernagel1887} on \emph{atta} from \emph{tta}, \emph{ouneka} from \emph{eneka} and related matters, as well as the \ili{Prakrit} enclitic\is{enclitics} \emph{m-iva}, \emph{mmiva} for \ili{Sanskrit} \emph{iva}, where the \emph{m} naturally arises from the final segment of neuters and accusatives\is{accusative} \citep[370]{Lassen1837}. See further \citet[423]{Tobler1877}, \citet[943f.]{Meyer1885}, \citet[1371]{Ziemer1885}, \citet[181]{Schuchardt1887}, \citet[167 fn]{Thielmann1889}.


\section{The position of enclitic pronouns in Archaic Greek}\label{enclitic-archaic}\il{Greek, Archaic|(}\il{Greek, Homeric|(}\is{pronouns|(}\is{enclitics|(}

The preference for putting \emph{min}, \emph{nin} and \emph{hoi} in the second position in the clause should be viewed in its broader context. \citet[177--178]{Bergaigne1877} already suggested that enclitic pronouns in general ``prefer to be placed after the first word in the clause''. As evidence he adduces (\ref{bergaigne1})
and (\ref{bergaigne2}).

\begin{exe}
\ex ὅ {ϲφιν} εὔ φρονέων ἀγορήϲατο καὶ μετέειπεν\\
\gll hó \emph{sphin} eú phronéōn agorḗsato kaì metéeipen\\
\textsc{pro} them.\textsc{dat} well be-minded.\textsc{ptcp.prs.m.nom.sg} assemble.\textsc{3sg.aor.mid} and address.\textsc{3sg.aor}\\
\trans `he with good intent addressed the gathering, and spoke among them' (Homer, \textit{Iliad} 1.73)\footnote{\emph{Translator's note}: The element marked \textsc{pro} is a rare Ionic\il{Greek, Ionic} form of a pronoun, glossed as `who, which' in Liddell-Scott,\ia{Liddell, Henry George}\ia{Scott, Robert} and not further specified in the gloss here.}
\label{bergaigne1}
\end{exe}

\begin{exe}
\ex ὅ {μοι} γέραϲ ἔρχεται ἄλλῃ\\
\gll hó \emph{moi} géras érkhetai állēi\\
which.\textsc{n.acc.sg} me.\textsc{dat} prize.\textsc{nom.sg} go.\textsc{3sg.pres.pass} elsewhere\\
\trans `that my prize goes elsewhere' (Homer, \textit{Iliad} 1.120)
\label{bergaigne2}
\end{exe}

This observation is confirmed as soon as one starts \hyperlink{p343}{\emph{[p343]}} to collect examples. Beginning with the third person pronouns to follow on from \emph{min}, \emph{nin} and \emph{hoi}, in the books ΝΠΡ that I have drawn upon as sources, \textit{he} (\textsc{3sg}) is found four times, always in the second position or as close as possible to second position (in what follows I will disregard this distinction). There are twelve instances of \textit{sphi(n)} (\textsc{3pl.dat}), of which eleven follow the rule; the only exception is (\ref{sphin1}) (see also (\ref{sphin2}), where \textit{sphin} has been inserted into the group \textit{tòn dé ánakta}). 

\begin{exe}
\ex ἐπὶ δὲ πτόλεμοϲ τέτατό {ϲφιν}\\
\gll epì dè ptólemos tétató \emph{sphin}\\
against but war.\textsc{nom.sg} strain.\textsc{3sg.plpf.pass}
them.\textsc{dat}\\
\trans `and against them was strained a conflict' (Homer, \textit{Iliad} 17.736)
\label{sphin1}
\end{exe}

\begin{exe}
\ex τὸν δέ {σφιν} ἄνακτ᾽ ἀγαθὸϲ Διομήδηϲ ἔκτανε\\
\gll tòn dé \emph{sphin} ánakt' agathòs Diomḗdēs éktane\\
the.\textsc{m.acc.sg} but them.\textsc{dat} king.\textsc{acc} good.\textsc{m.nom.sg} Diomedes.\textsc{nom} kill.\textsc{3sg.aor}\\
\trans `and brave Diomedes slew their lord' (Homer, \textit{Iliad} 10.559)
\label{sphin2}
\end{exe}

\textit{sphisi} (\textsc{3pl.dat}) is found six times, always following the rule. \textit{spheas} (\textsc{3pl.acc}) is found in (\ref{spheas}), and \textit{sphōe} (\textsc{3du.acc}) in (\ref{sphoe}). From elsewhere in Homer we can draw the hyperthetic example (\ref{hyperthetic}).

\begin{exe}
\ex μάλα γάρ {ϲφεαϲ} ὦκ᾽ ἐλέλιξεν\\
\gll mála gár \emph{spheas} ôk' elélixen\\
very for them.\textsc{acc} speedily whirl-round.\textsc{3sg.aor}\\
\trans `for full speedily did Aias rally them' (Homer, \textit{Iliad} 17.278)
\label{spheas}
\end{exe}

\begin{exe}
\ex εἰ μή {ϲφω᾽} Αἴαντε διέκριναν μεμαῶτε\\
\gll ei mḗ \emph{sphō'} Aíante diékrinan memaôte\\
if not them.\textsc{acc.du} Aiantes.\textsc{nom.du} separate.\textsc{3pl.aor} desire.\textsc{ptcp.pf.m.acc.du}\\
\trans `but that the twain Aiantes parted them in their fury'\\
(Homer, \textit{Iliad} 17.531)
\label{sphoe}
\end{exe}

\begin{exe}
\ex καί {ϲφεαϲ} φωνήϲαϲ ἔπεα πτερόεντα προϲηύδα\\
\gll kaí \emph{spheas} phōnḗsas épea pteróenta prosēúda\\
and them.\textsc{acc.pl} produce-a-sound.\textsc{ptcp.aor.m.nom.sg} word.\textsc{acc.pl} flying.\textsc{n.acc.pl} address.\textsc{3sg.imp}\\
\trans `and he spoke and addressed them with winged words' (Homer, \textit{Iliad} 4.284)
\label{hyperthetic}
\end{exe}

The same is true in the second person: \textit{seo} and \textit{seu} (\textsc{2sg.gen}) are found five times, always in second position (for more examples see below); \textit{toi} (\textsc{2sg.dat}, for which I am counting the cases where it is used as a particle,\footnote{\emph{Translator's note}: Homeric clitic \textit{toi} is translated by Liddell-Scott\ia{Liddell, Henry George}\ia{Scott, Robert} as `let me tell you, mark you, look you'.}\is{particles} for obvious reasons, but without \textit{ḗtoi} `either, or' and \textit{itoi}) can be found 47 times, 45 of which follow the rule, with only two exceptions: (\ref{toi1}) and (\ref{toi2}). In both cases the tendency for \isi{enclitics} to attach to the \isi{negation}\label{ouk3} has interfered with the operation of the main rule.

\begin{exe}
\ex ἐπει οὔ {τοι} ἐεδνωταὶ κακοί εἰμεν\\
\gll epei oú \emph{toi} eednōtaì kakoí eimen\\
because not you.\textsc{dat} father-of-bride.\textsc{dat.sg}
bad.\textsc{m.nom.pl} be.\textsc{1pl.prs}\\
\trans `because you may be sure we deal not hardly in exacting wedding gifts' (Homer, \textit{Iliad} 13.382)
\label{toi1}
\end{exe}

\begin{exe}
\ex ἀτὰρ οὔ {τοι} πάντεϲ ἐπαινέομεν θεοὶ ἄλλοι\\
\gll atàr oú \emph{toi} pántes epainéomen theoì álloi\\
but not \textsc{cl} all.\textsc{m.nom.pl} applaud.\textsc{1pl.prs} god.\textsc{nom.pl} other.\textsc{m.nom.pl}\\
\trans `but be sure that we other gods don't all applaud' (Homer, \textit{Iliad} 16.443)
\label{toi2}
\end{exe}

\textit{se} (\textsc{2sg.acc}) can be found 21 times, of which 19 instances follow the rule and two behave differently: (\ref{se1}) and (\ref{se2}).

\begin{exe}
\ex εἰ καὶ ἐγώ ϲε βάλοιμι\\
\gll ei kaì egṓ \emph{se} báloimi\\
if and I.\textsc{nom} you.\textsc{acc} throw.\textsc{1sg.aor.opt}\\
\trans `if so be I should smite thee' (Homer, \textit{Iliad} 16.623)
\label{se1}
\end{exe}

\begin{exe}
\ex ἦ τ᾽ ἐφάμην ϲε\\
\gll ê t' ephámēn \emph{se}\\
in-truth and speak.\textsc{1sg.imp} you.\textsc{acc}\\
\trans `in truth I deemed you ...' (Homer, \textit{Iliad} 17.171)
\label{se2}
\end{exe}

The same is true in the first person: \textit{meu} (\textsc{1sg.gen}) can be found in Ν 626 and Ρ 29, and in both it immediately follows the beginning of the clause; \textit{moi} (\textsc{1sg.dat}) is found 32 times (including \textit{hō moi}), including 27 following the rule, and (\ref{moipron1})
can probably be adduced as a 28th example.

\begin{exe}
\ex ἀλλὰ τί ἦ {μοι} ταῦτα φίλοϲ διελέξατο θυμόϲ\\
\gll allà tí ê \emph{moi} taûta phílos dieléxato thumós\\
but why in-truth me.\textsc{dat} these.\textsc{n.acc.pl} beloved.\textsc{m.nom.sg} converse.\textsc{3sg.aor.mid} soul.\textsc{nom.sg}\\
\trans `But why does my heart thus hold converse with me?'\\
(Homer, \textit{Iliad} 17.97)
\label{moipron1}
\end{exe}

The deviations are (\ref{moipron2}) (if written as \textit{héspeté nún moi}, the example would count as following the rule), (\ref{moipron3}), (\ref{moipron4}), and (\ref{moipron5}) -- exceptions which neither quantitatively nor qualitatively challenge the rule, while conversely an example like (\ref{patrokle}), in which the attachment of \textit{moi} to a \isi{vocative} was already striking to the ancients, is evidence for the consistent validity of the rule. 

\begin{exe}
\ex ἕϲπετε νῦν {μοι}\\
\gll héspete nûn \emph{moi}\\
say.\textsc{2pl.aor.imper} now me\textsc{.dat}\\
\trans `tell me now' (Homer, \textit{Iliad} 16.112)
\label{moipron2}
\end{exe}

\begin{exe}
\ex ἠδ᾽ ἔτι καὶ νῦν {μοι} τόδ᾽ ἐπικρήηνον ἐέλδωρ\\
\gll ēd' éti kaì nûn \emph{moi} tód' epikrḗēnon eéldōr\\
and yet even now me.\textsc{dat} this.\textsc{n.acc.sg} accomplish.\textsc{2sg.aor.imper} desire.\textsc{acc.sg}\\
\trans `even so now also fulfil you for me this my desire' (Homer, \textit{Iliad} 16.238)
\label{moipron3}
\end{exe}

\begin{exe}
\ex ἀλλὰ ϲύ πέρ {μοι} ἄναξ τόδε καρτερὸν ἕλκοϲ ἄκεϲϲαι\\
\gll allà sú pér \emph{moi} ánax tóde karteròn hélkos ákessai\\
but you.\textsc{nom} all me.\textsc{dat} king.\textsc{voc.sg} this.\textsc{n.acc.sg} strong.\textsc{n.acc.sg} wound.\textsc{acc.sg} heal.\textsc{2sg.aor.imper.mid}\\
\trans `Howbeit, do thou, O king, heal me of this grievous wound' (Homer, \textit{Iliad} 16.523)
\label{moipron4}
\end{exe}

\begin{exe}
\ex αἰνὸν ἄχοϲ τό {μοί} ἐϲτιν\\
\gll ainòn ákhos tó \emph{moí} estin\\
dire.\textsc{n.nom.sg} pain.\textsc{nom.sg} the.\textsc{n.nom.sg} me.\textsc{dat} be.\textsc{3sg.prs}\\
\trans `my pain is dire' (Homer, \textit{Iliad} 16.55)
\label{moipron5}
\end{exe}

\begin{exe}
\ex Πάτροκλέ {μοι} δειλῇ πλεῖϲτον κεχαριϲμένε θυμῷ\\
\gll Pátroklé \emph{moi} deilêi pleîston kekharisméne thumôi\\
Patroclus.\textsc{voc} me.\textsc{dat} cowardly.\textsc{m.dat.sg} most gratify.\textsc{ptcp.pf.pass.m.voc.sg} soul.\textsc{dat.sg}\\
\trans `Patroclus, dearest to my hapless heart' (Homer, \textit{Iliad} 19.287)
\label{patrokle}
\end{exe}

\hyperlink{p344}{\emph{[p344]}} Similarly striking is \textit{moi} after \textit{áll' áge}, as in (\ref{moiallage}). Finally, \textit{me} (\textsc{1sg.acc}) can be found 15 times, always following the rule. (\citealp[336ff.]{Monro1891} discusses exceptions from the other books, in some cases proposing emendations.)\is{emendation}

\begin{exe}
\ex ἄλλ᾽ ἄγε {μοι} τόδε εἰπέ\\
\gll áll' áge \emph{moi} tóde eipé\\
but bring.\textsc{2sg.prs.imper} me.\textsc{dat} this.\textsc{n.acc.sg} say.\textsc{3sg.aor.imper}\\
\trans `but, he said, come (and) tell me this' (Homer, \textit{Odyssey} 1.169)
\label{moiallage}
\end{exe}

Traces of the old rule can also be identified outside Homer. For instance, in the works of the elegiacs\is{elegiac poets} up to and including Theognis, \textit{me} is found 42 times in second position and 4 times later; \textit{moi} is found 36 times in second position and 5 times later; \textit{se} is found 27 times in second position and 6 times later. The same is true in the dialectal texts that depend more on the elegiac\is{elegiac poets} poets\is{poetry} than on Homer: although the Arcadians seem to have placed their \textit{spheis} relatively freely, the Doric\il{Greek, Doric} accusative\is{accusative} \textit{tu}\label{tu} fits the rule even better, e.g. (\ref{tu1})--(\ref{tu4}).

\begin{exe}
\ex καί {τυ} φίλιππον ἔθηκεν\\
\gll kaí \emph{tu} phílippon éthēken\\
and you.\textsc{acc} Philip.\textsc{acc} place.\textsc{3sg.aor}\\
\trans `and he placed Philip with you' (Theognis, \textit{Fragmenta Lyrica Adespota} 43; \citealp[701]{Bergk1882})
\label{tu1}
\end{exe}

\begin{exe}
\ex ἐκάλεϲε γάρ {τύ} τιϲ\\
\gll ekálese gár \emph{tú} tis\\
invite.\textsc{3sg.aor} for you.\textsc{acc} someone.\textsc{m.nom.sg}\\
\trans `Did someone invite you?' (Epicharmus in Athenaeus 4.16)
\label{tu2}
\end{exe}

\begin{exe}
\ex τί {τυ} ἐγὼν ποιέω\\
\gll tí \emph{tu} egṑn poiéō\\
what.\textsc{n.acc.sg} you.\textsc{acc} I.\textsc{nom} do.\textsc{1sg.prs}\\
\trans `What am I doing for you/to you?' (Sophron in Apollonius Dyscolus, \textit{De Pronominibus} 68B)
\label{tu3}
\end{exe}

\begin{exe}
\ex ἐπόθουν {τυ} ναί τὸν φίλιον ἇπερ ματέρα\\
\gll epóthoun \emph{tu} naí tòn phílion hâper matéra\\
long-for.\textsc{1sg.imp} you.\textsc{acc} by the.\textsc{m.acc.sg} friendly.\textsc{m.acc.sg} which.\textsc{f.dat.sg} mother.\textsc{acc.sg}\\
\trans `Let Zeus, the patron of friendship, witness, I regretted you as a mother {[}mourns her son{]}.' (Aristophanes, \textit{Acharnians} 730)
\label{tu4}
\end{exe}

In addition there is the Doric\il{Greek, Doric} oracle spell in (\ref{oracle}) (from Ephorus; not mentioned by \citealp[255]{Ahrens1843}) and the majority of the roughly thirty examples from Theocritus, of which particularly noteworthy are (\ref{theotu1}) (=Attic\il{Greek, Attic} \emph{mḗtis se eirṓta}), in which \emph{mḗtis} is split in two by \emph{tu}, and (\ref{theotu2}), in which \emph{tu} (confidently emended\is{emendation} by \citet[290]{Brunck1776} from the better attested but unmetrical \textit{toi}) is governed as an accusative\is{accusative} by \textit{zateûs(a)} but intervenes between the distant \textit{ha} and \textit{kṓra}.\footnote{\emph{Translator's note}: The modern Perseus edition has \textit{te} instead of \textit{tu}.} (The only example in Callimachus, (\ref{callimachus}), is an exception to the rule.)

\begin{exe}
\ex ποῖ {τυ} λαβὼν ἄξω καὶ ποῖ {τυ} καθίζω\\
\gll poî \emph{tu} labṑn áxō kaì poî \emph{tu} kathízō\\
where you.\textsc{acc} receive.\textsc{ptcp.aor.m.nom.sg} carry.\textsc{1sg.fut} and where you.\textsc{acc} place.\textsc{1sg.prs}\\
\trans `Where do I carry you and where do I place you?' (Stephanus Byzantinus 73.14 M)
\label{oracle}
\end{exe}

\begin{exe}
\ex μή {τύ} τιϲ ἠρώτη\\
\gll mḗ \emph{tú} tis ērṓtē\\
not you.\textsc{acc} someone.\textsc{m.nom.sg} ask.\textsc{3sg.imp}\\
\trans `one shouldn't ask you' (Theocritus 5.74)
\label{theotu1}
\end{exe}

\begin{exe}
\ex ἁ δέ {τυ} κώρα πάϲαϲ ἀνὰ κράναϲ, πάντ᾽ ἄλϲεα ποϲϲὶ φορεῖται ... ζατεῦϲ(α)\\
\gll ha dé \emph{tu} kṓra pásas anà kránas, pánt' álsea possì phoreîtai zateûs(a)\\
the.\textsc{f.nom.sg} but you.\textsc{acc} girl.\textsc{nom.sg} all.\textsc{f.acc.pl} up spring.\textsc{acc.pl} all.\textsc{n.acc.pl} grove.\textsc{acc.pl} foot.\textsc{dat.pl} carry.\textsc{3sg.prs.pass} seek.\textsc{ptcp.prs.f.nom.sg}\\
\trans `And the girl is borne on foot through all springs, all groves, seeking you.' (Theocritus 1.82)
\label{theotu2}
\end{exe}

\begin{exe}
\ex οὐδ᾽ ὅϲον ἀττάραγόν {τυ} δεδοίκαμεϲ\\
\gll oud' hóson attáragón \emph{tu} dedoíkames\\
neither as.much.\textsc{m.acc.sg} crumb.\textsc{acc.sg} you.\textsc{acc} fear.\textsc{1pl.pf}\\
\trans `and you couldn't fear the smallest thing' (Callimachus, Epigram 47.9 (46.9))
\label{callimachus}
\end{exe}

Finally, the only example that I have to hand from an inscription\is{inscriptions} is particularly remarkable: (\ref{inscriptiontu}) (=Attic\il{Greek, Attic} \emph{eán se hugiâ p}...), in which \textit{tu} occurs between the \isi{particles} \textit{aí} and \textit{ka}, which are otherwise closely connected. The only exceptional example from the pre-Alexandrine era, (\ref{sophrontu}), cannot be taken as a weighty counterexample as long as the reading cannot be established with any certainty.

\begin{exe}
\ex αἴ {τύ} κα ὑγιῆ ποιήϲω\\
\gll aí \emph{tú} ka hugiê poiḗsō\\
if you.\textsc{acc} \textsc{irr} healthy.\textsc{n.acc.pl} make.\textsc{1sg.aor.sbjv}\\
\trans `if I made healthy {[}things{]} to/for you' (Inscription 3339.70 Collitz, Epidauros)
\label{inscriptiontu}
\end{exe}

\begin{exe}
\ex οὐχ ὁδεῖν {τυ} ἐπίκαζε\\
\gll oukh hodeîn \emph{tu} epíkaze\\
not sell.\textsc{prs.inf} you.\textsc{acc} guess.\textsc{3sg.imp}\\
\trans `He did not suppose you to sell' (Sophron in Apollonius Dyscolus, \textit{De Pronominibus} 75A)
\label{sophrontu}
\end{exe}

The Aeolic\il{Greek, Aeolic} poets\is{poetry} also show a close affinity to Homer. In the fragments of their poetry,\is{poetry} which I cite following \citet{Bergk1882}, I count 38 (or, depending on the reading of Sappho Fragment 2.7 and Fragment 100 -- see the immediately \hyperlink{p345}{\emph{[p345]}} following -- 36) examples of the enclitic forms of personal\is{personal pronouns} pronouns.\footnote{\emph{Translator's note}: Our rendering of these examples is based on \citet{LobelPage1968}, whose numbering is added for convenience. Translations are adapted from \url{http://www.sacred-texts.com/cla/usappho/index.htm}. Note that not all of these examples are still attributed to Sappho.} 30 follow the Homeric rule, including 12 safe examples of \textit{me} and 10 of \textit{moi}. \textit{toi} behaves exceptionally three times (Sappho 2.2, 8, 70.1) and \textit{se} once (Sappho 104.2). There remain three examples with contested readings, for which I give the manuscript versions: (\ref{sappho1}), (\ref{sappho2}), and finally (\ref{sappho3}) following the fuller wording in Choricius \citep[97]{Graux1886}. 

\begin{exe}
\ex ὡϲ γάρ ϲ᾽ ἴδω βροχεώϲ {με} φωνὰϲ οὐδὲν ἔτ᾽ εἴκει\\
\gll hōs gár s' ídō brokheṓs \emph{me} phōnàs oudèn ét' eíkei\\
as for you.\textsc{acc} see.\textsc{1sg.aor.sbjv} shortly me.\textsc{acc} sound.\textsc{gen.sg} nothing.\textsc{acc} still resemble.\textsc{3sg.plpf}\\
\trans `As I saw you there soon seemed nothing left of my voice.' (Sappho, Fragment 2.7, \citealp[31.7]{LobelPage1968})\footnote{\emph{Translator's note}: \citet{LobelPage1968} have \textit{hōs gár és s' ídō brokhe' ṓs me phōnais' oud' èn ét' eíkei}.}
\label{sappho1}
\end{exe}

\begin{exe}
\ex ὄτα πάννυχοϲ {ἄσφι} κατάγρει\\
\gll óta pánnukhos \emph{ásphi} katágrei\\
when all-night.\textsc{m/f.nom.sg} them.\textsc{dat} overcome.\textsc{3sg.prs}\\
\trans `... when they are overcome all night ...' (Sappho, Fragment 43, \citealp[149.1]{LobelPage1968})
\label{sappho2}
\end{exe}

\begin{exe}
\ex ϲὲ τετίμηκεν ἐξόχωϲ ἡ Ἀφροδίτη\\
\gll sè tetímēken exókhōs hē Aphrodítē\\
you.\textsc{acc} honour.\textsc{3sg.pf} prominently the.\textsc{f.nom.sg} Aphrodite.\textsc{nom}\\
\trans `Aphrodite has honoured you especially' (Sappho, Fragment 100 {[}Choricius 5.1.19{]}; \citealp[97]{Graux1886})
\label{sappho3}
\end{exe}

In the first case, (\ref{sappho1}), the reading \textit{hṓs se gàr wídō ...} suggested by \citet[360]{Ahrens1839} and promoted by \citeauthor{Vahlen1887} in his edition of the text \textit{Perì Hýpsous} `On the Sublime' \citep[section x.2]{Vahlen1887} becomes more plausible, and the reading of \citet{Seidler1829}, followed by \citet{Bergk1854} and \citet{Hiller1890}, in which \textit{se} is moved to a position after \textit{brokheṓs} and \textit{me} is deleted, appears less plausible. In the second case, I can advocate the reading I suggested in \citet[141]{Wackernagel1887} (given in (\ref{sappho2wack}) below) with even more certainty. And in the third case, Weil's\ia{Weil, Henri} reading\footnote{\emph{Translator's note}: personal communication to Charles Graux, reproduced in \citet[98]{Graux1886}.} (given in (\ref{sappho3weil}) below), followed by \citet[Fragment 97]{Hiller1890}, is revealed to be distinctly improbable.

\begin{exe}
\ex ὄτά {ϲφι} πάννυχοϲ κατάγρειϲ\\
\gll ótá \emph{sphi} pánnukhos katágreis\\
when them.\textsc{dat} all-night.\textsc{m/f.nom.sg} overcome.\textsc{2sg.prs}\\
\trans `... when you overcome them all night ...' (Sappho 43, following \citealp[141]{Wackernagel1887}, \citealp[149.1]{LobelPage1968})
\label{sappho2wack}
\end{exe}

\begin{exe}
\ex τετίμακ᾽ ἔξοχά ϲ᾽ Ἀφροδίτα\\
\gll tetímak' éxokhá \emph{s'} Aphrodíta\\
honour.\textsc{3sg.pf} prominently you.\textsc{acc} Aphrodite.\textsc{nom}\\
\trans `Aphrodite has honoured you especially' (Sappho 100, \citealp[Fragment 97]{Hiller1890}, \citealp[112.5]{LobelPage1968})
\label{sappho3weil}
\end{exe}

By adding up the 30 cases discussed above, the \textit{se} and \textit{me} in Sappho 47, and the \textit{sphi} for \textit{ásphi} in Sappho 43, we reach 33 law-abiding examples against 4 exceptions and one (Sappho 100) where the textual transmission leaves us with a problem and we do not even know whether we are dealing with an enclitic.\is{enclitics} We take no account of Alcaeus 68, which some read as (\ref{alcaeus}) following \citet[175]{Bekker1833}, but in which \textit{d'} is much more robustly attested after \textit{ék}; compare Bergk's \citeyearpar[174]{Bergk1882} objections to \citeauthor{Bekker1833}'s reading.

\begin{exe}
\ex πάμπαν δὲ τυφὼϲ ἔκ ϲ᾽ ἕλετο φρέναϲ\\
\gll pámpan dè tuphṑs ék s' héleto phrénas\\
altogether but fever.\textsc{nom.sg} out you.\textsc{acc} take.\textsc{3sg.aor.mid} mind.\textsc{acc.pl}\\
\trans `and a fever has completely taken your wits' (Alcaeus, \textit{Fragments} 68, \citealp[336.1]{LobelPage1968})\footnote{\emph{Translator's note}: \citet[336.1]{LobelPage1968} have \textit{d' etúphōs} rather than \textit{dè tuphṑs}.}
\label{alcaeus}
\end{exe}

In some of the above 33 examples the enclitic pronoun breaks up a word group. The article and the noun are separated in (\ref{sappho4}) and (\ref{sappho5}).

\begin{exe}
\ex ἀ δέ μ᾽ ἰδρὼϲ ... κακχέεται\\
\gll a dé \emph{m'} idrṑs kakkhéetai\\
the.\textsc{f.nom.sg} but me.\textsc{acc} sweat.\textsc{nom.sg} pour-down.\textsc{3s.prs.pass}\\
\trans `down courses in streams the sweat of emotion' (Sappho 2.13, \citealp[31.13]{LobelPage1968})\footnote{\emph{Translator's note}: \citet[31.13]{LobelPage1968} have \textit{ékade} rather than \textit{a dé}.}
\label{sappho4}
\end{exe}

\begin{exe}
\ex Αιθοπίᾳ {με} κόρᾳ Λατοῦϲ ἀνέθηκεν Ἀρίϲτα\\
\gll Aithopíāi \emph{me} kórāi Latoûs anéthēken Arísta\\
Ethiopian.\textsc{f.dat.sg} me.\textsc{acc} girl.\textsc{dat.sg} Leto.\textsc{gen} dedicate.\textsc{3sg.aor} Aristas.\textsc{nom}\\
\trans `Aristas dedicated me to Leto's Ethiopian daughter' (Sappho 118.3, \citealp[Epigrammata 6.269]{LobelPage1968})
\label{sappho5}
\end{exe}

Adjective\is{adjectives} and noun are separated in (\ref{sappho6}). In (\ref{alcaeus2}), preposition\is{prepositions} and verb are separated.

\begin{exe}
\ex ϲμίκρα {μοι} πάϊϲ ἔμμεν ἐφαίνεο κἄχαριϲ\\
\gll smíkra \emph{moi} páïs émmen ephaíneo kákharis\\
small.\textsc{f.nom.sg} me.\textsc{dat} child.\textsc{nom.sg} be.\textsc{prs.inf} show.\textsc{2sg.imp.pass} and=graceless.\textsc{f.nom.sg}\\
\trans `to me you seemed to be a graceless little girl' (Sappho 34.1, \citealp[59.2]{LobelPage1968})
\label{sappho6}
\end{exe}

\begin{exe}
\ex ἔκ μ᾽ ἔλαϲαϲ ἀλγέων\\
\gll ék \emph{m'} élasas algéōn\\
out me.\textsc{acc} drive.\textsc{2sg.aor} pain.\textsc{gen.pl}\\
\trans `you have driven out my pains' (Alcaeus, \textit{Fragments} 95)
\label{alcaeus2}
\end{exe}

Cf. also (\ref{sappho7}) and (\ref{sappho8}), in which \textit{mán} and \textit{gár} could have laid claim to the position after \textit{tó} and \textit{hṓs} respectively.

\begin{exe}
\ex τό {μοι} μάν\\
\gll tó \emph{moi} mán\\
the.\textsc{n.nom.sg} me.\textsc{dat} truly\\
\trans (Sappho 2.5, \citealp[31.5]{LobelPage1968})\footnote{\emph{Translator's note}: \citet[31.5]{LobelPage1968} have \textit{tó m' ê mán}.}
\label{sappho7}
\end{exe}

\begin{exe}
\ex ὥϲ {ϲε} γάρ\\
\gll hṓs \emph{se} gár\\
as you.\textsc{acc} for\\
\trans (Sappho 2.7, \citealp[31.7]{LobelPage1968})
\label{sappho8}
\end{exe}

Equally noteworthy are the cases in which the pronoun is separated in an otherwise unusual way \hyperlink{p346}{\emph{[p346]}} from the words to which it syntactically belongs: (\ref{sappho9}), (\ref{sappho10}) and (\ref{sappho11}).

\begin{exe}
\ex τίϲ ϲ᾽, ὦ Ψάπφ᾽ ἀδικήει\\
\gll tís \emph{s'}, ô Psápph' adikḗei\\
who.\textsc{m.nom.sg} you.\textsc{acc} O Sappho.\textsc{voc} wrong.\textsc{3sg.prs}\\
\trans `Who has wronged you, O Sappho?' (Sappho 1.19, \citealp[1.19]{LobelPage1968})
\label{sappho9}
\end{exe}

\begin{exe}
\ex τίῳ ϲ᾽, ὦ φίλε γάμβρε, κάλωϲ ἐϊκάϲδω\\
\gll tíōi \emph{s'}, ô phíle gámbre, kálōs eïkásdō\\
what.\textsc{n.dat.sg} you.\textsc{acc} O dear.\textsc{m.voc.sg} in-law.\textsc{voc.sg} beautifully liken.\textsc{1sg.prs.sbjv}\\
\trans `To what, O dear bridegroom, shall I fairly compare thee?' (Sappho 104.1, \citealp[115.1]{LobelPage1968})
\label{sappho10}
\end{exe}

\begin{exe}
\ex τί {με} Πανδίονιϲ ὤραννα χελίδων\\
\gll tí \emph{me} Pandíonis ṓranna khelídōn\\
what.\textsc{n.nom.sg} me.\textsc{acc} of-Pandion.\textsc{f.nom.sg} O=Irene.\textsc{voc} swallow.\textsc{nom.sg}\\
\trans `What is that daughter of Pandion, the swallow, to me, Irene?' (Sappho 88, \citealp[135.1]{LobelPage1968})
\label{sappho11}
\end{exe}

In (\ref{sappho12}), \textit{moi} leans on a clause-introducing \isi{vocative}. Finally, I invite you to look at (\ref{sappho13}).

\begin{exe}
\ex ἄγε δὴ, χέλυ δῖά, {μοι} φωνάεϲϲα γένοιο\\
\gll áge dḕ, khélu dîá, \emph{moi} phōnáessa génoio\\
lead.\textsc{2sg.prs.imper} exactly lyre.\textsc{voc.sg} divine.\textsc{f.voc.sg} me.\textsc{dat} vocal.\textsc{f.nom.sg} become.\textsc{2sg.aor.opt.mid}\\
\trans `Come now, O divine lyre, begin to sing for me' (Sappho 45, \citealp[118.1]{LobelPage1968})
\label{sappho12}
\end{exe}

\begin{exe}
\ex ἤ {ϲε} Κύπροϲ ἢ Πάφοϲ ἢ Πάνορμοϲ\\
\gll ḗ \emph{se} Kúpros ḕ Páphos ḕ Pánormos\\
or you.\textsc{acc} Cyprus.\textsc{nom} or Paphos.\textsc{nom} or Panormus.\textsc{nom}\\
\trans (Sappho 6, \citealp[35.1]{LobelPage1968})
\label{sappho13}
\end{exe}

It is the general norm, without dialectal differentiation, to place the archaic \citep[13]{Klein1887} \textit{me} (\textsc{1sg.acc}) immediately after the first word in dedicatory and sculptors' inscriptions.\is{inscriptions|(} It will be useful to provide a full list of examples.\label{forAddenda1}\footnote{\emph{Translator's note}: Wackernagel indeed provides a full list of examples in the original, pp\hyperlink{p346}{346}--\hyperlink{p349}{9}. As these all serve to illustrate the same point, we have not glossed and translated all of them, taking only a representative example in each case.}

I begin with \emph{m' anéthēke} `me.\textsc{acc} dedicate'. (\ref{manetheke}) is an Attic\il{Greek, Attic} example.

\begin{exe}
\ex Ὀνήϲιμος μ᾽ ἀνέθηκεν ἀπαρχὴν τἀθηναίᾳ ὁ Σμικύθου υἱόϲ\\
\gll Onḗsimos \emph{m'} anéthēken aparkhḕn tathēnaíāi ho Smikúthou huiós\\
Onesimus.\textsc{nom} me.\textsc{acc} dedicate.\textsc{3sg.aor} offering.\textsc{acc} the=Athenaea.\textsc{dat} the.\textsc{m.nom.sg} Smikythus.\textsc{gen} son.\textsc{nom.sg}\\
\trans `Onesimos, the son of Smikythus, dedicated an offering to the Athenaea.' (\emph{Corpus inscriptionum atticarum} (CIA) 4$^2$.373.90)
\label{manetheke}
\end{exe}

Also CIA 4$^2$.373.87 \textit{-itos \emph{m'} anéthēken}, CIA 4$^2$.373.120 {[}\textit{ho deîna}{]} \textit{\emph{m'} anéthēken dekáthēn} (sic!) \textit{Athēnaíāi}, Inscriptiones graecae antiquissimae (IGA; \citealp{Roehl1882}) 1 (Attic\il{Greek, Attic} or Euboean) \textit{Sēmōnides \emph{m'} anéthēken}. Cf. CIA 4$^2$.373.100 {[}\textit{Strón}{]}\textit{gulós \emph{m'} anéthēke}, in which a \isi{dative} precedes, however. Many examples also in verse\is{poetry} (although there is of course no absence of counterexamples\label{versedeviations} here: CIA 1.343, CIA 1.374, 4$^2$.373.81 etc.): CIA 1.349 \textit{-thánēs \emph{m'} anéthēken Athēnaía}{[}\textit{i polioúkhps}{]}, 1.352 \textit{Iphidíkē \emph{m'} anéthēken}, 4$^2$.373.85 \textit{Alkímakhós \emph{m'} a}{[}\textit{néthēke}{]}, 4$^2$.373.99 \textit{Tímarkhós \emph{m'} anéthēke Diòs krateróphroni koúrēi}, 4$^2$.373.215 (cf. \citealp[145]{Studniczka1887}) \textit{Nēsiadēs kerameús \emph{me} kaì Andokídēs anéthēken}, 4$^2$.373.216 \textit{Palládi \emph{m'} egremákhāi Dionúsio}{[}\textit{s tó}{]}\textit{d' ágalma stêse Koloíou paîs }{[}\textit{euxá}{]}\textit{menos dekátēn}, 4$^2$.373.218 \textit{ané\-thē\-ke dé \emph{m'} Eudíkou huiós}, Acropolis inscription (ed. \citealp[160]{Foucart1889}) {[}\textit{Hermó}?{]}\textit{dō\-rós \emph{m'} anéthēken Aphrodítēi dôron aparkhḗn}. -- From Boeotia: inscription from \citet{Reinach1885} treated by \citet[123--125]{Kretschmer1891}, \textit{Timasíphilós \emph{m'} anétheike tōpóllōni toî Ptōieîi ho Praólleios}. -- From Corinth (in the following I will no longer distinguish between poetic\is{poetry} and \isi{prose} inscriptions): IGA 20.7 \textit{Simíōn \emph{m'} anéthēke Poteidáwōn}{[}\textit{i wánakti}{]}, 20.8 \textit{-ōn \emph{m'} anéthēke Poteidâni wán}{[}\textit{akti}{]}, 20.9(=10=11) \textit{Phlēbōn \emph{m'} anéthēke Poteidâ}{[}\textit{ni}{]}, 20.42 \textit{Dórkōn \emph{m'} anéthēk}{[}\textit{e}{]}, 20.43 \textit{Igrōn \emph{m'} an}{[}\textit{é\-thē\-ke}{]}, 20.47 \textit{Kuloídas \emph{m'} anéthēke}, 20.48 \textit{Eurumḗdēs \emph{m'} anéthēke}, 20.49 \textit{Lukiádas \emph{m'} }{[}\textit{anéthēke}{]}, 20.83 \textit{... \emph{m'} anéth}{[}\textit{ēke}{]}, 20.87 and 20.89 \textit{-s \emph{m'} anéthēke}, 20.87a \textit{... \emph{me} anéth(ē)ke tȭi}
, 20.94 \textit{... \emph{m'} anéthēke}, 20.102 {[}\textit{P}{]}\textit{érilós \emph{m'} ...}. -- Korkyra: IGA 341 (=3187 Collitz) \textit{Lóphiós \emph{m'} anéthēke}. \hyperlink{p347}{\emph{[p347]}} Hermione: \citet{Kaibel1878} 926 {[}\textit{Pan}{]}\textit{taklês \emph{m'} anéthēken}. -- Kyra at Aegina: Inscription (ed. \citealp[186]{Jamot1889}) \textit{hoi phrouroí \emph{m'} a}{[}\textit{néthesan}?{]} -- Laconia: IGA 62a (p174) \textit{Pleistiádas \emph{m'} a}{[}\textit{néthēke}{]}. -- Naxos: IGA 407 \textit{Nikándrē \emph{m'} anéthēken hekēbólōi iokheaírēi}, 408 \textit{Deinagórēs \emph{m'} anéthēken hekēbólōi Apóllōni}. -- Inscription found in Delos edited by \citet[464f.]{Homolle1888} \textit{Eì(th)ukartídēs \emph{m'} anéthēken ho Náxios poiēsas}. -- Samos: IGA 384 \textit{Khēramúes \emph{m'} anéth(ē)ken tḗrēi ágalma}.\label{IGA384} \citet[108]{Roehl1882} adds {[}\textit{Entháde}{]} at the beginning and observes: ``For now I leave aside the question as to whether the first word of the hexameter poem\is{poetry} was omitted by the person who made the inscription or by the one who copied its title''. It was certainly neither. Not the creator of the copy: Dümmler\ia{Dümmler, Ferdinand} (p.c.) points out to me that the copy he saw showed no trace of a word before \textit{Khēramúes}. But nor could it have been the mason: neither the sense nor (as we now know better than we did ten years ago) the metre demanded any additional material, and the placement of \textit{me} excludes any such addition. -- Kalymna: \citet{Kaibel1878} 778 \textit{Nikías \emph{me} anéthēken Apóllōni huiòs Thrasumḗdeos}. -- Cyprus: inscription in \citet[85]{Hoffmann1891} no. 163 \textit{(...) \emph{m'} a(né)thēkan tȭi Apól(l)ōni}, \citet{Kaibel1878} 794 (1st century CE) {[}\textit{Kekro}{]}\textit{pídēs \emph{m'} anéthēke}. -- Achaean (Magna Grecia): IGA 543 \textit{Kunískos \emph{me} anéthēken hṓrtamos wérgōn dekátan}. -- Syracuse: \citet{Kaibel1890} 5 \textit{Alkiádēs \emph{m'} }{[}\textit{anéthēken}{]}. -- Naukratis: \citet[60--63]{Gardner1886} no. 5 \textit{Parménōnm} (sic!) \textit{\emph{me} anéthēke tōppóllōni} (sic!), 24 \textit{-s \emph{me} a}{[}\textit{néthēke}{]}, 80 \textit{-s \emph{m'} anéthēken tōpollōn}{[}\textit{i}{]}, 114 \textit{-ōn \emph{m}}{[}\textit{e anéthēke}{]}, 137 \textit{-s \emph{m'} an}{[}\textit{éthēke}{]}, 177 \textit{Prṓtarkhós \emph{me} }{[}\textit{anéthēke t}{]}\textit{ōpóllōni}, 186 {[}\textit{P}{]}\textit{rṓtarkhós \emph{me} anéthēk}{[}\textit{e}{]}, 202 {[}\textit{ho deîna}{]} \textit{\emph{me} anéthēke}, 218 \textit{Phánēs \emph{me} anéthēke tōpóllōn}{[}\textit{i tôi Mi}{]}\textit{lēsíōi ho Glaúkou}, 220 \textit{Kharidíōn \emph{me} anéthē}{[}\textit{ke}{]}, 223 {[}\textit{Polú}{]}\textit{kestós \emph{m'} anéthēke t}{[}\textit{ōpóllōni}{]}, 235 \textit{Slēúēs \emph{m'} anéthēke tōpóllōni}, 237 {[}\textit{Kh}{]}\textit{ar(ó)phēs \emph{me} anéthēke tapó}{[}\textit{l\-lōni tôi M}{]}\textit{ilasíōi}, 255 \textit{-ēs \emph{m'} anéthēke}, 259 \textit{-s \emph{me} a}{[}\textit{néthēke}{]}, 326 \textit{Na}{[}\textit{úpli}{]}\textit{ós \emph{me}} {[}\textit{anéthēke}{]}, 327 \textit{-dēs \emph{m'} anéthēke tōpóllōni}, 446 \textit{-s \emph{me} ané}{[}\textit{thēke}{]}, vol. II \citep[62--29]{Gardner1888} no. 701 \textit{Sṓstratós \emph{m'} anéthēken tēphrodítēi}, 709 \textit{-os \emph{m'} anéthēke tê}{[}\textit{i Aphrodítēi}{]} \textit{epì tê ...}, 717 \textit{Kaîkos \emph{m'}} {[}\textit{ané}{]}\textit{thēken}, 720 \textit{-oros \emph{m'} an}{[}\textit{éthēken}{]}, 722 \textit{Musós \emph{m'} anéthēken Honomakrítou}, 723 \textit{Asos} \hyperlink{p348}{\emph{[p348]}} \textit{\emph{m'} anéthēken}, 734 \textit{-nax \emph{m'}} {[}\textit{anéthēken}{]}, 736 \textit{-ōn \emph{me} an}{[}\textit{éthēken}{]}, 738 {[}\textit{ho deîna}{]} \textit{\emph{m'} anéthēken Aphrodítēi} (?), 742 \textit{-ēilos \emph{m'} anéthēken}, 748 \textit{Hermēsiphánēs \emph{m'} anéthēken tēphrodítēi}, 770 \textit{-mēs \emph{me} an}{[}\textit{éthēke t}{]}\textit{ēphrodítē}{[}\textit{i}{]},
771 \textit{Khárm}{[}\textit{ē}{]}\textit{s \emph{me}} {[}\textit{anéthēke}{]}, 775 {[}\textit{K}{]}\textit{leódēmos \emph{me} a}{[}\textit{ne}{]}\textit{thēke tôi A}{[}\textit{phrodítēi}{]}, 776--777 \textit{Khármēs \emph{me} anéthēke tēphrodítēi} (or \textit{têi A-}) \textit{eukhōlēn}, 778 \textit{Roîkos \emph{m'} anéthēke t}{[}\textit{êi Aphr}{]}\textit{odítēi}, 780 \textit{Philís \emph{m'} anéthēke t}{[}\textit{êi Aphr}{]}\textit{odí}{[}\textit{tēi}{]}, 781 \textit{Thoútimós \emph{me} anéthēk}{[}\textit{en}{]}, 785 {[}\textit{ho deîna}{]} \textit{\emph{m'} an}{[}\textit{éthēke têi Aphr}{]}\textit{odítēi}, 794 \textit{Polúermós \emph{m'} an}{[}\textit{éthēke}{]} \textit{têi Aphrodítēi}, 799 \textit{Ōkhílos \emph{m'} anéthēke}, 817 {[}\textit{ho deîna}{]} \textit{kaì Kh}{[}\textit{rus}{]}\textit{ódōrós \emph{me} anéth}{[}\textit{ēkan}{]}, 819 {[}\textit{L}{]}\textit{ákri}{[}\textit{tó}{]}\textit{s \emph{m'} ané}{[}\textit{thē}{]}\textit{ke hourmo}{[}\textit{th}{]}\textit{ém}{[}\textit{ios}{]} \textit{tēphrodí}{[}\textit{tēi}, 876 \textit{Hermagórēs \emph{m'} anéthēke ho T}{[}\textit{ḗios}{]} \textit{tōpóllōni} (verse!),\is{poetry} 877 \textit{Púr}{[}\textit{rh}{]}\textit{os \emph{me} anéthēken}. (Metapontum: 1643 Collitz \textit{hó toi kerameús \emph{m'} anéthēke}.)
 
The only deviations from the norm (with some poetic\is{poetry} exceptions, see above p\pageref{versedeviations}) are Naukratis 1.303 {[}\textit{ho deîna anéthēké}{]} \textit{\emph{me}} and 1.307 {[}\textit{ho deîna anéthēk}{]}\textit{é \emph{me}}\label{naukratisx2} -- both inscriptions which have been falsely expanded, as is now clear -- and the two-line inscription Naukratis 2.750, in which the first line reads {[}\textit{têi Aphrodí}{]}\textit{tēi} and the second \textit{Hermagathînós \emph{m'} anéth}{[}\textit{ēken}{]}.\label{Naukratis2750} \citet{Gardner1888} thus gives the reading \textit{têi Aphrodítēi Hermagathînós \emph{m'} anéthēken}. However, Dümmler\ia{Dümmler, Ferdinand} (p.c.) points out to me that the top line cannot be the first line, because it is shorter and does not fill the space, and hence must instead have been the conclusion of the lower, longer line. As a consequence it is necessary to read \textit{Hermagathînós \emph{m'} anéth}{[}\textit{ēke}{]} {[}\textit{tēi Aphrodí}{]}\textit{tēi}, quite independently of our positional rule.

Something quite analogous is true of the inscriptions formed with synonyms of \textit{anéthēke}. \textit{me katéthēke} `me.\textsc{acc} down-lay': Cyprus: \citet{Deecke1884} 1 \textit{Kás \emph{mi} katéthēke tâi Paphíai Aphrodítai}, and (\ref{katetheke}).

\begin{exe}
\ex αὐτάρ μι κατέ{[}θηκε{]} Ὀναϲίθεμιϲ\\
\gll autár \emph{mi} katé{[}thēke{]} Onasíthemis\\
besides me.\textsc{acc} down-lay.\textsc{3sg.aor} Onasithemis\\
\trans `Besides, Onesithemis laid me down' (Cyprus, \citealp{Deecke1884}, 2)
\label{katetheke}
\end{exe}

Also \citet{Deecke1884} 3 \textit{autár \emph{me}} {[}\textit{katéthēke Onasí}{]}\textit{themi}{[}\textit{s}{]} and 15 \textit{autár \emph{me} katéthēke} {[}\textit{A}{]}\textit{kestóthemis}. -- Naukratis II \citep{Gardner1888} no. 790 {[}\textit{ho deîna \emph{m}}{]}\textit{\emph{e} káththē}{[}\textit{ke}{]} \textit{o Mutilḗnaios}, 840 \textit{Néarkhós \emph{me} ká}{[}\textit{ththēke to}{]}\textit{îs D}{[}\textit{ioskóroisi}{]}. -- \textit{m' epéthēke} `me.\textsc{acc} on-put': Aegina: IGA 362 \textit{Diótimós \emph{m'} epéthēke}. -- \textit{me (kat)éstase} `me.\textsc{acc} erect': Cyprus: \citet{Deecke1884} 71 \textit{ká \emph{men} éstasan} {[}\textit{ka}{]}\textit{sígnētoi} (verse!),\is{poetry} \citet[46]{Hoffmann1891} no. 67 \textit{Gil(l)íka \emph{me} katéstase ho Stasikréteos}. -- \textit{me éwexe} `me.\textsc{acc} grant': Cyprus: \citet[46]{Hoffmann1891} no. 66 {[}\textit{au}{]}\textit{tár \emph{me} éwexe} {[}\textit{Onasí}{]}\textit{the\-mis}. -- \textit{m' édōke} `me.\textsc{acc} give/grant/allow': Sicyon: IGA 22 \textit{Epaínetos \emph{m'} édōken Kharópōi}. The Boeotian inscription (\ref{failed}) deviates from the rule.

\begin{exe}
\ex Χάρηϲ ἔδωκεν Εὐπλοίωνί με\\
\gll Khárēs édōken Euploíōní \emph{me}\\
Charis.\textsc{nom} give.\textsc{3sg.aor} Euplion.\textsc{dat} me.\textsc{acc}\\
\trans `Charis gave me to Euplion' (IGA 2019)
\label{failed}
\end{exe}

\citet[56]{Roehl1882} comments as follows: ``Chares attempted to include a dedication in the form of a trimeter verse,\is{poetry} but his attempt failed.'' (Compare also the \hyperlink{p349}{\emph{[p349]}} position of \textit{soi} (\textsc{2sg.dat}) in the Attic\il{Greek, Attic} inscription IGA 2 \textit{tēndí \emph{soi} Thoúdēmos dídōsi}.)

In poetic\is{poetry} dedicatory inscriptions, \textit{me} is found in this position as late as the Roman era: (\ref{kaibelex1})--(\ref{kaibelex3}). Compare also (\ref{kaibelex4}). (\citet{Kaibel1878} 809, 813 and 843 have a different position for \textit{me}.)

\begin{exe}
\ex Βάκχῳ {μ{[}ε{]}} Βάκχον καὶ προϲυμναίᾳ θεῷ {ϲτάϲαντο}\\
\gll Bákkhōi \emph{m{[}e{]}} Bákkhon kaì prosumnaíāi theôi {stásanto}\\
Bacchus.\textsc{dat} me.\textsc{acc} Bacchus.\textsc{acc} and Prosymnian.\textsc{f.dat.}sg goddess.\textsc{dat.sg} set-up.\textsc{3pl.aor.mid}\\
\trans `To Bacchus (= Dionysus) and to the goddess praised in hymns (=Demeter); they set me up' (\citet{Kaibel1878}, 821)
\label{kaibelex1}
\end{exe}

\begin{exe}
\ex Δᾳδοῦχοϲ {με} Κόρηϲ, Βαϲιλᾶν, Διόϲ, ἱερὰ ϲηκῶν Ἥραϲ κλείθρα φέρων βωμὸν {ἔθηκε} Ῥέῃ\\
\gll Dāidoûkhos \emph{me} Kórēs, Basilân, Diós, hierà sēkôn Hḗras kleíthra phérōn bōmòn {éthēke} Rhéēi\\
torch-bearer me.\textsc{acc} Kore.\textsc{gen} queen.\textsc{acc} Zeus.\textsc{gen} sacred.\textsc{n.acc.pl} precinct.\textsc{gen.pl} Hera.\textsc{gen} key.\textsc{acc.pl} bear.\textsc{ptcp.prs.m.nom.sg} altar.\textsc{acc.sg} put.\textsc{3sg.aor} Rhea.\textsc{dat}\\
\trans `The torch-bearer of Kore, bearing the sacred keys to the sanctuary of Queen Hera, has dedicated me, the altar, to Rhea' (\citet{Kaibel1878}, 822.9)
\label{kaibelex2}
\end{exe}

\begin{exe}
\ex {ἄνθετο} μεν μ᾽ Ἐπίδαυροϲ\\
\gll {ántheto} men \emph{m'} Epídauros\\
dedicate.\textsc{3sg.aor.mid} then me.\textsc{acc} Epidaurus\\
\trans `Epidaurus dedicated me' (\citet[XIX]{Kaibel1878}, 877b)
\label{kaibelex3}
\end{exe}

\begin{exe}
\ex Ἀϲκληπιοῦ {με} δμῶα πυρφόρο{[}ν θεοῦ/ξένε{]} Πείϲωνα {λεύϲϲειϲ}\\
\gll Asklēpioû \emph{me} dmôa purphóro{[}n theoû /xéne{]} Peísōna {leússeis}\\
Asclepius.\textsc{gen} me.\textsc{acc} slave.\textsc{acc.sg} fire-bearing.\textsc{m.acc.sg} god.\textsc{gen.sg} stranger.\textsc{voc.sg} Peison.\textsc{acc} see.\textsc{2sg.prs}\\
\trans `Behold me, Peison, the fire-bearing slave of the god Asclepius(/of Asclepius, O stranger)' (\citet{Kaibel1878}, 868)
\label{kaibelex4}
\end{exe}

The artists' inscriptions behave the same.\label{forAddenda3} \emph{m' epoíēse, m' epoíei} `me.\textsc{acc} create': (\ref{me1}).

\begin{exe}
\ex {[}Ε{]}ὐθυκλῆϲ μ᾽ ἐποίηϲεν\\
\gll {[}E{]}uthuklês \emph{m'} epoíēsen\\
Euthycles.\textsc{nom} me.\textsc{acc} create.\textsc{3sg.aor}\\
\trans `Euthycles created me' (CIA 4$^2$ 373.206)
\label{me1}
\end{exe}

Also IGA 492 (Attic\il{Greek, Attic} inscription from Sigeum) \textit{kaì \emph{m'} epo(íē)sen Haísōpos kaì hadelphoí}, CIA 1.466 \textit{Aristíōn \emph{m'} epoēsen}, 1.469 (cf. \citealp[15]{Loewy1885}) \textit{Aristíōn Pári}{[}\textit{ós \emph{m'} ep}{]}\textit{ó}{[}\textit{ēs}{]}\textit{e} (the emendation\is{emendation} is certain!), IGA 378 (Thasos) \textit{Parménōn \emph{me} e}{[}\textit{poíēse}{]}, IGA 485 (Miletus) \textit{Eúdēmós \emph{me} epoíein}, IGA 557 (Elis?) \textit{Koîós \emph{m'} apóēsen}, IGA 22 (=\citealp[40]{Klein1887}) \textit{Eksēkias \emph{m'} epoíēse}, \citet[41]{Klein1887} \textit{Eksēkias \emph{m'} epoíēsen eû}, \citet[31]{Klein1887} \textit{Theózotós \emph{m'} epoēse}, \citet[34]{Klein1887} \textit{Ergótimós \emph{m'} epoíēsen}, \citet[43, 45 b, 48]{Klein1887} \textit{Amasís \emph{m'} epoíēsen}, \citet[48]{Klein1887} \textit{Khólkhos \emph{m'} epoíēsen}, \citet[66]{Klein1887} \textit{-s \emph{m'} epoíēsen}, \citet[71]{Klein1887} \textit{Nikosthénes \emph{m'} epoíēsen}, \citet[75]{Klein1887} \textit{Anaklês \emph{me} epoíēsen} and \textit{Nikosthénes \emph{me} epoíēsen}, \citet[76]{Klein1887} \textit{Arkheklês \emph{m'} epoíēsen}, \citet[77]{Klein1887} \textit{Glaukítēs \emph{m'} epoíēsen}, \citet[84 b]{Klein1887} \textit{Tlēnpólemós \emph{m'} epoíēsen}, \citet[85]{Klein1887} \textit{Gageos \emph{m'} epoíēsen}, \citet[90]{Klein1887} \textit{Panphaîós \emph{m'} epoíēsen}, \citet[213]{Klein1887} \textit{Lusías \emph{m'} epoíēsen hēmikhṓnēi}, as well as the metrical inscription IGA 536 {[}\textit{Glaukía}{]}\textit{i \emph{me} Kálōn ge}{[}\textit{neâi w}{]}\textit{aleî}{[}\textit{o}{]}\textit{s epoíei}. On the other hand, \citet[281]{Loewy1885} no. 411 {[}\textit{Arté}{]}\textit{mōn \emph{me} epoíēse} falls away because of the treatment of the inscription by \citet[7]{Koehler1888} in CIA 2.1181. -- (\ref{rulebreaker}) breaks the rule. Here it is likely that {<}\textit{e}{>}\textit{mé} was either originally present or at least intended.\footnote{\emph{Translator's note}: \textit{emé} is the non-clitic counterpart of \textit{me}, also a first person \isi{accusative} pronoun form.} (On \textit{emé} see below, page \pageref{eme}).

\begin{exe}
\ex Χαριταῖοϲ ἐποίηϲέν με\\
\gll Kharitaîos epoíēsén \emph{me}\\
Charitaeus.\textsc{nom} create.\textsc{3sg.aor} me.\textsc{acc}\\
\trans `Charitaeus created me' \citep[51]{Klein1887}
\label{rulebreaker}
\end{exe}

\emph{m' égrapse, m' égraphe} `me.\textsc{acc} write': IGA 20.102 (Corinth) \textit{-ōn \emph{m'}} {[}\textit{égrapse}{]} following the emendation\is{emendation} by \citet[65]{Blass1888} no. 3119e Collitz. Cypriot\il{Greek, Cypriot} inscriptions in \citet[90]{Hoffmann1891} no. 189 \textit{-oikós \emph{me} gráphei Selamínios}, \citet[29]{Klein1887} \textit{Timōnídas \emph{m'} égraphe}, \citet[30]{Klein1887} \textit{Kharēs \emph{m'} égrapse},
and (\ref{egrapsen}). 

\begin{exe}
\ex Νέαρχόϲ μ᾽ ἔγραψεν καὶ {<}ἐποίηϲεν{>}\\
\gll Néarkhós \emph{m'} égrapsen kaì epoíēsen\\
Nearchus.\textsc{nom} me.\textsc{acc} write.\textsc{3sg.aor} and make.\textsc{3sg.aor}\\
\trans `Nearchus engraved and made me' \citep[38]{Klein1887}
\label{egrapsen}
\end{exe}

IGA 474 (Crete) \textit{-mōn égraphé \emph{me}}\label{IGA474} deviates from the rule, but this exception can be set aside if we accept the reading \textit{égraph' emé}: compare the inscription in \citet[40]{Klein1887} \hyperlink{p350}{\emph{[p350]}} \textit{kapoíēs' emé} with just such an elision,\is{ellipsis} in which \textit{emé} can be read securely because of other instances of the same inscription with \textit{epóēse emé}. (With regard to \textit{me} in inscriptions, see also the \hyperlink{addenda}{Addenda}.)

The inscriptions transmitted to us on stones and vases include some truly ancient ones brought to us from Olympia by Pausanias: (\ref{paus1})--(\ref{paus3}).

\begin{exe}
\ex υἱόϲ μέν με Μίκωνοϲ Ὀνάταϲ ἐξετέλεϲϲεν\\
\gll huiós mén \emph{me} Míkōnos Onátas exetélessen\\
son.\textsc{nom} then me.\textsc{acc} Mikon.\textsc{gen} Onatas.\textsc{nom} complete.\textsc{3sg.aor}\\
\trans `Onatas, the son of Mikon, completed me' (Pausanias 5.25.13 = 8.42.10 from Thasos)
\label{paus1}
\end{exe}

\begin{exe}
\ex Κλεοϲθένηϲ μ᾽ ἀνέθηκεν ὁ Πόντιοϲ ἐξ Ἐπιδάμνου\\
\gll Kleosthénēs \emph{m'} anéthēken ho Póntios ex Epidámnou\\
Kleosthenes.\textsc{nom} me.\textsc{acc} dedicate.\textsc{3sg.aor} the.\textsc{m.nom.sg} of.Pontus.\textsc{m.nom.sg} from Epidamnus.\textsc{gen}\\
\trans `Kleosthenes, the Pontic man from Epidamnus, dedicated me' (Pausanias 6.10.17, fifth century)
\label{paus2}
\end{exe}

\begin{exe}
\ex Ζηνί μ᾽ ἄγαλμ᾽ ἀνέθηκαν\\
\gll Zēní \emph{m'} ágalm' anéthēkan\\
Zeus.\textsc{dat} me.\textsc{acc} statue.\textsc{acc} devote.\textsc{3pl.aor}\\
\trans `They raised me, a statue, for Zeus' (Pausanias 6.19.6, ancient Attic)
\label{paus3}
\end{exe}

F. Dümmler\ia{Dümmler, Ferdinand} (p.c.) emends\is{emendation} (\ref{paus4}) to read \textit{\emph{me} Kleitoríois} `me.\textsc{acc} Cleitorian.\textsc{dat.pl}' in place of \textit{metreît'}.\footnote{\emph{Translator's note}: This yields the translation `And Ariston and Telestas, the Laconian brothers, were good to the Cleitorians for me.'}

\begin{exe}
\ex καὶ μετρεῖτ᾽ Ἀρίϲτων ἠδὲ Τελέϲταϲ αὐτοκαϲίγνητοι καλὰ Λάκωνεϲ *ἔϲαν\\
\gll kaì metreît' Arístōn ēdè Teléstas autokasígnētoi kalà Lákōnes ésan\\
and count.\textsc{3sg.prs.pass} Ariston.\textsc{nom} and Telestas.\textsc{nom} own-brothers.\textsc{nom.pl} well Laconians.\textsc{nom.pl} be.\textsc{3pl.imp}\\
\trans `And Ariston and Telestas were well considered brothers and they were Laconians' (Pausanias 5.23.7, epigram)
\label{paus4}
\end{exe}

The examples brought to us by Herodotus from the Ismenion Hill also belong here: (\ref{hdtMe1}) and (\ref{hdtMe2}), of which the latter is the only counterexample to the rule in this group, and moreover, since it is metrical, is of little consequence.

\begin{exe}
\ex Ἀμφιτρύων μ᾽ ἀνέθηκεν *ἐὼν ἀπὸ Τηλεβοάων\\
\gll Amphitrúōn \emph{m'} anéthēken *eṑn apò Tēleboáōn\\
Amphitryon.\textsc{nom} me.\textsc{acc} devote.\textsc{3sg.aor} be.\textsc{ptcp.m.nom.sg} from Teleboan.\textsc{gen.pl}\\
\trans `Amphitryon, being from Teleboae, dedicated me' (Herodotus, 5.59.1)\footnote{\emph{Translator's note}: The Perseus edition has \emph{Amphitrúōn m' anethēk' enárōn apò tēleboáōn} `Amphitryon dedicated me from the spoils of Teleboae.'}
\label{hdtMe1}
\end{exe}

\begin{exe}
\ex Σκαῖοϲ πυγμαχέων με ἑκηβόλῳ Ἀπόλλωνι νικήϲαϲ ἀνέθηκε\\
\gll Skaîos pugmakhéōn me hekēbólōi Apóllōni nikḗsas anéthēke\\
Scaeus.\textsc{nom} boxer.\textsc{nom.sg} me.\textsc{acc} archer.\textsc{dat.sg} Apollo.\textsc{dat} win.\textsc{ptcp.aor.m.nom.sg} dedicate.\textsc{3sg.aor}\\
\trans `Scaeus the boxer, victorious in the contest, gave me to Apollo, the archer god.' (Herodotus, 5.60.1)
\label{hdtMe2}
\end{exe}

The later epigram-writers also kept to the norm with striking rigidity when they used the archaic \textit{me} in their poetic\is{poetry} inscriptions: (\ref{epi1})--(\ref{epi7}).

\begin{exe}
\ex ὅϲτιϲ ἐμὸν παρὰ σῆμα φέρειϲ πόδα, Καλλιμάχου {με} ἴϲθι Κυρηναίου παῖδά τε καὶ γενέτην\\
\gll hóstis emòn parà sêma phéreis póda, Kallimákhou \emph{me} ísthi Kurēnaíou paîdá te kaì genétēn\\
whoever.\textsc{m.nom.sg} my.\textsc{n.acc.sg} by tomb.\textsc{acc.sg} bring.\textsc{2sg.prs} foot.\textsc{acc.sg} Callimachus.\textsc{gen} me.\textsc{acc} know.\textsc{2sg.pf.imp} of-Cyrene.\textsc{m.gen.sg} child.\textsc{acc.sg} and and offspring.\textsc{acc.sg}\\
\trans `Whoever you are who walks past my tomb, know that I am the son of Callimachus of Cyrene.' (Callimachus, Epigram 23.1 (21.1 Wilamowitz))
\label{epi1}
\end{exe}

\begin{exe}
\ex τίν {με}, λεοντάγχ᾽ ὦνα ϲυοκτόνε, φήγινον ὄζον θῆκε\\
\gll tín \emph{me}, leontánkh' ôna suoktóne, phḗginon ózon thêke\\
you.\textsc{nom} me.\textsc{acc} lion-strangling.\textsc{m.voc.sg} O=lord.\textsc{voc.sg} swine-slaying.\textsc{m.voc.sg} oaken.\textsc{m.acc.sg} branch.\textsc{acc.sg} put.\textsc{3sg.aor}\\
\trans `O lion-strangling, swine-slaying lord, you have placed an oaken bough upon me' (Callimachus, Epigram 36.1 (34.1 Wilamowitz))
\label{epi2}
\end{exe}

\begin{exe}
\ex τῆϲ Ἀγοράνακτοϲ {με} λέγε, ξένε, κωμικὸν ὄντωϲ ἀγκεῖϲθαι νίκηϲ μάρτυρα τοῦ Ῥοδίου Πάμφιλον\\
\gll tês Agoránaktos \emph{me} lége, xéne, kōmikòn óntōs ankeîsthai níkēs mártura toû Rhodíou Pámphilon\\
the.\textsc{f.gen.sg} Agoranax.\textsc{gen} me.\textsc{acc} say.\textsc{2sg.prs.imper} stranger.\textsc{voc.sg} funny.\textsc{n.acc.sg} truly lay-up.\textsc{prs.inf.pass} victory.\textsc{gen.sg} witness.\textsc{acc.sg} the.\textsc{m.gen.}sg of-Rhodes.\textsc{m.gen.sg} Pamphilus.\textsc{acc}\\
\trans `Tell me, O foreigner from Agoranax, whether it is truly funny for Pamphilus of Rhodes to be laid up as witness to the victory.' (Callimachus, Epigram 50.1 (49.1 Wilamowitz))
\label{epi3}
\end{exe}

\begin{exe}
\ex τῷ {με} Κανωπίτῃ Καλλίϲτιον εἴκοϲι μύξαιϲ πλούϲιον ἡ Κριτίου λύχνον ἔθηκε θεῷ\\
\gll tôi \emph{me} Kanôpítēi Kallístion eíkosi múxais ploúsion hē Kritíou lúkhnon éthēke theôi\\
therefore me.\textsc{acc} of-Canopus.\textsc{m.dat.sg} Callistion.\textsc{nom} twenty wick.\textsc{dat.pl} rich.\textsc{m.acc.sg} the.\textsc{f.nom.sg} Critias.\textsc{gen} lamp.\textsc{acc.sg} put.\textsc{3sg.aor} god.\textsc{dat.sg}\\
\trans `Therefore Callistion, the daughter of Critias, dedicated me, a costly lamp with twenty wicks, to the god of Canopus.' (Callimachus, Epigram 56.1 (55.1 Wilamowitz))
\label{epi4}
\end{exe}

\begin{exe}
\ex Θαλῆϲ {με} τῷ μεδεῦντι Νείλεω δήμου δίδωϲι, τοῦτο δὶϲ λαβὼν ἀριϲτεῖον\\
\gll Thalês \emph{me} tôi medeûnti Neíleō dḗmou dídōsi, toûto dìs labṑn aristeîon\\
Thales.\textsc{nom} me.\textsc{acc} the.\textsc{m.dat.sg} protect.\textsc{ptcp.prs.m.dat.sg} Neleus.\textsc{gen} people.\textsc{gen.sg} give.\textsc{3pl.prs} this.\textsc{n.acc.sg} twice take.\textsc{ptcp.aor.n.nom.sg} prize.\textsc{acc}\\
\trans `Thales is giving me to the guardian of the people of Neleus, having received this as a prize twice' (Diogenes Laërtius 1.1.29 (Fragment 95))
\label{epi5}
\end{exe}

\begin{exe}
\ex καί μ᾽ ἐπὶ Πατρόκλῳ θῆκεν πόδαϲ ὠκὺϲ Ἀχιλλεύϲ\\
\gll kaí \emph{m'} epì Patróklōi thêken pódas ōkùs Akhilleús\\
And me.\textsc{acc} on Patroclus.\textsc{dat} place.\textsc{3sg.aor} foot.\textsc{acc.pl} swift.\textsc{m.nom.sg} Achilles.\textsc{nom}\\
\trans `And swift Achilles placed (his) feet on Patroclus' (Athen. 6, 232 B = Palatine Anthology 6.49)
\label{epi6}
\end{exe}

\begin{exe}
\ex δέξαι μ᾽ Ἡράκλειϲ Ἀρχεϲτράτου ἱερὸν ὅπλον\\
\gll déxai \emph{m'} Hērákleis Arkhestrátou hieròn hóplon\\
accept.\textsc{2sg.aor.imper.mid} me.\textsc{acc} Hercules.\textsc{voc} Archestratus.\textsc{gen} holy.\textsc{n.acc.sg} weapon.\textsc{acc.sg}\\
\trans `Accept me, Hercules, the holy weapon of Archestratus' (Palatine Anthology 6.178.1)
\label{epi7}
\end{exe}

(\ref{epi8})--(\ref{epi10}) are deviations, but not significant ones. 

\begin{exe}
\ex Βιθυνὶϲ Κυθέρη {με} τεῆϲ ἀνεθήκατο, Κύπρι, μορφῆϲ εἴδωλον λύγδινον εὐξαμένη\\
\gll Bithunìs Kuthérē \emph{me} teês anethḗkato, Kúpri, morphês eídōlon lúgdinon euxaménē\\
Bithynian.\textsc{f.nom.sg} Cytherea.\textsc{nom} me.\textsc{acc} your.\textsc{f.gen.sg} dedicate.\textsc{3sg.aor.mid} Cypris.\textsc{voc} form.\textsc{gen.sg} image.\textsc{acc.sg} marble.\textsc{n.acc.sg} pray.\textsc{ptcp.aor.mid.f.nom.sg}\\
\trans `O Cypris, Bythinian Cytherea dedicated my marble image of your form with a prayer' (Palatine Anthology 6.209.1)
\label{epi8}
\end{exe}

\begin{exe}
\ex ϲμήνεοϲ ἔκ {με} ταμὼν γλυκερὸν θέροϲ ἀντὶ νομαίων γηραιὸϲ Κλείτων ϲπεῖϲε μελιϲϲοπόνοϲ\\
\gll smḗneos ék \emph{me} tamṑn glukeròn théros antì nomaíōn gēraiòs Kleítōn speîse melissopónos\\
hive.\textsc{gen.sg} out me.\textsc{acc} cut.\textsc{ptcp.aor.m.nom.sg} sweet.\textsc{n.acc.sg} harvest.\textsc{acc} against customary.\textsc{n.gen.pl} aged.\textsc{m.nom.sg} Cleiton.\textsc{nom} libate.\textsc{3sg.aor} bee-keeping.\textsc{m.nom.sg}\\
\trans `Aged Cleiton the beekeeper makes a libation of me, cutting a sweet harvest from the hive against custom' (Palatine Anthology 6.239.1)
\label{epi9}
\end{exe}

\begin{exe}
\ex χάλκεον ἀργυρέῳ {με} πανείκελον, Ἰνδικὸν ἔργον, ὄλπην ... πέμπεν γηθομένῃ ϲὺν φρενὶ Κριναγόρηϲ\\
\gll khálkeon arguréōi \emph{me} paneíkelon, Indikòn, érgon ólpēn pémpen gēthoménēi sùn phrenì Krinagórēs\\
brazen.\textsc{n.acc.sg} silver.\textsc{n.dat.sg} me.\textsc{acc} just-like.\textsc{n.acc.sg} Indian.\textsc{n.acc.sg} work.\textsc{acc.sg} flask.\textsc{acc} send.\textsc{3sg.imp} rejoice.\textsc{ptcp.prs.f.dat.sg} with midriff.\textsc{dat} Crinagoras.\textsc{nom}\\
\trans `With joyous heart Crinagoras sent me a flask of Indian work, of bronze but exactly like silver' (Palatine Anthology 6.261.1)
\label{epi10}
\end{exe}

On the other hand, for (\ref{epi11}) the version transmitted in the Palatine Anthology has been superseded by the original in stone that has come to light, \hyperlink{p351}{\emph{[p351]}} CIA 1.381 (=\citet{Kaibel1878} 578), which contains no \textit{m'}. This also reveals the \textit{m'} expanded by \citet[147]{Hecker1852} in (\ref{epi12}) to be superfluous.

\begin{exe}
\ex πρὶν μὲν Καλλιτέληϲ (μ᾽) ἱδρύϲατο\\
\gll prìn mèn Kallitélēs (\emph{m'}) hidrúsato\\
before then Kallitelis.\textsc{nom} (me.\textsc{acc}) place.\textsc{3sg.aor.mid}\\
\trans `Before Kalliteles placed (me)' (Palatine Anthology 6.138.1)
\label{epi11}
\end{exe}

\begin{exe}
\ex παιδὶ φιλοϲτεφάνῳ Σεμέλαϲ (μ᾽) ἀνέθηκε\\
\gll paîdì philostephánōi Semelas (\emph{m'}) anéthēke\\
child.\textsc{dat.sg} wreath-loving.\textsc{m.dat.sg} Semele.\textsc{gen} (me.\textsc{acc}) devote.\textsc{3sg.aor}\\
\trans `{[}Melanthus{]} devoted (me) to the wreath-loving child of Semele' (Palatine Anthology 6.140.1)
\label{epi12}
\end{exe}

Our survey of the examples with \textit{me} thus reveals that this element is placed in second position preferentially in poetic\is{poetry} compositions and almost exceptionlessly in \isi{prose}. If we divide up IGA 474 as \textit{égraph' emé} `wrote me', discount as uncertain Naukratis 1.303 and 1.307 in which only \textit{ME} or \textit{EME} is transmitted, and finally restore the sequence of words intended by the writer of the inscription in Naukratis 2.750, then only IGA 219 (=(\ref{failed}) above), which is not a verse\is{poetry} but an attempt at a verse,\is{poetry} and \citet[51]{Klein1887}, example (\ref{rulebreaker}) above, remain. The latter is therefore the only real exception, which strengthens our suspicion that an error has crept in here.

On the other hand, our rule receives further confirmation. First, from the fact that, in archaic inscriptions in which the monument or the person commemorated by the monument speaks, \textit{me} is in second position: (\ref{archaic1}) and (\ref{archaic2}).

\begin{exe}
\ex Κοϲμία ἠμί, ἆγε δέ {με} Κλιτομίαϲ\\
\gll Kosmía ēmí, âge dé \emph{me} Klitomías\\
Kosmia.\textsc{voc} say.\textsc{1sg.prs} bring.\textsc{2sg.prs.imper} but me.\textsc{acc} Klitomia.\textsc{gen}\\
\trans `And I say, ``Kosmia, bring me Klitomia''.' (IGA 473, Rhodes)
\label{archaic1}
\end{exe}

\begin{exe}
\ex ὃϲ δ᾽ ἄν {με} κλέψει\\
\gll hòs d' án \emph{me} klépsei\\
who.\textsc{m.nom.sg} then \textsc{irr} me.\textsc{acc} steal.\textsc{3sg.aor.sbjv}\\
\trans `who then might steal me' (IGA 524 (Cumae) = \citet{Kaibel1890} 865)
\label{archaic2}
\end{exe}

Secondly (to anticipate \hyperlink{latinpron}{a later section}) from the analogous \ili{Latin} inscriptions: \textit{Manios med fefaked} `Manios me.\textsc{acc} made', \textit{Duenos med feced} `Duenos me.\textsc{acc} made', and (\ref{cista}).

\begin{exe}
\ex
\gll Novios Plautios med Romai fecid\\
Novios Plautios me.\textsc{acc} Rome.\textsc{loc} made\\
\trans `Novios Plautios made me in Rome'
\label{cista}
\end{exe}

\label{eme}Particularly instructive, however, are the few inscriptions with \textit{emé} `me.\textsc{acc}'. In two cases, (\ref{eme1}) and (\ref{eme2}), this \textit{emé} is also in second position.

\begin{exe}
\ex Ἀπολλόδωροϲ {ἐμὲ} ἀνέθ{[}ηκε{]}\\
\gll Apollódōros \emph{emè} anéth{[}ēke{]}\\
Apollodorus.\textsc{nom} me.\textsc{acc} devote.\textsc{3sg.aor}\\
\trans `Apollodorus dedicated me' (IGA 20.8, Corinth)
\label{eme1}
\end{exe}

\begin{exe}
\ex Μεναΐδαϲ {ἐμ᾽} ἐποί(ϝ)ηϲε Χάροπ{[}ι{]}\\
\gll Menaḯdas \emph{em'} epoí(w)ēse Khárop{[}i{]}\\
Menaidas.\textsc{nom} me.\textsc{acc} create.\textsc{3sg.aor} Charopus.\textsc{dat}\\
\trans `Menaidas created me for Charopus' \citep[168]{Pottier1888}
\label{eme2}
\end{exe}

But in six cases \textit{emé} is in a different position: (\ref{eme3})--(\ref{eme5}), as well as \citet[82]{Klein1887} \textit{Ermogénēs epoíēsen \emph{éme}}, \citet[83]{Klein1887} \textit{Ermogénēs epoíēsen \emph{éne}} (read \textit{emé}), and \citet[85]{Klein1887} \textit{Sakōnídēs égrapsen \emph{éme}}.

\begin{exe}
\ex Ἐξεκίαϲ ἔγραψε κἀπόηϲε᾽ {ἐμέ}\\
\gll Exekías égrapse kapóēse \emph{emé}\\
Execius.\textsc{nom} write.\textsc{aor.3sg} and=create.\textsc{3sg.aor} me.\textsc{acc}\\
\trans `Execius wrote and created me' (\citealp[39]{Klein1887}; verse?)\is{poetry}
\label{eme3}
\end{exe}

\begin{exe}
\ex Ἐξεκίαϲ ἔγραψε κἀ(ι)ποίηϲ᾽ {ἐμέ}\\
\gll Exekías égrapse ka(i)poíēs' \emph{emé}\\
Execius.\textsc{nom} write.\textsc{aor.3sg} and=create.\textsc{3sg.aor} me.\textsc{acc}\\
\trans `Execius wrote and created me' (\citealp[40]{Klein1887}; verse?)\is{poetry}
\label{eme4}
\end{exe}

\begin{exe}
\ex Χαριταῖοϲ ἐποίηϲεν {ἔμ᾽} εὖ\\
\gll Kharitaîos epoíēsen \emph{ém'} eû\\
Charitaeus.\textsc{nom} create.\textsc{3sg.aor} me.\textsc{acc} well\\
\trans `Charitaeus created me well' \citep[51]{Klein1887}
\label{eme5}
\end{exe}

These instances show that the regular positioning of \textit{me} after the first word is not a coincidence, and that it is determined by its enclitic nature. (See also the \hyperlink{addenda}{Addenda}.)\label{forAddenda2}\il{Greek, Archaic|)}\il{Greek, Homeric|)}\is{inscriptions|)}


\section{The position of enclitic pronouns in later Greek}\label{enclitic-later}\il{Greek, Classical|(}

More important for this question (as indeed for any linguistic research that goes beyond etymological trivialities) are, of course, the more extensive texts of Ionic\il{Greek, Ionic} and \hyperlink{p352}{\emph{[p352]}} Attic\il{Greek, Attic} literature, especially Herodotus. He, however, followed the old rule with the other enclitic pronouns just as little as he did with \textit{min} and \textit{hoi}.

In the seventh book of Herodotus, \textit{spheōn} (\textsc{3pl.gen}) is found 13 times, including 6 in second position; \textit{sphi} (\textsc{3pl.dat}) 70 times, including 46 in second position; \textit{spheas} (\textsc{3pl.acc}) 32 times, including 20 in second position; \textit{sphea} (\textsc{3pl.acc}) once, not in second position. Overall, of 116 instances of \textit{sph}-forms, 72 follow the rule, i.e. roughly 62\%. Incomplete collections from the other books revealed a similar ratio.

As for second-person pronouns, in Herodotus VII we have \textit{seo} (\textsc{2sg.gen}) once, following the rule; \textit{toi} (\textsc{2sg.dat}, excluding the cases in which it is clearly a particle)\is{particles} 45 times, including 18--20 in second position; \textit{se} (\textsc{2sg.acc}) 16 times, including 10 in second position. As for first-person pronouns: \textit{meo} (\textsc{1sg.gen}) 3 times, of which one follows the rule; \textit{moi} (\textsc{1sg.dat}) 37 times, including 24 in second position, if (\ref{moi1})--(\ref{moi3}) can be included here; \textit{me} (\textsc{2sg.acc}) 6 times, including two instances following the rule. Thus, in the first and second person, we have 58 examples following the rule and 50 examples breaking it.

\begin{exe}
\ex ἔγνων δὲ ταῦτα μοι ποιητέα ἐόντα\\
\gll égnōn dè taûtá \emph{moi} poiētéa eónta\\
know.\textsc{1sg.aor} but this.\textsc{n.acc.pl} me.\textsc{dat} do.\textsc{gdv.n.acc.pl} be.\textsc{ptcp.prs.n.acc.pl}\\
\trans `And I knew that these things were necessary for me to do.' (Herodotus 7.15.2)
\label{moi1}
\end{exe}

\begin{exe}
\ex φέρε τοῦτό μοι ἀτρεκέωϲ εἰπέ\\
\gll phére toûtó \emph{moi} atrekéōs eipé\\
bear.\textsc{2sg.prs.imper} this.\textsc{n.acc.sg} me.\textsc{dat}
truly say.\textsc{2sg.aor.imper}\\
\trans `Come, tell me this truly.' (Herodotus 7.47.1)
\label{moi2}
\end{exe}

\begin{exe}
\ex ἄγε εἰπέ μοι\\
\gll áge eipé \emph{moi}\\
lead.\textsc{2sg.prs.imper} say.\textsc{2sg.prs.imper}
me.\textsc{dat}\\
\trans `Come, tell me.' (Herodotus 7.103.1)
\label{moi3}
\end{exe}

These statistics show very clearly that the old rule cannot be said to be uncontroversially operative in Herodotus, and that other positional rules have come into force. But they also show that despite, and alongside, these new rules the old rule still had strength enough to determine the position of the pronoun in more than half of the cases: admittedly this larger half includes those examples in which second position would also have been natural according to the newer rules.

Counts I have made in the works of the Attic\il{Greek, Attic} poets\is{poetry} demonstrate a further decline of the old rule. But unmistakable traces of this rule can still be found in particular set phrases and collocations in their work, as in Herodotus and the post-Homeric authors in general.

Every reader of the Attic\il{Greek, Attic} orators is struck by how often the imperative clause permitting the reading of a charter or the calling of witnesses begins with \textit{kaí moi}: it can safely be said that \hyperlink{p353}{\emph{[p353]}} any clause beginning with \textit{kaí} `and' and containing \textit{moi} (\textsc{1sg.dat}) will exceptionlessly have \textit{moi} immediately following \textit{kaí}. In what follows I arrange the examples following the chronology of poets\is{poetry} and the phrases following the dating of the earliest example.

\textit{kaí moi kálei} `and me.\textsc{dat} call' with a following object: (\ref{kaimoi1}), Andocides 1.28, 1.112, Lysias 13.79, 17.2, 17.3, 17.9, 19.59, 31.16, Isocrates 17.12, 17.16, 18.8, 18.54, Isaeus 6.37, 7.10, 8.42, 10.7, Demosthenes 29.12, 29.18, 41.6, 57.12, 57.38, 57.39, 57.46, (Demosthenes) 44.14, 44.44, 58.32, 58.33, 59.25, 59.28, 59.32, 59.34, 59.40, Aeschines 1.100. Or with a different position for the object (\ref{kaimoi2})--(\ref{kaimoi4}).

\begin{exe}
\ex καὶ μοι κάλει Διόγνητον\\
\gll kaí \emph{moi} kálei Diógnēton\\
and me.\textsc{dat} call.\textsc{2sg.prs.imper} Diognetus\\
\trans `And call Diognetus for me.' (Andocides 1.14)\\
\label{kaimoi1}
\end{exe}

\begin{exe}
\ex καί μοι μάρτυραϲ τούτων κάλει\\
\gll kaí \emph{moi} márturas toútōn kálei\\
and me.\textsc{dat} witness.\textsc{acc.pl} this.\textsc{n.gen.pl}
call.\textsc{2sg.prs.imper}\\
\trans `And call witnesses of these things for me.' (Antiphon 5.56)
\label{kaimoi2}
\end{exe}

\begin{exe}
\ex καί μοι ἁπάντων τούτων τοὺϲ μάρτυραϲ κάλει\\
\gll kaí \emph{moi} hapántōn toútōn toùs márturas kálei\\
and me.\textsc{dat} quite.all.\textsc{gen.pl} this.\textsc{n.gen.pl}
the.\textsc{m.acc.pl} witness.\textsc{acc.pl} call.\textsc{2g.prs.imper}\\
\trans `And call witnesses of all these things for me.' (Andocides 1.127)
\label{kaimoi3}
\end{exe}

\begin{exe}
\ex καί μοι τούτουϲ κάλει πρῶτον\\
\gll kaí \emph{moi} toútous kálei prôton\\
and me.\textsc{dat} this.\textsc{m.acc.pl} call.\textsc{2sg.prs.imper}
first\\
\trans `And call these people for me first.' (Isaeus 6.11)
\label{kaimoi4}
\end{exe}

\textit{kaí moi labè kaì anágnōthi} with a following object: (\ref{kaimoi5}) (also Andocides 1.15).

\begin{exe}
\ex καί μοι λαβὲ καὶ ἀνάγνωθι αὐτῶν τὰ ὀνόματα\\
\gll kaí \emph{moi} labè kaì anágnōthi autôn tà onómata\\
and me.\textsc{dat} take.\textsc{2sg.aor.imper} and
read.\textsc{2sg.aor.imper} them.\textsc{gen} the.\textsc{n.acc.pl} name.\textsc{acc.pl}\\
\trans `And take and read their names for me.' (Andocides 1.13)\\
\label{kaimoi5}
\end{exe}

\textit{kaí moi anágnōthi} `and me.\textsc{dat} read' with a following object: (\ref{kaimoi6}), Andocides 1.76, 1.82, 1.85, 1.86, 1.87, 1.96, Lysias 10.14, 10.15, 13.35, 13.50, 14.8, Isocrates 15.29, 17.52, Isaeus 5.2b, 5.4, 6.7, 6.8, (Demosthenes) 34.10, 34.11, 34.20, 34.39, 43.16, 46.26, 47.17, 47.20, 47.40, 47.44, 48.30, 59.52, Aeschines 3.24. Or with a different position for the object (\ref{kaimoi7})--(\ref{kaimoi9}). Without an object, (Demosthenes) 47.24.

\begin{exe}
\ex καί μοι ἀνάγνωθι αὐτῶν τὰ ὀνόματα\\
\gll kaí \emph{moi} anágnōthi autôn tà onómata\\
and me.\textsc{dat} read.\textsc{2sg.aor.imper} them.\textsc{gen}
the.\textsc{n.acc.pl} name.\textsc{acc.pl}\\
\trans `And read their names for me.' (Andocides 1.34)
\label{kaimoi6}
\end{exe}

\begin{exe}
\ex καί μοι τὰϲ μαρτυρίαϲ ἀνάγνωθι ταύταϲ\\
\gll kaí \emph{moi} tàs marturías anágnōthi taútas\\
and me.\textsc{dat} the.\textsc{f.acc.pl} testimony.\textsc{acc.pl}
read.\textsc{2sg.aor.imper} this.\textsc{f.acc.pl}\\
\trans `And read these depositions for me.' (Isaeus 2.16; cf. Isaeus 2.34 with synonymous \emph{tautasí} for \textit{taútas})
\label{kaimoi7}
\end{exe}

\begin{exe}
\ex καί μοι τούτων ἀνάγνωθι τὴν μαρτυρίαν\\
\gll kaí \emph{moi} toútōn anágnōthi tḕn marturían\\
and me.\textsc{dat} this.\textsc{n.gen.pl} read.\textsc{2sg.aor.imper}
the.\textsc{f.acc.sg} testimony.\textsc{acc.sg}\\
\trans `And read the deposition of these things for me.' ({[}Demosthenes{]} 50.42)
\label{kaimoi8}
\end{exe}

\begin{exe}
\ex καί μοι λαβὼν ἀνάγνωθι πρῶτον τὸν Σόλωνοϲ νόμον\\
\gll kaí \emph{moi} labṑn anágnōthi prôton tòn Sólōnos nómon\\
and me.\textsc{dat} take.\textsc{ptcp.aor.m.nom.sg}
read.\textsc{2sg.aor.imper} first the.\textsc{m.acc.sg} Solon.\textsc{gen} law.\textsc{acc.sg}\\
\trans `And having taken the law of Solon, read (it) first for me.' (Demosthenes 57.31)\\
\label{kaimoi9}
\end{exe}

\textit{kaí moi anábēte mártures} (or \textit{toútōn mártures}): (\ref{kaimoi10}), Lysias 1.42, 13.64, 16.14, 16.17, 32.37; contra Aeschinem Fragment 1 \citep[172.26]{BaiterSauppe1850} in Athen. 13.612 F, Isocrates 17.37, 17.41; \textit{kaí moi toútōn anábēte mártures} (\ref{kaimoi11}); \textit{kaí moi anábēte deûro} (\ref{kaimoi12}); \textit{kaí moi anábēthi} (\ref{kaimoi13}) and also Isocrates 17.32).

\begin{exe}
\ex καί μοι ἀνάβητε τούτων μάρτυρεϲ\\
\gll kaí \emph{moi} anábēte toútōn mártures\\
and me.\textsc{dat} ascend.\textsc{2pl.aor.imper}
this.\textsc{n.gen.pl} witness.\textsc{voc.pl}\\
\trans `And witnesses of these things, come forward for me.' (Lysias 1.29)\\\label{kaimoi10}
\end{exe}

\begin{exe}
\ex καί μοι τούτων ἀνάβητε μάρτυρεϲ\\
\gll kaí \emph{moi} toútōn anábēte mártures\\
and me.\textsc{dat} this.\textsc{n.gen.pl}
ascend.\textsc{2pl.aor.imper} witness.\textsc{voc.pl}\\
\trans `And witnesses of these things, come forward for me.' (Isocrates 17.14)
\label{kaimoi11}
\end{exe}

\begin{exe}
\ex καί μοι ἀνάβητε δεῦρο\\
\gll kaí \emph{moi} anábēte deûro\\
and me.\textsc{dat} ascend.\textsc{2pl.aor.imper} hither\\
\trans `And come up here for me.' (Lysias 20.29)
\label{kaimoi12}
\end{exe}

\begin{exe}
\ex καί μοι ἀνάβηθι\\
\gll kaí \emph{moi} anábēthi\\
and me.\textsc{dat} ascend.\textsc{2sg.aor.imper}\\
\trans `And come forward for me.' (Lysias 16.13)
\label{kaimoi13}
\end{exe}

\textit{kaí moi deûr' íte mártures}: (\ref{kaimoi14}).

\begin{exe}
\ex καί μοι δεῦρ᾽ ἴτε μάρτυρεϲ\\
\gll kaí \emph{moi} deûr' íte mártures\\
and me.\textsc{dat} hither go.\textsc{2sg.prs.imper}
witness.\textsc{voc.pl}\\
\trans `And come here for me, witnesses.' (Lysias 7 10)
\label{kaimoi14}
\end{exe}

\textit{kaì moi labè} `and me.\textsc{dat} take' with a following object: (\ref{kaimoi15}), Isocrates 18.19, 19.14, Isaeus 6.16, 6.48, 8.17, 12.11, Lycurgus 125, Demosthenes 18.222, 30.10, 30.32, 30.34, 31.4, 36.4, 41.24, 41.28, 55.14, 55.35, 57.19, 57.25, (Demosthenes) 34.7, 34.17, 44.14, 48.3, 58.51, 59.87, 59.104, Aeschines 2.65; \textit{kaí moi pálin labè} (\ref{kaimoi16}).

\begin{exe}
\ex καὶ μοι λαβὲ τὸν νόμον\\
\gll kaì \emph{moi} labè tòn nómon\\
and me.\textsc{dat} take.\textsc{2sg.aor.imper} the.\textsc{m.acc.sg} law.\textsc{acc.sg}\\
\trans `And take the law for me.' (Lysias 9.8)
\label{kaimoi15}
\end{exe}

\begin{exe}
\ex καί μοι πάλιν λαβὲ τὸν νόμον τοῦτον\\
\gll kaí \emph{moi} pálin labè tòn nómon toûton\\
and me.\textsc{dat} again take.\textsc{2sg.aor.imper}
the\textsc{.m.acc.sg} law\textsc{.acc.sg} this.\textsc{m.acc.sg}\\
\trans `And take this law again for me.' (Demosthenes 58.49)
\label{kaimoi16}
\end{exe}

\textit{kaí moi apókrinai}: (\ref{kaimoi17}).

\begin{exe}
\ex καί μοι ἀπόκριναι\\
\gll kaí \emph{moi} apókrinai\\
and me.\textsc{dat} answer.\textsc{2sg.aor.imper.mid}\\
\trans `And answer me.' (Lysias 13.32)
\label{kaimoi17}
\end{exe}

\hyperlink{p354}{\emph{[p354]}} \textit{kaí moi epílabe tò húdor}: (\ref{kaimoi18}), and Lysias 23.8, 23.11, 23.14, and 23.15.

\begin{exe}
\ex καί μοι ἐπίλαβε τὸ ὕδορ\\
\gll kaí \emph{moi} epílabe tò húdor\\
and me.\textsc{dat} hold.\textsc{2sg.aor.imper} the.\textsc{n.acc.sg} water.\textsc{acc.sg}\\
\trans `And stop the water for me.' (Lysias 23.4)\\
\label{kaimoi18}
\end{exe}

\textit{kaí moi anagígnōske} with a following object: (\ref{kaimoi19}) and {[}Demosthenes{]} 35.37.

\begin{exe}
\ex καί μοι ἀναγίγνωϲκε λαβὼν ταύτην τὴν μαρτυρίαν\\
\gll kaí \emph{moi} anagígnōske labṑn taútēn tḕn marturían\\
and me.\textsc{dat} read.\textsc{2sg.prs.imper}
take.\textsc{ptcp.aor.m.nom.sg} this.\textsc{f.acc.sg} the.\textsc{f.acc.sg} testimony.\textsc{acc.sg}\\
\trans `And having taken this testimony, read (it) for me.' (Demosthenes 27.8)
\label{kaimoi19}
\end{exe}

\textit{kaí moi lége} `and me.\textsc{dat} say' with a following object: (\ref{kaimoi20}), Demosthenes 19.154, 19.276, 18.53, 18.83, 18.105, 18.163, 18.218, 32.13, 37.17, 38.3, 38.14, (Demosthenes) 34.9, 56.38, Aeschines 2.91, 3.27, 3.32, 3.39.

\begin{exe}
\ex καί μοι λέγε τὸ ψήφιϲμα\\
\gll kaí \emph{moi} lége tò psḗphisma\\
and me.\textsc{dat} say.\textsc{2sg.prs.imper} the\textsc{.n.acc.sg} decree.\textsc{acc.sg}\\
\trans `And read the decree for me.' (Demosthenes 19.130)
\label{kaimoi20}
\end{exe}

\textit{kaí moi phére tò psḗphisma tò tóte genómenon}: (\ref{kaimoi21}).

\begin{exe}
\ex καί μοι φέρε τὸ ψήφισμα τὸ τότε γενόμενον\\
\gll kaí \emph{moi} phére tò psḗphisma tò tóte genómenon\\
and me.\textsc{dat} bear.\textsc{2sg.prs.imper} the.\textsc{n.acc.sg} decree.\textsc{acc.sg} the.\textsc{n.acc.sg} then become.\textsc{ptcp.aor.mid.n.acc.sg}\\
\trans `And bring me the decree made then.' (Demosthenes 18.179)
\label{kaimoi21}
\end{exe}

The only deviation is (\ref{kaimoi22}). Here, however, we have not just \textit{kaí} `and' but \textit{kaí ... dé} `and ... then', and before this \textit{dé} (and thus after \textit{kaí}) a strongly emphasized word was required, ruling out \textit{moi}.\footnote{\emph{Translator's note}: The Perseus edition lacks \textit{kaí}.}

\begin{exe}
\ex καί τελευταίαν δέ μοι λαβὲ τὴν αὐτοῦ Μιϲγόλα μαρτυρίαν\\
\gll kaí teleutaían dé \emph{moi} labè tḕn autoû Misgóla marturían\\
and final.\textsc{f.acc.sg} but me.\textsc{dat}
take.\textsc{2sg.aor.imper} the.\textsc{f.acc.sg} same.\textsc{m.gen.sg} Misgolas.\textsc{gen.sg} testimony.\textsc{acc.sg}\\
\trans `And finally take for me the affidavit of Misgolas himself.' (Aeschines 1.50)
\label{kaimoi22}
\end{exe}

Even outside this poetic\is{poetry} usage, though, \textit{kaí moi} clause-initially is particularly frequent (cf. Blass on Demosthenes 18.199).\footnote{\emph{Translator's note}: We have been unable to discover what work by Blass Wackernagel is referring to here. It may be his edition of Demosthenes \citep[306--307]{DindorfBlass1887}, but there is no explicit comment on this passage.} Here I give just a few examples, (\ref{kaimoi23})--(\ref{kaimoi58}); similar ones can be found in any text. 

\begin{exe}
\ex καί μ᾽ οὔτ᾽ ἰάμβων οὔτε τερπωλέων μέλει\\
\gll kaí \emph{m'} oút' iámbōn oúte terpōléōn mélei\\
and me.\textsc{dat} nor iamb.\textsc{gen.pl} nor
delight.\textsc{gen.pl} matter.\textsc{3sg.prs}\\
\trans `And neither poetry nor pleasures matter to me.' (Archilochus, Fragment 22)
\label{kaimoi23}
\end{exe}

\begin{exe}
\ex καί μοι ϲύμμαχοϲ γουνουμένῳ ἵλαοϲ γενεῦ\\
\gll kaí \emph{moi} súmmakhos gounouménōi hílaos geneû\\
and me.\textsc{dat} ally.\textsc{nom.sg}
beg.\textsc{ptcp.prs.pass.m.dat.sg} gracious.\textsc{m.nom.sg} become.\textsc{2sg.aor.imper.mid}\\
\trans `And be a gracious ally to me, I beg you.' (Archilochus, Fragment 108)
\label{kaimoi24}
\end{exe}

\begin{exe}
\ex καί μοι ...\\
\gll kaí \emph{moi}\\
and me.\textsc{dat}\\
\trans (Sappho 79, \citealp[58.25]{LobelPage1968})\footnote{\emph{Translator's note}: \citet{LobelPage1968} have \textit{toûto} before \textit{kaí \emph{moi}} here.}\\
\label{kaimoi25}
\end{exe}

\begin{exe}
\ex γιγνώϲκω, καί μοι φρενὸϲ ἔνδοθεν ἄλγεα κεῖται, πρεϲβυτάτην ἐϲορῶν γαῖαν Ἰαονίαϲ κλινομένην\\
\gll gignṓskō, kaí \emph{moi} phrenòs éndothen álgea keîtai, presbutátēn esorôn gaîan Iaonías klinoménēn\\
perceive.\textsc{1sg.prs} and me.\textsc{dat} midriff.\textsc{gen.sg} within pain.\textsc{nom.pl}
lie.\textsc{3sg.prs.pass} oldest.\textsc{f.acc.sg} behold.\textsc{ptcp.prs.m.nom.sg} land.\textsc{acc.sg} Ionia.\textsc{gen.sg} decline.\textsc{ptcp.prs.f.acc.sg}\\
\trans `I perceive Ionia's oldest land declining, and seeing (this) sorrows lie within my breast.' (Aristotle, \textit{Constitution of the Athenians} 5.2 \citep[14, line 3]{Kenyon1891})\\
\label{kaimoi26}
\end{exe}

\begin{exe}
\ex καί μοι τοῦτ᾽ ἀνιηρότατον\\
\gll kaí \emph{moi} toût' aniērótaton\\
and me.\textsc{dat} this.\textsc{n.nom.sg}
troublesome.\textsc{supl.n.nom.sg}\\
\trans `... and this (is) most troublesome to me.' (Theognis, \textit{Elegies} 257)
\label{kaimoi27}
\end{exe}

\begin{exe}
\ex καί μοι κραδίην ἐπάταξε μέλαιναν\\
\gll kaí \emph{moi} kradíēn epátaxe mélainan\\
and me.\textsc{dat} heart.\textsc{acc.sg} beat.\textsc{3sg.aor}
black.\textsc{f.acc.sg}\\
\trans `And it bruised my heart black and blue.' (Theognis, \textit{Elegies} 1198)
\label{kaimoi28}
\end{exe}

\begin{exe}
\ex καί μοι τὸν ἐμὸν πέμψατ᾽ άδελφόν\\
\gll kaí \emph{moi} tòn emòn pémpsat' ádelphón\\ 
and me.\textsc{dat} the.\textsc{m.acc.sg} my.\textsc{m.acc.sg}
send.\textsc{2pl.aor.imper} brother.\textsc{acc.sg}\\
\trans `... and send me my brother.' (Sophocles, \textit{Electra} 117)
\label{kaimoi29}
\end{exe}

\begin{exe}
\ex καί μοι τρίτον ῥίπτοντι Δωτιεὺϲ ἀνὴρ ἀγχοῦ προϲῆψεν Ἔλατοϲ ἐν διϲκήματι\\
\gll kaí \emph{moi} tríton rhíptonti Dōtieùs anḕr ankhoû prosêpsen Élatos en diskḗmati\\
and me.\textsc{dat} third throw.\textsc{ptcp.prs.m.dat.sg} Dotian.\textsc{nom.sg} man.\textsc{nom.sg} near fasten.\textsc{3sg.aor} Elatus.\textsc{nom} in quoit.\textsc{dat.sg}\\
\trans `... and thirdly, a Dotian man, Elatus, came near my throw at quoits.' (Sophocles, Fragment 380)
\label{kaimoi30}
\end{exe}

\begin{exe}
\ex καί μοι μέχρι Μακεδονίηϲ ἐλάϲαντι {[}...{]} οὐδεὶϲ ἠντιώθη\\
\gll kaí \emph{moi} mékhri Makedoníēs elásanti oudeìs ēntiṓthē\\
and me.\textsc{dat} until Macedonia.\textsc{gen} drive.\textsc{ptcp.aor.m.dat.sg} nobody.\textsc{m.nom.sg} oppose.\textsc{3sg.aor.pass}\\
\trans `And no one opposed me marching as far as Macedonia.' (Herodotus 7.9A.2)
\label{kaimoi31}
\end{exe}

\begin{exe}
\ex καί μοι τοῦτο τὸ ἔποϲ ἐχέτω ἐϲ πάντα λόγον\\
\gll kaí \emph{moi} toûto tò épos ekhétō es pánta lógon\\
and me.\textsc{dat} this.\textsc{n.nom.sg} the.\textsc{n.nom.sg}
word.\textsc{nom.sg} have.\textsc{3sg.prs.imper} into all.\textsc{m.acc.sg} account.\textsc{acc.sg}\\
\trans `And let this statement hold for my entire history.' (Herodotus 7.152.3)
\label{kaimoi32}
\end{exe}

\begin{exe}
\ex καί μοι τὸ μὲν ϲὸν ἐκποδὼν ἔϲτω λόγου\\
\gll kaí \emph{moi} tò mèn sòn ekpodṑn éstō lógou\\
and me.\textsc{dat} the.\textsc{n.acc.sg} then yours.\textsc{n.acc.sg}
away be.\textsc{3sg.prs.imper} account.\textsc{gen.sg}\\
\trans `And as for me, let your (fate) be unspoken.' (Euripides, \textit{Medea} 1222)
\label{kaimoi33}
\end{exe}

\begin{exe}
\ex καί μοι εὐεργεϲία ὀφείλεται\\
\gll kaí \emph{moi} euergesía opheíletai\\
and me.\textsc{dat} good.work.\textsc{nom.sg} owe.\textsc{3sg.prs.pass}\\
\trans `And a good turn is owed to me.' (Thucydides 1.137.4)
\label{kaimoi34}
\end{exe}

\begin{exe}
\ex καί μοι φράϲον\\
\gll kaí \emph{moi} phráson\\
and me.\textsc{dat} tell.\textsc{2sg.aor.imper}\\
\trans `And tell me ...' (Aristophanes, \textit{Frogs} 755)
\label{kaimoi35}
\end{exe}

\begin{exe}
\ex καί μοι δοκεῖ κατὰ ϲχολὴν παρὰ τἀνδρὸϲ ἐξελθεῖν μόνη\\
\gll kaí \emph{moi} dokeî katà skholḕn parà tandròs exeltheîn mónē\\
and me.\textsc{dat} seem.\textsc{3sg.prs} down ease.\textsc{acc.sg}
from the=man.\textsc{gen.sg} leave.\textsc{aor.inf} alone.\textsc{f.nom.sg}\\
\trans `She alone seems to me to have got away from her husband with ease.' (Aristophanes, \textit{Ecclesiazusae} 47)\footnote{\emph{Translator's note}: The Perseus edition has \textit{kaítoi} for \textit{kaí moi}.}
\label{kaimoi36}
\end{exe}

\begin{exe}
\ex καί μοι ταὐτὰ ταῦτα ἔδοξε\\
\gll kaí \emph{moi} tautà taûta édoxe\\
and me.\textsc{dat} the=same.\textsc{n.nom.pl} this.\textsc{n.nom.pl}
seem.\textsc{3sg.aor}\\
\trans `... and these same things seemed (true) to me.' (Plato, \textit{Apology} 21d)
\label{kaimoi37}
\end{exe}

\begin{exe}
\ex καί μοι ἀπόκριναι\\
\gll kaí \emph{moi} apókrinai\\
and me.\textsc{dat} answer.\textsc{2sg.aor.imper.mid}\\
\trans `And answer me.' (Plato, \textit{Apology} 25a = Plato, \textit{Gorgias} 462b)
\label{kaimoi38}
\end{exe}

\begin{exe}
\ex καί μοι μὴ ἄχθεϲθε λέγοντι τἀληθῆ\\
\gll kaí \emph{moi} mḕ ákhthesthe légonti talēthê\\
and me.\textsc{dat} not grieve.\textsc{2pl.prs.imper.pass}
say.\textsc{ptcp.prs.m.dat.sg} the=true.\textsc{n.acc.pl}\\
\trans `And do not be angry with me for speaking the truth.' (Plato, \textit{Apology} 31e)
\label{kaimoi39}
\end{exe}

\begin{exe}
\ex καί μοι δοκεῖ ... μῦθον ἂν ϲυνθεῖναι\\
\gll kaí \emph{moi} dokeî mûthon àn suntheînai\\
and me.\textsc{dat} seem.\textsc{3sg.prs} myth.\textsc{acc.sg} \textsc{irr}
assemble.\textsc{aor.inf}\\
\trans `And it seems to me that he (Æsop) would have made a fable.' (Plato, \textit{Phaedo} 60c)\\
\label{kaimoi40}
\end{exe}

\begin{exe}
\ex καί μοι δοκεῖ Κέβηϲ εἰϲ ϲὲ τείνειν τὸν λόγον\\
\gll kaí \emph{moi} dokeî Kébēs eis sè teínein tòn lógon\\
and me.\textsc{dat} seem.\textsc{3sg.prs} Cebes.\textsc{nom} into
you.\textsc{acc} spread.\textsc{prs.inf} the.\textsc{m.acc.sg}
account.\textsc{acc.sg}\\
\trans `And Cebes seems to me to be aiming his argument at you.' (Plato, \textit{Phaedo} 63a)\\
\label{kaimoi41}
\end{exe}

\begin{exe}
\ex καί μοι φράϲειν\\
\gll kaí \emph{moi} phrásein\\
and me.\textsc{dat} tell.\textsc{fut.inf}\\
`And to tell me ...' \trans (Plato, \textit{Phaedo} 97d)
\label{kaimoi42}
\end{exe}

\begin{exe}
\ex καί μοι ἔδοξεν ὁμοιότατον πεπονθέναι\\
\gll kaí \emph{moi} édoxen homoiótaton peponthénai\\
and me.\textsc{dat} seem.\textsc{3sg.aor} similar.\textsc{supl.n.nom.sg} suffer.\textsc{prf.inf}\\
\trans `And to me it seemed most similar to having it be ...' (Plato, \textit{Phaedo} 98c)
\label{kaimoi43}
\end{exe}

\largerpage
\begin{exe}
\ex καί μοι ὡμολόγει\\
\gll kaí \emph{moi} hōmológei\\
and me.\textsc{dat} agree.\textsc{3sg.imp}\\
\trans `... and he agreed with me ...' (Plato, \textit{Symposium} 173b)
\label{kaimoi44}
\end{exe}

\hyperlink{p355}{\emph{[p355]}}

\begin{exe}
\ex καί μοι ἔϲτω ἄρρητα τὰ εἰρημένα\\
\gll kaí \emph{moi} éstō árrhēta tà eirēména\\
and me.\textsc{dat} be.\textsc{3sg.prs.imper} unsaid.\textsc{n.nom.pl}
the.\textsc{n.nom.pl} say.\textsc{ptcp.prf.n.nom.pl}\\
\trans `And let the things said be unsaid for me.' (Plato, \textit{Symposium} 189b)
\label{kaimoi45}
\end{exe}

\begin{exe}
\ex καί μοι φαίνῃ ὀκνεῖν\\
\gll kaí \emph{moi} phaínēi okneîn\\
and me.\textsc{dat} appear.\textsc{2sg.prs.pass} hesitate.\textsc{prs.inf}\\
\trans `And you appear to me to be hesitating ...' (Plato, \textit{Symposium} 218c)
\label{kaimoi46}
\end{exe}

\begin{exe}
\ex καί μοι ἐπίδειξειν αὐτοῦ τούτου ποίηϲαι\\
\gll kaí \emph{moi} epídeixein autoû toútou poíēsai\\
and me.\textsc{dat} display.\textsc{fut.inf} same.\textsc{n.gen.sg}
this.\textsc{n.gen.sg} make.\textsc{2sg.aor.imper}\\
\trans `And make sure to display to me this very thing.' (Plato, \textit{Gorgias} 449c)\footnote{\emph{Translator's note}: The Perseus edition has a nominal form \textit{epídeixin} for \textit{epídeixein}.}
\label{kaimoi47}
\end{exe}

\begin{exe}
\ex καί μοι ἐϲτιν τῶν ἑτέρων παιδικῶν πολὺ ἧττον ἔμπληκτοϲ\\
\gll kaí \emph{moi} estin tôn hetérōn paidikôn polù hêtton émplēktos\\
and me.\textsc{dat} be.\textsc{3sg.prs} the.\textsc{gen.pl} other.\textsc{gen.pl} darling.\textsc{gen.pl} much less capricious.\textsc{f.nom.sg}\\
\trans `And (philosophy) is far less fickle to me than my other darlings.' (Plato, \textit{Gorgias} 482a)
\label{kaimoi48}
\end{exe}

\begin{exe}
\ex καί μοι δοκεῖ δουλοπρεπέϲ τι εἶναι\\
\gll kaí \emph{moi} dokeî douloprepés ti eînai\\
and me.\textsc{dat} seem.\textsc{3sg.prs} slavish.\textsc{n.acc.sg}
something.\textsc{n.acc.sg} be.\textsc{prs.inf}\\
\trans `And it seems to me to be somehow more suitable for a slave.' (Plato, \textit{Gorgias} 485b)
\label{kaimoi49}
\end{exe}

\begin{exe}
\ex καί μοι λέγε\\
\gll kaí \emph{moi} lége\\
and me.\textsc{dat} say.\textsc{2sg.prs.imper}\\
\trans `And tell me ...' (Plato, \textit{Gorgias} 492d = 494b)
\label{kaimoi50}
\end{exe}

\begin{exe}
\ex καί μοι ὥϲπερ παιδὶ χρῇ\\
\gll kaí \emph{moi} hṓsper paidì khrêi\\
and me.\textsc{dat} like child.\textsc{dat.sg} use.\textsc{2sg.prs.pass}\\
\trans `And you are treating me like a child.' (Plato, \textit{Gorgias} 499b)
\label{kaimoi51}
\end{exe}

\begin{exe}
\ex καί μοι πάνυ ϲφόδρα ἐνετέλλετο\\
\gll kaí \emph{moi} pánu sphódra enetélleto\\
and me.\textsc{dat} quite exceedingly enjoin.\textsc{3sg.imp.pass}\\
\trans `And he most particularly enjoined me ...' (Plato, \textit{Charmides} 157b)
\label{kaimoi52}
\end{exe}

\begin{exe}
\ex καί μοι δοκεῖ θεὸϲ μὲν ἁνὴρ οὐδαμῶϲ εἶναι\\
\gll kaí \emph{moi} dokeî theòs mèn hanḕr oudamôs eînai\\
and me.\textsc{dat} seem.\textsc{3sg.prs} god.\textsc{nom.sg} then the=man.\textsc{nom.sg} in.no.way be.\textsc{prs.inf}\\
\trans `And the man seems to me not to be a god at all.' (Plato, \textit{Sophist} 216b)
\label{kaimoi53}
\end{exe}

\begin{exe}
\ex καί μοι πειρῶ προϲέχων τὸν νοῦν εὖ μάλα ἀποκρίναϲθαι\\
\gll kaí \emph{moi} peirô prosékhōn tòn noûn eû mála apokrínasthai\\
and me.\textsc{dat} try.\textsc{2sg.prs.imper.pass}
direct.\textsc{ptcp.prs.m.nom.sg} the.\textsc{m.acc.sg} mind.\textsc{acc.sg} well very answer.\textsc{aor.inf.mid}\\
\trans `And, focusing your mind, try to answer me very well.' (Plato, \textit{Sophist} 233d; \textit{moi} is separated from its governing verb by \textit{peirô})
\label{kaimoi54}
\end{exe}

\begin{exe}
\ex καί μοι νῦν ἥ τε φωνὴ προϲφιλὴϲ ὑμῶν\\
\gll kaí \emph{moi} nûn hḗ te phōnḕ prosphilḕs humôn\\
and me.\textsc{dat} now the.\textsc{f.nom.sg} and
sound.\textsc{nom.sg} dear.\textsc{f.nom.sg} you.\textsc{gen.pl}\\
\trans `And your accent is now dear to me.' (Plato, \textit{Laws} 1.642c)
\label{kaimoi55}
\end{exe}

\begin{exe}
\ex καί μοι δοκεῖϲ ... προελέϲθαι\\
\gll kaí \emph{moi} dokeîs proelésthai\\
and me.\textsc{dat} seem.\textsc{2sg.prs} choose.\textsc{aor.inf.mid}\\
\trans `And you seem to me to have chosen ...' (Demosthenes 18 280)
\label{kaimoi56}
\end{exe}

\begin{exe}
\ex καί μοι λέγειν τοῦτ᾽ ἔϲτιν ἁρμοϲτόν, Σόλων\\
\gll kaí \emph{moi} légein toût' éstin harmostón, Sólōn\\
and me.\textsc{dat} say.\textsc{prs.inf} this.\textsc{n.acc.sg}
be.\textsc{3sg.prs} fit.\textsc{n.nom.sg} Solon.\textsc{voc}\\
\trans `And it is fitting to say this to me, Solon.' (Philemon, Fragment 4.4 \citep[479]{Kock1884})
\label{kaimoi57}
\end{exe}

\begin{exe}
\ex καί μοι τέκν᾽ ἐγένοντο δύ᾽ ἄρσενα\\
\gll kaí \emph{moi} tékn' egénonto dú' ársena\\
and me.\textsc{dat} child.\textsc{acc.pl} become.\textsc{3pl.aor.mid} two male.\textsc{n.acc.pl}\\
\trans `And two male children were born to me.' (Callimachus, \textit{Epigrams} 41.5; 40.5 in \citealp{Wilamowitz1882})
\label{kaimoi58}
\end{exe}

It is very rare for \textit{moi} not to be attached to a clause-initial \textit{kaí}: (\ref{kaimoi59}), (\ref{kaimoi60}), (\ref{kaimoi61}). (\textit{kaí moi} also in Euripides, \textit{Hippolytus} 377.1373.)

\begin{exe}
\ex καὶ πρέπειν μοι δοκεῖ\\
\gll kaì prépein \emph{moi} dokeî\\
and befit.\textsc{prs.inf} me.\textsc{dat} seem.\textsc{prs.inf}\\
\trans `And it seems suitable to me.' (Plato, \textit{Gorgias} 485c)
\label{kaimoi59}
\end{exe}

\begin{exe}
\ex καὶ οὐδέν μοι δεῖ ἄλληϲ βαϲάνου\\
\gll kaì oudén \emph{moi} deî állēs basánou\\
and nothing.\textsc{acc.sg} me.\textsc{dat} lack.\textsc{3sg.prs}
other.\textsc{f.gen.sg} touchstone.\textsc{gen.sg}\\
\trans `And I would have no need of another touchstone.' (Plato, \textit{Gorgias} 486d)\footnote{\emph{Translator's note}: The Perseus edition has \isi{accusative} \textit{me} for \textit{moi}.}
\label{kaimoi60}
\end{exe}

\begin{exe}
\ex καὶ ταῦτά μοι πάντα πεποίηται\\
\gll kaì taûtá \emph{moi} pánta pepoíētai\\
and this.\textsc{n.acc.pl} me\textsc{.dat} all.\textsc{n.acc.pl}
do.\textsc{3sg.prf.pass}\\
\trans `And on my part all these things have been done.' (Demosthenes 18.246)
\label{kaimoi61}
\end{exe}

As examples of so-called \isi{prodiorthosis} (Blass\ia{Blass, Friedrich} on Demosthenes 18.199), the following examples particularly belong together: (\ref{kaimoi62}) (cf. the example (\ref{kaimoi39}) discussed above), (\ref{kaimoi63}), (\ref{kaimoi64}), and (\ref{kaimoi65}).

\begin{exe}
\ex καί μοι, ὦ ἄνδρεϲ Ἀθηναῖοι, μὴ θορυβήϲητε\\
\gll kaí \emph{moi}, ô ándres Athēnaîoi, mḕ thorubḗsēte\\
and me.\textsc{dat} O man.\textsc{voc.pl} Athenian.\textsc{m.voc.pl}
not clamour.\textsc{2pl.aor.sbjv}\\
\trans `And do not interrupt me, men of Athens.' (Plato, \textit{Apology} 20e)
\label{kaimoi62}
\end{exe}

\begin{exe}
\ex καί μοι μηδὲν ἀχθεϲθῇϲ\\
\gll kaí \emph{moi} mēdèn akhthesthêis\\
and me.\textsc{dat} nothing.\textsc{acc.sg} grieve.\textsc{2sg.aor.sbjv.pass}\\
\trans `And do not be at all angry with me.' (Plato, \textit{Gorgias} 486a)
\label{kaimoi63}
\end{exe}

\begin{exe}
\ex καί μοι μὴ θορυβήϲῃ μηδείϲ\\
\gll kaí \emph{moi} mḕ thorubḗsēi mēdeís\\
and me.\textsc{dat} not clamour.\textsc{3sg.aor.sbjv} nobody.\textsc{m.nom.sg}\\
\trans `And let no one interrupt me.' (Demosthenes 5.15)
\label{kaimoi64}
\end{exe}

\begin{exe}
\ex καί μοι μηδὲν ὀργιϲθῇϲ\\
\gll kaí \emph{moi} mēdèn orgisthêis\\
and me.\textsc{dat} nothing.\textsc{acc.sg} anger.\textsc{2sg.aor.sbjv.pass}\\
\trans `And do not be at all angry.' (Demosthenes 20.102)
\label{kaimoi65}
\end{exe}

And the following examples are very similar, except with a \isi{genitive} pronoun: (\ref{kaimou1}) and (\ref{kaimou2}).

\begin{exe}
\ex καί μου πρὸϲ Διὸϲ καὶ θεῶν μηδὲ εἷϲ τὴν ὑπερβολὴν θαυμάϲῃ\\
\gll kaí \emph{mou} pròs Diòs kaì theôn mēdè heîs tḕn huperbolḕn thaumásēi\\
and me.\textsc{gen} to Zeus.\textsc{gen} and god.\textsc{m.gen.pl} nor
one\textsc{.m.nom.sg} the.\textsc{f.acc.sg} hyperbole.\textsc{acc.sg}
wonder.\textsc{3sg.aor.sbjv}\\
\trans `And before Zeus and the gods, let not one of you wonder at my exaggeration.' (Demosthenes 18.199)\footnote{\emph{Translator's note}: The Perseus edition has \isi{nominative} \textit{mēdeìs} for \textit{mēdè heîs}.}
\label{kaimou1}
\end{exe}

\begin{exe}
\ex καί μου πρὸϲ Διὸϲ μηδεμίαν ψυχρότητα καταγνῷ μηδείϲ\\
\gll kaí \emph{mou} pròs Diòs mēdemían psukhrótēta katagnôi mēdeís\\
and me.\textsc{gen} to Zeus.\textsc{gen} no.\textsc{f.acc.sg}
coldness.\textsc{acc.sg} condemn.\textsc{3sg.aor.subj} nobody.\textsc{m.nom.sg}\\
\trans `And before Zeus, let no one condemn me for any coldness.' (Demosthenes 18.256)
\label{kaimou2}
\end{exe}

The tendency to attach the pronoun to clause-initial \textit{kaí} is by no means restricted to \textit{moi}. \textit{kaí mou} can be found in (\ref{kaimou3})--(\ref{kaimou7}).

\begin{exe}
\ex καί μου παῦρ᾽ ἐπάκουϲον ἔπη\\
\gll kaí \emph{mou} paûr' epákouson épē\\
and me.\textsc{gen} few.\textsc{n.acc.pl} listen.\textsc{2sg.aor.imper} word.\textsc{acc.pl}\\
\trans `And listen to my few words.' (Theognis, \textit{Elegies} 1366)
\label{kaimou3}
\end{exe}

\begin{exe}
\ex καί μου τὰ ϲπλάγχν᾽ ἀγανακτεῖ\\
\gll kaí \emph{mou} tà splánkhn' aganakteî\\
and me.\textsc{gen} the.\textsc{n.acc.pl} innard.\textsc{acc.pl}
irritate.\textsc{3sg.prs}\\
\trans `And it gripes my guts.' (Aristophanes, \textit{Frogs} 1006)
\label{kaimou4}
\end{exe}

\begin{exe}
\ex καί μου ταύτῃ ϲοφώτεροι ἦϲαν\\
\gll kaí \emph{mou} taútēi sophṓteroi êsan\\
and me.\textsc{gen} thus wiser.\textsc{nom.pl} be.\textsc{3pl.imp}\\
\trans `And thus they were wiser than I.' (Plato, \textit{Apology} 22d)
\label{kaimou5}
\end{exe}

\begin{exe}
\ex καί μου ὄπιϲθεν ὁ παῖϲ λαβόμενοϲ τοῦ ἱματίου\\
\gll kaí \emph{mou} ópisthen ho paîs labómenos toû himatíou\\
and me.\textsc{gen} behind the.\textsc{m.nom.sg} child.\textsc{m.nom.sg} take.\textsc{ptcp.aor.mid.m.nom.sg} the.\textsc{n.gen.sg} garment.\textsc{gen.sg}\\
\trans `And the boy, taking hold of my garment from behind ...' (Plato, \textit{Republic} 1.327b)
\label{kaimou6}
\end{exe}

\begin{exe}
\ex καί μου λαβόμενοϲ τῆϲ χειρόϲ\\
\gll kaí \emph{mou} labómenos tês kheirós\\
and me.\textsc{gen} take.\textsc{ptcp.aor.mid.m.nom.sg} the.\textsc{f.gen.sg} hand.\textsc{gen.sg}\\
\trans `And, taking my hand ...' (Plato, \textit{Parmenides} 126a)
\label{kaimou7}
\end{exe}

For \textit{kaí me} I refer the reader to the previously-mentioned dedicatory and artists' inscriptions\is{inscriptions|(} which contain it: IGA 492, Cypriot\il{Greek, Cypriot} \citet{Deecke1884} 1.71, Pausanias 5.23.7 (=(\ref{paus4}) above), Palatine Anthology 6.49 (=(\ref{epi6}) above). Cf. (\ref{kaime1}) and the younger Cypriot\il{Greek, Cypriot} inscription (\ref{kaime2}). 

\begin{exe}
\ex καί μ᾽ ἔϲτεψε πατὴρ (ε)ἰϲαρίθμοιϲ ἔπεϲι\\
\gll kaí \emph{m'} éstepse patḕr (e)isaríthmois épesi\\
and me.\textsc{acc} crown.\textsc{3sg.aor} father.\textsc{nom.sg}
equivalent.\textsc{n.dat.pl} word.\textsc{dat.pl}\\
\trans `And (his) father garlanded me with an equal number of verses.' (\citealp{Kaibel1878}, 806)
\label{kaime1}
\end{exe}

\begin{exe}
\ex καί με χθὼν ἧδε καλύπτει\\
\gll kaí \emph{me} khthṑn hêde kalúptei\\
and me.\textsc{acc} earth.\textsc{nom} this.\textsc{f.nom.sg} hide.\textsc{3sg.prs}\\
\trans `And this earth hides me.' (\citet{Deecke1884}, no. 30)
\label{kaime2}
\end{exe}\is{inscriptions|)}

In addition, \hyperlink{p356}{\emph{[p356]}} we have (\ref{kaime3})--(\ref{kaime20}).

\begin{exe}
\ex κἀδόκουν ἕκαϲτοϲ αὐτῶν ὄλβον εὑρήϲειν πολὺν καί με κωτίλλοντα λείωϲ τραχὺν ἐκφανεῖν νόον\\
\gll kadókoun hékastos autôn ólbon heurḗsein polùn kaí \emph{me} kōtíllonta leíōs trakhùn ekphaneîn nóon\\
and=think.\textsc{3pl.imp} each.\textsc{m.nom.sg} them.\textsc{gen} wealth.\textsc{acc} find.\textsc{fut.inf} much.\textsc{m.acc.sg} and me.\textsc{acc} coax.\textsc{ptcp.prs.m.acc.sg} smoothly harsh.\textsc{m.acc.sg} reveal.\textsc{fut.inf} mind.\textsc{acc}\\
\trans `And they thought, each of them, that they would find great wealth and that I, while coaxing gently, would reveal a harsh mind.' (Solon in Aristotle, \textit{Constitution of the Athenians}; \citealp[30, line 1]{Kenyon1891})
\label{kaime3}
\end{exe}

\begin{exe}
\ex καί μ᾽ ἐπίβωτον κατὰ γείτοναϲ ποιήϲειϲ\\
\gll kaí \emph{m'} epíbōton katà geítonas poiḗseis\\
and me.\textsc{acc} notorious.\textsc{m.acc.sg} down neighbour.\textsc{acc.pl} make.\textsc{2sg.fut}\\
\trans `And you will make me notorious among the neighbours.' (Anacreon, Fragment 9)
\label{kaime4}
\end{exe}

\begin{exe}
\ex καί με δεϲπότεω βεβροῦ λαχόντα λίϲϲομαι ϲε μὴ ῥαπίζεϲθαι\\
\gll kaí \emph{me} despóteō bebroû lakhónta líssomai se mḕ rhapízesthai\\
and me.\textsc{acc} master.\textsc{gen.sg} foolish.\textsc{m.gen.sg}
obtain.\textsc{ptcp.aor.m.acc.sg} pray.\textsc{1sg.prs} you.\textsc{acc} not beat.\textsc{prs.inf}\\
\trans `And I pray you not to beat me for having found a foolish master.' (Hipponax, Fragment 64)
\label{kaime5}
\end{exe}

\begin{exe}
\ex καί με βιᾶται οἶνοϲ\\
\gll kaí \emph{me} biâtai oînos\\
and me.\textsc{acc} constrain.\textsc{3sg.prs.pass} wine.\textsc{nom.sg}\\
\trans `And wine has got the better of me.' (Theognis, \textit{Elegies} 503)
\label{kaime6}
\end{exe}

\begin{exe}
\ex καί μ᾽ ἐφίλευν προφρόνωϲ πάντεϲ ἐπερχόμενον\\
\gll kaí \emph{m'} ephíleun prophrónōs pántes eperkhómenon\\
and me.\textsc{acc} like.\textsc{3pl.imp} willingly all.\textsc{m.nom.pl} approach.\textsc{ptcp.prs.m.acc.sg}\\
\trans `And they all freely enjoyed my approaching.' (Theognis, \textit{Elegies} 785)
\label{kaime7}
\end{exe}

\begin{exe}
\ex καί μ' ἦμαρ ἤδη ξυμμετρούμενον χρόνῳ λυπεῖ τί πράϲϲει\\
\gll kaí \emph{m'} êmar ḗdē xummetroúmenon khrónōi lupeî tí prássei\\
and me.\textsc{acc} day.\textsc{acc.sg} already reckon.\textsc{ptcp.prs.n.acc.sg} time.\textsc{dat.sg} trouble.\textsc{3sg.prs} what.\textsc{acc.sg} do.\textsc{3sg.prs}\\
\trans `And what he is doing troubles me, with the days reckoned in time.' (Sophocles, \textit{Oedipus Rex} 73)
\label{kaime8}
\end{exe}

\begin{exe}
\ex φάναι Πέρϲαϲ τε λέγειν ἀληθέα καί με μὴ ϲωφρονέειν\\
\gll phánai Pérsas te légein alēthéa kaí \emph{me} mḕ sōphronéein\\
say.\textsc{prs.inf} Persian.\textsc{acc.pl} and speak.\textsc{prs.}inf true.\textsc{n.acc.pl} and me.\textsc{acc} not be.sane.\textsc{prs.inf}\\
\trans `Say that the Persians are telling the truth and that I am out of my mind.' (Herodotus 3.35.2)
\label{kaime9}
\end{exe}

\begin{exe}
\ex καί μ᾽ οὐ νομίζω παῖδα ϲὸν πεφυκέναι\\
\gll kaí \emph{m'} ou nomízō paîda sòn pephukénai\\
and me.\textsc{acc} not consider.\textsc{1sg.prs} child.\textsc{acc.sg} your.\textsc{m.acc.sg} beget.\textsc{prf.inf}\\
\trans `And I do not consider myself your begotten son.' (Euripides, \textit{Alcestis} 641)
\label{kaime10}
\end{exe}

\begin{exe}
\ex τέθνηκα τῇ ϲῇ θυγατρὶ καί μ᾽ ἀπώλεϲε\\
\gll téthnēka têi sêi thugatrì kaí \emph{m'} apṓlese\\
die.\textsc{1sg.prf} the.\textsc{f.dat.sg} your.\textsc{f.dat.sg}
daughter.\textsc{dat.sg} and me.\textsc{acc} destroy.\textsc{3sg.aor}\\
\trans `I have been killed by your daughter and she has destroyed me.' (Euripides, \textit{Andromache} 335)
\label{kaime11}
\end{exe}

\begin{exe}
\ex καί μ᾽ ἀπάλλαξον πόνων\\
\gll kaí \emph{m'} apállaxon pónōn\\
and me.\textsc{acc} deliver.\textsc{2sg.aor.imper} trouble.\textsc{gen.pl}\\
\trans `And free me from my troubles.' (Euripides, \textit{Medea} 333)
\label{kaime12}
\end{exe}

\begin{exe}
\ex πόϲιν ποθ᾽ ἥξειν καί μ᾽ ἀπαλλάξειν κακῶν\\
\gll pósin poth' hḗxein kaí \emph{m'} apalláxein kakôn\\
husband.\textsc{acc} sometime arrive.\textsc{fut.inf} and
me.\textsc{acc} deliver.\textsc{fut.inf} evil.\textsc{n.gen.pl}\\
\trans `... for my husband to come one day and free me from these evils.' (Euripides, \textit{Helen} 278)
\label{kaime13}
\end{exe}

\begin{exe}
\ex καί μ᾽ ἑλὼν θέλει δοῦναι τυράννοιϲ\\
\gll kaí \emph{m'} helṑn thélei doûnai turánnois\\
and me.\textsc{acc} take.\textsc{ptcp.aor.m.nom.sg} want.\textsc{3sg.prs} give.\textsc{aor.inf} king.\textsc{dat.pl}\\
\trans `And having taken me, he wants to give me to the royal house.' (Euripides, \textit{Helen} 551)
\label{kaime14}
\end{exe}

\begin{exe}
\ex καί με πρὸϲ τύμβον πόρευϲα πατρόϲ\\
\gll kaí \emph{me} pròs túmbon póreusa patrós\\
and me.\textsc{acc} to tomb.\textsc{acc} convey.\textsc{2sg.aor.imper}
father.\textsc{gen.sg}\\
\trans `And guide me to my father's tomb.' (Euripides, \textit{Orestes} 796)
\label{kaime15}
\end{exe}

\begin{exe}
\ex καί μ᾽ ἔφερβε ϲὸϲ δόμοϲ\\
\gll kaí \emph{m'} épherbe sòs dómos\\
and me.\textsc{acc} foster.\textsc{3sg.imp} your.\textsc{m.nom.sg}
house.\textsc{nom.sg}\\
\trans `And your house reared me.' (Euripides, \textit{Orestes} 866)
\label{kaime16}
\end{exe}

\begin{exe}
\ex καί μ᾽ ἀϲφαλῶϲ πανήμερον παῖϲαί τε καὶ χορεῦϲαι\\
\gll kaí \emph{m'} asphalôs panḗmeron paîsaí te kaì khoreûsai\\
and me.\textsc{acc} safely all.day sport.\textsc{aor.inf} and and dance.\textsc{aor.inf}\\
\trans `...and (allow) me to sport and dance safely all day.' (Aristophanes, \textit{Frogs} 338; cf. \textit{Knights} 862, and \textit{Frogs} 389 \textit{kaí ... me})
\label{kaime17}
\end{exe}

\begin{exe}
\ex καί με τοῦτ᾽ ἔτερπεν\\
\gll kaí \emph{me} toût' éterpen\\
and me.\textsc{acc} this.\textsc{n.nom.sg} delight.\textsc{3sg.imp}\\
\trans `And this delighted me.' (Aristophanes, \textit{Frogs} 916)
\label{kaime18}
\end{exe}

\begin{exe}
\ex καί μ᾽ οὐκ ἀρέϲκει\\
\gll kaí \emph{m'} ouk aréskei\\
and me.\textsc{acc} not please.\textsc{3sg.prs}\\
\trans `... and it does not please me.' (Aristophanes, \textit{Plutus} 353)
\label{kaime19}
\end{exe}

\begin{exe}
\ex καί με μηδεὶϲ ἀπαρτᾶν νομίϲῃ τὸν λόγον τῆϲ γραφῆϲ\\
\gll kaí \emph{me} mēdeìs apartân nomísēi tòn lógon tês graphês\\
and me.\textsc{acc} nobody.\textsc{m.nom.sg} detach.\textsc{prs.inf}
consider.\textsc{3sg.aor.sbjv} the.\textsc{m.acc.sg} account.\textsc{acc.sg} the.\textsc{f.gen.sg} writ.\textsc{gen.sg}\\
\trans `And let no one consider that I am changing the subject from the indictment.' (Demosthenes 18.59)\footnote{\emph{Translator's note}: The Perseus edition has \textit{hupolábēi}.}
\label{kaime20}
\end{exe}

Second person pronouns: (\ref{kaise1})--(\ref{kaisoi3}).

\begin{exe}
\ex καί ϲε {[}...{]} νέοι ἄνδρεϲ {[}...{]} ᾄϲονται\\
\gll kaí \emph{se} néoi ándres ā́isontai\\
and you.\textsc{acc} young.\textsc{m.nom.pl} men.\textsc{nom.pl} sing.\textsc{3pl.fut.mid}\\
\trans `And young men will sing of you.' (Theognis, \textit{Elegies} 241)
\label{kaise1}
\end{exe}

\begin{exe}
\ex καί ϲοι τὰ δίκαια φίλ᾽ ἔϲτω\\
\gll kaí \emph{soi} tà díkaia phíl' éstō\\
and you.\textsc{dat} the.\textsc{n.nom.pl} righteous.\textsc{n.nom.pl}
dear.\textsc{n.nom.pl} be.\textsc{3sg.prs.imper}\\
\trans `And let the righteous things be dear to you.' (Theognis, \textit{Elegies} 465)
\label{kaisoi1}
\end{exe}

\begin{exe}
\ex καί ϲε Ποϲειδάων χάρμα φίλοιϲ ἀνάγοι\\
\gll kaí \emph{se} Poseidáōn khárma phílois anágoi\\
and you.\textsc{acc} Poseidon.\textsc{nom} joy.\textsc{acc.sg}
friend.\textsc{dat.pl} lead.\textsc{3sg.prs.opt}\\
\trans `And may Poseidon bring you, a delight to your friends.' (Theognis, \textit{Elegies} 692)
\label{kaise2}
\end{exe}

\begin{exe}
\ex καί τοι ταύτην τὴν ἀτιμίην προϲτίθημι ἐόντι κακῷ καὶ ἀθύμῳ\\
\gll kaí \emph{toi} taútēn tḕn atimíēn prostíthēmi eónti kakôi kaì athúmōi\\
and you.\textsc{dat} this.\textsc{f.acc.sg} the.\textsc{f.acc.sg}
disgrace.\textsc{acc.sg} impose.\textsc{1sg.prs} be.\textsc{ptcp.prs.m.dat.sg} bad.\textsc{m.dat.sg} and spiritless.\textsc{m.dat.sg}\\
\trans `And on you, being base and spiritless, I lay this disgrace.' (Herodotus 7.11.1)
\label{kaitoi1}
\end{exe}

\begin{exe}
\ex καί ϲ᾽ ἐβουλόμην μένειν\\
\gll kaí \emph{s'} eboulómēn ménein\\
and you.\textsc{acc} wish.\textsc{1sg.imp.pass} remain.\textsc{prs.inf}\\
\trans `...and I wanted you to stay.' (Euripides, \textit{Medea} 456)
\label{kaise3}
\end{exe}

\begin{exe}
\ex καί ϲ᾽ οὐ κεναῖϲι χερϲὶ γῆϲ ἀποϲτελῶ\\
\gll kaí \emph{s'} ou kenaîsi khersì gês apostelô\\
and you.\textsc{acc} not empty.\textsc{f.dat.pl} hand.\textsc{dat.pl} land.\textsc{gen.sg} dispatch.\textsc{1sg.fut}\\
\trans `And I will not send you away from the land with empty hands.' (Euripides, \textit{Helen} 1280)
\label{kaise4}
\end{exe}

\begin{exe}
\ex καί ϲε προϲποιούμεθα\\
\gll kaí \emph{se} prospoioúmetha\\
and you.\textsc{acc} claim.\textsc{1pl.prs.pass}\\
\trans `...and we claim from you...' (Euripides, \textit{Helen} 1387)\footnote{\emph{Translator's note}: The Perseus edition has the tonic form `sè', which Wackernagel also cites as a variant reading.}
\label{kaise5}
\end{exe}

\begin{exe}
\ex καί ϲ᾽ ἀναγκαῖον θανεῖν\\
\gll kaí \emph{s'} anankaîon thaneîn\\
and you.\textsc{acc} necessary.\textsc{n.nom.sg} die.\textsc{aor.inf}\\
\trans `And (it is) necessary for you to die.' (Euripides, \textit{Orestes} 755)
\label{kaise6}
\end{exe}

\begin{exe}
\ex καί ϲ᾽ ἀμείψαϲθαι θέλω φιλότητι χειρῶν\\
\gll kaí \emph{s'} ameípsasthai thélō philótēti kheirôn\\
and you.\textsc{acc} repay.\textsc{aor.inf.mid} want.\textsc{1sg.prs} affection.\textsc{dat.sg} hand.\textsc{gen.pl}\\
\trans `And I want to give you back a fond embrace.' (Euripides, \textit{Orestes} 1047)
\label{kaise7}
\end{exe}

\begin{exe}
\ex ὁρῶ καί ϲε δέξομαι ϲύγκωμον\\
\gll horô kaí \emph{se} déxomai súnkōmon\\
see.\textsc{1sg.prs} and you.\textsc{acc} receive.\textsc{1sg.fut.mid}
fellow.reveller.\textsc{acc.sg}\\
\trans `I see and I will accept you as a fellow reveller.' (Euripides, \textit{Bacchae} 1172)
\label{kaise8}
\end{exe}

\largerpage[2]
\begin{exe}
\ex καί ϲε φαίνω τοῖϲ πρυτάνεϲιν\\
\gll kaí \emph{se} phaínō toîs prutánesin\\
and you.\textsc{acc} show.\textsc{1sg.prs} the.\textsc{m.dat.pl}
magistrate.\textsc{dat.pl}\\
\trans `...and I am exposing you to the magistrates.'\\
(Aristophanes, \textit{Knights} 300)\footnote{\emph{Translator's note}: The Perseus edition has \textit{phanô se}.}
\label{kaise9}
\end{exe}
\clearpage

\begin{exe}
\ex καί ϲε θυϲίαιϲιν ἱεραῖϲι {[}...{]} ἀγαλοῦμεν\\
\gll kaí \emph{se} thusíaisin hieraîsi agaloûmen\\
and you.\textsc{acc} sacrifice.\textsc{dat.pl} holy.\textsc{f.dat.pl} glorify.\textsc{1pl.fut}\\
\trans `And we will glorify you with holy sacrifices.' (Aristophanes, \textit{Peace} 396)
\label{kaise10}
\end{exe}

\begin{exe}
\ex καί ϲοι\footnote{\emph{Translator's note}: Wackernagel cites `kaì soì' as a variant
reading.} τὰ μεγάλ᾽ ἡμεῖϲ Παναθήναι᾽ ἄξομεν\\
\gll kaí \emph{soi} tà megál' hēmeîs Panathḗnai' áxomen\\
and you.\textsc{dat} the.\textsc{n.acc.pl} great.\textsc{n.acc.pl} we.\textsc{nom} Panathenaea.\textsc{acc} lead.\textsc{1pl.fut}\\
\trans `...and we will celebrate the great Panathenaea in your honour.' (Aristophanes, \textit{Peace} 418)
\label{kaisoi2}
\end{exe}

\begin{exe}
\ex καί ϲου κατεγέλα\\
\gll kaí \emph{sou} kategéla\\
and you.\textsc{gen} mock.\textsc{3sg.imp}\\
\trans `And he was mocking you.' (Plato, \textit{Gorgias} 482d)\footnote{\emph{Translator's note}: The Perseus edition has infinitive \textit{katagelân}.}
\label{kaisou1}
\end{exe}

\begin{exe}
\ex καί ϲε ἴϲωϲ τυπτήϲει τιϲ\\
\gll kaí \emph{se} ísōs tuptḗsei tis\\
and you.\textsc{acc} perhaps hit.\textsc{3sg.fut} someone.\textsc{m.nom.sg}\\
\trans `...and perhaps someone will hit you.' (Plato, \textit{Gorgias} 527a)
\label{kaise11}
\end{exe}

\begin{exe}
\ex καί ϲοι ἐπιρρέξει Γόργοϲ χιμάροιο νομαίηϲ αἷμα\\
\gll kaí \emph{soi} epirrhéxei Górgos khimároio nomaíēs haîma\\
and you.\textsc{dat} sacrifice.\textsc{3sg.fut} Gorgos.\textsc{nom}
goat.\textsc{gen.sg} pastoral.\textsc{f.gen.sg} blood.\textsc{acc}\\
\trans `And Gorgos will sacrifice the blood of a herdsman's goat to you.' (Anthologia Graeca 6.157.3).
\label{kaisoi3}
\end{exe}

Cf. also example (\ref{tu1}) cited above.

Third person pronouns: (\ref{kaispheas})--(\ref{kaispheas2}).

\begin{exe}
\ex καί ϲφεαϲ ὄλλυ᾽ ὥϲπερ ὀλλύειϲ\\
\gll kaí \emph{spheas} óllu' hṓsper ollúeis\\
and them.\textsc{acc} destroy.\textsc{2sg.prs.imper} like
destroy.\textsc{2sg.prs}\\
\trans `... and destroy them as you destroy.' (Archilochus, Fragment 27.2)
\label{kaispheas}
\end{exe}

\begin{exe}
\ex καί ϲφιν θαλάϲϲηϲ ἠχέεντα κύματα φίλτερ᾽ ἠπείρου γένηται\\
\gll kaí \emph{sphin} thalássēs ēkhéenta kúmata phílter' ēpeírou génētai\\
and them.\textsc{dat} sea.\textsc{gen.sg} roaring.\textsc{n.acc.pl} billow.\textsc{acc.pl} dearer.\textsc{n.acc.pl} land.\textsc{gen.sg} become.\textsc{3sg.aor.subj.mid}\\
\trans `... and the sea's roaring billows shall become dearer than land to them.' (Archilochus, Fragment 74.8)
\label{kaisphin1}
\end{exe}

\begin{exe}
\ex καί μιν ἐπ᾽ ἀνθρώπουϲ βάξιϲ ἔχει χαλεπή\\
\gll kaí \emph{min} ep' anthrṓpous báxis ékhei khalepḗ\\
and him.\textsc{acc} upon person.\textsc{acc.pl} rumour.\textsc{nom.sg}
have.\textsc{3sg.pr} harsh.\textsc{f.nom.sg}\\
\trans `And a harsh rumour keeps him against people.' (Mimnermus, Fragment 15)
\label{kaimin1}
\end{exe}

\begin{exe}
\ex καί οἱ ἔθηκε δοκεῖν\\
\gll kaí \emph{hoi} éthēke dokeîn\\
and him.\textsc{dat} put.\textsc{3sg.aor} seem.\textsc{prs.inf}\\
\trans `... and he made him think ...' (Theognis, \textit{Elegies} 405)
\label{kaihoi}
\end{exe}

\begin{exe}
\ex καί ϲφιν πολλ᾽ ἀμέλητα μέλει\\
\gll kaí \emph{sphin} poll' amélēta mélei\\
and them.\textsc{dat} many.\textsc{n.nom.pl} unimportant.\textsc{n.nom.pl} matter.\textsc{3sg.prs}\\
\trans `And many unimportant things occupy them.' (Theognis, \textit{Elegies} 422)
\label{kaisphin2}
\end{exe}

\begin{exe}
\ex καί σφιν τοῦτο γένοιτο φίλον\\
\gll kaí \emph{sphin} toûto génoito phílon\\
and them.\textsc{dat} this.\textsc{n.nom.sg} become.\textsc{3sg.aor.opt.mid} dear.\textsc{n.nom.sg}\\
\trans `... and this would become dear to them ...' (Theognis, \textit{Elegies} 732)
\label{kaisphin3}
\end{exe}

\begin{exe}
\ex καί μιν ἔθηκεν δαίμονα\\
\gll kaí \emph{min} éthēken daímona\\
and him.\textsc{acc} put.\textsc{3sg.aor} demon.\textsc{acc.sg}\\
\trans `... and he made him divine.' (Theognis, \textit{Elegies} 1348)
\label{kaimin2}
\end{exe}

\hyperlink{p357}{\emph{[p357]}}

\begin{exe}
\ex καί ϲφεων ἐϲχίϲθηϲαν αἱ γνῶμαι\\
\gll kaí \emph{spheōn} eskhísthēsan hai gnômai\\
and them.\textsc{gen} split.\textsc{3pl.aor.pass} the.\textsc{f.nom.pl} opinion.\textsc{nom.pl}\\
\trans `... and their opinions were divided.' (Herodotus 4.119.1)
\label{kaispheon}
\end{exe}

\begin{exe}
\ex καί νιν δοκῶ\\
\gll kaí \emph{nin} dokô\\
and him.\textsc{acc} think.\textsc{1sg.prs}\\
\trans `And I think that he ...' (Euripides, \textit{Orestes} 1200)
\label{kainin}
\end{exe}

\begin{exe}
\ex καί ϲφαϲ ϲιδηραῖϲ ἁρμόϲαϲ ἐν ἄρκυϲι παύϲω {[}...{]} τῆϲδε βακχείαϲ\\
\gll kaí \emph{sphas} sidēraîs harmósas en árkusi paúsō têsde bakkheías\\
and them.\textsc{acc} iron.\textsc{f.dat.pl} fit.\textsc{ptcp.aor.m.nom.sg} in net.\textsc{dat.pl} stop.\textsc{1sg.fut} this.\textsc{f.gen.sg} frenzy.\textsc{gen.sg}\\
\trans `And having put them in iron fetters, I will keep them from this frenzy.' (Euripides, \textit{Bacchae} 231)
\label{kaisphas}
\end{exe}

\begin{exe}
\ex καί ϲφιν ἀνιηρὸν μὲν ἐρεῖϲ ἔποϲ, ἔμπα δὲ λέξειϲ\\
\gll kaí \emph{sphin} aniēròn mèn ereîs épos, émpa dè léxeis\\
and them.\textsc{dat} troublesome.\textsc{n.acc} then say.\textsc{2sg.fut} word.\textsc{acc.sg} all but say.\textsc{2sg.fut}\\
\trans `And you will say a troublesome thing to them, and still you will say ...' (Callimachus, Epigram 14.3; 12.3 in \citealp{Wilamowitz1882})\footnote{\emph{Translator's note}: Both Perseus editions have \textit{léxai} for \textit{léxeis}.}
\label{kaisphin4}
\end{exe}

One example of \textit{kaí me} and one of \textit{kaí spheas} are particularly noteworthy: (\ref{kaime}) and (\ref{kaispheas2}). In both examples the pronoun is extracted from the subordinate\is{subordination} clause in which it belongs and attached to \textit{kaí}. Moreover, \textit{kaí} with a following enclitic pronoun is also found very often in Homer.\il{Greek, Homeric}

\largerpage[2]
\begin{exe}
\ex καί με ἐὰν ἐξελέγχῃϲ, οὐκ ἀπεχθήϲομαί ϲοι\\
\gll kaí \emph{me} eàn exelénkhēis, ouk apekhthḗsomaí soi\\
and me.\textsc{acc} if refute.\textsc{2sj.prs.sbjv} not hate.\textsc{1sg.fut.mid} you.\textsc{dat}\\
\trans `And if you refute me, I will not be angry with you.' (Plato, \textit{Gorgias} 506c)\footnote{\emph{Translator's note}: The Perseus edition has \textit{akhthesthḗsomaí} for \textit{apekhthḗsomaí}.}
\label{kaime}
\end{exe}

\begin{exe}
\ex καί ϲφεαϲ ὡϲ οὐδεὶϲ ἐκάλεε, ἐκτράπονται ἐπ᾽ Ἀθηνέων\\
\gll kaí \emph{spheas} hōs oudeìs ekálee, ektrápontai ep' Athēnéōn\\
and them.\textsc{acc} as nobody.\textsc{m.nom.sg} call.\textsc{3sg.imp} turn.\textsc{3pl.prs.pass} upon Athens.\textsc{gen}\\
\trans `And as no one invited them, they turned toward Athens.' (Herodotus 6.34.2)\footnote{\emph{Translator's note}: The Perseus edition has \textit{ektrépontai} for \textit{ektrápontai}.}
\label{kaispheas2}
\end{exe}
\clearpage

This attracting force also inheres in other \isi{particles} that regularly or often occur clause-initially, e.g. \textit{ou}, \textit{mḗ} (\textsc{neg}), \textit{gár} `since', \textit{ei}, \textit{eán} `if'. \textit{allá} `but' also belongs to this group, as in examples (\ref{alla1})--(\ref{alla7}) (the latter is Euripidizing).

\begin{exe}
\ex ἀλλά μοι ϲμικρόϲ τιϲ εἴη\\
\gll allá \emph{moi} smikrós tis eíē\\
but me.\textsc{dat} small.\textsc{m.nom.sg} someone.\textsc{m.nom.sg} be.\textsc{3sg.prs.opt}\\
\trans `...but someone would be unimportant to me...' (Archilochus, Fragment 58.3)
\label{alla1}
\end{exe}

\begin{exe}
\ex ἀλλά μ᾽ ὁ λυϲιμελήϲ, ὦταῖρε δάμναται πόθοϲ\\
\gll allá \emph{m'} ho lusimelḗs, ôtaîre dámnatai póthos\\
but me.\textsc{acc} the.\textsc{m.nom.sg} limb-relaxing.\textsc{m.nom.sg} O=companion.\textsc{voc.sg} overpower.\textsc{3sg.prs.pass} longing.\textsc{nom.sg}\\
\trans `But the limb-relaxing longing overpowers me, my friend.' (Archilochus, Fragment 85)
\label{alla2}
\end{exe}

\begin{exe}
\ex θέλω τι ϝείπην, ἀλλά με κωλύει αἴδωϲ\\
\gll thélō ti weípēn, allá \emph{me} kōlúei aídōs\\
want.\textsc{1sg.prs} something.\textsc{acc} say.\textsc{aor.inf} but me.\textsc{acc} prevent.\textsc{3sg.prs} shame.\textsc{nom}\\
\trans `I want to say something, but shame prevents me.' (Alcaeus, Fragment 55.2)
\label{alla3}
\end{exe}

\begin{exe}
\ex ἀλλά μ᾽ ἑταῖροϲ ἐκλείπει\\
\gll allá \emph{m'} hetaîros ekleípei\\
but me.\textsc{acc} companion fail.\textsc{3sg.prs}\\
\trans `But my companion fails me.' (Theognis, \textit{Elegies} 941)
\label{alla4}
\end{exe}

\begin{exe}
\ex ἀλλά μοι εἴη ζῆν ἀπὸ τῶν ὀλίγων\\
\gll allá \emph{moi} eíē zên apò tôn olígōn\\
but me.\textsc{dat} be.\textsc{3sg.prs.opt} live.\textsc{prs.inf} of
the.\textsc{gen.pl} little.\textsc{gen.pl}\\
\trans `...but for me (what I ask) would be to live on little.' (Theognis, \textit{Elegies} 1155)
\label{alla5}
\end{exe}

\begin{exe}
\ex ἀλλά μοι φόβοϲ τιϲ εἰϲελήλυθ(ε)\\
\gll allá \emph{moi} phóbos tis eiselḗluth(e)\\
but me.\textsc{dat} fear.\textsc{nom.sg} some.\textsc{m.nom.sg} enter.\textsc{3sg.prf}\\
\trans `But some fear has entered me.' (Euripides, \textit{Orestes} 1323)
\label{alla6}
\end{exe}

\begin{exe}
\ex ἀλλά μοι ἀμφίπολοι λύχνον ἅψατε\\
\gll allá \emph{moi} amphípoloi lúkhnon hápsate\\
but me.\textsc{dat} attendant.\textsc{m.voc.pl} lamp.\textsc{acc.sg}
touch.\textsc{2pl.aor.imper}\\
\trans `But, servants, light the lamp for me.' (Aristophanes, \textit{Frogs} 1338)\\
\label{alla7}
\end{exe}

\emph{allá moi} `but me.\textsc{dat}' is common in Plato (\textit{Apology} 39E, 41D, \textit{Phaedo} 63E, 72D, \textit{Symposium} 207C, 213A, \textit{Gorgias} 453A, 476B, 517B etc.), and \emph{allá se} `but you.\textsc{acc}' is found in Theognis 1287, 1333, Euripides, \textit{Medea} 759, 1389, etc.

Furthermore, as with Homer and Sappho, we even find enclitic pronouns attached to a \isi{vocative} when it is the first word of a clause or follows the first word of a clause: (\ref{voc1})--(\ref{voc9}).

\begin{exe}
\ex Μοῦϲά μοι Εὐρυμεδοντιάδεα {[}...{]} ἐννεφ᾽ {[}...{]}\\
\gll Moûsá \emph{moi} Eurumedontiádea enneph'\\
muse.\textsc{voc.sg} me.\textsc{dat} wide.ruling.\textsc{f.voc.sg}
tell.\textsc{2sg.prs.imper}\\
\trans `Wide-ruling Muse, tell me...' (Hipponax, Fragment 85.1)
\label{voc1}
\end{exe}

\begin{exe}
\ex Μοῖϲά μοι ἀμφὶ Σκάμανδρον ἐύρροον ἄρχομ᾽ ἀείδεν\\
\gll Moîsá \emph{moi} amphì Skámandron eúrrhoon árkhom' aeíden\\
muse.\textsc{voc.sg} me.\textsc{dat} about Scamander.\textsc{acc} well-flowing.\textsc{m.acc.sg} begin.\textsc{1sg.prs.pass} sing.\textsc{prs.inf}\\
\trans `Muse, I begin to sing for myself about the well-flowing Scamander.' (Fragmenta Lyrica Adespota 30A; \citealp[696]{Bergk1882})
\label{voc2}
\end{exe}

\begin{exe}
\ex μήτοι καϲιγνήτη μ᾽ ἀτιμαϲῃϲ\\
\gll mḗtoi kasignḗtē \emph{m'} atimasēis\\
not sister.\textsc{voc} me.\textsc{acc} dishonour.\textsc{2sg.aor.sbjv}\\
\trans `No, sister, do not deem me unworthy.' (Sophocles, \textit{Antigone} 544)
\label{voc3}
\end{exe}

\begin{exe}
\ex ὁδ᾽ ὦ ξένοι με, ϲοὺϲ ἀτιμάζων θεούϲ, ἕλκει\\
\gll hod' ô xénoi \emph{me}, soùs atimázōn theoús, hélkei\\
this.\textsc{m.nom.sg} O stranger.\textsc{voc.pl} me.\textsc{acc} your.\textsc{m.acc.pl} dishonour.\textsc{ptcp.m.nom.sg} god.\textsc{acc.pl} drag.\textsc{3sg.prs}\\
\trans `Dishonouring your gods, strangers, this man drags me...' (Euripides, \textit{Heracleidae} 78)
\label{voc4}
\end{exe}

\begin{exe}
\ex ὁ Διόϲ, ὁ Διόϲ, ὦ πόϲι με παῖϲ Ἑρμᾶϲ ἐπέλαϲεν Νείλῳ\\
\gll ho Diós, ho Diós, ô pósi \emph{me} paîs Hermâs epélasen Neílōi\\
the.\textsc{m.nom.sg} Zeus.\textsc{gen} the.\textsc{m.nom.sg} Zeus.\textsc{gen} O husband.\textsc{voc.sg} me.\textsc{acc} child.\textsc{nom.sg} Hermes.\textsc{nom} bring.\textsc{3sg.aor} Nile.\textsc{dat}\\
\trans `Zeus's, Zeus's son Hermes, brought me to the Nile, husband.' (Euripides, \textit{Helen} 670)\footnote{\emph{Translator's note}: For \textit{me paîs Hermâs} the Perseus edition has \textit{paîs m'} followed by a lacuna.}
\label{voc5}
\end{exe}

\begin{exe}
\ex οἴκτιρε δ᾽ ὦ μῆτέρ με\\
\gll oíktire d' ô mêtér \emph{me}\\
pity.\textsc{2sg.prs.imper} then O mother.\textsc{voc.sg} me.\textsc{acc}\\
\trans `So pity me, Mother.' (Euripides, \textit{Bacchae} 1120)
\label{voc6}
\end{exe}

\begin{exe}
\ex ἔαϲον Ἀχοῖ με ϲὺν φίλαιϲιν γόου κόρον λαβεῖν\\
\gll éason Akhoî \emph{me} sùn phílaisin góou kóron labeîn\\
let.\textsc{2sg.aor.imper} echo.\textsc{voc.pl} me.\textsc{acc} with
friend.\textsc{f.dat.pl} wailing.\textsc{gen.sg} surfeit.\textsc{acc.sg} take.\textsc{aor.inf}\\
\trans `Echoes, let me have my fill of wailing with my friends.' (Euripides, \textit{Andromeda} Fragment 118)
\label{voc7}
\end{exe}

\begin{exe}
\ex μέμνηϲο Περϲεῦ μ᾽ ὡϲ καταλείπειϲ\\
\gll mémnēso Perseû \emph{m'} hōs kataleípeis\\
remember.\textsc{2sg.prf.imper.pass} Perseus\textsc{.voc} me.\textsc{acc} how leave.\textsc{2sg.prs}\\
\trans `Remember, Perseus, how you are leaving me behind.' (Aristophanes, \textit{Thesmophoriazusae} 1134)
\label{voc8}
\end{exe}

\begin{exe}
\ex εἶ᾽ ἄγε Θεϲτυλί μοι χαλεπᾶϲ νόϲω εὑρέ τι μᾶχοϲ\\
\gll eî' áge Thestulí \emph{moi} khalepâs nósō heuré ti mâkhos\\
on lead.\textsc{2sg.prs.imper} Thestylis.\textsc{voc} me.\textsc{dat} harsh.\textsc{f.gen.sg} illness.\textsc{gen.sg} find.\textsc{2sg.aor.imper} some.\textsc{n.acc.sg} remedy.\textsc{acc.sg}\\
\trans `Come now, Thestylis, find me some remedy for a harsh illness.' (Theocritus 2.95)\footnote{\emph{Translator's note}: The Perseus edition has \textit{ei d' áge ... mêkhos}.}
\label{voc9}
\end{exe}

Related to this is the attachment of the enclitic to a preceding \hyperlink{p358}{\emph{[p358]}} imperative element, as in Homeric\il{Greek, Homeric} \textit{all' áge moi}: (\ref{imper1})--(\ref{imper5}).

\begin{exe}
\ex δεῦρό ϲου ϲτέψω κάρα\\
\gll deûró \emph{sou} stépsō kára\\
hither you.\textsc{gen} crown.\textsc{1sg.fut} head.\textsc{acc.sg}\\
\trans `Come here; I will crown your head.' (Euripides, \textit{Bacchae} 341)
\label{imper1}
\end{exe}

\begin{exe}
\ex παῦϲαί με μὴ κάκιζε\\
\gll paûsaí \emph{me} mḕ kákize\\
stop.\textsc{2sg.aor.imper.mid} me.\textsc{acc} not abuse.\textsc{2sg.prs.imper}\\
\trans `Stop; do not make me a coward.' (Euripides, \textit{Iphigenia in Aulis} 1435)
\label{imper2}
\end{exe}

\begin{exe}
\ex φέρε δέ ϲοι, ἐὰν δύνωμαι, ϲαφέϲτερον ἀποδείξω\\
\gll phére dé \emph{soi}, eàn dúnōmai, saphésteron apodeíxō\\
bear.\textsc{2sg.prs.imper} but you.\textsc{dat} if can.\textsc{1sg.prs.sbjv} clearly.\textsc{comp} show.\textsc{1sg.aor.sbjv}\\
\trans `But come, let me show you more clearly, if I can ...' (Plato, \textit{Gorgias} 464b)\footnote{\emph{Translator's note}: The Perseus edition has \textit{epideíxō}.}
\label{imper3}
\end{exe}

\begin{exe}
\ex ἴθι δή μοι, ἐπειδὴ {[}...{]}, διελοῦ τάδε\\
\gll íthi dḗ \emph{moi}, epeidḕ dieloû táde\\
go.\textsc{2sg.prs.imper} exactly me.\textsc{dat} since decide.\textsc{2sg.aor.imper.mid} this.\textsc{n.acc.pl}\\
\trans `Go on, decide these things for me, since ...' (Plato,
\textit{Gorgias} 495c)
\label{imper4}
\end{exe}

\begin{exe}
\ex ἔχε δή μοι τόδε εἰπέ\\
\gll ékhe dḗ \emph{moi} tóde eipé\\
have.\textsc{2sg.prs.imper} exactly me.\textsc{dat} this.\textsc{n.acc.sg} say.\textsc{2sg.aor.imper}\\
\trans `Stop now and tell me this...' (Plato, \textit{Ion} 535b)
\label{imper5}
\end{exe}

Also attachment to \textit{boúlei} `wish.\textsc{2sg.prs}' when a first person singular \isi{subjunctive} follows: (\ref{boulei1})--(\ref{boulei4}). Broadly similar are (\ref{moidokein}) and (\ref{moiapokr}).

\begin{exe}
\ex βούλει ϲε γεύϲω\\
\gll boúlei \emph{se} geúsō\\
wish.\textsc{2sg.prs} you.\textsc{acc} taste.\textsc{1sg.aor.sbjv}\\
\trans `Do you want me to give you a taste?' (Euripides, \textit{Cyclops} 149)\footnote{\emph{Translator's note}: The Perseus edition has \isi{subjunctive} \textit{boulēi}.}
\label{boulei1}
\end{exe}

\begin{exe}
\ex βούλει ϲοι ὁμολογήϲω\\
\gll boúlei \emph{soi} homologḗsō\\
wish.\textsc{2sg.prs} you.\textsc{dat} agree.\textsc{1sg.aor.sbjv}\\
\trans `Do you want me to agree with you?' (Plato, \textit{Gorgias} 516c)
\label{boulei2}
\end{exe}

\begin{exe}
\ex βούλει ϲοι εἴπω\\
\gll boúlei \emph{soi} eípō\\
wish.\textsc{2sg.prs} you.\textsc{dat} say.\textsc{1sg.aor.sbjv}\\
\trans `Do you want me to tell you...' (Plato, \textit{Gorgias} 521d)
\label{boulei3}
\end{exe}

\begin{exe}
\ex βούλει ϲε θῶ φοβηθῆναι\\
\gll boúlei \emph{se} thô phobēthênai\\
wish.\textsc{2sg.prs} you.\textsc{acc} put.\textsc{1sg.aor.sbjv}
frighten.\textsc{aor.inf.pass}\\
\trans `Do you want me to assume that you were frightened?'\\
(Aeschines 3.163)
\label{boulei4}
\end{exe}

\begin{exe}
\ex νεωϲτί, μοι δοκεῖν, καταπεπλευκότι\\
\gll neōstí, \emph{moi} dokeîn, katapepleukóti\\
newly me.\textsc{dat} seem.\textsc{prs.inf} land.\textsc{ptcp.prf.m.dat.sg}\\
\trans `... freshly, I fancy, arrived on shore ...' (Plato, \textit{Euthydemus} 297c)
\label{moidokein}
\end{exe}

\begin{exe}
\ex τί οὖν, εἰπεῖν, μοι ἀποκρινεῖται\\
\gll tí oûn, eipeîn, \emph{moi} apokrineîtai\\
what.\textsc{acc.sg} so say.\textsc{aor.inf} me.\textsc{dat} answer.\textsc{3sg.fut.mid}\\
\trans `{``}Why, then,'' he said, ``shall I be answered?''' (Plato, \textit{Parmenides} 137b)\footnote{\emph{Translator's note}: The Perseus edition has \textit{tís}.}
\label{moiapokr}
\end{exe}

Often, however, we find such a pronoun that has been separated from the words to which it syntactically belongs in order to be placed in clausal second position, e.g. (\ref{pro2nd1}). Differently again (\ref{pro2nd2})--(\ref{pro2nd4}). See above p\pageref{kaime} on \textit{kaí me} and \textit{kaí spheas}. With \isi{participles}: (\ref{ptcp1})--(\ref{ptcp6}).

\begin{exe}
\ex λῷϲτά ϲε μήτε λίην ἀφνεὸν κτεάτεϲϲι γενέϲθαι μήτε ϲέ γ᾽ἐϲ πολλὴν χρημοϲύνην ἐλάϲαι\\
\gll lôistá \emph{se} mḗte líēn aphneòn kteátessi genésthai mḗte sé g'es pollḕn khrēmosúnēn elásai\\
best you.\textsc{acc} nor very rich.\textsc{m.acc.sg} possession.\textsc{dat.pl} become.\textsc{aor.inf.mid} nor you.\textsc{acc} then=into much.\textsc{f.acc.sg} need.\textsc{acc.sg} drive.\textsc{aor.inf}\\
\trans `(It is) best for you neither to become very rich in possessions nor to plunge into great poverty.' (Theognis, \textit{Elegies} 559)\footnote{\emph{Translator's note}: The Teubner edition \citep{Hiller1890} has \textit{hṓste}.}
\label{pro2nd1}
\end{exe}

\begin{exe}
\ex οὐδέ μ᾽ εἰ θανεῖν χρεών\\
\gll oudé \emph{m'} ei thaneîn khreṓn\\
nor me.\textsc{acc} if die.\textsc{aor.inf} need\\
\trans `... not even if (it is) necessary for me to die.' (Euripides, \textit{Iphigenia in Tauris} 987)\footnote{\emph{Translator's note}: The Persus edition has \textit{s'}.}
\label{pro2nd2}
\end{exe}

\begin{exe}
\ex ἵνα μ᾽ εἰ καταλάβοι ὁ τόκοϲ ἔτ᾽ ἐν πόλει, τέκοιμι\\
\gll hína \emph{m'} ei kataláboi ho tókos ét' en pólei, tékoimi\\
that me.\textsc{acc} if seize.\textsc{3sg.aor.opt} the.\textsc{m.nom.sg} childbirth.\textsc{nom} still in city.\textsc{dat.sg} beget.\textsc{1sg.aor.opt}\\
\trans `So that if labour should seize me while still in these precincts, I could give birth ...' (Aristophanes, \textit{Lysistrata} 753)
\label{pro2nd3}
\end{exe}

\begin{exe}
\ex ὅϲ μοι δωδεκαταῖοϲ ἀφ᾽ ὧ τάλαϲ οὐδέποθ᾽ ἵκει\\
\gll hós \emph{moi} dōdekataîos aph' hô tálas oudépoth' híkei\\
who.\textsc{m.nom.sg} me.\textsc{acc} twelfth.day.\textsc{m.nom.sg} of
which.\textsc{gen.sg} wretched.\textsc{m.nom.sg} never come.\textsc{3sg.prs}\\
\trans `... who, wretched one, (has been) twelve days since he ever came to me.' (Theocritus 2.4)
\label{pro2nd4}
\end{exe}

\begin{exe}
\ex οὐ γάρ τί μοι Ζεὺϲ ἦν ὁ κηρύξαϲ τάδε\\
\gll ou gár tí \emph{moi} Zeùs ên ho kērúxas táde\\
not for what.\textsc{acc.sg} me.\textsc{dat} Zeus.\textsc{nom} be.\textsc{3sg.imp} the.\textsc{m.nom.sg} proclaim.\textsc{ptcp.aor.m.nom.sg} this.\textsc{n.acc.pl}\\
\trans `Why, because Zeus was not the one proclaiming these things to me.' (Sophocles, \textit{Antigone} 450)
\label{ptcp1}
\end{exe}

\begin{exe}
\ex τίϲ μ᾽ εἶϲιν ἄξων\\
\gll tís \emph{m'} eîsin áxōn\\
who.\textsc{m.nom.sg} me.\textsc{acc} go.\textsc{3sg.prs} lead.\textsc{ptcp.fut.m.nom.sg}\\
\trans `Who will go as my escort?' (Euripides, \textit{Iphigenia in Aulis} 1458)
\label{ptcp2}
\end{exe}

\begin{exe}
\ex πονηρόϲ τίϲ μ᾽ ἔϲται ὁ εἰϲάγων\\
\gll ponērós tís \emph{m'} éstai ho eiságōn\\
evil.\textsc{m.nom.sg} some.\textsc{m.nom.sg} me.\textsc{acc} be.\textsc{3sg.fut.mid} the.\textsc{m.nom.sg} bring.\textsc{ptcp.prs.m.nom.sg}\\
\trans `It will be some villain who brings me there.' (Plato, \textit{Gorgias} 521d)
\label{ptcp3}
\end{exe}

\begin{exe}
\ex πολλά με τὰ παρακαλοῦντα ἦν\\
\gll pollá \emph{me} tà parakaloûnta ên\\
many.\textsc{n.nom.pl} me.\textsc{acc} the.\textsc{n.nom.pl}
urge.\textsc{ptcp.prs.n.nom.pl} be.\textsc{3sg.imp}\\
\trans `Many were the things urging me ...' ({[}Demosthenes{]} 59.1; cf. also \citealp[64]{Kock1864} on Aristophanes, \textit{Birds} 95)
\label{ptcp4}
\end{exe}

\begin{exe}
\ex τάδε τοι προϲδόκα ἔϲεϲθαι\\
\gll táde \emph{toi} prosdóka ésesthai\\
this.\textsc{n.acc.pl} you.\textsc{dat} expect.\textsc{2sg.prs.imper} be.\textsc{fut.inf.mid}\\
\trans `Expect these things for yourself.' (Herodotus, 7.235.4)
\label{ptcp5}
\end{exe}

\begin{exe}
\ex μή μοι θάνῃϲ ϲὺ κοινά\\
\gll mḗ \emph{moi} thánēis sù koiná\\
not me.\textsc{dat} die.\textsc{2sg.aor.sbjv} you.\textsc{nom} common.\textsc{f.nom.sg}\\
\trans `Do not die together with me.' (Sophocles, \textit{Antigone} 546)
\label{ptcp6}
\end{exe}

In taking such a position, the pronoun easily separates words which belong tightly together. Thus, for instance, in (\ref{pronsep1}) and (\ref{pronsep2}) we find the particle\is{particles} \textit{oukéti} `no longer' split apart by \textit{me} and \textit{moi} (\textsc{1sg}); similarly (\ref{pronsep3})--(\ref{pronsep6}), even though otherwise \textit{ei mḗ} and \textit{eàn mḗ} `if not' always occur closely connected to one another. (\ref{pronsep6}) is also an example of this, as well as (\ref{pronsep7}), since otherwise it is normal for \textit{ôn} `then' to occur immediately after the first word in the clause. 

\begin{exe}
\ex οὔ μ᾽ ἔτι, παρθενικαὶ μελιγάρυεϲ ἱμερόφωνοι, γυῖα φέρειν δύναται\\
\gll oú \emph{m'} éti, parthenikaì meligárues himeróphōnoi, guîa phérein dúnatai\\
not me.\textsc{acc} still maiden.\textsc{voc.pl} sweet.voiced.\textsc{f.voc.pl} lovely.sounding.\textsc{f.voc.pl} limb.\textsc{acc.pl} bear.\textsc{inf.prs} can.\textsc{3sg.prs}\\
\trans `Sweet-voiced, lovely-sounding maidens, I can no longer hold out my hands.' (Alcman 26.1)
\label{pronsep1}
\end{exe}

\begin{exe}
\ex οὔ μοι ἔτ᾽ εὐκελάδων ὕμνων μέλει\\
\gll oú \emph{moi} ét' eukeládōn húmnōn mélei\\
not me.\textsc{dat} still melodious.\textsc{m.gen.pl} hymn.\textsc{gen.pl} matter.\textsc{3sg.prs}\\
\trans `Melodious hymns no longer matter to me.' (Fragmenta Lyrica Adespota 5; \citealp[690]{Bergk1882})
\label{pronsep2}
\end{exe}

\begin{exe}
\ex εἴ ϲε μἢν δειναῖϲιν ὄντα ϲυμφοραῖϲ ἐπαρκέϲω\\
\gll eí \emph{se} mḕn deinaîsin ónta sumphoraîs eparkésō\\
if you.\textsc{acc} not=in terrible.\textsc{f.dat.pl} be.\textsc{ptcp.prs.m.acc.sg} mishap.\textsc{dat.pl} help.\textsc{1sg.aor.sbjv}\\
\trans `If I do not help you in these terrible straits ...' (Euripides, \textit{Orestes} 803)\footnote{\emph{Translator's note}: The Perseus edition has \textit{mḕ 'n}.}
\label{pronsep3}
\end{exe}

\begin{exe}
\ex ἐάν μοι μὴ δοκῇ\\
\gll eán \emph{moi} mḕ dokêi\\
if me.\textsc{dat} not seem.\textsc{3sg.prs.sbjv}\\
\trans `If he does not seem to me ...' (Plato, \textit{Apology} 29e)
\label{pronsep4}
\end{exe}

\begin{exe}
\ex ἐάν μοι μὴ εἴπῃϲ\\
\gll eán \emph{moi} mḕ eípēis\\
if me.\textsc{dat} not say.\textsc{2sg.aor.sbjv}\\
\trans `If you do not tell me ...' (Plato, \textit{Phaedrus} 236e)
\label{pronsep5}
\end{exe}

\begin{exe}
\ex οὐδείϲ μέ πω ἠρώτηκεν καινὸν οὐδέν\\
\gll oudeís \emph{mé} pō ērṓtēken kainòn oudén\\
nobody.\textsc{m.nom.sg} me.\textsc{acc} yet ask.\textsc{3sg.prf}
new.\textsc{n.acc.sg} nothing.\textsc{acc.sg}\\
\trans `No one has yet asked me anything new.' (Plato, \textit{Gorgias} 448a)
\label{pronsep6}
\end{exe}

\begin{exe}
\ex θωῦμά μοι ὦν καὶ τοῦτο γέγονεν\\
\gll thōûmá \emph{moi} ôn kaì toûto gégonen\\
wonder.\textsc{nom.sg} me.\textsc{dat} then also this.\textsc{n.nom.sg} become.\textsc{3sg.prf}\\
\trans `So this too is a wonder to me ...' (Herodotus 7.153.4)\footnote{\emph{Translator's note}: The Perseus edition has \textit{thôma ... gégone}.}
\label{pronsep7}
\end{exe}

An attributive \isi{genitive} is separated from its governing word \hyperlink{p359}{\emph{[p359]}} by Ion when he writes (\ref{gensep1}) at the beginning of his \textit{Triagmoí}. Similarly (\ref{gensep2})--(\ref{gensep7}) and (\ref{voc5}) above. (But \textit{emé} is also found in this configuration: (\ref{gensep8}).)

\begin{exe}
\ex ἀρχὴ δέ μοι τοῦ λόγου\\
\gll arkhḕ dé \emph{moi} toû lógou\\
beginning.\textsc{nom.sg} but me.\textsc{dat} the.\textsc{m.gen.sg}
account.\textsc{gen.sg}\\
\trans `And (this is) the beginning of my speech.' (Harpocration s.v. \textit{Íōn})\footnote{\emph{Translator's note}: The Perseus edition has \textit{hḗde}; Wackernagel also
cites \textit{hêdé} as a variant reading \citep[385]{Lobeck1829}.}
\label{gensep1}
\end{exe}

\begin{exe}
\ex τίνοϲ μ᾽ ἕκατι γῆϲ ἀποϲτέλλειϲ\\
\gll tínos \emph{m'} hékati gês apostélleis\\
what.\textsc{gen.sg} me.\textsc{acc} for land.\textsc{gen.sg}
dispatch.\textsc{2sg.prs}\\
\trans `For what reason are you exiling me from this land?' (Euripides, \textit{Medea} 281)
\label{gensep2}
\end{exe}

\begin{exe}
\ex ἁ Δίοϲ μ᾽ ἄλοχοϲ ὤλεϲεν\\
\gll ha Díos \emph{m'} álokhos ṓlesen\\
the.\textsc{f.nom.sg} Zeus.\textsc{gen} me.\textsc{acc} bedfellow.\textsc{nom.sg} destroy.\textsc{3sg.aor}\\
\trans `The wife of Zeus has ruined me.' (Euripides, \textit{Helen} 674)
\label{gensep3}
\end{exe}

\begin{exe}
\ex εἰ οὖν τί ϲε τούτων ἀρέϲκει\\
\gll ei oûn tí \emph{se} toútōn aréskei\\
if so anything.\textsc{nom.sg} you.\textsc{acc} this.\textsc{n.gen.pl}
please.\textsc{3sg.prs}\\
\trans `So if any of this pleases you ...' (Thucydides 1.128.7)
\label{gensep5}
\end{exe}

\begin{exe}
\ex ὅϲουϲ μοι τῶν ϲυγγόνων ἀπώλλυεν\\
\gll hósous \emph{moi} tôn sungónōn apṓlluen\\
how.many.\textsc{m.acc.pl} me.\textsc{dat} the.\textsc{m.gen.pl}
relative.\textsc{m.gen.pl} destroy.\textsc{3sg.imp}\\
\trans `... how many of my relatives he was ruining.' (Andocides 1.47)
\label{gensep6}
\end{exe}

\begin{exe}
\ex Ζηνόϲ τοι θυγάτηρ ὑπὸ τὰν μίαν ἵκετο χλαῖαν\\
\gll Zēnós \emph{toi} thugátēr hupò tàn mían híketo khlaîan\\
Zeus.\textsc{gen} you.\textsc{dat} daughter.\textsc{nom.sg} under the.\textsc{f.acc.sg} one.\textsc{f.acc.sg} come.\textsc{3sg.aor.mid} blanket.\textsc{acc.sg}\\
\trans `Zeus's daughter has come under the same blanket as you.' (Theocritus 18.19)
\label{gensep7}
\end{exe}

\begin{exe}
\ex οὐδεὶϲ ἔμ᾽ ἐχθρῶν προϲβλέπων ἀνέξεται\\
\gll oudeìs \emph{ém'} ekhthrôn prosblépōn anéxetai\\
nobody.\textsc{m.nom.sg} me.\textsc{acc} enemy.\textsc{gen.pl}
behold.\textsc{ptcp.prs.m.nom.sg} sustain.\textsc{3sg.fut}\\
\trans `None of the enemies will be able to bear looking at me.' (Euripides, \textit{Heracleidae} 691)
\label{gensep8}
\end{exe}

In (\ref{attsep1})--(\ref{attsep30}) an attributive adjective\is{adjectives} or pronoun or an appositive is separated from the phrase to which it belongs by an enclitic pronoun.

\begin{exe}
\ex δεϲπότηϲ ϲε Καμβύϲηϲ, Ψαμμήνιτε, εἰρωτᾷ\\
\gll despótēs \emph{se} Kambúsēs, Psammḗnite, eirōtâi\\
master.\textsc{nom.sg} you.\textsc{acc} Cambyses.\textsc{nom}
Psammenitus.\textsc{voc} ask.\textsc{3sg.prs}\\
\trans `Psammenitus, Lord Cambyses asks you ...' (Herodotus 3.14.9)
\label{attsep1}
\end{exe}

\begin{exe}
\ex ἀπὸ ταύτηϲ ϲφι τῆϲ μάχηϲ ... κατεύχεται ὁ κῆρυξ ... Πλαταιεῦϲι\\
\gll apò taútēs \emph{sphi} tês mákhēs kateúkhetai ho kêrux Plataieûsi\\
of this.\textsc{f.gen.sg} them.\textsc{dat} the.\textsc{f.gen.sg} battle.\textsc{gen.sg} pray.\textsc{3sg.prs} the.\textsc{m.nom.sg} herald.\textsc{nom.sg} Plataean.\textsc{dat.pl}\\
\trans `Since this battle, the herald prays for them, the Plataeans.' (Herodotus 6.111.2; here \textit{Plataieûsi} resumes the distantly removed \textit{sphi})
\label{attsep2}
\end{exe}

\begin{exe}
\ex τά ϲε καὶ ἀμφότερα περιήκοντα ἀνθρώπων κακῶν ὁμιλίαι ϲφάλλουϲιν\\
\gll tá se kaì amphótera periḗkonta anthrṓpōn kakôn homilíai sphállousin\\
the.\textsc{n.acc.pl} you.\textsc{acc} also both.\textsc{n.acc.pl}
befall.\textsc{n.acc.pl} person.\textsc{gen.pl} bad.\textsc{m.gen.pl} company.\textsc{nom.pl} trip.\textsc{3pl.prs}\\
\trans `The company of bad men overcomes both the things that have come to you.' (Herodotus 7.16A.1; here \textit{tá} belongs with \textit{amphótera} and \textit{se} with \textit{periḗkonta})
\label{attsep3}
\end{exe}

\begin{exe}
\ex ὀλίγων γάρ ϲφι ἡμερέων\\
\gll olígōn gár \emph{sphi} hēmeréōn\\
few.\textsc{f.gen.pl} for them\textsc{.dat} day.\textsc{gen.pl}\\
\trans `For few days' (provisions are left) to them.' (Herodotus 9.45.2)\footnote{\emph{Translator's note}: The Perseus edition has \textit{oligéōn}.}
\label{attsep4}
\end{exe}

\begin{exe}
\ex ωὑτὸϲ δέ μοι λόγοϲ καὶ ὑπὲρ τῶν ἄλλων\\
\gll hōutòs dé \emph{moi} lógos kaì hupèr tôn állōn\\
the.\textsc{m.nom.sg} but me.\textsc{dat} account.\textsc{nom.sg}
also over the.\textsc{n.gen.pl} other.\textsc{n.gen.pl}\\
\trans `And this (is) also my argument for the others.' ({[}Hippocrates,{]} \textit{De arte}; \citealp[52.18]{Gomperz1890})\\
\label{attsep5}
\end{exe}

\begin{exe}
\ex πολλή μ᾽ ἀνάγκη\\
\gll pollḗ \emph{m'} anánkē\\
much.\textsc{f.nom.sg} me.\textsc{dat} need.\textsc{nom.sg}\\
\trans `(There is) great need for me (to do so).' (Euripides, \textit{Medea} 1013)
\label{attsep6}
\end{exe}

\begin{exe}
\ex Αἴαϲ μ᾽ ἀδελφὸϲ ὤλεϲ᾽ ἐν Τροίᾳ θανών\\
\gll Aías \emph{m'} adelphòs ṓles' en Troíāi thanṓn\\
Ajax.\textsc{nom} me.\textsc{acc} brother.\textsc{nom.sg} destroy.\textsc{3sg.aor} in Troy.\textsc{dat} die.\textsc{ptcp.aor.m.nom.sg}\\
\trans `My brother Ajax undid me, dying at Troy.' (Euripides, \textit{Helen} 94)
\label{attsep7}
\end{exe}

\begin{exe}
\ex τοὐκεῖ με μέγεθοϲ τῶν πόνων πείθει\\
\gll toukeî \emph{me} mégethos tôn pónōn peíthei\\
the=there me.\textsc{acc} magnitude.\textsc{n.nom.sg} the.\textsc{m.gen.pl} trouble.\textsc{gen.pl} persuade.\textsc{3sg.prs}\\
\trans `The magnitude of my troubles there convinces me.' (Euripides, \textit{Helen} 593)
\label{attsep8}
\end{exe}

\begin{exe}
\ex φήμαϲ δέ μοι ἐϲθλὰϲ ἐνεγκών\\
\gll phḗmas dé \emph{moi} esthlàs enenkṓn\\
report.\textsc{acc.pl} but me.\textsc{dat} goodly.\textsc{f.acc.pl}
bear.\textsc{ptcp.aor.m.nom.sg}\\
\trans `And having brought me good news ...' (Euripides, \textit{Helen} 1281)\footnote{\emph{Translator's note}: The Perseus edition has \textit{d' emoì}.}
\label{attsep9}
\end{exe}

\begin{exe}
\ex διϲϲοὶ δέ ϲε Διόϲκοροι καλοῦϲιν\\
\gll dissoì dé \emph{se} Dióskoroi kaloûsin\\
twin.\textsc{m.nom.pl} but you.\textsc{acc} Dioscuri.\textsc{nom.pl} call.\textsc{3pl.prs}\\
\trans `And the twin Dioscuri are calling you.' (Euripides, \textit{Helen} 1643)\footnote{\emph{Translator's note}: The Perseus edition has \textit{kaloûmen}.}
\label{attsep10}
\end{exe}

\begin{exe}
\ex Ἑλένη ϲ᾽ ἀδελφὴ ταῖϲδε δωρεῖται χοαίϲ\\
\gll Helénē \emph{s'} adelphḕ taîsde dōreîtai khoaís\\
Helen.\textsc{nom} you.\textsc{acc} sister.\textsc{nom} the.\textsc{f.dat.pl} present.\textsc{3sg.prs} libation.\textsc{dat.pl}\\
\trans `Helen, your sister, presents you with these libations.' (Euripides, \textit{Orestes} 117)
\label{attsep11}
\end{exe}

\begin{exe}
\ex φίλου μοι πατρόϲ ἐϲτιν ἔκγονοϲ\\
\gll phílou \emph{moi} patrós estin ékgonos\\
dear.\textsc{m.gen.sg} me.\textsc{dat} father.\textsc{gen.sg}
be.\textsc{3sg.prs} offspring.\textsc{nom.sg}\\
\trans `He is the son of a father dear to me.' (Euripides, \textit{Orestes} 482)
\label{Or482}
\end{exe}

\begin{exe}
\ex Φοιβόϲ μ᾽ ὁ Λητοῦϲ παῖϲ ὁδ᾽ ἐγγὺϲ ὢν καλῶ\\
\gll Phoibós \emph{m'} ho Lētoûs paîs hod' engùs ṑn kalô\\
Phoebus.\textsc{nom} me.\textsc{acc} the\textsc{.m.nom.sg} Leto.\textsc{gen} child.\textsc{nom.sg} this.\textsc{m.nom.sg} near be.\textsc{ptcp.prs.m.nom.sg} call.\textsc{1sg.prs}\\
\trans `Being near, I call myself Phoebus, this son of Leto.' (Euripides, \textit{Orestes} 1626)\footnote{\emph{Translator's note}: The Perseus edition has \textit{s'}.}
\label{attsep13}
\end{exe}

\begin{exe}
\ex χρύϲεαι δή μοι πτέρυγεϲ περὶ νώτῳ\\
\gll khrúseai dḗ \emph{moi} ptéruges perì nṓtōi\\
golden.\textsc{f.nom.pl} exactly me.\textsc{dat} wing.\textsc{nom.pl} around back.\textsc{dat.sg}\\
\trans `Golden (are) the wings upon my back.' (Euripides, Fragment 911)
\label{attsep14}
\end{exe}

\begin{exe}
\ex τίϲ γάρ ϲε κήρυξ ἢ γερουϲία Φρυγῶν {[}...{]} οὐκ ἐπέϲκηψεν πόλει\\
\gll tís gár \emph{se} kḗrux ḕ gerousía Phrugôn ouk epéskēpsen pólei\\
which.\textsc{m.nom.sg} for you.\textsc{acc} herald.\textsc{nom.sg}
or senate.\textsc{nom.sg} Phrygian.\textsc{gen.pl} not adjure.\textsc{3sg.aor} city.\textsc{dat}\\
\trans `For what herald or embassy from Phrygia did not summon you for the city?' (Euripides, \textit{Rhesus} 401)
\label{attsep15}
\end{exe}

\begin{exe}
\ex τίνα μοι δύϲτανον ὄνειρον πέμπειϲ\\
\gll tína \emph{moi} dústanon óneiron pémpeis\\
which.\textsc{m.acc.sg} me.\textsc{dat} wretched.\textsc{m.acc.sg}
dream.\textsc{m.acc.sg} send.\textsc{2sg.prs}\\
\trans `What woeful dream do you send to me?' (Aristophanes, \textit{Frogs} 1332, imitating Euripides)
\label{attsep16}
\end{exe}

\begin{exe}
\ex αὐτή τέ μοι δέϲποινα μακαριωτάτη\\
\gll autḗ té \emph{moi} déspoina makariōtátē\\
same.\textsc{f.nom.sg} and me.\textsc{dat} mistress.\textsc{nom.sg}
blessed.\textsc{supl.f.nom.sg}\\
\trans `... and my mistress herself (is) happiest.' (Aristophanes, \textit{Ecclesiazusae} 1113)
\label{attsep17}
\end{exe}

\begin{exe}
\ex πολλὴ μέντἄν με φιλοψυχία ἔχοι\\
\gll pollḕ méntán \emph{me} philopsukhía ékhoi\\
much.\textsc{f.nom.sg} yet=\textsc{irr} me.\textsc{acc}
love.of.life\textsc{.nom.sg} have.\textsc{3sg.prs.opt}\\
\trans `Yet a great love of life would possess me ...' (Plato, \textit{Apology} 37c)
\label{attsep18}
\end{exe}

\begin{exe}
\ex μέγα μοι τεκμήριον τούτου γέγονεν\\
\gll méga \emph{moi} tekmḗrion toútou gégonen\\
great.\textsc{n.nom.sg} me.\textsc{dat} sign.\textsc{nom.sg}
this.\textsc{n.gen.sg} become.\textsc{3sg.prf}\\
\trans `A convincing proof of this has come to me.' (Plato, \textit{Apology} 40c)
\label{attsep19}
\end{exe}

\begin{exe}
\ex οὗτοϲ οὖν ϲοι ὁ λόγοϲ ἐκείνῳ πῶϲ ξυνᾴϲεται\\
\gll hoûtos oûn \emph{soi} ho lógos ekeínōi pôs xunā́isetai\\
this.\textsc{m.nom.sg} so you.\textsc{dat} the.\textsc{m.nom.sg}
account.\textsc{nom.sg} that.\textsc{m.dat.sg} how harmonize.\textsc{3sg.fut.mid}\\
\trans `So how can this theory be brought into harmony with that one for you?' (Plato, \textit{Phaedo} 92c)
\label{attsep20}
\end{exe}

\begin{exe}
\ex μέγα δέ ϲοι τεκμήριον ἐρῶ\\
\gll méga dé \emph{soi} tekmḗrion erô\\
great.\textsc{n.acc.sg} but you.\textsc{dat} sign.\textsc{acc.sg} say.\textsc{1sg.fut}\\
\trans `And I will tell you a striking proof.' (Plato, \textit{Gorgias} 456b)
\label{attsep21}
\end{exe}

\begin{exe}
\ex ἱκανόν μοι τεκμήριον ἐϲτιν\\
\gll hikanón \emph{moi} tekmḗrion estin\\
sufficient.\textsc{n.nom.sg} me.\textsc{dat} sign.\textsc{nom.sg}
be.\textsc{3sg.prs}\\
\trans `There is proof enough for me.' (Plato, \textit{Gorgias} 487d)
\label{attsep22}
\end{exe}

\begin{exe}
\ex τοῦτό μοι αὐτὸ ϲαφῶϲ διόριϲον\\
\gll toûtó \emph{moi} autò saphôs diórison\\
this.\textsc{n.acc.sg} me.\textsc{dat} same.\textsc{n.acc.sg} clearly define.\textsc{2sg.aor.imper}\\
\trans `Declare this very thing clearly for me.' (Plato, \textit{Gorgias} 488d)
\label{attsep23}
\end{exe}

\begin{exe}
\ex φέρε δή, ἄλλην ϲοι εἰκόνα λέγω\\
\gll phére dḗ, állēn \emph{soi} eikóna légō\\
bear.\textsc{2sg.prs.imper} exactly other.\textsc{f.acc.sg} you.\textsc{dat} likeness.\textsc{acc} say.\textsc{1sg.prs.sbjv}\\
\trans `Come now, let me tell you another parable.' (Plato, \textit{Gorgias} 493d)
\label{attsep24}
\end{exe}

\begin{exe}
\ex ὅντινά μοι τρόπον δοκεῖϲ εὖ λέγειν\\
\gll hóntiná \emph{moi} trópon dokeîs eû légein\\
which.\textsc{m.acc.sg} me.\textsc{dat} way.\textsc{acc.sg} seem.\textsc{2sg.prs} well say.\textsc{prs.inf}\\
\trans `... in what way you seem to me to be speaking correctly.' (Plato, \textit{Gorgias} 513c)
\label{attsep25}
\end{exe}

\begin{exe}
\ex τετάρτου μοι γένουϲ αὖ προϲδεῖν φαίνεται\\
\gll tetártou \emph{moi} génous aû prosdeîn phaínetai\\
fourth.\textsc{n.gen.sg} me.\textsc{dat} kind.\textsc{gen.sg} again still.lack.\textsc{prs.inf} appear.\textsc{3sg.prs}\\
\trans `A fourth class as well appears to me to be necessary.' (Plato, \textit{Philebus} 23d)
\label{attsep26}
\end{exe}

\begin{exe}
\ex ὁ ἀνήρ ϲοι ὁ ἐμὸϲ καὶ τἆλλα φίλοϲ ἦν\\
\gll ho anḗr \emph{soi} ho emòs kaì tâlla phílos ên\\
the.\textsc{m.nom.sg} man.\textsc{nom.sg} you.\textsc{dat} the.\textsc{m.nom.sg} my.\textsc{m.nom.sg} also the=other.\textsc{n.acc.pl} friend.\textsc{nom.sg}
be.\textsc{3sg.imp}\\
\trans `My husband was also a friend to you in all other ways.' (Xenophon, \textit{Hellenica} 3.1.11)
\label{attsep27}
\end{exe}

\begin{exe}
\ex δύο δέ μοι τῆϲ κατηγορίαϲ εἴδη λέλειπται\\
\gll dúo dé \emph{moi} tês katēgorías eídē léleiptai\\
two but me.\textsc{dat} the.\textsc{f.gen.sg} charge.\textsc{gen.sg} form.\textsc{n.nom.pl} leave.\textsc{3sg.prf.pass}\\
\trans `Two points of my plea remain for me.' (Aeschines 1.116)\footnote{\emph{Translator's note}: The Perseus edition has \textit{leípetai}.}
\label{attsep28}
\end{exe}

\begin{exe}
\ex ἁ μεγάλα μοι Κύπριϲ ἔθ᾽ ὑπνώντι παρέϲτα\\
\gll ha megála \emph{moi} Kúpris éth' hupnṓnti parésta\\
the.\textsc{f.nom.sg} great.\textsc{f.nom.sg} me.\textsc{dat}
Cypris.\textsc{nom} still sleep.\textsc{ptcp.prs.m.dat.sg} stand.by\textsc{.3sg.aor}\\
\trans `Great Cypris still stood by me in my sleep.' (Bion, Fragment 5.1)
\label{attsep29}
\end{exe}

\hyperlink{p360}{\emph{[p360]}}

\begin{exe}
\ex Ξεῖνε, Συρηκόϲιόϲ τοι ἀνὴρ τόδ᾽ ἐφίεται Ὄρθων\\
\gll Xeîne, Surēkósiós \emph{toi} anḕr tód' ephíetai Órthōn\\
stranger.\textsc{voc} Syracusan.\textsc{m.nom.sg} you.\textsc{dat} man.\textsc{nom.sg} this.\textsc{n.acc.sg} bid.\textsc{3sg.prs.pass} Orthon.\textsc{nom}\\
\trans `Stranger, Orthon, a Syracusan man, asks this of you.' (Anthologia Graeca 7.660)
\label{attsep30}
\end{exe}

I will not exhaustively list the numerous instances in which the verb immediately follows a pronoun so inserted, such as (\ref{pronverb1})--(\ref{pronverb3}), although they too belong here, in my view. In a different respect, (\ref{PlatApol28a}) and similar examples also belong here.

\begin{exe}
\ex τριϲϲαί μ᾽ ἀναγκάζουϲιν ϲυμφορᾶϲ ὁδοί\\
\gll trissaí \emph{m'} anankázousin sumphorâs hodoí\\
triple.\textsc{f.nom.pl} me.\textsc{acc} compel.\textsc{3pl.prs}
circumstance.\textsc{gen.sg} way.\textsc{nom.pl}\\
\trans `Three paths of circumstance compel me ...' (Euripides, \textit{Heracleidae} 232)\footnote{\emph{Translator's note}: The Perseus edition has \textit{sunnoías}.}
\label{pronverb1}
\end{exe}

\begin{exe}
\ex ταύτηϲ μοι δοκεῖ {[}...{]} πολλὰ {[}...{]} μόρια εἶναι\\
\gll taútēs \emph{moi} dokeî pollà mória eînai\\
this.\textsc{f.nom.sg} me.\textsc{dat} seem.\textsc{3sg.prs}
many.\textsc{n.acc.pl} part.\textsc{acc.pl} be.\textsc{prs.inf}\\
\trans `This seems to me to have many branches.' (Plato, \textit{Gorgias} 463b)
\label{pronverb2}
\end{exe}

\begin{exe}
\ex δοῖόϲ με καλεῖ γάμοϲ\\
\gll doîós \emph{me} kaleî gámos\\
double.\textsc{m.nom.sg} me.\textsc{acc} call.\textsc{3sg.prs} marriage.\textsc{nom.sg}\\
\trans `A double marriage calls me.' (Callimachus Epigram 1.3)
\label{pronverb3}
\end{exe}

\begin{exe}
\ex ὅτι πολλή μοι ἀπέχθια γέγονεν καὶ πρὸϲ πολλούϲ\\
\gll hóti pollḗ \emph{moi} apékhthia gégonen kaì pròs polloús\\
that much.\textsc{f.nom.sg} me.\textsc{dat} hatred.\textsc{nom.sg}
become.\textsc{3sg.prf} also to many.\textsc{m.acc.pl}\\
\trans `... that great hatred toward me has also arisen among many.' (Plato, \textit{Apology} 28a)
\label{PlatApol28a}
\end{exe}

In other cases, the pronoun is attached to the article. Sometimes immediately: (\ref{artpron1})--(\ref{artpron3}). Mostly the article is immediately followed by a `postpositive'\is{postpositive particles} particle:\is{particles} (\ref{artpron4})--(\ref{artpron11}). (See also example (\ref{theotu2}) above on p\pageref{theotu2}.)

\begin{exe}
\ex οἵ με φίλοι προδιδοῦϲιν\\
\gll hoí \emph{me} phíloi prodidoûsin\\
the.\textsc{m.nom.pl} me.\textsc{acc} friend.\textsc{nom.pl}
forsake.\textsc{3pl.prs}\\
\trans `My friends forsake me.' (Theognis, \textit{Elegies} 575; cf. also Theognis, \textit{Elegies} 861)
\label{artpron1}
\end{exe}

\begin{exe}
\ex οἵ με φίλοι προὔδωκαν\\
\gll hoí \emph{me} phíloi proúdōkan\\
the.\textsc{m.nom.pl} me.\textsc{acc} friend.\textsc{nom.pl} forsake.\textsc{3pl.prs}\\
\trans `My friends have forsaken me.' (Theognis, \textit{Elegies} 813)
\label{artpron2}
\end{exe}

\begin{exe}
\ex τάν τοι, ἔφα, κορύναν δωρύττομαι\\
\gll tán \emph{toi}, épha, korúnan dōrúttomai\\
the.\textsc{f.acc.sg} you.\textsc{dat} say.\textsc{3sg.imp}
club.\textsc{acc.sg} present.\textsc{1sg.prs.pass}\\
\trans `{``}I present,'' he said, ``the club to you.''' (Theocritus 7.43)
\label{artpron3}
\end{exe}

\begin{exe}
\ex οἱ δέ ϲφι βόεϲ οὐ παρεγένοντο\\
\gll hoi dé \emph{sphi} bóes ou paregénonto\\
the.\textsc{m.nom.pl} but them.\textsc{dat} ox.\textsc{nom.pl} not be.present\textsc{.3pl.imp}\\
\trans `But the oxen had not returned to them.' (Herodotus 1.31.2)
\label{artpron4}
\end{exe}

\begin{exe}
\ex οἱ γάρ με ἐκ τῆϲ κώμηϲ παῖδεϲ {[}...{]} ἐϲτήϲαντο βαϲιλέα\\
\gll hoi gár \emph{me} ek tês kṓmēs paîdes estḗsanto basiléa\\
the.\textsc{m.nom.pl} for me.\textsc{acc} from the.\textsc{f.gen.sg} village.\textsc{gen.sg} child.\textsc{nom.pl} stand.\textsc{3pl.aor.mid} king.\textsc{acc.sg}\\
\trans `For the boys of the village chose me as king.' (Herodotus 1.115.2)
\label{artpron5}
\end{exe}

\begin{exe}
\ex τὰ δέ μοι παθήματα τὰ ἐόντα ἀχάριτα μαθήματα γέγονε\\
\gll tà dé \emph{moi} pathḗmata tà eónta akhárita mathḗmata gégone\\
the.\textsc{n.nom.pl} but me.\textsc{dat} suffering.\textsc{nom.pl}
the.\textsc{n.nom.pl} be.\textsc{ptcp.prs.n.nom.pl} graceless.\textsc{n.nom.pl} lesson.\textsc{nom.pl} become.\textsc{3sg.prf}\\
\trans `And disastrous misfortunes have come to be lessons for me.' (Herodotus 1.207.1)
\label{artpron6}
\end{exe}

\begin{exe}
\ex ὁ δέ μοι μάγοϲ {[}...{]} ταῦτα ἐνετείλατο\\
\gll ho dé \emph{moi} mágos taûta eneteílato\\
the.\textsc{m.nom.sg} but me.\textsc{dat} magus.\textsc{nom.sg}
this.\textsc{n.acc.pl} command.\textsc{3sg.aor.mid}\\
\trans `But the magus gave me this message.' (Herodotus 3.63.2)
\label{artpron7}
\end{exe}

\begin{exe}
\ex ἡ γάρ μοι μήτηρ βέβηκεν ἄλλῃ\\
\gll hē gár \emph{moi} mḗtēr bébēken állēi\\
the.\textsc{f.nom.sg} for me.\textsc{dat} mother.\textsc{nom.sg}
step.\textsc{3sg.prf} elsewhere\\
\trans `For my mother has gone out.' (Aristophanes, \textit{Ecclesiazusae} 913)\footnote{\emph{Translator's note}: The Perseus edition has \textit{állēi bébēke}.}
\label{Eccles913}
\end{exe}

\begin{exe}
\ex ὁ δέ μοι λόγοϲ ὅρκοϲ ἔϲται\\
\gll ho dé \emph{moi} lógos hórkos éstai\\
the.\textsc{m.nom.sg} but me.\textsc{dat} account.\textsc{nom.sg}
oath.\textsc{nom.sg} be.\textsc{3sg.fut.mid}\\
\trans `And what I say will be an oath.' (Plato, \textit{Phaedrus} 236d)
\label{artpron9}
\end{exe}

\begin{exe}
\ex ἡ μέν μοι ἀρχὴ τοῦ λόγου ἐϲτὶ κατὰ τὴν Εὐριπίδου Μελανίππην\\
\gll hē mén \emph{moi} arkhḕ toû lógou estì katà tḕn Euripídou Melaníppēn\\ 
the.\textsc{f.nom.sg} then me.\textsc{dat} beginning.\textsc{nom.sg}
the.\textsc{m.gen.sg} account.\textsc{gen.sg} be.\textsc{3sg.prs} down the.\textsc{f.acc.sg} Euripides.\textsc{gen} Melanippe\textsc{.acc}\\
\trans `The beginning of my speech is in accordance with Euripides' Melanippe.' (Plato, \textit{Symposium} 177a)
\label{artpron10}
\end{exe}

\begin{exe}
\ex τὰ δέ τοι ϲία καρπὸν ἐνείκαι\\
\gll tà dé \emph{toi} sía karpòn eneíkai\\
the.\textsc{n.nom.pl} but you.\textsc{dat} water.parsnip.\textsc{nom.pl} fruit.\textsc{acc.sg} bear.\textsc{3sg.aor.opt}\\
\trans `And the water parsnip would bear fruit for you.' (Theocritus 5.125; cf. also Theocritus 1.82)\footnote{\emph{Translator's note}: The Perseus edition has \textit{t' oísua}.}
\label{artpron11}
\end{exe}

Or the pronoun is attached to a preposition\is{prepositions} and thus separates it from its case: (\ref{preppron1})--(\ref{preppron3}). The preposition\is{prepositions} is followed immediately by a particle\is{particles} in (\ref{preppron4})--(\ref{preppron6}).

\begin{exe}
\ex ἀμφί μοι αὖτε ἄναχθ᾽ ἑκαταβόλον ᾀδέτω ἁ φρήν\\
\gll amphí \emph{moi} aûte ánakhth' hekatabólon āidétō ha phrḗn\\
about me.\textsc{dat} again lord.\textsc{acc} far.shooting\textsc{.m.acc.sg} sing.\textsc{3sg.prs.imper} the.\textsc{f.nom.sg} midriff.\textsc{nom.sg}\\
\trans `Let my heart again sing for me of the far-shooting lord.' (Terpander, Fragment 2)
\label{preppron1}
\end{exe}

\begin{exe}
\ex ἀμφί μοι Ἑρμαίαο φίλον γόνον ἔννεπε Μοῦϲα\\
\gll amphí \emph{moi} Hermaíao phílon gónon énnepe Moûsa\\
about me.\textsc{dat} Hermes.\textsc{gen} dear.\textsc{m.acc.sg}
offspring.\textsc{acc.sg} say.\textsc{2sg.imper} Muse.\textsc{voc}\\
\trans `Tell me, Muse, about the dear son of Hermes.' (Homeric Hymns 19.1)
\label{preppron2}
\end{exe}

\begin{exe}
\ex κατά με γᾶϲ ζῶντα πόρευϲον\\
\gll katá \emph{me} gâs zônta póreuson\\
down me.\textsc{acc} earth.\textsc{gen.sg} live.\textsc{ptcp.prs.m.acc.sg} send.\textsc{2sg.aor.imper}\\
\trans `Bury me alive beneath the earth.' (Euripides, \textit{Rhesus} 831)
\label{preppron3}
\end{exe}

\begin{exe}
\ex ἐν γάρ ϲε τῇ νυκτὶ ταύτῃ ἀναιρέομαι\\
\gll en gár \emph{se} têi nuktì taútēi anairéomai\\
in for you.\textsc{acc} the.\textsc{f.dat.sg} night.\textsc{dat.sg}
this.\textsc{f.dat.sg} take.\textsc{1sg.prs.pass}\\
\trans `For I conceived you that night.' (Herodotus 6.69.4)
\label{preppron4}
\end{exe}

\begin{exe}
\ex ἐν δέ ϲε Παρραϲίῃ Ῥείη τέκεν\\
\gll en dé \emph{se} Parrhasíēi Rheíē téken\\
in but you.\textsc{acc} Parrhasia.\textsc{dat} Rhea.\textsc{nom} beget.\textsc{3sg.aor}\\
\trans `And Rhea gave birth to you in Parrhasia.' (Callimachus, \textit{Hymns} 1.10)
\label{preppron5}
\end{exe}

\begin{exe}
\ex ἐϲ δέ με δάκρυ ἤγαγεν\\
\gll es dé \emph{me} dákru ḗgagen\\
into but me.\textsc{acc} tear.\textsc{acc.sg} lead.\textsc{3sg.aor}\\
\trans `And it brought me to tears.' (Callimachus Epigram 2.1)
\label{preppron6}
\end{exe}

There is also the well-known case in which a \textit{se} (\textsc{2sg.acc}) dependent on a verb of asking (either one that is really present, or one whose reading can be supplied) occurs between \textit{prós} `to' and the \isi{genitive} it `governs', as in (\ref{ask1}). Similar instances are Sophocles, \textit{Philoctetes} 468 (=(\ref{ask4}) below), \textit{Oedipus at Colonus} 250 and 1333 (=(\ref{ask5}) below), and Euripides, \textit{Suppliants} 277. (In contrast, see (\ref{ask2}).) 

\begin{exe}
\ex μή, πρόϲ ϲε τοῦ ϲπείραντοϲ ἄντομαι Διόϲ\\
\gll mḗ, prós \emph{se} toû speírantos ántomai Diós\\
not to you.\textsc{acc} the.\textsc{m.gen.sg} sow.\textsc{ptcp.prs.m.gen.sg} pray.\textsc{1sg.prs} Zeus.\textsc{gen}\\
\trans `I beg you not to, by Zeus who begot you.' (Euripides, \textit{Alcestis} 1098)
\label{ask1}
\end{exe}

\begin{exe}
\ex μή, πρὸϲ γονάτων ϲε πάντωϲ πάντη ϲ᾽ ἱκετεύομεν\\
\gll mḗ, pròs gonátōn \emph{se} pántōs pántē s' hiketeúomen\\
not to knee.\textsc{gen.pl} you.\textsc{acc} all.ways every.way
you.\textsc{acc} beseech\textsc{.1pl.prs}\\
\trans `We beseech you not to, by your knees and in each and every way ...' (Euripides, \textit{Medea} 853)\footnote{\emph{Translator's note}: The Perseus edition has \textit{pántāi pántōs}.}
\label{ask2}
\end{exe}

The verb of asking is to be supplied in (\ref{ask3}), \hyperlink{p361}{\emph{[p361]}} as well as in Euripides, \textit{Medea} 324 and \textit{Andromache} 89 (cf. \textit{Iphigenia in Tauris} 1068). In all these instances, \textit{se} takes second position following the nearest preceding punctuation; (\ref{ask4})--(\ref{ask6}), where the enclitic \textit{nún} `now' precedes \textit{se}, do not of course constitute exceptions.

\begin{exe}
\ex μή, πρόϲ ϲε τοῦ κατ᾽ ἄκρον Οἰταῖον πάγον Διὸϲ καταϲτράπτοντοϲ, ἐκκλέψῃϲ λόγον\\
\gll mḗ, prós \emph{se} toû kat' ákron Oitaîon págon Diòs katastráptontos, ekklépsēis lógon\\
not to you.\textsc{acc} the.\textsc{m.gen.sg} down high.\textsc{m.acc.sg} Oetan\textsc{.m.acc.sg} rock.\textsc{acc} Zeus.\textsc{gen} strike.\textsc{ptcp.pres.m.gen.sg} steal.\textsc{2sg.aor.sbjv} account.\textsc{acc.sg}\\
\trans `By Zeus who hurls lightning down upon the high rock of Oeta, do not rob me of the truth.' (Sophocles, \textit{Women of Trachis} 436)\footnote{\emph{Translator's note}: The Perseus
edition has \textit{nápos}.}
\label{ask3}
\end{exe}

\begin{exe}
\ex πρόϲ νύν ϲε πατρόϲ\\
\gll prós nún \emph{se} patrós\\
to now you.\textsc{acc} father.\textsc{gen.sg}\\
\trans `Now by your father (I beg) you ...' (Sophocles, \textit{Philoctetes} 468)
\label{ask4}
\end{exe}

\begin{exe}
\ex πρόϲ νύν ϲε κρηνῶν\\
\gll prós nún \emph{se} krēnôn\\
to now you.\textsc{acc} spring.\textsc{gen.pl}\\
\trans `Now by the streams (I ask) you ...' (Sophocles, \textit{Oedipus at Colonus} 1333)
\label{ask5}
\end{exe}

\begin{exe}
\ex πρόϲ νύν ϲε γονάτων τῶνδ(ε)\\
\gll prós nún \emph{se} gonátōn tônd(e)\\
to now you.\textsc{acc} knee.\textsc{gen.pl} this.\textsc{n.gen.pl}\\
\trans `Now by these knees (I ask) you ...' (Euripides, \textit{Helen} 1233)
\label{ask6}
\end{exe}

From the non-Attic\il{Greek, Attic} poets\is{poetry} one can adduce (\ref{ask7}). Apollonius, whom we have to thank for this fragment, seems however to treat \textit{te} as orthotonic and to recognize only \textit{tu} as enclitic \isi{accusative} form in Doric.\il{Greek, Doric} But enclitic Doric\il{Greek, Doric} \textit{te} is confirmed by the words of the Megarian in (\ref{ask8}), in which, because of unwillingness to recognize \textit{tè}, one feels obliged to insert \textit{tu} with an unattractive hiatus.

\begin{exe}
\ex πρὸϲ δέ τε τῶν φίλων\\
\gll pròs dé \emph{te} tôn phílōn\\
to but you.\textsc{acc} the.\textsc{m.gen.pl} friend.\textsc{gen.pl}\\
\trans `And by your friends (I sigh to) you.' (Alcman, Fragment 52.1)
\label{ask7}
\end{exe}

\begin{exe}
\ex πάλιν τ᾽ ἀποιϲῶ ναὶ τὸν Ἑρμᾶν οἴκαδιϲ\\
\gll pálin \emph{t'} apoisô naì tòn Hermân oíkadis\\
again you.\textsc{acc} take.\textsc{1sg.fut} yes the.\textsc{m.acc.sg}
Hermes.\textsc{acc} home\\
\trans `By Hermes, I will take you back home.' (Aristophanes, \textit{Acharnians} 779)\footnote{\emph{Translator's note}: The Perseus edition has \textit{tu}.}
\label{ask8}
\end{exe}

In particular, though, we should compare example (\ref{ask9}): \textit{potí te Zēnòs} (from Codex Palatinus \textit{potitezēnos}). \citet[234]{Blomfield1815} unnecessarily emends\is{emendation} to enclitic \textit{tu}. Still, the accusation levelled at him by \citet[383]{Schneider1873} that he `erred horribly' should be turned back against Schneider himself and his preferred Vulgate reading \textit{potì tè Zanòs} with senseless accenting and false \isi{genitive} \textit{Zanòs}.

\begin{exe}
\ex ποτί τε Ζηνὸϲ ἱκνεῦμαι λιμενοϲκόπω\\
\gll potí \emph{te} Zēnòs hikneûmai limenoskópō\\
to you.\textsc{acc} Zeus.\textsc{gen} beseech.\textsc{1sg.prs.pass}
harbour.watching.\textsc{m.gen.sg}\\
\trans `I beseech you by Zeus, the guardian of the harbour.' (Callimachus, Fragment 114; Anthologia Graeca 13.10.1)
\label{ask9}
\end{exe}

Without taking into consideration these last two examples, \citet[4f.]{Christ1891} has expressed the opinion with regard to (\ref{Pind1.48}) that the \textit{te}, which makes an unpromising particle,\is{particles} should be read as the \isi{accusative} of the pronoun, much as \citet[17]{Bergk1866} wanted to insert \textit{se}. The position of \textit{te} speaks in favour of this reading.

\begin{exe}
\ex ὕδατοϲ ὅτι τε πυρὶ ζέοιϲαν εἰϲ ἀκμὰν μαχαίρᾳ τάμον κατὰ μέλη\\
\gll húdatos hóti \emph{te} purì zéoisan eis akmàn makhaírāi támon katà mélē\\
water.\textsc{gen.sg} that you.\textsc{acc} fire.\textsc{dat.sg}
boil.\textsc{ptcp.prs.f.acc.sg} into edge.\textsc{acc.sg} knife.\textsc{dat.sg} cut.\textsc{3pl.aor} down limb.\textsc{acc.pl}\\
\trans `... that they cut you limb from limb with a knife into the full boiling of the water on the fire.' (Pindar, \textit{Olympian Ode} 1.48)\footnote{\emph{Translator's note}: The Perseus edition has \textit{se}.}
\label{Pind1.48}
\end{exe}

The old positional law also makes its influence known with regard to the connection between the preposition\is{prepositions} and the verb\label{VPtmesis} \citep[§68.48.3]{Krueger1871}. The following examples of post-Homeric tmesis\is{tmesis|(} can be adduced: (\ref{tmesis13})--(\ref{tmesis20}) \hyperlink{p362}{\emph{[p362]}} and (\ref{tmesis21})--(\ref{tmesis27}).

\begin{exe}
\ex ἔκ μ᾽ ἔλαϲαϲ ἀλγέων\\
\gll ék \emph{m'} élasas algéōn\\
from me.\textsc{acc} drive.\textsc{2sg.aor} pain.\textsc{gen.pl}\\
\trans `You have driven out my pains.' (Alcaeus, Fragment 95)
\label{tmesis13}
\end{exe}

\begin{exe}
\ex ἀπό μοι θανεῖν γένοιτ(ο)\\
\gll apó \emph{moi} thaneîn génoit(o)\\
off me.\textsc{dat} die.\textsc{aor.inf}
become.\textsc{3sg.aor.mid.opt}\\
\trans `May death come to me.' (Anacreon, Fragment 50.1)
\label{tmesis14}
\end{exe}

\begin{exe}
\ex ἀπό ϲ᾽ ὀλέϲειεν Ἄρτεμιϲ, ϲὲ δὲ κὠπόλλων\\
\gll apó \emph{s'} oléseien Ártemis, sè dè kōpóllōn\\
off you.\textsc{acc} destroy.\textsc{3sg.aor.opt} Artemis.\textsc{nom}
you.\textsc{acc} but also=Apollo.\textsc{nom}\\
\trans `May Artemis destroy you, and Apollo too.' (Hipponax, Fragment 31)
\label{tmesis15}
\end{exe}

\begin{exe}
\ex κατά μοι βόαϲον\\
\gll katá \emph{moi} bóason\\
down me.\textsc{dat} shout.\textsc{2sg.aor.imper}\\
\trans `Shout down to me.' (Sophocles, \textit{Electra} 1067)
\label{tmesis16}
\end{exe}

\begin{exe}
\ex ἀπό μ᾽ ὀλεῖϲ\\
\gll apó \emph{m'} oleîs\\
off me.\textsc{acc} destroy.\textsc{2sg.fut}\\
\trans `You will destroy me.' (Sophocles, \textit{Philoctetes} 817)
\label{tmesis17}
\end{exe}

\begin{exe}
\ex κατά με φόνιοϲ Ἀίδαϲ ἕλοι\\
\gll katá \emph{me} phónios Aídas héloi\\
down me.\textsc{acc} murderous.\textsc{m.nom.sg} Hades.\textsc{nom}
take.\textsc{3sg.aor.opt}\\
\trans `May murderous Hades take me.' (Sophocles, \textit{Oedipus at Colonus} 1689)
\label{tmesis18}
\end{exe}

\begin{exe}
\ex διά μ᾽ ὀλεῖτε\\
\gll diá \emph{m'} oleîte\\
through me.\textsc{acc} destroy.\textsc{2pl.fut}\\
\trans `You will be my ruin.' (Euripides, \textit{Heracleidae} 1053)
\label{tmesis19}
\end{exe}

\begin{exe}
\ex ἀνά μοι τέκνα λῦϲαι\\
\gll aná \emph{moi} tékna lûsai\\
up me.\textsc{dat} child.\textsc{acc.pl} loose.\textsc{2sg.aor.imper.mid}\\
\trans `Release my children.' (Euripides, \textit{Suppliants} 45)\footnote{\emph{Translator's note}: The Perseus edition has \textit{ánomoi}.}
\label{tmesis20}
\end{exe}

\begin{exe}
\ex κατά με πέδον γᾶϲ ἕλοι\\
\gll katá \emph{me} pédon gâs héloi\\
down me.\textsc{acc} ground.\textsc{nom.sg} earth.\textsc{gen.sg}
take.\textsc{3sg.aor.opt.mid}\\
\trans `May the earth's floor swallow me.' (Euripides, \textit{Suppliants} 829)
\label{tmesis21}
\end{exe}

\begin{exe}
\ex διά μ᾽ ἔφθειραϲ\\
\gll diá \emph{m'} éphtheiras\\
through me.\textsc{acc} destroy.\textsc{2sg.aor}\\
\trans `You have destroyed me.' (Euripides, \textit{Hippolytus} 1357)
\label{tmesis22}
\end{exe}

\begin{exe}
\ex ἀνά μ᾽ ἐκάλεϲεν\\
\gll aná \emph{m'} ekálesen\\
up me.\textsc{acc} call.\textsc{3sg.aor}\\
\trans `(Whence did the voice) summon me?' (Euripides, \textit{Bacchae} 579)
\label{tmesis23}
\end{exe}

\begin{exe}
\ex κατά ϲε χώϲομεν\\
\gll katá \emph{se} khṓsomen\\
down you.\textsc{acc} bury.\textsc{1pl.fut}\\
\trans `We will bury you.' (Aristophanes, \textit{Acharnians} 295)
\label{tmesis24}
\end{exe}

\begin{exe}
\ex ἀπό ϲ᾽ ὀλῶ κακὸν κακῶϲ\\
\gll apó \emph{s'} olô kakòn kakôs\\
off you.\textsc{acc} destroy.\textsc{1sg.fut} bad.\textsc{n.acc.sg} badly\\
\trans `I will do you great harm.' (Aristophanes, \textit{Plutus} 65)
\label{tmesis25}
\end{exe}

\begin{exe}
\ex ξύμ μοι λαβέϲθε τοῦ μύθου\\
\gll xúm \emph{moi} labésthe toû múthou\\
with me.\textsc{dat} take.\textsc{2pl.aor.imper.mid} the.\textsc{m.gen.sg} myth.\textsc{gen.sg}\\
\trans `Assist me with the tale.' (Plato, \textit{Phaedrus} 237a)
\label{tmesis26}
\end{exe}

\begin{exe}
\ex εἰ δ᾽ ἄγε, ϲύμ μοι βούλευϲον\\
\gll ei d' áge, súm \emph{moi} boúleuson\\
if then lead.\textsc{2sg.prs.imper} with me.\textsc{dat} advise.\textsc{2sg.aor.imper}\\
\trans `But come now, advise me.' (Callimachus Epigram 1.5)
\label{tmesis27}
\end{exe}

With a preceding particle\is{particles} or similar: (\ref{tmesis28})--(\ref{tmesis31}). See above p\pageref{nin} for similar examples with \textit{nin}.

\begin{exe}
\ex ἀπὸ νύν με λείπετ᾽ ἤδη\\
\gll apò nún \emph{me} leípet' ḗdē\\
off now me.\textsc{acc} leave.\textsc{2pl.prs.imper} already\\
\trans `Leave me now immediately' (Sophocles, Phil. 1177)
\label{tmesis28}
\end{exe}

\begin{exe}
\ex ἔκ τοί με τήξειϲ\\
\gll ék toí \emph{me} tḗxeis\\
out lo me.\textsc{acc} melt.\textsc{2sg.fut}\\
\trans `Oh, you will melt my heart.' (Euripides, \textit{Orestes} 1047)
\label{tmesis29}
\end{exe}

\begin{exe}
\ex ἔν τί ϲοι παγήϲεται\\
\gll én tí \emph{soi} pagḗsetai\\
in something.\textsc{nom} you.\textsc{dat} stick.\textsc{3sg.fut.pass}\\
\trans `Something will get stuck into you.' (Aristophanes, \textit{Wasps} 437)
\label{tmesis30}
\end{exe}

\begin{exe}
\ex ἀνά τοί με πείθειϲ\\
\gll aná toí \emph{me} peítheis\\
up lo me.\textsc{acc} persuade.\textsc{2sg.prs}\\
\trans `You are convincing me.' (Aristophanes, \textit{Wasps} 784)
\label{tmesis31}
\end{exe}

If in isolated cases (Alcaeus Fragment 68 given by \citealp{Bekker1833}, erroneously, as (\ref{tmesis32})) the pronoun does not come to be in second position through such tmesis, this should not bother us much.

\begin{exe}
\ex τύφωϲ ἔκ ϲ᾽ ἕλετο φρέναϲ\\
\gll túphōs ék \emph{s'} héleto phrénas\\
fever.\textsc{nom.sg} out you.\textsc{acc} take.\textsc{3sg.aor.mid}
midriff.\textsc{acc.pl}\\
\trans `A fever has taken your wits.' (Alcaeus, Fragment 68)\footnote{\emph{Translator's note}: The TLG edition \citep{LobelPage1968} has \textit{etúphōs}.}
\label{tmesis32}
\end{exe}\il{Greek, Classical|)}\is{pronouns|)}\is{tmesis|)}\is{enclitics|)}


\section{Genitives}\label{genitives}\is{genitive|(}

The \isi{pronouns} \textit{moi} (\textsc{1sg}), \textit{toi} (\textsc{2sg}), (\textit{sphi} \textsc{3pl},) \textit{meo}/\textit{meu}/\textit{mou} (me.\textsc{gen}), \textit{seo}/\textit{seu}/\textit{sou} (you.\textsc{gen}), and \textit{spheōn} (\textsc{3pl.gen}) as attributive genitives deserve special consideration. I regard it as certain that \textit{moi} and \textit{toi}, like \textit{hoi}, did not take on the genitive function only later, but rather had this function from the start, like their Indic\il{Indo-Iranian} correlates \emph{mē}, \emph{tē} and \emph{sē}, and have nothing to do with the locative (cf. \citealp[205]{Delbrueck1888}). That the genitive function is retained in Greek not only in Homer (see \citealp[819]{Brugmann1890}, \citealp[39]{Wackernagel1891}) and the Ionic\il{Greek, Ionic} poets\is{poetry} can be seen above all in Wilamowitz's \citeyearpar[167]{Wilamowitz1889} comment\label{wilamowitz} on example (\ref{EurHer626}): ``In the address, the drama is conveyed by the expression of the possessive relation in kinship terms using the \isi{dative}, \textit{thúgatér moi} `daughter.\textsc{voc} me.\textsc{dat}', \textit{téknon moi} `child.\textsc{voc} me.\textsc{dat}' (Euripides \textit{Ion} 1399, \textit{Orestes} 124, \textit{Iphigenia in Aulis} 613), \textit{gúnai moi} `woman.\textsc{voc} me.\textsc{dat}'. The genitive is not at all common; its entrance into the language, for instance in the Jewish-Christian literature, is rather a sign of the common folk.''

\begin{exe}
\ex ϲύ τ᾽ ὦ γύναι μοι, ϲύλλογον ψυχῆϲ λαβέ\\
\gll sú t' ô gúnai \emph{moi}, súllogon psukhês labé\\
you.\textsc{nom} and O woman.\textsc{voc} me.\textsc{dat}
collection.\textsc{acc.sg} soul.\textsc{gen.sg} take.\textsc{2sg.aor.imper}\\
\trans `You too, my wife, collect your courage.' (Euripides, \textit{Heracleidae} 626)
\label{EurHer626}
\end{exe}

The most natural position for these genitives seems to us to be following their nouns. As is well known, although this position often occurs, for instance in the \isi{vocative} constructions discussed by \citet{Wilamowitz1889}, the equally justifiable position preceding the noun and its attributives (including the article) is also found. The origin of this strange positioning becomes clear when we look at the oldest examples.\il{Greek, Homeric|(} Homer has this positioning in examples (\ref{Homer1})--(\ref{Homer9}). In all of these cases, our positional rule effects this ordering. Later authors allowed themselves to remove these genitives further from the beginning of the clause, but nevertheless frequently retained the preposing that followed from the old positional rule. Various effects of the original connection between preposing and the old positional rule can, however, be seen.\il{Greek, Classical|(}

\begin{exe}
\ex καὶ μέν μευ βουλέων ξύνειν\\
\gll kaì mén \emph{meu} bouléōn xúnein\\
and then me.\textsc{gen} will.\textsc{gen.pl} heed.\textsc{3pl.imp}\\
\trans `And they listened to my counsel.' (Homer, \textit{Iliad} 1.273)
\label{Homer1}
\end{exe}

\hyperlink{p363}{\emph{[p363]}}

\begin{exe}
\ex οἵ μευ κουρδίην ἄλοχον καὶ κτήματα πολλὰ μάψ᾽ οἴχεϲθ᾽ ἀνάγοντεϲ\\
\gll hoí \emph{meu} kourdíēn álokhon kaì ktḗmata pollà máps' oíkhesth' anágontes\\
who.\textsc{m.nom.pl} me.\textsc{gen} wedded.\textsc{f.acc.sg} bedfellow.\textsc{acc.sg} and property.\textsc{acc.pl} much.\textsc{n.acc.pl} vainly go.\textsc{2pl.prs.pass} take.\textsc{ptcp.prs.m.nom.pl}\\
\trans `For you bare forth wantonly over sea my wedded wife and therewithal much treasure.' (Homer, \textit{Iliad} 13.626)
\label{Homer2}
\end{exe}

\begin{exe}
\ex καί μευ κλέοϲ ἦγον Ἀχαιοί\\
\gll kaí \emph{meu} kléos êgon Akhaioí\\
and me.\textsc{gen} fame.\textsc{acc.sg} lead.\textsc{3pl.imp} Achaean.\textsc{m.nom.pl}\\
\trans `And the Achaeans would have spread my fame.' (Homer, \textit{Odyssey} 5.311)
\label{Homer3}
\end{exe}

\begin{exe}
\ex καί μευ κλέοϲ οὐρανὸν ἵκει\\
\gll kaí \emph{meu} kléos ouranòn híkei\\
and me.\textsc{gen} fame.\textsc{nom.sg} heaven.\textsc{acc.sg} come.\textsc{3sg.prs}\\
\trans `And my fame reaches unto heaven.' (Homer, \textit{Odyssey} 9.20)
\label{Homer4}
\end{exe}

\begin{exe}
\ex ἦ μή τίϲ ϲευ μῆλα βροτῶν ἀέκοντοϲ ἐλαύνει\\
\gll ê mḗ tís \emph{seu} mêla brotôn aékontos elaúnei\\
in.truth not some.\textsc{m.nom.sg} you.\textsc{gen} sheep.\textsc{acc.pl} mortal.\textsc{gen.pl} unwilling.\textsc{m.gen.sg} drive.\textsc{3sg.prs}\\
\trans `Can it be that some mortal man is driving off your flocks against your will?' (Homer, \textit{Odyssey} 9.405)
\label{Homer5}
\end{exe}

\begin{exe}
\ex οἵ μευ βοῦϲ ἔκτειναν\\
\gll hoí \emph{meu} boûs ékteinan\\
who.\textsc{m.nom.pl} me.\textsc{gen} cow.\textsc{acc.pl} kill.\textsc{3pl.aor}\\
\trans `... who have slain my cows ...' (Homer, \textit{Odyssey} 12.379)
\label{Homer6}
\end{exe}

\begin{exe}
\ex οἵ μευ πατέρ᾽ ἀμφεπένοντο\\
\gll hoí \emph{meu} patér' amphepénonto\\
who.\textsc{m.nom.pl} me.\textsc{gen} father.\textsc{acc.sg} serve.\textsc{3pl.imp.pass}\\
\trans `... who waited on my father' (Homer, \textit{Odyssey} 15.467)
\label{Homer7}
\end{exe}

\begin{exe}
\ex καί ϲευ φίλα γούναθ᾽ ἱκάνω\\
\gll kaí \emph{seu} phíla goúnath' hikánō\\
and you.\textsc{gen} dear.\textsc{n.acc.pl} knee.\textsc{n.acc.pl} come.\textsc{1sg.prs}\\
\trans `I am come to your dear knees.' (Homer, \textit{Odyssey} 13.231)
\label{Homer8}
\end{exe}

\begin{exe}
\ex τῷ κέ ϲφεων γούνατ᾽ ἔλυϲα\\
\gll tôi ké \emph{spheōn} goúnat' élusa\\
thus \textsc{irr} them.\textsc{gen} knee.\textsc{acc.pl} loose.\textsc{1sg.aor}\\
\trans `So should I have loosened the knees of many of them.' (Homer, \textit{Odyssey} 24.381)
\label{Homer9}
\end{exe}\il{Greek, Homeric|)}

First, preposed genitives often occupy the second position in the clause after all. For \textit{moi} and \textit{toi} I refer you to examples (\ref{Herodotusmoi})--\ref{Sophocles,moi}. 

\begin{exe}
\ex μαρτυρέει δέ μοι τῇ γνώμῃ καὶ Ὁμήρου ἔποϲ\\
\gll marturéei dé \emph{moi} têi gnṓmēi kaì Homḗrou épos\\
testify.\textsc{3sg.prs} but me.\textsc{dat} the.\textsc{f.dat.sg}
opinion.\textsc{dat.sg} also Homer.\textsc{gen} word.\textsc{nom.sg}\\
\trans `A verse of Homer also supports my opinion.' (Herodotus 4.29.1)
\label{Herodotusmoi}
\end{exe}

\begin{exe}
\ex ὅϲ τοι τὸν πατέρα δωρήϲατο\\
\gll hós \emph{toi} tòn patéra dōrḗsato\\
who.\textsc{m.nom.sg} you.\textsc{dat} the.\textsc{m.acc.sg}
father.\textsc{acc.sg} present.\textsc{3sg.aor.mid}\\
\trans `... who presented to your father ...' (Herodotus 7.27.2)\footnote{\emph{Translator's note}: The Perseus edition has \textit{edōrḗsato} for \textit{dōrḗsato}.}
\label{Herodotustoi}
\end{exe}

\begin{exe}
\ex ἥ μοι μητρὶ μὲν θανεῖν μόνη μεταίτιοϲ\\
\gll hḗ \emph{moi} mētrì mèn thaneîn mónē metaítios\\
where me.\textsc{dat} mother.\textsc{dat} then die.\textsc{aor.inf} alone.\textsc{f.nom.sg} guilty.\textsc{f.acc.sg}\\
\trans `... when she alone (is) to blame for my mother's death' (Sophocles, \textit{Women of Trachis} 1233)
\label{Sophocles,moi}
\end{exe}

For the actual genitive forms see example (\ref{kaimou4}) above and examples (\ref{gen1})--(\ref{gen26}) below, which of course do not come close to being an exhaustive list of attestations.

\begin{exe}
\ex λαιμᾷ δέ ϲευ τὸ χεῖλοϲ\\
\gll laimâi dé \emph{seu} tò kheîlos\\
hunger.\textsc{3sg.prs} but you.\textsc{gen} the.\textsc{n.nom.sg} lip.\textsc{nom.sg}\\
\trans `Your lips are hungry.' (Hipponax, Fragment 76)
\label{gen1}
\end{exe}

\begin{exe}
\ex λάβετέ μευ θαἰμάτια\\
\gll lábeté \emph{meu} thaimátia\\
take.\textsc{2pl.aor.imper} me.\textsc{gen} the=garment.\textsc{acc.pl}\\
\trans `Take my clothes.' (Hipponax, Fragment 83)
\label{gen2}
\end{exe}

\begin{exe}
\ex ἔχειϲ δέ μευ τὸν ἀδελφεόν\\
\gll ékheis dé \emph{meu} tòn adelpheón\\
have.\textsc{2sg.prs} but me.\textsc{gen} the.\textsc{m.acc.sg} brother.\textsc{acc.sg}\\
\trans `You have my brother with you.' (Herodotus 4.80.3)
\label{gen3}
\end{exe}

\begin{exe}
\ex ϲὺ δέ μευ ϲυμβουλίην ἔνδεξαι\\
\gll sù dé \emph{meu} sumboulíēn éndexai\\
you.\textsc{nom} but me.\textsc{gen} advice.\textsc{acc.sg} accept.\textsc{2sg.aor.imper.mid}\\
\trans `But take my advice.' (Herodotus 7.51.1)
\label{gen4}
\end{exe}

\begin{exe}
\ex ὥϲ ϲου ϲυμφορὰϲ οἰκτίρομεν\\
\gll hṓs \emph{sou} sumphoràs oiktíromen\\
how you.\textsc{gen} mishap.\textsc{acc.pl} pity.\textsc{1pl.prs}\\
\trans `How we pity your misfortune.' (Euripides, \textit{Medea} 1233)
\label{gen5}
\end{exe}

\begin{exe}
\ex ἥ μου τὰϲ τύχαϲ ὤχει μόνη\\
\gll hḗ \emph{mou} tàs túkhas ṓkhei mónē\\
which.\textsc{f.nom.sg} me.\textsc{gen} the.\textsc{f.acc.pl} fortune.\textsc{acc.pl} sustain.\textsc{3sg.imp} alone.\textsc{f.nom.sg}\\
\trans `... which alone sustained my fortunes ...' (Euripides, \textit{Helen} 277)
\label{gen6}
\end{exe}

\begin{exe}
\ex ἔθιγέ μου φρενῶν\\
\gll éthigé \emph{mou} phrenôn\\
touch.\textsc{3sg.aor} me.\textsc{gen} midriff.\textsc{gen.pl}\\
\trans `It touches my heart.' (Euripides, \textit{Suppliants} 1162)
\label{gen7}
\end{exe}

\begin{exe}
\ex ϲύ μου τὸ δεινὸν καὶ διαφθαρὲν φρενῶν ἴϲχναινε\\
\gll sú \emph{mou} tò deinòn kaì diaphtharèn phrenôn ískhnaine\\
you.\textsc{nom} me.\textsc{gen} the.\textsc{n.acc.sg} terrible.\textsc{n.acc.sg} and corrupt.\textsc{ptcp.aor.pass.n.acc.sg} midriff.\textsc{gen.pl} reduce.\textsc{2sg.prs.imper}\\
\trans `It is for you to calm the terrors and distorted fancies of my brain.' (Euripides, \textit{Orestes} 297)
\label{gen8}
\end{exe}

\begin{exe}
\ex κυνοκοπήϲω ϲου τὸ νῶτον\\
\gll kunokopḗsō \emph{sou} tò nôton\\
dog.whip.\textsc{1sg.fut} you.\textsc{gen} the.\textsc{n.acc.sg} back.\textsc{acc.sg}\\
\trans `I will beat your back like a dog.' (Aristophanes, \textit{Knights} 289)
\label{gen9}
\end{exe}

\begin{exe}
\ex ἀπονυχιῶ ϲου τἀν πρυτανείῳ ϲιτία\\
\gll aponukhiô \emph{sou} tan prutaneíōi sitía\\
clip.\textsc{1sg.pl} you.\textsc{gen} the=in court.\textsc{dat.sg}
loaf.\textsc{acc.pl}\\
\trans `I will cut off your meals at the town hall.' (Aristophanes, \textit{Knights} 709)
\label{gen10}
\end{exe}

\begin{exe}
\ex ἀπώλεϲάϲ μου τὴν τέχνην καὶ τὸν βίον\\
\gll apṓlesás \emph{mou} tḕn tékhnēn kaì tòn bíon\\
destroy.\textsc{2sg.aor} me.\textsc{gen} the.\textsc{f.acc.sg} craft.\textsc{acc.sg} and the.\textsc{m.acc.sg} living.\textsc{acc.sg}\\
\trans `You have ruined my business and my livelihood.' (Aristophanes, \textit{Peace} 1212)
\label{gen11}
\end{exe}

\begin{exe}
\ex καλῶϲ γέ μου τὸν υἱόν ὦ Στιλβωνίδη {[}...{]} οὐκ ἔκυϲαϲ\\
\gll kalôs gé \emph{mou} tòn huión ô Stilbōnídē ouk ékusas\\
well even me.\textsc{gen} the.\textsc{m.acc.sg} son.\textsc{acc} O Stilbonides.\textsc{voc} not kiss.\textsc{2sg.aor}\\
\trans `Stilbonides, you kindly did not even kiss my son.' (Aristophanes, \textit{Birds} 139)
\label{gen12}
\end{exe}

\begin{exe}
\ex ὀρχουμένηϲ μου τῆϲ γυναικὸϲ ἑϲπέραϲ ἡ βάλανοϲ ἐκπέπτωκεν\\
\gll orkhouménēs \emph{mou} tês gunaikòs hespéras hē bálanos ekpéptōken\\
dance.\textsc{ptcp.prs.pass.f.gen.sg} me.\textsc{gen} the.\textsc{f.gen.sg} woman.\textsc{gen.sg} evening.\textsc{gen.sg} the.\textsc{f.nom.sg} clasp.\textsc{nom.sg} fall.out.\textsc{3sg.prf}\\
\trans `The clasp fell off one night while my wife was dancing.' (Aristophanes, \textit{Lysistrata} 409)
\label{gen13}
\end{exe}

\begin{exe}
\ex διττούϲ μου τοὺϲ κατηγόρουϲ γεγονέναι\\
\gll dittoús \emph{mou} toùs katēgórous gegonénai\\
double.\textsc{m.acc.pl} me.\textsc{gen} the.\textsc{m.acc.pl} accuser.\textsc{acc.pl} become.\textsc{inf.prf}\\
\trans `My accusers are twofold.' (Plato, \textit{Apology} 18d)
\label{gen14}
\end{exe}

\begin{exe}
\ex εἰ μέν ϲου τὼ υἱέε πώλω ἢ μόϲχω ἐγενέϲθην\\
\gll ei mén \emph{sou} tṑ huiée pṓlō ḕ móskhō egenésthēn\\
if then you.\textsc{gen} the.\textsc{m.nom.du} son.\textsc{nom.du} foal.\textsc{nom.du} or calf.\textsc{nom.du} become.\textsc{3du.aor.mid}\\
\trans `If your two sons had been born foals or calves...' (Plato, \textit{Apology} 20a)\footnote{\emph{Translator's note}: The Perseus edition has \textit{hueî} for \textit{huiée}.}
\label{gen15}
\end{exe}

\begin{exe}
\ex καταψήϲαϲ οὖν μου τὴν κεφαλὴν\\
\gll katapsḗsas oûn \emph{mou} tḕn kephalḕn\\
stroke.\textsc{ptcp.aor.m.nom.sg} so me.\textsc{gen} the.\textsc{f.acc.sg} head.\textsc{acc.sg}\\
\trans `So, stroking my head ...' (Plato, \textit{Phaedo} 89b)
\label{gen16}
\end{exe}

\begin{exe}
\ex ἐβίαϲέ μου τὴν γυναῖκα\\
\gll ebíasé \emph{mou} tḕn gunaîka\\
force.\textsc{3sg.aor} me.\textsc{gen} the.\textsc{f.acc.sg} woman.\textsc{acc.sg}\\
\trans `He has violated my wife.' (Alcaeus, Comic Fragment 29; \citealp{Kock1880})
\label{gen17}
\end{exe}

\begin{exe}
\ex ἀφομοιοῖ γάρ μου τὴν φύϲιν τοῖϲ Σειρῆϲιν\\
\gll aphomoioî gár \emph{mou} tḕn phúsin toîs Seirêsin\\
liken.\textsc{3sg.prs} for me.\textsc{gen} the.\textsc{f.acc.sg}
nature.\textsc{acc.sg} the.\textsc{m.dat.pl} Siren.\textsc{dat.pl}\\
\trans `For he likens my nature to the Sirens.' (Aeschines 3.228)\footnote{\emph{Translator's note}: The Perseus edition has \textit{taîs} for \textit{toîs}.}
\label{gen18}
\end{exe}

\begin{exe}
\ex τί μευ μέλαν ἐκ χροὸϲ αἷμα {[}...{]} πέπωκαϲ\\
\gll tí \emph{meu} mélan ek khroòs haîma pépōkas\\
what.\textsc{acc.sg} me.\textsc{gen} black.\textsc{n.acc.sg} out skin.\textsc{gen.sg} blood.\textsc{acc.sg} drink.\textsc{2sg.prf}\\
\trans `Why have you drunk the dark blood from under my skin?' (Theocritus 2.55)
\label{gen19}
\end{exe}

\begin{exe}
\ex φράζεό μευ τὸν ἔρωθ᾽ ὅθεν ἵκετο\\
\gll phrázeó \emph{meu} tòn érōth' hóthen híketo\\
tell.\textsc{2sg.prs.imper.pass} me.\textsc{gen} the.\textsc{m.acc.sg} love.\textsc{m.acc.sg} whence come.\textsc{3sg.aor.mid}\\
\trans `Tell me whence my love has come.' (Theocritus 2.69)
\label{gen20}
\end{exe}

\begin{exe}
\ex τόν μευ τὰν ϲύριγγα πρόαν κλέψαντα Κομάταν\\
\gll tón \emph{meu} tàn súringa próan klépsanta Komátan\\
the.\textsc{m.acc.sg} me.\textsc{gen} the.\textsc{f.acc.sg} pipe.\textsc{acc.sg} lately steal.\textsc{ptcp.aor.m.acc.sg} Comatas.\textsc{acc}\\
\trans `... Comatas, who has just stolen my pipes.' (Theocritus 5.4)
\label{gen21}
\end{exe}

\begin{exe}
\ex οὔ τευ τὰν ϲύριγγα λαθὼν ἔκλεψε Κομάταϲ\\
\gll oú \emph{teu} tàn súringa lathṑn éklepse Komátas\\
not you.\textsc{gen} the.\textsc{f.acc.sg} pipes.\textsc{acc.sg} hide.\textsc{ptcp.aor.m.nom.sg} steal.\textsc{aor.3sg} Comatas.\textsc{nom}\\
\trans `Comatas has not stolen your pipes unnoticed.' (Theocritus 5.19)
\label{gen22}
\end{exe}

\begin{exe}
\ex καλὰ δέ μευ ἁ μία κώρα\\
\gll kalà dé \emph{meu} ha mía kṓra\\
beautiful.\textsc{f.nom.sg} but me.\textsc{gen} the.\textsc{f.nom.sg}
one.\textsc{f.nom.sg} girl.\textsc{nom.sg}\\
\trans `And beautiful is my one girl.' (Theocritus 6.36)
\label{gen23}
\end{exe}

\begin{exe}
\ex τί μευ τὸ χιτώνιον ἄρδειϲ\\
\gll tí \emph{meu} tò khitṓnion árdeis\\
what.\textsc{acc} me.\textsc{gen} the.\textsc{n.acc.sg} frock.\textsc{acc.sg} water.\textsc{2sg.prs}\\
\trans `Why are you wetting my frock?' (Theocritus 15.31)
\label{gen24}
\end{exe}

\hyperlink{p364}{\emph{[p364]}}

\begin{exe}
\ex δίχα μευ τὸ θέριϲτριον ἤδη ἔϲχιϲται\\
\gll díkha \emph{meu} tò théristrion ḗdē éskhistai\\
apart me.\textsc{gen} the.\textsc{n.nom.sg} garment.\textsc{n.nom.sg} already split.\textsc{3sg.prf.pass}\\
\trans `My garment is already torn apart.' (Theocritus 15.69)
\label{gen25}
\end{exe}

\begin{exe}
\ex οἱ δέ ϲφεων κατὰ πρύμναν ἀείραντεϲ μέγα κῦμα\\
\gll hoi dé \emph{spheōn} katà prúmnan aeírantes méga kûma\\
the.\textsc{m.nom.sg} but them.\textsc{gen} down stern.\textsc{acc}
raise.\textsc{ptcp.aor.m.nom.p} great.\textsc{n.acc.sg} billow.\textsc{acc.sg}\\
\trans `And they, raising a great billow along their stern ...' (Theocritus 22.10)
\label{gen26}
\end{exe}

The influence of our positional law can be seen even more decisively in the striking examples in which the preceding pronominal\is{pronouns} genitive is separated from its governing noun by other words. This can be seen in the \textit{toi} of (\ref{gensep9}); compare Meineke's \citeyearpar[256]{Meineke1856} comments.

\begin{exe}
\ex ὥϲ τοι ἐγὼν ἐνόμευον ἀν᾽ ὤρεα τὰϲ καλὰϲ αἶγαϲ φωνᾶϲ εἰϲαΐων\\
\gll hṓs \emph{toi} egṑn enómeuon an' ṓrea tàs kalàs aîgas phōnâs eisaḯōn\\
as you.\textsc{dat} I.\textsc{nom} pasture.\textsc{1sg.imp} on mountain.\textsc{acc.pl} the.\textsc{f.acc.pl} beautiful.\textsc{f.acc.pl} goat.\textsc{acc.pl} sound.\textsc{gen.sg} hear.\textsc{ptcp.prs.m.nom.sg}\\
\trans `... that I might be tending your beautiful goats on the hillside, listening to your voice ...' (Theocritus 7.87)
\label{gensep9}
\end{exe}\il{Greek, Classical|)}

Furthermore, in the examples of the genitive in this category in Homer,\il{Greek, Homeric|(} the genitive is regularly in second position: (\ref{gensep10}), where the position of the pronoun\is{pronouns} is particularly remarkable; (\ref{gensep11})--(\ref{gensep15}). (Only (\ref{gensep16}), in which \textit{meu} stands in third position, constitutes a counterexample, and not a very serious one at that.)

\begin{exe}
\ex ἀλλά ϲευ ἢ κάματοϲ πολυᾶϊξ γυῖα δέδυκεν ἤ νύ ϲέ που δέοϲ ἴϲχει\\
\gll allá \emph{seu} ḕ kámatos poluâïx guîa déduken ḗ nú sé pou déos ískhei\\
but you.\textsc{gen} or weariness.\textsc{nom.sg} much.rushing.\textsc{m.nom.sg} limb.\textsc{acc.pl} enter.\textsc{3sg.prf} or now you.\textsc{acc} somewhere
fear.\textsc{nom.sg} hold.\textsc{3sg.prs}\\
\trans `Yet either weariness born of your many onsets has entered into your limbs, or perhaps terror possesses you.' (Homer, \textit{Iliad} 5.811)
\label{gensep10}
\end{exe}

\begin{exe}
\ex μόγιϲ δέ μευ ἔκφυγεν ὁρμήν\\
\gll mógis dé \emph{meu} ékphugen hormḗn\\
hardly but me.\textsc{gen} escape.\textsc{3sg.aor} onslaught.\textsc{acc.sg}\\
\trans `And hardly did he escape my onset.' (Homer, \textit{Iliad} 9.355)
\label{gensep11}
\end{exe}

\begin{exe}
\ex νῦν δέ ϲευ ὠνοϲάμην πάγχυ φρέναϲ\\
\gll nûn dé \emph{seu} ōnosámēn pánkhu phrénas\\
now but you.\textsc{gen} scorn.\textsc{1sg.aor.mid} wholly midriff.\textsc{acc.pl}\\
\trans `But now have I altogether scorn of your wits.' (Homer, \textit{Iliad} 14.95; cf. also 17.173)
\label{gensep12}
\end{exe}

\begin{exe}
\ex χαίρω ϲευ Λαερτιάδη τὸν μῦθον ἀκούϲαϲ\\
\gll khaírō \emph{seu} Laertiádē tòn mûthon akoúsas\\
rejoice.\textsc{1sg.prs} you.\textsc{gen} Laertes.\textsc{patron.voc.sg} the.\textsc{m.acc.sg} myth.\textsc{acc.sg} hear.\textsc{ptcp.aor.m.nom.sg}\\
\trans `Glad am I, son of Laertes, to hear your words.' (Homer, \textit{Iliad} 19.185)
\label{gensep13}
\end{exe}

\begin{exe}
\ex θεὰ δέ μευ ἔκλυεν αὐδῆϲ\\
\gll theà dé \emph{meu} ékluen audês\\
goddess.\textsc{nom.sg} but me.\textsc{gen} hear.\textsc{3sg.aor} voice.\textsc{gen.sg}\\
\trans `And the goddess heard my voice.' (Homer, \textit{Odyssey} 10.311)
\label{gensep14}
\end{exe}

\begin{exe}
\ex οἵ μευ φθινύθουϲι φίλον κῆρ\\
\gll hoí \emph{meu} phthinúthousi phílon kêr\\
who.\textsc{m.nom.pl} me.\textsc{gen} waste.\textsc{3pl.prs} dear.\textsc{n.acc.sg} heart.\textsc{acc.sg}\\
\trans `... who make my poor heart to pine.' (Homer, \textit{Odyssey} 10.485)
\label{gensep15}
\end{exe}

\begin{exe}
\ex ἦ μάλα μευ καταδάπτετ᾽ ἀκούοντοϲ φίλον ἦτορ\\
\gll ê mála \emph{meu} katadáptet' akoúontos phílon êtor\\ 
in.truth greatly me.\textsc{gen} devour.\textsc{2pl.prs} hear.\textsc{ptcp.prs.m.gen.sg} dear.\textsc{n.acc.sg} heart.\textsc{n.acc.sg}\\
\trans `Truly you rend my poor heart, as I hear your words.' (Homer, \textit{Odyssey} 16.92)
\label{gensep16}
\end{exe}\il{Greek, Homeric|)}

And in the late authors\il{Greek, Classical|(} a pronominal\is{pronouns} genitive separated from its noun also takes second position, if not regularly then at least very frequently: examples (\ref{kaimou1}), (\ref{kaimou6})--(\ref{kaimou7}), (\ref{kaispheon}) and (\ref{imper1}) above, and (\ref{gensep17})--(\ref{gensep27}) below (cf. also Menander Fragment 498).

\begin{exe}
\ex πρίν ϲου κατὰ πάντα δαῆναι ἤθεα\\
\gll prín \emph{sou} katà pánta daênai ḗthea\\
before you.\textsc{gen} down all.\textsc{n.acc.pl} learn.\textsc{aor.inf} custom.\textsc{acc.pl}\\
\trans `... before learning in accordance with all your customs.' (Theognis, \textit{Elegies} 969)
\label{gensep17}
\end{exe}

\begin{exe}
\ex μή μου κατείπῃϲ ϲῷ καϲιγνήτῳ πόϲιν\\
\gll mḗ \emph{mou} kateípēis sôi kasignḗtōi pósin\\
not me.\textsc{gen} denounce.\textsc{2sg.aor.sbjv} your.\textsc{m.dat.sg} brother.\textsc{dat.sg} husband.\textsc{acc}\\
\trans `Do not tell your brother that my husband ...' (Euripides, \textit{Helen} 898)
\label{gensep18}
\end{exe}

\begin{exe}
\ex οὐδέ ϲου ϲυνῆψε χείρα\\
\gll oudé \emph{sou} sunêpse kheíra\\
nor you.\textsc{gen} bind.\textsc{3sg.aor} hand.\textsc{acc.sg}\\
\trans `But did he not tie your hand?' (Euripides, \textit{Bacchae} 615)\footnote{\emph{Translator's note}: The Perseus edition has \textit{kheîre} for \textit{kheíra}.}
\label{gensep19}
\end{exe}

\begin{exe}
\ex ἐμπλήϲθητί μου πιὼν κελαινὸν αἷμα\\
\gll emplḗsthētí \emph{mou} piṑn kelainòn haîma\\
fill.up.\textsc{2sg.aor.imper.pass} me.\textsc{gen} drink.\textsc{ptcp.aor.m.nom.sg} dark.\textsc{n.acc.sg} blood.\textsc{acc.sg}\\
\trans `Have your fill drinking my dark blood.' (Euripides, Fragment 687.1)
\label{gensep20}
\end{exe}

\begin{exe}
\ex οἴμοι, δράκων μου γίγνεται τὸ ἥμιϲυ\\
\gll oímoi, drákōn \emph{mou} gígnetai tò hḗmisu\\
ah.me dragon.\textsc{nom.sg} me.\textsc{gen} become.\textsc{3sg.prs.pass} the.\textsc{n.nom.sg} half.\textsc{nom.sg}\\
\trans `Woe is me; half of me is becoming a dragon.' (Euripides, Fragment 930)
\label{gensep21}
\end{exe}

\begin{exe}
\ex ἐξαρπάϲομαί ϲου τοῖϲ ὄνυξι τἄντερα\\
\gll exarpásomaí \emph{sou} toîs ónuxi tántera\\
tear.out.\textsc{1sg.fut.mid} you.\textsc{gen} the.\textsc{m.dat.pl} nail.\textsc{dat.pl} the=gut.\textsc{acc.pl}\\
\trans `I will tear out your guts with my nails.' (Aristophanes, \textit{Knights} 708)
\label{gensep22}
\end{exe}

\begin{exe}
\ex εἴθε ϲου εἶναι ὤφελεν, ὦ λαζών, οὑτωϲὶ θερμὸϲ ὁ πλευμων\\
\gll eíthe \emph{sou} eînai ṓphelen, ô lazṓn, houtōsì thermòs ho pleumōn\\
if.only you.\textsc{gen} be.\textsc{prs.inf} owe.\textsc{3sg.aor} O kick.\textsc{ptcp.prs.m.nom.sg} so hot.\textsc{m.nom.sg} the.\textsc{m.nom.sg} lung.\textsc{nom.sg}\\
\trans `You trouble-maker, if only your lungs could get this hot.' (Aristophanes, \textit{Peace} 1068)\footnote{\emph{Translator's note}: The Perseus edition has \textit{ôlazṑn}, with crasis.}
\label{gensep23}
\end{exe}

\begin{exe}
\ex οἷϲ μου κατέφαγεϲ τὰ φορτία\\
\gll hoîs \emph{mou} katéphages tà phortía\\
which.\textsc{m.dat.pl} me.\textsc{gen} eat.up.\textsc{2sg.aor} the.\textsc{n.acc.pl} ware.\textsc{acc.pl}\\
\trans `... with which you ate up my wares.' (Aristophanes, \textit{Frogs} 573)
\label{gensep24}
\end{exe}

\begin{exe}
\ex ἕωϲ ἄν ϲου βάροϲ ἐν τοῖϲ ϲκέλεϲι γένηται\\
\gll héōs án \emph{sou} báros en toîs skélesi génētai\\
until \textsc{irr} you.\textsc{gen} weight.\textsc{nom.sg} in the.\textsc{n.dat.pl} leg.\textsc{dat.pl} become.\textsc{3sg.aor.sbjv.mid}\\
\trans `... until your legs feel heavy.' (Plato, \textit{Phaedo} 117a)
\label{gensep25}
\end{exe}

\begin{exe}
\ex ὥϲ μευ περὶ θυμὸϲ ἰάφθη\\
\gll hṓs \emph{meu} perì thumòs iáphthē\\
so me.\textsc{gen} around spirit.\textsc{nom.sg} wound.\textsc{3sg.aor.pass}\\
\trans `So all my heart was fired.' (Theocritus 2.82)
\label{gensep26}
\end{exe}

\begin{exe}
\ex εἴ μευ καλὰ πέλει τὰ μελύδρια\\
\gll eí \emph{meu} kalà pélei tà melúdria\\
if me.\textsc{gen} beautiful.\textsc{n.nom.pl} become.\textsc{3sg.prs}
the.\textsc{n.nom.pl} song.\textsc{dim.nom.pl}\\
\trans `If my little songs prove beautiful ...' (Bion 7.2)\footnote{\emph{Translator's note}: The Perseus edition has \textit{moi} for \textit{meu}.}
\label{gensep27}
\end{exe}

We have seen something very similar with the genitive \textit{hoi} (see above p\pageref{oi}f). And just as this word can occur in the middle of the governing phrase, i.e. after the first word, so can the forms to be discussed here. For instance: a) Following a particle,\is{particles} (\ref{part1})--(\ref{part7}). \hyperlink{p365}{\emph{[p365]}}

\begin{exe}
\ex οἱ δέ μευ πάντεϲ ὀδόντεϲ ἐντὸϲ ἐν γνάθοιϲ κεκινέαται\\
\gll hoi dé \emph{meu} pántes odóntes entòs en gnáthois kekinéatai\\
the.\textsc{m.nom.pl} but me.\textsc{gen} all.\textsc{m.nom.pl}
tooth.\textsc{nom.pl} inside in jaw.\textsc{dat.pl} move.\textsc{3pl.prf.pass}\\
\trans `And all the teeth inside my jaw have been moved.' (Hipponax, Fragment 62)
\label{part1}
\end{exe}

\begin{exe}
\ex αἱ δέ μευ φρένεϲ ἐκκεκωφέαται\\
\gll hai dé \emph{meu} phrénes ekkekōphéatai\\
the.\textsc{f.nom.pl} but me.\textsc{gen} midriff.\textsc{nom.pl}
deafen.\textsc{3pl.prf.pass}\\
\trans `And my wits have been dulled.' (Anacreon, Fragment 81)
\label{part2}
\end{exe}

\begin{exe}
\ex αἱ γάρ ϲφι κάμηλοι ἵππων οὐκ ἔϲϲονέϲ {[}...{]} εἰϲιν\\
\gll hai gár \emph{sphi} kámēloi híppōn ouk éssonés eisin\\
the.\textsc{f.nom.pl} for them.\textsc{dat} camel.\textsc{nom.pl}
horse.\textsc{gen.pl} not inferior.\textsc{f.nom.pl} be.\textsc{3pl.prs}\\
\trans `For their camels are not inferior to horses.' (Herodotus 3.102.3)
\label{part3}
\end{exe}

\begin{exe}
\ex τῶν δέ ϲφι γυναικῶν τοὺϲ μαζοὺϲ ἀποταμοῦϲα\\
\gll tôn dé \emph{sphi} gunaikôn toùs mazoùs apotamoûsa\\
the.\textsc{f.gen.pl} but them.\textsc{dat} woman.\textsc{gen.pl}
the.\textsc{m.acc.pl} breast.\textsc{acc.pl} cut.off.\textsc{ptcp.aor.f.nom.sg}\\
\trans `And, cutting off the breasts of their women ...' (Herodotus 4.202.1)
\label{part4}
\end{exe}

\begin{exe}
\ex οἵ τέ ϲφεων ὀπέωνεϲ {[}...{]} ἀπεκεκληίατο\\
\gll hoí té \emph{spheōn} opéōnes apekeklēíato\\
the.\textsc{m.nom.pl} and them.\textsc{gen} follower.\textsc{nom.pl}
exclude.\textsc{3pl.plup.pass}\\
\trans `And their followers had been cut off.' (Herodotus 9.50.1)
\label{part5}
\end{exe}

\begin{exe}
\ex τοῦτό γέ τοί ϲου τοὖργον ἀληθῶϲ γενναῖον καὶ φιλόδημον\\
\gll toûtó gé toí \emph{sou} toûrgon alēthôs gennaîon kaì philódēmon\\
this.\textsc{n.nom.sg} even lo you.\textsc{gen} the=work.\textsc{nom.sg} truly noble.\textsc{n.nom.sg} and popular.\textsc{n.nom.sg}\\
\trans `Well, this deed of yours at least is truly noble and democratic.' (Aristophanes, \textit{Knights} 787)
\label{part6}
\end{exe}

\begin{exe}
\ex ταὶ δέ μοι αἶγεϲ βόϲκονται κατ᾽ ὄροϲ\\
\gll taì dé \emph{moi} aîges bóskontai kat' óros\\
the.\textsc{f.nom.pl} but me.\textsc{dat} goat.\textsc{nom.pl}
feed.\textsc{3pl.pres.pass} down mountain.\textsc{acc.sg}\\
\trans `And my goats are grazing on the hillside.' (Theocritus 3.1)
\label{part7}
\end{exe}

(See also examples (\ref{Or482}), (\ref{attsep17}) and (\ref{Eccles913}) with \textit{moi}, presented above on p\pageref{Or482} and p\pageref{Eccles913}.)

b) Immediately following an article or preposition:\is{prepositions} (\ref{artprep1}). Identical is the Cy\-priot\il{Greek, Cypriot} example (\ref{artprep2}), which \citet[323]{Hoffmann1891} describes as ``very peculiar'', while \citet[139--140]{Meister1889} even felt compelled to construct a new word \textit{homoíposis} ``fellow husband''.\Footnote{2}{At the request of Dr. Meister I should comment here that because of Wilamowitz's \citeyearpar{Wilamowitz1889}
observation on Euripides, \textit{Heracleidae} 626 (example (\ref{EurHer626}); see above p\pageref{wilamowitz}) he became aware of the right reading of these words some time ago and intended to publicly withdraw his earlier explanation.}

\begin{exe}
\ex ϲὺ δέ, ὦ βαϲιλεῦ, ἐμὲ ἐϲ τόδε ἡλικίηϲ ἥκοντα οἰκτίραϲ, τῶν μοι παίδων παράλυϲον ἕνα τῆϲ ϲτρατιῆϲ\\
\gll sù dé, ô basileû, emè es tóde hēlikíēs hḗkonta oiktíras, tôn \emph{moi} paídōn paráluson héna tês stratiês\\
you.\textsc{nom} then O king.\textsc{voc} me.\textsc{acc} into this.\textsc{n.acc.sg} age.\textsc{gen.sg} arrive.\textsc{ptcp.prs.m.acc.sg} pity.\textsc{ptcp.aor.m.nom.sg} the.\textsc{m.gen.pl} me.\textsc{dat} child.\textsc{gen.pl} release.\textsc{2sg.aor.imper} one.\textsc{m.acc.sg}
the.\textsc{f.gen.sg} service.\textsc{gen.sg}\\
\trans `And you, O king, pitying me in my advanced age, release one of my sons from service.' (Herodotus 7.38.3)
\label{artprep1}
\end{exe}

\begin{exe}
\ex ὄ μοι πόϲιϲ Ὀναϲίτιμοϲ\\
\gll ó \emph{moi} pósis Onasítimos\\
the.\textsc{m.nom.sg} me.\textsc{dat} husband.\textsc{nom.sg} Onasitimos\\
\trans `My husband (is) Onasitimus.' \citep[no. 26]{Deecke1884}
\label{artprep2}
\end{exe}

In addition, from the Attic\il{Greek, Attic} poets\is{poetry} we have examples (\ref{artprep3})--(\ref{artprep6}). Cf. also (\ref{artprep7}). Other than at the beginning of the clause, however, \textit{mou} etc. are inserted in this way only extremely rarely. In the examples where it happens, such as (\ref{Ran485}), we can assume that the insertion that occurred at the start of the clause was mirrored later in the clause.

\begin{exe}
\ex διά μου κεφαλᾶϲ φλὸξ οὐρανία βαίη\\
\gll diá \emph{mou} kephalâs phlòx ouranía baíē\\
through me.\textsc{gen} head.\textsc{gen.sg} flame.\textsc{nom.sg}
heavenly.\textsc{f.nom.sg} pass.\textsc{3sg.aor.opt}\\
\trans `May fire from heaven strike through my head.' (Euripides, \textit{Medea} 144)
\label{artprep3}
\end{exe}

\begin{exe}
\ex διά μου κεφαλᾶϲ ᾄϲϲουϲ᾽ ὀδύναι\\
\gll diá mou kephalâs ā́issous' odúnai\\
through me.\textsc{gen} head.\textsc{gen.sg} dart.\textsc{3pl.prs} pain.\textsc{nom.pl}\\
\trans `Pains dart through my head.' (Euripides, \textit{Hippolytus} 1351)
\label{artprep4}
\end{exe}

\begin{exe}
\ex εἷϲ μου λόγοϲ ϲοι πάντα ϲημανεῖ τάδε\\
\gll heîs \emph{mou} lógos soi pánta sēmaneî táde\\
one.\textsc{m.nom.sg} me.\textsc{gen} account.\textsc{nom.sg} you.\textsc{dat} all.\textsc{n.acc.pl} signify.\textsc{3sg.fu} this.\textsc{n.acc.pl}\\
\trans `One account from me will tell you all these things.' (Euripides, \textit{Heracleidae} 799)
\label{artprep5}
\end{exe}

\begin{exe}
\ex ὦ ϲκυτοτόμε, τῆϲ μου γυναικὸϲ τοὺϲ πόδαϲ\\
\gll ô skutotóme, tês \emph{mou} gunaikòs toùs pódas\\
O cobbler.\textsc{voc} the.\textsc{f.gen.sg} me.\textsc{gen} woman.\textsc{gen.sg} the.\textsc{m.acc.pl} foot.\textsc{acc.pl}\\
\trans `Cobbler, my wife's feet ...' (Aristophanes, \textit{Lysistrata} 416)\footnote{\emph{Translator's note}: The Perseus edition has \textit{toû podòs} for \textit{toùs pódas}.}
\label{artprep6}
\end{exe}

\begin{exe}
\ex τό μευ νάκοϲ ἐχθὲϲ ἔκλεψεν\\
\gll tó \emph{meu} nákos ekhthès éklepsen\\
the.\textsc{n.acc.sg} me.\textsc{gen} fleece.\textsc{acc} yesterday steal.\textsc{3sg.aor}\\
\trans `He stole my fleece yesterday.' (Theocritus 5.2)
\label{artprep7}
\end{exe}

\begin{exe}
\ex δείϲαϲα γὰρ εἰϲ τὴν κάτω μου κοιλίαν καθείρπυϲεν\\
\gll deísasa gàr eis tḕn kátō \emph{mou} koilían katheírpusen\\
fear.\textsc{ptcp.aor.f.nom.sg} then into the.\textsc{f.acc.sg}
downwards me.\textsc{gen} belly.\textsc{acc.sg} creep.\textsc{3sg.aor}\\
\trans `For, taking fright, it crept down into my bowels.' (Aristophanes, \textit{Frogs} 485)
\label{Ran485}
\end{exe}\il{Greek, Classical|)}

I will not here investigate the position of the barytonic, hence originally enclitic,\is{enclitics} plural forms \textit{hḗmōn}, \textit{hêmin} (\textsc{1pl}) etc., due to the difficulty of distinguishing them from genuinely orthotonic forms (but see example (\ref{IGA486}), just like \textit{m' anéthēken} `me.\textsc{acc} dedicate' otherwise, and \hyperlink{p366}{\emph{[p366]}} (\ref{IGA482a5})); however, I would like to reiterate that, according to the evidence provided by \citet{Krueger1871}, to whose incisive categorization we owe the finer laws for the positioning of these genitives, \textit{autoû}, \textit{autês}, \textit{autôn} `self/same' with anaphoric meaning follow the same positional rules as \textit{mou}.

\begin{exe}
\ex {[}Ἑρ{]}μηϲιάναξ ἥμεαϲ ἀνέθηκεν {[}ὁ ...{]}\\
\gll {[}Her{]}mēsiánax \emph{hḗmeas} anéthēken {[}ho~...{]}\\
Hermesianax us.\textsc{acc} dedicate.\textsc{3sg.aor} the.\textsc{m.nom.sg}\\
\trans `Hermesianax the ... dedicated us.' (IGA 486, Miletus)
\label{IGA486}
\end{exe}

\begin{exe}
\ex ἔγραφε δ᾽ἇμε Ἄρχων Ἁμοιβίχου\\
\gll égraphe d' \emph{hâme} Árkhōn Hamoibíkhou\\
write.\textsc{3sg.imp} then us.\textsc{acc} ruler.\textsc{nom.sg} Hamoibikhos.\textsc{gen}\\
\trans `And the governor of Hamoibikhos inscribed us.' (IGA 482a.5, Elephantine)
\label{IGA482a5}
\end{exe}

It is true that this does not hold for Homer,\il{Greek, Homeric|(} for whom the anaphoric meaning and the loss of tone on \textit{autoû} are in their early stages, and who therefore also places it far from the start of the sentence even in places where we would render it with \emph{eius} (\textsc{3sg.gen}), as in (\ref{autos1}), (\ref{autos2}) (in \textit{Odyssey} 7.263, on the other hand, the same expression contains emphasis on \textit{autês}), and (\ref{autos3}), which provides very valuable indirect evidence for our positional rule. However,\il{Greek, Classical|(} the Attic\il{Greek, Attic} poets\is{poetry} place \textit{autoû}, \textit{autês}, \textit{autôn} before their governing nouns just as freely as \textit{mou}, and then, just like \textit{mou}, it is often near to the start of the clause, e.g. (\ref{autos4}), (\ref{autos5}), and (\ref{autos6}). Similarly, \textit{autoû}, like \textit{mou}, is also found preceding its noun such that the two are separated by one or more words, and in this case, like \textit{mou}, it freely occurs in second position, e.g. (\ref{autos7}). 

\begin{exe}
\ex ἄνυϲιϲ δ᾽ οὐκ ἔϲϲεται αὐτῶν\\
\gll ánusis d' ouk éssetai \emph{autôn}\\
accomplishment.\textsc{nom.sg} then not be.\textsc{3sg.fut.mid} them.\textsc{gen}\\
\trans `Yet no accomplishment shall come therefrom.' (Homer, \textit{Iliad} 2.347)
\label{autos1}
\end{exe}

\begin{exe}
\ex δὴ γὰρ νόοϲ ἐτράπετ᾽ αὐτοῦ\\
\gll dḕ gàr nóos etrápet' \emph{autoû}\\
exactly then mind.\textsc{nom.sg} turn.\textsc{3sg.aor.mid} him.\textsc{gen}\\
\trans `For lo, his mind was turned.' (Homer, \textit{Iliad} 17.546)
\label{autos2}
\end{exe}

\begin{exe}
\ex γόνοϲ δ᾽ οὐ γίγνεται αὐτῶν\\
\gll gónos d' ou gígnetai \emph{autôn}\\
offspring.\textsc{nom.sg} then not become.\textsc{3sg.prs} them.\textsc{gen}\\
\trans `And these bear no young.' (Homer, \textit{Odyssey} 12.130)
\label{autos3}
\end{exe}\il{Greek, Homeric|)}

\begin{exe}
\ex ἐθαύμαϲέ τε αὐτοῦ τὴν διάνοιαν\\
\gll ethaúmasé te \emph{autoû} tḕn diánoian\\
wonder.\textsc{3sg.aor} and him.\textsc{gen} the.\textsc{f.acc.sg} intention.\textsc{acc.sg}\\
\trans `And he approved his intention.' (Thucydides 1.138.1)
\label{autos4}
\end{exe}

\begin{exe}
\ex καὶ αὐτῶν τὴν χώραν ἐμμείναϲ τῷ ϲτρατῷ ἐδῄου\\
\gll kaì \emph{autôn} tḕn khṓran emmeínas tôi stratôi edḗiou\\
and them.\textsc{gen} the.\textsc{f.acc.sg} land.\textsc{acc.sg} abide.\textsc{ptcp.aor.m.nom.sg} the.\textsc{m.dat.sg} army.\textsc{acc.sg} ravage.\textsc{3sg.imp}\\
\trans `And standing firm, he ravaged their land with his army.' (Thucydides 4.109.5)
\label{autos5}
\end{exe}

\begin{exe}
\ex ἐγκωμιάζειϲ μὲν αὐτοῦ τὴν τέχνην\\
\gll enkōmiázeis mèn \emph{autoû} tḕn tékhnēn\\
extol.\textsc{2sg.prs} then him.\textsc{gen} the.\textsc{f.acc.sg} craft.\textsc{acc.sg}\\
\trans `You simply extol his art.' (Plato, \textit{Gorgias} 448e)
\label{autos6}
\end{exe}

\begin{exe}
\ex ἐπεὶ γὰρ αὐτῶν γῆϲ ἀπηλλάχθη πατήρ\\
\gll epeì gàr \emph{autôn} gês apēllákhthē patḗr\\
when then them.\textsc{gen} earth.\textsc{gen.sg} deliver.\textsc{3sg.aor.pass} father.\textsc{nom.sg}\\
\trans `For when their father was released from this world ...' (Euripides, \textit{Heracleidae} 12)
\label{autos7}
\end{exe}

Finally, anyone who looks at the Herodotan examples adduced by \citet[142]{Stein1866} on 6.30.1, in which \textit{autoû} stands between the article and the noun, will find \textit{autoû} in second position in all of them (and also in 1.146.2, 1.177.1, 2.149.19, and 7.129.1), including (\ref{autos8}) -- just as with intervening \textit{moi} and \textit{mou}. The Attic\il{Greek, Attic} poets\is{poetry} are freer: (\ref{autos9})--(\ref{autos10}). Perhaps it is relevant for the \textit{autoû} in Isocrates, as for the \textit{mou} in example (\ref{Ran485}) above (see p\pageref{Ran485}), that the genitive is attached not to the article but to an adjective.\is{adjectives}

\begin{exe}
\ex Μεγαρέαϲ τε τοὺϲ ἐν Σικελίῃ, ὡϲ {[}...{]} προϲεχώρηϲαν, τοὺϲ μὲν αὐτῶν παχέαϲ {[}...{]} πολιήταϲ ἐποίηϲε\\
\gll Megaréas te toùs en Sikelíēi, hōs prosekhṓrēsan, toùs mèn \emph{autôn} pakhéas poliḗtas epoíēse\\
Megarian.\textsc{m.acc.pl} and the.\textsc{m.acc.pl} in Sicily.\textsc{dat} as surrender.\textsc{3pl.aor} the.\textsc{m.acc.pl} then them.\textsc{gen} thick.\textsc{m.acc.pl} citizen.\textsc{acc.pl} make.\textsc{3sg.aor}\\
\trans `When they surrendered he made the Megarians in Sicily, the wealthiest of them, citizens.' (Herodotus 7.156.2)
\label{autos8}
\end{exe}

\begin{exe}
\ex γνώϲεϲθε τὴν ἄλλην αὐτοῦ πονηρίαν\\
\gll gnṓsesthe tḕn állēn \emph{autoû} ponērían\\
know.\textsc{2pl.fut.mid} the.\textsc{f.acc.sg} other.\textsc{f.acc.sg} him.\textsc{gen} wickedness.\textsc{acc.sg}\\
\trans `You will recognize his other wickedness.' (Isocrates 18.52)\footnote{\emph{Translator's note}: The Perseus edition places \textit{gnṓsesthe} after \textit{ponērían}.}
\label{autos9}
\end{exe}

\begin{exe}
\ex ὅπωϲ {[}...{]} αὐτοὶ καὶ οἱ αὐτῶν ϲτρατιῶται ἐκπλεύϲειαν\\
\gll hópōs autoì kaì hoi \emph{autôn} stratiôtai ekpleúseian\\
so they.\textsc{nom} and the.\textsc{m.nom.pl} them.\textsc{gen} soldier.\textsc{nom.pl} sail.away.\textsc{3pl.aor.opt}\\
\trans `... so that they and their soldiers could sail away ...' (Xenophon, \textit{Anabasis} 6.2.14)
\label{autos10}
\end{exe}\il{Greek, Classical|)}\is{genitive|)}


\section{Indefinite pronouns and other enclitics}\is{pronouns|(}\is{indefinites|(}

\citet[177--178]{Bergaigne1877} assumes that the positional law for enclitic\is{enclitics} personal\is{personal pronouns} pronouns laid out in sections \ref{enclitic-archaic}--\ref{genitives} originated with the \hyperlink{p367}{\emph{[p367]}} anaphoric pronouns; there was a desire to place these as close as possible to the preceding clause in order to better mark the connection between them. From the anaphoric pronouns, so the account goes, this positional rule was then extended to the pronouns of the first and second person, and, because they were placed after and attached to the first word of the clause, the affected pronouns became enclitic.\is{enclitics}

This assumption has little to recommend it, since precisely the factor that favoured the position after the beginning of the clause for \textit{hoi} (\textsc{3sg.dat}) and \textit{sphin} `them.\textsc{dat}' according to Bergaigne\ia{Bergaigne, Abel} -- the connection to the preceding clause -- is absent for \textit{moi} `me.\textsc{dat}' and \textit{mou} `me.\textsc{gen}'. On the other hand, the possibility, rejected by Bergaigne,\ia{Bergaigne, Abel} that ``the language became accustomed to place them after the first word because they were devoid of accent'' is shown to be correct by the fact that \isi{enclitics} other than personal\is{personal pronouns} pronouns were also subject to this positional rule. \citet[268, note 8]{Kuehner1869} has already observed that ``with the free word order of the Greek language it is no wonder that \isi{enclitics} are often attached not to the word to which they belong but to another to which they do not belong''. \citeauthor{Kuehner1869} does not discuss the direction in which these deviations go, but many examples that he presents in that section can be resolved by our positional rule.

Among the declinable \isi{enclitics}, the indefinite pronoun should now be considered. It is very clear that the positional rule did not apply to this pronoun: if it seems significant that the archaic\il{Greek, Archaic} forms \textit{tou} `someone.\textsc{gen}' and \textit{tōi} `someone.\textsc{dat}', with the exception of (\ref{CIA461a15}), occur only immediately following \textit{ei} or \textit{eán} `if' (cf. the examples in \citealp[123, note 1106]{Meisterhans1888}), it is sufficient to point to Thucydides, who shows these forms in all kinds of positions in the clause.

\begin{exe}
\ex {[}...{]} ἔχοντόϲ του\\
\gll ékhontós \emph{tou}\\
have.\textsc{ptcp.prs.m/n.gen.sg} someone.\textsc{gen}\\
\trans (CIA. 4.61a.15)
\label{CIA461a15}
\end{exe}


Nevertheless, the tendency in Homer\il{Greek, Homeric|(} to place \textit{tis} initially is unmistakable. Other than \textit{hóstis} and related forms, one should look at \textit{ei tis} and \textit{mḗ tis}, particularly the following examples:\label{greekseparation} in separation from the governing noun, (\ref{tis1})--(\ref{tis6}).

\begin{exe}
\ex εἰ δέ τευ ἐξ ἄλλου γε θεῶν\\
\gll ei dé \emph{teu} ex állou ge theôn\\
if but some.\textsc{m.gen.sg} out other.\textsc{m.gen.sg} even god.\textsc{gen.pl}\\
\trans `But (were you born) of any other god ...' (Homer, \textit{Iliad} 5.897)
\label{tis1}
\end{exe}

\begin{exe}
\ex ἵνα τιϲ ϲτυγέῃϲι καὶ ἄλλοϲ\\
\gll hína \emph{tis} stugéēisi kaì állos\\
that some.\textsc{m.nom.sg} hate.\textsc{3sg.prs.sbjv} also other.\textsc{m.nom.sg}\\
\trans `... that so others may dread to ...' (Homer, \textit{Iliad} 8.515)
\label{tis2}
\end{exe}

\hyperlink{p368}{\emph{[p368]}}

\begin{exe}
\ex εἴ πέρ τί ϲε κῆδοϲ ἱκάνει\\
\gll eí pér \emph{tí} se kêdos hikánei\\
if all some.\textsc{n.nom.sg} you.\textsc{acc} grief.\textsc{n.nom.sg}
come.\textsc{3sg.prs}\\
\trans `... if in any wise grief for your kin cometh upon thee.' (Homer, \textit{Iliad} 13.464; also preceding the enclitic\is{enclitics} \textit{se}!)
\label{tis3}
\end{exe}

\begin{exe}
\ex ἤ τευ ϲῆμα βροτοῖο πάλαι κατατεθνηῶτοϲ\\
\gll ḗ \emph{teu} sêma brotoîo pálai katatethnēôtos\\
or some.\textsc{m.gen.sg} sign.\textsc{nom.sg} mortal.\textsc{gen.sg} long.ago die.\textsc{ptcp.prf.m.gen.sg}\\
\trans `Haply (it is) a monument of some man long ago dead' (Homer, \textit{Iliad} 23.331)
\label{tis4}
\end{exe}

\largerpage[2]
\begin{exe}
\ex {[}...{]} ὡϲ ὑμεῖϲ παρ᾽ ἐμεῖο θοὴν ἐπὶ νῆα κίοιτε ὥϲ τέ τευ ἢ παρὰ πάμπαν ἀνείμονοϲ ἠὲ πενιχροῦ\\
\gll hōs humeîs par' emeîo thoḕn epì nêa kíoite hṓs té \emph{teu} ḕ parà pámpan aneímonos ēè penikhroû\\
that you.\textsc{nom.pl} from me.\textsc{gen} swift.\textsc{f.acc.sg} upon ship.\textsc{acc.sg} go.\textsc{2pl.prs.opt} as and someone.\textsc{m.gen.sg} or from altogether unclad.\textsc{m.gen.sg} or poor.\textsc{m.gen.sg}\\
\trans `... that you should go from my house to your swift ship as from one utterly without raiment or poor.' (Homer, \textit{Odyssey} 3.347)\footnote{\emph{Translator's note}: The Perseus edition
has \textit{ê ... ēdè}.}
\label{tis5}
\end{exe}
\clearpage

\begin{exe}
\ex μηδέ τι μεϲϲηγύϲ γε κακὸν καὶ πῆμα πάθῃϲιν\\
\gll mēdé \emph{ti} messēgús ge kakòn kaì pêma páthēisin\\
nor some.\textsc{n.acc.sg} meanwhile even ill.\textsc{n.acc.sg} and harm.\textsc{acc.sg} suffer.\textsc{3sg.aor.sbjv}\\
\trans `Nor shall he meanwhile suffer any evil or harm' (Homer, \textit{Odyssey} 7.195)
\label{tis6}
\end{exe}

With \textit{tis} placed before a word that would otherwise be entitled to second position (cf. (\ref{tis3})): (\ref{tis7})--(\ref{tis8}) (cf. (\ref{tis9}), \citealp[559]{Peppmueller1890}). Here belongs the not infrequent \textit{hṓs tís te} instead of \textit{hṓste tis} as in, for instance, (\ref{tis10}).

\begin{exe}
\ex καί τινά τοι παρ Ζηνὸϲ ἐπέφραδε πότνια μήτηρ\\
\gll kaí \emph{tiná} toi par Zēnòs epéphrade pótnia mḗtēr\\
and some.\textsc{f.acc.sg} you.\textsc{dat} from Zeus.\textsc{gen} tell.\textsc{3sg.aor} mistress.\textsc{nom.sg} mother.\textsc{nom.sg}\\
\trans `and (if) your queenly mother has declared anything to you from Zeus ...' (Homer, \textit{Iliad} 16.37)
\label{tis7}
\end{exe}

\begin{exe}
\ex ὅτε τίϲ κε θάνῃϲι\\
\gll hóte \emph{tís} ke thánēisi\\
when someone.\textsc{m.nom.sg} \textsc{irr} die.\textsc{3sg.aor.sbjv}\\
\trans `... whenever someone dies.' (Homer, \textit{Odyssey} 11.218)
\label{tis8}
\end{exe}

\begin{exe}
\ex εἰ γάρ τίϲ κ᾽ ἐθέλῃ\\
\gll ei gár \emph{tís} k' ethélēi\\
if for someone.\textsc{m.som.sg} \textsc{irr} want.\textsc{3sg.prs.sbjv}\\
\trans `For whoever wishes to ...' (Hesiod, \textit{Works and Days} 280)
\label{tis9}
\end{exe}

\begin{exe}
\ex βῆ δ᾽ ἰέναι ὥϲ τίϲ τε λέων ἀπὸ μεϲϲαύλοιο\\
\gll bê d' iénai hṓs \emph{tís} te léōn apò messaúloio\\
pass.\textsc{3sg.aor} then go.\textsc{prs.inf} as some.\textsc{m.nom.sg} and lion.\textsc{nom.sg} of court.\textsc{gen.sg}\\
\trans `...but (he) went his way as a lion from a steading' (Homer, \textit{Iliad} 17.657)
\label{tis10}
\end{exe}\il{Greek, Homeric|)}

Examples in the first category can also be adduced from the later\il{Greek, Classical|(} period \citep[572, note 6]{Kuehner1870}: (\ref{tis11})--(\ref{tis34}); in addition, (\ref{tis35}), in which the attachment of \textit{tis} to the \isi{vocative} is also noteworthy, cf. the comments above p\pageref{patrokle} on example (\ref{patrokle}).

\begin{exe}
\ex οὐδέ τιϲ ἡμῖν αἴτιοϲ ἀθανάτων\\
\gll oudé \emph{tis} hēmîn aítios athanátōn\\
nor someone.\textsc{m.nom.sg} us.\textsc{dat} guilty.\textsc{m.nom.sg}
immortal.\textsc{gen.pl}\\
\trans `Nor is one of the immortals guilty towards us.' (Theognis, \textit{Elegies} 833)
\label{tis11}
\end{exe}

\begin{exe}
\ex εἴ τι παθὼν ἀπ᾽ ἐμεῦ ἀγαθὸν μέγα μὴ χάριν οἶδαϲ\\
\gll eí \emph{ti} pathṑn ap' emeû agathòn méga mḕ khárin oîdas\\
if something.\textsc{acc.sg} suffer.\textsc{ptcp.aor.m.nom.}sg of me.\textsc{gen} good.\textsc{n.acc.sg} great.\textsc{n.acc.sg} not grace.\textsc{acc} know.\textsc{2sg.prf}\\
\trans `If, receiving some great good from me, you know no gratitude ...' (Theognis, \textit{Elegies} 957)
\label{tis12}
\end{exe}

\begin{exe}
\ex ἀλλά τί μοι ζῶντι γένοιτ᾽ ἀγαθόν\\
\gll allá \emph{tí} moi zônti génoit' agathón\\
but something.\textsc{nom.sg} me.\textsc{dat} live.\textsc{ptcp.prs.m.dat.sg} become.\textsc{3sg.aor.opt.mid} good.\textsc{n.nom.sg}\\
\trans `Rather, let some good come to me while I live.' (Theognis, \textit{Elegies} 1192)
\label{tis13}
\end{exe}

\begin{exe}
\ex οὐδέ τιϲ ἀντ᾽ ἀγαθῶν ἐϲτι χάριϲ παρὰ ϲοί\\
\gll oudé \emph{tis} ant' agathôn esti kháris parà soí\\
nor some.\textsc{f.nom.sg} against good.\textsc{n.gen.pl} be.\textsc{3sg.prs} grace.\textsc{nom.sg} from you.\textsc{dat}\\
\trans `Yet there is not any gratitude from you for good things.' (Theognis, \textit{Elegies} 1264)
\label{tis14}
\end{exe}

\begin{exe}
\ex οὔπω τιϲ Ἀκταίων᾽ ἄθηροϲ ἡμέρα {[}...{]} ἔπεμψεν ἐϲ δόμουϲ\\
\gll oúpō \emph{tis} Aktaíōn' áthēros hēméra épempsen es dómous\\
not.yet some.\textsc{f.nom.sg} Actaeon.\textsc{acc} beastless.\textsc{f.nom.sg} day.\textsc{nom.sg} send.\textsc{3sg.aor} into house.\textsc{acc.pl}\\
\trans `No day has yet sent Actaeon home without game.' (Aeschylus, Fragment 241)
\label{tis15}
\end{exe}

\begin{exe}
\ex οὐ γάρ τινα ἔγωγε οἶδα ποταμὸν Ὠκεανὸν ἐόντα\\
\gll ou gár \emph{tina} égōge oîda potamòn Ōkeanòn eónta\\
not for some.\textsc{m.acc.sg} I.\textsc{nom.emph} know.\textsc{1sg.prf} river.\textsc{acc.sg} Ocean.\textsc{acc.sg} be.\textsc{ptcp.prs.m.acc.sg}\\
\trans `For I know of no Ocean river.' (Herodotus 2.23.1)
\label{tis16}
\end{exe}

\begin{exe}
\ex αἰεί τι προϲδοκῶν ἀπ᾽ αὐτῆϲ τοιοῦτο ἔϲεϲθαι\\
\gll aieí \emph{ti} prosdokôn ap' autês toioûto ésesthai\\
always some.\textsc{n.acc.sg} expect.\textsc{ptcp.m.nom.sg} of it.\textsc{f.gen.sg} such.\textsc{n.acc.sg} be.\textsc{fut.inf.mid}\\
\trans `... always expecting that some such thing would take place from there.' (Herodotus 7.235.2)
\label{tis17}
\end{exe}

\begin{exe}
\ex μή μοί τι δράϲῃϲ παῖδ᾽ ἀνήκεϲτον κακόν\\
\gll mḗ moí \emph{ti} drásēis paîd' anḗkeston kakón\\
not me.\textsc{dat} some.\textsc{n.acc.sg} do.\textsc{2sg.aor.sbjv} child.\textsc{acc.sg} fatal.\textsc{n.acc.sg} evil.\textsc{n.acc.sg}\\
\trans `... lest you should do some deadly harm to my daughter.' (Euripides, \textit{Medea} 283)
\label{tis18}
\end{exe}

\begin{exe}
\ex μή τῳ λαθραίωϲ τέκνα γένναίῳ τέκοι\\
\gll mḗ \emph{tōi} lathraíōs tékna génnaíōi tékoi\\
not some.\textsc{m.dat.sg} secretly child.\textsc{acc.pl} noble.\textsc{m.dat.sg} bear.\textsc{3sg.aor.opt}\\
\trans `... lest she should secretly bear children to some nobleman.' (Euripdides, \textit{Electra} 26)
\label{tis19}
\end{exe}

\begin{exe}
\ex ἔϲτι γάρ τιϲ ἐν δόμοιϲ τύχη\\
\gll ésti gár \emph{tis} en dómois túkhē\\
be.\textsc{3sg.prs} for some.\textsc{f.nom.sg} in house.\textsc{dat.pl} fortune.\textsc{nom.sg}\\
\trans `For something is happening within.' (Euripides, \textit{Helena} 477)
\label{tis20}
\end{exe}

\begin{exe}
\ex εἴ τι τῶν τότε πόλιϲμα\\
\gll eí \emph{ti} tôn tóte pólisma\\
if some.\textsc{n.nom.sg} the.\textsc{gen.pl} then town.\textsc{nom.sg}\\
\trans `If some town of that age ...' (Thucydides 1.10.1)
\label{tis21}
\end{exe}

\begin{exe}
\ex καί τίϲ ἐϲτιν ἀϲτήρ\\
\gll kaí \emph{tís} estin astḗr\\
and who.\textsc{m.nom.sg} be.\textsc{3sg.prs} star.\textsc{nom.sg}\\
\trans `And who is (that) star?' (Aristophanes, \textit{Peace} 834)
\label{tis22}
\end{exe}

\begin{exe}
\ex καὶ γάρ τιν᾽ ἐκφέρουϲι τουτονὶ νεκρόν\\
\gll kaì gár \emph{tin'} ekphérousi toutonì nekrón\\
and for some.\textsc{m.acc.sg} bear.out.\textsc{3sg.prs} this.\textsc{m.acc.sg} corpse.\textsc{acc.sg}\\
\trans `And now they're bringing out some corpse here.' (Aristophanes, \textit{Frogs} 170)
\label{tis23}
\end{exe}

\begin{exe}
\ex μή τιϲ ἡμῖν βαϲκανία περιτρέψῃ τὸν λόγον\\
\gll mḗ \emph{tis} hēmîn baskanía peritrépsēi tòn lógon\\
not some.\textsc{f.nom.sg} us.\textsc{dat} sorcery.\textsc{nom.sg} divert.\textsc{3sg.aor.sbjv} the.\textsc{m.acc.sg} account.\textsc{acc.sg}\\
\trans `... lest some sorcery disrupt our argument.' (Plato, \textit{Phaedo} 95b)
\label{tis24}
\end{exe}

\begin{exe}
\ex μή τίϲ ϲοι ἐναντίοϲ λόγοϲ ἀπαντήϲῃ\\
\gll mḗ \emph{tís} soi enantíos lógos apantḗsēi\\
not some.\textsc{m.nom.sg} you.\textsc{dat} opposite.\textsc{m.nom.sg} account.\textsc{nom.sg} encounter.\textsc{3sg.aor.sbjv}\\
\trans `... lest a certain counter-argument should meet you' (Plato, \textit{Phaedo} 101a)
\label{tis25}
\end{exe}

\begin{exe}
\ex καί τι ἔφη αὐτόθι γελοῖον παθεῖν\\
\gll kaí \emph{ti} éphē autóthi geloîon patheîn\\
and something.\textsc{acc.sg} say.\textsc{3sg.imp} just.there funny.\textsc{n.acc.sg} suffer.\textsc{aor.inf}\\
\trans `And he said that just there he had a ridiculous experience.' (Plato, \textit{Symposium} 174e)
\label{tis26}
\end{exe}

\begin{exe}
\ex καί τίϲ ἐϲτ᾽ ἐν ἐμοὶ δύναμιϲ\\
\gll kaí \emph{tís} est' en emoì dúnamis\\
and some.\textsc{f.nom.sg} be.\textsc{3sg.prs} in me.\textsc{dat} power.\textsc{nom.sg}\\
\trans `And there is a certain power in me' (Plato, \textit{Symposium} 218e)
\label{tis27}
\end{exe}

\begin{exe}
\ex ἤδη του ἔγωγε καὶ ἤκουϲα τῶν ϲοφῶν\\
\gll ḗdē \emph{tou} égōge kaì ḗkousa tôn sophôn\\
already someone.\textsc{m.gen.sg} I.\textsc{nom.emph} also hear.\textsc{1sg.aor} the.\textsc{m.gen.pl} wise.\textsc{m.gen.pl}\\
\trans `Once I even heard from one of the sages ...' (Plato, \textit{Gorgias} 493e)
\label{tis28}
\end{exe}

\begin{exe}
\ex ὅταν τι τοῖϲ φίλοιϲ ἀγαθὸν εὑρίϲκω\\
\gll hótan \emph{ti} toîs phílois agathòn heurískō\\
whenever some.\textsc{n.acc.sg} the.\textsc{m.dat.pl} friend.\textsc{dat.pl} good.\textsc{n.acc.sg} find.\textsc{1sg.prs}\\
\trans `... whenever I find some good thing for my friends.' (Xenophon, \textit{Hellenica} 4.1.10)\footnote{\emph{Translator's note}: The Perseus edition has \textit{exeurískō} for \textit{heurískō}.}
\label{tis29}
\end{exe}

\begin{exe}
\ex εἴ τί που λαμβάνοι Ἀθηναίων πλοῖον\\
\gll eí \emph{tí} pou lambánoi Athēnaíōn ploîon\\
if some.\textsc{n.acc.sg} somewhere take.\textsc{3sg.prs.opt} Athenian.\textsc{m.gen.pl} vessel.\textsc{acc.sg}\\
\trans `... whatever vessel of the Athenians' he might capture anywhere.' (Xenophon, \textit{Hellenica} 4.8.33)
\label{tis30}
\end{exe}

\begin{exe}
\ex ἀλλά τιϲ ἦν ἄκριτοϲ καὶ παρὰ τούτοιϲ καὶ παρὰ τοῖϲ ἄλλοιϲ ἔριϲ\\
\gll allá \emph{tis} ên ákritos kaì parà toútois kaì parà toîs állois éris\\
but some.\textsc{f.nom.sg} be.\textsc{3sg.imp} indiscriminate.\textsc{f.nom.sg} and from this.\textsc{n.dat.pl} and from the.\textsc{n.dat.pl} other.\textsc{n.dat.pl} strife.\textsc{nom.sg}\\
\trans `But in these and in the others was an indiscriminate strife.' (Demosthenes 18.18)
\label{tis31}
\end{exe}

\begin{exe}
\ex ἦν ἄν τιϲ κατὰ τῶν ἐναντιωθέντων οἷϲ ἔπραττεν ἐκεῖνοϲ, μέμψιϲ καὶ κατηγορία\\
\gll ên án \emph{tis} katà tôn enantiōthéntōn hoîs épratten ekeînos, mémpsis kaì katēgoría\\
be.\textsc{3sg.imp} \textsc{irr} some.\textsc{f.nom.sg} down the.\textsc{m.gen.pl} oppose.\textsc{ptcp.aor.pass.m.gen.pl} what.\textsc{dat.pl} do.\textsc{3sg.imp} that.\textsc{m.nom.sg} blame.\textsc{nom.sg} and charge.\textsc{nom.sg}\\
\trans `... there might have been some reproach and charge against those opposed to what that man was doing.' (Demosthenes 18.65)
\label{tis32}
\end{exe}

\begin{exe}
\ex ὅταν τι πράττῃϲ ὅϲιον\\
\gll hótan \emph{ti} práttēis hósion\\
whenever something.\textsc{acc.sg} do.\textsc{2sg.prs.sbjv} holy.\textsc{n.acc.sg}\\
\trans `Whenever you perform anything religious ...' (Menander, Fragment 572; \citealp{Kock1888})
\label{tis33}
\end{exe}

\begin{exe}
\ex ἀλλά τιϲ ἄμμι δαίμων\\
\gll allá \emph{tis} ámmi daímōn\\
but some.\textsc{m.nom.sg} us.\textsc{dat} demon.\textsc{nom.sg}\\
\trans (Fragmenta Lyrica Adespota 58; \citealp[706]{Bergk1882})
\label{tis34}
\end{exe}

\begin{exe}
\ex εἰ γοῦν, ὦ ξένε, τιϲ ἡμῖν ὑπόϲχοιτο θεόϲ\\
\gll ei goûn, ô xéne, \emph{tis} hēmîn hupóskhoito theós\\
if at.least O stranger.\textsc{voc} some.\textsc{m.nom.sg} us.\textsc{dat} supply.\textsc{3sg.aor.opt} god.\textsc{nom.sg}\\
\trans `At least, stranger, if some god were to grant us that ...' (Plato, \textit{Laws} 3.683b)
\label{tis35}
\end{exe}

The word order in examples like \hyperlink{p369}{\emph{[p369]}} (\ref{tis36}) can be explained as imitation of this positioning, in which \textit{tis} clause-medially is separated from the following part of the clause by other words.

\begin{exe}
\ex καὶ αὐτῶν μέροϲ {[}...{]} ἐϲέπεϲεν ἔϲ του χωρίον ἰδιώτου\\
\gll kaì autôn méros esépesen és \emph{tou} khōríon idiṓtou\\
and them.\textsc{gen} member.\textsc{nom.sg} in.fall.\textsc{3sg.aor} into some.\textsc{m.gen.sg} place.\textsc{acc.sg} individual.\textsc{gen.sg}\\
\trans `And a division of them dashed into a field on some private property.' (Thucydides 1.106.1)\footnote{\emph{Translator's note}: The Perseus edition has \textit{kaí ti}.}
\label{tis36}
\end{exe}

And just like its Homeric\il{Greek, Homeric} counterpart, the post-Homeric \textit{tis}\label{tis} prevents other words from being placed in the second position they would otherwise receive. In Attic\il{Greek, Attic} literature, for instance, this is illustrated by the \isi{tmesis} in (\ref{tmesis30}) above and examples such as (\ref{tis37}).

\begin{exe}
\ex ὅντιν᾽ ἄν τιϲ τρόπον ὡϲ βέλτιϲτοϲ εἴη\\
\gll hóntin' án \emph{tis} trópon hōs béltistos eíē\\
what.\textsc{m.acc.sg} \textsc{irr} someone.\textsc{m.nom.sg} way.\textsc{acc.sg} as best.\textsc{m.nom.sg} be.\textsc{3sg.prs.opt}\\
\trans `... in what way someone can be as good as possible ...' (Plato, \textit{Gorgias} 520e)
\label{tis37}
\end{exe}\il{Greek, Classical|)}

But the word order \textit{tis ke} following the introductory word of a conjoined clause, which, in the epic poetry,\is{poetry} is only found in one Homeric\il{Greek, Homeric} and one Hesiodic example (disregarding the common \textit{hóstis ke}), is almost the rule in Doric,\il{Greek, Doric} though of course with \textit{ka} instead of \textit{ke}.\label{aitiska} Compare \citet[383]{Ahrens1843}. In the Gortyn code,\is{inscriptions|(} for instance, we have (\ref{tiska1})--(\ref{tiska5}).

\begin{exe}
\ex αἴ τιϲ κα\\
\gll aí \emph{tis} ka\\
if someone.\textsc{m.nom.sg} \textsc{irr}\\
\trans (Gortyn Code 9.43)
\label{tiska1}
\end{exe}

\begin{exe}
\ex αἴ τινά κα\\
\gll aí \emph{tiná} ka\\
if someone.\textsc{m.acc.sg} \textsc{irr}\\
\trans (Gortyn Code 7.13)
\label{tiska2}
\end{exe}

\begin{exe}
\ex καἴ τί κ᾽\\
\gll kaí \emph{tí} k'\\
and something \textsc{irr}\\
\trans (Gortyn Code 3.29; identically 6.23, 6.43, 9.13)\\
\label{tiska3}
\end{exe}

\begin{exe}
\ex καί μέν τίϲ κ᾽\\
\gll kaí mén \emph{tís} k'\\
and then someone.\textsc{m.nom.sg} \textsc{irr}\\
\trans (Gortyn Code 8.17)
\label{tiska4}
\end{exe}

\begin{exe}
\ex ὅτι δέ τίϲ κα\\
\gll hóti dé \emph{tís} ka\\
that but someone.\textsc{m.nom.sg} \textsc{irr}\\
\trans (Gortyn Code 3.9)
\label{tiska5}
\end{exe}

Deviating from this pattern are (\ref{tiska6}) and (\ref{tiska7}), where \textit{mḗ} `not' has attracted the indefinite, as well as (\ref{tiska8}).

\begin{exe}
\ex αἰ δέ κα μή τιϲ\\
\gll ai dé ka mḗ \emph{tis}\\
if but \textsc{irr} not someone.\textsc{m.nom.sg}\\
\trans (Gortyn Code 5.13; also 5.17, 5.22)
\label{tiska6}
\end{exe}

\begin{exe}
\ex ᾧ δέ κα μή τιϲ ᾖ ϲτέγα\\
\gll hôi dé ka mḗ \emph{tis} êi stéga\\
which.\textsc{dat.sg} but \textsc{irr} not some.\textsc{f.nom.sg} be.\textsc{3sg.prs.sbjv} roof.\textsc{nom.sg}\\
\trans (Gortyn Code 4.14)
\label{tiska7}
\end{exe}

\begin{exe}
\ex ὁπῶ κά τιλ λῇ\\
\gll hopô ká \emph{til} lêi\\
whence \textsc{irr} some wish.\textsc{3sg.prs.sbjv}\\
\trans (Gortyn Code 10.33)
\label{tiska8}
\end{exe}

In later Cretan inscriptions, (\ref{tiska9}) (identically CIG 3049.9, 3058.13) and (\ref{tiska10}) (identically CIG 3049.14, 3058.16).

\begin{exe}
\ex εἰ δέ τινέϲ κα τῶν ὁρμιωμένων\\
\gll ei dé \emph{tinés} ka tôn hormiōménōn\\
if but some.\textsc{m.nom.pl} \textsc{irr} the.\textsc{m.gen.pl} rush.\textsc{ptcp.prs.pass.m.gen.pl}\\
\trans (CIG 3048.33; \citealp[82, no. 123]{Cauer1883})
\label{tiska9}
\end{exe}

\begin{exe}
\ex εἴ τίϲ κα ἄγῃ\\
\gll eí \emph{tís} ka ágēi\\
if someone.\textsc{m.nom.sg} \textsc{irr} lead.\textsc{3sg.prs.sbjv}\\
\trans `If anyone should bring ...' (CIG 3048.38)
\label{tiska10}
\end{exe}

On the Heraclean Tablets, (\ref{tiska11})--(\ref{tiska16}).

\begin{exe}
\ex καὶ αἴ τινί κα ἄλλῳ\\
\gll kaì aí \emph{tiní} ka állōi\\
and if some.\textsc{dat.sg} \textsc{irr} other.\textsc{dat.sg}\\
\trans (Heraclean Tablets 1.105)
\label{tiska11}
\end{exe}

\begin{exe}
\ex καὶ αἴ τινάϲ κα ἄλλουϲ\\
\gll kaì aí \emph{tinás} ka állous\\
and if some.\textsc{m.acc.pl} \textsc{irr} other.\textsc{m.acc.pl}\\ 
\trans (Heraclean Tablets 1.117)
\label{tiska12}
\end{exe}

\begin{exe}
\ex αἰ δέ τινά κα γήρᾳ {[}...{]} ἐκπέτωντι\\
\gll ai dé \emph{tiná} ka gḗrāi ekpétōnti\\
if but someone.\textsc{m.acc.sg} \textsc{irr} age.\textsc{dat.sg}
depart.\textsc{ptcp.aor.dat.sg}\\
\trans (Heraclean Tablets 1.119; also 1.173, without \textit{dé})
\label{tiska13}
\end{exe}

\begin{exe}
\ex καὶ εἴ τινέϲ κα μὴ πεφυτεύκωντι\\
\gll kaì eí \emph{tinés} ka mḕ pephuteúkōnti\\
and if some.\textsc{m.nom.pl} \textsc{irr} not plant.\textsc{ptcp.prf.dat.sg}\\
\trans (Heraclean Tablets 1.127)
\label{tiska14}
\end{exe}

\begin{exe}
\ex αἰ δέ τίϲ κα ἐπιβῇ\\
\gll ai dé \emph{tís} ka epibêi\\
if but someone.\textsc{m.nom.sg} \textsc{irr} enter.\textsc{3sg.aor.sbjv}\\
\trans `And if anyone should enter ...' (Heraclean Tablets 1.128)
\label{tiska15}
\end{exe}

\begin{exe}
\ex αἰ δέ τιϲ κα τῶν καρπιζομένων ἀποθάνει\\
\gll ai dé \emph{tis} ka tôn karpizoménōn apothánei\\ 
if but someone.\textsc{m.nom.sg} \textsc{irr} the.\textsc{gen.pl}
enjoy.\textsc{ptcp.prs.pass.gen.pl} die.\textsc{3sg.fut}\\
\trans `And if anyone dies of these enjoyments ...' (Heraclean Tablets 1.151)
\label{tiska16}
\end{exe}

In the inscription of Orchomenos, (\ref{tiska17}). In the inscription from Mycenae, (\ref{tiska18}).

\begin{exe}
\ex καὶ εἴ τίϲ κα μὴ ἐμμένῃ\\
\gll kaì eí \emph{tís} ka mḕ emménēi\\
and if someone.\textsc{m.nom.sg} \textsc{irr} not abide.\textsc{3sg.prs.sbjv}\\
\trans `And if no one should remain ...' (Orchomenos Inscription 178.10; \citealp[278]{Dittenberger1883})
\label{tiska17}
\end{exe}

\begin{exe}
\ex αἰ δέ τί κα πένηται\\
\gll ai dé \emph{tí} ka pénētai\\
if but something \textsc{irr} labour.\textsc{3sg.prs.sbjv.pass}\\
\trans (Mycenae Inscription 3316.8; \citep[137]{Prellwitz1889})
\label{tiska18}
\end{exe}

In the Korkyra inscriptions \citep[93--98]{Blass1888}, (\ref{tiska19})--(\ref{tiska21}).

\begin{exe}
\ex εἰ δέ τί κ᾽ ἀδύνατον γένοιτο\\
\gll ei dé \emph{tí} k' adúnaton génoito\\
if but something.\textsc{nom.sg} \textsc{irr} unable.\textsc{n.nom.sg}
become.\textsc{3sg.aor.opt.mid}\\
\trans `And if anything impossible should come to pass ...' (Korkyra Inscription 3206.25)
\label{tiska19}
\end{exe}

\begin{exe}
\ex εἰ δέ τί κα {[}...{]} μὴ ὀρθῶϲ ἀπολογίξωνται {[}sic{]}\\
\gll ei dé \emph{tí} ka mḕ orthôs apologíxōntai\\
if but something.\textsc{acc.sg} \textsc{irr} not straight
reckon.\textsc{3pl.prs.sbjv.pass}\\
\trans `But if they should give an incorrect account of anything ...' (Korkyra Inscription 3206.103)
\label{tiska20}
\end{exe}

\begin{exe}
\ex εἴ τινόϲ κα ἄλλου δοκῆ\\
\gll eí \emph{tinós} ka állou dokê\\
if some.\textsc{gen.sg} \textsc{irr} other.\textsc{gen.sg}
seem.\textsc{3sg.imp}\\ 
\trans (Korkyra Inscription 3206.114)
\label{tiska21}
\end{exe}

Perhaps also (\ref{tiska22}). (See below p\pageref{Theocritus2159}.)

\begin{exe}
\ex αἰ δέ τί κα με {[}...{]} λυπῇ\\
\gll ai dé \emph{tí} ka me lupêi\\
if but something.\textsc{nom.sg} \textsc{irr} me.\textsc{acc}
trouble.\textsc{3sg.prs.sbjv}\\
\trans `And if anything should pain me ...' (Theocritus 2.159)\footnote{\emph{Translator's note}: The Perseus edition has \textit{d' éti kēmè lupêi}.}
\label{tiska22}
\end{exe}

In view of such constant usage, in contrast to which the only counterexamples I can find (other than the Gortyn exceptions, in which sometimes \textit{mḗ} `not' is present and sometimes \textit{ei} `if' does not precede) are (\ref{tiska23}) and (\ref{tiska24}), it seems clear to me that in the Korkyra inscription 3213.3 \hyperlink{p370}{\emph{[p370]}} the transmitted sequence \textit{aí ka páskhē} should not be emended,\is{emendation} with \citet[27]{Boeckh1843}, to \textit{aí ka \emph{tí} páskhē}, but rather to \textit{aí \emph{tí} ka páskhē}, as shown in (\ref{tiska25}).

\begin{exe}
\ex καἴ κά τιϲ ἀντίον \textless{}\emph{τι}\textgreater{} λῇ τήνῳ λέγειν\\
\gll kaí ká \emph{tis} antíon ti lêi tḗnōi légein\\
and=if \textsc{irr} someone.\textsc{m.nom.sg} contrary.\textsc{n.acc.sg} something.\textsc{acc.sg} wish.\textsc{3sg.prs.sbjv} that.\textsc{m.dat.sg} say.\textsc{prs.inf}\\
\trans `And if anyone should want to say something against that man ...' (Epicharmus in Athenaeus 6.28; \citealp[227]{Lorenz1864} line 5)
\label{tiska23}
\end{exe}

\begin{exe}
\ex αἴ κά τιϲ ἐκτρίψαϲ καλῶϲ παρατιθῇ νιν\\
\gll aí ká \emph{tis} ektrípsas kalôs paratithêi nin\\
if \textsc{irr} someone.\textsc{m.nom.sg} rub.out.\textsc{ptcp.aor.m.nom.sg} well serve.\textsc{3sg.prs.sbjv} \textsc{cl}\\
\trans `If, having bruised them well, one were to serve them ...' (Epicharmus in Athenaeus 2.83; \citealp[281]{Lorenz1864})
\label{tiska24}
\end{exe}

\begin{exe}
\ex αἴ \textless{}\emph{τί}\textgreater{} κα πάϲχη\\
\gll aí \emph{tí} ka páskhē\\
if something \textsc{irr} suffer.\textsc{3sg.prs.sbjv}\\
\trans(Korkyra Inscription 3213.3; \citealp[100]{Blass1888}; = CIG 1850)
\label{tiska25}
\end{exe}

Moreover, this positional custom is not only Doric:\il{Greek, Doric} the Idalion Tablet line 29 gives us example (\ref{tiska26}). See also (\ref{Ath3.75}), with separation of \textit{árton turônta} `cheese bread'.

\begin{exe}
\ex ὄπι ϲίϲ κε τὰϲ ϝρήταϲ τάϲδε λύϲη\\
\gll ópi \emph{sís} ke tàs wrḗtas tásde lúsē\\
that someone.\textsc{m.nom.sg} \textsc{irr} the.\textsc{f.acc.pl}
stated.\textsc{f.acc.pl} this.\textsc{f.acc.pl} loose.\textsc{3sg.prs.sbjv}\\
\trans `... that someone rescind what was stated ...' (Idalion Tablet 29)
\label{tiska26}
\end{exe}

\begin{exe}
\ex ἄρτον γάρ τιϲ τυρῶντα τοῖϲ παιδίοιϲ ἴαλε\\
\gll árton gár \emph{tis} turônta toîs paidíois íale\\
loaf.\textsc{acc.sg} for someone.\textsc{m.nom.sg} cheese-flavour.\textsc{ptcp.prs.m.acc.sg} the.\textsc{m.dat.pl} child.\textsc{dat.pl} send.\textsc{3sg.aor}\\
\trans `For someone has given a loaf of cheese bread to the children.' (Epicharmus in Athenaeus 3.75)
\label{Ath3.75}
\end{exe}

Finally, one might ask whether the insertion of \textit{tis} between the article (and adjective\is{adjectives} if present) and the noun of the governed partitive \isi{genitive} (e.g. (\ref{example1})--(\ref{example3})), common from Herodotus to the \isi{prose} writers, might have occurred in clauses where this separation caused \textit{tis} to appear in second position.

\begin{exe}
\ex τῶν τινα Λυδῶν\\
\gll tôn \emph{tina} Ludôn\\
the.\textsc{m.gen.pl} someone.\textsc{m.acc.sg} Lydian.\textsc{m.gen.pl}\\
\trans `one of the Lydians'
\label{example1}
\end{exe}

\begin{exe}
\ex ἐϲ τῶν ἐκείνων τι χωρίων\\
\gll es tôn ekeínōn \emph{ti} khōríōn\\
into the.\textsc{n.gen.pl} that.\textsc{n.gen.pl} something.\textsc{acc.sg} place.\textsc{gen.pl}\\
\trans `into some of that property'
\label{example2}
\end{exe}

\begin{exe}
\ex τῶν ἄλλων τινὰϲ Ἑλλήνων\\
\gll tôn állōn \emph{tinàs} Hellḗnōn\\
the.\textsc{m.gen.pl} other.\textsc{m.gen.pl} some.\textsc{m.acc.pl}
Greek.\textsc{m.gen.pl}\\
\trans `some of the other Greeks'
\label{example3}
\end{exe}\is{inscriptions|)}

The \isi{adverbs} derived from the indefinite pronoun follow our rule quite strictly in Homer.\il{Greek, Homeric|(} In books 13, 16 and 17 of the \textit{Iliad}, \textit{pou} `somewhere' can be found 14 times, always in second position: particularly noteworthy among these examples are (\ref{adv1}), with separation of \textit{mḗ} `not' and \textit{tis} `someone', and (\ref{adv2}). \textit{pothi} `somewhere' is found twice, in (\ref{adv3})--(\ref{adv4}), where it is preceded by \textit{ou} `not'.

\begin{exe}
\ex μή πού τιϲ ὑπερφιάλωϲ νεμεϲήϲῃ\\
\gll mḗ \emph{poú} tis huperphiálōs nemesḗsēi\\
not somewhere someone.\textsc{m.nom.sg} excessively resent.\textsc{3sg.aor.sbjv}\\
\trans `... lest haply some man wax wroth beyond measure' (Homer, \textit{Iliad} 13.293)
\label{adv1}
\end{exe}

\begin{exe}
\ex ἀλλά που\\
\gll allá \emph{pou}\\
but somewhere\\
\trans (Homer, \textit{Iliad} 13.225)
\label{adv2}
\end{exe}

\begin{exe}
\ex ἀλλά ποθι\\
\gll allá \emph{pothi}\\
but somewhere\\
\trans (Homer, \textit{Iliad} 13.630)
\label{adv3}
\end{exe}

\begin{exe}
\ex ἐπὶ οὔ ποθι ἔλπομαι\\
\gll epì oú \emph{pothi} élpomai\\
upon not somewhere hope.\textsc{1sg.prs.pass}\\
\trans `Verily, methinks, in no other place ...' (Homer, \textit{Iliad} 13.309)\footnote{\emph{Translator's note}: The Perseus edition has \textit{epeì}.}
\label{adv4}
\end{exe}

Nine instances of \textit{pōs} `somehow' are found, seven of which are in second position, as well as (\ref{adv5}) (twice).

\begin{exe}
\ex ἀλλ᾽ οὔ πωϲ\\
\gll all' oú \emph{pōs}\\
but not somehow\\ 
\trans (Homer, \textit{Iliad} 13.729 and 17.354)
\label{adv5}
\end{exe}

\textit{pote} `sometime' is found four times, twice in second position, as well as (\ref{adv6}) and (\ref{adv7}).

\begin{exe}
\ex ἄλλοτε δή ποτε μᾶλλον ἐρωῆϲαι πολέμοιο μέλλω\\
\gll állote dḗ \emph{pote} mâllon erōêsai polémoio méllō\\
another.time exactly sometime more withdraw.\textsc{aor.inf}
war.\textsc{gen.sg} be.going.to.\textsc{1sg.prs}\\
\trans `At some other time have I haply withdrawn me from war rather than now' (Homer, \textit{Iliad} 13.776)
\label{adv6}
\end{exe}

\begin{exe}
\ex ἠμὲν δή ποτ᾽ ἐμὸν ἔποϲ ἔκλυεϲ εὐξαμένοιο\\
\gll ēmèn dḗ \emph{pot'} emòn épos éklues euxaménoio\\
both exactly sometime my.\textsc{n.acc.sg} word.\textsc{acc.sg}
hear.\textsc{2sg.aor} pray.\textsc{ptcp.aor.m.gen.sg}\\
\trans `Aforetime verily you did hear my word, when I prayed' (Homer, \textit{Iliad} 16.236)
\label{adv7}
\end{exe}

\textit{pêi} `somehow' is found only once (16.110), correctly. \textit{pō} `yet' is found five times correctly, and also in (\ref{adv8}) and (\ref{adv9}). (\citealp[336ff.]{Monro1891} provides exceptions from the other books.)

\begin{exe}
\ex θέων δ᾽ ἐκίχανεν ἑταίρουϲ ὦκα μάλ᾽, οὔ πω τῆλε, ποϲὶ κραιπνοῖϲι μεταϲπών\\
\gll théōn d' ekíkhanen hetaírous ôka mál', oú \emph{pō} têle, posì kraipnoîsi metaspṓn\\
run.\textsc{ptcp.prs.m.nom.sg} then reach.\textsc{3sg.imp}
companion.\textsc{acc.pl} swiftly very not yet far foot.\textsc{dat.pl} swift.\textsc{m.dat.pl} pursue.\textsc{ptcp.aor.m.nom.sg}\\
\trans `(He) ran, and speedily reached his comrades not yet far off, hastening after them with swift steps' (Homer, \textit{Iliad} 17.189)
\label{adv8}
\end{exe}

\begin{exe}
\ex δύο δ᾽ οὔ πω φῶτε πεπύϲθην\\
\gll dúo d' oú \emph{pō} phôte pepústhēn\\
two then not yet man.\textsc{nom.du} learn.\textsc{3du.plup.pass}\\
\trans `Howbeit two men had not yet learned ...' (Homer, \textit{Iliad} 17.377)
\label{adv9}
\end{exe}\il{Greek, Homeric|)}

Texts from the post-Homeric\il{Greek, Classical|(} period allow these \isi{particles} a great deal of freedom. Remnants of the old rule can be seen (other than in \textit{ēpou} and \textit{dḗpou}) in examples such as (\ref{adv10}) and (\ref{adv11}). (Following such a template also (\ref{adv12}) and (\ref{adv13}).) Compare also (\ref{adv14}) and (\ref{adv15}).

\begin{exe}
\ex ἔν ποκ᾽ ἄρα Σπάρτᾳ\\
\gll én \emph{pok'} ára Spártāi\\
in sometime then Sparta.\textsc{dat}\\
\trans `So once in Sparta ...' (Theocritus 18.1)
\label{adv10}
\end{exe}

\begin{exe}
\ex ἔκ ποτέ τιϲ φρικτοῖο θεᾶϲ ϲεϲοβημένοϲ οἴϲτρῳ\\
\gll ék \emph{poté} tis phriktoîo theâs sesobēménos oístrōi\\
out sometime someone.\textsc{m.nom.sg} awful.\textsc{f.gen.sg} goddess.\textsc{gen.sg} scare.\textsc{ptcp.prf.pass.m.nom.sg} sting.\textsc{dat.sg}\\
\trans `Someone agitated at some time by a sting from an awful
goddess ...' (Anthologia Graeca 6.219.1)
\label{adv11}
\end{exe}

\begin{exe}
\ex ὅτι τε μεγαλοκευθέεϲιν ἔν ποτε θαλάμοιϲ\\
\gll hóti te megalokeuthéesin én \emph{pote} thalámois\\
that and much-concealing.\textsc{m.dat.pl} in sometime chamber.\textsc{dat.pl}\\
\trans `... and because once, in the vast recesses of the bridal
chamber ...' (Pindar, \textit{Pythian} 2.33)
\label{adv12}
\end{exe}

\begin{exe}
\ex Ἴξαλοϲ εὐπώγων αἰγὸϲ πόϲιϲ ἔν ποθ᾽ ἁλωῇ\\
\gll Íxalos eupṓgōn aigòs pósis én \emph{poth'} halōêi\\
bounding.\textsc{m.nom.sg} well-bearded.\textsc{m.nom.sg} goat.\textsc{gen.sg} husband.\textsc{nom.sg} in sometime yard.\textsc{dat.sg}\\
\trans `Once in a vineyard, the bounding, well-bearded husband of the she-goat ...' (Anthologia Graeca 9.99.1)
\label{adv13}
\end{exe}

\begin{exe}
\ex ἄλλη που ἐπιϲτήμη ἀνθρώπου καὶ λύραϲ\\
\gll állē \emph{pou} epistḗmē anthrṓpou kaì lúras\\
other.\textsc{f.nom.sg} somewhere knowledge.\textsc{nom.sg}
person.\textsc{gen.sg} and lyre.\textsc{gen.sg}\\
\trans `Knowledge of a man and of a lyre (are) in some way different.' (Plato, \textit{Phaedo} 73d)
\label{adv14}
\end{exe}

\begin{exe}
\ex ὁ αὐτὸϲ γάρ που φόβοϲ\\
\gll ho autòs gár \emph{pou} phóbos\\
the.\textsc{m.nom.sg} same.\textsc{m.nom.sg} for somewhere
fear.\textsc{nom.sg}\\
\trans `For in some way (there would be) the same fear.' (Plato, \textit{Phaedo} 101b)
\label{adv15}
\end{exe}\il{Greek, Classical|)}\is{pronouns|)}\is{indefinites|)}

Looking at other enclitic\is{enclitics} \isi{particles} is much more fruitful. It is true that the consistent appearance of \textit{te} `and/also' and \textit{rha} `so, then, therefore' in second position (in (\ref{adv16}), \hyperlink{p371}{\emph{[p371]}} the participle\is{participles} has the same role as a subordinate\is{subordination} clause) could be explained with reference to their function as clausal connectors. 

\begin{exe}
\ex βωμοῦ ὑπαΐξαϲ πρόϲ ῥα πλατάνιϲτον ὄρουϲεν\\
\gll bōmoû hupaḯxas prós \emph{rha} platániston órousen\\
altar.\textsc{gen.sg} glide.\textsc{ptcp.aor.m.nom.sg} to then plane.\textsc{acc.sg} dart.\textsc{3sg.aor}\\
\trans `(It) glided from beneath the altar and darted to the plane tree.' (Homer, \textit{Iliad} 2.310)
\label{adv16}
\end{exe}

On the other hand, \textit{ge} `at least/only/in fact' is immune to any such consistent positional rule, because it may not occur on the word on which the main weight of affirmation falls; at most one could point out that in Thucydides there are several examples of a \textit{ge} that belongs to a participle\is{participles} but is attached to a preceding word (\citealp[79]{PoppoStahl1889} on Thucydides 2.38.1): (\ref{greekge})--(\ref{adv18}). Cf. example (\ref{adv19}) (instead of \textit{hṓs émoìge dokeî}). What has been said for \textit{ge} holds also for \textit{per}.

\begin{exe}
\ex ἀγῶϲι μέν γε καὶ θυϲίαιϲ διετηϲίοιϲ νομίζοντεϲ\\
\gll agôsi mén \emph{ge} kaì thusíais dietēsíois nomízontes\\
gathering.\textsc{dat.pl} then even and sacrifice.\textsc{dat.pl}
year.round.\textsc{m.dat.pl} practise.\textsc{ptcp.prs.m.nom.pl}\\
\trans `... celebrating games and sacrifices all the year round ...' (Thucydides 2.38.1)
\label{greekge}
\end{exe}

\begin{exe}
\ex οὕτω τῇ γε παρούϲῃ εὐτυχίᾳ χρώμενοι\\
\gll hoútō têi \emph{ge} paroúsēi eutukhíāi khrṓmenoi\\
so the.\textsc{f.dat.sg} even be.present.\textsc{ptcp.prs.f.dat.sg}
success.\textsc{dat.sg} use.\textsc{ptcp.prs.pass.m.nom.pl}\\
\trans `Being so used to the present prosperity ...' (Thucydides 4.65.4)\footnote{\emph{Translator's note}: The Perseus edition has \textit{te}.}
\label{adv17}
\end{exe}

\begin{exe}
\ex πίϲτειϲ γε διδοὺϲ τὰϲ μεγίϲταϲ\\
\gll písteis \emph{ge} didoùs tàs megístas\\
faith.\textsc{acc.pl} even give.\textsc{ptcp.aor.m.nom.sg} the.\textsc{f.acc.pl} greatest.\textsc{f.acc.pl}\\
\trans `... having given the greatest possible guarantees ...' (Thucydides 4.86.2)
\label{adv18}
\end{exe}

\begin{exe}
\ex ὥϲ γ᾽ ἐμοὶ δοκεῖ\\
\gll hṓs \emph{g'} emoì dokeî\\
as even me.\textsc{dat} seem.\textsc{3pl.prs}\\
\trans `... as it seems to me at least ...' (Demosthenes 18.226)
\label{adv19}
\end{exe}

But there is one constantly enclitic\is{enclitics} particle\is{particles} that, although not serving to connect clauses, has a wholly unmistakable preference for second position, namely \textit{ke} (\textit{ken}, \textit{ka}; \textsc{irr}). \citet[7]{Hermann1831} has already indicated this with the words ``\textit{ken}, which is barred from the beginning of an utterance because it is enclitic,\is{enclitics} can also be placed before those words with whose meaning it is associated, as long as some word in the same sentence precedes it'', and illustrates this with the example (\ref{ke1}).

\begin{exe}
\ex ἦ κε μέγ᾽ οἰμώξειε γέρων ἱππηλάτα Πηλεύϲ\\
\gll ê \emph{ke} még' oimṓxeie gérōn hippēláta Pēleús\\
in.truth \textsc{irr} greatly wail.\textsc{3sg.aor.opt} old.\textsc{m.nom.sg} driver.\textsc{nom.sg} Peleus.\textsc{nom}\\
\trans `Verily aloud would old Peleus groan, the driver of chariots' (Homer, \textit{Iliad} 7.125)
\label{ke1}
\end{exe}

However, it does not occur to Hermann\ia{Hermann, Gottfried} that the particle\is{particles} belongs in the second position in the clause. And even the most recent overview of the Homeric\il{Greek, Homeric} use of \textit{ke}, \citet{Eberhard1885}, although devoting seven closely printed columns to its position, does not go beyond Hermann\ia{Hermann, Gottfried} theoretically, even though one would have thought that the material he had collected would put him on the right track -- for instance, when he emphasizes, following \citet[34]{Schnorr1864}, that \textit{ke} follows the verb only when it is clause-initial and follows the participle\is{participles} only in (\ref{ke2}), or that this attachment of \textit{ke} to a preceding word is found only ``at the start of a verse''.\is{poetry}

\begin{exe}
\ex ἰδοῦϲα κε θυμὸν ἰάνθηϲ\\
\gll idoûsa \emph{ke} thumòn iánthēs\\
see.\textsc{ptcp.aor.f.nom.sg} \textsc{irr} spirit.\textsc{acc.sg} warm.\textsc{2sg.aor.pass}\\
\trans `The sight would have warmed your heart with cheer.' (Homer, \textit{Odyssey} 23.47)
\label{ke2}
\end{exe}

It is generally recognized that, in every Greek dialect that has a form of \textit{ke} at all, the particle\is{particles} immediately follows the clause-initial pronoun\is{pronouns} or subordinating\is{subordination} conjunction without exception, unless other \isi{enclitics} or quasi-\isi{enclitics} like \textit{te}, \textit{dé}, \textit{gár}, \textit{mén} and occasionally also \textit{tis} (see above \hyperlink{p372}{\emph{[p372]}} p\pageref{tis}), \textit{tu} (see above p\pageref{tu}) and \textit{toi} (as in example (\ref{toi3})) intervene: \textit{hós ke}, \textit{eis hó ke}, \textit{eí ke}, \textit{aí ke}, \textit{epeíke}, \textit{hóte ke} (Doric\il{Greek, Doric} \textit{hókka}), \textit{éōs ke}, \textit{hóphra ke}, \textit{hṓs ke}, \textit{ho(p)pōs ke} or \textit{hos dé ke}, \textit{ei dé ke} and similar. (But see (\ref{epicharmos628}) and (\ref{theocritus15}) as well as (\ref{theocritus110}) etc.)

\begin{exe}
\ex ὅ τοί κ᾽ ἐπὶ τὸν νόον ἔλθῃ\\
\gll hó toí \emph{k'} epì tòn nóon élthēi\\ 
which.\textsc{n.nom.sg} lo \textsc{irr} upon the.\textsc{m.acc.sg}
mind.\textsc{acc.sg} go.\textsc{3sg.aor.sbjv}\\
\trans `... which, you see, would come to mind.' (Theognis, \textit{Elegies} 633)
\label{toi3}
\end{exe}

\begin{exe}
\ex αἴκα δ᾽ ἐντύχω τοῖϲ περιπόλοιϲ\\
\gll aí\emph{ka} d' entúkhō toîs peripólois\\
if=\textsc{irr} then encounter.\textsc{1sg.aor.sbjv} the.\textsc{m.dat.pl} watchman.\textsc{dat.pl}\\
\trans `And if I should ever encounter the watchmen ...' (Epicharmus in Athenaeus 6.28; \citealp[225]{Lorenz1864})
\label{epicharmos628}
\end{exe}

\begin{exe}
\ex αἴκα δ᾽ αἶγα λάβῃ τῆνοϲ γέραϲ\\
\gll aí\emph{ka} d' aîga lábēi tênos géras\\
if=\textsc{irr} then goat.\textsc{acc.sg} take.\textsc{3sg.aor.sbjv} that.\textsc{m.nom.sg} prize.\textsc{acc}\\
\trans `And if that one should win a goat as a prize ...' (Theocritus 1.5)
\label{theocritus15}
\end{exe}

\begin{exe}
\ex αἰ δέ κ᾽ ἀρέϲκῃ\\
\gll ai dé \emph{k'} aréskēi\\
if but \textsc{irr} please.\textsc{3sg.prs.sbjv}\\
\trans `And if it should please ...' (Theocritus 1.10)
\label{theocritus110}
\end{exe}

Ahrens' \citeyearpar[24]{Ahrens1855} suggestion of \textit{ai d' étí ká me ... lupēi} for Theocritus 2.159 (=(\ref{tiska22}) above)\label{Theocritus2159}\footnote{\emph{Translator's note}: Wackernagel here cites Theocritus 1.159 in the original, but this must be an error.} accepted by \citet[28, 213]{Meineke1856} and \citet[75]{FritzscheHiller1881}, so that \textit{ai} is separated from \textit{ka} by \textit{éti}, seems inconceivable to me. The context does not preclude the only grammatical possibility \textit{ai dé tí ka me} and counting this example among those mentioned above on p\pageref{aitiska} with \textit{tís} between \textit{ai} and \textit{ka}. (\citealp[12]{Hermann1817} has \textit{ei d' étí kaí me ... lupeî}, which is less promising.)

Other clause types show a corresponding pattern. In Homer,\il{Greek, Homeric|(} main clauses and interrogative\is{interrogatives} subordinate\is{subordination} clauses with a \isi{subjunctive} verb have \textit{ke} exceptionlessly in second position, as in examples (\ref{ke3})--(\ref{ke5}) from books 13, 16 and 17 of the \textit{Iliad}. 

\begin{exe}
\ex ἐγὼ δέ κε λαὸν ἀγείρω\\
\gll egṑ dé \emph{ke} laòn ageírō\\
I.\textsc{nom} but \textsc{irr} people.\textsc{acc} gather.\textsc{1sg.prs}\\
\trans `And I will gather the host.' (Homer, \textit{Iliad} 16.129)
\label{ke3}
\end{exe}

\begin{exe}
\ex (ἐπιφραϲϲαίμεθα βουλήν) ἤ κεν ἐνὶ νήεϲϲι πολυκλήιϲι πέϲωμεν {[}...{]} ἤ κεν ἔπειτα παρ νηῶν ἔλθωμεν\\
\gll epiphrassaímetha boulḗn ḗ \emph{ken} enì nḗessi poluklḗisi pésōmen ḗ \emph{ken} épeita par nēôn élthōmen\\
consider.\textsc{1pl.aor.opt.mid} counsel.\textsc{acc} or \textsc{irr} in ship.\textsc{dat.pl} many-benched.\textsc{f.dat.pl} fall.\textsc{1pl.aor.sbjv} or
\textsc{irr} then from ship.\textsc{gen.pl} go.\textsc{1pl.aor.sbjv}\\
\trans `(We shall consider counsel,) whether we shall fall upon the many-benched ships or thereafter shall return back from the ships.' (Homer, \textit{Iliad} 13.741)
\label{ke4}
\end{exe}

\begin{exe}
\ex ἤ κ᾽ αὐτὸϲ ἐνὶ πρώτοιϲιν ἁλώῃ\\
\gll ḗ \emph{k'} autòs enì prṓtoisin halṓēi\\
or \textsc{irr} same.\textsc{m.nom.sg} in first.\textsc{m.dat.pl} succumb.\textsc{3sg.aor.opt}\\
\trans `... or haply himself be slain amid the foremost.' (Homer, \textit{Iliad} 17.506)
\label{ke5}
\end{exe}

The same is true of \isi{future} clauses: (\ref{ke6})--(\ref{ke8}). (This is true more generally, even to the extent of separating words which belong together: (\ref{ke9}).)

\begin{exe}
\ex ὥϲ κε τάχα Τρώων κορέει κύναϲ ἠδ᾽ οἰωνούϲ\\
\gll hṓs \emph{ke} tákha Trṓōn koréei kúnas ēd' oiōnoús\\
as \textsc{irr} quickly Trojan.\textsc{gen.pl} glut.\textsc{3sg.fut} dog.\textsc{acc.pl} and raptor.\textsc{acc.pl}\\
\trans `... as it shall presently glut the dogs and birds of the Trojans' (Homer, \textit{Iliad} 17.241)\footnote{\emph{Translator's note}: The Perseus edition has \textit{hós} for \textit{hṓs}.}
\label{ke6}
\end{exe}

\begin{exe}
\ex εἴ κ᾽ Ἀχιλῆοϲ ἀγαυοῦ πιϲτὸν ἑταῖρον τείχει ὕπο Τρώων ταχέεϲ κύνεϲ ἑλκήϲουϲιν\\
\gll eí \emph{k'} Akhilêos agauoû pistòn hetaîron teíkhei húpo Trṓōn takhées kúnes helkḗsousin\\
if \textsc{irr} Achilles.\textsc{gen} noble.\textsc{m.gen.sg} trustworthy.\textsc{m.acc.sg} companion.\textsc{acc.sg} wall.\textsc{dat.sg} under Trojan.\textsc{gen.pl} swift.\textsc{m.nom.pl} dog.\textsc{nom.pl} tear.\textsc{3pl.fut}\\
\trans `... if the trusty comrade of lordly Achilles be torn by swift dogs beneath the wall of the Trojans.' (Homer, \textit{Iliad} 17.557)
\label{ke7}
\end{exe}

\begin{exe}
\ex τὰ δέ κεν Διὶ πάντα μελήϲει\\
\gll tà dé \emph{ken} Diì pánta melḗsei\\
the.\textsc{n.nom.pl} but \textsc{irr} Zeus.\textsc{dat} all.\textsc{n.nom.pl} matter.\textsc{3sg.fut}\\
\trans `... and the issue shall rest with Zeus.' (Homer, \textit{Iliad} 17.515)
\label{ke8}
\end{exe}

\begin{exe}
\ex τῷ δέ κε νικήϲαντι φίλη κεκλήϲῃ ἄκοιτιϲ\\
\gll tôi dé \emph{ke} nikḗsanti phílē keklḗsēi ákoitis\\
the.\textsc{m.dat.sg} but \textsc{irr} win.\textsc{ptcp.aor.m.dat.sg}
dear.\textsc{f.nom.sg} call.\textsc{2sg.fprf.pass} bedfellow.\textsc{nom.sg}\\
\trans `And whoso shall conquer, his dear wife shall you be called.' (Homer, \textit{Iliad} 3.138)
\label{ke9}
\end{exe}

Usage with the \isi{optative} and \isi{preterite} is no different. In books 13, 16 and 17 we have 28 instances of \textit{ke} in second or near-second position in \isi{optative} clauses (including (\ref{ke10})--(\ref{ke11})) and seven instances in \isi{preterite} clauses. Among these 35 examples, the following are particularly noteworthy: \textit{allá ken} in \textit{Iliad} 13.290 (as well as three instances in the Odyssey) and \textit{kaí ken} in 13.377,
17.613 (and many other examples; see \citet[733]{Eberhard1885}; alos cf. \textit{kaí moi}), as well as (\ref{ke12}) in which \textit{ke} precedes \isi{negation}. There is only one counterexample: (\ref{ke13}), where the shift of interrogative\is{interrogatives} \textit{tís} from its usual position clause-initially has taken \textit{ke} \hyperlink{p373}{\emph{[p373]}} along with it, as the latter may not precede \textit{tís}.

\begin{exe}
\ex ἃϲ οὔτ᾽ ἄν κεν Ἄρηϲ ὀνόϲαιτο μετελθών οὔτε κ᾽ Ἀθηναίη\\
\gll hàs oút' án \emph{ken} Árēs onósaito metelthṓn oúte \emph{k'} Athēnaíē\\
which.\textsc{f.acc.pl} nor \textsc{irr} \textsc{irr} Ares.\textsc{nom}
scorn.\textsc{3sg.aor.opt.mid} enter.\textsc{ptcp.aor.m.nom.sg} nor \textsc{irr} Athene.\textsc{nom}\\
\trans `... that not Ares might have entered in and made light of them, nor yet Athene' (Homer, \textit{Iliad} 13.127)
\label{ke10}
\end{exe}

\begin{exe}
\ex ὢ πόποι, ἤδη μέν κε {[}...{]} γνοίη\\
\gll ṑ pópoi, ḗdē mén \emph{ke} gnoíē\\
O fie already then \textsc{irr} know.\textsc{3sg.aor.opt}\\
\trans `Out upon it, now may (any man) know ...' (Homer, \textit{Iliad} 17.629)
\label{ke11}
\end{exe}

\begin{exe}
\ex ἀνδρὶ δέ κ᾽ οὐκ εἴξειε μέγαϲ Τελαμώνιοϲ Αἴαϲ\\
\gll andrì dé \emph{k'} ouk eíxeie mégas Telamṓnios Aías\\
man.\textsc{dat.sg} but \textsc{irr} not yield.\textsc{3sg.aor.opt}
great.\textsc{m.nom.sg} Telamonian.\textsc{m.nom.sg} Ajax.\textsc{nom}\\
\trans `But to no man would great Telamonian Aias yield' (Homer, \textit{Iliad} 13.321)
\label{ke12}
\end{exe}

\begin{exe}
\ex τῶν δ᾽ ἄλλων τίϲ κεν ᾗϲι φρεϲὶν οὐνόματ᾽ εἴποι\\
\gll tôn d' állōn tís \emph{ken} hêisi phresìn ounómat' eípoi\\
the.\textsc{m.gen.pl} then other.\textsc{m.gen.pl} who.\textsc{m.nom.sg} \textsc{irr} his.\textsc{f.dat.pl} midriff.\textsc{dat.pl} name.\textsc{acc.pl}
say.\textsc{3sg.aor.opt}\\
\trans `But of the rest, what man of his own wit could name the names?' (Homer, \textit{Iliad} 17.260)
\label{ke13}
\end{exe}

If we cast the net more widely in Homer, we can observe that the rule recognized for \isi{subjunctive} embedded\is{subordination} clauses, that \textit{ke} should immediately follow the clause-initial word, also holds for the \isi{optative} and indicative,\is{indicative (mood)} and that in these clause types \textit{hós ke}, \textit{hoîos ke}, \textit{hóthen ke}, \textit{hóte ke}, \textit{eis hó ke}, \textit{éōs ke}, \textit{hóphra ke}, \textit{hṓs ke}, \textit{eí ke} and \textit{aí ke} belong just as tightly together as in \isi{subjunctive} clauses. The exceptions to this rule, as for other \textit{ke} clauses, are vanishingly rare: (\ref{ke14}), in which \textit{ei kaí} forms a unit similar to \textit{eíper}; cf. \textit{ei kaí min} `if and \textsc{3.acc}' in \textit{Iliad} 13.58. Also, just as with \textit{min}, several examples with \textit{ou} (\textsc{neg}): (\ref{ouke1})--(\ref{ouke4}), and perhaps some others too. Then also (\ref{Homer1256}).

\begin{exe}
\ex εἰ καί νύ κεν οἴκοθεν ἄλλο μεῖζον ἐπαιτήϲειαϲ\\
\gll ei kaí nú \emph{ken} oíkothen állo meîzon epaitḗseias\\
if and now \textsc{irr} from.home other.\textsc{n.acc.sg} greater.\textsc{n.acc.sg} ask.\textsc{2sg.aor.opt}\\
\trans `And if you should ask some other better thing from out my house ...' (Homer, \textit{Iliad} 23.592)
\label{ke14}
\end{exe}

\begin{exe}
\ex μῦθον ὃν οὔ κεν ἀνήρ γε διὰ ϲτόμα πάμπαν ἄγοιτο\\
\gll mûthon hòn oú \emph{ken} anḗr ge dià stóma pámpan ágoito\\
myth.\textsc{acc.sg} which.\textsc{m.acc.sg} not \textsc{irr} man.\textsc{nom.sg} even through mouth.\textsc{acc.sg} altogether lead.\textsc{3sg.prs.opt.pass}\\
\trans `... (this) word, that no man should in any wise suffer to pass through his mouth at all' (Homer, \textit{Iliad} 14.91)
\label{ouke1}
\end{exe}

\begin{exe}
\ex ἐπεὶ οὔ κε θανόντι περ ὧδ᾽ ἀκαχοίμην\\
\gll epeì oú \emph{ke} thanónti per hôd' akakhoímēn\\
since not \textsc{irr} die.\textsc{ptcp.aor.m.dat.sg} all thus
grieve.\textsc{1sg.aor.opt.mid}\\
\trans `For I should not so grieve for his death ...' (Homer, \textit{Odyssey} 1.236)
\label{ouke2}
\end{exe}

\begin{exe}
\ex ἐπεὶ οὔ κε κακοὶ τοιούϲδε τέκοιεν\\
\gll epeì oú \emph{ke} kakoì toioúsde tékoien\\
since not \textsc{irr} bad.\textsc{m.nom.pl} such.\textsc{m.acc.pl} beget.\textsc{3pl.aor.opt}\\
\trans `For base churls could not beget such sons as you.' (Homer, \textit{Odyssey} 4.64)
\label{ouke3}
\end{exe}

\begin{exe}
\ex τά γ᾽ οὔ κέ τιϲ οὐδὲ ἴδοιτο\\
\gll tá g' oú \emph{ké} tis oudè ídoito\\
the.\textsc{n.acc.pl} even not \textsc{irr} someone.\textsc{m.nom.sg} nor see.\textsc{3sg.aor.opt.mid}\\
\trans `... that no one could see ...' (Homer, \textit{Odyssey} 8.280)
\label{ouke4}
\end{exe}

\begin{exe}
\ex ἄλλοι τε Τρῶεϲ μέγα κεν κεχαροίατο θυμῷ\\
\gll álloi te Trôes méga \emph{ken} kekharoíato thumôi\\ 
other.\textsc{m.nom.pl} and Trojan.\textsc{nom.pl} greatly \textsc{irr} rejoice.\textsc{3pl.aor.opt.mid} spirit.\textsc{dat.sg}\\
\trans `... and the rest of the Trojans would be most glad at heart' (Homer, \textit{Iliad} 1.256)
\label{Homer1256}
\end{exe}

A much rarer exception, insofar as \textit{eí ke} is otherwise always indivisible, is (\ref{Homer5273}). But numerous editors, most recently also \citet[112, 187]{Nauck1877}, have inserted the \textit{ge} that the meaning requires. Nauck's \citeyearpar[41]{Nauck1874} emendation\is{emendation} of \textit{Odyssey} 3.219 given in (\ref{Homer3219}), with \textit{ke} as opposed to the \textit{ge} found in all the manuscripts, is all the more striking.

\begin{exe}
\ex εἰ τούτω κε λάβοιμεν, ἀροίμεθά κεν κλέοϲ ἐϲθλόν\\
\gll ei toútō \emph{ke} láboimen, aroímethá ken kléos esthlón\\
if this.\textsc{m.acc.du} \textsc{irr} take.\textsc{1pl.aor.opt}
get.\textsc{1pl.aor.opt.mid} \textsc{irr} fame.\textsc{acc.sg} goodly.\textsc{n.acc.sg}\\
\trans `Could we but take these two, we should win us goodly renown.' (Homer, \textit{Iliad} 5.273; cf. also 8.196)
\label{Homer5273}
\end{exe}

\begin{exe}
\ex ὅθεν οὐκ ἔλποιτό κε θυμῷ, ἐλθέμεν\\
\gll hóthen ouk élpoitó \emph{ke} thumôi, elthémen\\
whence not hope.\textsc{3sg.prs.opt} \textsc{irr} spirit.\textsc{dat.sg} go.\textsc{aor.inf}\\
\trans `... whence no one would hope in his heart to return' (Homer, \textit{Odyssey} 3.219)\footnote{\emph{Translator's note}: The Perseus edition has \textit{ge} for \textit{ke}, following the manuscripts and Wackernagel rather than \citet{Nauck1874}.}
\label{Homer3219}
\end{exe}\il{Greek, Homeric|)}

In\is{inscriptions|(} the inscriptions written in the dialects that possess \textit{ke}/\textit{ka}, the particle\is{particles} rarely occurs outside the aforementioned \isi{subjunctive} subordinate\is{subordination} clauses, which makes sense given the content of most of these. In Aeolic\il{Greek, Aeolic} we have a couple of examples of \textit{hṓs ke} with the \isi{optative}, and in Cypriot\il{Greek, Cypriot} the very remarkable (\ref{Idalion30}), where \textit{ke} is in second position between the article and the noun with a \isi{future} verb (cf. \citet[70, 73]{Hoffmann1891}, who recognized the right reading rather than the previously read \textit{ge}). In Argive we have (\ref{Argive}); in Korkyra we have (\ref{Korkyra}); in Epidaurian we have (\ref{Epidaurian60}) on line 60 of the large healing inscription, but line 84 (\ref{Epidaurian84}), and in Isyllus both (\ref{Isyllus26}) (line 26) \hyperlink{p374}{\emph{[p374]}} in verse\is{poetry} and (\ref{Isyllus35}) (line 35ff) in \isi{prose}.

\begin{exe}
\ex τάϲ κε ζᾶϲ τάϲδε {[}...{]} ἔξο(ν)ϲι αἰϝεί\\
\gll tás \emph{ke} zâs tásde éxo(n)si aiweí\\
the.\textsc{f.acc.pl} \textsc{irr} land.\textsc{acc.pl} this.\textsc{f.acc.pl} have.\textsc{3pl.fut} always\\
\trans `They shall have these lands forever.' (Tablet of Idalion 30)
\label{Idalion30}
\end{exe}

\begin{exe}
\ex ἇι κα δικάϲϲαιεν\\
\gll hâi \emph{ka} dikássaien\\
who.\textsc{f.nom.pl} \textsc{irr} judge.\textsc{3pl.aor.opt}\\ 
\trans (Inscription 3277.8; \citealp[127]{Prellwitz1889})
\label{Argive}
\end{exe}

\begin{exe}
\ex ἀφ᾽ οὗ κ᾽ ἀρχ(ὰ) γένοιτο\\
\gll aph' hoû \emph{k'} arkh(à) génoito\\
of which.\textsc{gen.sg} \textsc{irr} beginning.\textsc{nom.sg} become.\textsc{3sg.aor.opt.mid}\\
\trans (Inscription 3206.84; \citealp[95]{Blass1888})
\label{Korkyra}
\end{exe}

\begin{exe}
\ex αἴ κα ὑγιῆ νιν ποιήϲαι\\
\gll aí \emph{ka} hugiê nin poiḗsai\\
if \textsc{irr} healthy.\textsc{acc.sg} \textsc{3.acc} make.\textsc{3sg.aor.opt}\\
\trans `And if he would make him healthy' (Inscription 3339.60; \citealp[151--157]{Prellwitz1889})
\label{Epidaurian60}
\end{exe}

\begin{exe}
\ex τοῦτον γὰρ οὐδέ κα ὁ ἐν Ἐπιδαύρωι Ἀϲκλαπιὸϲ ὑγιῆ ποιῆϲαι δύναιτο\\
\gll toûton gàr oudé \emph{ka} ho en Epidaúrōi Asklapiòs hugiê poiêsai dúnaito\\
this.\textsc{m.acc.sg} then nor \textsc{irr} the.\textsc{m.nom.sg} in Epidaurus.\textsc{dat} Asclepius.\textsc{nom} healthy.\textsc{acc.sg} make.\textsc{aor.inf} can.\textsc{3sg.prs.opt.pass}\\
\trans `For nor could the Epidauran Asclepius heal this man' (Inscription 3339.84; \citealp[151--157]{Prellwitz1889})
\label{Epidaurian84}
\end{exe}

\begin{exe}
\ex οὕτω τοί κ᾽ ἀμῶν περιφείδοιτ᾽ εὐρύοπα Ζεύϲ\\
\gll hoútō toí \emph{k'} amôn peripheídoit' eurúopa Zeús\\
thus lo \textsc{irr} us.\textsc{gen} spare.\textsc{3sg.prs.opt.pass}
wide-eyed.\textsc{m.nom.sg} Zeus.\textsc{nom}\\
\trans `So thus might wide-eyed Zeus spare us.' (Inscription 3342.26; \citealp[162--166]{Prellwitz1889})
\label{Isyllus26}
\end{exe}

\begin{exe}
\ex ἢ λώιον οἷ κα εἴη ἀγγράφοντι τὸν παιᾶνα. Ἐμάντευϲε λώιόν οἵ κα εἶμεν ἀγγραφοντι.\\
\gll ḕ lṓion hoî \emph{ka} eíē angráphonti tòn paiâna. Emánteuse lṓión hoí ka eîmen angraphonti.\\
or better him.\textsc{dat} \textsc{irr} be.\textsc{3sg.prs.opt} engrave.\textsc{ptcp.prs.m.dat.sg} the.\textsc{m.acc.sg} paean.\textsc{acc} prophesy.\textsc{3sg.aor} better him.\textsc{dat} \textsc{irr} be.\textsc{prs.inf}
engrave.\textsc{ptcp.prs.m.dat}\\
\trans `Or it would be better for him, the engraver of the paean. It was prophesied that it would be better for him, the engraver.' (Inscription 3324.35; \citealp[162--166]{Prellwitz1889})
\label{Isyllus35}
\end{exe}

The Dodonian and Elian inscriptions furnish more examples for \textit{ka}. And here we observe that questions to the Dodonian oracle beginning with \textit{tíni theōn thúontes} or similar and ending in an \isi{optative} verb always place \textit{ka} (if they have it) immediately after \textit{tíni} `whom.\textsc{dat}' and thus separate \textit{tíni} from the nearest \isi{genitive} it governs, clear evidence of the pressure to put \textit{ka} in second position: \citet{Hoffmann1890} 1562, 1563, 1566, 1582a and 1582b, e.g. (\ref{dodonian}). Example (\ref{dodonian2})
is similar.

\begin{exe}
\ex τίνι κα θεῶν {[ἢ{]} ἡρώων θύοντεϲ καὶ εὐχ{[}ό{]}(μ)ενο(ι)
ὁμονοοῖεν ἐ{[}π{]}ὶ τὠγαθόν}\\
\gll tíni \emph{ka} theôn {[}ḕ{]} hērṓōn thúontes kaì eukh{[}ó{]}(m)eno(i) homonooîen e{[}p{]}ì tōgathón\\
whom.\textsc{m.dat.sg} \textsc{irr} god.\textsc{gen.pl} or hero.\textsc{gen.pl}
sacrifice.\textsc{ptcp.prs.m.nom.pl} and pray.\textsc{ptcp.prs.pass.m.nom.pl} agree.\textsc{3pl.prs.opt} upon the=good.\textsc{n.acc.sg}\\
\trans `By sacrificing and praying to which of the gods or heroes would they agree for good?' (Inscription 1563; \citealp{Hoffmann1890})
\label{dodonian}
\end{exe}

\begin{exe}
\ex τί κα θύϲαϲ {[}...{]}\\
\gll tí \emph{ka} thúsas\\
what.\textsc{acc} \textsc{irr} sacrifice.\textsc{ptcp.aor.m.nom.sg}\\
\trans (Inscription 1572a; \citealp{Hoffmann1890})
\label{dodonian2}
\end{exe}

When \citet[82--83]{Blass1888} emends\is{emendation} inscription 3184 (=1564) (\ref{dodonian3}) to insert the particle\is{particles} \textit{ka}, which certainly cannot have followed \textit{tínas}, at the end of a line following \textit{lṓion} `better' because it is supposedly necessary, he overlooks the fact that the Dodonian inscriptions potentially use the \isi{optative} without \textit{ka} many times, e.g. (\ref{dodonian4})--(\ref{dodonian6}).

\begin{exe}
\ex τίναϲ θεῶν ἱλαϲκόμενοϲ λώιον καὶ ἄμεινον πράϲϲοι\\
\gll tínas theôn hilaskómenos lṓion kaì ámeinon prássoi\\
whom.\textsc{m.acc.pl} god.\textsc{gen.pl} appease.\textsc{ptcp.prs.pass.m.nom.sg} better and stronger do.\textsc{3sg.prs.opt}\\
\trans `By appeasing which gods would he do better and more desirably?' (Inscription 3184 = Inscription 1564; \citealp{Hoffmann1890})
\label{dodonian3}
\end{exe}

\begin{exe}
\ex τίνι θεῶν θύουϲα λώιον καὶ ἄμεινον πράϲϲοι καὶ τᾶϲ νόϲου παύϲαιτο\\
\gll tíni theôn thúousa lṓion kaì ámeinon prássoi kaì tâs nósou paúsaito\\
whom.\textsc{m.dat.sg} god.\textsc{gen.pl} sacrifice.\textsc{ptcp.aor.f.nom.sg} better and stronger do.\textsc{3sg.prs.opt} and the.\textsc{f.gen.sg}
illness.\textsc{gen.sg} stop.\textsc{3sg.aor.opt}\\
\trans `By sacrificing to which of the gods would she do better and more desirably, and put an end to the illness?' (Inscription 1562B; \citealp{Hoffmann1890})
\label{dodonian4}
\end{exe}

\newpage
\begin{exe}
\ex ἦ μὴ ν{[α{]}(υ)κλαρῆ(ν) λώιογ καὶ ἄμεινον πράϲϲοιμι}\\
\gll ê mḕ n{[}a{]}(u)klarê(n) lṓiong kaì ámeinon prássoimi\\
in.truth not captaincy{[}?{]}.\textsc{acc} better and stronger do.\textsc{1sg.prs.opt}\\
\trans `Truly I would not carry out the captaincy better and more desirably' (Inscription 1583.2; \citealp{Hoffmann1890})
\label{dodonian5}
\end{exe}

\begin{exe}
\ex τίνα θεῶν ἢ ἡρώων τιμᾶντι λώιον καὶ ἄμεινον εἴη\\
\gll tína theôn ḕ hērṓōn timânti lṓion kaì ámeinon eíē\\
whom.\textsc{m.acc.sg} god.\textsc{gen.pl} or hero.\textsc{gen.pl}
honour.\textsc{ptcp.prs.m.dat.sg} better and stronger be.\textsc{3sg.prs.opt}\\
\trans `By honouring which of the gods or heroes would it be better and more desirable?' (Inscription 1587a; \citealp{Hoffmann1890})
\label{dodonian6}
\end{exe}

Outside this fixed formula beginning with \textit{tis} `what', however, the position of \textit{ka} in these inscriptions is free, as shown by examples (\ref{dodonian7})--(\ref{dodonian8}).

\begin{exe}
\ex ἦ τυγχάνοιμί κα\\
\gll ê tunkhánoimí \emph{ka}\\
in.truth happen.\textsc{1sg.prs.opt} \textsc{irr}\\
\trans (Inscription 1568.1; \citealp{Hoffmann1890})
\label{dodonian7}
\end{exe}

\begin{exe}
\ex {[}...{]} βέλτιομ μοί κ᾽ εἴη\\
\gll béltiom moí \emph{k'} eíē\\
better me.\textsc{dat} \textsc{irr} be.\textsc{3sg.prs.opt}\\
\trans `... would be better for me' (Inscription 1573; \citealp{Hoffmann1890})
\label{dodonian8}
\end{exe}

Among the Elian inscriptions, 1151.12, 1154.7, 1157.4 and 1158.2 must be left out of consideration because, although \textit{ka} is transmitted, its position in the sentence is not recognizable; the same holds for all examples in which \textit{ka} has been inserted, except 1151.19, in which the position of the inserted \textit{ka} can at least be determined negatively. That leaves 28 examples: 21 have \textit{ka} in second or near-second position, including (\ref{Elian1}) and (\ref{Elian2}); these 21 stand opposite 7 counterexamples.

\begin{exe}
\ex ἐν τἠπιάροι κ᾽ ἐνέχοιτο\\
\gll en tēpiároi \emph{k'} enékhoito\\
in the=sacrifice.\textsc{dat} \textsc{irr} hold.\textsc{3sg.prs.opt}\\
\trans (Inscription 1149.9 Collitz)
\label{Elian1}
\end{exe}

\begin{exe}
\ex ἐν ταῖ ζεκαμναίαι κ᾽ ἐνέχοιτο\\
\gll en taî zekamnaíai \emph{k'} enékhoito\\
in the.\textsc{f.dat.sg} ten.minae.\textsc{dat} \textsc{irr} hold.\textsc{3sg.prs.opt}\\
\trans (Inscription 1152.7 Collitz)
\label{Elian2}
\end{exe}

The import of these figures is strengthened by the composition of examples (\ref{Elian3})--(\ref{Elian7}), \hyperlink{p375}{\emph{[p375]}} in all of which \textit{ka} separates the article or an adjective\is{adjectives} from its noun. In addition there is (\ref{Elian8}), in which, although \textit{ka} is not in second position, the \isi{tmesis} nevertheless betrays a pressure to move the particle\is{particles} towards the start of the clause.

\begin{exe}
\ex τοὶ ζέ κα θεοκόλοι\\
\gll toì zé \emph{ka} theokóloi\\
the.\textsc{m.dat.sg} then \textsc{irr} priest.\textsc{m.dat.sg}\\
\trans (Inscription 1154.1 Collitz)
\label{Elian3}
\end{exe}

\begin{exe}
\ex πεντακατίαϲ κα δαρχμάϲ {[}sic{]}\\
\gll pentakatías \emph{ka} darkhmás\\
five.hundred.\textsc{f.acc.pl} \textsc{irr} drachma.\textsc{acc.pl}\\
\trans (Inscription 1154.3 Collitz)
\label{Elian4}
\end{exe}

\begin{exe}
\ex ἀ δέ κα ϝράτρα\\
\gll a dé \emph{ka} wrátra\\
the.\textsc{f.nom.sg} but \textsc{irr} agreement.\textsc{nom.sg}\\
\trans (Inscription 1156.2 Collitz)
\label{Elian5}
\end{exe}

\begin{exe}
\ex τῶν δέ κα γραφέων\\
\gll tôn dé \emph{ka} graphéōn\\
the.\textsc{gen.pl} but \textsc{irr} scribe.\textsc{gen.pl}\\
\trans (Inscription 1156.3 Collitz)
\label{Elian6}
\end{exe}

\begin{exe}
\ex ὀ {[}sic{]} δέ κα ξένοϲ\\
\gll o dé \emph{ka} xénos\\
the.\textsc{m.nom.sg} but \textsc{irr} stranger.\textsc{nom.sg}\\
\trans (Inscription 1158.1 Collitz)
\label{Elian7}
\end{exe}

\begin{exe}
\ex τῶν ζὲ προϲτιζίων οὐζέ κα μί᾽ εἴη\\
\gll tôn zè prostizíōn ouzé \emph{ka} mí' eíē\\
the.\textsc{gen} then former.\textsc{gen.pl} nor \textsc{irr}
one.\textsc{f.nom.sg} be.\textsc{3sg.prs.opt}\\
\trans `... nor would be one of the former' (Inscription 1157.7 Collitz)
\label{Elian8}
\end{exe}\is{inscriptions|)}

For the post-Homeric poets,\is{poetry} despite the sparsity of attestations, one can maintain that the rule remained in force until the end of the sixth century. The fragments of the pre-Pindarian Melic poets,\is{poetry} like those of the elegiacs\is{elegiac poets} before Theognis, yield \textit{ke}/\textit{ka} only in second position (see in particular also (\ref{Xenophanes210})). 

\begin{exe}
\ex ταῦτά χ᾽ ἅπαντα λάχοι\\
\gll taûtá \emph{kh'} hápanta lákhoi\\
this.\textsc{n.acc.pl} \textsc{irr} quite.all.\textsc{n.acc.pl} obtain.\textsc{3sg.aor.opt}\\
\trans `All these things would fall to him' (Xenophanes 2.10)
\label{Xenophanes210}
\end{exe}

Sappho Fragment 66 ((\ref{Sappho66})) is poorly attested; \citet[177]{Bergk1882} writes Alcaeus 83 as (\ref{Alcaeus83}), but neither \textit{autós} `same' nor \textit{ke} is attested. It will now be necessary to seek other ways to improve this sentence.

\begin{exe}
\ex ὀ δ᾽ Ἄρευϲ φαῖϲί κεν Ἄφαιϲτον ἄγην\\
\gll o d' Áreus phaîsí \emph{ken} Áphaiston ágēn\\
the.\textsc{m.nom.sg} then Ares.\textsc{nom} say.\textsc{3sg.prs} \textsc{irr} Hephaestus.\textsc{acc} lead.\textsc{prs.inf}\\
\trans `And Ares says that he would bring Hephaestus' (Sappho, Fragment 66)
\label{Sappho66}
\end{exe}

\begin{exe}
\ex αἴ κ᾽ εἴπῃϲ, τὰ θέλειϲ, \textless{}αὐτὸϲ\textgreater{} ἀκούϲαιϲ \textless{}κε\textgreater{}, τά κ᾽ οὐ θέλοιϲ\\
\gll aí \emph{k'} eípēis, tà théleis, \textless{}autòs\textgreater{} akoúsais \textless{}\emph{ke}\textgreater{}, tá \emph{k'} ou thélois\\
if \textsc{irr} say.\textsc{2sg.aor.sbjv} the.\textsc{n.acc.pl} want.\textsc{2sg.prs} same.\textsc{m.nom.sg} hear.\textsc{2sg.aor.opt} \textsc{irr} the.\textsc{n.acc.pl} \textsc{irr} not want.\textsc{2sg.prs.opt}\\
\trans `If you said what you want, you yourself would hear what you would not want' (Alcaeus, Fragment 83)
\label{Alcaeus83}
\end{exe}

Then it is clear that the Theognideian gnomic poems,\is{poetry} Pindar and Epicharmus deviate from the old norm: Theognis (in addition to instances such as (\ref{Theognis900})) 645, 653, 747, 765; many examples in Pindar; Epicharmus (against normal usage \citealp[223]{Lorenz1864} Busiris fragment 1, \citeyearpar[264]{Lorenz1864} fragment 33.1, and \citeyearpar[267]{Lorenz1864} verse 12) fragment 7.1, \citet[257]{Lorenz1864}; \citeyearpar[267]{Lorenz1864} verse 9; \citeyearpar[268]{Lorenz1864} verse 16; \citeyearpar[269]{Lorenz1864} verse 11; \citeyearpar[274]{Lorenz1864} fragment 53; verse 167 in \citet[141]{Mullach1860}; for which one can let the question of the genuineness of the individual examples rest.

\begin{exe}
\ex μέγα κεν πῆμα βροτοῖϲιν ἐπῆν\\
\gll méga ken pêma brotoîsin epên\\
great.\textsc{n.nom.sg} \textsc{irr} harm.\textsc{nom.sg} mortal.\textsc{dat.pl} be.upon.\textsc{3sg.imp}\\
\trans `... a great calamity would be at hand for mortals.' (Theognis, \textit{Elegies} 900)
\label{Theognis900}
\end{exe}

Of the remaining enclitic\is{enclitics} \isi{particles} \textit{thēn} `surely', \textit{nu} `now' and \textit{toi} `certainly', in Homer\il{Greek, Homeric} \textit{thḗn} is always found in second position (naturally including (\ref{then1}) and (\ref{then2})); the same is true in (\ref{then3}); the same is true of Theocritus in the inherited phrases (\ref{then4}) (cf. Aeschylus in example (\ref{then3})) and \textit{kaì gár thēn} in 6.34 (cf. (\ref{then1}) from Homer), as well as in (\ref{then5}) and (\ref{then6}).

\begin{exe}
\ex καὶ γάρ θην\\
\gll kaì gár \emph{thēn}\\
and for surely\\
\trans (Homer, \textit{Iliad} 21.568)
\label{then1}
\end{exe}

\begin{exe}
\ex οὐ μέν θην\\
\gll ou mén \emph{thēn}\\
not then surely\\
\trans (Homer, \textit{Iliad} 8.448)
\label{then2}
\end{exe}

\begin{exe}
\ex ϲύ θην ἃ χρῄζειϲ, ταῦτ᾽ ἐπιγλωϲϲᾷ Διόϲ\\
\gll sú \emph{thēn} hà khrḗizeis, taût' epiglōssâi Diós\\
you.\textsc{nom} surely what.\textsc{acc.pl} want.\textsc{2sg.prs}
this.\textsc{n.acc.pl} reproach.\textsc{2sg.prs.pass} Zeus.\textsc{gen}\\
\trans `Surely, it is only your own desire that you utter as a curse against Zeus.' (Aeschylus, \textit{Prometheus Bound} 928)
\label{then3}
\end{exe}

\begin{exe}
\ex τύ θην\\
\gll tú \emph{thēn}\\
you.\textsc{nom} surely\\
\trans (Theocritus 1.97 and 7.83)
\label{then4}
\end{exe}

\begin{exe}
\ex αἶνόϲ θην\\
\gll aînós \emph{thēn}\\
fable.\textsc{nom} surely\\
\trans (Theocritus 14.43)
\label{then5}
\end{exe}

\begin{exe}
\ex πείρᾳ θην\\
\gll peírāi \emph{thēn}\\
attempt.\textsc{dat.sg} surely\\
\trans (Theocritus 15.62)
\label{then6}
\end{exe}

Theocritus broke the rule twice (2.114, 5.111); before him also Epicharmus ((\ref{then7})).

\begin{exe}
\ex καίτοι νῦν γά θην εὔωνον αἰνεῖ ϲῖτον\\
\gll kaítoi nûn gá \emph{thēn} eúōnon aineî sîton\\
and.yet now even surely cheap.\textsc{m.acc.sg} praise.\textsc{3sg.prs} bread.\textsc{acc.sg}\\
\trans `Yet now, surely, he at least praises cheap bread.' (Epicharmus in Athenaeus 6.28; \citealp[226]{Lorenz1864} verse 2)\footnote{\emph{Translator's note}: The Perseus edition has \textit{kàt tò ... aeí}.}
\label{then7}
\end{exe}

\textit{nu} and \textit{nun} `now' in Homer\il{Greek, Homeric|(} are almost always in second position, if we go by the remark of \citet{Ebeling} on this word: ``as the particle\is{particles} is enclitic,\is{enclitics} it attaches itself to whatever is the most important word''. I do not consider (\ref{nun1}) to be a counterexample.

\begin{exe}
\ex καὶ γὰρ δή νύ ποτε Ζεὺϲ ἄϲατο\\
\gll kaì gàr dḗ \emph{nú} pote Zeùs ásato\\
and then exactly now sometime Zeus.\textsc{nom} mislead.\textsc{3sg.aor.mid}\\
\trans `Aye, and on a time she blinded Zeus' (Homer, \textit{Iliad} 19.95)
\label{nun1}
\end{exe}

By contrast, it is striking \hyperlink{p376}{\emph{[p376]}} that \textit{nu} regularly precedes other \isi{enclitics} like \textit{moi}, \textit{toi}, \textit{hoi}, \textit{se}, \textit{tis}, \textit{ti}, \textit{pote}, \textit{pou} (though (\ref{nun2})), \textit{per} and \textit{ken}, and is only preceded by \textit{dé}: we also see \textit{nù gár} `now then' in \textit{Iliad} 13.257 next to \textit{gár nu} `then now' in \textit{Odyssey} 15.239 and \textit{gár dḗ nu} `then exactly now' in \textit{Iliad} 19.95.

\begin{exe}
\ex ὅϲα πού νυν ἐέλπεται\\
\gll hósa poú \emph{nun} eélpetai\\
as.much.\textsc{n.acc.pl} somewhere now hope.\textsc{3sg.prs.pass}\\
\trans `... even all that now he thinks' (Homer, \textit{Iliad} 10.105)
\label{nun2}
\end{exe}

It is also striking that it often separates, or assists in separating, close connections: adjective\is{adjectives} and noun ((\ref{nun6})--(\ref{nun8})); article and noun ((\ref{nun9})--(\ref{nun10})); preposition\is{prepositions} and noun ((\ref{nun11})). The only rule-breaking example, as far as I can see, is ((\ref{nun12})).

\begin{exe}
\ex ἠπεδανὸϲ δέ νύ τοι θεράπων\\
\gll ēpedanòs dé \emph{nú} toi therápōn\\
weakly.\textsc{m.nom.sg} but now you.\textsc{dat} attendant.\textsc{nom.sg}\\
\trans `... and your squire is a weakling' (Homer, \textit{Iliad} 8.104)
\label{nun6}
\end{exe}

\begin{exe}
\ex θαρϲαλέον νύ τοι ἦτορ ἐνὶ φρεϲίν\\
\gll tharsaléon \emph{nú} toi êtor enì phresín\\
confident.\textsc{n.nom.sg} now you.\textsc{dat} heart.\textsc{nom.sg} in midriff.\textsc{dat.pl}\\
\trans `Your heart within you is of good cheer' (Homer, \textit{Iliad} 19.169)\footnote{\emph{Translator's note}: The Perseus edition has \textit{hoi} for \textit{toi}.}
\label{nun7}
\end{exe}

\begin{exe}
\ex ϲιδήρειόν νύ τοι ἦτορ\\
\gll sidḗreión \emph{nú} toi êtor\\
iron.\textsc{n.nom.sg} now you.\textsc{dat} heart.\textsc{nom.sg}\\
\trans `Of iron truly is your heart.' (Homer, \textit{Iliad} 24.205 = \textit{Iliad} 24.521)
\label{nun8}
\end{exe}

\begin{exe}
\ex οἱ δέ νυ λαοὶ θνῆϲκον\\
\gll hoi dé \emph{nu} laoì thnêskon\\
the.\textsc{m.nom.pl} but now people.\textsc{nom.pl}
die.\textsc{3pl.imp}\\
\trans `Then the people began to die' (Homer, \textit{Iliad} 1.382)
\label{nun9}
\end{exe}

\begin{exe}
\ex ἡ δέ νυ μήτηρ τίλλε κόμην\\
\gll hē dé \emph{nu} mḗtēr tílle kómēn\\ 
the.\textsc{f.nom.sg} but now mother.\textsc{nom.sg} pluck.\textsc{3sg.imp} hair.\textsc{acc.sg}\\
\trans `But his mother tore her hair' (Homer, \textit{Iliad} 22.405)
\label{nun10}
\end{exe}

\begin{exe}
\ex ἀντί νυ πολλῶν λαῶν ἐϲτὶν ἀνήρ\\
\gll antí \emph{nu} pollôn laôn estìn anḗr\\
against now many.\textsc{m.gen.pl} people.\textsc{gen.pl} be.\textsc{3sg.prs} man.\textsc{nom.sg}\\
\trans `Of the worth of many hosts is the man ...' (Homer, \textit{Iliad} 9.116)
\label{nun11}
\end{exe}

\begin{exe}
\ex ὡϲ δὴ ἔγωγ᾽ ὄφελον μάκαρόϲ νύ τευ ἔμμεναι υἱὸϲ ἀνέροϲ\\
\gll hōs dḕ égōg' óphelon mákarós \emph{nú} teu émmenai huiòs anéros\\
as exactly I.\textsc{nom.emph} owe.\textsc{1sg.aor} blessed.\textsc{m.gen.sg} now some.\textsc{m.gen.sg} be.\textsc{prs.inf} son.\textsc{nom.sg} man.\textsc{gen.sg}\\
\trans `Ah, would that I had been the son of some blessed man' (Homer, \textit{Odyssey} 1.217)
\label{nun12}
\end{exe}\il{Greek, Homeric|)}

For post-Homeric\il{Greek, Classical|(} usage I refer to \textit{phére nun}, \textit{áge nun} ((\ref{nun13})), \textit{mḗ nun}, and to the \textit{mén nun} so often found in second position in Herodotus, and finally to (\ref{nun14})--(\ref{nun18}). Cf. also \citet[475]{Lobeck1835} on \textit{Ajax} verse 1332.

\begin{exe}
\ex ἄγε νυν\\
\gll áge \emph{nun}\\
lead.\textsc{2sg.prs.imper} now\\
\trans `Come now!' (Aristophanes, \textit{Peace} 1056)
\label{nun13}
\end{exe}

\begin{exe}
\ex πρόϲ νύν ϲε πατρὸϲ πρόϲ τε μητρόϲ {[}...{]} ἱκέτηϲ ἱκνοῦμαι\\
\gll prós \emph{nún} se patròs prós te mētrós hikétēs hiknoûmai\\
to now you.\textsc{acc} father.\textsc{gen.sg} to and mother.\textsc{gen.sg} suppliant.\textsc{nom.sg} beseech.\textsc{1sg.pass}\\
\trans `Now by your father and by your mother, I beseech you as a suppliant.' (Sophocles, \textit{Philoctetes} 468)
\label{nun14}
\end{exe}

\begin{exe}
\ex πρόϲ νύν ϲε κρηνῶν καὶ θεῶν ὁμογνίων αἰτῶ πιθέϲθαι\\
\gll prós \emph{nún} se krēnôn kaì theôn homogníōn aitô pithésthai\\
to now you.\textsc{acc} spring.\textsc{gen.pl} and god.\textsc{gen.pl}
akin.\textsc{m.gen.pl} ask.\textsc{1sg.prs} persuade.\textsc{aor.inf.mid}\\
\trans `Then, by the streams of water and gods of our race, I ask you to listen' (Sophocles, \textit{Oedipus at Colonus} 1333)
\label{nun15}
\end{exe}

\begin{exe}
\ex πρόϲ νύν ϲε γονάτων τῶνδ(ε)\\
\gll prós \emph{nún} se gonátōn tônd(e)\\
to now you.\textsc{acc} knee.\textsc{gen.pl} this.\textsc{n.gen.pl}\\
\trans `Now, by your knees ...' (Euripides, \textit{Helen} 1237)
\label{nun16}
\end{exe}

\begin{exe}
\ex ἀπό νυν με λείπετ᾽ ἤδη\\
\gll apó \emph{nun} me leípet' ḗdē\\
of now me.\textsc{acc} leave.\textsc{2pl.prs.imper} already\\
\trans `Leave me then, immediately' (Sophocles, \textit{Philoctetes} 1177)
\label{nun17}
\end{exe}

\begin{exe}
\ex μετά νυν δόϲ\\
\gll metá \emph{nun} dós\\
after now give.\textsc{2sg.aor.imper}\\
\trans `Then share ...' (Euripides, \textit{Suppliants} 56)
\label{nun18}
\end{exe}\il{Greek, Classical|)}

In Cypriot,\il{Greek, Cypriot}\is{inscriptions|(} the position of \textit{nu} is freer: (\ref{nun19})--(\ref{nun20}). The same is true in Boeotian: (\ref{nun21}) (equivalent to Attic\il{Greek, Attic} \textit{kaì hai huperēmériai ákuroi éstōn}). It seems highly doubtful to me that the Cypriot\il{Greek, Cypriot} words \textit{hónu} ``this.\textsc{m.nom}'', \textit{tónu} ``this.\textsc{m.acc}'' and Arcadian \textit{tánu} ``this.\textsc{f.acc}'' contain the particle\is{particles} \textit{nu}. It is more likely to be the \textit{u} of \textit{hoûtos} ``this''; cf. Arcadian \textit{tōní}, \textit{tanní}.

\begin{exe}
\ex ἢ δυϝάνοι νυ\\
\gll ḕ duwánoi \emph{nu}\\
or give.\textsc{3sg.aor.opt} now\\
\trans (Idalion 6)
\label{nun19}
\end{exe}

\begin{exe}
\ex ἢ δώκοι νυ\\
\gll ḕ dṓkoi \emph{nu}\\
or give.\textsc{3sg.aor.opt} now\\
\trans (Idalion 16)
\label{nun20}
\end{exe}

\begin{exe}
\ex κὴ τὴ οὑπεραμερίη ἄκουρύ νυ ἔνθω\\
\gll kḕ tḕ houperameríē ákourú \emph{nu} énthō\\
and the.\textsc{f.nom.pl} default.\textsc{nom.pl} invalid.\textsc{f.nom.pl} now be.\textsc{3pl.prs.imper}\\
\trans `And let the overdue amounts now be annulled.' (Inscription 488.88; \citealp[183]{Meister1884})
\label{nun21}
\end{exe}\is{inscriptions|)}

Finally a word on \textit{toi}, insofar as it has become a pure particle\is{particles} for which positioning according to our rule is generally recognized: cf. \textit{kaítoi} and \textit{méntoi}. Here we have 1) \isi{tmesis}: (\ref{toi4}), as well as examples (\ref{tmesis29}) and (\ref{tmesis31}) cited above. 

\begin{exe}
\ex ἔκ τοι πέπληγμαι\\
\gll ék \emph{toi} péplēgmai\\
out lo strike.\textsc{1sg.prf.pass}\\
\trans `Surely I am stunned' (Euripides, \textit{Heracleidae} 1105)
\label{toi4}
\end{exe}

2) (\ref{toi5}); also, with \textit{gár toí} `then lo', examples (\ref{toi6})--(\ref{toi8}).

\begin{exe}
\ex διά τοι ϲὲ πόνουϲ ἔχω\\
\gll diá \emph{toi} sè pónous ékhō\\
through lo you.\textsc{acc} trouble.\textsc{acc.pl} have.\textsc{1sg.prs}\\
\trans `Because of you I have these pangs' (Aristophanes, \textit{Ecclesiazusae} 975)
\label{toi5}
\end{exe}

\begin{exe}
\ex ἐν γάρ τοι πόλει ὧδε κακοψόγῳ ἁνδάνει οὐδέν\\
\gll en gár \emph{toi} pólei hôde kakopsógōi handánei oudén\\
in for lo city.\textsc{dat.sg} thus censorious.\textsc{f.dat.sg}
please.\textsc{3sg.prs} nothing.\textsc{nom.sg}\\
\trans `For nothing is pleasant in a censorious city.' (Theognis, \textit{Elegies} 287)
\label{toi6}
\end{exe}

\begin{exe}
\ex περὶ γάρ τοι τῶν ποιημάτων\\
\gll perì gár \emph{toi} tôn poiēmátōn\\
about for lo the.\textsc{n.gen.pl} poem.\textsc{gen.pl}\\
\trans `For about the poems ...' (Plato, \textit{Phaedo} 60c)
\label{toi7}
\end{exe}

\begin{exe}
\ex περὶ γάρ τοι γῆϲ {[}...{]} πολλὰ ἀκήκοα\\
\gll perì gár \emph{toi} gês pollà akḗkoa\\
about for lo earth.\textsc{gen.sg} much.\textsc{n.pl} hear.\textsc{1sg.prf}\\
\trans `For I have heard many things about the earth.' (Plato, \textit{Phaedo} 108d)
\label{toi8}
\end{exe}

3) (\ref{toi9})--(\ref{toi12}); also, with \textit{gár toí} `then lo', examples (\ref{toi13})--(\ref{toi14}).

\begin{exe}
\ex ὦ παῖδεϲ, ἥ τοι Κύπριϲ οὐ Κύπριϲ μόνον\\
\gll ô paîdes, hḗ \emph{toi} Kúpris ou Kúpris mónon\\
O child.\textsc{voc.pl} the.\textsc{f.nom.sg} lo Cypris.\textsc{nom} not Cypris.\textsc{nom} alone\\
\trans `You see, children, Cypris is not just Cypris.' (Sophocles, Fragment 855.1)
\label{toi9}
\end{exe}

\begin{exe}
\ex τήν τοι Δίκην λέγουϲι παῖδ᾽ εἶναι Χρόνου\\
\gll tḗn \emph{toi} Díkēn légousi paîd' eînai Khrónou\\
the.\textsc{f.acc.sg} lo Justice.\textsc{acc.sg} say.\textsc{3pl.prs} child.\textsc{acc.sg} be.\textsc{prs.inf} Time.\textsc{gen.sg}\\
\trans `They say that Justice is the child of Time.' (Euripides, Fragment 222)
\label{toi10}
\end{exe}

\begin{exe}
\ex οἵ τοι γεωργοὶ τοὖργον ἐξέλκουϲι\\
\gll hoí \emph{toi} geōrgoì toûrgon exélkousi\\
the.\textsc{m.nom.pl} lo farmer.\textsc{nom.pl} the=work.\textsc{acc.sg} extract.\textsc{3pl.prs}\\
\trans `The husbandmen are doing the work.' (Aristophanes, \textit{Peace} 511)
\label{toi11}
\end{exe}

\begin{exe}
\ex ἥ τοι τῆϲ διανοίαϲ ὄψιϲ\\
\gll hḗ \emph{toi} tês dianoías ópsis\\
the.\textsc{f.nom.sg} lo the.\textsc{f.gen.sg} intellect.\textsc{gen.sg} sight.\textsc{nom.sg}\\
\trans `The intellectual sight ...' (Plato, \textit{Symposium} 219a)
\label{toi12}
\end{exe}

\hyperlink{p377}{\emph{[p377]}}

\begin{exe}
\ex τὸ γάρ τοι πρᾶγμα ϲυμφορὰν ἔχει\\
\gll tò gár \emph{toi} prâgma sumphoràn ékhei\\
the.\textsc{n.nom.sg} for lo deed.\textsc{nom.sg} mishap.\textsc{acc.sc} have.\textsc{3sg.prs}\\
\trans `This matter is surely an unfortunate one.' (Euripides, \textit{Helen} 93)
\label{toi13}
\end{exe}

\begin{exe}
\ex τὸ γάρ τοι θάνατον δεδιέναι\\
\gll tò gár \emph{toi} thánaton dediénai\\
the.\textsc{n.nom.sg} for lo death.\textsc{acc.sg}
fear.\textsc{prf.inf}\\
\trans `The fear of death...' (Plato, \textit{Apology} 29a)
\label{toi14}
\end{exe}

4) Examples (\ref{toi15})--(\ref{toi22}) etc.

\begin{exe}
\ex τοιοῦτόϲ τοι ἑταῖροϲ ἀνὴρ φίλοϲ\\
\gll toioûtós \emph{toi} hetaîros anḕr phílos\\
such.\textsc{m.nom.sg} lo companion.\textsc{nom.sg} man.\textsc{nom.sg} dear.\textsc{m.nom.sg}\\
\trans `Such a man (is) a dear companion.' (Theognis, \textit{Elegies} 95; cf. Bergk's \textit{hetaírōi})
\label{toi15}
\end{exe}

\begin{exe}
\ex πολλῷ τοι πλέοναϲ λιμοῦ κόροϲ ὤλεϲεν ἤδη ἄνδραϲ\\
\gll pollôi \emph{toi} pléonas limoû kóros ṓlesen ḗdē ándras\\
much.\textsc{dat.sg} lo more.\textsc{m.acc.sg} hunger.\textsc{gen.sg} surfeit.\textsc{nom.sg} destroy.\textsc{3sg.aor} already men.\textsc{acc.pl}\\
\trans `At present excess has ruined far more men than hunger.' (Theognis, \textit{Elegies} 605)
\label{toi16}
\end{exe}

\begin{exe}
\ex διϲϲαί τοι πόϲιοϲ κῆρεϲ δειλοῖϲι βροτοῖϲιν\\
\gll dissaí \emph{toi} pósios kêres deiloîsi brotoîsin\\
double.\textsc{f.nom.pl} lo drink.\textsc{gen.sg} doom.\textsc{nom.pl}
wretched.\textsc{m.dat.pl} mortal.\textsc{dat.pl}\\
\trans `The perils of drink are twofold for wretched mortals.' (Theognis, \textit{Elegies} 837)
\label{toi17}
\end{exe}

\begin{exe}
\ex πολλοί τοι κίβδηλοι {[}...{]} κρύπτουϲ(ι)\\
\gll polloí \emph{toi} kíbdēloi krúptous(i)\\
many.\textsc{m.nom.pl} lo base.\textsc{m.nom.pl} hide.\textsc{3pl.prs}\\
\trans `Many false men hide ...' (Theognis, \textit{Elegies} 965)\footnote{\emph{Translator's note}: The Teubner ed. \citep{Hiller1890} has \textit{kíbdēlon}.}
\label{toi18}
\end{exe}

\begin{exe}
\ex ῥηιδίη τοι πρῆξιϲ ἐν ἀνθρώποιϲ κακότητοϲ\\
\gll rhēidíē \emph{toi} prêxis en anthrṓpois kakótētos\\
easy.\textsc{f.nom.sg} lo practice.\textsc{nom.sg} in person.\textsc{dat.pl} badness.\textsc{gen.sg}\\
\trans `The practice of evil is easy for people.' (Theognis, \textit{Elegies} 1027)
\label{toi19}
\end{exe}

\begin{exe}
\ex δειλῶν τοι κραδίη γίγνεται ὀξυτέρη\\
\gll deilôn \emph{toi} kradíē gígnetai oxutérē\\ 
wretched.\textsc{m.gen.pl} lo heart.\textsc{nom.sg} become.\textsc{3sg.prs.pass} sharper.\textsc{f.nom.sg}\\
\trans `The heart of the wretched becomes sharper.' (Theognis, \textit{Elegies} 1030)
\label{toi20}
\end{exe}

\begin{exe}
\ex Δία τοι ξένιον μέγαν αἰδοῦμαι\\
\gll Día \emph{toi} xénion mégan aidoûmai\\
Zeus.\textsc{acc} lo hospitable.\textsc{m.acc.sg} great.\textsc{m.acc.sg} revere.\textsc{1sg.prs.pass}\\
\trans `I revere great Zeus, protector of guests' (Aeschylus, \textit{Agamemnon} 363)
\label{toi21}
\end{exe}

\begin{exe}
\ex ἀμήχανόν τοι κάλλοϲ\\
\gll amḗkhanón \emph{toi} kállos\\
immense.\textsc{n.acc.sg} lo beauty.\textsc{acc.sg}\\
\trans `immense beauty'
(Plato, \textit{Symposium} 218e; cf. also Euripides, \textit{Orestes} 1167)
\label{toi22}
\end{exe}

Attic\il{Greek, Attic} \textit{toigártoí} is also a sign of the particle's\is{particles} forward movement. In Homer,\il{Greek, Homeric} \textit{toigártoí} does not yet occur. In its place we have several instances of (\ref{toigartoi}) (or another \isi{future} verb), where it is easy to punctuate after \textit{toigár}: ``because it is so (\textit{toí} = instrumental \textit{tṓ} + \textit{i}?), ...''. 
\begin{exe}
\ex τοιγὰρ ἐγώ τοι {[}...{]} καταλέξω\\
\gll toigàr egṓ toi kataléxō\\
therefore I lo tell.\textsc{1sg.fut}\\
\trans `Therefore I will tell ...'
\label{toigartoi}
\end{exe}

In the post-Homeric period, \textit{toi} -- and also \textit{oûn} -- was attached directly to \textit{toigár}; \textit{toigártoí} is to \textit{toigár ... toí} as \ili{Latin} \textit{utrumne} is to \textit{utrum ... ne} (see below p\pageref{utrumne}).


\section{Postpositive particles: \emph{án} in subordinate clauses}\is{particles|(}\is{subordination|(}\is{postpositive particles|(}

Similar to the \isi{enclitics} is a group of words that \citet{Krueger1871} %not 100% sure this is the one
appropriately calls postpositive particles, because they are just as incapable of appearing clause-initially as the \isi{enclitics}: \emph{án}, \emph{ár}, \emph{ára}, \emph{aû}, \emph{gár}, \emph{dé}, \emph{dêta}, \emph{mén}, \emph{mḗn}, \emph{oûn}, \emph{toínun}. Investigating the origins of this similarity is not the goal of my investigation. However, various factors appear to come into consideration: one of these particles, \emph{aû} `again, further' could have originally been a true enclitic,\is{enclitics} since it corresponds to the \ili{Sanskrit} \emph{u}, as I maintain against \citet[364]{Kretschmer1892}. Then, \emph{toínun} `therefore' is composed of two \isi{enclitics} \emph{toi} `lo' and \emph{nun} `now'. The original, however, was, for example, \emph{autós toí nun} `self/same lo now'. It cannot be established how long \emph{autòs toínun} `self/same therefore' has been in use. For others it is conceivable that they were initially in general use postpositively, just like \ili{Latin} \emph{enim} `namely' and, following this example, \emph{namque} `for/since' (\emph{itaque} `therefore/and so' following \emph{igitur} `therefore'). It is difficult to thus distinguish \emph{án} from the \ili{Latin} and \ili{Gothic} question particle \emph{an}, which in both languages is prepositive. It seems plausible to say that in Greek the particle was drawn away \hyperlink{p378}{\emph{[p378]}} from the first position in the clause and became postpositive under the influence of \textit{ke} (\textsc{irr}), with which it had become identical in meaning.\is{semantic change} Before our very eyes a similar change is happening with \emph{dḗ} `now/truly/exactly', which can introduce a clause in the language of Homer\il{Greek, Homeric} and the poets\is{poetry} who follow his style, but which is already becoming decisively postpositive in Homer's writings and is exclusively postpositive in \isi{prose}.

But for both types of particles -- those that were enclitic\is{enclitics} from the start, like \emph{aû}, and those that became postpositive under the influence of an enclitic,\is{enclitics} like \emph{án} -- the question arises of whether they participate in the special positional rule for \isi{enclitics} that has been established through our investigation. For those that serve as sentence connectives -- in fact, for all but \emph{án} -- it is recognized that they do so, and well known that, just like the actual \isi{enclitics}, they are able to induce \isi{tmesis} and similar, e.g. (\ref{SophAnt601}) and (\ref{EurHer1085}).

\begin{exe}
\ex κατ᾽ αὖ νιν φοινία θεῶν τῶν νερτέρων ἀμᾷ κοπίϲ\\
\gll kat' \emph{aû} nin phoinía theôn tôn nertérōn amâi kopís\\
down again \textsc{cl} bloody.\textsc{f.nom.sg} god.\textsc{gen.pl}
the.\textsc{m.gen.pl} nether.\textsc{gen.pl} mow.\textsc{3sg.prs} dust.\textsc{nom.sg}\\
\trans `The blood-stained dust of the infernal gods cuts it down again' (Sophocles, \textit{Antigone} 601)
\label{SophAnt601}
\end{exe}

\begin{exe}
\ex ἀν᾽ αὖ βακχεύϲει Καδμείων πόλιν\\
\gll an' \emph{aû} bakkheúsei Kadmeíōn pólin\\
up again riot.\textsc{3sg.fut} Cadmean.\textsc{m.gen.pl} city.\textsc{acc.sg}\\
\trans `He will run riot again through the Cadmeans' city.' (Euripides, \textit{Heracleidae} 1085)
\label{EurHer1085}
\end{exe}

\emph{oûn} `then' often occurs between preposition\is{prepositions} and case, or between article and noun. \emph{dé} `but/and' does this quite regularly, and with this word the rule is at its most effective, since it takes precedence over all \isi{enclitics} and \isi{enclitoids} and only extremely rarely takes third position. For the other particles, the rule is subject to certain restrictions: \textit{ára} `so/then', for instance, follows the verb, e.g. (\ref{HomIl5748}), (\ref{Hdt4454}). 

\begin{exe}
\ex Ἥρη δὲ μάϲτιγι θοῶϲ ἐπεμαίετ᾽ ἄρ᾽ ἵππουϲ\\
\gll Hḗrē dè mástigi thoôs epemaíet' \emph{ár'} híppous\\
Hera.\textsc{nom} but whip.\textsc{dat.sg} quickly touch.\textsc{3sg.imp.pass} then horse.\textsc{acc.pl}\\
\trans `And Hera swiftly touched the horses with the lash.' (Homer, \textit{Iliad} 5.748)
\label{HomIl5748}
\end{exe}

\begin{exe}
\ex πρότερον δὲ ἦν ἄρα ἀνώνυμοϲ\\
\gll próteron dè ên \emph{ára} anṓnumos\\
before but be.\textsc{3sg.imp} then anonymous.\textsc{m.nom.sg}\\
\trans `Before, it was apparently nameless.' (Herodotus 4.45.4)
\label{Hdt4454}
\end{exe}

\emph{oûn} `then' is often attracted by the preposition\is{prepositions} connected to a verb, and then occurs between it and the verb. This is found particularly often in Herodotus and Hippocrates: (\ref{Hipponax61})--(\ref{Ath1034}). The position of \emph{dḗ} `now/truly/exactly' is very free.

\begin{exe}
\ex ἑϲπέρηϲ καθεύδοντα ἀπ᾽ οὖν ἔδυϲε\\
\gll hespérēs katheúdonta ap' \emph{oûn} éduse\\
evening.\textsc{gen.sg} sleep.\textsc{ptcp.prs.m.acc.sg} of so clothe.\textsc{3sg.aor}\\
\trans `In the evening he undresses the one going to bed.' (Hipponax (?), Fragment 61)
\label{Hipponax61}
\end{exe}

\begin{exe}
\ex τήνῳ κυδάζομαί τε κἀπ᾽ ὦν ἠχθόμαν\\
\gll tḗnōi kudázomaí te kap' \emph{ôn} ēkhthóman\\
that.\textsc{m.dat.sg} revile.\textsc{1sg.prs.pass} and and=of so grieve.\textsc{1sg.imp.pass}\\
\trans `Then I revile him and am vexed.' (Epicharmus in Athenaeus 6.28)
\label{Ath628}
\end{exe}

\begin{exe}
\ex τάχα δὴ τάχα τοὶ μὲν ἀπ᾽ ὦν ὄλοντο\\
\gll tákha dḕ tákha toì mèn ap' \emph{ôn} ólonto\\
quickly exactly quickly lo then of so destroy.\textsc{3pl.aor.mid}\\
\trans `So they are ruined quickly, quickly.' (Melanippides in Athenaeus 10.34)\footnote{\emph{Translator's note}: The Perseus edition has \textit{oûn apōllúonto}.}
\label{Ath1034}
\end{exe}

\emph{án} has a special position. \citet[7]{Hermann1831} tells us ``Given that \textit{án} is not enclitic,\is{enclitics} but that it nevertheless cannot be placed in first position, it is clear that it must be placed after one of those words whose meaning it contributes to'', and sharply contrasts \emph{án} with \textit{ke}. According to \citeauthor{Hermann1831}, the difference between the two can be observed as early as the works of Homer,\il{Greek, Homeric} based on the examples \hyperlink{p379}{\emph{[p379]}} \textit{Iliad} 7.125 \textit{ê \emph{ke} még' oimṓxeie} ((\ref{ke1}) above), in which \textit{ke} immediately follows \textit{ê}, and (\ref{HomIl2220}), in which \emph{án} attaches to the second word, \textit{se}. This difference between \emph{án} and \textit{ken} is surprising. If the assumption that \emph{án} became postpositive under the influence of \textit{ke} is correct, then we should expect the position of \emph{án} to be no different from that of \textit{ken}.

\begin{exe}
\ex ἦ ϲ᾽ ἂν τιϲαίμην\\
\gll ê s' \emph{àn} tisaímēn\\
in.truth you.\textsc{acc} \textsc{irr} pay\textsc{.1sg.aor.opt.mid}\\
\trans `Verily I would avenge me on thee' (Homer, \textit{Iliad} 22.20)
\label{HomIl2220}
\end{exe}

Does the distinction reported by \citeauthor{Hermann1831} really exist, though? At any rate, it is not found in an extensive category of clauses, namely subordinate clauses with a \isi{subjunctive} verb. For here immediate attachment to the clause-initial word is just as much the rule for \emph{án} as it is for \textit{ke}(\textit{n}). In this context \textit{hóstis} `who.\textsc{m.nom.sg}' is counted as a single unitary word, as is \textit{hopoîós tis}: (\ref{PlatPhaedo81e}), (\ref{XenWays1}).

\begin{exe}
\ex ὁποῖ᾽ ἄττ᾽ ἂν καὶ μεμελετηκυῖαι τύχωϲι\\
\gll hopoî' átt' \emph{àn} kaì memeletēkuîai túkhōsi\\
of.what.sort.\textsc{n.nom.pl} whatever.\textsc{n.acc.pl} \textsc{irr} also
practise.\textsc{ptcp.prf.f.dat.sg} happen.\textsc{3pl.aor.sbjv}\\
\trans `... which correspond to the practices ...' (Plato, \textit{Phaedo} 81e)
\label{PlatPhaedo81e}
\end{exe}

\begin{exe}
\ex ὁποῖοί τινεϲ ἂν οἱ προϲτάται ὦϲι\\
\gll hopoîoí tines \emph{àn} hoi prostátai ôsi\\
of.what.sort.\textsc{m.nom.pl} some.\textsc{m.nom.pl} \textsc{irr}
the.\textsc{m.nom.pl} leader.\textsc{nom.pl} be.\textsc{3pl.prs.sbjv}\\
\trans `... as the leaders are ...' (Xenophon, \textit{Ways} 1)
\label{XenWays1}
\end{exe}

Furthermore, certain particles that themselves are required to appear at the start of the clause, namely \textit{gár}, \textit{ge}, \textit{dé}, \textit{mén}, -\textit{per}, and \textit{te}, regularly precede \emph{án}; there are also isolated examples of \textit{dḗ} `exactly' behaving like this, e.g. (\ref{PlatPhaedo114b}), as well as \emph{méntoi} `yet', e.g. (\ref{XenCyrop219}), and \textit{oûn} `so', e.g. (\ref{AristophFrogs1420}) (although Herodotus in some instances gives \emph{án} precedence over \textit{mén} and \textit{dé} `but', e.g. (\ref{Hdt11381})--(\ref{Hdt78D1})).

\begin{exe}
\ex οἳ δὲ δὴ ἂν δόξωϲι διαφερόντωϲ προκεκρίϲθαι\\
\gll hoì dè \emph{dḕ} \emph{àn} dóxōsi diapheróntōs prokekrísthai\\
who.\textsc{m.nom.pl} but exactly \textsc{irr} seem.\textsc{3sg.aor.sbjv} differently prejudge.\textsc{prf.inf.pass}\\
\trans `But whichever ones seem to have been found excellent ...' (Plato, \textit{Phaedo} 114b)\footnote{\emph{Translator's note}: \textit{prokekrísthai} not in Perseus edition}
\label{PlatPhaedo114b}
\end{exe}

\begin{exe}
\ex οἵ γε μέντ᾽ ἂν αὐτῶν φεύγωϲι\\
\gll hoí ge \emph{mént'} \emph{àn} autôn pheúgōsi\\
who.\textsc{m.nom.pl} even yet \textsc{irr} them.\textsc{gen} flee.\textsc{3pl.prs.sbjv}\\
\trans `... while whichever of them flee ...' (Xenophon, \textit{Cyropaedia} 2.1.9)
\label{XenCyrop219}
\end{exe}

\begin{exe}
\ex ὁπότεροϲ οὖν ἂν τῇ πόλει παραινέϲειν μέλλει τι χρηϲτόν\\
\gll hopóteros \emph{oûn} \emph{àn} têi pólei parainésein méllei ti khrēstón\\
which.\textsc{m.nom.sg} so \textsc{irr} the.\textsc{f.dat.sg}
city.\textsc{dat.sg} advise.\textsc{3sg.aor.sbjv} be.going.to.\textsc{3sg.prs} something.\textsc{nom.sg} useful.\textsc{n.nom.sg}\\
\trans `Whichever one advises the city is going to be of some use.' (Aristophanes, \textit{Frogs} 1420)\footnote{\emph{Translator's note}: The Perseus edition has \textit{mâllón}.}
\label{AristophFrogs1420}
\end{exe}

\begin{exe}
\ex ὃϲ ἂν δὲ τῶν ἀϲτῶν λέπρην {[}...{]} ἔχῃ\\
\gll hòs \emph{àn} \emph{dè} tôn astôn léprēn ékhēi\\
who.\textsc{m.nom.sg} \textsc{irr} then the.\textsc{m.gen.pl} townsman.\textsc{gen.pl} leprosy.\textsc{acc.sg} have.\textsc{3sg.prs.sbjv}\\
\trans `And whoever among the citizens has leprosy ...' (Herodotus 1.138.1)
\label{Hdt11381}
\end{exe}

\begin{exe}
\ex ὃϲ ἂν μέν νυν τῶν πυλωρῶν ἑκὼν παρίῃ\\
\gll hòs \emph{àn} \emph{mén} nun tôn pulōrôn hekṑn paríēi\\
who.\textsc{m.nom.sg} \textsc{irr} then now the.\textsc{m.gen.pl} guard.\textsc{gen.pl} willing.\textsc{m.nom.sg} pass.\textsc{3sg.prs.sbjv}\\
\trans `Now whoever of the guards willingly admits us ...' (Herodotus 3.72.5)
\label{Hdt3725}
\end{exe}

\begin{exe}
\ex ὃϲ ἂν δὲ ἔχων ἥκῃ\\
\gll hòs \emph{àn} \emph{dè} ékhōn hḗkēi\\
who.\textsc{m.nom.sg} \textsc{irr} then have.\textsc{ptcp.prs.m.nom.sg}
arrive.\textsc{3sg.prs.sbjv}\\
\trans `And whoever comes having ...' (Herodotus 7.8D.1)
\label{Hdt78D1}
\end{exe}

But \emph{án} takes precedence over all other words. The inexcusable counterexample (\ref{Antiph538}), which cannot be explained away, has long since been corrected by \citet[78]{Maetzner1838} based on the Oxoniensis manuscript's \textit{àn mēnúēi}.

\begin{exe}
\ex καθ᾽ ὧν μηνύῃ ἄν τιϲ\\
\gll kath' hôn mēnúēi \emph{án} tis\\
down whom.\textsc{gen.pl} inform.\textsc{3sg.prs.sbjv} \textsc{irr} someone.\textsc{m.nom.sg}\\
\trans `... against whom someone informs ...' (Antiphon 5.38)
\label{Antiph538}
\end{exe}

In \citet[688]{Nauck1889} we encounter the even more unexpected verses in (\ref{EurFragm1029}). Dümmler\ia{Dümmler, Ferdinand} (p.c.) proposes \emph{àn pléon} `\textsc{irr} more' instead of the problematic \emph{mâllon àn}. Or should \emph{thélēis} be changed to \emph{thélois}?

\begin{exe}
\ex ἀρετὴ δ᾽ ὅϲῳπερ μᾶλλον ἂν χρῆϲθαι θέλῃϲ, τοϲῷδε μείζων γίγνεται καθ᾽ ἡμέραν\\
\gll aretḕ d' hósōiper mâllon \emph{àn} khrêsthai thélēis, tosôide meízōn gígnetai kath' hēméran\\
goodness.\textsc{voc.sg} then how.much.\textsc{dat.sg} more
\textsc{irr} use.\textsc{prs.inf.pass} want.\textsc{2sg.prs.sbjv} so.much.\textsc{dat.sg} greater.\textsc{m.nom.sg} become.\textsc{3sg.prs} down day.\textsc{acc.sg}\\
\trans `And, Your Excellency, however much more you wish to use, it becomes greater by so much day by day.' (Euripides, Fragment 1029)
\label{EurFragm1029}
\end{exe}

We are on firmer ground with the correction of a third example where \emph{án} is wrongly placed, (\ref{AristophFrogs259}). We should simply reorder this to read \textit{hē phárunx hopóson àn hēmôn}, which does not negatively affect the reply in verse 264 ((\ref{AristophFrogs264})).

\begin{exe}
\ex ὁπόϲον ἡ φάρυγξ ἂν ἡμῶν χανδάνῃ\\
\gll hopóson hē phárunx \emph{àn} hēmôn khandánēi\\
as.much the.\textsc{f.nom.sg} throat.\textsc{nom.sg} \textsc{irr} us.\textsc{gen} contain.\textsc{3sg.prs.sbjv}\\
\trans `... as much as ever our throats can hold.' (Aristophanes, \textit{Frogs} 259)
\label{AristophFrogs259}
\end{exe}

\begin{exe}
\ex οὐδέποτε· κεκράξομαι γάρ\\
\gll oudépote; kekráxomai gár\\
nor.ever croak.\textsc{1sg.fprf} then\\
\trans `... never, for I will croak ...' (Aristophanes, \textit{Frogs} 264)
\label{AristophFrogs264}
\end{exe}

The attachment of \textit{án} to the connective has become very close in Ionic\il{Greek, Ionic} \textit{ḗn} \hyperlink{p380}{\emph{[p380]}} and Attic\il{Greek, Attic} \textit{án}, in which the usual \textit{eán} `if' has arisen through \textit{ei} `if' repeatedly preceding \emph{án}, and in \textit{hótan}, \textit{epeidán}, \textit{epán} = Ionic\il{Greek, Ionic} \textit{epḗn} `whenever', where the requirement for \emph{án} to be preceded by no more than one word is lost.

But in other clause types there is also no difference to be observed between the positions of \textit{án} and \textit{ke}(\textit{n}) in the earliest texts. In main clauses, as well as in indicative\is{indicative (mood)} and \isi{optative} subordinate clauses, we find that \textit{án} in Homer\il{Greek, Homeric|(} follows the positional rule of the \isi{enclitics}. There are only a few cases in which \textit{án} strays from the rule. First, following \textit{ou}: (\ref{ouk5})--(\ref{ouk9}).\label{ouk4}

\begin{exe}
\ex τῶν οὐκ ἄν τι φέροιϲ\\
\gll tôn ouk \emph{án} ti phérois\\
the.\textsc{n.gen.pl} not \textsc{irr} something.\textsc{n.acc.sg}
bear.\textsc{2sg.prs.opt}\\
\trans `... nothing will you take ...' (Homer, \textit{Iliad} 1.301)
\label{ouk5}
\end{exe}

\begin{exe}
\ex πληθὺν δ᾽ οὐκ ἂν ἐγὼ μυθήϲομαι οὐδ᾽ ὀνομήνω\\
\gll plēthùn d' ouk \emph{àn} egṑ muthḗsomai oud' onomḗnō\\
multitude.\textsc{acc.sg} then not \textsc{irr} I.\textsc{nom}
tell.\textsc{1sg.aor.sbjv.mid} nor name.\textsc{1sg.aor.sbjv}\\
\trans `But the common folk I could not tell nor name' (Homer, \textit{Iliad} 2.488)
\label{ouk6}
\end{exe}

\begin{exe}
\ex ἑκὼν δ᾽ οὐκ ἄν τιϲ ἕλοιτο\\
\gll hekṑn d' ouk \emph{án} tis héloito\\
willing.\textsc{m.nom.sg} then not \textsc{irr}
someone.\textsc{m.nom.sg} take.\textsc{3sg.aor.opt.mid}\\
\trans `... whereas by his own will could no man win them.' (Homer, \textit{Iliad} 3.66)
\label{ouk7}
\end{exe}

\begin{exe}
\ex τὸ μὲν οὐκ ἂν ἐγώ ποτε μὰψ ὀμόϲαιμι\\
\gll tò mèn ouk \emph{àn} egṓ pote màps omósaimi\\
the.\textsc{n.acc.sg} then not \textsc{irr} I.\textsc{nom} sometime
vainly swear.\textsc{1sg.aor.opt}\\
\trans `... whereby I verily would never forswear myself' (Homer, \textit{Iliad} 15.40)
\label{ouk8}
\end{exe}

\begin{exe}
\ex ἐπεὶ οὐκ ἂν ἐφορμηθέντε γε νῶϊ τλαῖεν ἐναντίβιον ϲτάντεϲ μαχέϲαϲθαι Ἄρηι\\
\gll epeì ouk \emph{àn} ephormēthénte ge nôï tlaîen enantíbion stántes makhésasthai Árēi\\
since not \textsc{irr} rouse.\textsc{ptcp.aor.pass.m.acc.du} even us.\textsc{acc.du} endure.\textsc{3pl.aor.opt} opposing stand.\textsc{ptcp.aor.m.nom.pl} fight.\textsc{aor.inf.mid} Ares.\textsc{dat}\\
\trans `... seeing the men would not abide the oncoming of us two, and stand to contend with us in battle.' (Homer, \textit{Iliad} 17.489)
\label{ouk9}
\end{exe}

Now, we have already observed repeatedly that \isi{enclitics} tend to attach after \isi{negation}. And if this phenomenon is less often seen with \emph{ke} than with \emph{án}, we should remember Fick's \citeyearpar[xxiii]{Fick1886} remark that \emph{ouk an}, which occurs strikingly often in the transmitted text, often appears to occur in the place of \emph{ou ken}. (Against this, however, see \citealp[330]{Monro1891}.) There are three other relevant examples, one with \textit{kaì án}: (\ref{kaian1}), while in (\ref{kaian2}) the \textit{kaì án} can be viewed as the start of a new clause.

\begin{exe}
\ex ὃϲ νῦν γε καὶ ἂν Διὶ πατρὶ μάχοιτο\\
\gll hòs nûn ge \emph{kaì} \emph{àn} Diì patrì mákhoito\\
who.\textsc{m.nom.sg} now even also \textsc{irr} Zeus.\textsc{dat} father.\textsc{dat.sg} fight.\textsc{3sg.prs.opt.pass}\\
\trans `... that would now fight even with father Zeus.' (Homer, \textit{Iliad} 5.362; cf. also 5.457)
\label{kaian1}
\end{exe}

\begin{exe}
\ex ἄλλον μέν κεν ἔγωγε θεῶν αἰειγενετάων ῥεῖα κατευνήϲαιμι καὶ ἂν ποταμοῖο ῥέεθρα Ὠκεανοῦ\\
\gll állon mén ken égōge theôn aieigenetáōn rheîa kateunḗsaimi \emph{kaì} \emph{àn} potamoîo rhéethra Ōkeanoû\\
other.\textsc{m.acc.sg} then \textsc{irr} I.\textsc{nom.emph} god.\textsc{gen.pl} everlasting.\textsc{m.gen.pl} easily lull.\textsc{1sg.aor.opt} also \textsc{irr} river.\textsc{gen.sg}
stream.\textsc{acc.pl} Ocean.\textsc{gen.sg}\\
\trans `... another of the gods, that are for ever, might I lightly lull to sleep, aye, were it even the streams of the river Oceanus' (Homer, \textit{Iliad} 14.244)
\label{kaian2}
\end{exe}

One with \textit{tákh' án}: (\ref{takhan1}). (Cf. \textit{tákh' án} at the beginning of the clause in (\ref{takhan2})).

\begin{exe}
\ex ᾗϲ ὑπεροπλίῃϲι τάχ᾽ ἄν ποτε θυμὸν ὀλέϲϲῃ\\
\gll hêis huperoplíēisi \emph{tákh'} \emph{án} pote thumòn oléssēi\\
his.\textsc{f.dat.pl} insolence.\textsc{dat.pl} quickly \textsc{irr} sometime spirit.\textsc{acc.sg} destroy.\textsc{3sg.aor.sbjv}\\
\trans `Through his own excessive pride shall he presently lose his life.' (Homer, \textit{Iliad} 1.205)
\label{takhan1}
\end{exe}

\begin{exe}
\ex τάχ᾽ ἄν ποτε καὶ τίϲιϲ εἴη\\
\gll \emph{tákh'} \emph{án} pote kaì tísis eíē\\
quickly \textsc{irr} sometime also compensation.\textsc{nom.sg}
be.\textsc{3sg.prs.opt}\\
\trans `Recompense would haply be made some day' (Homer, \textit{Odyssey} 2.76)
\label{takhan2}
\end{exe}

Finally one with \textit{tót' án}: (\ref{totan1}). (Cf. \textit{tót' án} at the beginning of the clause in Homer, \textit{Iliad} 18.397, 24.213,\footnote{\emph{Translator's note}: The Perseus edition has a different reading.} and \textit{Odyssey} 9.211).

\begin{exe}
\ex ἐμοὶ δὲ τότ᾽ ἂν πολὺ κέρδιον εἴη\\
\gll emoì dè \emph{tót'} \emph{àn} polù kérdion eíē\\
me.\textsc{dat} but then \textsc{irr} much better.\textsc{n.nom.sg}
be.\textsc{3sg.prs.opt}\\
\trans `... but for me it were better far ...' (Homer, \textit{Iliad} 22.108)
\label{totan1}
\end{exe}

These few examples, however, are certainly not enough to justify Hermann's clear-cut division between \emph{án} and \emph{ke}(\emph{n}). His own example \citep[7]{Hermann1831}, \textit{ê s' àn tisaímēn} `truly you \textsc{irr} pay.\textsc{1pl.opt}' as opposed to \textit{ê ke még' oimṓxeie} `truly \textsc{irr} greatly wail.\textsc{3sg.opt}', demonstrates nothing, because \emph{s}(\emph{e}) is enclitic.\footnote{\emph{Translator's note}: These two examples are also included above as (\ref{HomIl2220}) and (\ref{ke1}) respectively.}\is{enclitics} Similarly, of course, no conclusions can be drawn from \textit{eí per án} `if all \textsc{irr}' as opposed to example (\ref{aike1}). Compare, moreover, the collocations \textit{óphr' àn mén ken} `that \textsc{irr} then \textsc{irr}' and \textit{oút' án ken} `nor \textsc{irr} \textsc{irr}', although admittedly these are contested.\footnote{\emph{Translator's note}: See e.g. \textit{Iliad} 11.187 and 13.127 respectively.}

\begin{exe}
\ex αἴ κέ περ ὔμμι φίλον καὶ ἡδὺ γένοιτο\\
\gll aí \emph{ké} per úmmi phílon kaì hēdù génoito\\
if \textsc{irr} all you.\textsc{dat.pl} dear.\textsc{n.nom.sg} and sweet.\textsc{n.nom.sg} become.\textsc{3sg.aor.opt.mid}\\
\trans `... if haply it be your wish and your good pleasure ...' (Homer, \textit{Iliad} 7.387)
\label{aike1}
\end{exe}\il{Greek, Homeric|)}

Post-Homeric literature has \emph{án} firmly following the old rule in \isi{subjunctive} subordinate clauses. Its \hyperlink{p381}{\emph{[p381]}} use in subordinate clauses of other moods is more variable. However, even here \emph{án} attached firmly to the first word in certain cases. The compounds \textit{hōs án} `as \textsc{irr}', \textit{hópōs án} `so \textsc{irr}', and \textit{hṓsper án} `like \textsc{irr}' are particularly worthy of consideration in this connection.

The situation is clearest in final and consecutive clauses beginning with \emph{hōs} `as' and \emph{hópōs} `so' and containing the \isi{optative} or indicative\is{indicative (mood)} with \emph{án}, thanks to the collections that \citet{Weber1884,Weber1885} has assembled and published. In such clauses we have \emph{hōs án} adjacent to each other not only in Homer (e.g. (\ref{hosan3})) but also in (\ref{hosan4})--(\ref{hosan12}), and in (\ref{hosan13}), in which \emph{hōs án} should probably be read as consecutive.

\begin{exe}
\ex ὡϲ ἂν πύρνα κατὰ μνηϲτῆραϲ ἀγείροι\\
\gll \emph{hōs} \emph{àn} púrna katà mnēstêras ageíroi\\
as \textsc{irr} bread.\textsc{acc.pl} down suitor.\textsc{acc.pl} gather.\textsc{3sg.prs.opt}\\
\trans `... to go among the wooers and gather bits of bread ...' (Homer, \textit{Odyssey} 17.362)
\label{hosan3}
\end{exe}

\begin{exe}
\ex ὡϲ ἂν καὶ γέρων ἠράϲϲατο\\
\gll \emph{hōs} \emph{àn} kaì gérōn ērássato\\
as \textsc{irr} also old.\textsc{m.nom.sg} love.\textsc{3sg.aor.mid}\\
\trans `... that even an old man should love' (Archilochus, Fragment 30)
\label{hosan4}
\end{exe}

\begin{exe}
\ex ὡϲ ἄν ϲε θωϊὴ λάβοι\\
\gll \emph{hōs} \emph{án} se thōïḕ láboi\\
as \textsc{irr} you.\textsc{acc} penalty.\textsc{nom.sg}
take.\textsc{3sg.aor.opt}\\
\trans `... that a penalty should overtake you' (Archilochus, Fragment 101)
\label{hosan5}
\end{exe}

\begin{exe}
\ex ὡϲ ἂν θεᾷ πρῶτοι κτίϲαιεν βωμόν\\
\gll \emph{hōs} \emph{àn} theâi prôtoi ktísaien bōmón\\
as \textsc{irr} goddess.\textsc{dat.sg} first.\textsc{m.nom.pl}
build.\textsc{3pl.aor.opt} altar.\textsc{acc.sg}\\
\trans `... that they should be the first to build an altar for the goddess' (Pindar, \textit{Olympian Ode} 7.42)
\label{hosan6}
\end{exe}

\begin{exe}
\ex ὡϲ ἂν ποταθείην\\
\gll \emph{hōs} \emph{àn} potatheíēn\\
as \textsc{irr} soar.\textsc{1sg.aor.opt.pass}\\
\trans `... that I might soar ...' (Aristophanes, \textit{Birds} 1338)\footnote{\emph{Translator's note}: The Perseus edition has \textit{ampotatheíēn}.}
\label{hosan7}
\end{exe}

\begin{exe}
\ex ὡϲ ἂν πυνθανόμενοι πλεῖϲτοι ϲυνέλθοιεν Σπαρτιητέων\\
\gll \emph{hōs} \emph{àn} punthanómenoi pleîstoi sunélthoien Spartiētéōn\\
as \textsc{irr} learn.\textsc{ptcp.prs.pass.m.nom.pl} most.\textsc{m.nom.pl} assemble.\textsc{3pl.aor.opt} Spartan.\textsc{gen.pl}\\
\trans `... so that as many as possible of the Spartans might assemble to hear him' (Herodotus 1.152.1; cf. also 5.37.2, 7.176.4, 8.7.1, 9.22.3, 9.51.3)
\label{hosan8}
\end{exe}

\begin{exe}
\ex ὡϲ ἂν μάλιϲτα τὸν υἱὸν ἐχθρὸν ἑαυτῷ καὶ τῇ πόλει ποιήϲειε\\
\gll \emph{hōs} \emph{àn} málista tòn huiòn ekhthròn heautôi kaì têi pólei poiḗseie\\
as \textsc{irr} most the.\textsc{m.acc.sg} son.\textsc{acc.sg}
enemy.\textsc{acc.sg} himself.\textsc{dat} and the.\textsc{f.dat.sg} city.\textsc{dat} make.\textsc{3sg.aor.opt}\\
\trans `... so as best to make his son an enemy of himself and of the city' ({[}Andocides{]} 4.23)
\label{hosan9}
\end{exe}

\begin{exe}
\ex ὡϲ ἂν μάλιϲτα αὐτὸϲ ὁ δεδεμένοϲ ξυλλήπτωρ εἴη τοῦ δεδέϲθαι\\
\gll \emph{hōs} \emph{àn} málista autòs ho dedeménos xullḗptōr eíē toû dedésthai\\
as \textsc{irr} most same.\textsc{m.nom.sg} the.\textsc{m.nom.sg} bind.\textsc{ptcp.prf.pass.m.nom.sg} accomplice.\textsc{nom.sg} be.\textsc{3sg.prs.opt} the.\textsc{n.gen.sg} bind.\textsc{prf.inf.pass}\\
\trans `... so that the prisoner himself would be the greatest assistant in his imprisonment' (Plato, \textit{Phaedo} 82e)
\label{hosan10}
\end{exe}

\begin{exe}
\ex τοῖϲ μὲν κοϲμίοιϲ τῶν ἀνθρώπων, καὶ ὡϲ ἂν κοϲμιώτεροι γίγνοιντο οἱ μή πω ὄντεϲ, δεῖ χαρίζεϲθαι\\
\gll toîs mèn kosmíois tôn anthrṓpōn, kaì \emph{hōs} \emph{àn} kosmiṓteroi gígnointo hoi mḗ pō óntes, deî kharízesthai\\
the.\textsc{m.dat.pl} then orderly.\textsc{m.dat.pl} the.\textsc{m.gen.pl} person.\textsc{gen.pl} and as \textsc{irr} orderly.\textsc{comp.m.nom.pl} become.\textsc{3pl.prs.opt} the.\textsc{m.nom.pl} not yet be.\textsc{ptcp.prs.m.nom.pl} lack.\textsc{3sg.prs} gratify.\textsc{prs.inf.pass}\\
\trans `It is necessary to indulge the orderly, and so that those who are not yet so may become more orderly.' (Plato, \textit{Symposium} 187d)
\label{hosan11}
\end{exe}

\begin{exe}
\ex δοκῶ μοι {[}...{]} ἔχειν μηχανήν, ὡϲ ἂν εἶεν ἄνθρωποι καὶ παύϲαιντο τῆϲ ἀκολαϲίαϲ\\
\gll dokô moi ékhein mēkhanḗn, \emph{hōs} \emph{àn} eîen ánthrōpoi kaì paúsainto tês akolasías\\ 
think.\textsc{1sg.prs} me.\textsc{dat} have.\textsc{prs.inf} means.\textsc{acc.sg} as \textsc{irr} be.\textsc{3pl.prs.opt} person.\textsc{nom.pl} and stop.\textsc{3pl.aor.mid} the.\textsc{f.gen.sg} intemperance.\textsc{gen.sg}\\
\trans `I think I have a means for man to be and yet cease his iniquity.' (Plato, \textit{Symposium} 190c)\footnote{\emph{Translator's note}: The Persus ed. has \textit{eîen te} for \textit{eîen}.}
\label{hosan12}
\end{exe}

\begin{exe}
\ex ὡϲ δ᾽ ἂν ἐξεταϲθείη μάλιϲτ᾽ ἀκριβῶϲ, μὴ γένοιτο\\
\gll \emph{hōs} d' \emph{àn} exetastheíē málist' akribôs, mḕ génoito\\
as then \textsc{irr} examine.\textsc{3sg.aor.opt.pass} most strictly not become.\textsc{3sg.aor.opt.mid}\\
\trans `May it not come to pass that this be tested in the severest way.' (Demosthenes 6.37)
\label{hosan13}
\end{exe}

Very frequent in Xenophon, the only Attic\il{Greek, Attic} \isi{prose} writer who often connects \emph{hōs} with \emph{án} and the \isi{optative} in a purely final sense. Of the seventeen examples given in \citet[83ff.]{Weber1885}, fourteen have \emph{án} immediately after \emph{hōs}, and only three are separated from it: final (\ref{hosan14}) and (\ref{hosan15}), and consecutive (\ref{hosan16}). These are the only three cases in which the tradition demanding adjacency of \emph{hōs} and \emph{án} is broken.

\begin{exe}
\ex ὡϲ μηδενὸϲ ἂν δέοιτο\\
\gll \emph{hōs} mēdenòs \emph{àn} déoito\\
as nothing.\textsc{gen.sg} \textsc{irr} lack.\textsc{3sg.prs.opt}\\
\trans `... so that he should lack for nothing' (Xenophon, \textit{Cyropaedia} 5.1.18)\footnote{\emph{Translator's note}: The Perseus edition has \textit{endéoito} for \textit{àn déoito}.}
\label{hosan14}
\end{exe}

\begin{exe}
\ex ὡϲ ὅτι ἥκιϲτα ἂν ἐπιφθόνοιϲ ϲπάνιοϲ τε καὶ ϲεμνὸϲ φανείη\\
\gll \emph{hōs} hóti hḗkista \emph{àn} epiphthónois spánios te kaì semnòs phaneíē\\
as that least \textsc{irr} envious.\textsc{m.dat.pl} rare.\textsc{m.nom.sg} and and solemn.\textsc{m.nom.sg} show.\textsc{3sg.aor.opt.pass}\\
\trans `... in such a way that he would appear seldom and solemnly, and with as little envy as possible.' (Xenophon, \textit{Cyropaedia} 7.5.37)\footnote{\emph{Translator's note}: The Perseus edition has \textit{epiphthónōs} for \textit{epiphthónois}.}
\label{hosan15}
\end{exe}

\begin{exe}
\ex ὡϲ πᾶϲ ἂν ἔγνω, ὅτι ἀϲμένη ἤκουϲε\\
\gll \emph{hōs} pâs \emph{àn} égnō, hóti asménē ḗkouse\\
as all.\textsc{m.nom.sg} \textsc{irr} know.\textsc{3sg.aor} that
glad.\textsc{f.nom.sg} hear.\textsc{3sg.aor}\\
\trans `... so that everyone would know that she was glad to hear' (Xenophon, \textit{Symposium} 9.3)
\label{hosan16}
\end{exe}

However, according to the transmitted manuscripts, there are a further two examples from Euripidean verse:\is{poetry} (\ref{hosan17}) and (\ref{hosan18}). But the first verse has been treated with suspicion by critics since \citet[178]{Markland1811}, and in the \hyperlink{p382}{\emph{[p382]}} second the usual reading is \textit{hōs esidoíman}.\footnote{\emph{Translator's note}: The Perseus edition follows this usual reading} (In (\ref{hosan19}), \emph{hōs} is relative.)\is{relative pronouns}

\begin{exe}
\ex ὡϲ δὴ ϲκότοϲ λαβόντεϲ ἐκϲωθεῖμεν ἄν\\
\gll \emph{hōs} dḕ skótos labóntes eksōtheîmen {án}\\
as exactly dark.\textsc{acc.sg} take.\textsc{ptcp.aor.m.nom.pl}
save.\textsc{1pl.aor.opt.pass} \textsc{irr}\\
\trans `... so that we might keep safe using the darkness' (Euripides, \textit{Iphigenia in Tauris} 1025)\footnote{\emph{Translator's note}: The Perseus edition has \textit{skóton} for \textit{skótos}.}
\label{hosan17}
\end{exe}

\begin{exe}
\ex Ἀχαιῶν ϲτρατιὰν ὡϲ ἴδοιμ᾽ ἄν\\
\gll Akhaiôn stratiàn \emph{hōs} ídoim' \emph{án}\\
Achaean.\textsc{gen.pl} army.\textsc{acc.sg} as see.\textsc{1sg.aor.opt} \textsc{irr}\\
\trans `... so that I might see the army of the Achaeans' (Euripides, \textit{Iphigenia in Aulis} 171)
\label{hosan18}
\end{exe}

\begin{exe}
\ex οὕτω προΐῃ, ὡϲ μάλιϲτ᾽ ἂν {[}...{]} ποιοίη\\
\gll hoútō proḯēi, \emph{hōs} málist' \emph{àn} poioíē\\
so proceed.\textsc{3sg.prs.sbjv} as most \textsc{irr} make.\textsc{3sg.prs.opt}\\
\trans `... so proceed as best to make ...' (Plato, \textit{Gorgias} 453c)
\label{hosan19}
\end{exe}

The collocation \emph{hópōs an} `so \textsc{irr}' is even more fixed in such clauses: (\ref{oposan1})--(\ref{oposan5}).\label{oposan}

\begin{exe}
\ex ὅπωϲ ἄν μήτε πρὸ καιροῦ μήθ᾽ ὑπὲρ ἄϲτρων βέλοϲ ἠλίθιον ϲκήψειεν\\
\gll \emph{hópōs} \emph{án} mḗte prò kairoû mḗth' hupèr ástrōn bélos ēlíthion skḗpseien\\
so \textsc{irr} nor before point.\textsc{gen.sg} nor over star.\textsc{gen.pl} dart.\textsc{nom.sg} vain.\textsc{n.nom.sg} land.\textsc{3sg.aor.opt}\\
\trans `... so that his bolt would not land in vain either short of the target or beyond the stars.' (Aeschylus, \textit{Agamemnon} 364)
\label{oposan1}
\end{exe}

\begin{exe}
\ex ὅκωϲ ἂν τὸ ϲτρατόπεδον ἱδρυμένον κατὰ νώτου λάβοι\\
\gll \emph{hókōs} \emph{àn} tò stratópedon hidruménon katà nṓtou láboi\\
so \textsc{irr} the.\textsc{n.acc.sg} camp.\textsc{acc.sg}
settle.\textsc{ptcp.prf.pass.n.acc.sg} down back.\textsc{gen.sg} take.\textsc{3sg.aor.opt}\\
\trans `... so that it would arrive behind where the camp was
situated.' (Herodotus 1.75.5; see also 1.91.2, 1.110.3, 2.126.1, 3.44.1, 5.98.4, 8.13.1)
\label{oposan2}
\end{exe}

\begin{exe}
\ex ὅπωϲ ἂν ἀπολιϲθάνοι καὶ μὴ ἔχοι ἀντιλαβὴν ἡ χείρ\\
\gll \emph{hópōs} \emph{àn} apolisthánoi kaì mḕ ékhoi antilabḕn hē kheír\\
so \textsc{irr} slip.off.\textsc{3sg.prs.opt} and not have.\textsc{3sg.prs.opt} hold.\textsc{acc.sg} the.\textsc{f.nom.sg} hand.\textsc{nom.sg}\\
\trans `... so that the hook would slip off and not take hold.' (Thucydides 7.65.2)
\label{oposan3}
\end{exe}

\newpage
\begin{exe}
\ex ὅπωϲ ἂν περιλάβοιμ᾽ αὐτῶν τινα\\
\gll \emph{hópōs} \emph{àn} periláboim' autôn tina\\
so \textsc{irr} catch.\textsc{1sg.aor.opt} them.\textsc{gen}
someone.\textsc{m.acc.sg}\\
\trans `... so that I might catch one of them.' (Aristophanes, \textit{Ecclesiazusae} 881)\footnote{\emph{Translator's note}: The Perseus edition has interrog.\is{interrogatives} \textit{pôs ... ?}.}
\label{oposan4}
\end{exe}

\begin{exe}
\ex ὅπωϲ ἂν εὐδαιμονοίηϲ\\
\gll \emph{hópōs} \emph{àn} eudaimonoíēs\\
so \textsc{irr} prosper.\textsc{2sg.prs.opt}\\
\trans `... for you to be happy.' (Plato, \textit{Lysis} 207e)
\label{oposan5}
\end{exe}

Very frequent in Xenophon, twelve times (not counting \emph{hópōs} `how' following verbs of advising and thinking) according to the evidence of \citet[83ff.]{Weber1885}, and always such that \emph{án} immediately follows \emph{hópōs}; (\ref{oposan6}) is a typical case.

\begin{exe}
\ex ϲκοπῶ, ὅπωϲ ἂν ὁ μὲν παῖϲ ὅδε ὁ ϲὸϲ καὶ ἡ παῖϲ ἧδε ὡϲ ῥᾷϲτα
διάγοιεν, ἡμεῖϲ δ᾽ ἂν μάλιϲτα (ἂν) εὐφραινοίμεθα\\
\gll skopô, \emph{hópōs} \emph{àn} ho mèn paîs hóde ho sòs kaì hē paîs hêde hōs rhâista diágoien, hēmeîs d' \emph{àn} málista (\emph{àn}) euphrainoímetha\\
consider.\textsc{1sg.prs} so \textsc{irr} the.\textsc{m.nom.sg} then child.\textsc{nom.sg} this.\textsc{m.nom.sg} the.\textsc{m.nom.sg} your.\textsc{m.nom.sg} and the.\textsc{f.nom.sg} child.\textsc{nom.sg} this.\textsc{f.nom.sg} as easily.\textsc{supl} continue.\textsc{3pl.prs.opt} we.\textsc{nom} then \textsc{irr} most \textsc{irr} cheer.\textsc{1pl.prs.opt.pass}\\
\trans `I am considering how this boy of yours and this girl could proceed as easily as possible while we took the most pleasure.' (Xenophon, \textit{Symposium} 7.2)
\label{oposan6}
\end{exe}

In\is{inscriptions|(} (\ref{oposan7}), the \isi{subjunctive} \emph{apallagē} recommended by \citet[75--76]{Herwerden1880} and \citet[3]{Weber1885} is too short for the gap in the inscription, whose extent can be determined by the spelling \emph{stoikhēdon}.

\begin{exe}
\ex ὅπωϲ ἂν ὁ δῆμο{[ϲ ἀπαλλαγείη τ{]}οῦ πολέμου}\\
\gll \emph{hópōs} \emph{àn} ho dêmo{[}s apallageíē t{]}oû polémou\\
so \textsc{irr} the.\textsc{m.nom.sg} people.\textsc{nom.sg}
deliver.\textsc{3sg.aor.opt.pass} the.\textsc{m.gen.sg} war.\textsc{gen.sg}\\
\trans `...so that the people may be delivered from war.' (CIA 2.300.20; \citealp[123--124]{Koehler1877}, 295/4 BCE)
\label{oposan7}
\end{exe}\is{inscriptions|)}

After all of this there can be no doubt that \citet[746]{Hermann1816} and \citet[77]{Velsen1883} are wrong to want to read Aristophanes, \textit{Ecclesiazusae} 916 as (\ref{oposan8}), and that, if \emph{án} is to be inserted here at all, it should be in its normal position immediately following \emph{hópōs}.

\begin{exe}
\ex ὅπωϲ ϲαυτῆϲ \textless{}ἂν\textgreater{} κατόναι(ο)\\
\gll \emph{hópōs} sautês \textless{}\emph{àn}\textgreater{} katónai(o)\\
so yourself.\textsc{f.gen.sg} \textsc{irr} bless.\textsc{2sg.aor.opt.mid}\\
\trans `... so you may be blessed.' (Aristophanes, \textit{Ecclesiazusae} 916)
\label{oposan8}
\end{exe}

Similar to final clauses with \emph{hōs} and \emph{hópōs} are indirect questions in the \isi{optative} and containing \emph{án}, introduced by the same particles or by \textit{pôs} `how'.

a) \emph{hōs án} are immediately adjacent: (\ref{hosan20})--(\ref{hosan22}). The only deviation, as far as I can tell, is the second part of the Demosthenian example (\ref{hosan23}). On Demosthenes 10.45 see below (example (\ref{hosan31})).

\begin{exe}
\ex ἐὰν οἷοί τε γενώμεθα εὑρεῖν, ὡϲ ἂν ἐγγύτατα τῶν εἰρημένων πόλιϲ οἰκήϲειεν\\
\gll eàn hoîoí te genṓmetha heureîn, \emph{hōs} \emph{àn} engútata tôn eirēménōn pólis oikḗseien\\
if such.as.\textsc{m.nom.pl} and become.\textsc{1pl.aor.sbjv.mid} find.\textsc{aor.inf} as \textsc{irr} nearest the.\textsc{n.gen.pl} say.\textsc{ptcp.prf.pass.n.gen.pl} city.\textsc{nom.sg} settle.\textsc{3sg.aor.opt}\\
\trans `If such as we may come to find how a state may be governed as closely as possible to what has been said ...' (Plato, \textit{Republic} 5.473a)
\label{hosan20}
\end{exe}

\begin{exe}
\ex διδάϲκει, ὡϲ ἂν καλλιϲτά τιϲ αὐτῇ χρῷτο\\
\gll didáskei, \emph{hōs} \emph{àn} kallistá tis autêi khrôito\\
teach.\textsc{3sg.prs} as \textsc{irr} well.\textsc{supl}
someone.\textsc{m.nom.sg} her.\textsc{dat.sg} use.\textsc{3sg.prs.opt.pass}\\
\trans `She teaches how one may treat her best.'\\
(Xenophon, \textit{Oeconomicus} 19.18)
\label{hosan21}
\end{exe}

\begin{exe}
\ex τἆλλ᾽ ὡϲ ἂν μοι βέλτιϲτα καὶ τάχιϲτα δοκεῖ παραϲκευαϲθῆναι, καὶ δὴ πειράϲομαι λέγειν\\
\gll tâll' \emph{hōs} \emph{àn} moi béltista kaì tákhista dokeî paraskeuasthênai, kaì dḕ peirásomai légein\\
the=other.\textsc{n.acc.pl} as \textsc{irr} me.\textsc{dat} best and fastest seem.\textsc{3sg.prs.ind} equip.\textsc{aor.inf.pass} also exactly try.\textsc{1sg.fut.mid} say.\textsc{prs.inf}\\
\trans `I shall now attempt to speak of providing the rest in the way that seems best and fastest to me.' (Demosthenes 4.13; cf. also 20.87)
\label{hosan22}
\end{exe}

\begin{exe}
\ex ὡϲ μὲν ἂν εἴποιτε καὶ {[}...{]} ϲυνεῖτε, ἄμεινον Φιλίππου παρεϲκεύαϲθε, ὡϲ δὲ κωλύϲαιτ᾽ ἂν ἐκεῖνον {[}...{]}, παντελῶϲ ἀργῶϲ ἔχετε\\
\gll \emph{hōs} mèn \emph{àn} eípoite kaì suneîte, ámeinon Philíppou pareskeúasthe, \emph{hōs} dè kōlúsait' \emph{àn} ekeînon pantelôs argôs ékhete\\
as then \textsc{irr} say.\textsc{2pl.aor.opt} and perceive.\textsc{2pl.aor.opt} better Philip.\textsc{gen} equip.\textsc{2pl.prf.pass} as but hinder.\textsc{2pl.aor.opt} \textsc{irr} that.\textsc{m.acc.sg} completely idly have.\textsc{2pl.prs}\\
\trans `While you are better equipped than Philip for speaking and listening, as for hindering him you remain completely idle.' (Demosthenes 6.3)
\label{hosan23}
\end{exe}

b) \textit{hópōs án} are immediately adjacent: (\ref{oposan9}). Also frequent in Xenophon: (\ref{oposan10}). Likewise \textit{Anabasis} \hyperlink{p383}{\emph{[p383]}} 3.2.27, 4.3.14, and 5.7.20, \textit{Hellenica} 2.3.13, 3.2.1, 7.1.27, and 7.1.33, and \textit{Cyropaedia} 1.4.13 and 2.1.4. I have no counterexamples to hand. (Cf., however, (\ref{opeian}).)

\begin{exe}
\ex οὐκ οἶδ᾽ ὅπωϲ ἄν τιϲ αὐτὰ νομίϲειε μὴ ἐόντα\\
\gll ouk oîd' \emph{hópōs} \emph{án} tis autà nomíseie mḕ eónta\\
not know.\textsc{1sg.prf} so \textsc{irr} someone.\textsc{m.nom.sg} them.\textsc{n.acc.pl} consider.\textsc{3sg.aor.opt} not be.\textsc{ptcp.prs.n.acc.pl}\\
\trans `I do not know how anyone could consider them not to be so.' ({[}Hippocrates,{]} \textit{De arte}; \citealp[42, line 20]{Gomperz1890})
\label{oposan9}
\end{exe}

\begin{exe}
\ex τὸν γὰρ θεῶν πόλεμον οὐκ οἶδα {[}...{]}, ὅπωϲ ἂν εἰϲ
ἐχυρὸν χωρίον ἀποϲταίη\\
\gll tòn gàr theôn pólemon ouk oîda \emph{hópōs} \emph{àn} eis ekhuròn khōríon apostaíē\\
the.\textsc{m.acc.sg} then god.\textsc{gen.pl} war.\textsc{acc.sg} not know.\textsc{1sg.prf} so \textsc{irr} into secure.\textsc{n.acc.sg} place.\textsc{acc.sg} withdraw.\textsc{3sg.aor.opt}\\
\trans `For in war with the gods I know not how one could withdraw to a place of safety.' (Xenophon, \textit{Anabasis} 2.5.7)
\label{oposan10}
\end{exe}

\begin{exe}
\ex ὡϲ τύχω μαντευμάτων, ὅπῃ νεὼϲ ϲτείλαιμ᾽ ἂν οὔριον πτερόν\\
\gll hōs túkhō manteumátōn, \emph{hópēi} neṑs steílaim' \emph{àn} oúrion pterón\\
as happen.\textsc{1sg.aor.sbjv} oracle.\textsc{gen.pl} whereby
ship.\textsc{gen.sg} prepare.\textsc{1s.aor.opt} \textsc{irr} 
fair-winded.\textsc{n.acc.sg} wing.\textsc{acc.sg}\\
\trans `... so I might obtain an oracle: how I should steer a favourable course ...' (Euripides, \textit{Helen} 146)
\label{opeian}
\end{exe}

c) \emph{pôs an} are immediately adjacent, e.g. (\ref{posan1}) and (\ref{posan2}). I have no counterexamples here either.

\begin{exe}
\ex ϲυνεβουλεύτο, πῶϲ ἂν τὴν μάχην ποιοῖτο\\
\gll sunebouleúto, \emph{pôs} \emph{àn} tḕn mákhēn poioîto\\
counsel.\textsc{3sg.imp.pass} how \textsc{irr} the.\textsc{f.acc.sg} battle.\textsc{acc.sg} make.\textsc{3sg.prs.opt.pass}\\
\trans `He took counsel as to how he should fight the battle.' (Xenophon, \textit{Anabasis} 1.7.2)\footnote{\emph{Translator's note}: The Perseus edition adds \textit{te} after the first word.}
\label{posan1}
\end{exe}

\begin{exe}
\ex εἰ {[}...{]} ἐϲκόπει {[}...{]}, πῶϲ ἂν ἄριϲτ᾽ ἐναντιωθείη τῇ εἰρήνῃ\\
\gll ei eskópei \emph{pôs} \emph{àn} árist' enantiōtheíē têi eirḗnēi\\
if consider.\textsc{3sg.imp} how \textsc{irr} best oppose.\textsc{3sg.aor.opt.pass} the.\textsc{f.dat.sg} peace.\textsc{dat.sg}\\
\trans `If he had considered how he might best oppose the peace ...' (Demosthenes 19.14)
\label{posan2}
\end{exe}

But also the relativizer \emph{hōs}, \emph{hṓsper} `as, how' shows the property of bonding \emph{án} tightly to itself. To begin with \emph{hōs}, it is true that we have cases such as (\ref{hosan24})--(\ref{hosan31}).

\begin{exe}
\ex ὡϲ μάλιϲτ᾽ ἂν ἐν πόθῳ λάβοιϲ\\
\gll \emph{hōs} málist' \emph{àn} en póthōi lábois\\
as most \textsc{irr} in longing.\textsc{dat.sg} take.\textsc{2sg.aor.opt}\\
\trans `Just as you might have most longed for' (Sophocles, \textit{Oedipus at Colonus} 1678)
\label{hosan24}
\end{exe}

\begin{exe}
\ex ὡϲ εἰκὸϲ δόξειεν ἂν εἶναι παρόντι πένθει\\
\gll \emph{hōs} eikòs dóxeien \emph{àn} eînai parónti pénthei\\
as resemble.\textsc{ptcp.prf.n.nom.sg} seem.\textsc{3sg.aor.opt} \textsc{irr}
be.\textsc{prs.inf} be.present.\textsc{ptcp.prs.m.dat.sg} mourning.\textsc{dat}\\
\trans `... as might seem to be likely for one present at a scene of mourning' (Plato, \textit{Phaedo} 59a)
\label{hosan25}
\end{exe}

\begin{exe}
\ex ὡϲ ἡμεῖϲ φαῖμεν ἄν\\
\gll \emph{hōs} hēmeîs phaîmen \emph{án}\\
as we.\textsc{nom} say.\textsc{1pl.prs.opt} \textsc{irr}\\
\trans `... as we may say ...' (Plato, \textit{Phaedo} 118a)
\label{hosan26}
\end{exe}

\begin{exe}
\ex ὡϲ ἀπὸ τούτων ἄν τιϲ εἰκάϲειεν\\
\gll \emph{hōs} apò toútōn \emph{án} tis eikáseien\\
as of this.\textsc{n.gen.pl} \textsc{irr} someone.\textsc{m.nom.sg}
represent.\textsc{3sg.aor.opt}\\
\trans `... as one might infer from this.' (Plato, \textit{Symposium} 190a)
\label{hosan27}
\end{exe}

\begin{exe}
\ex ὡϲ γοῦν ἐγὼ φαίην ἄν\\
\gll \emph{hōs} goûn egṑ phaíēn \emph{án}\\
as at.least I.\textsc{nom} say.\textsc{1sg.prs.opt} \textsc{irr}\\
\trans `So I should say, at least.' (Plato, \textit{Philebus} 15c)
\label{hosan28}
\end{exe}

\begin{exe}
\ex ὥϲ γ᾽ ἡμεῖϲ ἂν οἰηθεῖμεν\\
\gll \emph{hṓs} g' hēmeîs \emph{àn} oiētheîmen\\
as even we.\textsc{nom} \textsc{irr} think.\textsc{1pl.aor.opt.pass}\\
\trans `... which we might otherwise suppose.' (Plato, \textit{Laws} 4.712c)
\label{hosan29}
\end{exe}

\begin{exe}
\ex θᾶττον ἢ ὥϲ τιϲ ἂν ᾤετο\\
\gll thâtton ḕ \emph{hṓs} tis \emph{àn} ṓieto\\
faster than as someone.\textsc{m.nom.sg} \textsc{irr} think.\textsc{3sg.imp.pass}\\
\trans `... faster than one would have imagined' (Xenophon, \textit{Anabasis} 1.5.8)
\label{hosan30}
\end{exe}

\begin{exe}
\ex ὡϲ μὲν οὖν εἴποι τιϲ ἄν, {[}...{]} ταῦτ᾽ ἴϲωϲ ἐϲτιν\\
\gll \emph{hōs} mèn oûn eípoi tis \emph{án}, taût' ísōs estin\\
as then so say.\textsc{3sg.aor.opt} someone.\textsc{m.nom.sg} \textsc{irr} this.\textsc{n.nom.pl} perhaps be.\textsc{3sg.prs}\\
\trans `So while this is perhaps what one might say ...' ({[}Demosthenes{]} 10.45)
\label{hosan31}
\end{exe}

(The remainder of the sentence in (\ref{hosan31}), given in (\ref{hosan31a}), contains interrogative\is{interrogatives} \emph{hōs}.)

\begin{exe}
\ex ὡϲ δὲ καὶ γένοιτ᾽ ἄν, νόμῳ διορθώϲαϲθαι δεῖ\\
\gll \emph{hōs} dè kaì génoit' \emph{án}, nómōi diorthṓsasthai deî\\
as but also become.\textsc{3sg.aor.opt.mid} \textsc{irr} law.\textsc{dat.sg} arrange.\textsc{aor.inf.mid} need.\textsc{3sg.prs}\\
\trans `It is necessary to arrange by law how it should come about.' ({[}Demosthenes{]} 10.45)\footnote{\emph{Translator's note}: The Perseus edition has \textit{en} for \textit{án}.}
\label{hosan31a}
\end{exe}

However, in opposition to these we have not only the examples in (\ref{hosan32})--(\ref{hosan36}); rather, we should also take into account the elliptical\is{ellipsis} use of \emph{hōs án}, which only makes sense if the close connection between \emph{hōs} and \emph{án} was firmly ingrained into linguistic consciousness. In fact, with such uses the verb of the main clause is to be understood as repeated in \isi{optative} form, and we find such repetition realized in (\ref{hosan34}) and (\ref{hosan35}).

\begin{exe}
\ex ἑκόντεϲ, ὡϲ ἂν ἄριϲτα περὶ τῶν οἰκείων βουλεύϲαιντο, πρὸϲ τὴν δύναμιν τὴν αὑτῶν εὖ ποιοῦϲιν\\
\gll hekóntes, \emph{hōs} \emph{àn} árista perì tôn oikeíōn bouleúsainto, pròs tḕn dúnamin tḕn hautôn eû poioûsin\\
willing.\textsc{m.nom.pl} as \textsc{irr} best about the.\textsc{n.gen.pl} domestic.\textsc{n.gen.pl} counsel.\textsc{3pl.aor.opt.mid} to the.\textsc{f.acc.sg} power.\textsc{acc.sg} the.\textsc{f.acc.sg} themselves.\textsc{gen} well do.\textsc{3pl.prs}\\
\trans `They do good willingly, as seems advisable according to their interests, to the best of their own ability.' (Plato, \textit{Phaedrus} 231a; cf. also Plato, \textit{Apology} 34c)
\label{hosan32}
\end{exe}

\begin{exe}
\ex ὡϲ ἂν ϲυντομώτατ᾽ εἴποι τιϲ\\
\gll \emph{hōs} \emph{àn} suntomṓtat' eípoi tis\\
as \textsc{irr} briefly.\textsc{supl} say.\textsc{3sg.aor.opt} someone.\textsc{m.nom.sg}\\
\trans `... as one might say as briefly as possible ...' (Demosthenes 27.7)
\label{hosan33}
\end{exe}

\begin{exe}
\ex ϲτέρξαϲ ὡϲ ἄν υἱόν τιϲ ϲτέρξαι\\
\gll stérxas \emph{hōs} \emph{án} huión tis stérxai\\
love.\textsc{ptcp.aor.m.nom.sg} as \textsc{irr} son.\textsc{acc.sg}
someone.\textsc{m.nom.sg} love.\textsc{3sg.aor.opt}\\
\trans `... having loved as one might love a son ...' (Demosthenes 39.22)
\label{hosan34}
\end{exe}

\begin{exe}
\ex οὐδὲ μεμαρτύρηκεν ἁπλῶϲ, ὡϲ ἄν τιϲ τἀληθῆ μαρτυρήϲειε\\
\gll oudè memartúrēken haplôs, \emph{hōs} \emph{án} tis talēthê marturḗseie\\
nor testify.\textsc{3sg.prf} simply as \textsc{irr} someone.\textsc{m.nom.sg} the=true.\textsc{n.acc.pl} testify.\textsc{3sg.aor.opt}\\
\trans `... nor has (anyone) testified simply, as one would testify to the truth' (Demosthenes 45.18)
\label{hosan35}
\end{exe}

\begin{exe}
\ex τὸ {[}...{]} μὴ πάνθ᾽ ὡϲ ἂν ἡμεῖϲ βουλοίμεθ᾽
ἔχειν {[}...{]}, οὐδέν ἐϲτι θαυμαϲτόν\\
\gll tò mḕ pánth' \emph{hōs} \emph{àn} hēmeîs bouloímeth' ékhein oudén esti thaumastón\\
the.\textsc{n.nom.sg} not all.\textsc{n.acc.pl} as \textsc{irr} we.\textsc{nom} wish.\textsc{1pl.prs.opt.pass} have.\textsc{prs.inf} nothing.\textsc{nom.sg} be.\textsc{3sg.prs} wonderful.\textsc{n.nom.sg}\\
\trans `That everything is not going as we might wish is nothing astonishing.' (Demosthenes, \textit{Exordia} 2.3)
\label{hosan36}
\end{exe}

This \emph{hōs án} is found a) before \emph{ei} `if' in (\ref{hosan37}); cf. the \emph{hōsaneí} of post-classical Greek;\il{Greek, post-Classical} 

\begin{exe}
\ex ὡϲ ὰν εἰ λέγοι\\
\gll \emph{hōs} \emph{àn} ei légoi\\
as \textsc{irr} if say.\textsc{3sg.prs.opt}\\
\trans `... as if he were speaking' (Plato, \textit{Protagoras} 344b)
\label{hosan37}
\end{exe}

b) before \isi{participles}: α) with a new subject: (\ref{hosan38})--(\ref{hosan42}).

\begin{exe}
\ex καὶ τὸν Κῦρον ἐρέϲθαι προπετῶϲ, ὡϲ ἂν παῖϲ μηδέπω ὑποπτήϲϲων\\
\gll kaì tòn Kûron erésthai propetôs, \emph{hōs} \emph{àn} paîs mēdépō hupoptḗssōn\\
and the.\textsc{m.acc.sg} Cyrus.\textsc{acc.sg} ask.\textsc{aor.inf.mid} precipitously as \textsc{irr} child.\textsc{m.nom.sg} nor.yet crouch.\textsc{ptcp.prs.m.nom.sg}\\
\trans `And Cyrus asked precipitously, like a boy not yet shy...' (Xenophon, \textit{Cyropaedia} 1.3.8)\footnote{\emph{Translator's note}: The Perseus edition has \textit{eperésthai} for \textit{erésthai}.}
\label{hosan38}
\end{exe}

\begin{exe}
\ex ἀπεκρίνατο, οὐχ ὥϲπερ οἱ φυλαττόμενοι {[}...{]}, ἀλλ᾽ ὡϲ ἂν πεπειϲμένοι μάλιϲτα πράττειν τὰ δέοντα\\
\gll apekrínato, oukh hṓsper hoi phulattómenoi all' \emph{hōs} \emph{àn} pepeisménoi málista práttein tà déonta\\
answer.\textsc{3sg.aor.mid} not like the.\textsc{m.nom.pl} guard.\textsc{ptcp.prs.pass.m.nom.pl} but as \textsc{irr} 
persuade.\textsc{ptcp.prf.pass.m.nom.pl} most do.\textsc{prs.inf} the.\textsc{n.acc.pl} need.\textsc{ptcp.prs.n.acc.pl}\\
\trans `He answered not like those defending themselves but like those most determined to do what is necessary.' (Xenophon, \textit{Memorabilia} 3.8.1)
\label{hosan39}
\end{exe}

\begin{exe}
\ex ἔχει τὰ μέν, ὡϲ ἂν ἑλών τιϲ πολέμῳ\\
\gll ékhei tà mén, \emph{hōs} \emph{àn} helṓn tis polémōi\\
have.\textsc{3sg.prs} the.\textsc{n.acc.pl} then as \textsc{irr}
take.\textsc{ptcp.aor.m.nom.sg} someone.\textsc{m.nom.sg} war.\textsc{dat.sg}\\
\trans `He holds these as one who has taken by force' (Demosthenes 4.6)\footnote{\emph{Translator's note}: The Perseus edition adds \textit{ékhoi} after \textit{tis}.}
\label{hosan40}
\end{exe}

\begin{exe}
\ex οὐδὲ ταῦθ᾽ ἁπλῶϲ {[}...{]} φανήϲεται γεγραφώϲ, ἀλλ᾽ ὡϲ ἂν μάλιϲτά τιϲ ὑμᾶϲ ἐξαπατῆϲαι καὶ παρακρούϲαϲθαι βουλόμενοϲ\\
\gll oudè taûth' haplôs phanḗsetai gegraphṓs, all' \emph{hōs} \emph{àn} málistá tis humâs exapatêsai kaì parakroúsasthai boulómenos\\
nor this.\textsc{n.acc.pl} simply show.\textsc{3sg.fut.pass} write.\textsc{ptcp.prf.m.nom.sg} but as \textsc{irr} most someone.\textsc{m.nom.sg} you.\textsc{acc.pl} deceive.\textsc{aor.inf} and mislead.\textsc{aor.inf.mid} wish.\textsc{ptcp.prs.pass.m.nom.sg}\\
\trans `Nor does he appear having written these things simply, but as one wanting most to deceive and mislead you.' (Demosthenes 24.79)
\label{hosan41}
\end{exe}

\hyperlink{p384}{\emph{[p384]}}

\begin{exe}
\ex ϲυγγραφὰϲ ἐποιήϲαντο {[}...{]}, ὡϲ ἂν οἱ μάλιϲτα ἀπιϲτοῦντεϲ\\
\gll sungraphàs epoiḗsanto \emph{hōs} \emph{àn} hoi málista apistoûntes\\
contract.\textsc{acc.pl} make.\textsc{3pl.aor.mid} as \textsc{irr} the.\textsc{m.nom.pl} most distrust.\textsc{ptcp.prs.m.nom.pl}\\
\trans `They drew up two contracts, as with the greatest distrust' ({[}Demosthenes{]} 34.32)
\label{hosan42}
\end{exe}

More frequently, β) without explicit mention of the indefinite\is{indefinites} subject actually intended (``as someone did in such and such a condition''), where \emph{hōs án} comes very close to the meaning of \emph{háte} `as' and the participle\is{participles} takes the case of the word in the main clause whose referent is specified by the participle.\is{participles} Thus, for instance, (\ref{hosan43})--(\ref{hosan51}).

\begin{exe}
\ex γλῶϲϲαν οὐκέτ᾽ Ἀττικὴν ἱένταϲ, ὡϲ ἂν πολλαχοῦ πλανωμένουϲ\\
\gll glôssan oukét' Attikḕn hiéntas, \emph{hōs} \emph{àn} pollakhoû planōménous\\
tongue.\textsc{acc.sg} no.more Attic.\textsc{f.acc.sg} send.\textsc{ptcp.prs.m.acc.pl} as \textsc{irr} many.places lead.astray.\textsc{ptcp.prs.pass.m.acc.pl}\\
\trans `... no longer uttering the Attic tongue, as wanderers in many places ...' (Solon in Aristotle, \textit{Constitution of the Athenians} 12.4; now confirmed by \citealp[31 line 10]{Kenyon1891})\footnote{\emph{Translator's note}: The Perseus edition has \textit{pollakhêi} for \textit{pollakhoû}.}
\label{hosan43}
\end{exe}

\begin{exe}
\ex ἡ γυνὴ οὐκ ἤθελεν ἀπιέναi, ὡϲ ἂν ἀϲμένη με ἑορακυῖα\\
\gll hē gunḕ ouk ḗthelen apiénai, \emph{hōs} \emph{àn} asménē me heorakuîa\\
the.\textsc{f.nom.sg} woman.\textsc{nom.sg} not want.\textsc{3sg.imp} go.away.\textsc{prs.inf} as \textsc{irr} glad.\textsc{f.nom.sg} me.\textsc{acc}
see.\textsc{ptcp.prf.f.nom.sg}\\
\trans `My wife was unwilling to go, as if (she were) glad to see me.' (Lysias 1.12)\footnote{\emph{Translator's note}: The Perseus edition has \textit{hē dè tò mèn prôton ouk ḗthelen}.}
\label{hosan44}
\end{exe}

\begin{exe}
\ex διεϲιώπηϲεν, ὡϲ ἂν τότε ϲκοπῶν, ὁπόθεν ἄρχοιτο\\
\gll diesiṓpēsen, \emph{hōs} \emph{àn} tóte skopôn, hopóthen árkhoito\\
remain.silent.\textsc{3sg.aor} as \textsc{irr} then consider.\textsc{ptcp.prs.m.nom.sg} whence begin.\textsc{3sg.prs.opt.pass}\\
\trans `(He) remained silent, as if now considering how he should begin.' (Xenophon, \textit{Memorabilia} 3.6.4)
\label{hosan45}
\end{exe}

\begin{exe}
\ex κρότον τοιοῦτον ὡϲ ἂν ἐπαινοῦντέϲ τε καὶ ϲυνηϲθέντεϲ ἐποιήϲατε\\
\gll króton toioûton \emph{hōs} \emph{àn} epainoûntés te kaì sunēsthéntes epoiḗsate\\
applause.\textsc{acc.sg} such.\textsc{m.acc.sg} as \textsc{irr}
praise.\textsc{ptcp.prs.m.nom.pl} and and sympathize.\textsc{ptcp.aor.pass.m.nom.pl} make.\textsc{2pl.aor}\\
\trans `You made such applause as would those who approve of and rejoice with me.' (Demosthenes 21.14)
\label{hosan46}
\end{exe}

\begin{exe}
\ex θρυλοῦντοϲ ἀεί, τὸ μὲν πρῶτον ὡϲ ἂν εἰϲ κοινὴν γνώμην ἀποφαινομένου\\
\gll thruloûntos aeí, tò mèn prôton \emph{hōs} \emph{àn} eis koinḕn gnṓmēn apophainoménou\\
chatter.\textsc{ptcp.prs.m.gen.sg} always the.\textsc{n.acc.sg} then first.\textsc{n.acc.sg} as \textsc{irr} into common.\textsc{f.acc.sg} opinion.\textsc{acc.sg} display.\textsc{ptcp.prs.pass.m.gen.sg}\\
\trans `... always talking, at first as one communicating his opinion ...' (Demosthenes 19.156)\footnote{\emph{Translator's note}: The Perseus edition has \textit{koinòn} for \textit{koinḕn}.}
\label{hosan47}
\end{exe}

\begin{exe}
\ex διαλεχθείϲ τι πρὸϲ αὑτὸν οὕτωϲ ὡϲ ἂν μεθύων\\
\gll dialekhtheís ti pròs hautòn hoútōs \emph{hōs} \emph{àn} methúōn\\
discuss.\textsc{ptcp.aor.pass.m.nom.sg} something.\textsc{acc} to himself.\textsc{acc} so as \textsc{irr} be.drunk.\textsc{ptcp.prs.m.nom.sg}\\
\trans `... saying something to himself, as a drunken man does... ' (Demosthenes 54.7)
\label{hosan48}
\end{exe}

\begin{exe}
\ex ϲυνεδείπνει ἐναντίον πολλῶν Νέαιρα, ὡϲ ἂν ἑταίρα οὖϲα\\
\gll sunedeípnei enantíon pollôn Néaira, \emph{hōs} \emph{àn} hetaíra oûsa\\
dine.together.\textsc{3sg.imp} before many.\textsc{m.gen.pl} Neaera.\textsc{nom} as \textsc{irr} companion.\textsc{nom.sg} be.\textsc{ptcp.prs.f.nom.sg}\\
\trans `Neaera dined with them in public, as would one who was a courtesan.' ({[}Demosthenes{]} 59.24)\footnote{\emph{Translator's note}: The Perseus edition adds \textit{hautēì} after \textit{Néaira}.}
\label{hosan49}
\end{exe}

\begin{exe}
\ex ϲημεῖον δ᾽ ἐ\textless{}πι\textgreater{}φέρουϲι τό τε ὄνομα τοῦ τέλουϲ, ὡϲ ἂν ἀπὸ τοῦ πράγματοϲ κείμενον\\
\gll sēmeîon d' e\textless{}pi\textgreater{}phérousi tó te ónoma toû télous, \emph{hōs} \emph{àn} apò toû prágmatos keímenon\\
sign.\textsc{acc.sg} then bring.\textsc{3pl.prs} the.\textsc{n.acc.sg} and name.\textsc{acc.sg} the.\textsc{n.gen.sg} end.\textsc{gen.sg} as \textsc{irr} of the.\textsc{n.gen.sg} deed.\textsc{gen.sg} lie.\textsc{ptcp.prs.pass.n.acc.sg}\\
\newpage
\trans `... and they adduce as a proof the name of the rating as being derived from the fact' (Aristotle, \textit{Constitution of the Athenians} 7.4; \citealp[19, line 12]{Kenyon1891})\footnote{\emph{Translator's note}: The Perseus edition has \textit{dè phérousi} for \textit{d' e\textless{}pi\textgreater{}phérousi}.}
\label{hosan50}
\end{exe}

\begin{exe}
\ex ἔπτη δ᾽ ὡϲ ἂν ἔχων τοὺϲ πόδαϲ ἡμετέρουϲ\\
\gll éptē d' \emph{hōs} \emph{àn} ékhōn toùs pódas hēmetérous\\
fly.\textsc{3sg.aor} then as \textsc{irr} have.\textsc{ptcp.prs.m.nom.sg} the.\textsc{m.acc.pl} feet.\textsc{acc.pl} our.\textsc{m.acc.pl}\\
\trans `He flew as if he had our feet.' (Anthologia Graeca 6.259)
\label{hosan51}
\end{exe}

c) Other: (\ref{hosan52})--(\ref{hosan60}).

\begin{exe}
\ex ἄγαν καλῶϲ κλύουϲά γ᾽ ὡϲ ἂν οὐ φίλη\\
\gll ágan kalôs klúousá g' \emph{hōs} \emph{àn} ou phílē\\
too well hear.\textsc{ptcp.prs.f.nom.sg} even as \textsc{irr} not dear.\textsc{f.nom.sg}\\
\trans `... obeying rather too well, as if unfriendly' (Aeschylus, \textit{Suppliants} 718)\footnote{\emph{Translator's note}: The Perseus edition has \textit{toîsin} for \textit{hōs àn}.}
\label{hosan52}
\end{exe}

\begin{exe}
\ex ὡϲ ἂν μάλιϲτα, μετὰ ἀειμνήϲτου μαρτυρίου τὴν χάριν καταθήϲεϲθε\\
\gll \emph{hōs} \emph{àn} málista, metà aeimnḗstou marturíou tḕn khárin katathḗsesthe\\
as \textsc{irr} most after ever-remembered.\textsc{n.gen.sg} testimony.\textsc{gen.sg} the.\textsc{f.acc.sg} grace.\textsc{acc} place.\textsc{2pl.fut.mid}\\
\trans `As far as possible, you will store up gratitude as an everlasting witness' (Thucydides 1.33.1)
\label{hosan53}
\end{exe}

\begin{exe}
\ex ἀπεριϲκέπτωϲ προϲπεϲόντεϲ καὶ ὡϲ ἂν μάλιϲτα δι᾽ ὀργῆϲ\\
\gll aperisképtōs prospesóntes kaì \emph{hōs} \emph{àn} málista di' orgês\\
recklessly fall.upon.\textsc{ptcp.aor.m.nom.pl} and as \textsc{irr} most through anger.\textsc{gen.sg}\\
\trans `... recklessly falling upon him in the greatest possible rage ...' (Thucydides 6.57.3)
\label{hosan54}
\end{exe}

\begin{exe}
\ex δῶρα πολλὰ {[}...{]} φέρων καὶ ἄγων, ὡϲ ἂν ἐξ οἴκου μεγάλου\\ 
\gll dôra pollà phérōn kaì ágōn, \emph{hōs} \emph{àn} ex oíkou megálou\\
gift.\textsc{acc.pl} many.\textsc{n.acc.pl} bear.\textsc{ptcp.prs.m.nom.sg} and lead.\textsc{ptcp.prs.m.nom.sg} as \textsc{irr} out house.\textsc{gen.sg} great.\textsc{m.gen.sg}\\
\trans `... bearing and bringing many gifts, as from a great house' (Xenophon, \textit{Cyropaedia} 5.4.29)
\label{hosan55}
\end{exe}

\begin{exe}
\ex εἴ ϲοι πείϲαιμι {[}...{]} (ἐπιτρέπειν) τὴν πόλιν ψευδόμενοϲ, ὡϲ ἂν ϲτρατηγικῷ τε καὶ δικαϲτικῷ καὶ πολιτικῷ\\
\gll eí soi peísaimi (epitrépein) tḕn pólin pseudómenos, \emph{hōs} \emph{àn} stratēgikôi te kaì dikastikôi kaì politikôi\\
if you.\textsc{dat} persuade.\textsc{1sg.aor.opt} entrust.\textsc{prs.inf} the.\textsc{f.acc.sg} city.\textsc{acc.sg} lie.\textsc{ptcp.prs.pass.m.nom.sg} as \textsc{irr} military.\textsc{m.dat.sg} and and juridical.\textsc{m.dat.sg} and political.\textsc{m.dat.sg}\\
\trans `If by lying I persuaded (them to entrust) the city to you, as if to a general and jurist and statesman ...' (Xenophon, \textit{Memorabilia} 2.6.38)
\label{hosan56}
\end{exe}

\begin{exe}
\ex οὐδ᾽ ὡϲ ἂν καλλιϲτ᾽ αὐτῷ τὰ παρόντ᾽ ἔχει\\
\gll oud' \emph{hōs} \emph{àn} kallist' autôi tà parónt' ékhei\\
nor as \textsc{irr} well.\textsc{supl} him.\textsc{dat} the.\textsc{n.nom.pl} be.present.\textsc{ptcp.prs.n.nom.pl} have.\textsc{3sg.prs}\\
\trans `Nor is the present situation going as well as possible for him' (Demosthenes 1.21)
\label{hosan57}
\end{exe}

\begin{exe}
\ex οὐχ ὡϲ ἂν εὔνουϲ καὶ δίκαιοϲ πολίτηϲ ἔϲχε τὴν γνώμην\\
\gll oukh \emph{hōs} \emph{àn} eúnous kaì díkaios polítēs éskhe tḕn gnṓmēn\\
not as \textsc{irr} right.minded.\textsc{m.nom.sg} and righteous.\textsc{m.nom.sg} citizen.\textsc{nom.sg} have.\textsc{3sg.aor} the.\textsc{f.acc.sg} opinion.\textsc{f.acc.sg}\\
\trans `... not as an honest and loyal citizen would have held' (Demosthenes 18.291)
\label{hosan58}
\end{exe}

\begin{exe}
\ex ἀφυλάκτων ὄντων, ὡϲ ἂν πρὸϲ φίλον τῶν ἐν τῇ χώρᾳ\\
\gll aphuláktōn óntōn, \emph{hōs} \emph{àn} pròs phílon tôn en têi khṓrāi\\
unguarded.\textsc{m.gen.pl} be.\textsc{ptcp.prs.m.gen.pl} as \textsc{irr} to friend.\textsc{acc.sg} the.\textsc{m.gen.pl} in the.\textsc{f.dat.sg} country.\textsc{dat.sg}\\
\trans `Those in the country being off their guard, as if toward a friend ...' (Demosthenes 23.154)
\label{hosan59}
\end{exe}

\begin{exe}
\ex ὑπὲρ τῶν ἱππέων τῶν αἰχμαλώτων ὡϲ ἂν ὑπὲρ πολιτῶν\\
\gll hupèr tôn hippéōn tôn aikhmalṓtōn \emph{hōs} \emph{àn} hupèr politôn\\
over the.\textsc{m.gen.pl} horseman.\textsc{gen.pl} the.\textsc{m.gen.pl} captive.\textsc{m.gen.pl} as \textsc{irr} over citizen.\textsc{gen.pl}\\
\trans `... over the captive horsemen as over citizens ...' (CIA 2.243.34)
\label{hosan60}
\end{exe}

The use of \emph{hṓsper} `like' is perhaps even more striking. It is true that we read \textit{hṓsper ... án} in (\ref{osperan1})--(\ref{osperan2}). On the other hand, though, in (\ref{osperan3}) we have \textit{hṓsper àn} etc., as well as in examples (\ref{osperan4}) and (\ref{osperan5}), the latter with a remarkable double \emph{án},\is{doubling} and in (\ref{osperan6}).

\begin{exe}
\ex ὥϲπερ ϲελήνηϲ ὄψιϲ εὐφρόναϲ δύο ϲτῆναι δύναιτ᾽ ἄν\\
\gll \emph{hṓsper} selḗnēs ópsis euphrónas dúo stênai dúnait' \emph{án}\\
like moon.\textsc{gen.sg} face.\textsc{nom.sg} night.\textsc{acc.pl} two stand.\textsc{aor.inf} can.\textsc{3sg.prs.opt.pass} \textsc{irr}\\
\trans `... just as if the face of the moon could stand still for two nights' (Sophocles, Fragment 787)
\label{osperan1}
\end{exe}

\begin{exe}
\ex τὸν αὐτὸν τρόπον, ὥϲπερ τῶν ϲτρατευμάτων ἀξιώϲειέ τιϲ ἂν τὸν ϲτρατηγὸν ἡγεῖϲθαι\\
\gll tòn autòn trópon, \emph{hṓsper} tôn strateumátōn axiṓseié tis \emph{àn} tòn stratēgòn hēgeîsthai\\
the.\textsc{m.acc.sg} same.\textsc{m.acc.sg} way.\textsc{acc.sg} like the.\textsc{n.gen.pl} troop.\textsc{gen.pl} deem.\textsc{3sg.aor.opt} someone.\textsc{m.nom.sg} \textsc{irr} the.\textsc{m.acc.sg} leader.\textsc{acc.sg} lead.\textsc{prs.inf.pass}\\
\trans `... in the same way that someone from the army would expect the general to lead ...' (Demosthenes 4.39)
\label{osperan2}
\end{exe}

\begin{exe}
\ex ὥϲπερ ἂν ἥδιϲτα καὶ ἐπιτηδειότατα ἀμφοτέροιϲ ἐγίγνετο, ἐγὼ μὲν ἐκέλευον\\
\gll \emph{hṓsper} \emph{àn} hḗdista kaì epitēdeiótata amphotérois egígneto, egṑ mèn ekéleuon\\
like \textsc{irr} sweetly.\textsc{supl} and suitably.\textsc{supl} both.\textsc{m.dat.pl} become.\textsc{3sg.imp.pass} I.\textsc{nom} then order.\textsc{1sg.imp}\\
\trans `Just as if it were happening in the most satisfactory and suitable way for both parties, I would urge ...' (Antiphon 6.11)
\label{osperan3}
\end{exe}

\begin{exe}
\ex δοκεῖ ὁμοίωϲ λέγεϲθαι ταῦτα, ὥϲπερ ἄν τιϲ περὶ ἀνθρώπου {[}...{]} λέγοι τοῦτον τὸν λόγον\\
\gll dokeî homoíōs légesthai taûta, \emph{hṓsper} \emph{án} tis perì anthrṓpou légoi toûton tòn lógon\\
seem.\textsc{3sg.prs} similarly say.\textsc{prs.inf.pass} this.\textsc{n.acc.pl} like \textsc{irr} someone.\textsc{m.nom.sg} about person.\textsc{gen.sg} say.\textsc{3sg.prs.opt} this.\textsc{m.acc.sg} the.\textsc{m.acc.sg} account.\textsc{acc.sg}\\
\trans `To say these things seems similar to one's making this argument about someone ...' (Plato, \textit{Phaedo} 87b)
\label{osperan4}
\end{exe}

\hyperlink{p385}{\emph{[p385]}}

\begin{exe}
\ex ἀλλ᾽ ὥϲπερ ἂν μουϲικὸϲ ἐντυχὼν ἀνδρὶ {[}...{]} οὐκ ἀγρίωϲ εἴποι ἄν\\
\gll all' \emph{hṓsper} \emph{àn} mousikòs entukhṑn andrì ouk agríōs eípoi án\\
but like \textsc{irr} musical.\textsc{m.nom.sg} encounter.\textsc{ptcp.aor.m.nom.sg} man.\textsc{dat} not roughly say.\textsc{3sg.aor.opt} \textsc{irr}\\
\trans `... but just as a musician, encountering (such) a man, would not say roughly ...' (Plato, \textit{Phaedrus} 268d)
\label{osperan5}
\end{exe}

\begin{exe}
\ex ἐκείνῳ δὲ πιϲτευούϲηϲ {[}...{]}, ὥϲπερ ἂν γυνὴ γαμβρὸν ἀϲπάζοιτο\\
\gll ekeínōi dè pisteuoúsēs, hṓsper àn gunḕ gambròn aspázoito\\
that.\textsc{m.dat.sg} but trust.\textsc{ptcp.prs.f.gen.sg} like \textsc{irr} woman.\textsc{nom.sg} son-in-law.\textsc{acc.sg} welcome.\textsc{3sg.prs.opt}\\
\trans `... but trusting him, just as a woman would her son-in-law ...' (Xenophon, \textit{Hellenica} 3.1.14)
\label{osperan6}
\end{exe}

In particular, when a conditional is inserted into the comparative clause, the word order \emph{hṓsper án eí} `like \textsc{irr} if' is found throughout: (\ref{osperanei1})--(\ref{osperanei8}).

\begin{exe}
\ex ὥϲπερ οὖν ἄν, εἴ τῷ ὄντι ξένοϲ ἐτύγχανον ὤν, ξυνεγιγνώϲκετε δήπου ἄν μοι\\
\gll \emph{hṓsper} oûn \emph{án}, \emph{eí} tôi ónti xénos etúnkhanon ṓn, xunegignṓskete dḗpou án moi\\
like so \textsc{irr} if the.\textsc{n.dat.sg} be.\textsc{ptcp.prs.n.dat.sg} stranger.\textsc{nom.sg} happen.\textsc{1sg.imp} be.\textsc{ptcp.prs.m.nom.sg}
agree.\textsc{2pl.imp} doubtless \textsc{irr} me.\textsc{dat}\\
\trans `So just as you would doubtless excuse me if I happened in fact to be a foreigner ...' (Plato, \textit{Apology} 17d)
\label{osperanei1}
\end{exe}

\begin{exe}
\ex ὥϲπερ ἄν, εἰ ἐτύγχανεν ὢν ὑποδημάτων δημιουργόϲ, ἀποκρίναιτο ἂν δήπου ϲοι\\
\gll \emph{hṓsper} \emph{án}, \emph{ei} etúnkhanen ṑn hupodēmátōn dēmiourgós, apokrínaito àn dḗpou soi\\
like \textsc{irr} if happen.\textsc{3sg.imp} be.\textsc{ptcp.prs.m.nom.sg} shoe.\textsc{gen.pl} craftsman.\textsc{nom.sg} answer.\textsc{3sg.aor.mid} \textsc{irr} doubtless you.\textsc{dat}\\
\trans `Just as, if he had happened to be a maker of shoes, he would have answered you...' (Plato, \textit{Gorgias} 447d)
\label{osperanei2}
\end{exe}

\begin{exe}
\ex ὥϲπερ ἄν, εἴ τίϲ με ἔροιτο {[}...{]}, εἴποιμ᾽ ἄν\\
\gll \emph{hṓsper} \emph{án}, \emph{eí} tís me éroito eípoim' án\\
like \textsc{irr} if someone.\textsc{m.nom.sg} me ask.\textsc{3sg.aor.opt.mid} say.\textsc{1sg.aor.opt} \textsc{irr}\\
\trans `Just as, if someone asked me ... I should say ...' (Plato, \textit{Gorgias} 451a)
\label{osperanei3}
\end{exe}

\begin{exe}
\ex ὥϲπερ ἄν, εἰ ἐτύγχανον {[}...{]}, ἆρ᾽ οὐκ ἂν δικαίωϲ ϲε ἠρόμην;\\
\gll \emph{hṓsper} \emph{án}, \emph{ei} etúnkhanon âr' ouk àn dikaíōs se ērómēn\\
like \textsc{irr} if happen.\textsc{1sg.imp} then not \textsc{irr}
righteously you.\textsc{acc} ask.\textsc{1sg.aor.mid}\\
\trans `Just as, if I happened to ... then could I not fairly
ask ... ?' (Plato, \textit{Gorgias} 453c)
\label{osperanei4}
\end{exe}

\begin{exe}
\ex ὥϲπερ ἄν, εἰ ἐπενόειϲ {[}...{]} ἀργύριον τελεῖν {[}...{]}, εἴ τίϲ ϲε ἤρετο {[}...{]}, τί ἂν ἀπεκρίνω\\
\gll \emph{hṓsper} \emph{án}, \emph{ei} epenóeis argúrion teleîn eí tís se ḗreto tí àn apekrínō\\
like \textsc{irr} if intend.\textsc{2sg.imp} money.\textsc{acc.sg} fulfil.\textsc{prs.inf} if someone.\textsc{m.nom.sg} you.\textsc{acc} ask.\textsc{3sg.aor.mid} what.\textsc{acc.sg} \textsc{irr} answer.\textsc{2sg.aor.mid}\\
\trans `Just as, if you intended to pay money, if someone asked you ... what would you answer?' (Plato, \textit{Protagoras} 311b)
\label{osperanei5}
\end{exe}

\begin{exe}
\ex ὥϲπερ ἄν, εἰ {[}...{]} Ἱπποκράτηϲ ὅδε ἐπιθυμήϲειε {[}...{]} καὶ {[}...{]} ἀκούϲειεν {[}...{]}, εἰ αὐτὸν ἐπανέροιτο {[}...{]}, εἴποι ἂν αὐτῷ\\
\gll \emph{hṓsper} \emph{án}, \emph{ei} Hippokrátēs hóde epithumḗseie kaì akoúseien ei autòn epanéroito eípoi àn autôi\\
like \textsc{irr} if Hippocrates this.\textsc{m.nom.sg} desire.\textsc{3sg.aor.opt} and hear.\textsc{3sg.aor.opt} if him.\textsc{acc} enquire.\textsc{3sg.aor.opt.mid} say.\textsc{3sg.aor.opt} \textsc{irr} him.\textsc{dat}\\
\trans `Just as, if Hippocrates here desired ... and heard ..., if he enquired of him ... he would say to him ...' (Plato, \textit{Protagoras} 318b)
\label{osperanei6}
\end{exe}

\begin{exe}
\ex ὥϲπερ ἄν, εἰ ζητοίηϲ, τίϲ διδάϲκαλοϲ τοῦ ἑλληνίζειν, οὐδ᾽ ἂν εἷϲ φανείη\\
\gll \emph{hṓsper} \emph{án}, \emph{ei} zētoíēs, tís didáskalos toû hellēnízein, oud' àn heîs phaneíē\\
like \textsc{irr} if seek.\textsc{2sg.prs.opt} who.\textsc{m.nom.sg}
teacher.\textsc{nom.sg} the.\textsc{n.gen.sg} Hellenize.\textsc{prs.inf} nor \textsc{irr} one.\textsc{m.nom.sg} show.\textsc{3sg.aor.opt.pass}\\
\trans `Just as, if you tried to find who (was) a teacher of Greek, no one would appear.' (Plato, \textit{Protagoras} 327e)
\label{osperanei7}
\end{exe}

\begin{exe}
\ex ὥϲπερ ἄν, εἴ τιϲ {[}...{]} τάττοι, οὐκ ἂν αὐτόϲ γ᾽ ἀδικεῖν παρεϲκευάϲθαι δόξαι\\
\gll \emph{hṓsper} \emph{án}, \emph{eí} tis táttoi, ouk àn autós g' adikeîn pareskeuásthai dóxai\\
like \textsc{irr} if someone.\textsc{m.nom.sg} assign.\textsc{3sg.prs.opt} not \textsc{irr} same.\textsc{m.nom.sg} even wrong.\textsc{prs.inf} equip.\textsc{prf.inf.pass} seem.\textsc{3sg.aor.opt}\\
\trans `Just as, if someone assigned ... he would not seem to be prepared to commit an offence himself.' (Demosthenes 20.143)
\label{osperanei8}
\end{exe}

Here, too, the tight attachment of \emph{án} occurs particularly because \emph{hṓsper án} is very often used elliptically\is{ellipsis} without (\isi{optative} or \isi{preterite}) verb -- either where a form of the verb \emph{eimí} `be' is to be understood, as in (\ref{osperan7}), or the verb of the higher clause: (\ref{osperan8}), which can be read as in (\ref{osperan9}); (\ref{osperan10})--(\ref{osperan18}).

\begin{exe}
\ex ὥϲπερ ἄν, εἰ υἱὸϲ {[}...{]} διῴκει τι μὴ καλῶϲ ἢ ὀρθῶϲ, αὐτὸ μὲν τοῦτ᾽ ἄξιον μέμψεοϲ\\
\gll \emph{hṓsper} \emph{án}, ei huiòs diṓikei ti mḕ kalôs ḕ orthôs, autò mèn toût' áxion mémpseos\\
like \textsc{irr} if son.\textsc{nom.sg} administer.\textsc{3sg.imp}
something.\textsc{n.acc.sg} not well or straight same.\textsc{n.acc.sg} then this.\textsc{n.acc.sg} worthy.\textsc{n.acc.sg} blame.\textsc{gen.sg}\\
\trans `Just as, if a son's management were in some way not good or correct, this itself (would be) worthy of blame' (Demosthenes 9.30)\footnote{\emph{Translator's note}: The Perseus edition has \textit{kat' autò} for \textit{autò}.}
\label{osperan7}
\end{exe}

\begin{exe}
\ex χρὴ {[}...{]} ἀνθρωπίνωϲ περὶ τῶν πραγμάτων ἐκλογίζεϲθαι, ὥϲπερ ἂν αὐτὸν ὄντα ἐν τῇ ϲυμφορᾷ\\
\gll khrḕ anthrōpínōs perì tôn pragmátōn eklogízesthai, \emph{hṓsper} \emph{àn} autòn ónta en têi sumphorâi\\
need.\textsc{3sg.prs} humanely about the.\textsc{n.gen.pl} deed.\textsc{gen.pl} consider.\textsc{fut.inf.mid} like \textsc{irr} same.\textsc{m.acc.sg} be.\textsc{ptcp.prs.m.acc.sg} in the.\textsc{f.dat.sg} mishap.\textsc{dat.sg}\\
\trans `It is necessary to judge a case humanely, as if one were in those circumstances oneself.' (Andocides 1.57)
\label{osperan8}
\end{exe}

\begin{exe}
\ex ὥϲπερ ἄν τιϲ αὐτὸϲ ὢν {[}...{]} ἐκλογίζοιτο\\
\gll \emph{hṓsper} \emph{án} tis autòs ṑn eklogízoito\\
like \textsc{irr} someone.\textsc{m.nom.sg} same.\textsc{m.nom.sg}
be.\textsc{ptcp.prs.m.nom.sg} consider.\textsc{3sg.prs.opt.pass}\\
\trans `... as if one were judging, being oneself ...'
\label{osperan9}
\end{exe}

\begin{exe}
\ex τοῦτ᾽ αὐτὸ ἐπιδεικνύτω ὥϲπερ ἂν ὑμῶν ἕκαϲτοϲ\\
\gll toût' autò epideiknútō \emph{hṓsper} \emph{àn} humôn hékastos\\
this.\textsc{n.acc.sg} same.\textsc{n.acc.sg} display.\textsc{3sg.prs.imper} like \textsc{irr} you.\textsc{gen.pl} each.\textsc{m.nom.sg}\\
\trans `Let him demonstrate this very thing, as each of you would.'\\
(Isaeus 6.64)
\label{osperan10}
\end{exe}

\begin{exe}
\ex οὐδὲ {[}...{]} ὁμοίωϲ ὑμῖν, ὣϲπερ ἂν τρυτάνη ῥέπων ἐπὶ τὸ λῆμμα ϲυμβεβούλευκα\\
\gll oudè homoíōs humîn, \emph{hṑsper} \emph{àn} trutánē rhépōn epì tò lêmma sumbeboúleuka\\
nor similarly you.\textsc{dat.pl} like \textsc{irr} scale.\textsc{nom.sg} tilt.\textsc{ptcp.prs.m.nom.sg} upon the.\textsc{n.acc.sg} profit.\textsc{acc.sg} advise.\textsc{1sg.prf}\\
\trans `Nor, like you, have I advised as if I were a scale biased toward profit.' (Demosthenes 18.298; V. C. has \textit{hṓsper àn ei}, \citet{DindorfBlass1888} has only \textit{hṓsper})
\label{osperan11}
\end{exe}

\begin{exe}
\ex ὥϲπερ ἂν παρεϲτηκότοϲ αὐτοῦ\\
\gll \emph{hṓsper} \emph{àn} parestēkótos autoû\\
like \textsc{irr} stand.by.\textsc{ptcp.prf.m.gen.sg} him.\textsc{gen}\\
\trans `... as if with him standing beside' (Demosthenes 19.226)
\label{osperan12}
\end{exe}

\begin{exe}
\ex χρώμενοϲ ὥϲπερ ἂν ἄλλοϲ τιϲ αὐτῷ τὰ πρὸ τούτου\\
\gll khrṓmenos \emph{hṓsper} \emph{àn} állos tis autôi tà prò toútou\\
use.\textsc{ptcp.prs.pass.m.nom.sg} like \textsc{irr} other.\textsc{m.nom.sg} someone.\textsc{m.nom.sg} him.\textsc{dat} the.\textsc{n.acc.sg} before this.\textsc{n.gen.sg}\\
\trans `... as familiar as anyone could be with him up to then' (Demosthenes 21.117)
\label{osperan13}
\end{exe}

\begin{exe}
\ex δεῖ τοίνυν τούτοιϲ βοηθεῖν, ὥϲπερ ἂν αὑτῷ τιϲ ἀδικουμένῳ\\
\gll deî toínun toútois boētheîn, \emph{hṓsper} \emph{àn} hautôi tis adikouménōi\\
need.\textsc{3sg.prs} therefore this.\textsc{m.dat.pl} help.\textsc{prs.inf} like \textsc{irr} himself.\textsc{dat} someone.\textsc{m.nom.sg} wrong.\textsc{ptcp.prs.pass.m.dat.sg}\\
\trans `Therefore you must help these people, just as anyone (would help) himself if wronged.' (Demosthenes 21.225)
\label{osperan14}
\end{exe}

\begin{exe}
\ex ὥϲπερ ἄν τιϲ ϲυκοφαντεῖν ἐπιχειρῶν\\
\gll \emph{hṓsper} \emph{án} tis sukophanteîn epikheirôn\\
like \textsc{irr} someone.\textsc{m.nom.sg} defraud.\textsc{prs.inf} attempt.\textsc{ptcp.prs.m.nom.sg}\\
\trans `... as would someone attempting to deceive' (Demosthenes 29.30; see \citet[354]{DindorfBlass1888} following A; most have \emph{hṓsper àn eí tis}, with which reading the example below should be understood.)\footnote{\emph{Translator's note}: The Perseus edition has \textit{àn eí}, which Wackernagel cites as a variant.}
\label{osperan15}
\end{exe}

\begin{exe}
\ex πλὴν εἰ ϲημεῖον ὥϲπερ ἂν ἄλλῳ τινί, τῷ χαλκίῳ προϲέϲται\\
\gll plḕn ei sēmeîon \emph{hṓsper} \emph{àn} állōi tiní, tôi khalkíōi proséstai\\
except if sign.\textsc{nom.sg} like \textsc{irr} other.\textsc{n.dat.sg} something.\textsc{dat.sg} the.\textsc{n.dat.sg} copper.\textsc{dat.sg} be.added.\textsc{3sg.fut.mid}\\
\trans `... unless some mark shall be attached to the tablet, as there might be to anything else' (Demosthenes 39.10)
\label{osperan16}
\end{exe}

\begin{exe}
\ex ὥϲπερ ἂν δοῦλοϲ δεϲπότῃ διδούϲ\\
\gll \emph{hṓsper} \emph{àn} doûlos despótēi didoús\\
like \textsc{irr} slave.\textsc{nom.sg} master.\textsc{dat.sg} give.\textsc{ptcp.prs.m.nom.sg}\\
\trans `... as a slave giving to his master' (Demosthenes 45.35)
\label{osperan17}
\end{exe}

\begin{exe}
\ex ὥϲπερ ἂν ἄλλοϲ τιϲ ἀποτυχών\\
\gll \emph{hṓsper} \emph{àn} állos tis apotukhṓn\\
as \textsc{irr} other.\textsc{m.nom.sg} someone.\textsc{m.nom.sg} miss.\textsc{ptcp.aor.m.nom.sg}\\
\trans `... as another might who failed to obtain what he wanted ...' (Demosthenes 49.27)
\label{osperan18}
\end{exe}

This is often found with a following \emph{ei} with \isi{optative} \hyperlink{p386}{\emph{[p386]}} or \isi{preterite} verb: (\ref{osperanei9})--(\ref{osperanei10}) and see 10.10, 15.2, 15.14, and 15.298 from Isocrates.

\begin{exe}
\ex ὥϲπερ ἂν εἰ πρὸϲ ἅπανταϲ ἀνθρώπουϲ ἐπολέμηϲαν\\
\gll \emph{hṓsper} \emph{àn} \emph{ei} pròs hápantas anthrṓpous epolémēsan\\
like \textsc{irr} if to quite.all.\textsc{m.acc.pl} person.\textsc{acc.pl} war.\textsc{3pl.aor}\\
\trans `... as if they had fought the whole world.' (Isocrates 4.69)
\label{osperanei9}
\end{exe}

\begin{exe}
\ex ὥϲπερ ἂν εἴ τῳ Φρυνώνδαϲ πανουργίαν ὀνειδίϲειεν\\
\gll \emph{hṓsper} \emph{àn} \emph{eí} tōi Phrunṓndas panourgían oneidíseien\\
like \textsc{irr} if someone.\textsc{m.dat.sg} Phrynondas.\textsc{nom}
villainy.\textsc{acc.sg} reproach.\textsc{3sg.aor.opt}\\
\trans `... as if Phrynondas should reproach someone with villainy' (Isocrates 18.57)
\label{osperanei10}
\end{exe}

The same is found in (\ref{osperanei11}) and (\ref{osperanei12}) from Plato. Cf. \textit{Cratylus} 430a, \textit{Gorgias} 479a, \textit{Phaedo} 98c and 109c, \textit{Symposium} 199d and 204e, \textit{Republic} 7.529d, etc.

\begin{exe}
\ex ὥϲπερ ἂν εἰ ἤκουεν\\
\gll \emph{hṓsper} \emph{àn} \emph{ei} ḗkouen\\
like \textsc{irr} if hear.\textsc{3sg.imp}\\
\trans `... as if he heard ...' (Plato, \textit{Protagoras} 341c)
\label{osperanei11}
\end{exe}

\begin{exe}
\ex ὥϲπερ ἂν εἴ τιϲ {[}...{]} ὀνομάϲειε καὶ εἴποι\\
\gll \emph{hṓsper} \emph{àn} \emph{eí} tis onomáseie kaì eípoi\\
like \textsc{irr} if someone.\textsc{m.nom.sg} name.\textsc{3sg.aor.opt} and say.\textsc{3sg.aor.opt}\\
\trans `... as if someone were to call and say ...' (Plato, \textit{Cratylus} 395e)
\label{osperanei12}
\end{exe}

The same is found in (\ref{osperanei13}) from Xenophon.

\begin{exe}
\ex ἠϲπάζετο αὐτόν, ὥϲπερ ἂν εἴ τιϲ {[}...{]} ἀϲπάζοιτο\\
\gll ēspázeto autón, \emph{hṓsper} \emph{àn} \emph{eí} tis aspázoito\\
welcome.\textsc{3sg.imp.pass} him.\textsc{acc} like \textsc{irr} if someone.\textsc{m.nom.sg} welcome.\textsc{3sg.prs.opt.pass}\\
\trans `He kissed him, just as someone would ...' (Xenophon, \textit{Cyropaedia} 1.3.2)\footnote{\emph{Translator's note}: The Perseus edition adds \textit{te} after the initial verb.}
\label{osperanei13}
\end{exe}

The same is true of Demosthenes ((\ref{osperanei14})--(\ref{osperanei15}); cf. §243) and other orators, (\ref{osperanei16}).

\begin{exe}
\ex ὥϲπερ ἂν εἰ πολεμοῦντεϲ τύχοιτε\\
\gll \emph{hṓsper} \emph{àn} \emph{ei} polemoûntes túkhoite\\
like \textsc{irr} if war.\textsc{ptcp.prs.m.nom.pl} happen.\textsc{2pl.aor.opt}\\
\trans `... as if you happened to be at war.' (Demosthenes 6.8)
\label{osperanei14}
\end{exe}

\begin{exe}
\ex ὥϲπερ ἂν εἴ τιϲ ναύκληρον {[}...{]} αἰτιῷτο\\
\gll \emph{hṓsper} \emph{àn} \emph{eí} tis naúklēron aitiôito\\
like \textsc{irr} if some.\textsc{m.nom.sg} shipowner.\textsc{nom.sg} accuse.\textsc{3sg.prs.opt.pass}\\
\trans `As if some shipowner were to be accused ...' (Demosthenes 18.194)
\label{osperanei15}
\end{exe}

\begin{exe}
\ex ὥϲπερ ἂν εἴ τιϲ εἰϲ Αἴγιναν ἢ εἰϲ Μέγαρα ὁρμίϲαιτο\\
\gll \emph{hṓsper} \emph{àn} \emph{eí} tis eis Aíginan ḕ eis Mégara hormísaito\\
like \textsc{irr} if someone.\textsc{m.nom.sg} into Aegina.\textsc{acc} or into Megara.\textsc{acc} anchor.\textsc{3sg.aor.opt.mid}\\
\trans `... as if one were to anchor in Aegina or Megara' ({[}Demosthenes{]} 35.28)
\label{osperanei16}
\end{exe}

In addition to this we find the sequence \emph{hṓsper àn ei} (usually written \emph{hōsperaneí}) in the sense of \emph{quasi} `how', cf. \emph{ōseí}, \emph{ōspereí}, without use of a finite verb, e.g. (\ref{osperanei17}), Isocrates 4.148, Xenophon, \textit{Symposium} 9.4, and Demosthenes 18.194.\footnote{\emph{Translator's note}: In these last two instances the Perseus edition has \textit{hṓsper àn ei} written separately.} On the use of \emph{hōsperaneí} and \emph{kathaperaneí} in Aristotle, see \citet[41]{Bonitz1870}.

\begin{exe}
\ex ὡϲπερανεὶ παῖϲ\\
\gll \emph{hōsperaneì} paîs\\
as.if child.\textsc{nom.sg}\\
\trans `like a child' (Plato, \textit{Gorgias} 479a)
\label{osperanei17}
\end{exe}

Relative\is{relative clauses|(} clauses also provide occasion for comment. First, in the sequence \emph{ouk éstin hóstis} `not be.\textsc{3sg.prs} who' (or also interrogative\is{interrogatives} \emph{éstin hóstis ...} `be.\textsc{3sg.prs} who ...'), in which the main clause only receives its content from the subordinate clause and hence the connection between the two clauses is particularly close, \emph{án} regularly follows the relativizer:\is{relative pronouns} (\ref{relan1})--(\ref{relan10}); cf. also (\ref{relan11}).

\begin{exe}
\ex οὐκ ἔϲτ᾽ ἀδελφόϲ, ὅϲτιϲ ἂν βλάϲτοι ποτέ\\
\gll \emph{ouk} \emph{ést'} adelphós, \emph{hóstis} \emph{àn} blástoi poté\\
not be.\textsc{3sg.prs} brother.\textsc{nom.sg} who.\textsc{m.nom.sg}
\textsc{irr} bud.\textsc{3sg.aor.opt} sometime\\
\trans `There is no brother that could ever bloom for me' (Sophocles, \textit{Antigone} 912)
\label{relan1}
\end{exe}

\begin{exe}
\ex οὐκ ἔϲτιν οὐδεὶϲ ὅϲτιϲ ἂν μέμψαιτό ϲε\\
\gll \emph{ouk} \emph{éstin} oudeìs \emph{hóstis} \emph{àn} mémpsaitó se\\
not be.\textsc{3sg.prs} nobody.\textsc{m.nom.sg} who.\textsc{m.nom.sg} \textsc{irr} blame.\textsc{3sg.aor.opt.mid} you.\textsc{acc}\\
\trans `There is no one who would blame you.' (Euripides, \textit{Electra} 903; cf. also \textit{Heracleidae} 972)
\label{relan2}
\end{exe}

\begin{exe}
\ex οὐκ ἔϲτιν εἰϲ ὅ τι ἂν ἀναγκαιότερον ἀναλίϲκοιτε χρήματα\\
\gll \emph{ouk} \emph{éstin} eis \emph{hó.ti} \emph{àn} anankaióteron analískoite khrḗmata\\
not be.\textsc{3sg.prs} into which.\textsc{n.acc.sg} \textsc{irr} necessary.\textsc{comp.n.acc.sg} spend.\textsc{2pl.prs.opt} property.\textsc{acc.pl}\\
\trans `There is nothing more necessary on which you could spend your money.' (Plato, \textit{Phaedo} 78a)\footnote{\emph{Translator's note}: The Perseus edition has \textit{eukairóteron} for \textit{anankaióteron}.}
\label{relan3}
\end{exe}

\begin{exe}
\ex οὐκ ἔϲτιν {[}...{]}, ὅτι ἄν τιϲ μεῖζον {[}...{]} πάθοι\\
\gll \emph{ouk} \emph{éstin}, \emph{hóti} \emph{án} tis meîzon páthoi\\
not be.\textsc{3sg.prs} which.\textsc{n.acc.sg} \textsc{irr} someone.\textsc{m.nom.sg} greater.\textsc{n.acc.sg} suffer.\textsc{3sg.aor.opt}\\
\trans `There is nothing greater that one can suffer ...' (Plato, \textit{Phaedo} 89d)
\label{relan4}
\end{exe}

\begin{exe}
\ex τουτωνὶ {[}...{]} οὐκ ἔϲτιν, ἅττ᾽ ἂν ἐμοὶ εἶπεϲ ἡδίω\\
\gll toutōnì \emph{ouk} \emph{éstin}, \emph{hátt'} \emph{àn} emoì eîpes hēdíō\\
this.\textsc{n.gen.pl.emph} not be.\textsc{3sg.prs} which.\textsc{n.acc.pl} \textsc{irr} me.\textsc{dat} say.\textsc{2sg.aor} sweeter.\textsc{n.acc.pl}\\
\trans `There is nothing more pleasant than this that you could say to me.' (Plato, \textit{Phaedrus} 243b)
\label{relan5}
\end{exe}

\begin{exe}
\ex οἶμαι γὰρ τοιοῦτον οὐδὲν εἶναι, ὅτου ἂν ἀπέϲχετο\\
\gll oîmai gàr toioûton \emph{oudèn} \emph{eînai}, \emph{hótou} \emph{àn} apéskheto\\
think.\textsc{1sg.prs.pass} then such.\textsc{m.acc.sg} nothing.\textsc{acc.sg} be.\textsc{prs.inf} which.\textsc{n.gen.sg} \textsc{irr} keep.off.\textsc{3sg.aor.mid}\\
\trans `For I think that there is nothing from which such a person would have kept his hands.' (Demosthenes 24.138)
\label{relan6}
\end{exe}

\begin{exe}
\ex ἔϲτιν, ὅϲτιϲ ἂν {[}...{]} ἐψήφιϲεν {[}...{]};\\
\gll \emph{éstin}, \emph{hóstis} \emph{àn} epsḗphisen?\\
be.\textsc{3sg.prs} who.\textsc{m.nom.sg} \textsc{irr} vote.\textsc{3sg.aor}\\
\trans `Is there anyone who would have voted ... ?' (Demosthenes 24.157)\footnote{\emph{Translator's note}: The Perseus edition has \textit{epipsḗphisen} for \textit{epsḗphisen}.}
\label{relan7}
\end{exe}

\begin{exe}
\ex ἔϲτιν, ὅϲτιϲ ἂν {[}...{]} ὑπέμεινεν {[}...{]};\\
\gll \emph{éstin}, \emph{hóstis} \emph{àn} hupémeinen?\\
be.\textsc{3sg.prs} who.\textsc{m.nom.sg} \textsc{irr} abide.\textsc{3sg.aor}\\
\trans `Is there anyone who could bear ... ?' (Demosthenes 19.309)\footnote{\emph{Translator's note}: The Perseus edition has \textit{ésth' hóstis} for \textit{éstin, hóstis}.}
\label{relan8}
\end{exe}

\begin{exe}
\ex οὐ γὰρ ἦν, ὅ τι ἂν ἐποιεῖτε\\
\gll \emph{ou} gàr \emph{ên}, \emph{hó.ti} \emph{àn} epoieîte\\
not then be.\textsc{3sg.imp} which.\textsc{n.acc.sg} \textsc{irr} do.\textsc{2pl.imp}\\
\trans `For there was nothing that you could do.' (Demosthenes 18.43)
\label{relan9}
\end{exe}

\begin{exe}
\ex ἔϲτιν οὖν, ὅϲτιϲ ἂν τοῦ ξύλου καὶ τοῦ χωρίου {[}...{]} τοϲαύτην ὑπέμεινε φέρειν μίϲθωϲιν; ἔϲτι δ᾽ ὅϲτιϲ ἂν {[}...{]} ἐπέτρεψεν;\\
\gll \emph{éstin} \emph{oûn}, \emph{hóstis} \emph{àn} toû xúlou kaì toû khōríou tosaútēn hupémeine phérein místhōsin? ésti d' hóstis àn epétrepsen?\\
be.\textsc{3sg.prs} so who.\textsc{m.nom.sg} \textsc{irr} the.\textsc{n.gen.sg} wood.\textsc{gen.sg} and the.\textsc{n.gen.sg} place.\textsc{gen.sg} so.much.\textsc{f.acc.sg} abide.\textsc{3sg.aor} bear.\textsc{prs.inf} rent.\textsc{acc.sg} be.\textsc{3sg.prs} then who.\textsc{m.nom.sg} \textsc{irr}
entrust.\textsc{3sg.aor}\\
\trans `Now, is there any man who would have submitted to the payment of so large a rental for the counter and the site? And is there any man who would have entrusted ... ?' (Demosthenes 45.33)
\label{relan10}
\end{exe}

\begin{exe}
\ex οὐκ ἔϲτ᾽ οὐδείϲ, ὅϲτιϲ ἂν εἴποι\\
\gll \emph{ouk} \emph{ést'} oudeís, \emph{hóstis} \emph{àn} eípoi\\
not be.\textsc{3sg.prs} nobody.\textsc{m.nom.sg} who.\textsc{m.nom.sg} \textsc{irr} say.\textsc{3sg.aor.opt}\\
\trans `There is no one who would say ...' ({[}Demosthenes{]} 13.22)\footnote{\emph{Translator's note}: The Perseus edition has \textit{oud'} for \textit{ouk}.}
\label{relan11}
\end{exe}

Almost on the same level as \emph{ouk éstin hóstis} are such phrasings as we find in (\ref{relan12}) or in (\ref{relan13}) and in (\ref{relan14}).

\begin{exe}
\ex οὐ γὰρ ἴδοιϲ ἂν ἀθρῶν βροτῶν ὅϲτιϲ ἂν εἰ θεὸϲ ἄγοι ἐκφυγεῖν δύναιτο\\
\gll \emph{ou} gàr ídois àn athrôn brotôn \emph{hóstis} \emph{àn} ei theòs ágoi ekphugeîn dúnaito\\
not then see.\textsc{2sg.aor.opt} \textsc{irr} observe.\textsc{ptcp.prs.m.nom.sg} mortal.\textsc{gen.pl} who.\textsc{m.nom.sg} \textsc{irr} if god.\textsc{nom.sg} lead.\textsc{3sg.prs.opt} escape.\textsc{prs.inf} can.\textsc{3sg.prs.opt.pass}\\
\trans `If you observed, you would not see any mortal who could escape if a god were to lead him on.' (Sophocles, \textit{Oedipus at Colonus} 252)
\label{relan12}
\end{exe}

\begin{exe}
\ex οὐκ οἶδα εἰϲ ὅντιν᾽ ἄν τιϲ ἄλλον καιρὸν ἀναβάλλοιτο\\
\gll \emph{ouk} oîda eis \emph{hóntin'} \emph{án} tis állon kairòn anabálloito\\
not know.\textsc{1sg.prf} into which.\textsc{m.nom.sg} \textsc{irr} someone.\textsc{m.nom.sg} other.\textsc{m.acc.sg} time.\textsc{acc.sg} defer.\textsc{3sg.prs.opt.pass}\\
\trans `I do not know to what other time one could delay' (Plato, \textit{Phaedo} 107a)
\label{relan13}
\end{exe}

\begin{exe}
\ex οὐκ οἶδα ὅ τι ἄν τιϲ χρήϲαιτο αὐτῷ\\
\gll \emph{ouk} oîda \emph{hó.ti} \emph{án} tis khrḗsaito autôi\\
not know.\textsc{1sg.prf} which.\textsc{n.acc.sg} \textsc{irr}
someone.\textsc{m.nom.sg} use.\textsc{3sg.aor.opt.mid} him.\textsc{dat}\\
\trans `I do not know what use one could make of him' (Xenophon, \textit{Anabasis} 3.1.40)\footnote{\emph{Translator's note}: The Perseus edition has \textit{autoîs} for \textit{autôi}.}
\label{relan14}
\end{exe}

And the connection between main clause and subordinate clause \hyperlink{p387}{\emph{[p387]}} is just as tight as in these examples when \emph{hóstis} is announced by \emph{hoútō}: (\ref{relan15}).

\begin{exe}
\ex οὐδεὶϲ γάρ ἐϲτιν οὕτω ῥᾴθυμοϲ ὅϲτιϲ ἂν δέξαιτο\\
\gll \emph{oudeìs} gár estin \emph{hoútō} rhā́ithumos \emph{hóstis} \emph{àn} déxaito\\
nobody.\textsc{m.nom.sg} for be.\textsc{3sg.prs} so indifferent.\textsc{m.nom.sg} who.\textsc{m.nom.sg} \textsc{irr} receive.\textsc{3sg.aor.opt.mid}\\
\trans `For there is no one so cavalier that he would receive ...' (Isocrates 9.35)
\label{relan15}
\end{exe}

The connection between \emph{hóstis} and \emph{án} can, however, be interrupted, first by \emph{pote} `sometime', which is quite natural: (\ref{relan16}), secondly by \emph{ouk} `not': (\ref{relan17})--(\ref{relan20}). (Cf. (\ref{relan21}).)

\begin{exe}
\ex τῶν δὲ κατὰ ταῦτα ἐχόντων οὐκ ἔϲτιν ὅτῳ ποτ᾽ ἂν ἄλλῳ ἐπιλάβοιο\\
\gll tôn dè katà taûta ekhóntōn \emph{ouk} \emph{éstin} \emph{hótōi} pot' \emph{àn} állōi epiláboio\\
the.\textsc{n.gen.pl} but down this.\textsc{n.acc.pl} have.\textsc{ptcp.prs.n.gen.pl} not be.\textsc{3sg.prs} which.\textsc{n.dat.sg} sometime \textsc{irr} other.\textsc{n.dat.sg} grasp.\textsc{2sg.aor.opt.mid}\\
\trans `But there is nothing else by which you could grasp the things that are always the same ...' (Plato, \textit{Phaedo} 79a)\footnote{\emph{Translator's note}: The Perseus edition has \textit{t'autà}, with crasis, for \textit{taûta}.}
\label{relan16}
\end{exe}

\begin{exe}
\ex ὧν οὐκ ἔϲτιν, ὅϲτιϲ οὐκ ἄν τιϲ καταφρονήϲειεν\\
\gll hôn \emph{ouk} \emph{éstin}, \emph{hóstis} ouk \emph{án} tis kataphronḗseien\\
whom.\textsc{m.gen.pl} not be.\textsc{3sg.prs} who.\textsc{m.nom.sg} not \textsc{irr} someone.\textsc{m.nom.sg} despise.\textsc{3sg.aor.opt}\\
\trans `... whom there is no one that would fail to despise' (Isocrates 8.52)\footnote{\emph{Translator's note}: The Perseus edition lacks \textit{tis}.}
\label{relan17}
\end{exe}

\begin{exe}
\ex οὐ γάρ ἐϲτιν, περὶ ὅτου οὐκ ἂν πιθανώτερον εἴποι ὁ ῥητορικόϲ\\
\gll \emph{ou} gár \emph{estin}, perì \emph{hótou} ouk \emph{àn} pithanṓteron eípoi ho rhētorikós\\
not for be.\textsc{3sg.prs} about which.\textsc{n.gen.sg} not \textsc{irr} persuasively.\textsc{comp} say.\textsc{3sg.aor.opt} the.\textsc{m.nom.sg}
rhetorician.\textsc{nom.sg}\\
\trans `There is nothing about which a rhetorician would not speak more persuasively' (Plato, \textit{Gorgias} 456c; cf. also 491e)
\label{relan18}
\end{exe}

\begin{exe}
\ex οὐδεὶϲ οὕτω κακόϲ, ὅντινα οὐκ ἂν αυτὸϲ ὁ Ἔρωϲ ἔνθεον ποιήϲειεν\\
\gll \emph{oudeìs} \emph{hoútō} kakós, \emph{hóntina} ouk \emph{àn} autòs ho Érōs éntheon poiḗseien\\
nobody.\textsc{m.nom.sg} so bad.\textsc{m.nom.sg} whom.\textsc{m.acc.sg} not \textsc{irr} same.\textsc{m.nom.sg} the.\textsc{m.nom.sg} Eros.\textsc{nom}
inspired.\textsc{m.acc.sg} make.\textsc{3sg.aor.opt}\\
\trans `(There is) no one so base whom Eros himself cannot inspire' (Plato, \textit{Symposium} 179a)
\label{relan19}
\end{exe}

\begin{exe}
\ex οὐδεὶϲ γάρ, ὅϲτιϲ οὐκ ἂν ἀξιώϲειεν\\
\gll \emph{oudeìs} gár, \emph{hóstis} ouk \emph{àn} axiṓseien\\
nobody.\textsc{m.nom.sg} then who.\textsc{m.nom.sg} not \textsc{irr} deem.\textsc{3sg.aor.opt}\\
\trans `There is no one who would think ...' (Xenophon, \textit{Cyropaedia} 7.5.61)
\label{relan20}
\end{exe}

\begin{exe}
\ex τίϲ ὅυτωϲ {[}...{]} φθονερόϲ ἐϲτιν {[}...{]} ὅϲ οὐκ ἂν εὔξαιτο {[}...{]};\\ 
\gll \emph{tís} hóutōs phthonerós \emph{estin} \emph{hós} ouk \emph{àn} eúxaito\\
who.\textsc{m.nom.sg} so envious.\textsc{m.nom.sg} be.\textsc{3sg.prs} who.\textsc{m.nom.sg} not \textsc{irr} pray.\textsc{3sg.aor.opt.mid}\\
\trans `Who is so envious that he would not have prayed ... ?' (Lycurgus 1.69)
\label{relan21}
\end{exe}

Note that none of the examples with immediately adjacent \emph{hóstis án} contain \isi{negation} in the relative clause, so that the insertion of \emph{ouk} can be said to be a rule. This is also not at all surprising: compare what was observed above on p\pageref{ouk1}, p\pageref{ouk2} and p\pageref{ouk3} on the placement of \emph{ouk} before \isi{enclitics} and on p\pageref{ouk4} on Homeric \emph{ouk án}. Demosthenes 18.206 is peculiar. Here the best source texts, S and L, give (\ref{relan22}). If the transmission is correct, the expression is based on a contamination driven by the need to conform to the usual sequences \emph{hóstis án} and (\emph{hóstis}) \emph{ouk án}.

\begin{exe}
\ex οὐκ ἔϲθ᾽ ὅϲτιϲ ἂν οὐκ ἂν εἰκότωϲ ἐπιτιμήϲειέ μοι\\
\gll ouk ésth' hóstis àn ouk àn eikótōs epitimḗseié moi\\
not be.\textsc{3sg.prs} who.\textsc{m.nom.sg} \textsc{irr} not \textsc{irr}
justly evaluate.\textsc{3sg.aor.opt} me.\textsc{dat}\\
\trans `There is no one who would not justly censure me.' (Demosthenes 18.206)\footnote{\emph{Translator's note}: The Perseus edition lacks the first \textit{àn}.}
\label{relan22}
\end{exe}

The words \emph{àn ouk àn} are also found immediately adjacent in Sophocles, \textit{Oedipus Rex} 446, \textit{Electra} 439, \textit{Oedipus at Colonus} 1366, Fragment 673, Euripides, \textit{Heracleidae} 74,\footnote{\emph{Translator's note}: Not found in Perseus edition} and Aristophanes, \textit{Lysistrata} 361. And \emph{àn oud' àn} in Sophocles, \textit{Electra} 97 (more common, and still found in Aristotle, is \emph{àn} ... \emph{ouk àn} or \emph{oudeìs án} separated by several words). Since in any case the sequence \emph{àn ouk àn} seems to be unknown in the fourth century\il{Greek, Classical} and the repetition of \emph{án} is only found after a lot of intervening material, the editors who delete the first \emph{án} and simply write \emph{hóstis ouk án} are perhaps right to do so.

Good Attic\il{Greek, Attic} poets\is{poetry} do not separate \emph{hóstis} and \emph{án} by words other than \emph{pote} `sometime' or \emph{ou} `not'. Admittedly, Xenophon writes (\ref{relan23}) and (\ref{relan24}).

\begin{exe}
\ex οὔτ᾽ ἔϲτιν ὅτου ἕνεκα βουλοίμεθα ἂν τὴν βαϲιλέωϲ χώραν κακῶϲ ποιεῖν\\
\gll oút' éstin \emph{hótou} héneka bouloímetha \emph{àn} tḕn basiléōs khṓran kakôs poieîn\\
nor be.\textsc{3sg.prs} which.\textsc{n.gen.sg} because.of wish.\textsc{1pl.prs.opt.pass} \textsc{irr} the.\textsc{f.acc.sg} king.\textsc{gen.sg} country.\textsc{acc.sg} ill do.\textsc{prs.inf}\\
\trans `... nor is there any reason why we should desire to do harm to the King's territory' (Xenophon, \textit{Anabasis} 2.3.23)
\label{relan23}
\end{exe}

\hyperlink{p388}{\emph{[p388]}}

\begin{exe}
\ex ἔϲτιν οὖν ὅϲτιϲ τοῦτο ἂν δύναιτο ὑμᾶϲ ἐξαπατῆϲαι\\
\gll éstin oûn \emph{hóstis} toûto \emph{àn} dúnaito humâs exapatêsai\\
be.\textsc{3sg.prs} so who.\textsc{m.nom.sg} this.\textsc{n.acc.sg} \textsc{irr} can.\textsc{prs.opt.pass} you.\textsc{pl.acc} deceive.\textsc{aor.inf}\\
\trans `Therefore, is there anyone who could deceive you in this ... ?' (Xenophon, \textit{Anabasis} 5.7.6)
\label{relan24}
\end{exe}

Strikingly, (\ref{relan25}) is similar.

\begin{exe}
\ex τίϲ δ᾽ ἦν οὗτω ἢ μιϲόδημοϲ τότε ἢ μιϲαθήναιοϲ, ὅϲτιϲ έδυνήθη ἄν\\
\gll tís d' ên hoûtō ḕ misódēmos tóte ḕ misathḗnaios, \emph{hóstis} édunḗthē \emph{án}\\
who.\textsc{m.nom.sg} then be.\textsc{3sg.imp} so or people-hating.\textsc{m.nom.sg} then or Athens-hating.\textsc{m.nom.sg} who.\textsc{m.nom.sg} can.\textsc{3sg.aor.pass} \textsc{irr}\\
\trans `And was there anyone then who hated either the people or Athens so much that he could have ... ?' (Lycurgus 1 39)
\label{relan25}
\end{exe}

Perhaps the observation by \citet[103]{Blass1880} is also applicable here: ``what strikes one {[}in Lycurgus{]} as non-classical or ungrammatical must be blamed on its acknowledged poor transmission.'' But in Blass's text for (\ref{relan26}) the \emph{állo} `other' is pure editorial conjecture. (However, see (\ref{relan27}). Read \textit{hḗtis àn tód'}?)

\begin{exe}
\ex οὐ γὰρ ἦν ὅ τι ἄλλ᾽ ἂν ἐποιεῖτε\\
\gll ou gàr ên \emph{hó.ti} áll' \emph{àn} epoieîte\\
not then be.\textsc{3sg.imp} which.\textsc{n.acc.sg} other.\textsc{n.acc.sg} \textsc{irr} do.\textsc{2pl.imp}\\
\trans `For there was nothing else that you could do.' (Demosthenes 18.43; \citealp{Blass1877})\footnote{\emph{Translator's note}: The Perseus edition lacks Blass's\ia{Blass, Friedrich} \textit{áll'}.}
\label{relan26}
\end{exe}

\begin{exe}
\ex οὐκ ἔϲτιν, ἥτιϲ τοῦτ᾽ ἂν Ἑλληνὶϲ γυνὴ ἔτλη\\
\gll ouk éstin, \emph{hḗtis} toût' \emph{àn} Hellēnìs gunḕ étlē\\
not be.\textsc{3sg.prs} who.\textsc{f.nom.sg} this.\textsc{n.acc.sg} \textsc{irr} Greek.\textsc{f.nom.sg} woman.\textsc{nom.sg} endure.\textsc{3sg.aor}\\
\trans `There is no Greek woman who would have dared this.' (Euripides, \textit{Medea} 1339)
\label{relan27}
\end{exe}

The tradition was less stable in clauses containing one of the relative \isi{adjectives} or \isi{adverbs}\is{relative pronouns} related to \emph{hóstis}, and in clauses where \emph{hóstis} itself was attached to a negative\is{negation} clause but was not absolutely necessary for its interpretation and therefore not so closely attached to it. From the first category we have (\ref{relan28}) (non-negative\is{negation} interrogative!)\is{interrogatives} and (\ref{relan29})--(\ref{relan33}).

\begin{exe}
\ex ἔϲτ᾽ οὖν ὅπωϲ ἂν ὡϲπερεὶ ϲπονδῆϲ θεοῦ κἀγὼ λαβοίμην {[}...{]};\\
\gll ést' oûn \emph{hópōs} \emph{àn} hōspereì spondês theoû kagṑ laboímēn\\
be.\textsc{3sg.prs} so how \textsc{irr} as.if libation.\textsc{gen.sg} god.\textsc{gen.sg} also=I.\textsc{nom} take.\textsc{1sg.aor.opt.mid}\\
\trans `Then is there any way in which, as with a libation to a god, I too could take ... ?' (Euripides, \textit{Cyclops} 469)
\label{relan28}
\end{exe}

\begin{exe}
\ex οὐκ ἔϲτιν ὅπωϲ ἂν ἐγώ ποθ᾽ ἑκὼν τῆϲ ϲῆϲ γνώμηϲ ἔτ᾽ ἀφείμην\\
\gll ouk éstin \emph{hópōs} \emph{àn} egṓ poth' hekṑn tês sês gnṓmēs ét' apheímēn\\
not be.\textsc{3sg.prs} how \textsc{irr} I.\textsc{nom} sometime willing.\textsc{m.nom.sg} the.\textsc{f.gen.sg} your.\textsc{f.gen.sg} opinion.\textsc{gen.sg} still discard.\textsc{1sg.aor.mid}\\
\trans `There is no way that I would ever again willingly ignore your advice.' (Aristophanes, \textit{Birds} 627)
\label{relan29}
\end{exe}

\begin{exe}
\ex οὐδὲν αὐτὸϲ ἐξηῦρον, ὁπόθεν ἂν εἰκότωϲ ὑπερείδετε τὴν ἐμὴν ὁμιλίαν\\
\gll oudèn autòs exēûron, \emph{hopóthen} \emph{àn} eikótōs hupereídete tḕn emḕn homilían\\
nothing.\textsc{acc.sg} same.\textsc{m.nom.sg} discover.\textsc{1sg.aor} whence \textsc{irr} justly despise.\textsc{2pl.aor} the.\textsc{f.acc.sg} my.\textsc{acc.sg} company.\textsc{acc.sg}\\
\trans `I myself have discovered nothing from which you could reasonably have despised my company.' (Lysias 8.7)
\label{relan30}
\end{exe}

\begin{exe}
\ex οὐκ ἔϲτιν, ὅπωϲ ἂν ἄμεινον οἰκήϲειαν τὴν ἑαυτῶν\\
\gll ouk éstin, \emph{hópōs} \emph{àn} ámeinon oikḗseian tḕn heautôn\\
not be.\textsc{3sg.prs} how \textsc{irr} better settle.\textsc{3pl.aor.opt} the.\textsc{f.acc.sg} themselves.\textsc{gen}\\
\trans `There is no way in which they could be better citizens of their country' (Plato, \textit{Symposium} 178e)
\label{relan31}
\end{exe}

\begin{exe}
\ex οὐκ ἔϲθ᾽ ὅπωϲ ἂν ἐνθάδε μείναιμι\\
\gll ouk ésth' \emph{hópōs} \emph{àn} entháde meínaimi\\
not be.\textsc{3sg.prs} how \textsc{irr} here stay.\textsc{1sg.aor.opt}\\
\trans `There is no way that I could stay here.' (Plato, \textit{Symposium} 223a)
\label{relan32}
\end{exe}

\begin{exe}
\ex οὐκ εἶναι ἔθνοϲ, ὁποίῳ ὰν ἀξιώϲειαν ὑπήκοοι εἶναι Θετταλοί\\
\gll ouk eînai éthnos, \emph{hopoíōi} \emph{àn} axiṓseian hupḗkooi eînai Thettaloí\\
not be.\textsc{prs.inf} people.\textsc{nom.sg} of.what.sort.\textsc{n.dat.sg} \textsc{irr} deem.\textsc{3pl.aor.opt} subject.\textsc{m.nom.pl} be.\textsc{prs.inf}
Thessalian.\textsc{nom.pl}\\
\trans `... that there would be no people such that the Thessalians would consider being subject to them.' (Xenophon, \textit{Hellenica} 6.1.9)
\label{relan33}
\end{exe}

We also have (\ref{relan34}) (although the revisor of Codex S has added a second \emph{án} above \emph{tis}, it is not legitimate to delete the \emph{án} after \emph{hópōs}, which is absent only in Augustanus, and transpose it to after \emph{enantiṓterá}, as done by \citealp[103]{Weil1886} and, following him, \citealp{DindorfBlass1888}), and (\ref{relan35}) (cf. also \emph{ouk oîd'}, \emph{hópōs àn} -- above p\pageref{oposan}).

\begin{exe}
\ex ἔϲτιν οὖν ὅπωϲ ἂν ἐναντιώτερά τιϲ δύο θείη\\
\gll éstin oûn \emph{hópōs} \emph{àn} enantiṓterá tis dúo theíē\\
be.\textsc{3sg.prs} so how \textsc{irr} opposite.\textsc{comp.n.acc.pl} someone.\textsc{m.nom.sg} two put.\textsc{3sg.aor.opt}\\
\trans `So is there any way in which one could propose two more contradictory things ... ?' (Demosthenes 24.64)
\label{relan34}
\end{exe}

\begin{exe}
\ex ἔϲτιν οὖν ὅπωϲ ἂν μᾶλλον ἄνθρθποι πάνθ᾽ ὑπὲρ Φιλίππου πράττοντεϲ ἐξελεγχθεῖεν\\
\gll éstin oûn \emph{hópōs} \emph{àn} mâllon ánthrthpoi pánth' hupèr Philíppou práttontes exelenkhtheîen\\
be.\textsc{3sg.prs} so how \textsc{irr} more person.\textsc{nom.pl} all.ways over Philip.\textsc{gen} do.\textsc{ptcp.prs.m.nom.pl} convict.\textsc{3pl.aor.opt.pass}\\
\trans `Now is there any way in which people could be more clearly convicted of acting for Philip in every way ... ?' (Demosthenes 19.165)
\label{relan35}
\end{exe}

These examples are not contradicted by (\ref{relan36}), and probably not by (\ref{relan37}); but the following are genuine counterexamples: (\ref{relan38})--(\ref{relan42}) and (\ref{relan43}) (for which sparser manuscripts have \emph{hópōs àn taût'}).

\begin{exe}
\ex οὐκ ἔχω {[}...{]} ὅκωϲ οὐκ ἂν ἴϲον πλῆθοϲ τοίϲ Πέρϲῃϲι ἐξέβαλε\\
\gll ouk ékhō \emph{hókōs} ouk \emph{àn} íson plêthos toís Pérsēisi exébale\\
not have.\textsc{1sg.prs} how not \textsc{irr} equal.\textsc{n.acc.sg} quantity.\textsc{acc.sg} the.\textsc{m.dat.pl} Persian.\textsc{dat.pl} cast.out.\textsc{3sg.aor}\\
\trans `I hold that there is no way in which he would have cast overboard a number equal to that of the Persians ...' (Herodotus 8.119.1)
\label{relan36}
\end{exe}

\begin{exe}
\ex τοῦτ᾽ οὖν ἐϲτιν ὅπωϲ τιϲ ἂν ὑμᾶϲ ἐξαπατήϲαι\\
\gll toût' oûn estin \emph{hópōs} tis \emph{àn} humâs exapatḗsai\\
this.\textsc{n.acc.sg} so be.\textsc{3sg.prs} how someone.\textsc{m.nom.sg} \textsc{irr} you.\textsc{acc.pl} deceive.\textsc{3sg.aor.opt}\\
\trans `Therefore, is there any way in which someone could deceive you in this ... ?' (Xenophon, \textit{Anabasis} 5.7.7)
\label{relan37}
\end{exe}

\begin{exe}
\ex οὐκ ἔϲθ᾽ ὁποῖον ϲτάντ᾽ ἂν ἀνθρώπου βίον οὔτ᾽ αἰνέϲαιμ᾽ ἂν οὔτε μεμψαίμην ποτέ\\
\gll ouk ésth' \emph{hopoîon} stánt' \emph{àn} anthrṓpou bíon oút' ainésaim' àn oúte mempsaímēn poté\\
not be.\textsc{3sg.prs} of.what.sort.\textsc{m.acc.sg} stand.\textsc{ptcp.aor.m.acc.sg} \textsc{irr} person.\textsc{gen.sg} life.\textsc{acc.sg} nor praise.\textsc{1sg.aor.opt} \textsc{irr} nor blame.\textsc{1sg.aor.opt.mid} sometime\\
\trans `There is no station of human life that I would ever praise or blame as being settled.' (Sophocles, \textit{Antigone} 1156)
\label{relan38}
\end{exe}

\begin{exe}
\ex οὐ γὰρ ἔϲθ᾽ ὅπωϲ μί᾽ ἡμέρα γένοιτ᾽ ἂν ἡμέραι δύο\\
\gll ou gàr ésth' \emph{hópōs} mí' hēméra génoit' \emph{àn} hēmérai dúo\\
not then be.\textsc{3sg.prs} how one.\textsc{f.nom.sg} day.\textsc{nom.sg} become.\textsc{3sg.aor.opt.mid} \textsc{irr} day.\textsc{nom.pl} two\\
\trans `For there is no way that one day could become two days.' (Aristophanes, \textit{Clouds} 1181)
\label{relan39}
\end{exe}

\begin{exe}
\ex κοὐκ ἔϲθ᾽ ὅπωϲ {[}...{]} ἂν {[}...{]} λάθοι\\
\gll kouk ésth' \emph{hópōs} \emph{àn} láthoi\\
and=not be.\textsc{3sg.prs} how \textsc{irr} hide.\textsc{3sg.aor.opt}\\
\trans `... and there is no way for him to escape notice.' (Aristophanes, \textit{Wasps} 212)
\label{relan40}
\end{exe}

\begin{exe}
\ex οὐ γὰρ ἔϲθ᾽ ὅπωϲ ἀπειπεῖν ἂν δοκῶ μοι τήμερον\\
\gll ou gàr ésth' \emph{hópōs} apeipeîn \emph{àn} dokô moi tḗmeron\\
not then be.\textsc{3sg.prs} how refuse.\textsc{aor.inf} \textsc{irr} think.\textsc{1sg.prs} me.\textsc{dat} today\\
\trans `For there is no way that I could think of refusing today ...' (Aristophanes, \textit{Peace} 306; cf. also Plato, \textit{Apology} 40c)
\label{relan41}
\end{exe}

\hyperlink{p389}{\emph{[p389]}}

\begin{exe}
\ex οὐ γὰρ ἔϲθ᾽ ὅπωϲ {[}...{]} εὖνοι γένοιντ᾽ ἄν\\
\gll ou gàr ésth' \emph{hópōs} eûnoi génoint' \emph{án}\\
not then be.\textsc{3sg.prs} how right-minded.\textsc{m.nom.pl} become.\textsc{3pl.aor.opt.mid} \textsc{irr}\\
\trans `... for there is no way in which they could become well-disposed ...' (Demosthenes 15.18)
\label{relan42}
\end{exe}

\begin{exe}
\ex ἔϲτιν οὖν, ὅπωϲ ταῦτ᾽ ἄν, ἐκεῖνα προειρηκώϲ, {[}...{]} ἐτόλμηϲεν εἰπεῖν\\
\gll éstin oûn, \emph{hópōs} taût' \emph{án}, ekeîna proeirēkṓs, etólmēsen eipeîn\\
be.\textsc{3sg.prs} so how this.\textsc{n.acc.pl} \textsc{irr} that.\textsc{n.acc.pl} say.before.\textsc{ptcp.prf.m.nom.sg} dare.\textsc{3sg.aor} say.\textsc{aor.inf}\\
\trans `So is there any way in which he could have dared to say these things, having previously said those ... ?' (Demosthenes 19.308)
\label{relan43}
\end{exe}

A similar reading is given to (\ref{relan44}) and (\ref{relan45}) on the one hand, but also (\ref{relan46}) on the other.

\begin{exe}
\ex ἀλλ᾽ οὐδὲ φίλων πέλαϲ οὐδείϲ, ὅϲτιϲ ἂν εἴποι\\
\gll all' oudè phílōn pélas oudeís, \emph{hóstis} \emph{àn} eípoi\\
but nor friend.\textsc{gen.pl} near nobody.\textsc{m.nom.sg} who.\textsc{m.nom.sg} \textsc{irr} say.\textsc{3sg.aor.opt}\\
\trans `And neither is there any of his kin nearby who might say ...' (Euripides, \textit{Alcestis} 80)\footnote{\emph{Translator's note}: The Perseus edition adds \textit{ést'} after \textit{pélas}.}
\label{relan44}
\end{exe}

\begin{exe}
\ex οὔτε τιϲ ξένοϲ ἀφῖκται {[}...{]}, ὅϲτιϲ ἂν ἡμῖν ϲαφέϲ τι ἀγγεῖλαι οἷόϲ τ᾽ ἦν περὶ τούτων\\
\gll oùte tis xénos aphîktai \emph{hóstis} \emph{àn} hēmîn saphés ti angeîlai hoîós t' ên perì toútōn\\
nor some.\textsc{m.nom.sg} stranger.\textsc{nom.sg} arrive.\textsc{3sg.prf.pass} who.\textsc{m.nom.sg} \textsc{irr} us.\textsc{dat} clear.\textsc{n.acc.sg} something.\textsc{acc.sg} announce.\textsc{3sg.aor.opt} such.as.\textsc{m.nom.sg} and be.\textsc{3sg.imp} about this.\textsc{gen.pl}\\
\trans `Nor has any stranger come who could tell us anything definite about this matter' (Plato, \textit{Phaedo} 57a)
\label{relan45}
\end{exe}

\begin{exe}
\ex οὐδ᾽ ἀγγελόϲ τιϲ οὐδὲ ϲυμπράκτωρ ὁδοῦ κατειδ᾽ ὅτου τιϲ ἐκμαθὼν ἐχρήϲατ᾽ ἄν\\
\gll oud' angelós tis oudè sumpráktōr hodoû kateid' \emph{hótou} tis ekmathṑn ekhrḗsat' \emph{án}\\
nor messenger some.\textsc{m.nom.sg} nor assistant.\textsc{nom.sg} way.\textsc{gen.sg} observe.\textsc{3sg.aor} whom.\textsc{m.gen.sg} someone.\textsc{m.nom.sg} learn.\textsc{ptcp.aor.m.nom.sg} use.\textsc{3sg.aor.mid} \textsc{irr}\\
\trans `(Was there) no messenger or travelling companion from whom one might have learned something of use?' (Sophocles, \textit{Oedipus Rex} 117)
\label{relan46}
\end{exe}

A second group of relative clauses to be considered here are those that are introduced by \emph{hóper} `which', in which the -\emph{per} conceptually serves to indicate sharp subordination to the main clause, and in which we would therefore expect to see \emph{án} immediately following the relativizer,\is{relative pronouns} based on what was observed with \emph{hóstis}. We find this position in full \emph{hósper}-sentences only in the majority of examples, however, and not always: (\ref{relan47})--(\ref{relan55}).

\begin{exe}
\ex κατήλπιζε εὐπετέωϲ τῆϲ θαλάϲϲηϲ κρατήϲειν, τάπερ ἂν καὶ ἦν\\
\gll katḗlpize eupetéōs tês thalássēs kratḗsein, \emph{táper} \emph{àn} kaì ên\\
hope.\textsc{3sg.imp} easily the.\textsc{f.gen.sg} sea.\textsc{gen.sg}
rule.\textsc{fut.inf} which.\textsc{n.nom.pl} \textsc{irr} also be.\textsc{3sg.imp}\\
\trans `He hoped that he would easily rule the seas, which might well have been.' (Herodotus 8.136.3)
\label{relan47}
\end{exe}

\begin{exe}
\ex τοιαῦτα θεραπεύϲαντεϲ ἑωυτούϲ, ὁποῖά περ ἂν ἐθεραπεύθηϲαν\\
\gll toiaûta therapeúsantes heōutoús, \emph{hopoîá} \emph{per} \emph{àn} etherapeúthēsan\\
such.\textsc{n.acc.pl} treat.\textsc{ptcp.aor.m.nom.pl} themselves.\textsc{acc} of.what.sort.\textsc{n.acc.pl} all \textsc{irr} treat.\textsc{3pl.aor.pass}\\
\trans `... treating themselves in just such a way as they would be treated' ({[}Hippocrates,{]} \textit{De arte} 46.12; \citealp[46, line 12]{Gomperz1890})
\label{relan48}
\end{exe}

\begin{exe}
\ex ἐνόμιζον {[}...{]} ὅϲον οὐκ ἐϲπλεῖν αὐτούϲ· ὅπερ ἂν, εἰ ἐβουλήθηϲαν μὴ κατοκνῆϲαι, ῥᾳδίωϲ ἂν ἐγένετο\\
\gll enómizon hóson ouk espleîn autoús; \emph{hóper} \emph{àn}, ei eboulḗthēsan mḕ katoknêsai, rhāidíōs àn egéneto\\
consider.\textsc{3pl.imp} how.much not sail.in.\textsc{prs.inf} them.\textsc{acc} which.\textsc{m.nom.sg} \textsc{irr} if wish.\textsc{3pl.aor.pass} not shrink.\textsc{aor.inf} easily \textsc{irr} become.\textsc{3sg.aor.mid}\\
\trans `They believed that they were not far from sailing in upon them, which might easily have come to pass if they had been unwilling to shrink from it.' (Thucydides 2.94.1)\footnote{\emph{Translator's note}: The Perseus edition has \textit{ṓionto} and lacks the
second \textit{àn}.}
\label{relan49}
\end{exe}

\begin{exe}
\ex ἐὰν ϲυμβούλουϲ ποιώμεθα τοιούτουϲ {[}...{]}, οἵουϲ περ ἂν περὶ τῶν ἰδίων ἡμῖν εἶναι βουληθεῖμεν\\
\gll eàn sumboúlous poiṓmetha toioútous \emph{hoíous} \emph{per} \emph{àn} perì tôn idíōn hēmîn eînai boulētheîmen\\
if advisor.\textsc{acc.pl} make.\textsc{1pl.prs.sbjv.pass} such.\textsc{m.acc.pl} such.as.\textsc{m.acc.pl} all \textsc{irr} about the.\textsc{n.gen.pl} private.\textsc{n.gen.pl} us.\textsc{dat} be.\textsc{prs.inf} wish.\textsc{1pl.aor.opt.pass}\\
\trans `... if we make such people advisors as we would wish to have for our private affairs ...' (Isocrates 8.133)
\label{relan50}
\end{exe}

\begin{exe}
\ex χρὴ τοιούτουϲ εἶναι κριτάϲ {[}...{]}, οἵων περ ἂν αὐτοὶ τυγχάνειν ἀξιώϲειαν\\
\gll khrḕ toioútous eînai kritás \emph{hoíōn} \emph{per} \emph{àn} autoì tunkhánein axiṓseian\\
need.\textsc{3sg.prs} such.\textsc{m.acc.pl} be.\textsc{prs.inf} judge.\textsc{acc.pl} such.as.\textsc{m.gen.pl} all \textsc{irr} same.\textsc{m.nom.pl} happen.\textsc{prs.inf} deem.\textsc{3pl.aor.opt}\\
\trans `It is necessary (for them) as judges to be such as they themselves would find worthy' (Isocrates 15.23)
\label{relan51}
\end{exe}

\begin{exe}
\ex ἀξιῶν τὴν αὐτὴν Παϲίωνι {[}...{]} γίγνεϲθαι ζημίαν, ἧϲπερ ἂν αὐτὸϲ ἐτύγχανεν\\
\gll axiôn tḕn autḕn Pasíōni gígnesthai zēmían, \emph{hêsper} \emph{àn} autòs etúnkhanen\\
deem.\textsc{ptcp.prs.m.nom.sg} the.\textsc{f.acc.sg} same.\textsc{f.acc.sg} Pasion.\textsc{dat} become.\textsc{prs.inf.pass} penalty.\textsc{acc.sg}
which.\textsc{f.gen.sg} \textsc{irr} same.\textsc{m.nom.sg} happen.\textsc{3sg.imp}\\
\trans `... expecting the same penalty for Pasion that he would have incurred himself' (Isocrates 17.21)
\label{relan52}
\end{exe}

\begin{exe}
\ex πράττειϲ ἅπερ ἂν δοῦλοϲ φαυλότατοϲ πράξειεν\\
\gll prátteis \emph{háper} \emph{àn} doûlos phaulótatos práxeien\\
do.\textsc{2sg.prs} which.\textsc{n.acc.pl} \textsc{irr} slave.\textsc{nom.sg} basest.\textsc{m.nom.sg} do.\textsc{3sg.aor.opt}\\
\trans `You are doing what the meanest slave would do' (Plato, \textit{Crito} 52c)\footnote{\emph{Translator's note}: The Perseus edition adds \textit{te} and also \textit{ho} before \textit{phaulótatos}.}
\label{relan53}
\end{exe}

\begin{exe}
\ex ᾤμην {[}...{]} διαλέξεϲθαι αὐτόν μοι, ἅπερ ἂν ἐραϲτὴϲ παιδικοῖϲ {[}...{]} διαλεχθείη\\
\gll ṓimēn dialéxesthai autón moi, \emph{háper} \emph{àn} erastḕs paidikoîs dialekhtheíē\\
think.\textsc{1sg.imp} discuss.\textsc{fut.inf.mid} him.\textsc{acc} me.\textsc{dat} which.\textsc{n.acc.pl} \textsc{irr} lover.\textsc{nom.sg} darling.\textsc{dat.pl} discuss.\textsc{3sg.aor.opt.pass}\\
\trans `I thought that he would say to me what a lover would say to his favourites.' (Plato, \textit{Symposium} 217b)
\label{relan54}
\end{exe}

\begin{exe}
\ex ἐποίουν ἅπερ ἂν ἄνθρωποι ἐν ἐρημίᾳ ποιήϲειαν\\
\gll epoíoun \emph{háper} \emph{àn} ánthrōpoi en erēmíāi poiḗseian\\
do.\textsc{3pl.imp} which.\textsc{n.acc.pl} \textsc{irr} person.\textsc{nom.pl} in solitude.\textsc{dat.sg} do.\textsc{3pl.aor.opt}\\
\trans `... they did what people would do in private.' (Xenophon, \textit{Anabasis} 5.4.34)
\label{relan55}
\end{exe}

But in examples (\ref{relan56})--(\ref{relan58}), \emph{án} is separated from \emph{hósper}:

\begin{exe}
\ex τὸν δὲ πόλεμον, δι᾽ ὅνπερ χρήϲιμοι ἂν εἶμεν, εἴ τιϲ ὑμῶν μὴ οἴεται ἔϲεϲθαι\\
\gll tòn dè pólemon, di' \emph{hónper} khrḗsimoi \emph{àn} eîmen, eí tis humôn mḕ oíetai ésesthai\\
the.\textsc{m.acc.sg} but war.\textsc{acc.sg} through which.\textsc{m.acc.sg} useful.\textsc{m.nom.pl} \textsc{irr} be.\textsc{1pl.prs} if someone.\textsc{m.nom.sg} you.\textsc{gen.pl} not think.\textsc{3sg.prs.pass} be.\textsc{fut.inf.mid}\\
\trans `But if any of you does not think there will be a war, through which we could be useful ...' (Thucydides 1.33.3)
\label{relan56}
\end{exe}

\begin{exe}
\ex Φίλιπποϲ δ᾽ ἅπερ εὔξαιϲθ᾽ ἂν ὑμεῖϲ, {[}...{]} πράξει\\
\gll Phílippos d' \emph{háper} eúxaisth' \emph{àn} humeîs práxei\\
Philip.\textsc{nom} then which.\textsc{n.acc.pl} pray.\textsc{2pl.aor.opt.mid} \textsc{irr} you.\textsc{nom.pl} do.\textsc{3sg.fut}\\
\trans `... and Philip will do just what you would have prayed for' (Demosthenes 6.30)
\label{relan57}
\end{exe}

\begin{exe}
\ex ὑμεῖϲ δ᾽, ἅπερ εὔξαιϲθ᾽ ἄν, ἐλπίϲαντεϲ {[}...{]}\\
\gll humeîs d', \emph{háper} eúxaisth' \emph{án}, elpísantes\\
you.\textsc{nom.pl} then which.\textsc{n.acc.pl} pray.\textsc{2pl.aor.opt.mid} \textsc{irr} hope.\textsc{ptcp.aor.m.nom.pl}\\
\trans `... and you, hoping for just what you would have prayed for ...' (Demosthenes 19.328)
\label{relan58}
\end{exe}

Awareness of the close connection between \emph{án} and \emph{hósper} becomes particularly clear in cases of verb ellipsis:\is{ellipsis} compare \isi{ellipsis} of the \isi{subjunctive} verb, e.g. (\ref{relan59})--(\ref{relan62}).

\begin{exe}
\ex φίλουϲ νομίζουϲ᾽ οὕϲπερ ἂν πόϲιϲ ϲέθεν\\
\gll phílous nomízous' \emph{hoúsper} \emph{àn} pósis séthen\\
dear.\textsc{m.acc.pl} consider.\textsc{ptcp.prs.f.nom.sg} whom.\textsc{m.acc.pl} \textsc{irr} husband.\textsc{nom.sg} you.\textsc{gen}\\
\trans `... holding them as dear as does your husband' (Euripides, \textit{Medea} 1153)
\label{relan59}
\end{exe}

\begin{exe}
\ex φιλεῖν οἴεϲθε δεῖν καὶ τιμᾶν, οὕϲπερ ἂν καὶ ὁ βαϲιλεύϲ\\
\gll phileîn oíesthe deîn kaì timân, \emph{hoúsper} \emph{àn} kaì ho basileús\\
like.\textsc{prs.inf} think.\textsc{2pl.prs.imper.pass} need.\textsc{prs.inf} and honour.\textsc{prs.inf} whom.\textsc{m.acc.pl} \textsc{irr} also the.\textsc{m.nom.sg} king.\textsc{nom.sg}\\
\trans `Believe that you should love and honour those whom your king loves and honours' (Isocrates 3.60)
\label{relan60}
\end{exe}

\begin{exe}
\ex τὸ τοὺϲ αὐτοὺϲ μιϲεῖν καὶ φιλείν, οὕϲπερ ἂν ἡ πατρίϲ\\
\gll tò toùs autoùs miseîn kaì phileín, \emph{hoúsper} \emph{àn} hē patrís\\ 
the.\textsc{n.nom.sg} the.\textsc{m.acc.pl} same.\textsc{m.acc.pl} hate.\textsc{prs.inf} and like.\textsc{prs.inf} whom.\textsc{m.acc.pl} \textsc{irr} the.\textsc{f.nom.sg} fatherland.\textsc{nom.sg}\\
\trans `... having the same friends and the same enemies as your country.' (Demosthenes 18.280)
\label{relan61}
\end{exe}

\hyperlink{p390}{\emph{[p390]}}

\begin{exe}
\ex τελεῖν δὲ αὐτὸν τὰ αὐτὰ τέλη ἐν τῷ δήμῳ ἅπερ ἂγ καὶ Πειραιεῖϲ\\
\gll teleîn dè autòn tà autà télē en tôi dḗmōi \emph{háper} \emph{àng} kaì Peiraieîs\\
fulfil.\textsc{prs.inf} but him.\textsc{m.acc.sg} the.\textsc{n.acc.pl} same.\textsc{n.acc.pl} end.\textsc{acc.pl} in the.\textsc{m.dat.sg} people.\textsc{dat.sg} which.\textsc{n.acc.pl} \textsc{irr} also Peiraean.\textsc{nom.pl}\\
\trans `... and for him to pay the same fees for the people that Peiraeans also would' (CIA 2.589.26; circa 300 BCE)
\label{relan62}
\end{exe}

\begin{exe}
\ex τοϲαύτην ποιηϲάμενοι ϲπουδὴν, ὅϲην περ ἂν τῆϲ αὑτῶν χώραϲ πορθουμένηϲ\\
\gll tosaútēn poiēsámenoi spoudḕn, \emph{hósēn} \emph{per} \emph{àn} tês hautôn khṓras porthouménēs\\
so.much.\textsc{f.acc.sg} make.\textsc{ptcp.aor.mid.m.nom.pl} speed.\textsc{acc.sg} as.much.\textsc{f.acc.sg} all \textsc{irr} the.\textsc{f.gen.sg} themselves.\textsc{gen} land.\textsc{gen.sg} ravage.\textsc{ptcp.prs.pass.f.gen.sg}\\
\trans `... having made as great haste as if it had been their own country that was being laid waste.' (Isocrates 4.86)
\label{relan63}
\end{exe}

The following serve as examples: (\ref{relan63})--(\ref{relan70}).

\begin{exe}
\ex νικῆϲαι {[}...{]} τοϲοῦτον, ὅϲον περ ἂν εἰ ταῖϲ γυναιξὶν αὐτῶν ϲυνέβαλον\\
\gll nikêsai tosoûton, \emph{hóson} \emph{per} \emph{àn} ei taîs gunaixìn autôn sunébalon\\
win.\textsc{aor.inf} so.much.\textsc{n.acc.sg} as.much.\textsc{n.acc.sg} all \textsc{irr} if the.\textsc{f.dat.pl} woman.\textsc{dat.pl} them.\textsc{gen} clash.\textsc{3pl.aor}\\
\trans `... to have won as complete a victory as if they had come to blows with their womenfolk' (Isocrates 5.90)
\label{relan64}
\end{exe}

\begin{exe}
\ex τοϲοῦτον ἐφρόνηϲαν, ὅϲον περ ἂν, εἰ πάντων ἡμῶν ἐκράτηϲαν\\
\gll tosoûton ephrónēsan, \emph{hóson} \emph{per} \emph{àn}, ei pántōn hēmôn ekrátēsan\\
so.much.\textsc{n.acc.sg} understand.\textsc{3pl.aor} as.much.\textsc{n.acc.sg} all \textsc{irr} if all.\textsc{m.gen.pl} us.\textsc{gen} rule.\textsc{3pl.aor}\\
\trans `... they were as filled with pride as if they had conquered us all' (Isocrates 10.49)
\label{relan65}
\end{exe}

\begin{exe}
\ex ἅπερ ἂν εἰϲ τοὺϲ πολεμιωτάτουϲ, ἐξαμαρτεῖν ἐτόλμηϲαν\\
\gll \emph{háper} \emph{àn} eis toùs polemiōtátous, examarteîn etólmēsan\\
which.\textsc{n.acc.pl} \textsc{irr} into the.\textsc{m.acc.pl} hostile.\textsc{supl.m.acc.pl} wrong.\textsc{aor.inf} dare.\textsc{3pl.aor}\\
\trans `They dared to do wrong as they would to their greatest enemies' (Isocrates 14.37)
\label{relan66}
\end{exe}

\begin{exe}
\ex εἰϲ τὸν αὐτὸν καθέϲτηκα κίνδυνον, εἰϲ ὅνπερ ἄν, εἰ πάνταϲ ἐτύγχανον ἠδικηκώϲ\\
\gll eis tòn autòn kathéstēka kíndunon, eis \emph{hónper} \emph{án}, ei pántas etúnkhanon ēdikēkṓs\\
into the.\textsc{m.acc.sg} same.\textsc{m.acc.sg} set.\textsc{1sg.prf} danger.\textsc{acc.sg} into which.\textsc{m.acc.sg} \textsc{irr} if all.\textsc{m.acc.pl} happen.\textsc{1sg.imp} wrong.\textsc{ptcp.prf.m.nom.sg}\\
\trans `I stand in the same peril in which I would stand if I happened to have wronged everyone' (Isocrates 15.28)
\label{relan67}
\end{exe}

\begin{exe}
\ex δοκεῖ μοι {[}...{]} τοιαύτην ποιήϲαϲθαι ζήτηϲιν αὐτοῦ, οἵαν περ ἄν, εἰ προϲέταξέ τιϲ\\
\gll dokeî moi toiaútēn poiḗsasthai zḗtēsin autoû, \emph{hoían} \emph{per} \emph{án}, ei prosétaxé tis\\
seem.\textsc{3sg.prs} me.\textsc{dat} such.\textsc{f.acc.sg} do.\textsc{aor.inf.mid} search.\textsc{acc.sg} it.\textsc{gen} such.as.\textsc{f.acc.sg} all \textsc{irr} if command.\textsc{3sg.aor} someone.\textsc{m.nom.sg}\\
\trans `It is apparent to me to use such an inquiry for this as we would if someone commanded ...' (Plato, \textit{Republic} 2.368d)\footnote{\emph{Translator's note}: The Perseus edition has \textit{dokô} for \textit{dokeî}.}
\label{relan68}
\end{exe}

\begin{exe}
\ex μόνοι τε ὄντεϲ ὅμοια ἔπραττον, ἅπερ ἂν μετ᾽ ἄλλων ὄντεϲ\\
\gll mónoi te óntes hómoia épratton, \emph{háper} \emph{àn} met' állōn óntes\\
alone.\textsc{m.nom.pl} and be.\textsc{ptcp.prs.m.nom.pl} similar.\textsc{n.acc.pl} do.\textsc{3pl.imp} which.\textsc{n.acc.pl} \textsc{irr} with other.\textsc{m.gen.pl} be.\textsc{ptcp.prs.m.nom.pl}\\
\trans `And being alone, they would do the same things that they would with others.' (Xenophon, \textit{Anabasis} 5.4.34)
\label{relan69}
\end{exe}

\newpage
\begin{exe}
\ex ἀπεκρινάμην αὐτῷ, ἅπερ ἂν νέοϲ ἄνθρωποϲ\\
\gll apekrinámēn autôi, \emph{háper} \emph{àn} néos ánthrōpos\\
answer.\textsc{1sg.aor.mid} him.\textsc{dat} which.\textsc{n.acc.pl} \textsc{irr} young.\textsc{m.nom.sg} person.\textsc{nom.sg}\\
\trans `I answered him as a young man would' (Demosthenes 53.12)
\label{relan70}
\end{exe}

Among the relative clauses introduced by \emph{hós} alone, those with an assimilated pronoun\is{pronouns} are most clearly marked as closely connected to the main clause. In accordance with this, most of the examples that I have to hand have \emph{án} after \emph{hós}: (\ref{relan71})--(\ref{relan74}). But the number of examples is too small to justify a general rule, and (\ref{relan75}) is a counterexample.

\begin{exe}
\ex ἐγὼ δεδηγμένοϲ {[}...{]} τὸ ἀλγεινότατον ὧν ἄν τιϲ δηχθείη\\
\gll egṑ dedēgménos tò algeinótaton \emph{hôn} \emph{án} tis dēkhtheíē\\
I.\textsc{nom} bite.\textsc{ptcp.prf.pass.m.nom.sg} the.\textsc{n.acc.sg} painful.\textsc{supl.n.acc.sg} which.\textsc{n.gen.pl} \textsc{irr} someone.\textsc{m.nom.sg} bite.\textsc{3sg.aor.opt.pass}\\
\trans `I have been bitten in the most painful way that one can be bitten' (Plato, \textit{Symposium} 218a)\footnote{\emph{Translator's note}: The Perseus edition adds \textit{oûn} after \textit{egṑ}.}
\label{relan71}
\end{exe}

\begin{exe}
\ex ἐμμενεῖν οἷϲ ἂν οὗτοι γνοῖεν\\
\gll emmeneîn \emph{hoîs} \emph{àn} hoûtoi gnoîen\\
abide.\textsc{prs.inf} which.\textsc{n.dat.pl} \textsc{irr} this.\textsc{m.nom.pl} know.\textsc{3pl.aor.opt}\\
\trans `... to abide by what these men would decide' (Isaeus 5.31)
\label{relan72}
\end{exe}

\begin{exe}
\ex ἐμμενεῖν οἷϲ ἂν αὐτοὶ γνοῖεν\\
\gll emmeneîn \emph{hoîs} \emph{àn} autoì gnoîen\\
abide.\textsc{prs.inf} which.\textsc{n.dat.pl} \textsc{irr} same.\textsc{m.nom.pl} know.\textsc{3pl.aor.opt}\\
\trans `... to abide by what they themselves would decide' (Isaeus 5.33)
\label{relan73}
\end{exe}

\begin{exe}
\ex πρὸϲ ἅπαϲιν {[}...{]} τοῖϲ ἄλλοιϲ, οἷϲ ἂν εἰπεῖν τιϲ ὑπὲρ Κτηϲιφῶντοϲ ἔχοι\\
\gll pròs hápasin toîs állois, \emph{hoîs} \emph{àn} eipeîn tis hupèr Ktēsiphôntos ékhoi\\
to quite.all the.\textsc{n.dat.pl} other.\textsc{n.dat.pl} which.\textsc{n.dat.pl} \textsc{irr} say.\textsc{aor.inf} someone.\textsc{m.nom.sg} over Ctesiphon.\textsc{gen} have.\textsc{3sg.prs.opt}\\
\trans `As well as all the other things with which one might speak for Ctesiphon ...' (Demosthenes 18.16)
\label{relan74}
\end{exe}

\begin{exe}
\ex μηδὲν ὧν ἰδίᾳ φυλάξαιϲθ᾽ ἄν\\
\gll mēdèn \emph{hôn} idíāi phuláxaisth' \emph{án}\\
nothing.\textsc{acc} which.\textsc{n.gen.pl} private.\textsc{f.dat.sg}
guard.\textsc{2pl.aor.opt.mid} \textsc{irr}\\
\trans `... none of the things against which you would guard in your private lives' (Demosthenes 20.136)
\label{relan75}
\end{exe}

In other types of relative clause, usage seems colourful and lawless. However, I think I can say that normal relative clauses have \emph{án} almost as often immediately after the pronoun\is{pronouns}\is{relative pronouns} as in a later position in the clause. A natural consequence of this variation is that it is not unusual\is{doubling|(} to find \emph{án} twice in relative clauses, e.g. (\ref{anan1})--(\ref{anan3}). Compare the double use of \emph{án} in main clauses, discussed below.

\begin{exe}
\ex ἀφ᾽ ὧν ἄν τιϲ ϲκοπῶν, εἴ ποτε καὶ αὖθιϲ ἐπιπέϲοι, μάλιϲτ᾽ ἂν ἔχοι τι προειδὼϲ μὴ ἀγνοεῖν\\
\gll aph' \emph{hôn} \emph{án} tis skopôn, eí pote kaì aûthis epipésoi, málist' \emph{àn} ékhoi ti proeidṑs mḕ agnoeîn\\
of which.\textsc{n.gen.pl} \textsc{irr} someone.\textsc{m.nom.sg} consider.\textsc{ptcp.prs.m.nom.sg} if sometime also again fall.on.\textsc{3sg.aor.opt} most \textsc{irr} have.\textsc{3sg.prs.opt} something.\textsc{acc.sg} foresee.\textsc{ptcp.prf.m.nom.sg} not overlook.\textsc{prs.inf}\\
\trans `... with which some observer, if it should ever come upon us again, may have something to predict and recognize.' (Thucydides 2.48.3)
\label{anan1}
\end{exe}

\begin{exe}
\ex ὅϲα γὰρ ἂν νῦν πορίϲαιτ᾽ ἄν\\
\gll \emph{hósa} gàr \emph{àn} nûn porísait' \emph{án}\\
as.much.\textsc{n.acc.pl} then \textsc{irr} now bring.\textsc{2pl.aor.opt.mid} \textsc{irr}\\
\trans `For the amount that you could provide ...' (Demosthenes 14.26)\footnote{\emph{Translator's note}: The Perseus edition lacks the first \textit{àn}.}
\label{anan2}
\end{exe}

\begin{exe}
\ex οὓϲ ἄν τιϲ δεόμενοϲ {[}...{]} εἴποι ἄν\\
\gll \emph{hoùs} \emph{án} tis deómenos eípoi \emph{án}\\
which.\textsc{m.acc.pl} \textsc{irr} someone.\textsc{m.nom.sg} lack.\textsc{ptcp.prs.pass.m.nom.sg} say.\textsc{3sg.aor.opt} \textsc{irr}\\
\trans `... which someone might say while beseeching ...' ({[}Demosthenes{]} 59.70)
\label{anan3}
\end{exe}\is{doubling|)}

Therefore, when we find an example like (\ref{relan76}), in which \emph{án} has clearly been omitted, it is completely impossible to establish, from our perspective, whether the right reading is \textit{di' hôn \emph{an} pausaímeth'} or rather \textit{di' hôn pausaímeth' \emph{an}} (as found in editions since \citealp[1539]{Bekker1823}).

\begin{exe}
\ex ἅ δεῖ καὶ δι᾽ ὧν παυϲαίμεθ᾽ αἰϲχύνην ὀφλιϲκάνοντεϲ\\
\gll há deî kaì di' \emph{hôn} pausaímeth' aiskhúnēn ophliskánontes\\
which.\textsc{n.acc.pl} need.\textsc{3sg.prs} and through which.\textsc{n.gen.pl} stop.\textsc{1pl.aor.opt.mid} shame.\textsc{acc.sg} incur.\textsc{ptcp.prs.m.nom.pl}\\
\trans `... what is necessary and by what means we may cease to incur disgrace' (Demosthenes, \textit{Exordia} 1.3)
\label{relan76}
\end{exe}

\hyperlink{p391}{\emph{[p391]}} On the other hand, where the relative pronoun\is{pronouns}\is{relative pronouns} simply serves in place of \emph{hoûtos} `this', Latin-style, to link two independent statements together -- in other words, when we are dealing with a main clause rather than a relative clause -- \emph{án} is never found after the pronoun;\is{pronouns}\is{relative pronouns} cf. (\ref{relan77})--(\ref{relan79}).

\begin{exe}
\ex ἐν οἷϲ ἐγὼ {[}...{]} δικαίωϲ ἂν ὑπὸ πάντων ἐλεηθείην\\
\gll en \emph{hoîs} egṑ dikaíōs \emph{àn} hupò pántōn eleētheíēn\\
in which.\textsc{n.dat.pl} I.\textsc{nom} righteously \textsc{irr} under all.\textsc{m.gen.pl} pity.\textsc{1sg.aor.opt.pass}\\
\trans `... in which (circumstances) I might rightly be pitied by all.' (Andocides 1.67)
\label{relan77}
\end{exe}

\begin{exe}
\ex ὅ τίϲ ἰδὼν οὐκ ἂν ἐφοβήθη {[}...{]};\\
\gll \emph{hó} tís idṑn ouk \emph{àn} ephobḗthē\\
which.\textsc{n.acc.sg} who.\textsc{m.nom.sg} see.\textsc{ptcp.aor.m.nom.sg} not \textsc{irr} frighten.\textsc{3sg.aor.pass}\\
\trans `Seeing which, who would not have been frightened?' (Lysias 2.34)\footnote{\emph{Translator's note}: The Perseus edition has \textit{ouk àn idṑn}.}
\label{relan78}
\end{exe}

\begin{exe}
\ex ἐξ ὧν ϲαφέϲτατ᾽ ἄν τιϲ ἴδοι\\
\gll ex \emph{hôn} saphéstat' \emph{án} tis ídoi\\
out which.\textsc{n.gen.sg} clearly.\textsc{supl} \textsc{irr} someone.\textsc{m.nom.sg} see.\textsc{3sg.aor.opt}\\
\trans `From which one can most clearly see ...' (Demosthenes 18.49)\footnote{\emph{Translator's note}: The Perseus edition adds \textit{kaì} after \textit{hôn}.}
\label{relan79}
\end{exe}\is{relative clauses|)}

Correspondingly, in all other subordinate clauses, for instance those containing \emph{án} with an \isi{optative} or \isi{preterite}, \emph{án} is usually found in a later position in the clause. This is of course because in all such cases the subordinate clause has the mood in question not by virtue of being a subordinate clause but because it stands in for a main clause. For \emph{hōs} `that/as', for instance, we have the example (\ref{hosan1}) (but also (\ref{hosan2})); for \emph{hṓste} `so that', for instance, (\ref{hoste}); for \emph{hóti} `that/because', for instance, (\ref{hoti1})--(\ref{hoti3}).

\begin{exe}
\ex ὡϲ ἐγὼ οὐδ᾽ ἂν ἕνα ἄλλον ἐπαινέϲαιμι\\
\gll hōs egṑ oud' \emph{àn} héna állon epainésaimi\\
as I.\textsc{nom} nor \textsc{irr} one.\textsc{m.acc.sg} other.\textsc{m.acc.sg} praise.\textsc{1sg.aor.opt}\\
\trans `... as I could not praise another' (Plato, \textit{Symposium} 214d)
\label{hosan1}
\end{exe}

\begin{exe}
\ex καὶ οὐκ ἂν ἐλπίϲαντεϲ ὡϲ ἂν ἐπεξέλθοι τιϲ\\
\gll kaì ouk àn elpísantes hōs \emph{àn} epexélthoi tis\\
and not \textsc{irr} hope.\textsc{ptcp.aor.m.nom.pl} as \textsc{irr} proceed.\textsc{3sg.aor.opt} someone.\textsc{m.nom.sg}\\
\trans `... and not expecting that anyone might sally forth ...' (Thucydides 5.9.3)\footnote{\emph{Translator's note}: The Perseus edition has \textit{elpísantas} for \textit{elpísantes}.}
\label{hosan2}
\end{exe}

\begin{exe}
\ex ὥϲτε καὶ οὗτοϲ Ἔρωτοϲ ἂν εἴη μαθητήϲ\\
\gll hṓste kaì hoûtos Érōtos \emph{àn} eíē mathētḗs\\
so also this.\textsc{m.nom.sg} Eros.\textsc{gen} \textsc{irr} be.\textsc{3sg.prs.opt} pupil.\textsc{nom.sg}\\
\trans `... so that he too would be a pupil of Eros' (Plato, \textit{Symposium} 197b)
\label{hoste}
\end{exe}

\begin{exe}
\ex δῆλον ὅτι τοιαῦτ᾽ ἄττ᾽ ἂν λέγοι\\
\gll dêlon hóti toiaût' átt' \emph{àn} légoi\\
evident.\textsc{n.nom.sg} that such.\textsc{n.nom.pl} which.\textsc{n.acc.pl} \textsc{irr} say.\textsc{3sg.prs.opt}\\
\trans `(It is) evident that such things (are) what one might say ...' (Plato, \textit{Phaedo} 93c)
\label{hoti1}
\end{exe}

\begin{exe}
\ex ὅτι οὕτωϲ ἂν ἡμῶν τὸ γένοϲ εὔδαιμον γένοιτο\\
\gll hóti hoútōs \emph{àn} hēmôn tò génos eúdaimon génoito\\
that so \textsc{irr} us.\textsc{gen} the.\textsc{n.nom.sg} kind.\textsc{nom.sg} fortunate.\textsc{n.nom.sg} become.\textsc{3sg.aor.opt.mid}\\
\trans `... that our kind would become happy in this way ...' (Plato, \textit{Symposium} 193c)
\label{hoti2}
\end{exe}

\begin{exe}
\ex ὅτι τῶν ἀδικημάτων ἂν ἐμέμνητο τῶν αὑτοῦ\\
\gll hóti tôn adikēmátōn \emph{àn} emémnēto tôn hautoû\\
that the.\textsc{n.gen.pl} wrong.\textsc{gen.pl} \textsc{irr} recall.\textsc{3sg.plpf.pass} the.\textsc{n.gen.pl} himself.\textsc{gen}\\
\trans `Because he would recall his own crimes ...' (Demosthenes 18.79)
\label{hoti3}
\end{exe}

The same holds for \emph{epeì} `as/since', for instance (\ref{epei1})--(\ref{epei2}).

\begin{exe}
\ex ἐπεὶ ἔχοι γ᾽ ἄν τιϲ εἰπεῖν περὶ αὐτῶν\\
\gll epeì ékhoi g' \emph{án} tis eipeîn perì autôn\\
when have.\textsc{3sg.prs.opt} even \textsc{irr} someone.\textsc{m.nom.sg} say.\textsc{aor.inf} about them.\textsc{n.gen.pl}\\
\trans `... while one might be able to speak about them' (Plato, \textit{Cratylus} 410a)
\label{epei1}
\end{exe}

\begin{exe}
\ex ἐπεὶ διὰ γ᾽ ὑμᾶϲ πάλαι ἂν ἀπωλώλειτε\\
\gll epeì dià g' humâs pálai \emph{àn} apōlṓleite\\
when through even you.\textsc{acc.pl} long.ago \textsc{irr} destroy.\textsc{2pl.plpf}\\
\trans `... when on your own you would have perished long ago.' (Demosthenes 18.49)\footnote{\emph{Translator's note}: The Perseus edition adds \textit{autoùs} after \textit{humâs}.}
\label{epei2}
\end{exe}

The transmission gives us cause for doubt when it comes to temporal particles: \emph{hótan} `when(ever)' with the \isi{optative} is transmitted in Aeschylus, \textit{Persians} 450, and \emph{héōs án} `until \textsc{irr}' with the \isi{optative} in Isocrates 17.15 and Plato, \textit{Phaedo} 101d.\footnote{\emph{Translator's note}: The Perseus edition of Isocrates 17.15 lacks \textit{án}.} (Since \citealp[453]{Elmsley1812}, Sophocles, \textit{Women of Trachis} 687 is no longer taken to contain this.) We can be confident in (\ref{temporalan1})--(\ref{temporalan3}). In example (\ref{temporalan4}), \emph{án} is deleted.

\begin{exe}
\ex ἡνίκ᾽ ἂν ἡμεῖϲ μὴ δυναίμεθ᾽ ἐκεῖϲ᾽ ἀφικέϲθαι\\
\gll hēník' \emph{àn} hēmeîs mḕ dunaímeth' ekeîs' aphikésthai\\
when \textsc{irr} we.\textsc{nom} not can.\textsc{1pl.prs.opt.pass} thither arrive.\textsc{aor.inf.mid}\\
\trans `... when we could not arrive there.' (Demosthenes 4.31)
\label{temporalan1}
\end{exe}

\begin{exe}
\ex πρὶν ἂν {[}...{]} μετέχοιεν\\
\gll prìn \emph{àn} metékhoien\\
before \textsc{irr} share.\textsc{3pl.prs.opt}\\
\trans `... until they could share ...' (Xenophon, \textit{Hellenica} 2.3.48)
\label{temporalan2}
\end{exe}

\begin{exe}
\ex πρὶν ἂν {[}...{]} καταϲτήϲειαν\\
\gll prìn \emph{àn} katastḗseian\\
before \textsc{irr} share.\textsc{3pl.prs.opt}\\
\trans `... until they could set ...' (Xenophon, \textit{Hellenica} 2.3.48)
\label{temporalan3}
\end{exe}

\begin{exe}
\ex πρὶν ἂν {[}...{]} ἢ πέϲοι τιϲ ἢ τρωθείη\\
\gll prìn \emph{àn} ḕ pésoi tis ḕ trōtheíē\\
before \textsc{irr} or fall.\textsc{3sg.aor.opt} someone.\textsc{m.nom.sg} or wound.\textsc{3sg.aor.opt.pass}\\
\trans `... until someone either fell or was wounded' (Xenophon, \textit{Hellenica} 2.4.18)
\label{temporalan4}
\end{exe}

Without exception, \emph{án} is separated from the conjunction in \isi{optative} \emph{ei}-clauses: \emph{ei} `if' introducing embedded\is{subordination} questions, e.g. (\ref{eian1}), and \emph{ei} `if' introducing adverbial\is{adverbs} clauses, e.g. (\ref{eian2})--(\ref{eian5}).\footnote{\emph{Translator's note}: The distinction here is between \ili{German} \textit{ob} `if/whether' and \textit{wenn} `if'.}

\begin{exe}
\ex οὐκ οἶδ᾽ εἰ οἷόϲ τ᾽ ἂν εἴηϲ\\
\gll ouk oîd' \emph{ei} hoîós t' \emph{àn} eíēs\\
not know.\textsc{1sg.prf} if such.as\textsc{.m.nom.sg} and \textsc{irr} be.\textsc{2gs.prs.opt}\\
\trans `I do not know if you would be of that sort.' (Plato, \textit{Symposium} 210a)
\label{eian1}
\end{exe}

\begin{exe}
\ex εἰ πῶϲ ἂν ἀναπείϲαιμεν ἱκετεύοντέ νιν\\
\gll \emph{ei} pôs \emph{àn} anapeísaimen hiketeúonté nin\\
if somehow \textsc{irr} persuade.\textsc{1pl.aor.opt} supplicate.\textsc{ptcp.prs.m.nom.du} \textsc{3.acc}\\
\trans `If somehow we could persuade by supplicating her ...' (Euripides, \textit{Helen} 825)\footnote{\emph{Translator's note}: The Perseus edition has \textit{ísōs} for \textit{pôs}.}
\label{eian2}
\end{exe}

\begin{exe}
\ex οὐδ᾽ εἰ μὴ ποιήϲαιτ᾽ ἂν ἤδη\\
\gll oud' \emph{ei} mḕ poiḗsait' \emph{àn} ḗdē\\
nor if not do.\textsc{2pl.aor.opt} \textsc{irr} already\\
\trans `Nor, if you should fail to do immediately ...' (Demosthenes 4.18)\footnote{\emph{Translator's note}: The Perseus edition lacks \textit{ḗdē}.}
\label{eian3}
\end{exe}

\begin{exe}
\ex οὐκοῦν αἰϲχρόν, εἰ μέλλοντεϲ μὲν εὖ πάϲχειν ϲυκοφάντην ἂν τὸν ταῦτα λέγονθ᾽ ἡγοῖϲθε, ἐπὶ τῷ δ᾽ ἀφελέϲθαι {[}...{]} ἀκούϲεϲθε\\
\gll oukoûn aiskhrón, \emph{ei} méllontes mèn eû páskhein sukophántēn \emph{àn} tòn taûta légonth' hēgoîsthe, epì tôi d' aphelésthai akoúsesthe\\
not.then shameful.\textsc{n.nom.sg} if be.going.to.\textsc{ptcp.prs.m.nom.pl} then well suffer.\textsc{prs.inf} informer.\textsc{acc.sg} \textsc{irr} the.\textsc{m.acc.sg} this.\textsc{n.acc.pl} say.\textsc{ptcp.prs.m.acc.sg} lead.\textsc{2pl.prs.opt.pass} upon the.\textsc{n.dat.sg} then remove.\textsc{aor.inf.mid} hear.\textsc{2pl.fut.mid}\\
\trans `(Is it) not then shameful if you consider the one saying these things a pettifogger when about to benefit, but you will listen about removing ...' (Demosthenes 20.62)
\label{eian4}
\end{exe}

\begin{exe}
\ex ἐξώληϲ ἀπολοίμην {[}...{]}, εἰ προϲλαβών γ᾽ ἂν ἀργύριον {[}...{]} ἐπρέϲβευϲα\\
\gll exṓlēs apoloímēn \emph{ei} proslabṓn g' \emph{àn} argúrion eprésbeusa\\
ruined.\textsc{m.nom.sg} destroy.\textsc{1sg.aor.opt.mid} if accept.\textsc{ptcp.aor.m.nom.sg} even \textsc{irr} money.\textsc{acc.sg} be.ambassador.\textsc{1sg.aor}\\
\trans `May I perish miserably if I would have become an ambassador even by accepting money' (Demosthenes 19.172)
\label{eian5}
\end{exe}

In these cases the hypothetical\is{hypotheticals} character of the clause provided by \emph{án} is not determined by \emph{ei}; see the commentators on the individual examples.

\hyperlink{p392}{\emph{[p392]}} The cases in which \emph{mḗ} `not' with the \isi{optative} and \emph{án} follow expressions of fear and expectation are particularly significant: (\ref{fearan1})--(\ref{fearan4}). Here it cannot be doubted that the use of the \isi{optative} with \emph{án} is due to the main clause influencing the subordinate clause with \emph{mḗ}, and here only one of four examples contains \emph{án} immediately following \emph{mḗ}.

\begin{exe}
\ex δέδοικα γάρ, μὴ πρῲ λέγοιϲ ἂν τὸν πόθον\\
\gll dédoika gár, \emph{mḕ} prṑi légois \emph{àn} tòn póthon\\
fear.\textsc{1sg.prf} then not early say.\textsc{2sg.prs.opt} \textsc{irr} the.\textsc{m.acc.sg} longing.\textsc{acc.sg}\\
\trans `I am afraid lest you tell my longing too soon' (Sophocles, \textit{Women of Trachis} 630)
\label{fearan1}
\end{exe}

\begin{exe}
\ex οὔτε προϲδοκία οὐδεμία ἦν, μὴ ἄν ποτε οἱ πολέμιοι ἐξαπιναίωϲ οὕτωϲ ἐπιπλεύϲειαν\\
\gll oúte prosdokía oudemía ên, \emph{mḕ} \emph{án} pote hoi polémioi exapinaíōs hoútōs epipleúseian\\
nor expectation.\textsc{nom.sg} none.\textsc{f.nom.sg} be.\textsc{3sg.imp} not \textsc{irr} sometime the.\textsc{m.nom.pl} hostile.\textsc{m.nom.pl} unexpectedly so sail.upon.\textsc{3pl.aor.opt}\\
\trans `Nor was there any expectation lest the enemy should ever launch an attack so unexpectedly.' (Thucydides 2.93.3)
\label{fearan2}
\end{exe}

\begin{exe}
\ex ἐκεῖνο ἐννοῶ, μὴ λίαν ἂν ταχὺ ϲωφρονιϲθείην\\
\gll ekeîno ennoô, \emph{mḕ} lían \emph{àn} takhù sōphronistheíēn\\
that.\textsc{n.acc.sg} consider.\textsc{1sg.prs} not very \textsc{irr} quickly chasten.\textsc{1sg.aor.opt.pass}\\
\trans `As for that I misdoubt that I should be recalled to my senses very quickly.' (Xenophon, \textit{Anabasis} 6.1.28)
\label{fearan3}
\end{exe}

\begin{exe}
\ex φοβοῦνται, μὴ ματαία ἂν γένοιτο αὕτη ἡ παραϲκευή\\
\gll phoboûntai, \emph{mḕ} mataía \emph{àn} génoito haútē hē paraskeuḗ\\
frighten.\textsc{3pl.prs.pass} not vain.\textsc{f.nom.sg} \textsc{irr} become.\textsc{3sg.aor.opt.mid} this.\textsc{f.nom.sg} the.\textsc{f.nom.sg} preparation.\textsc{nom.sg}\\
\trans `(Some) were afraid lest this plan should prove vain.' (Xenophon, \textit{Ways} 4.41)\footnote{\emph{Translator's note}: The Perseus edition has \textit{kataskeuḗ} for \textit{paraskeuḗ}.}
\label{fearan4}
\end{exe}

This makes it clear why the position of \emph{án} is so fixed in \isi{subjunctive} clauses and so flexible in other subordinate clauses. In Classical Greek,\il{Greek, Classical} \emph{án} with \isi{subjunctive} mood is found only in subordinate clauses; what would be the point in moving \emph{án} from its traditional position? Conversely, \emph{án} with the indicative\is{indicative (mood)} and with the \isi{optative} is not only more frequent in main than in subordinate clauses but also basically carried over to these subordinate clauses from the main clause. It was necessary for the positional tendencies of \emph{án} in main clauses to be carried over to the subordinate clauses in question.\is{subordination|)}


\section{Postpositive particles: \emph{án} in main clauses}
\exewidth{(1235)}

But what is going on with the free positioning of \emph{án} in main clauses? It is indisputable that \emph{án} can be found a long way from the initial position in such clauses. The only word that it must precede is the final finite or non-finite verb\is{verb position|(} modified by \emph{án} in the clause, and here I particularly emphasize that \isi{participles} equivalent to hypothetical\is{hypotheticals} subordinate\is{subordination} clauses happily precede \emph{án} (cf. e.g. (\ref{anmain1})).

\begin{exe}
\ex γόνιμον δὲ ποιητὴν ἂν οὐχ εὕροιϲ ἔτι ζητῶν ἄν\\
\gll gónimon dè poiētḕn àn oukh heúrois éti zētôn \emph{án}\\
fruitful.\textsc{m.acc.sg} but poet.\textsc{acc.sg} \textsc{irr} not find.\textsc{2sg.aor.opt} still seek.\textsc{ptcp.prs.m.nom.sg} \textsc{irr}\\
\trans `If you looked, you wouldn't find a fruitful poet any more.' (Aristophanes, \textit{Frogs} 96)
\label{anmain1}
\end{exe}

\emph{án} may only follow this verb if it occurs immediately attached to it. However, there are examples in which \emph{g'}, a single-syllable enclitic\is{enclitics} or other monosyllable intervenes between the verb and \emph{án}. For \emph{g'} `even': (\ref{verbgan}).

\begin{exe}
\ex ἐπεὶ ἔχοι γ᾽ ἄν τιϲ εἰπεῖν περὶ αὐτῶν\\
\gll epeì ékhoi g' \emph{án} tis eipeîn perì autôn\\
when have.\textsc{3sg.prs.opt} even \textsc{irr} someone.\textsc{m.nom.sg} say.\textsc{aor.inf} about them.\textsc{n.gen.pl}\\
\trans `... while one might be able to speak about them' (Plato, \textit{Cratylus} 410A)
\label{verbgan}
\end{exe}

For \emph{tis} `someone': {[}Euripides, \textit{Oresteia} 694 and{]} (\ref{verbtisan}).

\begin{exe}
\ex οὐ μὲν οὖν εἴποι τιϲ ἂν ἡλίκαϲ\\
\gll ou mèn oûn eípoi tis \emph{àn} hēlíkas\\
not then so say.\textsc{3sg.aor.opt} someone.\textsc{m.nom.sg} \textsc{irr} so.great.\textsc{f.acc.pl}\\
\trans `... indeed, one could not say enough ...' (Demosthenes 18.282)
\label{verbtisan}
\end{exe}

For \emph{pot'} `ever': (\ref{verbpotan}).

\begin{exe}
\ex κεῖνοϲ δὲ πῶϲ τὰ ζῶντα τοῖϲ θανοῦϲιν ἀποδοίη ποτ᾽ ἄν\\
\gll keînos dè pôs tà zônta toîs thanoûsin apodoíē pot' \emph{án}\\
that.\textsc{m.nom.sg} but how the.\textsc{n.acc.pl} live.\textsc{ptcp.prs.n.acc.pl} the.\textsc{m.dat.pl} die.\textsc{ptcp.aor.m.dat.pl} restore.\textsc{3sg.aor.opt} sometime \textsc{irr}\\
\trans `And how could he ever restore the living to the dead?' (Euripides, \textit{Helen} 912f.)
\label{verbpotan}
\end{exe}

\hyperlink{p393}{\emph{[p393]}} For \emph{ou} (\textsc{neg}): (\ref{verbouan}).

\begin{exe}
\ex ἦ γὰρ εἶην οὐκ ἂν εὖ φρονῶν\\
\gll ê gàr eîēn ouk \emph{àn} eû phronôn\\
in.truth then be.\textsc{1sg.prs.opt} not \textsc{irr} well reason.\textsc{ptcp.prs.m.nom.sg}\\
\trans `For truly I would not be in my right mind ...' (Sophocles, \textit{Ajax} 1330)
\label{verbouan}
\end{exe}

For \emph{takh'} `quickly': (\ref{verbtakhan}).

\begin{exe}
\ex τῇ δ᾽ ἐπιϲτήμῃ ϲύ μου τρούχοιϲ τάχ᾽ ἄν που\\
\gll têi d' epistḗmēi sú mou troúkhois tákh' \emph{án} pou\\
the.\textsc{f.dat.sg} then knowledge.\textsc{f.dat.sg} you.\textsc{nom} me.\textsc{gen} have.before.\textsc{2sg.prs.opt} quickly \textsc{irr} somewhere\\
\trans `But perhaps you have an advantage in knowledge over me ...' (Sophocles, \textit{Oedipus Rex} 1115f.)
\label{verbtakhan}
\end{exe}

For \emph{tad'} `this': (\ref{verbtadan}).

\begin{exe}
\ex τίϲ ϲωφρονῶν τλαίη τάδ᾽ ἄν\\
\gll tís sōphronôn tlaíē tád' \emph{án}\\
who.\textsc{m.nom.sg} be.sane.\textsc{ptcp.prs.m.nom.sg} endure.\textsc{3sg.aor.opt} this.\textsc{n.acc.pl} \textsc{irr}\\
\trans `Who in his senses would dare this?' (Euripides, \textit{Helen} 97)
\label{verbtadan}
\end{exe}

For \emph{taut'} `this': (\ref{verbtautan}).

\begin{exe}
\ex ϲυμμαρτυροίη ταῦτ᾽ ἂν ἐν δίκῃ\\
\gll summarturoíē taût' \emph{àn} en díkēi\\
corroborate.\textsc{3sg.prs.opt} this.\textsc{n.acc.pl} \textsc{irr} in judgement.\textsc{dat.sg}\\
\trans `(She) too would bear witness to these things in judgement ...' (Solon, Fragment 36.1)
\label{verbtautan}
\end{exe}

For \emph{ment'} `yet': (\ref{verbmentan1}), (\ref{verbmentan2}), and Plato, \textit{Apology} 30D.

\begin{exe}
\ex ᾤμωξε μέντ᾽ ἄν\\
\gll ṓimōxe mént' \emph{án}\\
lament.\textsc{3sg.aor} yet \textsc{irr}\\
\trans `He would certainly regret it.' (Aristophanes, \textit{Frogs} 743)
\label{verbmentan1}
\end{exe}

\begin{exe}
\ex βουλοίμην μέντ᾽ ἄν\\
\gll bouloímēn mént' \emph{án}\\
wish.\textsc{1sg.prs} yet \textsc{irr}\\
\trans `I would certainly wish so.' (Plato, \textit{Phaedo} 76B)
\label{verbmentan2}
\end{exe}

However, these last three examples ((\ref{verbtautan}), (\ref{verbmentan1}), (\ref{verbmentan2})) also permit a different explanation. When the verb is clause-initial, the rule discussed above seems not to hold, e.g. (\ref{verbfirst1})--(\ref{verbfirst3}).

\begin{exe}
\ex προϲέβα γὰρ οὐκ ἂν ἀϲτιβὲϲ ἄλϲοϲ ἔϲ\\
\gll proséba gàr ouk \emph{àn} astibès álsos és\\
approach.\textsc{3sg.aor} then not \textsc{irr} untrodden.\textsc{n.acc.sg} grove.\textsc{acc.sg} into\\
\trans `For he would not have entered the untrodden grove ...' (Sophocles, \textit{Oedipus at Colonus} 125)
\label{verbfirst1}
\end{exe}

\begin{exe}
\ex ὄλοιντ᾽ ἰδοῦϲαι τοῦϲδ᾽ ἄν\\
\gll óloint' idoûsai toûsd' \emph{án}\\
destroy.\textsc{3pl.aor.opt.mid} see.\textsc{ptcp.aor.f.nom.pl} this.\textsc{m.acc.pl} \textsc{irr}\\
\trans `They would be undone, seeing them.' (Euripides, \textit{Suppliants} 944)
\label{verbfirst2}
\end{exe}

\begin{exe}
\ex μάθοιτε δὲ τοῦτο μάλιϲτ᾽ ἄν\\
\gll máthoite dè toûto málist' \emph{án}\\
learn.\textsc{2pl.aor.opt} but this.\textsc{n.acc.sg} most \textsc{irr}\\
\trans `But you would understand this best ...' (Demosthenes 21)
\label{verbfirst3}
\end{exe}

Moreover, it is obvious that, if a clause contains multiple instances of \emph{án}, the rule will affect the last \emph{án}, as in (\ref{multiplean1}) and (\ref{multiplean2}). In (\ref{multiplean3}), the distance between the second \emph{án} and the verb can be explained by the initial position of the verb.

\begin{exe}
\ex ἔδραϲ᾽ ἂν (εὖ τόδ᾽ ἴϲθ᾽) ἄν\\
\gll édras' \emph{àn} (eû tód' ísth') \emph{án}\\
do.\textsc{1sg.aor} \textsc{irr} well this.\textsc{n.acc.sg} know.\textsc{2sg.prf.imper} \textsc{irr}\\
\trans `I could have done -- know this well -- ...' (Sophocles, \textit{Oedipus Rex} 1438)\footnote{\emph{Translator's note}: The Perseus edition has \textit{toût'} for \textit{tód'}.}
\label{multiplean1}
\end{exe}

\begin{exe}
\ex δύναιτ᾽ ἂν ούδ᾽ ἂν ἰϲχύων φυγεῖν\\
\gll dúnait' \emph{àn} oúd' \emph{àn} iskhúōn phugeîn\\
can.\textsc{3sg.prs.opt.pass} \textsc{irr} nor \textsc{irr} be.strong.\textsc{ptcp.prs.m.nom.sg} flee.\textsc{aor.inf}\\
\trans `... not even a strong man would be able to escape' (Sophocles, \textit{Electra} 697)
\label{multiplean2}
\end{exe}

\begin{exe}
\ex ἠλείψατο δ᾽ ἂν τοὐμφαλοῦ οὐδεὶϲ παῖϲ ὑπένερθεν τότ᾽ ἄν\\
\gll ēleípsato d' \emph{àn} toumphaloû oudeìs paîs hupénerthen tót' \emph{án}\\
anoint.\textsc{3sg.aor.mid} then \textsc{irr} the=navel.\textsc{gen.sg} none.\textsc{m.nom.sg} child.\textsc{nom.sg} beneath then \textsc{irr}\\
\trans `And no boy then would anoint himself below the navel.' (Aristophanes, \textit{Clouds} 977)
\label{multiplean3}
\end{exe}

The editors of Aristophanes's \emph{The Knights} were therefore right to change the transmitted \textit{phágois hḗdist'} in verse 707 to \textit{phagṑn hḗdoit'} (or \textit{hḗdoi}), as in (\ref{knights}).

\begin{exe}
\ex ἐπὶ τῷ φαγὼν ἥδοιτ᾽/ἥδοι᾽ ἄν\\
\gll epì tôi phagṑn hḗdoit'/hḗdoi' \emph{án}\\
upon what.\textsc{n.dat.sg} eat.\textsc{ptcp.aor.m.nom.sg} enjoy.\textsc{3sg(2sg).prs.opt.pass} \textsc{irr}\\
\trans `What would he (you) most enjoy dining on?' (Aristophanes, \textit{Knights} 707)
\label{knights}
\end{exe}

On the other hand, (\ref{latean1}) is only an apparent counterexample, since for each of the consecutive nominatives\is{nominative} an understood \emph{élegen} `speak' should be read. Cf. also Sophocles, \textit{Philoctetes} 292 \textit{pròs toût' \emph{án}} `to this.\textsc{n.acc.sg} \textsc{irr}' (and (\ref{latean2})).

\begin{exe}
\ex οὐδὲν παρῆκ᾽ ἂν ἀργόν, ἀλλ᾽ ἔλεγεν ἡ γυνή τέ μοι χὠ δοῦλοϲ οὐδὲν ἧττον χὠ δεϲπότηϲ χἠ παρθένοϲ χἠ γραῦϲ ἄν\\
\gll oudèn parêk' \emph{àn} argón, all' élegen hē gunḗ té moi khō doûlos oudèn hêtton khō despótēs khē parthénos khē graûs \emph{án}\\
nothing.\textsc{acc.sg} permit.\textsc{1sg.aor} \textsc{irr} idle.\textsc{n.acc.sg} but say.\textsc{3sg.imp} the.\textsc{f.nom.sg} woman.\textsc{nom.sg} and me.\textsc{dat} and=the.\textsc{m.nom.sg} slave nothing.\textsc{n.acc.sg} less.\textsc{n.acc.sg} and=the.\textsc{m.nom.sg} master.\textsc{nom.sg} and=the.\textsc{f.nom.sg} maiden.\textsc{nom.sg} and=the.\textsc{f.nom.sg} old.woman.\textsc{nom.sg} \textsc{irr}\\
\trans `I would permit nothing idle; instead, my woman would speak, and the slave no less, and the master and the maiden and the old woman.' (Aristophanes, \textit{Frogs} 949f.)
\label{latean1}
\end{exe}

\begin{exe}
\ex κοὐ φθάνοι θνήϲκων τιϲ ἄν\\
\gll kou phthánoi thnḗskōn tis \emph{án}\\
and=not arrive.\textsc{3sg.prs.opt} die.\textsc{ptcp.prs.m.nom.sg} someone.\textsc{m.nom.sg} \textsc{irr}\\
\trans `... and it would not be too soon for anyone to die' (Euripides, \textit{Orestes} 941)
\label{latean2}
\end{exe}

From this rule, though, one can recognize what sort of tendencies have led to \emph{án} being attracted away from the position it had occupied in Homeric\il{Greek, Homeric} times.\label{posthomerican} The verb whose modality was determined by \emph{án} attracted it to itself, along with \isi{negation}, \isi{adverbs}, particularly superlatives, and all those constituents for which the hypothetical\is{hypotheticals} character of the clause represented by \emph{án} was most relevant, in the same way that the enclitic\is{enclitics} \isi{pronouns} lost their traditional position because of the growing requirement to assign them the place in the clause that their function seemed to demand. However, as with the \hyperlink{p394}{\emph{[p394]}} enclitic\is{enclitics} \isi{pronouns}, the tradition retained a certain influence with \emph{án}.\is{verb position|)}

First, the tendency to attach to clause-initial words can also be demonstrated for \emph{án}.\label{tisposinitial} This is indisputable for \emph{tis} `someone' and its forms, particularly \emph{pōs}. (Cf. \citealp[175]{Jebb1889} on Sophocles, \textit{Oedipus at Colonus} 1100, who makes reference to (\ref{jebbagam}). Cf. Homer, \textit{Iliad} 9.77, 24.367, and \textit{Odyssey} 8.208 and 10.573.) 

\begin{exe}
\ex τίϲ ἂν ἐν τάχει μὴ περιώδυνοϲ μὴ δεμνιοτήρηϲ μόλοι\\
\gll tís \emph{àn} en tákhei mḕ periṓdunos mḕ demniotḗrēs móloi\\
someone.\textsc{f.nom.sg} \textsc{irr} in haste.\textsc{dat.sg} not very.painful.\textsc{f.nom.sg} not bed-confining.\textsc{f.nom.sg} come.\textsc{3sg.aor.opt}\\
\trans `May some (fate) come quickly, neither too painful nor too lingering ...' (Aeschylus, \textit{Agamemnon} 1448)\footnote{\emph{Translator's note}: Wackernagel mentions line 1402, but the correct reference in \citet{Jebb1889} is the similar example on line 1448. The Perseus edition has \textit{mēdè} instead of the second \textit{mḕ}.}
\label{jebbagam}
\end{exe}

Furthermore, we should make use of Werfer's \citeyearpar[264ff.]{Werfer1814} observation that there are `almost countless examples' of \emph{án} attaching to \emph{gàr} `then'. The number of examples makes it impossible to reproduce, or add to, \citeauthor{Werfer1814}'s collection here. I merely want to observe two things: first, although counterexamples can be adduced from all genres of literature, \emph{gàr an} is still infinitely more frequent than \emph{gàr ... an}; secondly, as a consequence of inserting \emph{án} immediately after \emph{gàr}, the need is often felt to insert \emph{án} again in a later position in the clause: (\ref{garan1})--(\ref{garan17}) (cf. \citealp[408]{Vahlen1865} on 1460b.7).

\begin{exe}
\ex τῷ γὰρ ἂν καὶ μείζονι λέξαιμ᾽ ἂν ἢ ϲοί\\
\gll tôi gàr \emph{àn} kaì meízoni léxaim' \emph{àn} ḕ soí\\
whom.\textsc{m.dat.sg} then \textsc{irr} also greater.\textsc{m.dat.sg} say.\textsc{1sg.aor.opt} \textsc{irr} than you.\textsc{dat}\\
\trans `For to whom more than to you would I speak ...' (Sophocles, \textit{Oedipus Rex} 772)
\label{garan1}
\end{exe}

\begin{exe}
\ex οὐδὲν γὰρ ἂν πράξαιμ᾽ ἄν\\
\gll oudèn gàr \emph{àn} práxaim' \emph{án}\\
nothing.\textsc{acc.sg} then \textsc{irr} do.\textsc{1sg.aor.opt} \textsc{irr}\\
\trans `For I would do nothing ...' (Sophocles, \textit{Oedipus Rex} 882)
\label{garan2}
\end{exe}

\begin{exe}
\ex κἀμοὶ γὰρ ἂν πατήρ γε δακρύων χάριν ἀνῆκτ᾽ ἂν εἰϲ φῶϲ\\
\gll kamoì gàr \emph{àn} patḗr ge dakrúōn khárin anêkt' \emph{àn} eis phôs\\
and=me.\textsc{dat} then \textsc{irr} father.\textsc{nom.sg} even tear.\textsc{gen.pl} grace.\textsc{acc.sg} lead.up.\textsc{3sg.plpf} \textsc{irr} into light.\textsc{acc.sg}\\
\trans `For my father would at least have brought gratitude for tears into the light' (Sophocles, Fragment 513.6; \citealp[254]{Nauck1889})
\label{garan3}
\end{exe}

\begin{exe}
\ex ἀλλ᾽ οὐ γὰρ ἂν τὰ θεῖα κρυπτόντων θεῶν μάθοιϲ ἄν\\
\gll all' ou gàr \emph{àn} tà theîa kruptóntōn theôn máthois \emph{án}\\
but not then \textsc{irr} the.\textsc{n.acc.pl} divine\textsc{.n.acc.pl} hide.\textsc{ptcp.prs.m.gen.pl} god.\textsc{gen.pl} learn.\textsc{2sg.aor.opt} \textsc{irr}\\
\trans `But you would not learn of divine things with the gods hiding them.' (Sophocles, Fragment 833)
\label{garan4}
\end{exe}

\begin{exe}
\ex μόλιϲ γὰρ ἄν τιϲ αὐτὰ τἀναγκαῖ᾽ ὁρᾶν δύναιτ᾽ ἂν ἑϲτὼϲ πολεμίοιϲ ἐναντίοϲ\\
\gll mólis gàr \emph{án} tis autà tanankaî' horân dúnait' \emph{àn} hestṑs polemíois enantíos\\
scarcely then \textsc{irr} someone.\textsc{m.nom.sg} them.\textsc{n.acc.pl} the=necessary.\textsc{n.acc.pl} see.\textsc{prs.inf} can.\textsc{3sg.prs.opt} \textsc{irr} stand.\textsc{ptcp.prf.m.nom.sg} hostile.\textsc{m.dat.pl} opposite.\textsc{m.nom.sg}\\
\trans `For one would scarcely be able to see that which was necessary, standing opposite the foe.' (Euripides, \textit{Suppliants} 855)
\label{garan5}
\end{exe}

\begin{exe}
\ex τὴν Τροίαν γὰρ ἂν δειλοὶ γενόμενοι πλεῖϲτον αἰϲχύνοιμεν ἄν\\
\gll tḕn Troían gàr \emph{àn} deiloì genómenoi pleîston aiskhúnoimen \emph{án}\\
the.\textsc{f.acc.sg} Troy.\textsc{acc} then \textsc{irr} wretched.\textsc{m.nom.pl} become.\textsc{ptcp.aor.mid.m.nom.sg} most shame.\textsc{1pl.prs.opt.act} \textsc{irr}\\
\trans `For we would most greatly disgrace Troy by becoming cowardly.' (Euripides, \textit{Helen} 948)
\label{garan6}
\end{exe}

\begin{exe}
\ex καὶ γὰρ ἂν κεῖνοϲ βλέπων ἀπέδωκεν ἄν ϲοι τῆνδ᾽ ἔχειν\\
\gll kaì gàr \emph{àn} keînos blépōn apédōken \emph{án} soi tênd' ékhein\\
and then \textsc{irr} that.\textsc{m.nom.sg} look.\textsc{ptcp.prs.m.nom.sg} restore.\textsc{3sg.aor} \textsc{irr} you.\textsc{dat} this.\textsc{f.acc.sg} have.\textsc{prs.inf}\\
\trans `For that man, if he could see, would have given that woman back to you to have' (Euripides, \textit{Helen} 1011)
\label{garan7}
\end{exe}

\begin{exe}
\ex εὐμενέϲτερον γὰρ ἂν τῷ φιλτάτῳ μοι Μενέλεῳ τὰ πρόϲφορα δρῴηϲ ἄν\\
\gll eumenésteron gàr \emph{àn} tôi philtátōi moi Menéleōi tà prósphora drṓiēs \emph{án}\\
favourably.\textsc{comp} then \textsc{irr} the.\textsc{m.dat.sg} dearest.\textsc{m.dat.sg} me.\textsc{dat} Menelaus.\textsc{dat} the.\textsc{n.acc.pl} suitable.\textsc{n.dat.pl} do.\textsc{2sg.prs.opt} \textsc{irr}\\
\trans `For you would be better disposed towards my dearest Menelaus while doing what is suitable ...' (Euripides, \textit{Helen} 1298)
\label{garan8}
\end{exe}

\begin{exe}
\ex οὐ γὰρ ἄν ποτε τρέφειν δύναιτ᾽ ἂν μία λόχμη κλέπτα δύο\\
\gll ou gàr \emph{án} pote tréphein dúnait' \emph{àn} mía lókhmē klépta dúo\\
not then \textsc{irr} sometime rear.\textsc{prs.inf} can.\textsc{3sg.prs.opt.pass} \textsc{irr} one.\textsc{f.nom.sg} lair.\textsc{nom.sg} thief.\textsc{acc.du} two\\
\trans `For the same lair can never support two thieves' (Aristophanes, \textit{Wasps} 927)
\label{garan9}
\end{exe}

\begin{exe}
\ex οὐ γὰρ ἂν χαίροντεϲ ἡμεῖϲ τήμερον παυϲαίμεθ᾽ ἄν\\
\gll ou gàr \emph{àn} khaírontes hēmeîs tḗmeron pausaímeth' \emph{án}\\
not then \textsc{irr} rejoice.\textsc{ptcp.prs.m.nom.pl} we.\textsc{nom} today stop.\textsc{1pl.aor.opt.mid} \textsc{irr}\\
\trans `For today we cannot cease rejoicing.' (Aristophanes, \textit{Peace} 321)
\label{garan10}
\end{exe}

\begin{exe}
\ex ἄλλωϲ γὰρ ἂν ἄμαχοι γυναῖκεϲ καὶ μιαραὶ κεκλῄμεθ᾽ ἄν\\
\gll állōs gàr \emph{àn} ámakhoi gunaîkes kaì miaraì keklḗimeth' \emph{án}\\
otherwise then \textsc{irr} invincible.\textsc{f.nom.pl} woman.\textsc{f.nom.pl} and polluted.\textsc{f.nom.pl} confine.\textsc{1pl.prf.pass} \textsc{irr}\\
\trans `For otherwise we would be confined as being unconquerable and foul women' (Aristophanes, \textit{Lysistrata} 252)
\label{garan11}
\end{exe}

\begin{exe}
\ex καὶ γὰρ ἂν μαινοίμεθ᾽ ἄν\\
\gll kaì gàr \emph{àn} mainoímeth' \emph{án}\\
and then \textsc{irr} rave.\textsc{1pl.prs.opt.pass} \textsc{irr}\\
\trans `For we would be mad.' (Aristophanes, \textit{Thesmophoriazusae} 196)
\label{garan12}
\end{exe}

\begin{exe}
\ex ϲαφῶϲ γὰρ ἄν, εἰ πείθοιμι ὑμᾶϲ ..., θεοὺϲ ἂν διδάϲκοιμι\\
\gll saphôs gàr \emph{án}, ei peíthoimi humâs theoùs \emph{àn} didáskoimi\\
clearly then \textsc{irr} if persuade.\textsc{1sg.prs.opt} you.\textsc{acc.pl} god.\textsc{acc.pl} \textsc{irr} teach.\textsc{1pl.prs.opt}\\
\trans `For clearly if I persuaded you, I should be teaching that the gods ...' (Plato, \textit{Apology} 35d)
\label{garan13}
\end{exe}

\begin{exe}
\ex ἐγὼ γὰρ ἂν οἶμαι, εἰ ... δέοι ..., οἶμαι ἂν ... τὸν μέγαν βαϲιλέα εὐαριθμήτουϲ ἂν εὑρεῖν\\
\gll egṑ gàr \emph{àn} oîmai, ei déoi oîmai \emph{àn} tòn mégan basiléa euarithmḗtous \emph{àn} heureîn\\
I.\textsc{nom} then \textsc{irr} think.\textsc{1sg.prs.pass} if lack.\textsc{3sg.prs.opt} think.\textsc{1sg.prs.pass} \textsc{irr} the.\textsc{m.acc.sg} great.\textsc{m.acc.sg} king.\textsc{acc.sg} easily-counted.\textsc{m.acc.pl} \textsc{irr} find.\textsc{aor.inf}\\
\trans `For I think, if it were necessary ... I think that the great king would find few ...' (Plato, \textit{Apology} 40d; cf. example (\ref{anan2}) above)
\label{garan14}
\end{exe}

\begin{exe}
\ex οὔτε γὰρ ἂν αἱ τῆϲ ϲελήνηϲ ἐκλείψειϲ τοιαύταϲ ἂν εἶχον τὰϲ ἀποτομάϲ\\
\gll oúte gàr \emph{àn} hai tês selḗnēs ekleípseis toiaútas \emph{àn} eîkhon tàs apotomás\\
nor then \textsc{irr} the.\textsc{f.nom.pl} the.\textsc{f.gen.sg} moon.\textsc{gen.sg} eclipse.\textsc{nom.pl} such.\textsc{f.acc.pl} \textsc{irr} have.\textsc{3pl.imp} the.\textsc{f.acc.pl} division.\textsc{acc.pl}\\
\trans `For neither would the eclipses of the moon have such divisions.' (Aristotle, \textit{On the Heavens} 227b.24)
\label{garan15}
\end{exe}

\begin{exe}
\ex μέλλων γὰρ ἂν βαδίζειν τιϲ οὐκ ἂν βαδίϲειεν\\
\gll méllōn gàr \emph{àn} badízein tis ouk \emph{àn} badíseien\\
be.going.to.\textsc{ptcp.prs.m.nom.sg} then \textsc{irr} walk.\textsc{prs.inf} someone.\textsc{m.nom.sg} not \textsc{irr} walk.\textsc{3sg.aor.opt}\\
\trans `For someone about to walk would not have walked' (Aristotle, \textit{On Generation and Corruption} 337b.7)
\label{garan16}
\end{exe}

\begin{exe}
\ex οὕτωϲ γὰρ ἂν ἔχον χρηϲιμώτατον ἂν εἴη\\
\gll hoútōs gàr \emph{àn} ékhon khrēsimṓtaton \emph{àn} eíē\\
so then \textsc{irr} have.\textsc{ptcp.prs.n.nom.sg} useful.\textsc{supl.n.nom.sg} \textsc{irr} be.\textsc{3sg.prs.opt}\\
\trans `For having (it) thus would be the most useful.' (Aristotle, \textit{Parts of Animals} 654a.18)
\label{garan17}
\end{exe}

\hyperlink{p395}{\emph{[p395]}} It should also be noted that the joined words \emph{kan} (from \emph{kaì an} `and \textsc{irr}') and \emph{takh' an} `soon \textsc{irr}', in which \emph{án} has coalesced with the previous word to the point of being completely bleached of its original meaning,\is{semantic change} are found at the start of the clause in the majority of cases. However, we should not put too much weight on this, because even \emph{kai an} and \emph{takh' an} can be found in clause-internal positions in Homer,\il{Greek, Homeric} and there is no reason to derive the tight connection of \emph{án} to \emph{kaì} and \emph{takha} from the instances in which \emph{kaì} and \emph{takha} are clause-initial. (The conjunction \emph{kaì} `and' immediately precedes \emph{án} in (\ref{kaian})).

\begin{exe}
\ex καὶ ἂν ἐδήλου\\
\gll kaì \emph{àn} edḗlou\\
and \textsc{irr} show.\textsc{3sg.imp}\\
\trans `... and he would show ...' (Herodotus 4.118.4)
\label{kaian}
\end{exe}

Secondly, \emph{án}, like the \isi{enclitics}, can occasionally be found after a \isi{vocative}, as in (\ref{vocan}).

\begin{exe}
\ex ἀλλ᾽ ὦ μέλ᾽ ἄν μοι ϲιτίων διπλῶν ἔδει\\
\gll all' ô mél' \emph{án} moi sitíōn diplôn édei\\
but O friend.\textsc{voc} \textsc{irr} me.\textsc{dat} food.\textsc{gen.pl} double.\textsc{n.gen.pl} lack.\textsc{3sg.prs}\\
\trans `But, my dear, I would need twice the food.' (Aristophanes, \textit{Peace} 137)
\label{vocan}
\end{exe}

Thirdly, it often displaces \emph{oûn} `so/then', and more rarely \emph{te} and \emph{dè} `and', from their positions: (\ref{displace1})--(\ref{displace13}).

\begin{exe}
\ex οὓτω ἂν ὦν εἶμεν\\
\gll hoùtō \emph{àn} ôn eîmen\\
so \textsc{irr} so be.\textsc{1pl.prs}\\
\trans `Therefore we would thus be ...' (Herodotus 7.150.2; cf. Euripides, \textit{Medea} 504)\footnote{\emph{Translator's note}: The Perseus edition has \textit{eíēmen} for \textit{eîmen}.}
\label{displace1}
\end{exe}

\begin{exe}
\ex τίϲ ἂν οὖν γένοιτ᾽ ἂν ὄρκοϲ\\
\gll tís \emph{àn} oûn génoit' àn órkos\\
what.\textsc{m.nom.sg} \textsc{irr} so become.\textsc{3sg.aor.opt} \textsc{irr} oath.\textsc{nom.sg}\\
\trans `What oath would suit us then?' (Aristophanes, \textit{Lysistrata} 191)
\label{displace2}
\end{exe}

\begin{exe}
\ex πῶϲ ἂν οὖν οὐκ ἂν δεινὰ πάϲχοιμεν\\
\gll pôs \emph{àn} oûn ouk àn deinà páskhoimen\\
how \textsc{irr} so not \textsc{irr} terrible.\textsc{n.acc.pl} suffer.\textsc{1pl.prs.opt}\\
\trans `Then how could we not suffer terrible things?' ({[}Lysias{]} 20.15)
\label{displace3}
\end{exe}

\begin{exe}
\ex πῶϲ ἂν οὖν δὴ τοῦθ᾽ οὕτωϲ ἔχοι ..., ἐγὼ πειράϲομαι φράϲαι\\
\gll pôs \emph{àn} oûn dḕ toûth' hoútōs ékhoi egṑ peirásomai phrásai\\
how \textsc{irr} so exactly this.\textsc{n.nom.sg} so have.\textsc{3sg.prs.opt} I.\textsc{nom} try.\textsc{1sg.fut.mid} tell.\textsc{aor.inf}\\
\trans `So I will try to tell you how this would be.' (Plato, \textit{Phaedo} 64a)
\label{displace4}
\end{exe}

\begin{exe}
\ex πῶϲ ἂν οὖν θεὸϲ εἴη ὅ γε τῶν καλῶν καὶ ἀγαθῶν ἄμοιροϲ\\
\gll pôs \emph{àn} oûn theòs eíē hó ge tôn kalôn kaì agathôn ámoiros\\
how \textsc{irr} so god.\textsc{nom.sg} be.\textsc{3sg.prs.opt} the.\textsc{m.nom.sg} even the.\textsc{n.gen.pl} beautiful.\textsc{n.gen.pl} and good.\textsc{n.gen.pl} devoid.\textsc{m.nom.sg}\\
\trans `How then can he be a god, if he is devoid of things beautiful and good?' (Plato, \textit{Symposium} 202d)
\label{displace5}
\end{exe}

\begin{exe}
\ex πῶϲ ἂν οὖν ἔχοντεϲ τοϲούτοϲ πόρουϲ ... ἔπειτα ἐκ τούτων πάντων τοῦτον ἂν τὸν τρόπον ἐξελοίμεθα ...\\
\gll pôs \emph{àn} oûn ékhontes tosoútos pórous épeita ek toútōn pántōn toûton àn tòn trópon exeloímetha\\
how \textsc{irr} so have.\textsc{ptcp.prs.m.nom.pl} so.many.\textsc{m.nom.pl} way.\textsc{nom.pl} then out this.\textsc{m.gen.pl} all.\textsc{m.gen.pl} this.\textsc{m.acc.sg} \textsc{irr} the.\textsc{m.acc.sg} way.\textsc{acc.sg} choose.\textsc{1pl.aor.opt.mid}\\
\trans `Therefore, having so many ways, how then could we choose this way out of all these ... ?' (Xenophon, \textit{Anabasis} 2.5.20)
\label{displace6}
\end{exe}

\begin{exe}
\ex πῶϲ ἂν οὖν ἐγὼ ἤ βιαϲαίμην ὑμᾶϲ ... ἢ ἐξαπατήϲαϲ ἄγοιμι\\
\gll pôs \emph{àn} oûn egṑ ḗ biasaímēn humâs ḕ exapatḗsas ágoimi\\
how \textsc{irr} so I.\textsc{nom} or force.\textsc{1sg.aor.opt.mid} you.\textsc{acc.pl} or deceive.\textsc{ptcp.aor.m.nom.sg} lead.\textsc{1sg.prs.opt}\\
\trans `Then how could I either force you or lead you by deception?' (Xenophon, \textit{Anabasis} 5.7.8)
\label{displace7}
\end{exe}

\begin{exe}
\ex πῶϲ ἂν οὖν ἀνὴρ μᾶλλον δοίη δίκην\\
\gll pôs \emph{àn} oûn anḕr mâllon doíē díkēn\\
how \textsc{irr} so man.\textsc{nom.sg} more give.\textsc{3sg.aor.opt} judgement.\textsc{acc.sg}\\
\trans `Then how could a man bring down punishment more surely ...' (Xenophon, \textit{Anabasis} 5.7.9)
\label{displace8}
\end{exe}

\begin{exe}
\ex οὐκ ἂν οὖν ῥᾳδίωϲ γέ τιϲ εὕροι Σπαρτιατῶν ... ὑγεινοτέρουϲ\\
\gll ouk \emph{àn} oûn rhāidíōs gé tis heúroi Spartiatôn hugeinotérous\\
not \textsc{irr} so easily even someone.\textsc{m.nom.sg} find.\textsc{3sg.aor.opt} Spartan.\textsc{gen.pl} healthier.\textsc{m.acc.pl}\\
\trans `So one could not easily find healthier men than the Spartans.' (Xenophon, \textit{Constitution of the Lacedaemonians} 5.9)
\label{displace9}
\end{exe}

\begin{exe}
\ex τίϲ ἂν οὖν εὖ φρονῶν αὑτὸν ἂν ἢ τὰ τῆϲ πατρίδοϲ ϲυμφέροντα ταύτῃ ϲυνάψειε\\
\gll tís \emph{àn} oûn eû phronôn hautòn àn ḕ tà tês patrídos sumphéronta taútēi sunápseie\\
who.\textsc{m.nom.sg} \textsc{irr} so well reason.\textsc{ptcp.prs.m.nom.sg} himself.\textsc{acc} \textsc{irr} or the.\textsc{n.acc.pl} the.\textsc{f.gen.sg} fatherland.\textsc{gen.sg} gather.\textsc{ptcp.prs.n.acc.pl} this.\textsc{f.dat.sg} join.\textsc{3sg.aor.opt}\\
\trans `Then who in his right mind would bind himself or his country's interests to this?' (Demosthenes 25.33)
\label{displace10}
\end{exe}

\begin{exe}
\ex πῶϲ ἂν οὖν μὴ εἰδὼϲ ὁ πατὴρ αὐτὸν Ἀθηναῖον ἐϲόμενον ἔδωκεν ἂν τὴν ἑαυτοῦ γυναῖκα\\
\gll pôs \emph{àn} oûn mḕ eidṑs ho patḕr autòn Athēnaîon esómenon édōken àn tḕn heautoû gunaîka\\
how \textsc{irr} so not know.\textsc{ptcp.prf.m.nom.sg} the.\textsc{m.nom.sg} father.\textsc{nom.sg} him.\textsc{acc} Athenian.\textsc{acc.sg} be.\textsc{ptcp.fut.mid.m.acc.sg} give.\textsc{3sg.aor} \textsc{irr} the.\textsc{f.acc.sg} himself.\textsc{gen} woman.\textsc{acc.sg}\\
\trans `How, then, could my father, not knowing that he was to become an Athenian citizen, have given him his own wife ...' ({[}Demosthenes{]} 46.13)
\label{displace11}
\end{exe}

\begin{exe}
\ex ἴϲωϲ ἂν οὖν τιϲ θαυμάϲειεν\\
\gll ísōs \emph{àn} oûn tis thaumáseien\\
perhaps \textsc{irr} so someone.\textsc{m.nom.sg} wonder.\textsc{3sg.aor.opt}\\
\trans `So perhaps someone might wonder ...' (Aeschines 1.17)
\label{displace12}
\end{exe}

\begin{exe}
\ex πῶϲ ἂν οὖν ἐγὼ προεδεικνύμην Ἀλεξάνδρῳ\\
\gll pôs \emph{àn} oûn egṑ proedeiknúmēn Alexándrōi\\
how \textsc{irr} so I.\textsc{nom} demonstrate.\textsc{1sg.imp.pass} Alexander.\textsc{dat}\\
\trans `How then could I have been already making a manifesto to Alexander?' (Aeschines 3.219)
\label{displace13}
\end{exe}

The fact that the \emph{án} that precedes \emph{oûn} is attached to \emph{tís} `what' or \emph{pôs} `how' fits with what was observed above on p\pageref{tisposinitial}. (It should not be denied that \emph{án} follows \emph{oûn} even more frequently.) In (\ref{ante}) \emph{án} precedes \emph{te}; it precedes \emph{de} in (\ref{dean}) and perhaps (\ref{dean2}) (the majority of the manuscripts and editions have \textit{tákha d' \emph{àn} ísōs}).\footnote{\emph{Translator's note}: This is also the version found in the modern Perseus edition.} However, in the last two examples the \hyperlink{p396}{\emph{[p396]}} collocation of \emph{takha} with \emph{án} is of more importance than the position itself.

\begin{exe}
\ex τάχιϲτ᾽ ἄν τε πόλιν οἱ τοιοῦτοι ἀπολέϲειαν\\
\gll tákhist' \emph{án} te pólin hoi toioûtoi apoléseian\\
fastest \textsc{irr} and city.\textsc{acc.sg} the.\textsc{m.nom.pl} such.\textsc{m.nom.pl} destroy.\textsc{3pl.aor.opt}\\
\trans `And such people would ruin a state most quickly' (Thucydides 2.62.3)
\label{ante}
\end{exe}

\begin{exe}
\ex τάχ᾽ ἂν δὲ καὶ ἄλλωϲ ἐϲπλεύϲαντεϲ\\
\gll tákh' \emph{àn} dè kaì állōs espleúsantes\\
quickly \textsc{irr} but also otherwise sail.in.\textsc{ptcp.aor.m.nom.pl}\\
\trans `... but perhaps also sailing in by another way' (Thucydides 6.2.4)\footnote{\emph{Translator's note}: The Perseus edition adds \textit{pōs} after \textit{állōs}.}
\label{dean}
\end{exe}

\begin{exe}
\ex ταχ᾽ ἂν δ᾽ ἴϲωϲ\\
\gll takh' \emph{àn} d' ísōs\\
quickly \textsc{irr} then perhaps\\
\trans (Thucydides 6.10.4)
\label{dean2}
\end{exe}

Fourth, \emph{án} is happy to be separated by an intervening clause from the main elements of the clause to which it belongs: (\ref{clausean1})--(\ref{clausean10}).

\begin{exe}
\ex οὐδ᾽ ἄν, μὰ τὴν Δήμητρα, φροντίϲαιμί γε\\
\gll oud' \emph{án}, mà tḕn Dḗmētra, phrontísaimí ge\\
nor \textsc{irr} by the.\textsc{f.acc.sg} Demeter.\textsc{acc} consider.\textsc{1sg.aor.opt} even\\
\trans `By Demeter, I wouldn't think of it.' (Aristophanes, \textit{Frogs} 1222)
\label{clausean1}
\end{exe}

\begin{exe}
\ex ϲὺ δ᾽ ... οἶμαι, ἄν, ὡϲ ἐγὼ λέγω, ποιοίηϲ\\
\gll sù d' oîmai, \emph{án}, hōs egṑ légō, poioíēs\\
you.\textsc{nom} then think.\textsc{1sg.prs.pass} \textsc{irr} as I.\textsc{nom} say.\textsc{1sg.prs} do.\textsc{2sg.prs.opt}\\
\trans `But you, I think, will do as I say.' (Plato, \textit{Phaedo} 101e)
\label{clausean2}
\end{exe}

\begin{exe}
\ex τί οὖν ἄν, ἔφη, εἴη ὁ Ἔρωϲ\\
\gll tí oûn \emph{án}, éphē, eíē ho Érōs\\
what.\textsc{n.nom.sg} so \textsc{irr} say.\textsc{3sg.imp} be.\textsc{3sg.prs.opt} the.\textsc{m.nom.sg} Eros.\textsc{nom}\\
\trans `{``}What, then,'' he said, ``could Eros be?''' (Plato, \textit{Symposium} 202d)\footnote{\emph{Translator's note}: The Perseus edition has \textit{éphēn} for \textit{éphē}.}
\label{clausean3}
\end{exe}

\begin{exe}
\ex καὶ πῶϲ ἄν, ἔφη, ὦ Σώκρατεϲ, ὁμολογοῖτο\\
\gll kaì pôs \emph{án}, éphē, ô Sṓkrates, homologoîto\\
and how \textsc{irr} say.\textsc{3sg.imp} O Socrates.\textsc{voc} agree.\textsc{3sg.prs.opt.pass}\\
\trans `{``}And how,'' she said, ``Socrates, could it be agreed ... ?''' (Plato, \textit{Symposium} 202b)
\label{clausean4}
\end{exe}

\begin{exe}
\ex πρόϲ γε ὑποδημάτων ἄν, οἶμαι φαίηϲ κτῆϲιν\\
\gll prós ge hupodēmátōn \emph{án}, oîmai phaíēs ktêsin\\
to even shoe.\textsc{gen.pl} \textsc{irr} think.\textsc{1sg.prs.pass} say.\textsc{2sg.prs.opt} acquisition.\textsc{acc.sg}\\
\trans `For obtaining shoes, I think, you would say?' (Plato, \textit{Republic} 1.333a)
\label{clausean5}
\end{exe}

\begin{exe}
\ex ἴϲωϲ γὰρ ἄν, ἔφη, δοκοίη τι λέγειν ὁ ταῦτα λέγων\\
\gll ísōs gàr \emph{án}, éphē, dokoíē ti légein ho taûta légōn\\
perhaps then \textsc{irr} say.\textsc{3sg.imp} seem.\textsc{3sg.prs.opt} something.\textsc{acc.sg} say.\textsc{prs.inf} the.\textsc{m.nom.sg} this.\textsc{n.acc.pl} say.\textsc{ptcp.prs.m.nom.sg}\\
\trans `{``}Perhaps, then,'' he said, ``someone saying this would seem to be saying something.''' (Plato, \textit{Republic} 4.438a)
\label{clausean6}
\end{exe}

\begin{exe}
\ex τί ἄν, εἰ ... (seven lines follow) τί ποτ᾽ ἂν ἡγούμεθα ἐκ ταύτηϲ τῆϲ προρρήϲεωϲ ξυμβαίνειν\\
\gll tí \emph{án}, ei tí pot' àn hēgoúmetha ek taútēs tês prorrhḗseōs xumbaínein\\
what.\textsc{n.acc.sg} \textsc{irr} if what.\textsc{n.acc.sg} sometime \textsc{irr} lead.\textsc{1pl.imp.pass} out this.\textsc{f.gen.sg} the.\textsc{f.gen.sg} proclamation.\textsc{gen.sg} occur.\textsc{prs.inf}\\
\trans `What, if ... what do we think would ever result from this proclamation?' (Plato, \textit{Laws} 2.658a)
\label{clausean7}
\end{exe}

\begin{exe}
\ex οἶμαι ἄν, αὐτῶν εἰ καλῶϲ τιϲ ἐπιμελοῖτο, οὐκ εἶναι ἔθνοϲ\\
\gll oîmai \emph{án}, autôn ei kalôs tis epimeloîto, ouk eînai éthnos\\
think.\textsc{1sg.prs.pass} \textsc{irr} them.\textsc{gen} if well someone.\textsc{m.nom.sg} manage.\textsc{3sg.prs.opt.pass} not be.\textsc{prs.inf} people.\textsc{m.nom.sg}\\
\trans `I think that, if one managed them well, there would be no people ...' (Xenophon, \textit{Hellenica} 6.1.9)
\label{clausean8}
\end{exe}

\begin{exe}
\ex ἐγὼ ἄν, εἰ ἔχοιμι, ὡϲ τάχιϲτα ὅπλα ἐποιούμην τοῖϲ Πέρϲαιϲ\\
\gll egṑ \emph{án}, ei ékhoimi, hōs tákhista hópla epoioúmēn toîs Pérsais\\
I.\textsc{nom} \textsc{irr} if have.\textsc{1sg.prs.opt} as fastest armour.\textsc{acc.pl} make.\textsc{1sg.imp.pass} the.\textsc{m.dat.pl} Persian.\textsc{dat.pl}\\
\trans `I, if I could have it, would have armour made for the Persians as quickly as possible.' (Xenophon, \textit{Cyropaedia} 2.1.9)\footnote{\emph{Translator's note}: The Perseus edition has \textit{egṑ mèn án, éphē ho Kûros, ei su eíēn, hōs tákhista hópla poioímēn}.}
\label{clausean9}
\end{exe}

\begin{exe}
\ex τί ἄν, εἴ που τῆϲ χώραϲ τοῦτο πάθοϲ ϲυνέβη, προϲδοκῆϲαι χρῆν\\
\gll tí \emph{án}, eí pou tês khṓras toûto páthos sunébē, prosdokêsai khrên\\
what.\textsc{n.acc.sg} \textsc{irr} if somewhere the.\textsc{f.gen.sg} country.\textsc{gen.sg} this.\textsc{n.nom.sg} experience.\textsc{nom.sg} occur.\textsc{3sg.aor} expect.\textsc{aor.inf} need.\textsc{3sg.prs}\\
\trans `What, if this misfortune occurred somewhere in our country, would it be necessary to expect?' (Demosthenes 18.195)
\label{clausean10}
\end{exe}

It is understandable that there is a tendency to insert \emph{án} again after the intervening clause: see example (\ref{multiplean1}) above, and (\ref{anrepeat1})--(\ref{anrepeat14}) (also Xenophon, \textit{Anabasis} 7.7.38).

\begin{exe}
\ex οὔτ᾽ ἄν, εἰ θέλοιϲ ἔτι πράϲϲειν, ἐμοῦ γ᾽ ἂν ἡδέωϲ πράϲϲοιϲ μέτα\\
\gll oút' \emph{án}, ei thélois éti prássein, emoû g' \emph{àn} hēdéōs prássois méta\\
nor \textsc{irr} if want.\textsc{2sg.prs.opt} still do.\textsc{prs.inf} me.\textsc{gen} even \textsc{irr} sweetly do.\textsc{2sg.prs.opt} with\\
\trans `... nor, even if you still wanted to do so, would you willingly do so with me.' (Sophocles, \textit{Antigone} 69)\footnote{\emph{Translator's note}: The Perseus edition has \textit{drṓiēs} for \textit{prássois}.}
\label{anrepeat1}
\end{exe}

\begin{exe}
\ex ἀλλ᾽ ἄν, εἰ τὸν ἐξ ἐμῆϲ μητρὸϲ θανόντ᾽ ἄθαπτον ἠνϲχόμην νέκυν, κείνοιϲ ἂν ἤλγουν\\
\gll all' \emph{án}, ei tòn ex emês mētròs thanónt' áthapton ēnskhómēn nékun, keínois \emph{àn} ḗlgoun\\
but \textsc{irr} if the.\textsc{m.acc.sg} out my.\textsc{f.gen.sg} mother.\textsc{gen.sg} die.\textsc{ptcp.aor.m.acc.sg} unburied.\textsc{m.acc.sg} sustain.\textsc{1sg.aor.mid} corpse.\textsc{acc.sg} that.\textsc{n.dat.pl} \textsc{irr} hurt.\textsc{1sg.imp}\\
\trans `But if I had endured the dead son of my mother as an unburied corpse, I would have suffered from that.' (Sophocles, \textit{Antigone} 466)
\label{anrepeat2}
\end{exe}

\begin{exe}
\ex ὥϲτ᾽ ἄν, εἰ ϲθένοϲ λάβοιμι, δηλώϲαιμ᾽ ἄν\\
\gll hṓst' \emph{án}, ei sthénos láboimi, dēlṓsaim' \emph{án}\\
so \textsc{irr} if strength.\textsc{acc.sg} take.\textsc{1sg.aor.opt} show.\textsc{1sg.aor.opt} \textsc{irr}\\
\trans `... so that, if I could find strength, I would show ...' (Sophocles, \textit{Electra} 333)
\label{anrepeat3}
\end{exe}

\begin{exe}
\ex ἀρχὴν δ᾽ ἄν, εἰ μὴ τλημονεϲτάτη γυνὴ παϲῶν ἔβλαϲτε, ... χοὰϲ οὐκ ἄν ποθ᾽ ὃν γ᾽ ἔκτεινε, τῷδ᾽ ἐπέϲτεφε\\
\gll arkhḕn d' \emph{án}, ei mḕ tlēmonestátē gunḕ pasôn éblaste khoàs ouk \emph{án} poth' hòn g' ékteine, tôid' epéstephe\\
beginning.\textsc{acc.sg} then \textsc{irr} if not audacious.\textsc{supl.f.nom.sg} woman.\textsc{nom.sg} all.\textsc{gen.pl} bud.\textsc{3sg.aor} libation.\textsc{acc.pl} not \textsc{irr} sometime whom.\textsc{m.acc.sg} even kill.\textsc{3sg.imp} this.\textsc{m.dat.sg} pour.\textsc{3sg.imp}\\
\trans `To begin with, if she had not been born the most audacious woman of all, she would never have poured offerings to this man whom she had killed' (Sophocles, \textit{Electra} 439)
\label{anrepeat4}
\end{exe}

\begin{exe}
\ex ἐκεῖνον δ᾽ ἄν, εἰ ἐκδοίη αὐτόν ..., ϲωτηρίαϲ ἂν τῆϲ ψυχῆϲ ἀποϲτερῆϲαι\\
\gll ekeînon d' \emph{án}, ei ekdoíē autón sōtērías \emph{àn} tês psukhês aposterêsai\\
that.\textsc{m.acc.sg} then \textsc{irr} if give.up.\textsc{3sg.aor.opt} him.\textsc{acc} salvation.\textsc{gen.sg} \textsc{irr} the.\textsc{f.gen.sg} soul.\textsc{gen.sg} rob.\textsc{3sg.aor.opt}\\
\trans `And if he gave him up, he would be depriving him of the safety of his life' (Thucydides 1.136.4)
\label{anrepeat5}
\end{exe}

\begin{exe}
\ex κἄν, ὑμῖν εἴ τιϲ ἐνῆν νοῦϲ, ἐκ τῶν ἐρίων τῶν ἡμετέρων ἐπολιτεύεϲθ᾽ ἂν ἅπαντα\\
\gll k\emph{án}, humîn eí tis enên noûs, ek tôn eríōn tôn hēmetérōn epoliteúesth' \emph{àn} hápanta\\
and=\textsc{irr} you.\textsc{dat.pl} if someone.\textsc{m.nom.sg} spin.\textsc{3sg.imp} mind.\textsc{acc.pl} out the.\textsc{n.gen.pl} wool.\textsc{gen.pl} the.\textsc{n.gen.pl} our.\textsc{n.gen.pl} be.citizen.\textsc{2pl.imp.pass} \textsc{irr} quite.all.\textsc{n.acc.pl}\\
\trans `And if someone could spin minds for you out of our wool, you could govern everything.' (Aristophanes, \textit{Lysistrata} 572)
\label{anrepeat6}
\end{exe}

\begin{exe}
\ex κἄν, εἴ με τύπτοιϲ, οὐκ ἂν ἀντείποιμί ϲοι\\
\gll k\emph{án}, eí me túptois, ouk \emph{àn} anteípoimí soi\\
and=\textsc{irr} if me.\textsc{acc} beat.\textsc{2sg.prs.opt} not \textsc{irr} contradict.\textsc{1sg.aor.opt} you.\textsc{dat}\\
\trans `Even if you beat me, I'd never contradict you.' (Aristophanes, \textit{Frogs} 585)
\label{anrepeat7}
\end{exe}

\begin{exe}
\ex κἄν, εἰ Ὀρθαγόρᾳ τῷ Θηβαίῳ ϲυγγενόμενοϲ ... ἐπανέροιτο αὐτόν ..., εἴποι ἄν\\
\gll k\emph{án}, ei Orthagórāi tôi Thēbaíōi sungenómenos epanéroito autón eípoi \emph{án}\\
and=\textsc{irr} if Orthagoras.\textsc{dat} the.\textsc{m.dat.sg} Theban.\textsc{dat.sg} converse.\textsc{ptcp.aor.mid.m.nom.sg} enquire.\textsc{3sg.aor.opt.mid} him.\textsc{acc} say.\textsc{3sg.aor.opt} \textsc{irr}\\
\trans `And if, having studied with Orthagoras the Theban, he enquired of him ... he would say ...' (Plato, \textit{Protagoras} 318c)
\label{anrepeat8}
\end{exe}

\begin{exe}
\ex τάχα δ᾽ ἄν, εἰ θεὸϲ ἐθέλοι, κἂν δυοῖν θάτερα βιαϲαίμεθα περὶ ἐρωτικῶν\\
\gll tákha d' \emph{án}, ei theòs ethéloi, k\emph{àn} duoîn thátera biasaímetha perì erōtikôn\\
quickly then \textsc{irr} if god.\textsc{nom.sg} want.\textsc{3sg.prs.opt} also=\textsc{irr} two.\textsc{n.gen.du} the=other.\textsc{n.acc.pl} force.\textsc{1pl.aor.opt.mid} about erotic.\textsc{n.gen.pl}\\
\trans `Possibly, should God so grant, we might forcibly effect one of two things in this matter of sex-relations' (Plato, \textit{Laws} 8.841c)
\label{anrepeat9}
\end{exe}

\begin{exe}
\ex ἐπιϲχὼν ἄν, ἕωϲ ..., εἰ ..., ἡϲυχίαν ἂν ἦγον\\
\gll episkhṑn \emph{án}, héōs ei hēsukhían \emph{àn} êgon\\
wait.\textsc{ptcp.aor.m.nom.sg} \textsc{irr} until if silence.\textsc{acc.sg} \textsc{irr} lead.\textsc{1sg.imp}\\
\trans `Having waited until ... if ... I should have held my peace.' (Demosthenes 4.1)
\label{anrepeat10}
\end{exe}

\begin{exe}
\ex ἆρ᾽ ἄν, εἴ γ᾽ εἶχε ..., ταῦτ᾽ ἂν εἴαϲεν\\
\gll âr' \emph{án}, eí g' eîkhe taût' \emph{àn} eíasen\\
then \textsc{irr} if even have.\textsc{3sg.imp} this.\textsc{n.acc.pl} \textsc{irr} allow.\textsc{3sg.aor}\\
\trans `So if he had even had ... would he have allowed these things?' (Demosthenes 21.115)
\label{anrepeat11}
\end{exe}

\begin{exe}
\ex οὐδ᾽ ἄν, εἴ τι γένοιτ᾽, ᾠήθην ἂν δίκην μοι λαχεῖν ποτε τοῦτον\\
\gll oud' \emph{án}, eí ti génoit', ōiḗthēn \emph{àn} díkēn moi lakheîn pote toûton\\
nor \textsc{irr} if something.\textsc{nom.sg} become.\textsc{3sg.aor.opt.mid} think.\textsc{1sg.aor.pass} \textsc{irr} judgement.\textsc{acc.sg} me.\textsc{dat} obtain.\textsc{aor.inf} sometime this.\textsc{m.acc.sg}\\
\trans `Nor, if anything happened, did I think that this man would ever bring a suit against me.' (Demosthenes 37.16)
\label{anrepeat12}
\end{exe}

\begin{exe}
\ex καίτοι πῶϲ ἄν, εἰ μὴ πεποριϲμένον τε ἦν ..., εὐθὺϲ ἂν ἀπέλαβον\\
\gll kaítoi pôs \emph{án}, ei mḕ peporisménon te ên euthùs \emph{àn} apélabon\\
and.yet how \textsc{irr} if not bring.\textsc{ptcp.prf.pass.n.nom.sg} and be.\textsc{3sg.imp} straight \textsc{irr} receive.\textsc{3pl.aor}\\
\trans `And yet how, if it had not been provided, would they have received it immediately?' ({[}Demosthenes{]} 47.66)
\label{anrepeat13}
\end{exe}

\begin{exe}
\ex οἶμαι δ᾽ ἄν, εἰ ..., ταῖϲ ὑμετέραιϲ μαρτυρίαιϲ ῥᾳδίωϲ ἂν ἀπολύϲαϲθαι τοὺϲ τοῦ κατηγόρου λόγουϲ\\
\gll oîmai d' \emph{án}, ei taîs humetérais marturíais rhāidíōs \emph{àn} apolúsasthai toùs toû katēgórou lógous\\
think.\textsc{1sg.prs.pass} then \textsc{irr} if the.\textsc{f.dat.pl} your.\textsc{f.dat.pl} testimony.\textsc{dat.pl} easily \textsc{irr} release.\textsc{aor.inf.mid} the.\textsc{m.acc.pl} the.\textsc{m.gen.sg} accuser.\textsc{gen.sg} account.\textsc{acc.pl}\\
\trans `And I think that if ... your testimony would easily refute my accuser's words.' (Aeschines 1.122)
\label{anrepeat14}
\end{exe}

The opposite tendency, so to speak, which nevertheless springs from the same positional rule, is found when an \hyperlink{p397}{\emph{[p397]}} \textit{án} belonging to an intervening clause or to a subordinate\is{subordination} clause is attracted to a position after the first word in the superordinate clause: (\ref{anraise1})--(\ref{anraise7}).

\begin{exe}
\ex ἄλλο τι οὖν, ἂν φαῖεν, ἢ ξυνθήκαϲ τὰϲ πρὸϲ ἡμᾶϲ αὐτοὺϲ ... παραβαίνειϲ\\
\gll állo ti oûn, \emph{àn} phaîen, ḕ xunthḗkas tàs pròs hēmâs autoùs parabaíneis\\
other.\textsc{n.acc.sg} something.\textsc{acc.sg} so \textsc{irr} say.\textsc{3pl.prs.opt} than compact.\textsc{acc.pl} the.\textsc{f.acc.pl} to us.\textsc{acc} same.\textsc{m.acc.pl} overstep.\textsc{2sg.prs}\\
\trans `{``}Then are you not,'' they would say, ``transgressing against something besides your agreements with us ourselves?''' (Plato, \textit{Crito} 52d)
\label{anraise1}
\end{exe}

\begin{exe}
\ex τί οὖν, ἂν φαίη ὁ λόγοϲ, ἔτι ἀπιϲτεῖϲ\\
\gll tí oûn, \emph{àn} phaíē ho lógos, éti apisteîs\\
what.\textsc{n.acc.sg} so \textsc{irr} say.\textsc{3sg.prs.opt} the.\textsc{m.nom.sg} account.\textsc{nom.sg} still distrust.\textsc{2sg.prs}\\
\trans `{``}Why, then,'' the argument might say, ``do you still disbelieve ... ?''' (Plato, \textit{Phaedo} 87a)
\label{anraise2}
\end{exe}


\exewidth{(1234567)}
\begin{exe}
\ex μανθάνω, ἂν ἴϲοϲ φαίη, καὶ ἐγώ\\
\gll manthánō, \emph{àn} ísos phaíē, kaì egṓ\\
learn.\textsc{1sg.prs} \textsc{irr} equal.\textsc{m.nom.sg} say.\textsc{3sg.prs.opt} also I.\textsc{nom}\\
\trans `{``}I too understand,'' he would likewise say ...' (Plato, \textit{Hippias Major} 299a)\footnote{\emph{Translator's note}: The Perseus edition has \textit{ísōs} for \textit{ísos}.}
\label{anraise3}
\end{exe}

\begin{exe}
\ex τί οὖν, ἄν τιϲ εἴποι, ταῦτα λέγειϲ\\
\gll tí oûn, \emph{án} tis eípoi, taûta légeis\\
what.\textsc{n.acc.sg} so \textsc{irr} someone.\textsc{m.nom.sg} say.\textsc{3sg.aor.opt} this.\textsc{n.acc.pl} say.\textsc{2sg.prs}\\
\trans `{``}Why, then,'' someone might say, ``do you say these things ... ?''' (Demosthenes 1.14)
\label{anraise4}
\end{exe}

\begin{exe}
\ex τί οὖν, ἄν τιϲ εἴποι, ϲὺ γράφειϲ ταῦτ᾽ εἶναι ϲτρατιωτικά\\
\gll tí oûn, \emph{án} tis eípoi, sù grápheis taût' eînai stratiōtiká\\
what.\textsc{n.acc.sg} so \textsc{irr} someone.\textsc{m.nom.sg} say.\textsc{3sg.aor.opt} you.\textsc{nom} write.\textsc{2sg.prs} this.\textsc{n.acc.pl} be.\textsc{prs.inf} military.\textsc{n.acc.pl}\\
\trans `{``}Why, then,'' someone might say, ``do you propose that these things should be for military purposes?''' (Demosthenes 1.19)
\label{anraise5}
\end{exe}

\begin{exe}
\ex τί οὖν, ἄν τιϲ εἴποι, ϲὺ παραινεῖϲ\\
\gll tí oûn, \emph{án} tis eípoi, sù paraineîs\\
what.\textsc{n.acc.sg} so \textsc{irr} someone.\textsc{m.nom.sg} say.\textsc{3sg.aor.opt} you.\textsc{nom} advise.\textsc{2sg.prs}\\
\trans `{``}What, then,'' someone might say, ``do you advise ... ?''' (Demosthenes, \textit{Exordia} 35.4)
\label{anraise6}
\end{exe}

\begin{exe}
\ex ὅτι νὴ Δί᾽, ἂν εἴποι, τοῦτον εἰϲπεποίηκα υἱόν\\
\gll hóti nḕ Dí', \emph{àn} eípoi, toûton eispepoíēka huión\\
that yes Zeus.\textsc{acc} \textsc{irr} say.\textsc{3sg.aor.opt} this.\textsc{m.acc.sg} adopt.\textsc{1sg.prf} son.\textsc{acc.sg}\\
\trans `{``}Yes, by Zeus,'' he might say, ``because I have had him adopted ...''' ({[}Demosthenes{]} 44.55)
\label{anraise7}
\end{exe}

Cf. also examples (\ref{anraise8}), (\ref{anraise9}), and similarly (\ref{anraise10}) in the interior of the clause in Demosthenes 45.7. The Euripidean usage in example (\ref{anraise11}) (also \textit{Alcestis} 48, with \textit{ou gàr} `not then' instead of \textit{ouk}) is, in turn, probably based on similar constructions. Thucydides 5.9.3 ((\ref{hosan2}) above) is peculiar, and the first \emph{án} can probably only be explained as an anticipation of the subordinate\is{subordination} clause.

\begin{exe}
\ex οὐκ ἂν οἶδ᾽ ὅ τι πλέον εὕροι τούτου\\
\gll ouk \emph{àn} oîd' hó~ti pléon heúroi toútou\\
not \textsc{irr} know.\textsc{1sg.prf} which.\textsc{n.acc.sg} more.\textsc{n.acc.sg} find.\textsc{3sg.aor.opt} this.\textsc{n.gen.sg}\\
\trans `I do not know how much more than this it would fetch.' (Isaeus 11.44)\footnote{\emph{Translator's note}: The Perseus edition has \textit{ou gàr ... hóti} for \textit{ouk àn ... hó ti}. Wackernagel cites this as Demosthenes 11.44 but the correct reference is Isaeus 11.44.}
\label{anraise8}
\end{exe}

\begin{exe}
\ex ἐγὼ γάρ, ἃ μὲν χθὲϲ ἤκουϲα, οὐκ ἂν οἶδ᾽ εἰ δυναίμην ἅπαντα ἐν μνήμῃ πάλιν λαβεῖν\\
\gll egṑ gár, hà mèn khthès ḗkousa, ouk \emph{àn} oîd' ei dunaímēn hápanta en mnḗmēi pálin labeîn\\
I.\textsc{nom} then which.\textsc{n.acc.pl} then yesterday hear.\textsc{1sg.aor} not \textsc{irr} know.\textsc{1sg.prf} if can.\textsc{1sg.prs.opt.pass} quite.all.\textsc{n.acc.pl} in memory.\textsc{dat.sg} again take.\textsc{aor.inf}\\
\trans `For I do not know if I could recall to mind everything that I heard yesterday.' (Plato, \textit{Timaeus} 26b)
\label{anraise9}
\end{exe}

\begin{exe}
\ex οὐκ ἂν οἶδ᾽ ὅ τι\\
\gll ouk \emph{àn} oîd' hó~ti\\
not \textsc{irr} know.\textsc{1sg.prf} which.\textsc{n.acc.sg}\\
\trans `... I do not know what ...' (Demosthenes 45.7)
\label{anraise10}
\end{exe}

\begin{exe}
\ex οὐκ οἶδ᾽ ἂν εἰ πείϲαιμι\\
\gll ouk oîd' \emph{àn} ei peísaimi\\
not \textsc{irr} know.\textsc{1sg.prf} if persuade.\textsc{1sg.aor.opt}\\
\trans `I do not know if I can persuade ...' (Euripides, \textit{Medea} 941)
\label{anraise11}
\end{exe}

Sixth, just like the \isi{enclitics}, \emph{án} often splits clause-initial word groups apart. Under this heading one could count \textit{oud' àn heîs}, as in (\ref{osperanei7}) above as well as (\ref{ansplit1})--(\ref{ansplit7}).

\begin{exe}
\ex οὐδ᾽ ἂν εἷϲ δύναιτ᾽ ἀνήρ\\
\gll oud' \emph{àn} heîs dúnait' anḗr\\
nor \textsc{irr} one.\textsc{m.nom.sg} can.\textsc{3sg.prs.opt.pass} man.\textsc{nom.sg}\\
\trans `Nor could any man ...' (Sophocles, \textit{Oedipus Rex} 281)
\label{ansplit1}
\end{exe}

\begin{exe}
\ex οὐδ᾽ ἂν εἷϲ θνητῶν φράϲειε\\
\gll oud' \emph{àn} heîs thnētôn phráseie\\
nor \textsc{irr} one.\textsc{m.nom.sg} mortal.\textsc{gen.pl} tell.\textsc{3sg.aor.opt}\\
\trans `Nor could any mortal tell ...' (Sophocles, \textit{Oedipus at Colonus} 1656)
\label{ansplit2}
\end{exe}

\begin{exe}
\ex οὐδ᾽ ἂν εἷϲ ἀμφιϲβητήϲειε\\
\gll oud' \emph{àn} heîs amphisbētḗseie\\
nor \textsc{irr} one.\textsc{m.nom.sg} dispute.\textsc{3sg.aor.opt}\\
\trans `Not one could compete ...' (Plato, \textit{First Alcibiades}, 122d)
\label{ansplit3}
\end{exe}

\begin{exe}
\ex οὐδ᾽ ἂν εἷϲ εὖ οἶδ᾽ ὅτι φήϲειεν\\
\gll oud' \emph{àn} heîs eû oîd' hóti phḗseien\\
nor \textsc{irr} one.\textsc{m.nom.sg} well know.\textsc{3sg.prf} that say.\textsc{3sg.aor.opt}\\
\trans `Nor does anyone not know well that he would say ...' (Demosthenes 19.312)
\label{ansplit4}
\end{exe}

\begin{exe}
\ex οὐδ᾽ ἂν εἷϲ ταῦτα φήϲειεν\\
\gll oud' \emph{àn} heîs taûta phḗseien\\
nor \textsc{irr} one.\textsc{m.nom.sg} this.\textsc{n.acc.pl} say.\textsc{3sg.aor.opt}\\
\trans `Nor would anyone say these things.' (Demosthenes 18.69)
\label{ansplit5}
\end{exe}

\begin{exe}
\ex οὐδ᾽ ἂν εἷϲ εἰπεῖν ἔχοι\\
\gll oud' \emph{àn} heîs eipeîn ékhoi\\
nor \textsc{irr} one.\textsc{m.nom.sg} say.\textsc{aor.inf} have.\textsc{3sg.prs.opt}\\
\trans `Nor would anyone say these things.' (Demosthenes 18.94)
\label{ansplit6}
\end{exe}

\begin{exe}
\ex οὐδ᾽ ἂν εἷϲ εἴποι\\
\gll oud' \emph{àn} heîs eípoi\\
nor \textsc{irr} one.\textsc{m.nom.sg} say.\textsc{3sg.aor.opt}\\
\trans `Nor would anyone say ...' (Aristotle, \textit{Constitution of the Athenians} 7.4)
\label{ansplit7}
\end{exe}

However, this \isi{tmesis} is found at least as often clause-internally (Lysias 19.60, 24.24, Isocrates 15.223, 21.20, Plato, \textit{Symposium} 192e, 214d, 216e, \textit{Gorgias} 512e, 519c, Demosthenes 14.1,\footnote{\emph{Translator's note}: The Perseus edition lacks \textit{án}.} 20.136, 18.68,\footnote{\emph{Translator's note}: The Perseus edition has \textit{oudeìs an}.} 18.128, Lycurgus 49.57), and thus appears to be due to the attracting force of \emph{oude} `nor'.

The two instances of \emph{g' an oun} `even \textsc{irr} so' instead of \emph{goun an} in Thucydides, (\ref{ganoun1}) and (\ref{ganoun2}), constitute better evidence, as well as examples (\ref{wordgroup1})--(\ref{wordgroup40}), in which \emph{án} is inserted into the middle of a word group.

\begin{exe}
\ex ἄλλουϲ γ᾽ ἂν οὖν οἰόμεθα τὰ ἡμέτερα λαβόντεϲ δεῖξαι ἄν\\
\gll állous g' \emph{àn} oûn oiómetha tà hēmétera labóntes deîxai án\\
other.\textsc{m.acc.pl} even \textsc{irr} so think.\textsc{1pl.prs.pass} the.\textsc{n.acc.pl} our.\textsc{n.acc.pl} take.\textsc{ptcp.aor.m.nom.pl} show.\textsc{aor.inf} \textsc{irr}\\
\trans `We think that by taking others it would at least show ours ...' (Thucydides 1.76.4)\footnote{\emph{Translator's note}: The Perseus edition has \textit{labóntas} for \textit{labóntes}.}
\label{ganoun1}
\end{exe}

\begin{exe}
\ex ὑμεῖϲ γ᾽ ἂν οὖν, εἰ ... ἄρξαιτε, τάχ᾽ ἂν ... μεταβάλοιτε\\
\gll humeîs g' \emph{àn} oûn, ei árxaite, tákh' àn metabáloite\\
you.\textsc{nom.pl} even \textsc{irr} so if begin.\textsc{2pl.aor.opt} quickly \textsc{irr} exchange.\textsc{2pl.aor.opt}\\
\trans `If you were to lead, then you would soon change ...' (Thucydides 1.77.6)
\label{ganoun2}
\end{exe}

\begin{exe}
\ex πολλῶν ἂν ἀνδρῶν ἧδ᾽ ἐχηρώθη πόλιϲ\\
\gll pollôn \emph{àn} andrôn hêd' ekhērṓthē pólis\\
many.\textsc{m.gen.pl} \textsc{irr} man.\textsc{gen.pl} this.\textsc{f.nom.sg} bereave.\textsc{3sg.aor.pass} city.\textsc{nom.sg}\\
\trans `This city would have been bereft of many men.' (Aristotle, \textit{Constitution of the Athenians} 12.4)
\label{wordgroup1}
\end{exe}

\begin{exe}
\ex μόνοϲ ἂν θνητῶν πέραϲ εἴποι\\
\gll mónos \emph{àn} thnētôn péras eípoi\\
alone.\textsc{m.nom.sg} \textsc{irr} mortal.\textsc{gen.pl} end.\textsc{acc.sg} say.\textsc{3sg.aor.opt}\\
\trans `He alone of mortals can declare how to bring it to accomplishment.' (Aeschylus, \textit{Persians} 632)
\label{wordgroup2}
\end{exe}

\begin{exe}
\ex ἀνθρώπεια δ᾽ ἄν τοι πήματ᾽ ἂν τύχοι βροτοῖϲ\\
\gll anthrṓpeia d' \emph{án} toi pḗmat' àn túkhoi brotoîs\\
human.\textsc{n.nom.pl} then \textsc{irr} lo harm.\textsc{nom.pl} \textsc{irr} happen.\textsc{3sg.aor.opt} mortal.\textsc{dat.pl}\\
\trans `Afflictions ordained for human life must, we know, befall mankind.' (Aeschylus, \textit{Persians} 706)
\label{wordgroup3}
\end{exe}

\begin{exe}
\ex κατὰ δ᾽ ἄν τιϲ ἐμοῦ τοιαῦτα λέγων οὐκ ἂν πείθοι\\
\gll katà d' \emph{án} tis emoû toiaûta légōn ouk àn peíthoi\\
down then \textsc{irr} someone.\textsc{m.nom.sg} me.\textsc{gen} such.\textsc{n.acc.pl} say.\textsc{ptcp.prs.mp.nom.sg} not \textsc{irr} persuade.\textsc{3sg.aor.opt}\\
\trans `But someone saying such things against me would fail to convince' (Sophocles, \textit{Ajax} 155)
\label{wordgroup4}
\end{exe}

\hyperlink{p398}{\emph{[p398]}}

\begin{exe}
\ex ἄλλον δ᾽ ἂν ἄλλῳ προϲίδοιϲ\\
\gll állon d' \emph{àn} állōi prosídois\\
other.\textsc{m.acc.sg} then \textsc{irr} other.\textsc{m.dat.sg} behold.\textsc{2sg.aor.opt}\\
\trans `And you can see one after another ...' (Sophocles, \textit{Oedipus Rex} 175)
\label{wordgroup5}
\end{exe}

\begin{exe}
\ex ϲοφίᾳ δ᾽ ἂν ϲοφίαν παραμείψειεν ἀνήρ\\
\gll sophíāi d' \emph{àn} sophían parameípseien anḗr\\
wisdom.\textsc{dat.sg} then \textsc{irr} wisdom.\textsc{acc.sg} pass.\textsc{3sg.aor.opt} man. \textsc{nom.sg}\\
\trans `... though man may surpass man in wisdom' (Sophocles, \textit{Oedipus Rex} 502)
\label{wordgroup6}
\end{exe}

\begin{exe}
\ex τίϲ οὖν ἂν ὑμῶν τοῖϲ ἔϲω φράϲειεν ἄν\\
\gll tís oûn \emph{àn} humôn toîs ésō phráseien án\\
who.\textsc{m.nom.sg} so \textsc{irr} you.\textsc{gen.pl} the.\textsc{m.dat.pl} inside tell.\textsc{3sg.aor.opt} \textsc{irr}\\
\trans `Which of you, then, would tell those inside ... ?' (Sophocles, \textit{Electra} 1103)
\label{wordgroup7}
\end{exe}

\begin{exe}
\ex τίϲ ἂν θεῶν ϲοι τόνδ᾽ ἄριϲτον ἄνδρ᾽ ἰδεῖν δοίη\\
\gll tís \emph{àn} theôn soi tónd' áriston ándr' ideîn doíē\\
who.\textsc{m.nom.sg} \textsc{irr} god.\textsc{gen.pl} you.\textsc{dat} this.\textsc{m.acc.sg} best.\textsc{m.acc.sg} man.\textsc{acc.sg} see.\textsc{aor.inf} give.\textsc{3sg.aor.opt}\\
\trans `Which of the gods might grant that you could see this best of men ... ?' (Sophocles, \textit{Oedipus at Colonus} 1100)
\label{wordgroup8}
\end{exe}

\begin{exe}
\ex ἐφρόντιζε ἱϲτορέων, τοὺϲ ἂν Ἑλλήνων δυνατωτάτουϲ ἐόνταϲ προϲκτήϲαιτο φίλουϲ\\
\gll ephróntize historéōn, toùs \emph{àn} Hellḗnōn dunatōtátous eóntas prosktḗsaito phílous\\
consider.\textsc{3sg.imp} enquire.\textsc{ptcp.prs.m.nom.sg} whom.\textsc{m.acc.pl} \textsc{irr} Greek.\textsc{gen.pl} mightiest.\textsc{acc.pl} be.\textsc{ptcp.prs.m.acc.pl} gain.\textsc{3sg.aor.opt.mid} friend.\textsc{acc.pl}\\
\trans `He took care to enquire about those whom he might win as friends, being the most powerful of the Greeks.' (Herodotus 1.56.1)
\label{wordgroup9}
\end{exe}

\begin{exe}
\ex ἐπειρώτεον, τίνα ἂν θεῶν ἱλαϲάμενοι κατύπερθε τῷ πολέμῳ Τεγεητέων γενοίατο\\
\gll epeirṓteon, tína \emph{àn} theôn hilasámenoi katúperthe tôi polémōi Tegeētéōn genoíato\\
enquire.\textsc{3pl.imp} whom.\textsc{m.acc.sg} \textsc{irr} god.\textsc{gen.pl} appease.\textsc{ptcp.aor.mid.m.nom.pl} above the.\textsc{m.dat.sg} war.\textsc{dat.sg} Tegean.\textsc{gen.pl} become.\textsc{3pl.aor.opt.mid}\\
\trans `They asked which god to appease so as to overcome the Tegeans in war.' (Herodotus 1.67.2)\footnote{\emph{Translator's note}: The Perseus edition has \textit{epeirṓtōn} for \textit{epeirṓteon}.}
\label{wordgroup10}
\end{exe}

\begin{exe}
\ex τὸ δὲ ἂν χρυϲίον ἐγίνετο ἀπὸ τῶν εὐειδέων παρθένων\\
\gll tò dè \emph{àn} khrusíon egíneto apò tôn eueidéōn parthénōn\\
the.\textsc{n.nom.sg} but \textsc{irr} money.\textsc{nom.sg} become.\textsc{3sg.aor.opt.mid} of the.\textsc{f.gen.pl} well-formed.\textsc{f.gen.pl} maiden.\textsc{gen.pl}\\
\trans `And the money would come from the attractive girls' (Herodotus 1.196.3)
\label{wordgroup11}
\end{exe}

\begin{exe}
\ex ϲτρατοῦ ἂν ἄλλου τιϲ τὴν ταχίϲτην ἄγερϲιν ποιέοιτο\\
\gll stratoû \emph{àn} állou tis tḕn takhístēn ágersin poiéoito\\
army.\textsc{gen.sg} \textsc{irr} other.\textsc{m.gen.sg} someone.\textsc{m.nom.sg} the.\textsc{f.acc.sg} fastest.\textsc{f.acc.sg} muster.\textsc{acc.sg} do.\textsc{3sg.prs.opt.pass}\\
\trans `Someone should muster another army as soon as possible' (Herodotus 7.48.1)
\label{wordgroup12}
\end{exe}


\newpage

\begin{exe}
\ex ἕκαϲτοϲ ἂν ὑμῶν ἄρχοι γῆϲ Ἑλλάδοϲ\\
\gll hékastos \emph{àn} humôn árkhoi gês Helládos\\
each.\textsc{m.nom.sg} \textsc{irr} you.\textsc{gen.pl} rule.\textsc{3sg.prs.opt} land.\textsc{gen.sg} Greece.\textsc{gen.sg}\\
\trans `Each of you might rule the land of Greece.' (Herodotus 7.135.2)
\label{wordgroup13}
\end{exe}

\begin{exe}
\ex κατά γε ἂν τὴν ἤπειρον τοιάδε ἐγίνετο\\
\gll katá ge \emph{àn} tḕn ḗpeiron toiáde egíneto\\
down even \textsc{irr} the.\textsc{f.acc.sg} mainland.\textsc{acc.sg} so.much.\textsc{f.nom.sg} become.\textsc{3sg.aor.opt.mid}\\
\trans `On land something like this would have happened' (Herodotus 7.139.2)
\label{wordgroup14}
\end{exe}

\begin{exe}
\ex ἐν ἄλλοιϲιν ἂν λόγοιϲιν ϲαφέϲτερον διδαχθείη\\
\gll en álloisin \emph{àn} lógoisin saphésteron didakhtheíē\\
in other.\textsc{m.dat.pl} \textsc{irr} account.\textsc{dat.pl} clearly.\textsc{comp} teach.\textsc{3sg.aor.opt.pass}\\
\trans `It could be taught more clearly in other words' (Hippocrates, \textit{De arte}; \citealp[44, line 8]{Gomperz1890})
\label{wordgroup15}
\end{exe}

\begin{exe}
\ex ἐπεὶ τῶν γε μὴ ἐόντων τίνα ἄν τιϲ οὐϲίην θεηϲάμενοϲ ἀπαγγείλειεν ὡϲ ἔϲτιν\\
\gll epeì tôn ge mḕ eóntōn tína \emph{án} tis ousíēn theēsámenos apangeíleien hōs éstin\\
since the.\textsc{gen.pl} even not be.\textsc{ptcp.prs.gen.pl} some.\textsc{f.acc.sg} \textsc{irr} someone.\textsc{m.nom.sg} being.\textsc{acc.sg} behold.\textsc{ptcp.aor.mid.m.nom.sg} report.\textsc{3sg.aor.opt} as be.\textsc{3sg.prs}\\
\trans `... since someone observing some essence of those that are not would report that it is so.' (Hippocrates, \textit{De arte}; \citealp[42, line 19]{Gomperz1890})
\label{wordgroup16}
\end{exe}

\begin{exe}
\ex πολλὴν ἂν οἶμαι ἀπιϲτίαν τῆϲ δυνάμεωϲ ... τοῖϲ ἔπειτα πρὸϲ τὸ κλέοϲ αὐτῶν εἶναι\\
\gll pollḕn \emph{àn} oîmai apistían tês dunámeōs toîs épeita pròs tò kléos autôn eînai\\
much.\textsc{f.acc.sg} \textsc{irr} think.\textsc{1sg.prs.pass} distrust.\textsc{acc.sg} the.\textsc{f.gen.sg} power.\textsc{gen.sg} the.\textsc{m.dat.pl} then to the.\textsc{n.acc.sg} fame.\textsc{acc.sg} them.\textsc{gen} be.\textsc{prs.inf}\\
\trans `I think that there would be much distrust among the people then of their power in regard to their fame' (Thucydides 1.10.2)
\label{wordgroup17}
\end{exe}

\begin{exe}
\ex βραχυτάτῳ δ᾽ ἂν κεφαλαίῳ ... τῷδ᾽ ἂν μὴ προέϲθαι ἡμᾶϲ μάθοιτε\\
\gll brakhutátōi d' \emph{àn} kephalaíōi tôid' àn mḕ proésthai hēmâs máthoite\\
shortest.\textsc{n.dat.sg} then \textsc{irr} heading.\textsc{n.dat.sg} this.\textsc{n.dat.sg} \textsc{irr} not abandon.\textsc{aor.inf.mid} us.\textsc{acc} learn.\textsc{2pl.aor.opt}\\
\trans `In summary, you should learn from this not to abandon us' (Thucydides 1.36.3)
\label{wordgroup18}
\end{exe}

\begin{exe}
\ex πρὸϲ γὰρ ἂν τοὺϲ Ἀθηναίουϲ, εἰ ἐξῆν χωρεῖν\\
\gll pròs gàr \emph{àn} toùs Athēnaíous, ei exên khōreîn\\
to then \textsc{irr} the.\textsc{m.acc.pl} Athenian.\textsc{acc.pl} if be.possible.\textsc{3sg.imp} withdraw.\textsc{prs.inf}\\
\trans `For if it were possible, the alliance of Athens would be shut against them.' (Thucydides 5.22.2)
\label{wordgroup19}
\end{exe}

\begin{exe}
\ex τίν᾽ οὖν ἂν ἄγγελον πέμψαιμ᾽ ἐπ᾽ αὐτόν\\
\gll tín' oûn \emph{àn} ángelon pémpsaim' ep' autón\\
what.\textsc{m.acc.sg} so \textsc{irr} messenger.\textsc{acc.sg} send.\textsc{1sg.aor.opt} upon him.\textsc{acc.sg}\\
\trans `What messenger could I send to him?' (Aristophanes, \textit{Thesmophoriazusae} 768)
\label{wordgroup20}
\end{exe}

\begin{exe}
\ex ϲκεπτέον, τί ἂν ἀγαθὸν αὐτὰϲ ἐργαϲάμενοϲ φανείηϲ ἄξια ... πεποιηκώϲ\\
\gll skeptéon, tí \emph{àn} agathòn autàs ergasámenos phaneíēs áxia pepoiēkṓs\\
look.\textsc{gdv.n.nom.sg} what.\textsc{n.acc.sg} \textsc{irr} good.\textsc{n.acc.sg} them.\textsc{f.acc.pl} work.\textsc{ptcp.aor.mid.m.nom.sg} show.\textsc{2sg.aor.opt.pass} worthy.\textsc{n.acc.pl} do.\textsc{ptcp.prf.m.nom.sg}\\
\trans `It must be considered by doing them what good you can be seen to have done worthy things ...' (Isocrates 5.35)
\label{wordgroup21}
\end{exe}

\begin{exe}
\ex πολλὴ γὰρ ἄν τιϲ εὐδαιμονία εἴη περὶ τοὺϲ νέουϲ\\
\gll pollḕ gàr \emph{án} tis eudaimonía eíē perì toùs néous\\
much.\textsc{f.nom.sg} then \textsc{irr} someone.\textsc{m.nom.sg} prosperity.\textsc{nom.sg} be.\textsc{3sg.prs.opt} about the.\textsc{m.acc.pl} young.\textsc{m.acc.pl}\\
\trans `For it would be a great blessing for the young ...' (Plato, \textit{Apology} 25b)
\label{wordgroup22}
\end{exe}

\begin{exe}
\ex πολλὴ ἂν ἐλπὶϲ εἴη καὶ καλὴ\\
\gll pollḕ \emph{àn} elpìs eíē kaì kalḕ\\
much.\textsc{f.nom.sg} \textsc{irr} hope.\textsc{nom.sg} be.\textsc{3sg.prs.opt} and beautiful.\textsc{f.nom.sg}\\
\trans `... it would be a great and beautiful hope ...' (Plato, \textit{Phaedo} 70a)
\label{wordgroup23}
\end{exe}

\begin{exe}
\ex ἄλλου ἄν του δέοι λόγου\\
\gll állou \emph{án} tou déoi lógou\\
other.\textsc{m.gen.sg} \textsc{irr} the.\textsc{m.gen.sg} lack.\textsc{3sg.prs.opt} word.\textsc{gen.sg}\\
\trans `... further argument would be needed' (Plato, \textit{Phaedo} 70d and 106d)
\label{wordgroup24}
\end{exe}

\begin{exe}
\ex οὐδεμία ἂν εἴη ἄλλη ἀποφυγή\\
\gll oudemía \emph{àn} eíē állē apophugḗ\\
none.\textsc{f.nom.sg} \textsc{irr} be.\textsc{3sg.prs.opt} other.\textsc{f.nom.sg} escape.\textsc{nom.sg}\\
\trans `... there would be no other escape ...' (Plato, \textit{Phaedo} 107c)
\label{wordgroup25}
\end{exe}

\begin{exe}
\ex ἐλθὼν δ᾽ ὁ Ξενοφῶν ἐπήρετο τὸν Ἀπόλλω, τίνι ἂν θεῶν θύων καὶ εὐχόμενοϲ κάλλιϲτα καὶ ἄριϲτα ἔλθοι τὴν ὁδόν, ἣν ἐπινοεῖ, καὶ καλῶϲ πράξαϲ ϲωθείη\\
\gll elthṑn d' ho Xenophôn epḗreto tòn Apóllō, tíni \emph{àn} theôn thúōn kaì eukhómenos kállista kaì árista élthoi tḕn hodón, hḕn epinoeî, kaì kalôs práxas sōtheíē\\
go.\textsc{ptcp.aor.m.nom.sg} then the.\textsc{m.nom.sg} Xenophon.\textsc{nom} enquire.\textsc{3sg.aor.mid} the.\textsc{m.acc.sg} Apollo.\textsc{acc} what.\textsc{m.dat.sg} \textsc{irr} god.\textsc{gen.pl} sacrifice.\textsc{ptcp.m.nom.sg} and pray.\textsc{ptcp.prs.pass.m.nom.sg} well.\textsc{supl} and best go.\textsc{3sg.aor.opt} the.\textsc{f.acc.sg} way.\textsc{acc.sg} which.\textsc{f.acc.sg} intend.\textsc{3sg.prs} and well do.\textsc{ptcp.aor.m.nom.sg} save.\textsc{3sg.aor.opt.pass}\\
\trans `So Xenophon went and asked Apollo to which of the gods he should sacrifice and pray in order best and most successfully to perform the journey which he had in mind and, after meeting with good fortune, to return home in safety' (Xenophon, \textit{Anabasis} 3.1.6; reminiscent of the \textit{tíni ka theōn} of example (\ref{dodonian}) above)
\label{wordgroup26}
\end{exe}

\begin{exe}
\ex ἐπερωτᾷ ὁ δῆμοϲ ... , ὅ τι ἂν δρῶϲιν ... εἵη\\
\gll eperōtâi ho dêmos, hó~ti \emph{àn} drôsin eíē\\
enquire.\textsc{3sg.prs} the.\textsc{m.nom.sg} people.\textsc{nom.sg} which.\textsc{n.acc.sg} \textsc{irr} do.\textsc{3pl.prs} be.\textsc{3sg.prs.opt}\\
\trans `The people enquire ... what they should do ... may be ...' ({[}Demosthenes{]} 43.66; cf. also example (\ref{wordgroup10}) above)
\label{wordgroup27}
\end{exe}

\begin{exe}
\ex λαβόντεϲ δὲ τοὺϲ ἄρχονταϲ, ἀναρχίᾳ ἂν καὶ ἀταξίᾳ ἐνόμιζον ἡμᾶϲ ἀπολέϲθαι\\
\gll labóntes dè toùs árkhontas, anarkhíāi \emph{àn} kaì ataxíāi enómizon hēmâs apolésthai\\
take.\textsc{ptcp.aor.m.nom.pl} but the.\textsc{m.acc.pl} ruler.\textsc{acc.pl} anarchy.\textsc{dat.sg} \textsc{irr} and disorder.\textsc{dat.sg} consider.\textsc{3pl.imp} us.\textsc{acc} destroy.\textsc{aor.inf.mid}\\
\trans `Having taken our commanders, they considered that we would be ruined through want of leadership and of discipline.' (Xenophon, \textit{Anabasis} 3.2.29)
\label{wordgroup28}
\end{exe}

\begin{exe}
\ex πολλὴ ἂν καὶ ἀπὸ τούτων πρόϲοδοϲ γίγνοιτο\\
\gll pollḕ \emph{àn} kaì apò toútōn prósodos gígnoito\\
much.\textsc{f.nom.sg} \textsc{irr} also of this.\textsc{f.gen.pl} revenue.\textsc{nom.sg} become.\textsc{3sg.prs.opt.pass}\\
\trans `A great revenue would also come from these.' (Xenophon, \textit{Ways} 3.14)
\label{wordgroup29}
\end{exe}

\begin{exe}
\ex πάμπολλα ἂν νομίζω χρήματα ... προϲιέναι\\
\gll pámpolla \emph{àn} nomízō khrḗmata prosiénai\\
very.much.\textsc{n.acc.pl} \textsc{irr} consider.\textsc{1sg.prs} property.\textsc{acc.pl} be.added.\textsc{prs.inf}\\
\trans `I consider that a great sum of money would be added ...' (Xenophon, \textit{Ways} 4.1)
\label{wordgroup30}
\end{exe}

\begin{exe}
\ex ἀντὶ πολλῶν ἄν, ὦ ἄνδρεϲ Ἀθηναῖοι, χρημάτων ὑμᾶϲ ἑλέϲθαι νομίζω\\
\gll antì pollôn \emph{án}, ô ándres Athēnaîoi, khrēmátōn humâs helésthai nomízō\\
against much.\textsc{n.gen.pl} \textsc{irr} O men.\textsc{voc.pl} Athenian.\textsc{m.voc.pl} property.\textsc{gen.pl} you.\textsc{acc.pl} take.\textsc{aor.inf.mid} consider.\textsc{1sg.prs}\\
\trans `You would, I expect, men of Athens, accept it as the equivalent of a large amount of money ...' (Demosthenes 1.1)
\label{wordgroup31}
\end{exe}

\begin{exe}
\ex πληϲίον μὲν ὄντεϲ, ἅπαϲιν ἂν τοῖϲ πράγμαϲιν τεταραγμένοιϲ ἐπιϲτάντεϲ ὅπωϲ βούλεϲθε διοικήϲαιϲθε\\
\gll plēsíon mèn óntes, hápasin \emph{àn} toîs prágmasin tetaragménois epistántes hópōs boúlesthe dioikḗsaisthe\\
near then be.\textsc{ptcp.prs.m.nom.pl} quite.all.\textsc{n.dat.pl} \textsc{irr} the.\textsc{n.dat.pl} deed.\textsc{dat.pl} disturb.\textsc{ptcp.prf.pass.n.dat.pl} establish.\textsc{ptcp.aor.m.nom.pl} so wish.\textsc{2pl.prs.pass} administer.\textsc{2pl.aor.opt.mid}\\
\trans `... being at hand, you could manage things as you wish by attending to the disturbances in everything' (Demosthenes 4.12)
\label{wordgroup32}
\end{exe}

\begin{exe}
\ex τί ἂν ποιῶν ὑμῖν χαρίϲαιτο\\
\gll tí \emph{àn} poiôn humîn kharísaito\\
what.\textsc{n.acc.sg} \textsc{irr} do.\textsc{ptcp.prs.m.nom.sg} you.\textsc{dat.pl} gratify.\textsc{3sg.aor.opt.mid}\\
\trans `... what he might oblige you by doing' (Demosthenes 19.48)
\label{wordgroup33}
\end{exe}

\begin{exe}
\ex τί ἂν εἰπών ϲέ τιϲ ὀρθῶϲ προϲείποι\\
\gll tí \emph{àn} eipṓn sé tis orthôs proseípoi\\
what.\textsc{n.acc.sg} \textsc{irr} say.\textsc{ptcp.aor.m.nom.sg} you.\textsc{acc} someone.\textsc{m.nom.sg} straight address.\textsc{3sg.aor.opt}\\
\trans `By saying what could someone call you correctly?' (Demosthenes 18.22)
\label{wordgroup34}
\end{exe}

\begin{exe}
\ex ὅτι πολλὰ μὲν ἂν χρήματα ἔδωκε Φιλιϲτίδηϲ\\
\gll hóti pollà mèn \emph{àn} khrḗmata édōke Philistídēs\\
that much.\textsc{n.acc.pl} then \textsc{irr} property.\textsc{acc.pl} give.\textsc{3sg.aor} Philistides.\textsc{nom}\\
\trans `... that Philistides would have paid a great sum of money ...' (Demosthenes 18.81)
\label{wordgroup35}
\end{exe}

\begin{exe}
\ex μείζων ἂν δοθείη δωρειά\\
\gll meízōn \emph{àn} dotheíē dōreiá\\
greater.\textsc{f.nom.sg} \textsc{irr} give.\textsc{3sg.aor.opt.pass} gift.\textsc{nom.sg}\\
\trans `... a greater gift would be given ...' (Demosthenes 18.293)
\label{wordgroup36}
\end{exe}

\begin{exe}
\ex θαυμαϲίωϲ ἂν ὡϲ εὐλαβούμην\\
\gll thaumasíōs \emph{àn} hōs eulaboúmēn\\
wonderfully \textsc{irr} as beware.\textsc{1sg.imp.pass}\\
\trans `... I should be wonderfully cautious ...' (Demosthenes 29.1)\footnote{\emph{Translator's note}: The Perseus edition has \textit{ēulaboúmēn} for \textit{eulaboúmēn}.}
\label{wordgroup37}
\end{exe}

\hyperlink{p399}{\emph{[p399]}}

\begin{exe}
\ex καίτοι, τίϲ ἂν ὑμῶν οἴεται τὴν μητέρα πέμψαι ... ;\\
\gll kaítoi, tís \emph{àn} humôn oíetai tḕn mētéra pémpsai\\
and.yet who.\textsc{m.nom.sg} \textsc{irr} you.\textsc{gen.pl} think.\textsc{3sg.prs.pass} the.\textsc{n.acc.sg} mother.\textsc{acc.sg} send.\textsc{aor.inf}\\
\trans `And yet, who among you thinks that his mother would have sent ... ?' (Demosthenes 39.24)
\label{wordgroup38}
\end{exe}

\begin{exe}
\ex τί ἂν εἰπὼν μήθ᾽ ἁμαρτεῖν δοκοίην μήτε ψευϲαίμην\\
\gll tí \emph{àn} eipṑn mḗth' hamarteîn dokoíēn mḗte pseusaímēn\\
what.\textsc{n.acc.sg} \textsc{irr} say.\textsc{ptcp.aor.m.nom.sg} nor miss.\textsc{aor.inf} seem.\textsc{1sg.prs.opt} nor lie.\textsc{1sg.aor.opt.mid}\\
\trans `By saying what could I neither seem to understate nor exaggerate?' (Demosthenes, \textit{Letters} 3.37)
\label{wordgroup39}
\end{exe}

\begin{exe}
\ex τί ἄν τιϲ ἄλλο ὄνομ᾽ ἔχοι θέϲθαι τῷ τοιούτῳ\\
\gll tí \emph{án} tis állo ónom' ékhoi thésthai tôi toioútōi\\
what.\textsc{n.acc.sg} \textsc{irr} someone.\textsc{m.nom.sg} other.\textsc{n.acc.sg} name.\textsc{acc.sg} have.\textsc{3sg.prs.opt} put.\textsc{aor.inf.mid} the.\textsc{m.dat.sg} such.\textsc{m.dat.sg}\\
\trans `What other name could one give to such a person ... ?' ({[}Demosthenes{]} 35.36)\footnote{\emph{Translator's note}: The Perseus edition has \textit{toîs toioútois} for \textit{tôi toioútōi}.}
\label{wordgroup40}
\end{exe}

In addition, there are numerous examples of the type in (\ref{wordgroup41}).

\begin{exe}
\ex οὐκ ἂν οἴεϲθε δημοϲίᾳ πάνταϲ ὑμᾶϲ προξένουϲ αὑτῶν ποιήϲαϲθαι\\
\gll ouk \emph{àn} oíesthe dēmosíāi pántas humâs proxénous hautôn poiḗsasthai\\
not \textsc{irr} think.\textsc{2pl.prs.pass} publicly all.\textsc{m.acc.pl} you.\textsc{acc.pl} patron.\textsc{acc.pl} themselves.\textsc{gen} make.\textsc{aor.inf.mid}\\
\trans `Do you not think that they would unanimously appoint you their protectors?' (Demosthenes 21.50)
\label{wordgroup41}
\end{exe}

Among these examples, whose number could moreover easily be doubled, there are several in which the later half of the clause contains a second\is{doubling|(} \emph{án} resuming the first \emph{án}, as in the preceding categories. Here is a particularly instructive case: for example (\ref{wordgroup31}) from Demosthenes there is a parallel version in \textit{Exordia} in which the second part of the clause is heavily expanded, with the text in example (\ref{wordgroup42}) instead of \textit{khrēmátōn humâs helésthai nomízō}, and here, because of the expanded version of the clause, \emph{án} is repeated after \emph{pántas} `all'. (Blass's \citeyearpar[360]{DindorfBlass1892} deletion of the first \emph{án} after \emph{pollōn}, against the better transmitted version, is wholly erroneous.)

\begin{exe}
\ex χρημάτων τὸ μέλλον ϲυνοίϲεν περὶ ὧν νῦν τυγχάνετε ϲκοποῦντεϲ οἶμαι πάνταϲ ὑμᾶϲ ἑλέϲθαι\\
\gll khrēmátōn tò méllon sunoísen perì hôn nûn tunkhánete skopoûntes oîmai pántas humâs helésthai\\
property.\textsc{gen.pl} the.\textsc{n.acc.sg} be.going.to.\textsc{ptcp.prs.n.acc.sg} profit.\textsc{fut.inf} about which.\textsc{n.gen.pl} now happen.\textsc{2pl.prs} consider.\textsc{ptcp.prs.m.nom.pl} think.\textsc{1pl.prs.pass} all.\textsc{m.acc.pl} you.\textsc{acc.pl} take.\textsc{aor.inf.mid}\\
\trans `(Instead of) money, I think that you would choose what will benefit in those things about which you now happen to be deliberating' (Demosthenes, \textit{Exordia} 3.1)\footnote{\emph{Translator's note}: The Perseus edition adds \textit{humîn} after \textit{sunoísen}.}
\label{wordgroup42}
\end{exe}

I believe we are able to say that, in all cases where \emph{án} is inserted more than once, this is a compromise between the traditional pressure to place \emph{án} near the beginning of the clause and the requirement -- emerging in the classical language -- to place \emph{án} nearer the verb and other constituents (see above p\pageref{posthomerican}). This also explains why doubled \emph{án} is not found in \isi{subjunctive} clauses. Thus, all clauses with multiple instances of \emph{án} in which the first \emph{án} occupies the second position are of relevance for us, and not only those that have already been adduced. The examples that I have to hand are (\ref{multiplean2})--(\ref{multiplean3}) above and (\ref{multian1})--(\ref{multian74}), excluding of course \emph{oút' án ... oút' án} `neither \textsc{irr} ... nor \textsc{irr}', which does not belong here.

\begin{exe}
\ex οὔ τἂν ἑλόντεϲ αὖθιϲ ἀνθαλοῖεν ἄν\\
\gll oú t\emph{àn} helóntes aûthis anthaloîen \emph{án}\\
not and=\textsc{irr} take.\textsc{ptcp.aor.m.nom.pl} again be.captured.\textsc{3pl.aor.opt} \textsc{irr}\\
\trans `... the captors shall not be made captives in their turn' (Aeschylus, \textit{Agamemnon} 340)
\label{multian1}
\end{exe}

\begin{exe}
\ex ἐντὸϲ δ᾽ ἂν οὖϲα μορϲίμων ἀγρευμάτων πείθοι᾽ ἄν\\
\gll entòs d' \emph{àn} oûsa morsímōn agreumátōn peíthoi' \emph{án}\\
inside then \textsc{irr} be.\textsc{ptcp.prs.f.nom.sg} destined.\textsc{n.gen.pl} snare.\textsc{gen.pl} persuade.\textsc{2sg.prs.opt.pass} \textsc{irr}\\
\trans `Since you are in the toils of destiny, perhaps you will obey' (Aeschylus, \textit{Agamemnon} 1048)
\label{multian2}
\end{exe}

\begin{exe}
\ex λιπὼν ἂν εὔκλειαν ἐν δόμοιϲιν ... πολύχωϲτον ἂν εἶχεϲ τάφον\\
\gll lipṑn \emph{àn} eúkleian en dómoisin polúkhōston \emph{àn} eîkhes táphon\\
leave.\textsc{ptcp.aor.m.nom.sg} \textsc{irr} renown.\textsc{acc.sg} in house.\textsc{dat.pl} high-heaped.\textsc{m.acc.sg} \textsc{irr} have.\textsc{2sg.imp} tomb.\textsc{acc.sg}\\
\trans `Having left a good name in your household, you would have found a high-heaped tomb ...' (Aeschylus, \textit{Libation Bearers} 349)
\label{multian3}
\end{exe}

\begin{exe}
\ex πῶϲ δ᾽ ἂν γαμῶν ἄκουϲαν ἄκοντοϲ πάρα ἁγνὸϲ γένοιτ᾽ ἄν\\
\gll pôs d' \emph{àn} gamôn ákousan àkontos pára hagnòs génoit' \emph{án}\\
how then \textsc{irr} marry.\textsc{ptcp.prs.m.nom.sg} unwilling.\textsc{f.acc.sg} unwilling.\textsc{m.gen.sg} from holy.\textsc{m.nom.sg} become.\textsc{3sg.aor.opt.mid} \textsc{irr}\\
\trans `And how can man be pure who would seize from an unwilling father an unwilling bride?' (Aeschylus, \textit{Suppliants} 227)
\label{multian4}
\end{exe}

\begin{exe}
\ex τί δῆτ᾽ ἂν ὡϲ ἐκ τῶνδ᾽ ἂν ὠφελοῖμί ϲε\\
\gll tí dêt' \emph{àn} hōs ek tônd' \emph{àn} ōpheloîmí se\\
what.\textsc{n.acc.sg} then \textsc{irr} as out this.\textsc{n.gen.pl} \textsc{irr} help.\textsc{1sg.prs.opt} you.\textsc{acc}\\
\trans `How, then, can I serve you, as things stand now?' (Sophocles, \textit{Ajax} 537)
\label{multian5}
\end{exe}

\begin{exe}
\ex ἡμεῖϲ μὲν ἂν τήνδ᾽ ἣν ὅδ᾽ εἴληχεν τύχην θανόντεϲ ἂν προὐκείμεθ᾽ αἰϲχίϲτῳ μόρῳ\\
\gll hēmeîs mèn \emph{àn} tḗnd' hḕn hód' eílēkhen túkhēn thanóntes \emph{àn} proukeímeth' aiskhístōi mórōi\\
we.\textsc{nom} then \textsc{irr} this.\textsc{f.acc.sg} which.\textsc{f.acc.sg} this.\textsc{m.nom.sg} obtain.\textsc{3sg.prf} fortune.\textsc{acc.sg} die.\textsc{ptcp.aor.m.nom.pl} \textsc{irr} forth=lie.\textsc{1pl.imp.pass} shameful.\textsc{supl.m.dat.sg} doom.\textsc{dat.sg}\\
\trans `We would have been allotted the fate which he now has, and we would be dead and lie prostrate by an ignoble doom' (Sophocles, \textit{Ajax} 1058)
\label{multian6}
\end{exe}

\begin{exe}
\ex ἀλλ᾽ ἄνδρα χρὴ ... δοκεῖν, πεϲεῖν ἂν κἂν ἀπὸ ϲμικροῦ κακοῦ\\
\gll all' ándra khrḕ dokeîn, peseîn \emph{àn} k\emph{àn} apò smikroû kakoû\\
but man.\textsc{m.acc.sg} need.\textsc{3sg.prs} think.\textsc{prs.inf} fall.\textsc{aor.inf} \textsc{irr} also=\textsc{irr} of small.\textsc{n.gen.sg} ill.\textsc{gen.sg}\\
\trans `It is necessary for a man to think that he shall fall, even from a slight harm.' (Sophocles, \textit{Ajax} 1078)
\label{multian7}
\end{exe}

\begin{exe}
\ex τάχ᾽ ἂν κἄμ᾽ ἂν τοιαύτῃ χειρὶ τιμωρεῖν θέλοι\\
\gll tákh' \emph{àn} kám' \emph{àn} toiaútēi kheirì timōreîn théloi\\
quickly \textsc{irr} also=me.\textsc{acc} \textsc{irr} such.\textsc{f.dat.sg} hand.\textsc{dat} avenge.\textsc{prs.inf} want.\textsc{3sg.prs.opt}\\
\trans `He might perhaps wish to take vengeance on me with such a hand.' (Sophocles, \textit{Oedipus Rex} 139)\footnote{\emph{Translator's note}: The Perseus edition has \textit{timōroûnth' héloi} for \textit{timōreîn théloi}.}
\label{multian8}
\end{exe}

\hyperlink{p400}{\emph{[p400]}}

\begin{exe}
\ex ϲυθείϲ τ᾽ ἂν οὐκ ἂν ἀλγύνοιϲ πλέον\\
\gll sutheís t' \emph{àn} ouk \emph{àn} algúnois pléon\\
drive.\textsc{ptcp.aor.pass.m.nom.sg} and \textsc{irr} not \textsc{irr} pain.\textsc{2sg.prs.opt.act} more\\
\trans `When you have gone, you will vex me no more.' (Sophocles, \textit{Oedipus Rex} 446)
\label{multian9}
\end{exe}

\begin{exe}
\ex οὔτ᾽ ἂν μετ᾽ ἄλλου δρῶντοϲ ἂν τλαίην ποτέ\\
\gll oút' \emph{àn} met' állou drôntos \emph{àn} tlaíēn poté\\
nor \textsc{irr} after other.\textsc{m.gen.sg} do.\textsc{ptcp.prs.m.nom.sg} \textsc{irr} endure.\textsc{1sg.aor.opt} sometime\\
\trans `Nor could I ever endure it after another's doing so.' (Sophocles, \textit{Oedipus Rex} 602)
\label{multian10}
\end{exe}

\begin{exe}
\ex ἧδ᾽ ἂν τάδ᾽ οὐχ᾽ ἥκιϲτ᾽ ἂν Ἰοκάϲτη λέγοι\\
\gll hêd' \emph{àn} tád' oukh' hḗkist' \emph{àn} Iokástē légoi\\
this.\textsc{f.nom.sg} \textsc{irr} this.\textsc{n.acc.pl} not least \textsc{irr} Jocasta say.\textsc{3sg.prs.opt}\\
\trans `Not least could this Jocasta say these things.' (Sophocles, \textit{Oedipus Rex} 1053)
\label{multian11}
\end{exe}

\begin{exe}
\ex τίϲ οὖν ἂν ἀξίαν γε ϲοῦ πεφηνότοϲ μεταβάλοιτ᾽ ἂν ὧδε ϲιγὰν λόγων\\
\gll tís oûn \emph{àn} axían ge soû pephēnótos metabáloit' \emph{àn} hôde sigàn lógōn\\
who.\textsc{m.nom.sg} so \textsc{irr} worthy.\textsc{f.acc.sg} even you.\textsc{gen} show.\textsc{ptcp.prf.m.nom.sg} exchange.\textsc{3sg.aor.opt.mid} \textsc{irr} thus silence.\textsc{acc.sg} account.\textsc{gen.pl}\\
\trans `You having appeared, who then would thus change fitting silence for words?' (Sophocles, \textit{Electra} 1260)
\label{multian12}
\end{exe}

\begin{exe}
\ex ποίαϲ ἂν ὑμᾶϲ πατρίδοϲ (or πόλεοϲ) ἢ γένουϲ ποτὲ τύχοιμ᾽ ἂν εἰπών\\
\gll poías \emph{àn} humâs patrídos/póleos ḕ génous potè túkhoim' \emph{àn} eipṓn\\
of.what.sort.\textsc{f.gen.sg} \textsc{irr} you.\textsc{acc.pl} fatherland.\textsc{gen}/city.\textsc{gen.sg} or kind.\textsc{gen.sg} sometime happen.\textsc{1sg.aor.opt} \textsc{irr} say.\textsc{ptcp.aor.m.nom.sg}\\
\trans `Of what country or family might I ever happen to say that you are?' (Sophocles, \textit{Philoctetes} 222)
\label{multian13}
\end{exe}

Example (\ref{multian13}) is what is read by \citet[304]{Dindorf1882} and \citet[18--19]{Heimreich1884} in place of the manuscript's \textit{poías pátras àn humâs ḕ génous poté}, in which the metrical error caused by the placement of \emph{humâs} is remedied less successfully by others.

\begin{exe}
\ex τίϲ δ᾽ ἂν τοιοῦδ᾽ ὑπ᾽ ἀνδρὸϲ εὖ πράξειεν ἄν\\
\gll tís d' \emph{àn} toioûd' hup' andròs eû práxeien \emph{án}\\
who.\textsc{m.nom.sg} then \textsc{irr} such.\textsc{m.gen.sg} under man.\textsc{gen.sg} well do.\textsc{3sg.aor.opt} \textsc{irr}\\
\trans `And who could profit from such a man?' (Sophocles, \textit{Oedipus at Colonus} 391)
\label{multian14}
\end{exe}

\begin{exe}
\ex ἆρ᾽ ἂν ματαίου τῆϲδ᾽ ἂν ἡδονῆϲ τύχοιϲ\\
\gll âr' \emph{àn} mataíou tesd' \emph{àn} hēdonês túkhois\\
then \textsc{irr} vain.\textsc{f.gen.sg} this.\textsc{f.gen.sg} \textsc{irr} pleasure.\textsc{gen.sg} happen.\textsc{2sg.aor.opt}\\
\trans `Would you then find this pleasure vain?' (Sophocles, \textit{Oedipus at Colonus} 780)
\label{multian15}
\end{exe}

\begin{exe}
\ex πῶϲ ἂν τό γ᾽ ἆκον πρᾶγμ᾽ ἂν εἰκότωϲ ψέγοιϲ\\
\gll pôs \emph{àn} tó g' âkon prâgm' \emph{àn} eikótōs pségois\\
how \textsc{irr} the.\textsc{n.acc.sg} even unwilling.\textsc{n.acc.sg} deed.\textsc{acc.sg} \textsc{irr} justly blame.\textsc{2sg.prs.opt.act}\\
\trans `How could you reasonably blame the unwitting deed?' (Sophocles, \textit{Oedipus at Colonus} 976)
\label{multian16}
\end{exe}

\begin{exe}
\ex ἦ τἂν οὐκ ἂν ἦ\\
\gll ê t\emph{àn} ouk \emph{àn} ê\\
in.truth and=\textsc{irr} not \textsc{irr} be.\textsc{1sg.imp}\\
\trans `In truth I would be no more' (Sophocles, \textit{Oedipus at Colonus} 1366)
\label{multian17}
\end{exe}

\begin{exe}
\ex οὐ γάρ ποτ᾽ ἂν γένοιτ᾽ ἂν ἀϲφαλὴϲ πόλιϲ\\
\gll ou gár pot' \emph{àn} génoit' \emph{àn} asphalḕs pólis\\
not for sometime \textsc{irr} become.\textsc{3sg.aor.opt.mid} \textsc{irr} safe.\textsc{f.nom.sg} city.\textsc{nom.sg}\\
\trans `For the city would never prove secure' (Sophocles, \textit{Phaedra} 622.1)
\label{multian18}
\end{exe}

Example (\ref{multian19}) has three \textit{án}s!

\begin{exe}
\ex πῶϲ ἂν οὐκ ἂν ἐν δίκῃ θάνοιμ᾽ ἄν\\
\gll pôs \emph{àn} ouk \emph{àn} en díkēi thánoim' {án}\\
how \textsc{irr} not \textsc{irr} in judgement.\textsc{dat.sg} die.\textsc{1sg.aor.opt} \textsc{irr}\\
\trans `How in justice could I not die?' (Sophocles, Fragment 673)
\label{multian19}
\end{exe}

\begin{exe}
\ex ὁ ἥλιοϲ ἂν ἀπελαυνόμενοϲ ἐκ μέϲου τοῦ οὐρανοῦ ... ἤιε ἂν τὰ ἄνω τῆϲ Εὐρώπηϲ\\
\gll ho hḗlios \emph{àn} apelaunómenos ek mésou toû ouranoû ḗie \emph{àn} tà ánō tês Eurṓpēs\\
the.\textsc{m.nom.sg} sun.\textsc{nom.sg} \textsc{irr} expel.\textsc{ptcp.prs.pass.m.nom.sg} out mid.\textsc{m.gen.sg} the.\textsc{m.gen.sg} heaven.\textsc{gen.sg} go.\textsc{3sg.imp} \textsc{irr} the.\textsc{n.acc.pl} upward the.\textsc{f.gen.sg} Europe.\textsc{gen.sg}\\
\trans `The sun, when driven from mid-heaven, would pass over the inland parts of Europe' (Herodotus 2.26.2)
\label{multian20}
\end{exe}

\begin{exe}
\ex διεξιόντα δ᾽ ἄν μιν διὰ πάϲηϲ Εὐρώπηϲ ἔλπομαι ποιέειν ἂν τὸν Ἴϲτρον\\
\gll diexiónta d' \emph{án} min dià pásēs Eurṓpēs élpomai poiéein \emph{àn} tòn Ístron\\
pass.through.\textsc{ptcp.prs.m.acc.sg} then \textsc{irr} \textsc{cl} through all.\textsc{gen.sg} Europe hope.\textsc{1sg.prs.pass} do.\textsc{prs.inf} \textsc{irr} the.\textsc{m.acc.sg} Ister.\textsc{acc}\\
\trans `... and I believe that passing across all Europe, it would do to the Ister ...' (Herodotus 2.26.2)
\label{multian21}
\end{exe}

\begin{exe}
\ex ούδ᾽ ἂν αὐτὸν ἔγωγε δοκέω τὸν θεὸν οὕτω ἂν κακῶϲ βαλεῖν\\
\gll oúd' \emph{àn} autòn égōge dokéō tòn theòn hoútō \emph{àn} kakôs baleîn\\
nor \textsc{irr} same.\textsc{m.acc.sg} I.\textsc{nom.emph} think.\textsc{1sg.prs} the.\textsc{m.acc.sg} god.\textsc{acc.sg} so \textsc{irr} badly throw.\textsc{aor.inf}\\
\trans `I think that not even the god himself could shoot so true.' (Herodotus 3.35.4)
\label{multian22}
\end{exe}

\largerpage[2]
\begin{exe}
\ex οὐδ᾽ ἂν τούτων ὑπὸ πλήθεοϲ οὐδεὶϲ ἂν εἴποι πλῆθοϲ\\
\gll oúd' \emph{àn} toútōn hupò plḗtheos oudeìs \emph{àn} eípoi plêthos\\
nor \textsc{irr} this.\textsc{gen.pl} under quantity.\textsc{gen.sg} nobody.\textsc{m.nom.sg} \textsc{irr} say.\textsc{3sg.aor.opt} quantity.\textsc{acc.sg}\\
\trans `And no one could tell the number, with such numbers of them.' (Herodotus 7.187.1)
\label{multian23}\footnote{\emph{Translator's note}: The Perseus edition has \textit{arithmón} for \textit{plêthos}.}
\end{exe}
\clearpage

\begin{exe}
\ex πόλλ᾽ ἂν ϲὺ λέξαϲ οὐδὲν ἂν πλέον λάβοιϲ\\
\gll póll' \emph{àn} sù léxas oudèn \emph{àn} pléon lábois\\
many.\textsc{n.acc.pl} \textsc{irr} you.\textsc{nom} say.\textsc{ptcp.aor.m.nom.sg} nothing.\textsc{acc.sg} \textsc{irr} more.\textsc{n.acc.sg} take.\textsc{2sg.aor.opt}\\
\trans `Having said much, you will get nothing more.' (Euripides, \textit{Alcestis} 72)
\label{multian24}
\end{exe}

\begin{exe}
\ex πῶϲ ἂν ἔρημον τάφον Ἄδμητοϲ κεδνῆϲ ἂν ἔπραξε γυναικόϲ\\
\gll pôs \emph{àn} érēmon táphon Ádmētos kednês \emph{àn} épraxe gunaikós\\
how \textsc{irr} solitary.\textsc{m.acc.sg} tomb.\textsc{acc.sg} Admetus.\textsc{num} dear.\textsc{f.gen.sg} \textsc{irr} do.\textsc{3sg.aor} woman.\textsc{gen.sg}\\
\trans `How would Admetus have held the funeral of his good wife without mourners?' (Euripides, \textit{Alcestis} 93)
\label{multian25}
\end{exe}

\begin{exe}
\ex οὐκ ἂν ἔν γ᾽ ἐμοῖϲ δόμοιϲ βλέπουϲ᾽ ἂν αὐγὰϲ τἄμ᾽ ἐκαρποῦτ᾽ ἂν λέχη\\
\gll ouk \emph{àn} én g' emoîs dómois blépous' \emph{àn} augàs tám' ekarpoût' \emph{àn} lékhē\\
not \textsc{irr} in even my.\textsc{m.dat.pl} house.\textsc{dat.pl} look.\textsc{ptcp.prs.f.nom.sg} \textsc{irr} daylight.\textsc{acc.pl} the=my.\textsc{n.acc.pl} harvest.\textsc{3sg.imp.pass} \textsc{irr} bed.\textsc{acc.pl}\\
\trans `She would never have reaped the fruits of my bed in my house and seen daylight' (Euripides, \textit{Andromache} 934)
\label{multian26}
\end{exe}

\begin{exe}
\ex ἄλγοϲ ἂν προϲθείμεθ᾽ ἄν\\
\gll álgos \emph{àn} prostheímeth' \emph{án}\\
pain.\textsc{acc.sg} \textsc{irr} add.\textsc{1sg.aor.opt.mid} \textsc{irr}\\
\trans `... I would add to my anguish' (Euripides, \textit{Hecuba} 742)
\label{multian27}
\end{exe}

\begin{exe}
\ex τῷδ᾽ ἂν εὐϲτόχῳ πτερῷ ἀπόλαυϲιν εἰκοῦϲ ἔθανεϲ ἂν Διὸϲ κόρηϲ\\
\gll tôid' \emph{àn} eustókhōi pterôi apólausin eikoûs éthanes \emph{àn} Diòs kórēs\\
this.\textsc{n.dat.sg} \textsc{irr} well.aimed.\textsc{n.dat.sg} feather.\textsc{dat.sg} reward.\textsc{acc.sg} likeness.\textsc{gen.sg} die.\textsc{2sg.aor} \textsc{irr} Zeus.\textsc{gen} girl.\textsc{gen.sg}\\
\trans `You would have died by this well-aimed arrow as a reward for your likeness to the daughter of Zeus.' (Euripides, \textit{Helen} 76)
\label{multian28}
\end{exe}

\begin{exe}
\ex φθάνοιϲ δ᾽ ἂν ούκ ἂν τοῖϲδε ϲὸν κρύπτων δέμαϲ\\
\gll phthánois d' \emph{àn} oúk \emph{àn} toîsde sòn krúptōn démas\\
arrive.\textsc{2sg.prs.opt} then \textsc{irr} not \textsc{irr} this.\textsc{n.dat.pl} your.\textsc{n.acc.sg} hide.\textsc{ptcp.prs.m.nom.sg} body.\textsc{acc.sg}\\
\trans `It would not be premature to put it on.' (Euripides, \textit{Heracleidae} 721; cf. \citealp[119]{Elmsley1821})
\label{multian29}
\end{exe}

\begin{exe}
\ex ἄλλοϲ τε πῶϲ ἂν μὴ διορθεύων λόγουϲ ὀρθῶϲ δύναιτ᾽ ἂν δῆμοϲ εὐθύνειν πόλιν\\
\gll állos te pôs \emph{àn} mḕ diortheúōn lógous orthôs dúnait' \emph{àn} dêmos euthúnein pólin\\
otherwise and how \textsc{irr} not judge.rightly.\textsc{ptcp.prs.m.nom.sg} account.\textsc{acc.pl} straight can.\textsc{3sg.prs.opt.pass} \textsc{irr} people.\textsc{m.nom.sg} direct.\textsc{prs.inf} city.\textsc{acc}\\
\trans `Besides, how would the people, if it cannot form true judgments, be able rightly to direct the state?' (Euripides, \textit{Suppliants} 417)
\label{multian30}
\end{exe}

\begin{exe}
\ex τίν᾽ ἂν λόγον, τάλαινα, ίν᾽ ἂν τῶνδ᾽ αἰτία λάβοιμι\\
\gll tín' \emph{àn} lógon, tálaina, tín' \emph{àn} tônd' aitía láboimi\\
what.\textsc{m.acc.sg} \textsc{irr} account.\textsc{acc.sg} wretched.\textsc{f.nom.sg} what.\textsc{m.acc.sg} \textsc{irr} this.\textsc{n.gen.pl} guilty.\textsc{f.nom.sg} take.\textsc{1sg.aor.opt.act}\\
\trans `What, alas! will be said of me, who am the cause of it?' (Euripides, \textit{Suppliants} 606)\footnote{\emph{Translator's note}: The Perseus edition has \textit{tálaina, tína lógon} for \textit{tín' àn lógon, tálaina}.}
\label{multian31}
\end{exe}

\begin{exe}
\ex οὐκ ἂν δυναίμην οὔτ᾽ ἐρωτῆϲαι τάδε οὔτ᾽ ἂν πιθέϲθαι\\
\gll ouk \emph{àn} dunaímēn oút' erōtêsai táde oút' \emph{àn} pithésthai\\
not \textsc{irr} can.\textsc{1sg.prs.opt.pass} nor ask.\textsc{aor.inf} this.\textsc{n.acc.pl} nor \textsc{irr} persuade.\textsc{aor.inf.mid}\\
\trans `I could neither ask nor believe these things.' (Euripides, \textit{Suppliants} 853)
\label{multian32}
\end{exe}

\begin{exe}
\ex ἦ τἆρ᾽ ἂν ὄψε γ᾽ ἄνδρεϲ ἐξεύροιεν ἄν\\
\gll ê târ' \emph{àn} ópse g' ándres exeúroien \emph{án}\\
in.truth and=then \textsc{irr} late even man.\textsc{nom.pl} discover.\textsc{3pl.aor.opt} \textsc{irr}\\
\trans `And so truly, men would not soon discover ...' (Euripides, \textit{Hippolytus} 480)
\label{multian33}
\end{exe}

\begin{exe}
\ex οὔτ᾽ ἂν ξένοιϲι τοῖϲι ϲοῖϲ χρηϲαίμεθ᾽ ἄν\\
\gll oút' \emph{àn} xénoisi toîsi soîs khrēsaímeth' \emph{án}\\
nor \textsc{irr} stranger.\textsc{dat.pl} the.\textsc{m.dat.pl} your.\textsc{m.dat.pl} use.\textsc{1pl.aor.opt.mid} \textsc{irr}\\
\trans `I will accept no help from your friends' (Euripides, \textit{Medea} 616)
\label{multian34}
\end{exe}

\begin{exe}
\ex ἆρ᾽ ἂν τύραννον διολέϲαι δυναίμεθ᾽ ἄν\\
\gll âr' \emph{àn} túrannon diolésai dunaímeth' \emph{án}\\
then \textsc{irr} king.\textsc{acc.sg} destroy.\textsc{aor.inf} can.\textsc{1pl.prs.opt.pass} \textsc{irr}\\
\trans `Could we murder the king?' (Euripides, \textit{Iphigenia in Tauris} 1020)
\label{multian35}
\end{exe}

\begin{exe}
\ex οὐκέτ᾽ ἂν φθάνοιϲ ἂν αὔραν ἱϲτίοιϲ καραδοκῶν\\
\gll oukét' \emph{àn} phthánois \emph{àn} aúran histíois karadokôn\\
no.more \textsc{irr} arrive.\textsc{2sg.prs.opt} \textsc{irr} breeze.\textsc{acc.sg} sheet.\textsc{dat.pl} await.\textsc{ptcp.prs.m.nom.sg}\\
\trans `It would no longer be too soon to await a breeze for your sails' (Euripides, \textit{Trojan Women} 456)
\label{multian36}
\end{exe}

\begin{exe}
\ex ἀφανεῖϲ ἂν ὄντεϲ οὐκ ἂν ὑμνηθεῖμεν ἄν\\
\gll aphaneîs \emph{àn} óntes ouk \emph{àn} humnētheîmen \emph{án}\\
unseen.\textsc{f.nom.pl} \textsc{irr} be.\textsc{ptcp.prs.f.nom.pl} not \textsc{irr} hymn.\textsc{1pl.aor.pass} \textsc{irr}\\
\trans `Being unknown, we should have been unsung.' (Euripides, \textit{Trojan Women} 1240)
\label{multian37}
\end{exe}

\begin{exe}
\ex μόνον δ᾽ ἂν ἀντὶ χρημάτων οὐκ ἂν λάβοιϲ\\
\gll mónon d' \emph{àn} antì khrēmátōn ouk \emph{àn} lábois\\
alone then \textsc{irr} against property.\textsc{gen.pl} not \textsc{irr} take.\textsc{2sg.aor.opt}\\
\trans `But you alone would not take for money ...' (Euripides, \textit{Meleagros} Fragment 527; \citealp[528--529]{Nauck1889} would prefer \textit{én} for the first \textit{án})
\label{multian38}
\end{exe}

\begin{exe}
\ex λέγω ... καὶ κάθ᾽ ἕκαϲτον, δοκεῖν ἄν μοι τὸν αὐτὸν ἄνδρα παρ᾽ ἡμῶν ἐπὶ πλεῖϲτ᾽ ἂν εἴδη κὰι μετὰ χαρίτων μάλιϲτ᾽ εὐτραπέλωϲ τὸ ϲῶμα αὔταρκεϲ παρέχεϲθαι\\
\gll légō kaì káth' hékaston, dokeîn \emph{án} moi tòn autòn ándra par' hēmôn epì pleîst' \emph{àn} eídē kài metà kharítōn málist' eutrapélōs tò sôma aútarkes parékhesthai\\
say.\textsc{1sg.prs} and down each.\textsc{m.acc.sg} seem.\textsc{prs.inf} \textsc{irr} me.\textsc{dat} the.\textsc{m.acc.sg} same.\textsc{m.acc.sg} man.\textsc{acc.sg} from us.\textsc{gen} upon most.\textsc{n.acc.pl} \textsc{irr} form.\textsc{acc.pl} and with grace.\textsc{gen.pl} most resourcefully the.\textsc{n.acc.sg} body.\textsc{acc.sg} independent.\textsc{n.acc.sg} supply.\textsc{prs.inf.pass}\\
\trans `I say ... and it seems to me that individually, the very man coming from us would display the most personal self-sufficiency in the most circumstances and with the greatest grace and resourcefulness.' (Thucydides 2.41.1; cf. \citealp[87]{PoppoStahl1889} \hyperlink{p401}{\emph{[p401]}} on this example)
\label{multian39}
\end{exe}

\begin{exe}
\ex οὐδ᾽ ἂν ϲφῶν πειραϲομένουϲ ... αὐτοὺϲ δακεῖν ἧϲϲον, ἀλλὰ πολλῷ μᾶλλον ... εὔνουϲ ἂν ϲφίϲι γενέϲθαι\\
\gll oud' \emph{àn} sphôn peirasoménous autoùs dakeîn hêsson, allà pollôi mâllon eúnous \emph{àn} sphísi genésthai\\
nor \textsc{irr} them.\textsc{gen} try.\textsc{ptcp.aor.mid.m.acc.pl} them.\textsc{acc} bite.\textsc{aor.inf} less but much.\textsc{n.dat.sg} more right-minded.\textsc{m.acc.pl} \textsc{irr} them.\textsc{dat} become.\textsc{aor.inf.mid}\\
\trans `... that by giving them a trial they would annoy them less, and yet become much better-disposed toward them' (Thucydides 4.114.4)\footnote{\emph{Translator's note}: The Perseus edition has \textit{dokeîn} for \textit{dakeîn}.}
\label{multian40}
\end{exe}

\begin{exe}
\ex τάχ᾽ ἂν δ᾽ ἴϲωϲ, εἰ ... λάβοιεν ... , καὶ πάνυ ἂν ξυνεπίθοιντο\\
\gll tákh' \emph{àn} d' ísōs, ei láboien kaì pánu \emph{àn} xunepíthointo\\
quickly \textsc{irr} than perhaps if take.\textsc{3pl.aor.opt} also quite \textsc{irr} join.in.\textsc{3pl.aor.opt.mid}\\
\trans `And it is only too probable that if they found ... they would attack us vigorously' (Thucydides 6.10.4)\footnote{\emph{Translator's note}: The Perseus edition starts with \textit{tákha d' àn ísōs}, which Wackernagel cites above as a variant.}
\label{multian41}
\end{exe}

\begin{exe}
\ex Σικελιῶται δ᾽ ἄν μοι δοκοῦϲιν, ὥϲ γε νῦν ἔχουϲιν, καὶ ἔτι ἂν ἧϲϲον δεινοὶ ἡμῖν γενέϲθαι\\
\gll Sikeliôtai d' \emph{án} moi dokoûsin, hṓs ge nûn ékhousin, kaì éti \emph{àn} hêsson deinoì hēmîn genésthai\\
Siceliot.\textsc{nom.pl} then \textsc{irr} me.\textsc{dat} seem.\textsc{3pl.prs} as even now have.\textsc{3pl.prs} also still \textsc{irr} less terrible.\textsc{m.nom.pl} us.\textsc{dat} become.\textsc{aor.inf.mid}\\
\trans `And the Siceliots seem to me, even as they are now, to have become even less dangerous still to us.' (Thucydides 6.11.2)
\label{multian42}
\end{exe}

\begin{exe}
\ex βραχὺ ἄν τι προϲκτώμενοι αὐτῇ περὶ αὐτῆϲ ἂν ταύτηϲ μᾶλλον κινδυνεύοιμεν\\
\gll brakhù \emph{án} ti prosktṓmenoi autêi perì autês \emph{àn} taútēs mâllon kinduneúoimen\\
short.\textsc{n.acc.sg} \textsc{irr} something.\textsc{acc.sg} gain.\textsc{ptcp.prs.pass.m.nom.pl} same.\textsc{f.dat.sg} about same.\textsc{f.gen.sg} \textsc{irr} this.\textsc{f.gen.sg} more endanger.\textsc{1pl.prs.opt}\\
\trans `We should make but few new conquests, and should imperil those we have already won.' (Thucydides 6.18.2)
\label{multian43}
\end{exe}

\begin{exe}
\ex γενομένηϲ δ᾽ ἂν ... ἀρχῆϲ ἀπορεῖν ἂν αὐτόν\\
\gll genoménēs d' \emph{àn} arkhês aporeîn \emph{àn} autón\\
become.\textsc{ptcp.aor.mid.f.gen.sg} and \textsc{irr} beginning.\textsc{gen.sg} puzzle.\textsc{prs.inf} \textsc{irr} him.\textsc{acc}\\
\trans `... and, the command having become ... he would be at a loss' (Thucydides 8.46.2)
\label{multian44}
\end{exe}

\begin{exe}
\ex οὔτε ἂν αὐτῷ τῷ λέγοντι οὔτε τοῖϲ ἀκούουϲι δῆλα ἂν εἴη\\
\gll oúte \emph{àn} autôi tôi légonti oúte toîs akoúousi dêla \emph{àn} eíē\\
nor \textsc{irr} same.\textsc{m.dat.sg} the.\textsc{m.dat.sg} say.\textsc{ptcp.prs.m.dat.sg} nor the.\textsc{m.dat.pl} hear.\textsc{ptcp.prs.m.dat.pl} clear.\textsc{n.nom.pl} \textsc{irr} be.\textsc{3sg.prs.opt}\\
\trans `Neither to the speaker nor to the hearers would it be clear ...' (Hippocrates, \textit{On Ancient Medicine} 1; \citealp[572]{Littre1839})
\label{multian45}
\end{exe}

\begin{exe}
\ex οὐδ᾽ ἂν ἐλαφρῶϲ ἂν ἀπεπλίξατο\\
\gll oud' \emph{àn} elaphrôs \emph{àn} apeplíxato\\
nor \textsc{irr} lightly \textsc{irr} trot.off.\textsc{3sg.aor.mid}\\
\trans `... nor would he have trotted off lightly' (Aristophanes, \textit{Acharnians} 218)
\label{multian46}
\end{exe}

\begin{exe}
\ex πώϲ δέ γ᾽ ἂν καλῶϲ λέγοιϲ ἄν\\
\gll pṓs dé g' \emph{àn} kalôs légois \emph{án}\\
how but even \textsc{irr} well say.\textsc{2sg.prs.opt} \textsc{irr}\\
\trans `But how can you say ``well'' ... ?' (Aristophanes, \textit{Acharnians} 308)
\label{multian47}
\end{exe}

\begin{exe}
\ex μαμμᾶν δ᾽ ἂν αἰτήϲαντοϲ ἧκόν ϲοι φέρων ἂν ἄρτον\\
\gll mammân d' \emph{àn} aitḗsantos hêkón soi phérōn \emph{àn} árton\\
mother.\textsc{acc.sg} then \textsc{irr} ask.\textsc{ptcp.aor.m.gen.sg} arrive.\textsc{1sg.imp} you.\textsc{dat} bear.\textsc{ptcp.prs.m.nom.sg} \textsc{irr} loaf.\textsc{acc.sg}\\
\trans `When you cried for food I would come to you bringing bread.' (Aristophanes, \textit{Clouds} 1383)
\label{multian48}
\end{exe}

\begin{exe}
\ex πῶϲ ἄν ποτ᾽ ἀφικοίμην ἂν εὐθὺ τοῦ Διόϲ\\
\gll pôs \emph{án} pot' aphikoímēn \emph{àn} euthù toû Diós\\
how \textsc{irr} sometime arrive.\textsc{1sg.aor.opt.mid} \textsc{irr} straight the.\textsc{m.gen.sg} Zeus.\textsc{gen}\\
\trans `However could I go straight to Zeus?' (Aristophanes, \textit{Peace} 68)
\label{multian49}
\end{exe}

\begin{exe}
\ex ἡ δ᾽ Ἑλλὰϲ ἂν έξερημωθεῖϲ᾽ ἂν ὑμᾶϲ ἔλαθε\\
\gll hē d' Hellàs \emph{àn} éxerēmōtheîs' \emph{àn} humâs élathe\\
the.\textsc{f.nom.sg} then Greece.\textsc{nom} \textsc{irr} desolate.\textsc{ptcp.aor.pass.f.nom.sg} \textsc{irr} you.\textsc{acc.pl} hide.\textsc{3sg.aor}\\
\trans `And Greece, having been left destitute, escaped your notice.' (Aristophanes, \textit{Peace} 646)
\label{multian50}
\end{exe}

\begin{exe}
\ex οὐκ ἂν πριαίμην οὐδ᾽ ἂν ἰϲχάδοϲ μιᾶϲ\\
\gll ouk \emph{àn} priaímēn oud' \emph{àn} iskhádos miâs\\
not \textsc{irr} buy.\textsc{1sg.prs.opt.pass} nor \textsc{irr} fig.\textsc{gen.sg} one.\textsc{f.gen.sg}\\
\trans `I would not buy, not even for one fig.' (Aristophanes, \textit{Peace} 1223)
\label{multian51}
\end{exe}

\begin{exe}
\ex καὶ πῶϲ ἂν ἔτι γένοιτ᾽ ἂν εὔτακτοϲ πόλιϲ\\
\gll kaì pôs \emph{àn} éti génoit' \emph{àn} eútaktos pólis\\
and how \textsc{irr} still become.\textsc{3sg.aor.opt.mid} \textsc{irr} well-ordered.\textsc{f.nom.sg} city.\textsc{nom.sg}\\
\trans `And how could a city become so well-ordered ... ?' (Aristophanes, \textit{Birds} 829)
\label{multian52}
\end{exe}

\begin{exe}
\ex ἐγὼ δέ τἂν κἄν, εἴ με χρείη ... ἐκπιεῖν\\
\gll egṑ dé t\emph{àn} k\emph{án}, eí me khreíē ekpieîn\\
I.\textsc{nom} but and=\textsc{irr} and=\textsc{irr} if me.\textsc{acc} need.\textsc{3sg.prs.opt} drink.up.\textsc{aor.inf}\\
\trans `And so would I, even if I had to drink up ...' (Aristophanes, \textit{Lysistrata} 113)
\label{multian53}
\end{exe}

\begin{exe}
\ex ἐγὼ δέ γ᾽ ἂν κἂν ὥϲπερ εἰ ψῆτταν δοκῶ δοῦναι ἂν ἐμαυτῆϲ παρταμοῦϲα θἤμιϲυ\\
\gll egṑ dé g' \emph{àn} k\emph{àn} hṓsper ei psêttan dokô doûnai \emph{àn} emautês partamoûsa thḗmisu\\
I.\textsc{nom} but even \textsc{irr} and=\textsc{irr} like if turbot.\textsc{acc.sg} think.\textsc{1sg.prs} give.\textsc{aor.inf} \textsc{irr} myself.\textsc{f.gen.sg} cut.off.\textsc{ptcp.fut.f.nom.sg} the=half.\textsc{acc.sg}\\
\trans `And I would too, even if I expected to cut off half of myself and give it like a turbot.' (Aristophanes, \textit{Lysistrata} 115)\footnote{\emph{Translator's note}: The Perseus edition has \textit{paratemoûsa} for \textit{partamoûsa}.}
\label{multian54}
\end{exe}

\begin{exe}
\ex μᾶλλον ἂν διὰ τουτογὶ γένοιτ᾽ ἂν εἰρήνη\\
\gll mâllon \emph{àn} dià toutogì génoit' \emph{àn} eirḗnē\\
more \textsc{irr} through this.\textsc{n.acc.sg.emph} become.\textsc{3sg.aor.opt.mid} \textsc{irr} peace.\textsc{nom.sg}\\
\trans `Would peace come to pass rather through this?' (Aristophanes, \textit{Lysistrata} 147)
\label{multian55}
\end{exe}

\begin{exe}
\ex φωνὴν ἂν οὐκ ἂν εἶχον\\
\gll phōnḕn \emph{àn} ouk \emph{àn} eîkhon\\
sound.\textsc{acc.sg} \textsc{irr} not \textsc{irr} have.\textsc{3pl.imp}\\
\trans `... they would not make a sound' (Aristophanes, \textit{Lysistrata} 361)
\label{multian56}
\end{exe}

\begin{exe}
\ex ἦ τἄν ϲε κωκύειν ἂν ἐκέλευον μακρά\\
\gll ê t\emph{án} se kōkúein \emph{àn} ekéleuon makrá\\
in.truth and=\textsc{irr} you.\textsc{acc} wail.\textsc{prs.inf} \textsc{irr} order.\textsc{1sg.imp} large.\textsc{n.acc.pl}\\
\trans `Truly I would make you wail more.' (Aristophanes, \textit{Frogs} 34)
\label{multian57}
\end{exe}

\begin{exe}
\ex οὐκ ἂν γενοίμην Ἡρακλῆϲ ἄν\\
\gll ouk \emph{àn} genoímēn Hēraklês \emph{án}\\
not \textsc{irr} become.\textsc{1sg.aor.opt.mid} Hercules.\textsc{nom} \textsc{irr}\\
\trans `I won't be Hercules' (Aristophanes, \textit{Frogs} 581)
\label{multian58}
\end{exe}

\begin{exe}
\ex οὐκ ἂν φθάνοιϲ τὸ γένειον ἂν περιδουμένη\\
\gll ouk \emph{àn} phthánois tò géneion \emph{àn} peridouménē\\
not \textsc{irr} arrive.\textsc{2sg.prs.opt} the.\textsc{n.acc.sg} beard.\textsc{acc.sg} \textsc{irr} bind.up\textsc{.ptcp.prs.pass.f.nom.sg}\\
\trans `It wouldn't be too soon to tie on your beard.' (Aristophanes, \textit{Ecclesiazusae} 118)
\label{multian59}
\end{exe}

\begin{exe}
\ex ἴϲωϲ ἂν ἐγὼ περὶ τοῦ μεθύϲκεϲθαι ... τἀληθῆ λέγων ἧττον ἂν εἴην ἀηδήϲ\\
\gll ísōs \emph{àn} egṑ perì toû methúskesthai talēthê légōn hêtton \emph{àn} eíēn aēdḗs\\
perhaps \textsc{irr} I.\textsc{nom} about the.\textsc{n.gen.sg} intoxicate.\textsc{prs.inf.pass} the=true.\textsc{n.acc.pl} say.\textsc{ptcp.prs.m.nom.sg} less \textsc{irr} be.\textsc{1sg.prs.opt} distasteful.\textsc{m.nom.sg}\\
\trans `Perhaps I would be less disagreeable speaking the truth about intoxication.' (Plato, \textit{Symposium} (\textit{Apology} 41a) 176c)
\label{multian60}
\end{exe}

\begin{exe}
\ex εἰκότωϲ ἂν τοὺϲ ἐρῶνταϲ μᾶλλον ἂν φοβοῖο\\
\gll eikótōs \emph{àn} toùs erôntas mâllon \emph{àn} phoboîo\\
justly \textsc{irr} the.\textsc{m.acc.pl} love.\textsc{ptcp.prs.m.acc.pl} more \textsc{irr} frighten.\textsc{2sg.prs.opt.pass}\\
\trans `You would reasonably be more frightened for the lovers' (Plato, \textit{Phaedrus} 232c; \citealp[7]{Schanz1882} has \textit{dḗ} for the first \textit{àn})
\label{multian61}
\end{exe}

\begin{exe}
\ex τάχ᾽ οὖν ἂν ὑπὸ φιλοτιμίαϲ ἐπίϲχοι ἡμῖν ἂν τοῦ γράφειν\\
\gll tákh' oûn \emph{àn} hupò philotimías epískhoi hēmîn \emph{àn} toû gráphein\\
quickly so \textsc{irr} under ambition.\textsc{gen.sg} hold.back.\textsc{3sg.aor.opt} us.\textsc{dat} \textsc{irr} the.\textsc{n.gen.sg} write.\textsc{prs.inf}\\
\trans `So perhaps out of pride he may refrain from writing to us.' (Plato, \textit{Phaedrus} 257c)
\label{multian62}
\end{exe}

\begin{exe}
\ex οὐκ ἂν ῥᾳδίωϲ οὐδὲ πολλὰ ἂν εὕροιϲ ὡϲ τοῦτο\\
\gll ouk \emph{àn} rhāidíōs oudè pollà \emph{àn} heúrois hōs toûto\\
not \textsc{irr} easily nor many.\textsc{n.acc.pl} \textsc{irr} find.\textsc{2sg.aor.opt} as this.\textsc{n.acc.sg}\\
\trans `You would not find many like this, nor easily.' (Plato, \textit{Republic} 7.526c)
\label{multian63}
\end{exe}

\begin{exe}
\ex κἂν ὀλίγου, εἴ με κελεύοιϲ ἀποδύντα ὀρχήϲαϲθαι, χαριϲαίμην ἄν\\
\gll k\emph{àn} olígou, eí me keleúois apodúnta orkhḗsasthai, kharisaímēn \emph{án}\\
and=\textsc{irr} little.\textsc{n.gen.sg} if me.\textsc{acc} order.\textsc{2sg.prs.opt} undress.\textsc{ptcp.aor.m.acc.sg} dance.\textsc{aor.inf.mid} gratify.\textsc{1sg.aor.opt.mid} \textsc{irr}\\
\trans `And I would almost gratify you if you were to bid me strip and dance' (Plato, \textit{Menexenus} 236d)
\label{multian64}
\end{exe}

\begin{exe}
\ex πῶϲ οὖν ἄν ποτέ τιϲ ... δύναιτ᾽ ἂν ὑγιέϲ τι λέγων ἀντειπεῖν\\
\gll pôs oûn \emph{án} poté tis dúnait' \emph{àn} hugiés ti légōn anteipeîn\\
how so \textsc{irr} sometime someone.\textsc{m.nom.sg} can.\textsc{3sg.prs.opt.pass} \textsc{irr} healthy.\textsc{n.acc.sg} something.\textsc{acc.sg} say.\textsc{ptcp.prs.m.nom.sg} argue.\textsc{aor.inf}\\
\trans `Then how could one ever argue ... saying anything sound?' (Plato, \textit{Sophist} 233a)
\label{multian65}
\end{exe}

\begin{exe}
\ex ϲχολῇ ποτ᾽ ἂν αὐτοῖϲ τιϲ χρήματα διδοὺϲ ἤθελεν ἂν ... μαθητὴϲ γίγνεϲθαι\\
\gll skholêi pot' \emph{àn} autoîs tis khrḗmata didoùs ḗthelen \emph{àn} mathētḕs gígnesthai\\
scarcely sometime \textsc{irr} them.\textsc{dat} someone.\textsc{m.nom.sg} property.\textsc{acc.pl} give.\textsc{ptcp.prs.m.nom.sg} want.\textsc{3sg.imp} \textsc{irr} pupil.\textsc{nom.sg} become.\textsc{prs.inf.pass}\\
\trans `Scarcely anyone would ever want to become (their) pupil, giving them money.' (Plato, \textit{Sophist} 233b; cf. also \textit{Laws} 5.742c)
\label{multian66}
\end{exe}

\begin{exe}
\ex ϲτὰϲ ἂν ὥϲπερ οὗτοϲ ἐπὶ τῇ εἰϲόδῳ ... λέγοιμ᾽ ἄν\\
\gll stàs \emph{àn} hṓsper hoûtos epì têi eisódōi légoim' \emph{án}\\
stand.\textsc{ptcp.aor.m.nom.sg} \textsc{irr} like this.\textsc{m.nom.sg} upon the.\textsc{f.dat.sg} entrance.\textsc{dat.sg} say.\textsc{1sg.prs.opt} \textsc{irr}\\
\trans `Standing at the door just like him, I would say ...' (Xenophon, \textit{Cyropaedia} 1.3.11)
\label{multian67}
\end{exe}

\begin{exe}
\ex ὑμῶν δ᾽ ἔρημοϲ ὤν, οὐκ ἂν ἱκανὸϲ οἶμαι εἶναι οὔτ᾽ ἂν φίλον ὠφελῆϲαι οὔτ᾽ ἂν ἐχθρὸν ἀλέξαϲθαι\\
\gll humôn d' érēmos ṓn, ouk \emph{àn} hikanòs oîmai eînai oút' \emph{àn} phílon ōphelêsai oút' \emph{àn} ekhthròn aléxasthai\\
you.\textsc{gen.pl} then solitary.\textsc{m.nom.sg} be.\textsc{ptcp.prs.m.nom.sg} not \textsc{irr} sufficient.\textsc{m.nom.sg} think.\textsc{1sg.prs.pass} be.\textsc{prs.inf} nor \textsc{irr} friend.\textsc{acc} help.\textsc{aor.inf} nor \textsc{irr} enemy.\textsc{acc.sg} ward.off.\textsc{aor.inf.mid}\\
\trans `But bereft of you I do not think I shall be able either to aid a friend or to ward off a foe.' (Xenophon, \textit{Anabasis} 1.3.6)
\label{multian68}
\end{exe}

\begin{exe}
\ex δοκοῦμεν δ᾽ ἄν μοι ταύτῃ προϲποιούμενοι προϲβαλεῖν ἐρημωτέρῳ ἂν τῷ ὄρει χρῆϲθαι\\
\gll dokoûmen d' \emph{án} moi taútēi prospoioúmenoi prosbaleîn erēmōtérōi \emph{àn} tôi órei khrêsthai\\
seem.\textsc{1pl.prs} then \textsc{irr} me.\textsc{dat} here pretend.\textsc{ptcp.prs.pass.m.nom.pl} attack.\textsc{aor.inf} solitary.\textsc{comp.n.dat.sg} \textsc{irr} the.\textsc{n.dat.sg} mountain.\textsc{dat.sg} use.\textsc{prs.inf.pass}\\
\trans `I do think, however, that if we should make a feint of attacking here, we should find the rest of the mountain all the more deserted' (Xenophon, \textit{Anabasis} 4.6.13)
\label{multian69}
\end{exe}

\begin{exe}
\ex διαϲπαϲθέντεϲ δ᾽ ἂν καὶ κατὰ μικρὰ γενομένηϲ τῆϲ δυνάμεωϲ οὔτ᾽ ἂν τροφὴν δύναιϲθε λαμβάνειν οὔτε χαίροντεϲ ἂν ἀπαλλάξαιτε\\
\gll diaspasthéntes d' \emph{àn} kaì katà mikrà genoménēs tês dunámeōs oút' \emph{àn} trophḕn dúnaisthe lambánein oúte khaírontes \emph{àn} apalláxaite\\
separate.\textsc{ptcp.aor.pass.m.nom.pl} then \textsc{irr} and down small.\textsc{n.acc.pl} become.\textsc{ptcp.aor.mid.f.gen.sg} the.\textsc{f.gen.sg} power.\textsc{gen.sg} nor \textsc{irr} food.\textsc{acc} can.\textsc{2pl.prs.opt.pass} take.\textsc{prs.inf} nor rejoice.\textsc{ptcp.prs.m.nom.pl} \textsc{irr} deliver.\textsc{2pl.aor.opt}\\
\trans `But separated and with your force in small parts, you could neither get food nor emerge safe.' (Xenophon, \textit{Anabasis} 5.6.32)
\label{multian70}
\end{exe}

\begin{exe}
\ex ὦδ᾽ ἂν ... ἐπιϲκοποῦντεϲ ... ἴϲωϲ ἂν καταμάθοιμεν\\
\gll ôd' \emph{àn} episkopoûntes ísōs \emph{àn} katamáthoimen\\
thus \textsc{irr} oversee.\textsc{ptcp.prs.m.nom.pl} perhaps \textsc{irr} perceive.\textsc{1pl.aor.opt}\\
\trans `Perhaps by considering ... we can thus discover ...' (Xenophon, \textit{The Economist} 4.5)
\label{multian71}
\end{exe}

\hyperlink{p402}{\emph{[p402]}} 

\begin{exe}
\ex εἶδεϲ δ᾽ ἂν αὐτῆϲ Φαρνάβαζον θᾶττον ἄν\\
\gll eîdes d' \emph{àn} autês Pharnábazon thâtton \emph{án}\\
see.\textsc{2sg.aor} then \textsc{irr} her.\textsc{gen} Pharnabazus.\textsc{acc} faster \textsc{irr}\\
\trans `And you could see Pharnabazus more quickly than her.' (Epicrates, Fragment 2/3, line 17; \citealp[283]{Kock1884})
\label{multian72}
\end{exe}

\begin{exe}
\ex οὐκ ἂν ἡγεῖϲθ᾽ αὐτὸν κἂν ἐπιδραμεῖν\\
\gll ouk \emph{àn} hēgeîsth' autòn k\emph{àn} epidrameîn\\
not \textsc{irr} lead.\textsc{2pl.prs.pass} him.\textsc{acc} also=\textsc{irr} rush.\textsc{aor.inf}\\
\trans `Don't you believe he would even have rushed ...' (Demosthenes 27.56)
\label{multian73}
\end{exe}

\begin{exe}
\ex ὧδ᾽ ἂν θεωροῦϲιν γένοιτ᾽ ἂν φανερόν\\
\gll hôd' \emph{àn} theōroûsin génoit' \emph{àn} phanerón\\
thus \textsc{irr} behold.\textsc{ptcp.prs.m.dat.sg} become.\textsc{3sg.aor.opt.mid} \textsc{irr} visible.\textsc{n.nom.sg}\\
\trans `It might become clear by considering thus ...' (Aristotle, \textit{Poetics} 1460b, among many others; cf. \citet[408, 438]{Vahlen1865} on this example)
\label{multian74}
\end{exe}

(Example (\ref{multian75}) does not belong here, since the repetition of \emph{án} is caused by the resumption of interrogative\is{interrogatives} \textit{tí}.)

\begin{exe}
\ex τί ἂν οἴεϲθ᾽ εἰ ... ἀπῆλθον ... , τί ποιεῖν ἂν ἢ τί λέγειν τοὺϲ ἀϲεβεῖϲ ἀνθρώπουϲ τουτουϲί\\
\gll tí \emph{àn} oíesth' ei apêlthon tí poieîn \emph{àn} ḕ tí légein toùs asebeîs anthrṓpous toutousí\\
what.\textsc{n.acc.sg} \textsc{irr} think.\textsc{2pl.prs.pass} if leave.\textsc{3pl.aor} what.\textsc{n.acc.sg} do.\textsc{prs.inf} \textsc{irr} or what.\textsc{n.acc.sg} say.\textsc{prs.inf} the.\textsc{m.acc.pl} impious.\textsc{acc.pl} person.\textsc{acc.pl} this.\textsc{m.acc.pl.emph}\\
\trans `What do you think, if they had gone off ... what (do you think) these ungodly persons would have done or said?' (Demosthenes 18.240)
\label{multian75}
\end{exe}

Assuming my incomplete collection of examples is not too unrepresentative, there is a large decrease in this type of \emph{án}-doubling in the fourth century. In particular, oratorical \isi{prose} contains only few examples; as is well known, Lysias never doubled \emph{án}. I do not doubt that this decrease is due to the gradual extinguishing of the tradition which demanded \emph{án} in second position of the clause.

We also find instances of \emph{án}-doubling in which \emph{án} does not take clausal second position but rather a later position. This is quite natural, as various elements are known to be happily followed by \emph{án}, and therefore, as soon as a clause became more extensive, various mutually conflicting demands had to take effect on the particle. It is beyond the scope of this research to consider the combinations this gives rise to and to adduce examples for each of them, since our task is only to investigate the remains of the old positional law -- however interesting and important it would be for our appreciation of the younger language to illustrate in detail the tendencies that have become dominant there.\il{Greek|)}\is{postpositive particles|)}\is{doubling|)}


\section{Indo-Iranian and Germanic}\is{particles|)}\il{Indo-Iranian|(}

The positional law whose validity for \ili{Greek} has been discussed over the preceding pages has long been recognized for certain of the Asian sister languages.

For \ili{Sanskrit} \isi{prose}, \citet[47]{Delbrueck1878} informs us: ``Enclitic\is{enclitics} words move as close as possible to the beginning of the clause.'' In essence, Bartholomae's \citeyearpar[3]{Bartholomae1886} comments on the \textit{\d{R}gveda} agree with this: ``Even on a superficial assessment it becomes \hyperlink{p403}{\emph{[p403]}} clear that in the \textit{\d{R}gveda} the enclitic\is{enclitics} forms of the personal\is{personal pronouns} \isi{pronouns}, as well as certain particles,\is{particles} in most cases take the second position within the verse or the verse section.'' See the same author \citeyearpar[30]{Bartholomae1887} on \emph{sīm} and \emph{smā̆} as well as the hard \isi{tmesis} in \textit{\d{R}gveda} 5.2.7 \emph{ṧunaṧ cic chēpam niditaṃ sahasrad yūpād amun̑caḥ}.

The same scholar \citeyearpar[3--31]{Bartholomae1886} has made corresponding observations on the Gathas of the Avesta.\il{Avestan} He proposes the following rule \citeyearpar[11ff.]{Bartholomae1886}: ``Enclitic\is{enclitics} \isi{pronouns} and \isi{particles} attach to the first high tone in the verse'', and recognizes exceptions only in the case of \emph{cīṭ}\label{cit}, which often needs to emphasize particular parts of the clause and is then attached to the relevant part. It is easy to see how this observation also relates to \citeauthor{Delbrueck1878}'s rule.

However, this rule is apparently retained to the letter in Middle Indic \isi{prose} (cf. e.g. \citealp[8]{Jacobi1886} line 18 \emph{jena se parikkhemi balavisesaṃ}, in which \emph{se} syntactically belongs to \emph{balavisesaṃ}), and certainly in Old Persian,\il{Persian, Old|(} whose cuneiform inscriptions\is{inscriptions|(} are particularly suitable for such observations due to their solemnly correct style and the precise differentiation of \isi{enclitics} in their script. I present the material in its entirety following \citet{Spiegel1881}, with the exception of the cases in which the enclitic\is{enclitics} is inserted editorially. The following are found exclusively in second position:

\emph{mai̯} (\textsc{1sg.gen}):\footnote{\emph{Translator's note}: Genitive\is{genitive} and \isi{dative} are collapsed together in Old Persian. We gloss them as \isi{genitive} in the following.} following the gendered nominatives\is{nominative} \emph{Auramazdā} `Ahura Mazda' in the Behistun inscription 1.25, 1.55, 1.87, 1.94, 2.24, 2.40, 2.60, 2.68, 3.6, 3.17, 3.37, 3.44, 3.60, 3.65, 3.86, 4.60, and Persepolis NR$^a$ 50, \emph{dahyāuš} `country' in Behistun 4.79, and \emph{hauv} `he' in Behistun 2.79 and 3.11; also following the neuter \emph{tya} (\textsc{rel}) (excluding Behistun 4.65, on which nothing can be said with certainty because of the lacuna) in Xerxes A 24, A 30, Ca 13 (twice), Cb 22 (twice), D 19, and Ea 19; finally after \emph{utā} `and' in Behistun 4.74, 4.78, and Xerxes D 15 (and also NR$^a$ 52 and Xerxes D 18, E 18, and A 29, although in these cases \emph{utā} links only constituents rather than clauses).

\emph{tai̯} (\textsc{2sg.gen}): following the gendered nominatives\is{nominative} \emph{Auramazdā} `Ahura Mazda' in Behistun 4.58, and 4.78, \emph{hauv} `he' in NR$^a$ 57 (where, however, according to \citeauthor{Thumb1887}'s analysis \emph{taiy} should be in fifth position!), \hyperlink{p404}{\emph{[p404]}} following the neuter \emph{ava} `that' in Behistun 4.76 and 4.79, following \emph{ada} `then' in NR$^a$ 43 and 45, and following \emph{utā} (\textsc{conj}) in Behistun 4.58, 4.75, and 4.79.

\emph{šai̯} (\textsc{3sg.gen}): following \emph{hauv} `he' in Darius H 3, following \emph{tayai̯} (\textsc{rel.nom.pl}) in Behistun 1.57, 2.77, 3.48, 3.51, and 3.73, following \emph{avaθā} `then' in 3.14, following \emph{utā} (\textsc{conj}) in 2.74, 2.89, and 5.11, and following \emph{pasāva} `afterward' in 2.88.

\emph{mai̯}, \emph{tai̯} and \emph{šai̯} thus follow the rule in all 56 instances, attaching to a wide variety of words, without a single counterexample. Particularly noteworthy is Behistun 1.57 ((\ref{behistun157})), as opposed to the \emph{utā martiyā tayai̯\emph{šai} fratamā} etc. of the other examples with \emph{tayai̯šai}, and also Behistun 4.74 = 4.78 ((\ref{behistun474})), in which \emph{mai̯} precedes the intervening clause while the verb comes after it;\footnote{\emph{Translator's note}: More recent editions (e.g. \citealp{Kent1953,Schmitt1991}) read this enclitic\is{enclitics} as \textit{tai̯} (\textsc{2sg.gen}) rather than \textit{mai̯} (\textsc{1sg.gen}), but this does not alter Wackernagel's general point.}\is{verb position} but especially Xerxes D 15 ((\ref{xerxesd15})) = \textit{kaí \emph{moi} átta ho patḕr epoíēsen} `and me.\textsc{dat} that.\textsc{n.acc.pl} the.\textsc{m.nom.sg} father.\textsc{nom.sg} make.\textsc{3sg.aor}', where the \emph{mai̯} that belongs to the relative clause\is{relative clauses} is placed before the relative pronoun\is{pronouns}\is{relative pronouns} in order to attach to \emph{utā}.

\begin{exe}
\ex
\gll avaθā adam hadā kamnai̯biš martiyai̯biš avam Gau̯mātam tayam magum avājanam utā tayai̯=\emph{šai̯} fratamā martiyā anušiyā āhantā\\
then \textsc{1.nom.sg} with few.\textsc{ins.pl.m} man.\textsc{ins.pl.m} \textsc{dem.acc.sg.m} G.\textsc{acc.sg.m} \textsc{rel.acc.sg.m} magus.\textsc{acc.sg.m} slay.\textsc{1.sg.pst} \textsc{conj} \textsc{rel.nom.pl.m}=\textsc{3.gen.sg} foremost.\textsc{nom.pl.m} man.\textsc{nom.pl.m} follower.\textsc{nom.pl.m} be.\textsc{3.pl.pst}\\
\trans `then I with a few men slew that Gaumāta the magus and the men who were his foremost followers' (Behistun 1.57)
\label{behistun157}%going into I.58; āhanta is middle voice, but I doubt this is relevant information
\end{exe}

\begin{exe}
\ex
\gll yadi imām dipim vai̯nāhi imai̯=vā patikarā, nai̯=diš vikanāhi utā=\emph{tai̯} yāvā tau̯mā ahati paribarāhi=diš\\
\textsc{conj} \textsc{dem.acc.sg.f} inscription.\textsc{acc.sg.f} see.\textsc{2.sg.sbjv} \textsc{dem.acc.pl.m}=\textsc{conj} sculpture.\textsc{acc.pl.m} \textsc{neg}=\textsc{3.acc.pl} destroy.\textsc{2.sg.sbjv} \textsc{conj}=\textsc{2.sg.gen} \textsc{conj} strength.\textsc{nom.sg.n} be.\textsc{3.sg.sbjv} protect.\textsc{2.sg.imp}=\textsc{3.acc.pl}\\
\trans `If you shall look at this inscription or these sculptures, (and) shall not destroy them and, as long as there is strength to you, shall care for them, ...' (Behistun 4.74 = 4.78)
\label{behistun474}
\end{exe}

\begin{exe}
\ex
\gll taya adam akunavam uta=mai̯ taya pitā akunau̯š\\
\textsc{rel.nom/acc.pl.n} \textsc{1.nom.sg} make.\textsc{1.sg.pst} \textsc{conj}=\textsc{1.sg.gen} \textsc{rel.nom/acc.pl.n} father.\textsc{nom.sg.m} make.\textsc{3.sg.pst}\\
\trans `... which I built and which my father built.' (Xerxes D 15)\footnote{\emph{Translator's note}: The standard reference is XPa 15.}
\label{xerxesd15}
\end{exe}

The other pronominal\is{pronouns} \isi{enclitics} yield very similar results. \rephrase{Enclitic}{The enclitic}\is{enclitics} \emph{mām} (\textsc{1sg.acc}), the only instance of which (Behistun 1.52) follows clause-initial \emph{mātya} `lest, that not'; \emph{šim} (\textsc{3sg.acc}) following the nominatives\is{nominative} \emph{āpi} `water' in Behistun 1.95, \emph{kāra} `people' in 1.50, \emph{adam} (\textsc{1sg.nom}) in 1.52, and \emph{haruva} `whole' in 2.75 and 2.90; following the \isi{accusative} \emph{šatram} `command, empire' in 1.59; following the \isi{particles} \emph{avadā} `there, then' in 1.59, 3.79, and 5.14, \emph{nai} (\textsc{neg}) in 4.49, and \emph{pasāva} `afterwards' in 2.90; \emph{šiš} (\textsc{3pl.acc}) following \emph{avadā} `there, then' in 3.52; \emph{šām} (\textsc{3sg.gen}) following the nominatives\is{nominative} \emph{adam} (\textsc{1sg.nom}) in NR$^a$ 18 and \emph{hya} (\textsc{rel.m}) in Behistun 2.13; following the \isi{accusative} \emph{avam} (\textsc{3sg.m.acc}) in Behistun 2.20 and 2.83; following the neuter \emph{tya} (\textsc{rel}) in Behistun 1.19 and NR$^a$ 20 and 36; following the \isi{particles} \emph{avaθā} `then' in 2.27, 2.37, 2.42, 2.62, 2.83, 2.98, 3.8, 3.19, 3.40, 3.47, 3.56, 3.63, 3.68, and 3.84, and \emph{utā} (\textsc{conj}) in 3.56.

These 35 instances can be added to the previous 56, but there are also three counterexamples, (\ref{behistun114})--(\ref{NRa35}).

\begin{exe}
\ex
\gll vašnā Auramazdāha adam=\emph{šām} xšāyaθiya āham\\
will.\textsc{ins.sg.m} A.\textsc{gen.sg.m} \textsc{1.nom.sg}=\textsc{3.gen.pl} king.\textsc{nom.sg.m} be.\textsc{1.sg.pst}\\
\trans 'By the favour of Auramazdā I was their king' (Behistun I.14)%beginning in I.13
\label{behistun114}
\end{exe}

\begin{exe}
\ex 
\gll vašna Auramazdāha adam=\emph{šiš} ajanam utā navā xšāyaθiyā agr̥bāyam\\
will.\textsc{ins.sg.m} A.\textsc{gen.sg.m} \textsc{1.nom.sg}=\textsc{3.acc.pl} defeat.\textsc{1.sg.pst} \textsc{conj} nine.\textsc{acc.pl.m} king.\textsc{acc.pl.m} capture.\textsc{1.sg.pst}\\
\trans `By the favour of Auramazdā I defeated them and captured nine kings' (Behistun IV.6)
\label{behistun46}
\end{exe}

\begin{exe}
\ex
\gll vašna Auramazdāhā adam=\emph{šim} gāθavā niyašādayam\\
will.\textsc{ins.sg.m} A.\textsc{gen.sg.m} \textsc{1.nom.sg}=\textsc{3.acc.sg} place.\textsc{loc.sg.m} set-down.\textsc{1.sg.pst}\\
\trans `By the favour of Auramazdā I put it in its proper place.' (NR$^a$ 35)\footnote{\emph{Translator's note}: The standard reference is DNa 35.}
\label{NRa35}
\end{exe}

In each of these cases the enclitic\is{enclitics} is attached to the subject \emph{adam} `I'. And these examples are more than compensated for by such instances as (\ref{behistun275}), in which the pronoun\is{pronouns} intervenes between adjective\is{adjectives} and noun, or (\ref{behistun356}), in which \emph{šām} (\textsc{3pl.gen}) belongs \hyperlink{p405}{\emph{[p405]}} syntactically to \emph{maθištam} `greatest'.

\begin{exe}
\ex
\gll haruva=\emph{šim} kāra avaina\\
whole.\textsc{nom}=\textsc{3.acc.sg} people.\textsc{nom} see.\textsc{3.sg.pst}\\
\trans `All the people saw him' (Behistun II.75 = II.90)
\label{behistun275}
\end{exe}

\begin{exe}
\ex 
\gll Vivāna nāma Pārsa, manā bandaka, Harauvatiyā xšaçapāvā, abi avam, utā=\emph{šam} ai̯vam martiyam maθištam akunau̯š\\
V.\textsc{nom.sg.m} name.\textsc{loc.sg.n} Persian.\textsc{nom.sg.m} \textsc{1.gen.sg} vassal.\textsc{nom.sg.m} Arachosia.\textsc{loc.sg.f} satrap.\textsc{nom.sg.m} against \textsc{dem.acc.sg.m} \textsc{conj}=\textsc{3.gen.pl} one.\textsc{acc.sg.m} man.\textsc{acc.sg.m} greatest.\textsc{acc.sg.m} make.\textsc{3.sg.pst}\\
\trans '(there was) a Persian, Vivana by name, my vassal, satrap in Arachosia, against him {[}he sent an army{]}, and he made one single man their chief' (Behistun III.56)
\label{behistun356}
\end{exe}

Setting aside \emph{hacāma} `from me' and \emph{haca avadaša} `from there', we are left with \emph{-ci̯} (= \ili{Sanskrit} \emph{cit}, \textsc{indf/q}) and \emph{dim}, \emph{diš} (\textsc{3sg.acc}, \textsc{3pl.acc}). The latter two follow the rule after the \isi{nominative} \emph{drauga} `lie' in Behistun 4.34, the neuter \emph{tya} (\textsc{rel}) in Behistun 1.65, the \isi{particles} \emph{nai̯} (\textsc{neg}) in 4.73, 4.78 and \emph{pasāva} `afterwards' in Behistun 4.35 and NR$^a$ 33, and the verbal form \emph{visanāha} `destroy' in Behistun 4.77. Behistun 4.74 (=(\ref{behistun474}) above) hardly counts as a counterexample: (\citealp{Spiegel1881}: ``but you, as long as your family lasts, you maintain me''), because although \emph{diš} (\textsc{3pl.acc}) is not attached to the first word in the clause, it is attached to the first word following the intervening clause. Then the only counterexample is NR$^a$ 42 {[}\textit{yath}{]}\textit{ā xšnās}{[}\textit{āha\emph{diš}}{]} ``so that you know them'', and there it is tempting to ask whether the editorial additions might not be wrong.\footnote{\emph{Translator's note}: Modern editions of this text such as that in \citet[137]{Kent1953} do not assume a pronominal\is{pronouns} clitic here.} 

\emph{ci̯} (\textsc{indf/q}), on the other hand, has freed itself from the rule. Although it occurs in Behistun 1.53 following \emph{kaš} `who', in I\footnote{\emph{Translator's note}: Wackernagel has S instead of I.} 23 following \emph{hauv} `he', and in Xerxes D 20, Ca 14 and Cb 24 in second position, it occurs in third position or later in Behistun 1.46 following \emph{kaš} `who', 1.53 following \emph{cis} `what', 1.63, 1.67, and 1.69 following \emph{paruvam} `before', and 4.46 and Xerxes D 13 following \emph{aniyaš} `other'. In these cases it follows the word that is to be emphasized;\is{inscriptions|)} cf. the position of \emph{cīṭ} in the Avesta\il{Avestan} (above p\pageref{cit}).\il{Persian, Old|)}

This is the situation in the Indo-Iranian languages.\il{Indo-Iranian|)} But instructive parallels can also be found outside these languages.\il{Germanic|(} First, the treatment of weakly stressed personal\is{personal pronouns} \isi{pronouns} in modern \ili{German} shows that our positional rule is not alien to the Germanic languages. Above all, when \emph{sich} (\textsc{3.refl}) occurs in a subordinate\is{subordination} clause and far removed from the verb, the rule makes itself known like an uncomfortable set of shackles, which we happily shed in written composition by placing the pronoun\is{pronouns} next to the verb.\is{verb position} We believe that this creates greater clarity, but we nevertheless view this placement as unattractive. And often in oral conversation we produce a double\is{doubling} \emph{sich}: one in its traditional position at the start, and another next to the verb,\is{verb position} just like the double\is{doubling} \emph{án} in \ili{Greek}. Such a tendency can also be observed with the other personal\is{personal pronouns} \isi{pronouns}.

However, I will not venture a more detailed commentary here, instead \rephrase{merely pointing}{I will merely point } to the cases of \isi{tmesis} in \ili{Gothic} that \citet[80]{Kluge1883} \hyperlink{p406}{\emph{[p406]}} has drawn attention to: \emph{ga-\emph{u}-laubeis} (\textsc{pfv}-\textsc{q}-believe-\textsc{2sg}), \emph{ga-\emph{u}-ƕa-sēƕi} (\textsc{pfv}-\textsc{q}-what-see-\textsc{sbjv.3sg}), \emph{us-\emph{nu}-gibiþ} (out-now-give-\textsc{imp.2pl}), as well as the examples \rephrase{in which}{where} \emph{u(h)} (\textsc{q}/\textsc{conj}) and similar \ili{Gothic} \isi{particles} separate a preposition\is{prepositions} from its case. Kluge correctly identifies a remnant of the language's prehistory in this pressure to attach enclitics immediately following the first word. The most informative example is undoubtedly \emph{ga-u-ƕa-sēƕi}, with its insertion of the indefinite\is{indefinites} \emph{ƕa} `what/something' = \ili{Greek} \textit{tì}.\il{Germanic|)}


\section{Latin: personal pronouns}\il{Latin|(}\is{pronouns|(}\is{personal pronouns|(}

\hypertarget{latinpron}{Leaving} aside the question of whether this could also shed some light on the \ili{Celtic} pronomen infixum\is{infixes} \citep[327ff]{Zeuss1871}, I turn now to Latin, and start by observing that old school Latinists have long taught that, at least in classical \isi{prose}, the position after the first word of the clause is connected with tonal weakness, and that the words which occupy this position are either inherently enclitic\is{enclitics} or become enclitic\is{enclitics} through being placed there (\citealp[818]{Reisig1839}; \citealp[43]{Madvig1839}; \citealp[49, 64]{SeyffertMueller1876}; \citealp[557]{StolzSchmalz1890}; etc.). For a detailed investigation, however, it is unfortunate that, unlike in \ili{Greek}, the manuscripts do not provide us with any external indication of the difference between orthotonic and enclitic\is{enclitics} forms. Nevertheless, we can proceed quite confidently. Because assuming we find, for example, an oblique\is{oblique (case)} case form of the personal pronoun which, according to the evidence of the context, bore no emphasis, and which shows exactly the same positional properties that we found for \emph{moi} and its friends, in my view this is evidence for both the enclitic\is{enclitics} stress of the pronoun in question and the validity in Latin of the rule that we have demonstrated for \ili{Greek}. And there are plenty of such cases.

First, instances of \isi{tmesis} between preposition\is{prepositions} and verb (cf. p\pageref{VPtmesis} above for \ili{Greek}), as in (\ref{adjWithPer1}). 

\begin{exe}
\ex
\gll sub \emph{vos} placo, ob \emph{vos} sacro\\
under you.\textsc{acc} plead.\textsc{pres.1sg} because-of you.\textsc{acc} consecrate.\textsc{pres.1sg}\\
\trans `I entreat you, I beseech you' (Festus, 190b.2, 309a.30)
\label{adjWithPer1}
\end{exe}

Secondly, separation of other collocations that otherwise form a fixed unit by a weakly stressed pronoun seeking second position:

a) \isi{adjectives} with \emph{per} `very': (\ref{adjWithPer2})--(\ref{adjWithPer7}). \hyperlink{p407}{\emph{[p407]}}

\begin{exe}
\ex
\gll in quo per \emph{mihi} mirum visum est\\
in which.\textsc{abl.sg} very I.\textsc{dat.sg} strange.\textsc{nom.sg.n} seem.\textsc{prf.3sg.pass} is.\textsc{pres.3sg}\\
\trans `wherein it seemed very strange to me' (Cicero, \textit{de Oratore} 1.214)
\label{adjWithPer2}
\end{exe}

\begin{exe}
\ex
\gll nam sicut, quod apud Catonem est ..., per \emph{mihi} scitum videtur ...: sic profecto se res habet\\
for just.as what in Cato.\textsc{acc.sg} is.\textsc{pres.3sg} ~ very I.\textsc{dat.sg} sensible.\textsc{nom.sg.n} seem.\textsc{pres.3sg.pass} ~ so really itself matter.\textsc{nom.sg.f} has.\textsc{pres.3sg}\\
\trans `for just as what Cato says ..., to me seems very sensible ...: so it really is' (Cicero, \textit{de Oratore} 2.271)
\label{adjWithPer3}
\end{exe}

\begin{exe}
\ex
\gll per \emph{mihi} benigne respondit\\
very I.\textsc{dat.sg} kindly answer.\textsc{imprf.3sg}\\
\trans `he answered me very kindly' (Cicero, \textit{ad Quintum Fratrem} 1.7(9).2)
\label{adjWithPer4}
\end{exe}%NOTE source may differ in newer editions

\begin{exe}
\ex
\gll quod ad me de Hermathena scribis, per \emph{mihi} gratum est\\
what.\textsc{acc.sg} at I.\textsc{acc.sg.} about Hermathena.\textsc{abl.sg} write.\textsc{pres.2sg} very I.\textsc{dat.sg} dear.\textsc{nom} be.\textsc{pres.3sg}\\
\trans `What you write to me about Hermathena I am very grateful for' (Cicero, \textit{ad Atticum} 1.4.3)
\label{adjWithPer5}
\end{exe}

\begin{exe}
\ex
\gll per \emph{mihi}, per, inquam, gratum feceris\\
very I.\textsc{dat.sg} very say.\textsc{verb.defect.1sg} dear do.\textsc{prf.2sg.subj}\\
\trans `You would, I say, make me very, very grateful' (Cicero, \textit{ad Atticum} 1.20.7)
\label{adjWithPer6}
\end{exe}

\begin{exe}
\ex
\gll pergratum \emph{mihi} feceris, spero item Scaevolae\\
very.grateful.\textsc{acc.sg} I.\textsc{dat.sg} do.\textsc{prf.2sg.subj} hope.\textsc{pres.1sg} likewise Scaevolae.\textsc{dat}\\
\trans `you would make me very grateful, and, I hope, Scaevola, too' (Cicero, \textit{Laelius de Amicitia} 16)
\label{adjWithPer7}
\end{exe}

The fact that in (\ref{adjWithPer7}) we find \emph{pergratum mihi} and not \emph{per mihi gratum} `very I.\textsc{dat.sg} grateful.\textsc{acc.sg}', as \citet{Orelli1828} would have it, serves to confirm our rule, as \emph{mihi} must have been heavily stressed because of its opposition to \emph{Scaevolae} \citep[95]{SeyffertMueller1876}. The other cases in which \emph{per} undergoes \isi{tmesis} will be discussed later, except (\ref{adjWithPer8}) and (\ref{adjWithPer9}), in both of which, moreover, a particle\is{particles} requiring second position has caused the \isi{tmesis}.

\begin{exe}
\ex
\gll ista sunt pergrata per\emph{que} iucunda\\
that.\textsc{nom.pl.n} are.\textsc{pres.3pl} very.pleasing.\textsc{acc.pl.n} very.and delightful.\textsc{acc.pl.n}\\
\trans `that is very pleasing and very delightful' (Cicero, \textit{de Oratore} 1.205)
\label{adjWithPer8}
\end{exe}

\begin{exe}
\ex
\gll per \emph{enim} magni aestimo\\
very namely great.\textsc{gen.sg.n} appraise.\textsc{pres.1sg}\\
\trans `for I very highly appraise' (Cicero, \textit{ad Atticum} 10.1.1)
\label{adjWithPer9}
\end{exe}

b) Of the pronoun \emph{quicumque} \citep[489]{NeueWagener1892} and its dependents (whose \isi{tmesis} in cases like (\ref{quicumque1})--(\ref{quicumque6}) and in the examples given by Neue from Gellius and Appuleius, as well as in (\ref{quicumque7}), is of a very special type): (\ref{quicumque8})--(\ref{quicumque17}), and following this (\ref{quicumque18}).

\begin{exe}
\ex
\gll quod \emph{iudicium} cunque subierat\\
what.\textsc{acc.sg.n} trial.\textsc{acc.sg.n} ever enter.\textsc{pstprf.3sg}\\
\trans `whatever trial they (=everyone) had faced' (Cicero, \textit{pro Sestio} 68)
\label{quicumque1}
\end{exe}

\begin{exe}
\ex
\gll qua \emph{re} cunque\\
what.\textsc{abl.sg.f} matter.\textsc{abl.sg.f} ever\\
\trans `because of whatever matter' (Cicero, \textit{de Divinatione} 2.7)
\label{quicumque2}
\end{exe}

\begin{exe}
\ex
\gll quae \emph{loca} cunque\\
what.\textsc{acc.pl.n} place.\textsc{acc.pl.n} ever\\
\trans `(to) all places' (Lucretius 4.867)
\label{quicumque3}
\end{exe}%NOTE source may differ in newer editions, e.g. 4.870 in Loeb

\begin{exe}
\ex
\gll qua \emph{de} \emph{causa} cunque\\
what.\textsc{abl.sg.f} for reason.\textsc{abl.sg.f} ever\\
\trans `for whatever reason' (Lucretius 6.85)
\label{quicumque4}
\end{exe}

\begin{exe}
\ex
\gll quae \emph{semina} cunque\\
what.\textsc{acc.pl.n} seed.\textsc{acc.pl.n} ever\\
\trans `all seeds whatsoever' (Lucretius 6.867)
\label{quicumque5}
\end{exe}

\begin{exe}
\ex
\gll quam \emph{rem} cunque\\
what.\textsc{acc.sg.f} matter.\textsc{acc.sg.f} ever\\
\trans `whatever' (Horace, \textit{Odes} 1.6.3)
\label{quicumque6}
\end{exe}

\begin{exe}
\ex
\gll quod \emph{ad} cunque legis genus\\
what.\textsc{acc.sg.n} to ever law.\textsc{gen.sg} type.\textsc{nom.sg}\\
\trans `to whatever type of law' (Cicero, \textit{de Legibus} 4.26)
\label{quicumque7}
\end{exe}

\begin{exe}
\ex
\gll quam \emph{se} cunque in partem dedisset\\
what.\textsc{acc.sg.f} himself.\textsc{refl.acc} ever in division.\textsc{acc.sg.f} give.\textsc{pstprf.3sg.subj}\\
\trans `whatever side he held on to' (Cicero, \textit{de Oratore} 3.60)
\label{quicumque8}
\end{exe}

\begin{exe}
\ex
\gll quo ea \emph{me} cunque ducet\\
where it.\textsc{nom.sg.f} I.\textsc{acc.sg.} ever lead.\textsc{fut.3sg}\\
\trans `wherever it will lead me' (Cicero, \textit{Tusculanae Disputationes} 2.15)
\label{quicumque9}
\end{exe}

\begin{exe}
\ex
\gll quo \emph{te} cunque verteris\\
where you.\textsc{acc.sg} ever turn.\textsc{prf.2sg.subj}\\
\trans `wherever you turn' (Cicero, \textit{de Divinatione} 2.149)
\label{quicumque10}
\end{exe}

\begin{exe}
\ex
\gll quae \emph{me} cunque vocant terrae\\
what.\textsc{nom.pl.f} I.\textsc{acc.sg} ever call.\textsc{pres.3pl} earth.\textsc{nom.pl.f}\\
\trans `whichever lands summon me' (Virgil, \textit{Aeneid} 1.610)
\label{quicumque11}
\end{exe}

\begin{exe}
\ex
\gll quo \emph{te} cunque lacus miserantem incommoda nostra fonte tenet\\
in.what.\textsc{abl.sg.n} you.\textsc{acc.sg} ever lake.\textsc{nom.sg.m} lament.\textsc{ptcp.pres.acc} trouble.\textsc{acc.pl.n} our.\textsc{acc.pl.n} spring.\textsc{abl.sg.m} hold.\textsc{pres.3sg}\\
\trans `in whatever spring the lake keeps you lamenting our troubles' (Virgil, \textit{Aeneid} 8.74)
\label{quicumque12}
\end{exe}

\begin{exe}
\ex
\gll qui \emph{te} cunque manent isto certamine casus\\
what.\textsc{nom.pl.m} you.\textsc{acc.sg} ever await.\textsc{pres.pl} that.\textsc{abl.sg.n} contest.\textsc{abl.sg.n} calamity.\textsc{nom.pl.m}\\
\trans `whatever calamities await you in that contest' (Virgil, \textit{Aeneid} 12.61)
\label{quicumque13}
\end{exe}

\begin{exe}
\ex
\gll quo \emph{nos} cunque feret melior Fortuna parente\\
where we.\textsc{acc.pl} ever take.\textsc{fut.3sg} good.\textsc{nom.sg.f.comparative} Fortuna.\textsc{nom.sg.f} father.\textsc{abl.sg.m}\\
\trans `wherever Fortune, who is better than my father, will take us' (Horace, \textit{Odes} 1.7.25)
\label{quicumque14}
\end{exe}

\begin{exe}
\ex
\gll quae \emph{te} cunque domat Venus\\
what.\textsc{nom.sg.f} you.\textsc{acc.sg} ever tame.\textsc{pres.3sg} Venus.\textsc{nom.sg.f}\\
\trans `whatever Venus tames you' (Horace, \textit{Odes} 1.27.14)
\label{quicumque15}
\end{exe}

\begin{exe}
\ex
\gll delicias legit qui \emph{tibi} cunque meas\\
delight.\textsc{acc.pl.f} read.\textsc{prf.3sg} who.\textsc{nom.sg} you.\textsc{dat.sg} ever my.\textsc{acc.pl.f}\\
\trans `whoever read to you my cheerful verse' (Ovid, \textit{Tristia} 2.78)
\label{quicumque16}
\end{exe}

\begin{exe}
\ex
\gll nomen quod \emph{tibi} cunque datur\\
name.\textsc{nom.sg.n} what.\textsc{nom.sg.n} you.\textsc{dat.sg} ever give.\textsc{pres.3sg.pass}\\
\trans `whatever name is given to you' (Martial 2.61.6)
\label{quicumque17}
\end{exe}

\begin{exe}
\ex
\gll quae \emph{meo} quomque animo lubitum est facere\\
what.\textsc{nom.pl.n} my.\textsc{dat.sg.m} ever heart.\textsc{dat.sg.m} please.\textsc{ptcp.prf.n} be.\textsc{pres.3sg} do.\textsc{inf.pres}\\
\trans `to do whatever pleased my heart' (Terence, \textit{Andria} 263)
\label{quicumque18}
\end{exe}

Other than in these examples and in the others that will be discussed below because of enclisis,\is{enclitics} we only have Lucretius 6.1002, Horace 1.9.14, 1.16.2, and \textit{Satires} 2.5.51, in which various words occur in between. (Cf. (\ref{quicumque19}).) In these examples we can safely recognize poetic\is{poetry} license.

\begin{exe}
\ex
\gll garrulus hunc quando \emph{consumet} cunque\\
talkative.\textsc{nom.sg.m} this.\textsc{acc.sg.m} at.what.time take.completely.\textsc{fut.3sg} ever\\
\trans `a chatterbox will devour him at some time or other' (Horace, \textit{Satires} 1.9.33)
\label{quicumque19}
\end{exe}

c) Of the adverb\is{adverbs} \emph{quomodo} `in what way': \hyperlink{p408}{\emph{[p408]}} (\ref{quomodo1})--(\ref{quomodo4}). Cf. (\ref{quomodo5}) and (\ref{quomodo6}). More below; separation by fully stressed words does not seem to be found. Cicero, \textit{de Lege agraria} 1.25 \emph{quo uno modo} `in the only way' is a special case.

\begin{exe}
\ex
\gll necesse est, quo tu \emph{me} modo voles esse, ita esse, mater\\
necessary be.\textsc{pres.3sg} how you.\textsc{nom.sg} I.\textsc{acc.sg} way.\textsc{abl.sg.m} want.\textsc{fut.2sg} be.\textsc{inf} so be.\textsc{inf} mother.\textsc{voc.sg}\\
\trans `it is necessary to be however you want me to be, mother' (Plautus, \textit{Cistellaria} 1.1.47)
\label{quomodo1}
\end{exe}

\begin{exe}
\ex
\gll quo \emph{te} modo iactaris\\
in.what.\textsc{abl.sg.n} you.\textsc{acc.sg} way.\textsc{abl.sg.m} throw.\textsc{imprf.2sg.subj}\\
\trans `how you would show off' (Cicero, \textit{pro Roscio Amerino} 89)
\label{quomodo2}
\end{exe}

\begin{exe}
\ex
\gll quo \emph{te} modo ad tuam intemperantiam innovasti\\
in.what.\textsc{abl.sg.m} you.\textsc{acc.sg} way.\textsc{abl.sg.m} to your.\textsc{acc.sg.f} intemperance.\textsc{acc.sg.f} renew.\textsc{prf.2sg}\\
\trans `how you returned to your extravagance' (Cicero, \textit{in Pisonem} 89)
\label{quomodo3}
\end{exe}

\begin{exe}
\ex
\gll quo \emph{te} nunc modo appelem\\
in.what.\textsc{abl.sg.m} you.\textsc{acc.sg} now way.\textsc{abl.sg.m} speak.to.\textsc{pres.1sg.subj}\\
\trans `how shall I address you' (Cicero, \textit{pro Scauro} 50)
\label{quomodo4}
\end{exe}

\begin{exe}
\ex
\gll quonam \emph{se} modo defendet\\
in.what\textsc{abl.sg.m} \textsc{encl.interr} way.\textsc{abl.sg.m} defend.\textsc{fut.3sg}\\
\trans `how will he defend himself (...)?' (Cicero, \textit{pro Rabirio Postumo} 19)
\label{quomodo5}
\end{exe}%NOTE Loeb edition has "defendent"

\begin{exe}
\ex
\gll quo cunque igitur \emph{te} modo ...\\
in.what.\textsc{abl.sg.m} ever therefore you.\textsc{acc.sg} way.\textsc{abl.sg.m} ~\\
\trans `therefore, in whatever manner (...) you' (Cicero, \textit{pro Scauro} 50)
\label{quomodo6}
\end{exe}%NOTE Loeb edition "quocumque"

Thirdly, the separation of preposition\is{prepositions} and governed case in the well-known request formula\is{request formula|(} should be mentioned: (\ref{separationPrepCase1})--(\ref{separationPrepCase9}). (In (\ref{separationPrepCase8}) and (\ref{separationPrepCase9}) the verb of requesting is elided.)\is{ellipsis} The \emph{per} `very', which the pronoun \emph{te} (\textsc{2sg}) or \emph{vos}/\emph{vobis} (\textsc{2pl}) is attached to, is thus always at the beginning of the clause.

\begin{exe}
\ex
\gll per \emph{te} ere obsecro deos immortales\\
by you.\textsc{acc.sg} master.\textsc{voc.sg.m} implore.\textsc{pres.1sg} god.\textsc{acc.pl.m} immortal.\textsc{acc.pl.m}\\
\trans `Master, I implore you by the immortal gods' (Plautus, \textit{Bacchides} 905)
\label{separationPrepCase1}
\end{exe}

\begin{exe}
\ex
\gll per ego \emph{vobis} deos atque homines dico\\
by there you.\textsc{dat.pl} god.\textsc{acc.pl.m} and man.\textsc{acc.pl.m} speak.\textsc{pres.1sg}\\
\trans `I assign to you by the gods and men' (Plautus, \textit{Menaechmi} 990)
\label{separationPrepCase2}
\end{exe}%NOTE Wackernagel has "eo", Loeb edition "ego"

\begin{exe}
\ex
\gll per \emph{te} deos oro et nostram amicitiam, Chremes\\
by you.\textsc{acc.sg} god.\textsc{acc.pl.m} pray.\textsc{pres.1sg} and our.\textsc{acc.sg.f} friendship.\textsc{acc.sg.f} Chremes.\textsc{voc.sg.m}\\
\trans `I beg you, Chremes, by the gods and our friendship' (Terence, \textit{Andria} 538)
\label{separationPrepCase3}
\end{exe}

\begin{exe}
\ex
\gll per ego \emph{te} deos oro\\
by I.\textsc{nom.sg} you.\textsc{acc.sg} god.\textsc{acc.pl.m} pray.\textsc{pres.1sg}\\
\trans `by the gods I beg you' (Terence, \textit{Andria} 834)
\label{separationPrepCase4}
\end{exe}

\begin{exe}
\ex
\gll per \emph{te} dulcissima furta perque tuos oculos per geniumque rogo\\
by you.\textsc{acc.sg} sweet.\textsc{elative.acc.pl.n} theft.\textsc{acc.pl.n} by-and your.\textsc{acc.pl.m} eye.\textsc{acc.pl.m} by spirit.\textsc{acc.sg.m} ask.\textsc{pres.1sg}\\
\trans `by the sweetest thefts, your eyes, and your Genius I beg' (Tibullus 3.11.7 (=4.5.7))
\label{separationPrepCase5}
\end{exe}

\begin{exe}
\ex
\gll per ego \emph{te}, inquit, fili, quaecunque iura iungunt liberos parentibus, precor quaesoque\\
by I.\textsc{nom.sg} you.\textsc{acc.sg} say.\textsc{prf.3sg} son.\textsc{voc.sg.m} whatever.\textsc{acc.pl.n} right.\textsc{acc.pl.n} attach.\textsc{pres.3pl} child.\textsc{acc.pl.m} parent.\textsc{dat.pl.m} pray.\textsc{pres.1sg.pass} beg.\textsc{pres.1sg}-and\\
\trans `{``}by whatever rights connect children to their parents,'' he said, ``I beg and entreat you, son''' (Livius 23.9.2)
\label{separationPrepCase6}
\end{exe}

\begin{exe}
\ex
\gll per ego \emph{vos} decora maiorum ... oro et obtestor\\ 
by I.\textsc{nom.sg} you.\textsc{acc.pl} ornament.\textsc{acc.pl.n} ancestors.\textsc{gen.pl.m} ~ beg.\textsc{pres.1sg} and beseech.\textsc{pres.1sg.pass}\\
\trans `by the dignity of your forbears ... I beg and beseech you' (Curtius 5.8.16)
\label{separationPrepCase7}
\end{exe}%NOTE maiores is comparative of magnus 'great' as a plural noun meaning 'ancestors'
%NOTE script has "ora", must be "oro"

\begin{exe}
\ex
\gll per \emph{te} quod fecimus una perdidimusque nefas ... ades\\
by you.\textsc{abl.sg} \textsc{rel.acc.sg.n} do.\textsc{prf.1pl} at.once squander.\textsc{prf.1pl}-and sin.\textsc{n.indecl} ~ be.present.\textsc{imp.pres.sg}\\
\trans `by the sin which we commited and squandered together with you ..., come' (Lucan 10.370)
\label{separationPrepCase8}
\end{exe}

\begin{exe}
\ex
\gll per \emph{vos} culta diu Rutulae primordia gentis ..., conservate pios\\
by you.\textsc{nom.pl} cared.for.\textsc{ptcp.prf.acc.pl.n} long Rutulian.\textsc{gen.sg.f} beginning.\textsc{acc.pl.n} race.\textsc{gen.sg.f} ~ preserve.\textsc{imp.pres.pl} pious.\textsc{acc.pl.m}\\
\trans `by the long-worshipped beginnings of the Rutulian race (...), leave the pious unharmed' (Silius 1.658)
\label{separationPrepCase9}
\end{exe}\is{request formula|)}

Fourthly, the examples of separation of less tightly linked word groups are given here which have been cited by the aforementioned Latinists as evidence for Cicero's tendency to insert the unstressed pronoun after the first word: (\ref{lessLinkedGroups1})--(\ref{lessLinkedGroups6}).

\begin{exe}
\ex
\gll his autem de rebus sol \emph{me} ille admonuit\\
this.\textsc{abl.pl.g} but about matter.\textsc{dat.pl.f} sun.\textsc{nom.sg.m} I.\textsc{acc.sg} that.\textsc{nom.sg.m} warn.\textsc{prf.3sg}\\
\trans `but that sun warned me about these things' (Cicero, \textit{de Oratore} 309)
\label{lessLinkedGroups1}
\end{exe}

\begin{exe}
\ex
\gll populus \emph{se} Romanus erexit\\
people.\textsc{nom.sg.m} itself.\textsc{refl.acc.sg} Roman.\textsc{nom.sg.m} set.up.\textsc{prf.3sg}\\
\trans `a Roman people rose' (Cicero, \textit{Brutus} 12)
\label{lessLinkedGroups2}
\end{exe}%NOTE must be Romanu-s

\begin{exe}
\ex
\gll sentiebam, non \emph{te} id sciscitari\\
feel.\textsc{imprf.1sg} not you.\textsc{acc.sg} that.\textsc{acc.sg.n} examine.\textsc{inf.pres}\\
\trans `I supposed that you did not inquire' (Cicero, \textit{de Oratore} 52)
\label{lessLinkedGroups3}
\end{exe}

\begin{exe}
\ex
\gll in agros \emph{se} possessionesque contulit\\
to estate.\textsc{acc.pl.m} itself.\textsc{refl.acc.sg} possessions.\textsc{acc.pl.}-and turn.to.\textsc{prf.3sg}\\
\trans `turned itself to country estates' (Cicero, \textit{de Officiis} 1.151)
\label{lessLinkedGroups4}
\end{exe}%NOTE mercatura ('trade') is the subject

\begin{exe}
\ex
\gll idque eo \emph{mihi} magis est cordi\\
this.\textsc{nom.sg.n}-and because.of.this.\textsc{abl.sg.n} I.\textsc{dat.sg} more be.\textsc{pres.3sg} heart.\textsc{dat.sg}\\
\trans `and this lies more at my heart for the reason (...)' (Cicero, \textit{Laelius de Amicitia} 15)
\label{lessLinkedGroups5}
\end{exe}

\begin{exe}
\ex
\gll ut aliquis \emph{nos} deus ex hac hominum frequentia tolleret\\
that some.\textsc{nom.sg.m} we.\textsc{acc.pl} god.\textsc{nom.sg.m} from this.\textsc{abl.sg.f} of.people.\textsc{gen.pl.m} crowd.\textsc{abl.sg.f} take.away.\textsc{imprf.}\\
\trans `that some god removes us from this crowd of people' (Cicero, \textit{Laelius de Amicitia} 87)
\label{lessLinkedGroups6}
\end{exe}

Fifthly, we can adduce some cases in which a pronoun belonging jointly to two clausal constituents is inserted into the first (see \citealp{SeyffertMueller1876} on \textit{Laelius de Amicitia} XX.72): (\ref{pronounInserted1})--(\ref{pronounInserted3}).

\begin{exe}
\ex
\gll sed item etiam illa vidi, neque \emph{te} consilium civilis belli ita gerendi nec copias Cn.~Pompeii ... probare\\
but likewise too that.\textsc{acc.pl.n} see.\textsc{prf.1sg} neither you.\textsc{acc.sg} plan.\textsc{acc.sg.n} civil.\textsc{gen.sg.n} war.\textsc{gen.sg.n} in.this.manner wage.\textsc{gerundium.gen.sg.n} nor troops.\textsc{acc.pl.f} of.Gnaeus.Pompeius.\textsc{gen.sg.m} ~ approve.of.\textsc{inf.pres}\\
\trans `but at the same time I also saw that you did not approve of the plan to wage a civil war in this manner nor of Gnaeus Pompeius' troops' (Cicero, \textit{Epistulae} 4.7.2)
\label{pronounInserted1}
\end{exe}%NOTE Loeb has idem <instead of item

\begin{exe}
\ex
\gll nec \emph{se} comitem illius furoris, sed ducem praebuit\\
and.not himself.\textsc{refl.acc.sg} comrade.\textsc{acc.sg.m} that.\textsc{gen.sg.m} fury.\textsc{gen.sg.f} but leader.\textsc{acc.sg.m} give.\textsc{prf.3sg}\\
\trans `and he did not present himself as a comrade of that person's fury, but as the leader' (Cicero, \textit{Laelius de Amicitia} 37)
\label{pronounInserted2}
\end{exe}%NOTE must be illius instead of illus

\begin{exe}
\ex
\gll neque \emph{te} provinciae neque leges neque di penates civem patiuntur\\
neither you.\textsc{acc.sg} province.\textsc{nom.pl.f} nor law.\textsc{nom.pl.f} nor god.\textsc{nom.pl.m} Penates.\textsc{nom.pl.m} citizen.\textsc{acc.sg.m} bear.\textsc{pres.3pl}\\
\trans `Neither the provinces nor the laws nor the tutelary gods tolerate you as a citizen' (Sallust, \textit{Oratio Philippi} 16)
\label{pronounInserted3}
\end{exe}

\hyperlink{p409}{\emph{[p409]}} (The same, but without influence of the positional rule, is found in (\ref{pronounInserted4}), on which \citet[XX]{Paul1889}, however, remarks: ``word order shows that \emph{se} should be deleted''.)

\begin{exe}
\ex
\gll quae omnia et \emph{se} tulisse patienter et esse laturum\\
which.\textsc{acc.pl.n.dem} all.\textsc{acc.pl.n} both he.\textsc{acc.sg.m.refl} endure.\textsc{inf.prf} patiently and be.\textsc{inf.pres} endure.\textsc{ptcp.fut.acc.sg.m}\\
\trans `(he said) that he had endured all this patiently and would further endure' (Caesar, \textit{de Bello Civili} 1.85.11)
\label{pronounInserted4}
\end{exe}

Previous research provides examples of a different use of the pronoun by the comic playwrights. Specifically, I would like to emphasize \citeauthor{Kaempf1886}'s \citeyearpar[31, 36]{Kaempf1886} observation that in the vast majority of cases the personal pronouns attach immediately to question words and clause-introducing conjunctions (cf. e.g. in \citealp[243]{Bach1891} the juxtaposition of the cases with \emph{quid tibi} `what you.\textsc{dat}' etc. with the accusative-governing\is{accusative} verbal substantives in -\emph{tio}), as well as to affirmative \isi{particles} such as \emph{hercle} `by Hercules', \emph{pol}, \emph{edepol} `by Pollux', etc. \citep[40]{Kaempf1886}, which, as will be discussed later, assume either the first or the second position in the clause. Also very worthy of note is \citeauthor{Kaempf1886}'s \citeyearpar{Kaempf1886} remark, coupled to an observation of Kellerhoff's,\ia{Kellerhoff, Eduard} that in the very numerous cases in which \isi{negation} is verse-initial a personal pronoun is attached to it wherever it is found.

Most informative of all, however, is Langen's \citeyearpar[426ff.]{Langen1857}\label{godformulae} evidence concerning the assertion, wish and curse formulae with \emph{di} `gods', \emph{di deaeque} `gods and goddesses', or the name of a specific god as subject and a \isi{subjunctive} (or \isi{future}) verb as predicate. (Cf. also \citealp[77f.]{Kellerhoff1891}). When \emph{di}, \emph{di deaeque} or the god's name in question is clause-initial, it is immediately followed by any \isi{accusative} or \isi{dative} personal pronoun \emph{me} (\textsc{1sg}), \emph{te}, \emph{tibi} (\textsc{2sg}) governed by the verb, and by the more rarely occurring \emph{vos}, \emph{vobis} (\textsc{2pl}), (\emph{istum} `that',) \emph{istunc}, \emph{istaec} `this', and \emph{illum} `that, him'. When the subject consists of multiple words, it is true that the pronoun is occasionally found immediately after the whole constituent, as in (\ref{pronounInserted5}). Cf. (\ref{formulaeWithDi1}), which \citet{Langen1857} and, following him, \citet{Goetz1878}, emends\is{emendation} to \emph{di me hercle omnes}, and (\ref{formulaeWithDi2}) (emended\is{emendation} to \emph{me omnes} by \citealp{Ritschl1852}).

\begin{exe}
\ex
\gll Hercules dique \emph{istam} perdant\\
Hercules.\textsc{nom.sg.m} god.\textsc{nom.pl.m}
that.\textsc{acc.sg.f} destroy.\textsc{pres.3pl.subj}\\
\trans `Hercules and the gods shall destroy her' (Plautus, \textit{Casina} 275)
\label{pronounInserted5}
\end{exe}

\begin{exe}
\ex
\gll di hercle omnes \emph{me} adiuvant, augent, amant\\
god.\textsc{nom.sg.m} Hercules.\textsc{abl.sg.m} all.\textsc{nom.pl.m} me help.\textsc{pres.3pl} bless.\textsc{pres.3pl} love.\textsc{pres.3pl}\\
\trans `all gods, by Hercules, help me, bless me, love me' (Plautus, \textit{Epidicus} 192)
\label{formulaeWithDi1}
\end{exe}

\begin{exe}
\ex
\gll di deaeque omnes \emph{me} pessumis exemplis interficant\\
god.\textsc{nom.pl.m} goddess.\textsc{nom.pl.f}-and all.\textsc{nom.pl.m} me bad.\textsc{abl.pl.n.superlative} manner.\textsc{abl.pl.n} kill.\textsc{pres.3pl.subj}\\
\trans `May all the gods and goddesses kill me in the worst ways' (Plautus, \textit{Mostellaria} 192)
\label{formulaeWithDi2}
\end{exe}

\hyperlink{p410}{\emph{[p410]}} More often the pronoun is inserted after the first word, as in (\ref{formulaeWithDi3}) (likewise in Plautus, \textit{Captivi} 868, \textit{Curculio} 317, \textit{Rudens} 1112) and (\ref{formulaeWithDi4})--(\ref{formulaeWithDi7}).

\begin{exe}
\ex
\gll Iuppiter \emph{te} dique perdant\\
Jupiter.\textsc{nom.sg.m} you.\textsc{acc.sg} god.\textsc{nom.pl.m}-and destroy.\textsc{pres.3pl.subj}\\
\trans `May Jupiter and the gods destroy you' (Plautus, \textit{Aulularia} 658)
\label{formulaeWithDi3}
\end{exe}

\begin{exe}
\ex
\gll Diespiter \emph{te} dique, Ergasile, perdant\\
Jupiter.\textsc{nom.sg.m} you.\textsc{acc.sg} god.\textsc{nom.pl.m}-and Ergasilus.\textsc{voc.sg.m} destroy.\textsc{pres.3pl.subj}\\
\trans `May Jupiter and the gods confound you, Ergasile' (Plautus, \textit{Captivi} 919)
\label{formulaeWithDi4}
\end{exe}

\begin{exe}
\ex
\gll di \emph{te} deaeque ament\\
god.\textsc{nom.pl.m} you.\textsc{acc.sg} goddess.\textsc{nom.pl.f}-and love.\textsc{pres.3pl.subj}\\
\trans `May the gods and goddesses love you' (Plautus, \textit{Pseudolus} 271)
\label{formulaeWithDi5}
\end{exe}

\begin{exe}
\ex
\gll di \emph{te} deaeque omnes faxint cum istoc omine\\
god.\textsc{nom.pl.m} you.\textsc{acc.sg} goddess.\textsc{nom.pl.f}-and all.\textsc{nom.pl.m} do.\textsc{prf.3pl.subj} with that.of.yours.\textsc{abl.sg.n} foreboding.\textsc{abl.sg.n}\\
\trans `may all the gods and goddesses confound you with your forebodings' (Plautus, \textit{Mostellaria} 463)
\label{formulaeWithDi6}
\end{exe}%NOTE faxint is unclear. Loeb edition has deleted it in the Latin text (tr. "With that omen may all the gods and goddesses—", others leave it in the Latin but don't translate it

\begin{exe}
\ex
\gll di \emph{te} deaeque omnes funditus perdant, senex\\
god.\textsc{nom.pl.m} you.\textsc{acc.sg} goddess.\textsc{nom.pl.f}-and all.\textsc{nom.pl.m} completely destroy.\textsc{pres.3pl.subj} old.man.\textsc{voc.sg.m}\\
\trans `may all the gods and goddesses completely destroy you, old man' (Plautus, \textit{Mostellaria} 684)
\label{formulaeWithDi7}
\end{exe}

Similarly with attributive groups! (\ref{formulaeWithDi8}) and (\ref{formulaeWithDi9}) illustrate. The example in (\ref{formulaeWithDi10}) takes an intermediate position; similarly \textit{Mostellaria} 192 according to \citet{Ritschl1852} (see (\ref{formulaeWithDi2}) above).

\begin{exe}
\ex
\gll di \emph{illum} omnes perdant\\
god.\textsc{nom.pl.m} that.\textsc{acc.sg.m} all.\textsc{nom.pl.m} destroy.\textsc{pres.3pl.subj}\\
\trans `may all the gods destroy him' (Plautus, \textit{Menaechmi} 596)
\label{formulaeWithDi8}
\end{exe}

\begin{exe}
\ex
\gll di \emph{tibi} omnes id quod es dignus duint\\
god.\textsc{nom.pl.m} you.\textsc{dat.sg} all.\textsc{nom.pl.m} this.\textsc{acc.sg.n} because be.\textsc{pres.2sg} worthy.\textsc{nom.sg.m} do.\textsc{pres.3pl.subj}\\
\trans `may all the gods do this to you because you deserve it' (Terence, \textit{Phormio} 519)
\label{formulaeWithDi9}
\end{exe}

\begin{exe}
\ex
\gll di deaeque \emph{me} omnes perdant\\
god.\textsc{nom.pl.m} goddess.\textsc{nom.pl.f}-and I.\textsc{acc.sg} all.\textsc{nom.pl.m} destroy.\textsc{pres.3pl.subj}\\
\trans `may all the gods and goddesses destroy me' (Plautus, \textit{Persa} 292)
\label{formulaeWithDi10}
\end{exe}

This alone is remarkable; however, what is particularly important is that, whenever an \emph{ita} `thus', \emph{itaque} `therefore', \emph{ut} (complementizer), \emph{utinam} `if only', \emph{hercle} `Hercules', \emph{qui} (relative pronoun)\is{relative pronouns} or \emph{at} `but, yet, whereas' is clause-initial, we find the pronoun preceding the nominal subject, and not, for instance, \emph{di} `gods' or the god's name and then the pronoun. Where \emph{at} and \emph{ita} are together, the pronoun follows both in (\ref{formulaeWithDi11}) and (\ref{formulaeWithDi12}), but intervenes between the two \isi{particles} in (\ref{formulaeWithDi13}), where for the sake of the metre I would rather emend\is{emendation} \emph{me} to \emph{med} than follow the reordering proposed by more recent editors, \emph{at ita me}.

\begin{exe}
\ex
\gll at ita \emph{me} machaera et clypeus bene iuvent\\
but as.truly.as I.\textsc{acc.sg} sword.\textsc{nom.sg.f} and shield.\textsc{nom.sg.m} well help.\textsc{pres.3pl.subj}\\
\trans `but as truly as sword and shield may help me well' (Plautus, \textit{Curculio} 574)
\label{formulaeWithDi11}
\end{exe} 

\begin{exe}
\ex
\gll at ita \emph{me} di deaeque omnes ament\\
but as.truly.as I.\textsc{acc.sg} god.\textsc{nom.pl.m} goddess.\textsc{nom.pl.f}-and all.\textsc{nom.pl.m} love.\textsc{pres.sg.subj}\\
\trans `but as truly as all the gods and goddesses may love me' (Plautus, \textit{Miles gloriosus} 501)
\label{formulaeWithDi12}
\end{exe}

\begin{exe}
\ex
\gll at \emph{me} ita dei servent\\
but I.\textsc{acc.sg} as.truly.as god.\textsc{nom.pl.m} serve.\textsc{pres.3pl.subj}\\
\trans `but as truly as the gods may serve me' (Plautus, \textit{Poenulus} 1258)
\label{formulaeWithDi13}
\end{exe}

The pronoun also precedes the subject \emph{di} after initial words other than the \isi{particles} mentioned: (\ref{formulaeWithDi14})--(\ref{formulaeWithDi16}) etc. In (\ref{formulaeWithDi16}), \emph{malum quod} = \ili{Greek} \emph{kakón ti} `bad.\textsc{acc} something'.

\begin{exe}
\ex
\gll si \emph{te} di ament\\
if you.\textsc{acc.sg} god.\textsc{nom.pl.m} love.\textsc{pres.pl.subj}\\
\trans `if the gods are to love you' (Plautus, \textit{Pseudolus} 430)
\label{formulaeWithDi14}
\end{exe}

\begin{exe}
\ex
\gll tantum \emph{tibi} boni di immortales duent\\
as.much.\textsc{correlative} you.\textsc{dat.sg} good.\textsc{gen.sg.n} god.\textsc{nom.pl.m} immortal.\textsc{nom.pl.m} give.\textsc{pres.pl.subj}\\
\trans `may the immortal gods give you as much good' (Plautus, \textit{Pseudolus} 936)
\label{formulaeWithDi15}
\end{exe}

\begin{exe}
\ex
\gll malum quod \emph{isti} di deaeque omnes duint\\
misfortune.\textsc{acc.sg.n} that he.\textsc{dat.sg} god.\textsc{nom.pl.m} goddess.\textsc{nom.pl.f}-and all.\textsc{nom.pl.m} give.\textsc{pres.3pl.subj}\\
\trans `may all the gods and goddesses give him misfortune' (Plautus, \textit{Mostellaria} 455)
\label{formulaeWithDi16}
\end{exe}

\citet{Langen1857}, followed by \citet[78]{Kellerhoff1891} and \citet[70]{Schoell1890} in his edition, wants to reorder the countervailing example (\ref{formulaeWithDi17}) to \emph{te di}, while \citet{Seyffert1874} seeks to mitigate the damage by punctuating it as \emph{``di te perdant''}.

\begin{exe}
\ex
\gll quin hercle di \emph{te} perdant\\
but Hercules.\textsc{abl.sg.m} god.\textsc{nom.pl.m} you.\textsc{acc.sg} destroy.\textsc{pres.3pl.subj}\\
\trans `but, by Hercules, may the gods destroy you' (Plautus, \textit{Casina} 609)
\label{formulaeWithDi17}
\end{exe}

Langen's \citeyearpar{Langen1857} observation also continues to be valid for classical Latin -- at least insofar as, in assertion formulae containing \emph{ita} and \emph{sic} `thus', the pronoun \emph{me} (\textsc{1sg}), \emph{te} (\textsc{2sg}) or \emph{mihi} (\textsc{1sg}) almost always immediately follows these words. With \emph{ita}: (\ref{assertionFormulae1})--(\ref{assertionFormulae9}).

\begin{exe}
\ex
\gll ita \emph{mihi} deos velim propitios\\
so I.\textsc{dat.sg} god.\textsc{acc.pl.m} wish.\textsc{pres.1sg.subj} favourable.\textsc{acc.pl.m}\\
\trans `I so want the gods to be favourable’ (Cicero, \textit{Divinatio in Caecilium} 41)
\label{assertionFormulae1}
\end{exe}

\begin{exe}
\ex
\gll ita \emph{mihi} meam voluntatem – vestra populique Romani existimatio comprobet\\
as.truly.as I.\textsc{dat.sg} my.\textsc{acc.sg.f} will.\textsc{acc.sg.f} ~ your.\textsc{nom.sg.f} and.people.\textsc{gen.sg.m} Roman.\textsc{gen.sg.m} judgment.\textsc{nom.sg.f} approve.\textsc{pres.3sg.subj}\\
\trans `as truly as your and the the Roman people's judgment may approve of my wishes’ (Cicero, \textit{in Verrem} 5.35)
\label{assertionFormulae2}
\end{exe}

\begin{exe}
\ex
\gll ita \emph{mihi} omnis deos propitios velim\\
so I.\textsc{dat.sg} god.\textsc{acc.pl.m} all.\textsc{acc.pl.m} favourable.\textsc{acc.pl.m} wish.\textsc{pres.1sg.subj}\\
\trans `as truly as I want all gods to be favourable’ (Cicero, \textit{in Verrem} 5.37)
\label{assertionFormulae3}
\end{exe}

\begin{exe}
\ex
\gll nam tecum esse, ita \emph{mihi} commoda omnia quae opto contingant, ut vehementer velim\\
for with.you.\textsc{abl.sg} be.\textsc{pres.inf} as I.\textsc{dat.sg} convenience.\textsc{nom.pl.n} all.\textsc{nom.pl.n} \textsc{nom.pl.n} wish.\textsc{pres.1sg} touch.\textsc{pres.3pl.subj} as eagerly wish.\textsc{pres.1sg.subj}\\
\trans `because as truly as I attain all conveniences I wish for I eagerly want to be with you’ (Cicero, \textit{Epistulae} 5.21.1)
\label{assertionFormulae4}
\end{exe}

\hyperlink{p411}{\emph{[p411]}}

\begin{exe}
\ex
\gll saepe, ita \emph{me} di iuvent, te ... desideravi\\
often so I.\textsc{acc.sg} god.\textsc{nom.pl.m} help.\textsc{pres.3pl.subj} you.\textsc{acc.sg} ~ desire.\textsc{prf.1sg}\\
\trans `I often called for you, so the gods help me’ (Cicero, \textit{ad Atticum} 1.16.1)
\label{assertionFormulae5}
\end{exe}

\begin{exe}
\ex
\gll iurat ``ita \emph{sibi} parentis honores consequi liceat''\\
swear.\textsc{pres.3sg} so himself.\textsc{dat.sg} father.\textsc{gen.sg} honour.\textsc{acc.pl} follow.\textsc{pres.inf} be.allowed.\textsc{pres.3sg.subj.impers}\\
\trans `he swears ``as true as it shall be granted him to follow the honors of his father''' (Cicero, \textit{ad Atticum} 15.16.3)
\label{assertionFormulae6}
\end{exe}%Loeb has the clever "Swears ‘by his hopes of rising to his father’s honours,’" avoiding a rather bulky translation of "liceat". Again, I translated "ita" with "as true/ly as" in order to be able to introduce an actual subclause.

\begin{exe}
\ex
\gll at marite, ita \emph{me} iuvent caelites, nihilo minus pulcer es\\ 
but husband.\textsc{voc.sg.m} so I.\textsc{acc.sg} help.\textsc{pres.3pl.subj} heavenly.\textsc{nom.pl.m} nothing.\textsc{abl.sg} less beautiful.\textsc{nom.sg.m} be.\textsc{pres.2sg}\\
\trans `but, husband, so the gods help me, you are not less beautiful’ (Catullus 61.196)
\label{assertionFormulae7}
\end{exe}%"at, marite, ..." corr. Scaliger, to be found in Loeb and Teubner edition of the text (Wackernagel has "ad marite")

\begin{exe}
\ex
\gll non (ita \emph{me} divi) vera gemunt (iuerint)\\
not so I.\textsc{acc.sg} god.\textsc{nom.pl.m} true.\textsc{acc.pl.n} lament.\textsc{pres.3pl} go.\textsc{prf.3pl.subj}\\
\trans `they, so may the gods help me, do not lament false things’ (Catullus 66.18)
\label{assertionFormulae8}
\end{exe}%Loeb translates it as "so may (!) the gods help me". Again, the background idea for ita is "as truly as the gods help me, ..."

\begin{exe}
\ex
\gll non, ita \emph{me} di ament, quicquam referre putavi\\
not so I.\textsc{acc.sg} god.\textsc{nom.pl.m} love.\textsc{pres.3pl.subj} anything.\textsc{acc.sg.n} bring.\textsc{pres.inf} believe.\textsc{prf.3sg}\\
\trans `I did not, so may the gods love me, think it mattered’ (Catullus 97.1)
\label{assertionFormulae9}
\end{exe}

This position is retained even when another particle\is{particles} is inserted before \emph{ita}, as in (\ref{assertionFormulae10}) and (\ref{assertionFormulae11}).

\begin{exe}
\ex
\gll nam ita \emph{mihi} salva republica vobiscum perfrui liceat, ut ...\\
because just.as I.\textsc{dat.sg} unhar\textit{Medea}\textsc{abl.sg.f} state.\textsc{abl.sg.f} with.you.\textsc{abl.pl} enjoy.\textsc{pres.inf} be.allowed.\textsc{pres.3sg.subj.impers} so.also.\textsc{correlative} ~\\
\trans `as truly as I shall be allowed to enjoy the saved Republic, so also ...' (Cicero, \textit{in Catilinam} 4.11)
\label{assertionFormulae10}
\end{exe}

\begin{exe}
\ex
\gll tamen ita \emph{te} victorem complectar ..., ut ...\\
yet as you.\textsc{acc.sg} victor.\textsc{acc.sg.m} embrace.\textsc{pres.1sg.subj} ~ so.also.\textsc{correlative} ~\\
\trans `nevertheless, as truly as I might embrace you victorious, ...' (Cicero, \textit{Epistulae} 10.12.1)
\label{assertionFormulae11}
\end{exe}

(\ref{assertionFormulae12}) and (\ref{assertionFormulae13}) do not, of course, come into consideration.

\begin{exe}
\ex
\gll ita ab imminentibus malis respublica \emph{me} adiuvante liberetur\\
so from threaten.\textsc{ptcp.pres.abl.pl} calamity.\textsc{abl.pl.n} state.\textsc{nom.sg.f} I.\textsc{abl.sg} help.\textsc{abl.sg} free.\textsc{pres.3sg.pass.subj}\\
\trans `so may the state be freed from menacing calamities with my help' (Plancus, \textit{ad Ciceronem epistulae} 10.9.2)
\label{assertionFormulae12}
\end{exe}

\begin{exe}
\ex
\gll ita genium \emph{meum} propitium habeam\\
just.as genius.\textsc{acc.sg.m} my.\textsc{acc.sg.m} favourable.\textsc{acc.sg.m} have.\textsc{pres.1sg.subj}\\
\trans `as truly as I wish to have my genius to be favourable' (Petronius, \textit{Satyricon} 74)
\label{assertionFormulae13}
\end{exe}

With \emph{sic} `so': (\ref{assertionFormulae14})--(\ref{assertionFormulae23}). Cf. (\ref{assertionFormulae24}), in which the pronoun is not in second position but is still immediately after \emph{sic}.

\begin{exe}
\ex
\gll sic \emph{tibi} bonus ex tua pons libidine fiat\\
so you.\textsc{dat.sg} good.\textsc{nom.sg.m} according.to your.\textsc{abl.sg.f} bridge.\textsc{nom.sg.m} desire.\textsc{abl.sg.f} happen.\textsc{pres.3sg.subj}\\
\trans `so may you receive a good bridge as you desire ’ (Catullus 17.5)
\label{assertionFormulae14}
\end{exe}

\begin{exe}
\ex
\gll sic \emph{tibi}, cum fluctus supterlabere Sicanos, Doris amara suam non intermisceat undam\\
so you.\textsc{dat.sg} when wave.\textsc{acc.pl.m} glide.over.\textsc{imprf.2sg.subj} of.Sicily.\textsc{acc.pl.m} Doris.\textsc{nom.sg.f} bitter.\textsc{nom.sg.f} her.\textsc{acc.sg.f} not intermix.\textsc{pres.3sg.subj} wave.\textsc{acc.sg.f}\\
\trans `so may not bitter Doris intermix her wave with you, when you glide over the waves of Sicily’ (Virgil, \textit{Eclogues} 10.4)
\label{assertionFormulae15}
\end{exe}

\begin{exe}
\ex
\gll sic \emph{te} diva potens Cypri ... regat\\
so you.\textsc{acc.sg} goddess.\textsc{nom.sg.f} strong.\textsc{nom.sg.f} Cyprus.\textsc{gen.sg.m} ~ lead.aright.\textsc{pres.3sg.subj}\\
\trans `may the goddess ruling over Cyprus guide you’ (Horace, \textit{Odes} 1.3.1)
\label{assertionFormulae16}
\end{exe}

\begin{exe}
\ex
\gll sic \emph{tibi} sint intonsi Phoebe capilli\\
so you.\textsc{dat.sg} be.\textsc{pres.3pl.subj} unshaven.\textsc{nom.pl.m} Phoebus.\textsc{voc.sg.m} hair.\textsc{nom.pl.m}\\
\trans `so may your hair be unshaven, Phoebus’ (Tibullus 2.5.121)
\label{assertionFormulae17}
\end{exe}

\begin{exe}
\ex
\gll sic \emph{mihi} te referas levis\\
so I.\textsc{dat.sg} you.\textsc{acc.sg} bring.back.\textsc{pres.2sg.subj} light.\textsc{voc.sg.m}\\
\trans `as truly as I wish that you come back to me, fickle one ’ (Propertius 1.18.11)
\label{assertionFormulae18}
\end{exe}%"sic" here corresponds to "ut" [sic mihi te referas, levis, ut non altera nostro limine formosos intulit ulla pedes], cf. Georges: sic c) (wie οὕτως) bei Versicherungen, Schwüren und Wünschen (bei Dichtern, in Prosa gew. ita), sic mit Konjunktiv ... ut mit Indikat., so wahr ich wünsche, daß ... so gewiß

\begin{exe}
\ex
\gll sic \emph{tibi} sint dominae Lygdame dempta iuga\\
so you.\textsc{dat.sg} be.\textsc{pres.3pl.subj} mistress.\textsc{gen.sg.f} Lygdamus.\textsc{voc.sg.m} remove.\textsc{ptcp.prf.nom.pl.n} yoke.\textsc{nom.pl.n}\\
\trans `may the mistress's yokes be removed from you, Lygdamus’ (Propertius 3.6.2)
\label{assertionFormulae19}
\end{exe}%the idea behind the dative "tibi" is "for your benefit"

\begin{exe}
\ex
\gll sic \emph{tibi} secretis agilis dea saltibus adsit\\
so you.\textsc{dat.sg} remote.\textsc{abl.pl.m} nimble.\textsc{nom.sg.f} goddess.\textsc{nom.sg.f} mountain.valley.\textsc{abl.pl.m} help.\textsc{pres.3sg.subj}\\
\trans `so may the nimble goddess help you in remote mountain valleys’ (Ovid, \textit{Heroides} 4.169)
\label{assertionFormulae20}
\end{exe}

\begin{exe}
\ex
\gll sic \emph{tibi} dent nymphae\\
so you.\textsc{dat.sg} give.\textsc{pres.3pl.subj} nymph.\textsc{nom.pl.f}\\
\trans `so may the nymphs give you’ (Ovid, \textit{Heroides} 4.173)
\label{assertionFormulae21}
\end{exe}

\begin{exe}
\ex
\gll sic \emph{tibi} nec vernum nascentia frigus adurat poma\\
so you.\textsc{dat.sg} not vernal.\textsc{nom.sg.n} growing.\textsc{ptcp.pres.acc.pl.n} cold.\textsc{nom.sg.n} burn.\textsc{pres.3sg.subj} fruit.\textsc{acc.pl.n}\\
\trans `so may not the coldness of spring burn your growing fruit’ (Ovid, \textit{Metamorphoses} 14.763)
\label{assertionFormulae22}
\end{exe}

\begin{exe}
\ex
\gll presta mi sinceru(m): sic \emph{te} amet qui custodit ortu(m) Venus\\
give.\textsc{pres.imp} I.\textsc{dat.sg} pure.\textsc{acc.sg} so you.\textsc{acc.sg} love.\textsc{pres.3sg.subj} who.\textsc{nom.sg.f} keep.\textsc{pres.3sg} garden\textsc{acc.sg} Venus.\textsc{nom.sg.f}\\
\trans `Give me pure {[}wine{]} and Venus who tends the garden will love you’ (Corpus Inscriptionum Latinarum 4.2776)
\label{assertionFormulae23}
\end{exe}\is{inscriptions}

\begin{exe}
\ex
\gll perpetuo liceat sic \emph{tibi} ponte frui\\
forever be.allowed so you.\textsc{dat.sg} bridge.\textsc{abl.sg.m} enjoy.\textsc{inf.pres}\\
\trans `so may you forever be allowed to enjoy (your) bridge’ (Martial 7.93.8)
\label{assertionFormulae24}
\end{exe}

With \isi{ablative} absolutes ((\ref{assertionFormulae25})) and possessives ((\ref{assertionFormulae26})) we have no right to expect the rule to hold (though cf. (\ref{assertionFormulae27})).

\begin{exe}
\ex
\gll sic ... Venusinae plectantur silvae \emph{te} sospite\\
so ~ of.Venusia.\textsc{nom.pl.f} beat.\textsc{pres.3pl.pass.subj} forest.\textsc{nom.pl.f} you.\textsc{abl.sg} unharmed.\textsc{abl.sg}\\
\trans `so ... may the woods of Venusia be beaten, while you are safe’ (Horace, \textit{Odes} 1.28.25)
\label{assertionFormulae25}
\end{exe}

\begin{exe}
\ex
\gll rogo, sic peculium \emph{tuum} fruniscaris\\
ask.\textsc{pres.1sg} so property.\textsc{acc.sg.n} your.\textsc{acc.sg.n} enjoy.\textsc{pres.2sg.subj.pass}\\
\trans `I ask you, as truly as you wish to enjoy your property’ (Petronius 65)
\label{assertionFormulae26}
\end{exe}

\begin{exe}
\ex
\gll sic \emph{tua} Cyrneas fugiant examina taxos\\
as your.\textsc{nom.pl.n} Corsican.\textsc{acc.pl.f} flee.\textsc{pres.3pl.subj} swarm.\textsc{nom.pl.n} yew.\textsc{acc.pl.f}\\
\trans `as truly as your swarms wish to flee the yews of Corsica’ (Virgil, \textit{Eclogues} 9.30)
\label{assertionFormulae27}
\end{exe}

We also cannot treat (\ref{assertionFormulae28}) as a violation of the rule. On the other hand, (\ref{assertionFormulae29}) and (\ref{assertionFormulae30}) are striking.

\begin{exe}
\ex
\gll (sic habites terras et \emph{te} desideret aether) sic ad pacta tibi sidera tardus eas\\
so dwell.\textsc{pres.3sg.subj} earth.\textsc{acc.pl.f} and you.\textsc{abl.sg} want.\textsc{pres.3sg.subj} ether so to promise.\textsc{ptcp.prf.acc.pl.n} you.\textsc{dat.sg} star.\textsc{acc.pl.n} late.\textsc{nom.sg.m} go.\textsc{pres.2sg.subj}\\
\trans `(so may you dwell on earth and heaven long for you) so may you go late to the stars promised to you’ (Ovid, \textit{Tristia} 5.2.51f.)
\label{assertionFormulae28}
\end{exe}

\begin{exe}
\ex
\gll sic umbrosa \emph{tibi} contingant tecta Priape\\
so shady.\textsc{nom.pl.n} you.\textsc{dat.sg} touch.\textsc{pres.3pl} shelter.\textsc{acc.pl.n} Priapus.\textsc{voc.sg.m}\\
\trans `so may you attain a shelter full of shade, Priapus’ (Tibullus 1.4.1)
\label{assertionFormulae29}
\end{exe}

\begin{exe}
\ex
\gll sic felicem \emph{me} videas\\
so fortunate.\textsc{acc.sg} I.\textsc{acc.sg} see.\textsc{pres.2sg.subj}\\
\trans `as truly as I wish that you see me fortunate’ (Petronius 61)
\label{assertionFormulae30}
\end{exe}%not happy with the translation; context: Oro te, sic felicem me videas, narra illud quod tibi usu venit – Loeb has: Do please, to make me happy, tell us of your adventure. I struggle to find any hint of this "so that" in lexicon entries for "sic"

The words \emph{mehercule} `by Hercules', \emph{mediusfidius} `by God', and \emph{mecastor} `by Castor' are well known to have developed out of expressions like the ones discussed. This also seems to me to explain their position. In the vast majority of examples they are in second position in the \hyperlink{p412}{\emph{[p412]}} clause. This is true exceptionlessly for the first two in Cicero's speeches. For \emph{mehercule}, cf. also Terence, \textit{Eunuchus} 416, Cicero, \textit{de Oratore} 2.7, \textit{Epistulae} 2.11.4, \textit{ad Atticum} 10.13.1, 16.15.3, Caesar in Cicero, \textit{ad Atticum} 9.7c 1, Caelius in Cicero, \textit{Epistulae} 8.2.1, Plancus ibid. 10.11.3, and Pliny, \textit{Epistulae} 6.30; for \emph{mediusfidius} also Cicero, \textit{Epistulae} 5.21.1, \textit{Tusculanae Disputationes} 1.74 (\ref{interjection1}), Sallust, \textit{Catiline} 35.2, Livius 5.6.1, 22.59.17, Seneca, \textit{Suasoriae} 6.5, and Pliny, \textit{Epistulae} 4.3.5.

\begin{exe}
\ex
\gll ne ille \emph{mediusfidius} vir sapiens\\
indeed that.\textsc{nom.sg.m} by.God man.\textsc{nom.sg.m} wise.\textsc{nom.sg.m}\\ 
\trans `indeed, by God, that wise man' (Cicero, \textit{Tusculanae Disputationes} 1.74)
\label{interjection1}
\end{exe}

Particularly probative is the not uncommon insertion of an assertion particle\is{particles} that belongs to a whole period after the first word of the clause: \emph{si mehercule} `if by Hercules' in Cicero, \textit{pro Caecina} 64, \textit{Catiline} 2.16, \textit{pro Scauro} Fragment 10 \citep[246]{Mueller1886}, and Sallust, \textit{Catiline} 52.35; \emph{quanto mehercule} in Sallust, \textit{Historiae, Oratio Philippi} 17; \emph{si mediusfidius} `if by God' in Cicero, \textit{pro Sulla} 83, \textit{pro Plancio} 9, and Livius 5.6.1 and 22.59.17. The examples in which one of these two \isi{particles} assumes a later position in the clause are significantly less numerous (\emph{mehercule}: Terence, \textit{Eunuchus} 67, Catullus 38.2, Phaedrus 3.5.4, and Pliny, \textit{Epistulae} 3.1.1; \emph{mediusfidius}: Cato in Gellius 10.14.3, Cicero, \textit{ad Atticum} 15.8A.2, Quintilian 5.12.17). Examples (\ref{interjection2}) and (\ref{interjection3}) are remarkable due to the very unusual placement of the particle.\is{particles}

\begin{exe}
\ex
\gll \emph{mediusfidius}, ne tu emisti locum preclarum\\
by.God indeed you.\textsc{nom} buy.\textsc{prf.2sg} place.\textsc{acc.sg.m} excellent.\textsc{acc.sg.m}\\
\trans `by God, you have indeed bought an excellent place' (Cicero, \textit{ad Atticum} 4.4b.2)
\label{interjection2}
\end{exe}

\begin{exe}
\ex
\gll \emph{mehercule} etiam adventu nostro reviviscunt\\
by.Hercules also arrival.\textsc{abl.sg.m} our.\textsc{abl.sg.m} revive.\textsc{3pl.pres}\\
\trans `by Hercules, they also come back to life by our arrival' (Cicero, \textit{ad Atticum} 5, 16, 3)
\label{interjection3}
\end{exe}

As regards preclassical \emph{mecastor}, (\ref{interjection4}) and also (\ref{interjection5}) obey the rule, while (\ref{interjection6}) contradicts it.

\begin{exe}
\ex
\gll noenum \emph{mecastor} quid ego ero dicam meo ... queo comminisci\\
not by.Castor what.\textsc{nom.sg.n} I.\textsc{nom} master.\textsc{dat.sg.m} say.\textsc{1sg.pres.subj} my.\textsc{dat.sg.m} ~ be.able.\textsc{1sg.pres} invent.\textsc{inf.pres}\\
\trans `by Castor, I cannot think of what I should say [has happened] to my master' (Plautus, \textit{Aulularia} 67)
\label{interjection4}
\end{exe}%Difficult to translate as both the verb (evenisse) causing the dative ero meo and the genitive complementing quid aren't included in the extract. In my eyes, particularly the verb seems crucial for understanding the sentence. The whole sentence could be translated as follows: By Castor, I cannot think of what unfortunate thing or madness I should say has happened to my master. Latin: Noenum mecastor, quid ego ero dicam meo malae rei evenisse quamve insaniam queo comminisci.

\begin{exe}
\ex
\gll ne istuc \emph{mecastor} iam patrem accersam meum\\
indeed to.this.place by.Castor now father.\textsc{acc.sg.m} summon.\textsc{1sg.fut} my.\textsc{acc.sg.m}\\
\trans `by Castor, I will indeed summon my father to this place now' (Plautus, \textit{Menaechmi} 734)
\label{interjection5}
\end{exe}%I am not really sure whether summon can be used this way. Another possible translation would be `make my father come'.

\begin{exe}
\ex
\gll novi hominem haud malum \emph{mecastor}\\
know.\textsc{1sg.prf.} man.\textsc{acc.sg.m} not.at.all bad.\textsc{acc.sg.m} by.Castor\\
\trans `I know the man. By Castor, he is not bad at all' (Plautus, \textit{Aulularia} 172)
\label{interjection6}
\end{exe} 

The difference between \isi{vocative} \emph{mehercule} `by Hercules' etc. on the one hand and \emph{hercule} `by Hercules' etc. on the other (see below) is that the forms with \emph{me-} are excluded from the first position in the clause (leaving aside the isolated examples in Cicero, \textit{ad Atticum} 4.4b.2 and 5.16.3). Therefore, the tendency for these forms to occur in second position should not be attributed to that observed for \emph{hercule} etc., but rather to the enclitic\is{enclitics} nature of \emph{me} (\textsc{1sg}).


\section{Latin: more personal pronouns and indefinites}\is{indefinites}

Let's move on to other forms! If the \isi{vocative} \emph{mī} `my' is really identical to the \emph{moi} (\textsc{1sg}) in \ili{Greek} \emph{téknon moi} `my child' etc. \hyperlink{p413}{\emph{[p413]}} (see above p\pageref{wilamowitz}), as \citet[819]{Brugmann1890} assumes, then this word's property of enclisis\is{enclitics} must already have been lost in prehistoric times, since as early as Plautus it is found in clause-initial position. It is not inconceivable that preposing of \emph{mi} before the noun it belongs to occurred in clauses in which the \isi{vocative} was not in first position, and in which, therefore, \emph{mi} had to be placed before the \isi{vocative} in order to be in the clausal second position it required.

We can be more confident that the oblique\is{oblique (case)} cases of \emph{is} `he, it, this, that', just like Attic\il{Greek, Attic} \emph{autoû} `here, there' and \ili{Sanskrit}'s enclitic\is{enclitics} \emph{asmāi} `this.\textsc{dat}', behaved the same as \emph{me} (\textsc{1sg}) and \emph{te} (\textsc{2sg}). We therefore read e.g. (\ref{enclitic1}) like (\ref{enclitic2}) (see example (\ref{lessLinkedGroups2}) above). We also find enclitic\is{enclitics} positioning with the demonstrative pronouns \emph{iste} `that (\textsc{prox})' and \emph{ille} `that (\textsc{dist})' in the clauses of wishing and cursing discussed above on pp\pageref{godformulae}ff.

\begin{exe}
\ex
\gll quam \emph{id} recte fecerim\\
to.what.degree it.\textsc{acc.sg.n} rightly do.\textsc{prf.1sg.subj}\\
\trans `to what degree I acted correctly' (Cicero, \textit{Laelius de Amicitia} 10) 
\label{enclitic1}
\end{exe}

\begin{exe}
\ex
\gll populus \emph{se} Romanus erexit\\
people.\textsc{nom.sg.m} himself.\textsc{acc.sg.m} Roman.\textsc{nom.sg.m} erexit.\textsc{prf.3sg}\\ 
\trans `the Roman people rose' (Cicero, \textit{Brutus} 12) 
\label{enclitic2}
\end{exe}

Some readers might have noticed, moreover, that in the examples where \emph{me} (\textsc{1sg}) or \emph{te} (\textsc{2sg}) disrupts a constituent because of its position it is often preceded by \emph{ego}: for instance, (\ref{enclitic3}) and (\ref{enclitic4}). In addition, we have (\ref{enclitic5}). Also the \isi{nominative} of \emph{is}, \emph{ea}, \emph{id}: (\ref{enclitic6}).

\begin{exe}
\ex
\gll per \emph{ego} \emph{vobis} deos ... dico\\
through I.\textsc{nom.sg} you.\textsc{dat.pl} god.\textsc{acc.pl.m} ~ say.\textsc{pres.1sg}\\ 
\trans `I order you in the name of the gods' (Plautus, \textit{Menaechmi} 990) 
\label{enclitic3}
\end{exe}

\begin{exe}
\ex
\gll per \emph{ego} \emph{te} deos oro\\
through I.\textsc{nom.sg.} you.\textsc{acc.sg} god.\textsc{acc.pl.m} say.\textsc{pres.1sg}\\ 
\trans `I beg you in the name of the gods' (Terence, \textit{Andria} 834) 
\label{enclitic4}
\end{exe}

\begin{exe}
\ex
\gll quo \emph{tu} \emph{me} modo voles esse\\
who.\textsc{abl.sg.m} you.\textsc{nom.sg} I.\textsc{acc.sg} way.\textsc{abl.sg.m} want.\textsc{fut.2sg} be.\textsc{inf.pres}\\
\trans `I'll behave the way you want me to' (Plautus, \textit{Cistellaria} 1.1.47) 
\label{enclitic5}
\end{exe}%expressing the future tense explicitly in the translation may not be necessary?

\begin{exe}
\ex
\gll quo \emph{ea} \emph{me} cunque duxit\\
where she.\textsc{nom.sg.f} I.\textsc{acc.sg} ever lead.\textsc{prf.3sg}\\
\trans `wherever it (=Reason) led me' (Cicero, \textit{Tusculanae Disputationes} 2.15) 
\label{enclitic6}
\end{exe}%in the context of the text, ea should be it instead of she (ratio/reason). may be confusing?

It is indisputable that in such cases \emph{ego}, \emph{tu} and \emph{ea} are also enclitic,\is{enclitics} and reminiscent of the enclisis\is{enclitics} of German \emph{er} (\textsc{3sg.nom.m}), \emph{sie} (\textsc{3sg.nom.f}), \emph{es} (\textsc{3sg.nom.n}) in subordinate\is{subordination} clauses as well as in inverted and interrogative\is{interrogatives} main clauses. In this way we can also explain examples like (\ref{enclitic7})--(\ref{enclitic12}). Furthermore, the \emph{ego} (\textsc{1sg.nom}) or \emph{tu} (\textsc{2sg.nom}) that immediately follows the verb, like \ili{Greek} \emph{egō} (\textsc{1sg.nom}) in the same position, should certainly also be considered enclitic.\is{enclitics}

\begin{exe}
\ex
\gll quantulum \emph{id} cunque est\\
how.little.\textsc{nom.sg.n} it.\textsc{nom.sg.n} ever is.\textsc{pres.3sg}\\
\trans `how little soever it is' (Cicero, \textit{de Oratore} 2.97) 
\label{enclitic7}
\end{exe}

\begin{exe}
\ex
\gll quale \emph{id} cunque est\\
of.what.quality.\textsc{nom.sg.n} it.\textsc{nom.sg.n} ever is.\textsc{pres.3sg}\\
\trans `of what quality soever it is' (Cicero, \textit{de Natura Deorum} 2.76) 
\label{enclitic8}
\end{exe}

\begin{exe}
\ex
\gll quonam igitur \emph{haec} modo gesta sunt\\
which.\textsc{abl.sg.m} then this.\textsc{nom.pl.n} way.\textsc{abl.sg.m} happen.\textsc{ptcl.prf.nom.pl.n} be.\textsc{pres.3pl}\\
\trans `Which way, then, did these things happen?' (Cicero, \textit{pro Cluentio} 66) 
\label{enclitic9}
\end{exe}

\begin{exe}
\ex
\gll cuius \emph{haec} cunque modi videntur\\
what.\textsc{gen.sg.m} this.\textsc{nom.pl.n} ever kind.\textsc{gen.sg.m} seem.\textsc{pres.3sg.pas}\\
\trans `of whatever kind these things seem to be' (Sallust, \textit{Catiline} 52.10) 
\label{enclitic10}
\end{exe}

\begin{exe}
\ex
\gll ne aut \emph{ille} alserit aut ceciderit\\
that.not either that.\textsc{nom.sg.m} suffer.from.cold.\textsc{perf.3sg.subj} or fall.\textsc{perf.3sg.subj}\\
\trans `that he has neither suffered from cold nor fallen' (Terence, \textit{Adelphoe} 36) 
\label{enclitic11}
\end{exe}

\begin{exe}
\ex
\gll quonam \emph{ille} modo cum regno distractus esset\\
what.\textsc{abl.sg.m} that.\textsc{nom.sg.m} way.\textsc{abl.sg.m} with realm.\textsc{abl.sg.n} tear.apart.\textsc{ptcp.perf.m} be.\textsc{imperf.3sg.subj}\\
\trans `how he would have been torn apart with his realm' (Cicero, \textit{pro rege Deiotaro} 15) 
\label{enclitic12}
\end{exe}\is{personal pronouns|)}

With indefinites,\is{indefinites|(} Latin holds more firmly to the old rule than \ili{Greek}, and this has been \hyperlink{p414}{\emph{[p414]}} recognized for a long time, although the formulation has not been entirely correct. If we jointly consider the linguistic usage of the ancient inscriptions,\is{inscriptions} the commentaries of Caesar and the speeches of Cicero, following the index of \emph{Corpus Inscriptionum Latinum} (CIL) I \citep{MommsenHenzen1863} and the lexica of \citet{Meusel1887} and \citet{Merguet1884}, the result is that \emph{quis} `who/what.\textsc{m/f}' and \emph{quid} `who/what.\textsc{n}' in the overwhelming majority of examples attach to clause-introducing words such as \emph{ē}- `out/away', \emph{nē} `no/not', \emph{dum nē} `provided-that not', \emph{num} `whether', the relativizer \emph{qui} and its forms, \emph{quo} `where/why', \emph{cum} `when/because/although', \emph{quamvis} `however/although', and \emph{neque} `and not'. Of course, -\emph{ve} (in \emph{neve} `and not', \emph{sive} `or/but if' etc.) takes precedence, and more rarely pronominal\is{pronouns} \isi{enclitics} (only once in Caesar): (\ref{indef.enclitic1}) and (\ref{indef.enclitic2}). Cf. (\ref{indef.enclitic3})--(\ref{indef.enclitic5}).

\begin{exe}
\ex
\gll ne\emph{ve} \emph{eorum} \emph{quod} saeptum clausumve habeto\\
and.not this.\textsc{gen.pl.m} who.\textsc{acc.sg.n} fence.in.\textsc{ptcl.acc.sg.n} close-or.\textsc{ptcl.acc.sg.n} or have.\textsc{imp.sg.fut}\\
\trans `and you shall not possess their belongings which have been fenced in or locked' (CIL I.206.71) 
\label{indef.enclitic1}
\end{exe}

\begin{exe}
\ex
\gll dum \emph{eorum} \emph{quid} faciet\\
while this.\textsc{gen.pl.m} something.\textsc{acc.sg.n} make.\textsc{fut.3sg}\\
\trans `while he will practice any of these' (CIL I.206.94, I.206.104) 
\label{indef.enclitic2}
\end{exe}

\begin{exe}
\ex
\gll qui ita \emph{quid} confessus erit\\
who.\textsc{nom.sg.m} so something.\textsc{acc.sg.n} confess.\textsc{ptcp.prf.m} be.\textsc{fut.3sg}\\
\trans `who will have confessed something this way' (CIL I.205.II.15, 41) 
\label{indef.enclitic3}
\end{exe}%The CIL entry listed seems really strange. I'm not sure whether that is indeed how it‘s meant or whether I just don't get it. As I'm not really familiar with the CIL and its structure, I honestly have no idea.

\begin{exe}
\ex
\gll quod eum \emph{quis} ignoret\\
because he.\textsc{acc.sg.m} someone.\textsc{nom.sg.m} not.know.\textsc{pres.3sg.subj}\\ 
\trans `because someone may not know him' (Cicero, \textit{in Verrem} 5.168) 
\label{indef.enclitic4}
\end{exe}

\begin{exe}
\ex
\gll qui horum \emph{quid} acerbissime crudelissimeque fecerat, is et vir et civis optimus habebatur\\
who.\textsc{nom.sg.m} this.\textsc{gen.pl.m} something.\textsc{nom.sg.n} most.violently most.cruelly-and do.\textsc{pstprf3.sg.} he.\textsc{nom.sg.m} and man.\textsc{nom.sg.m} and citizen.\textsc{nom.sg.m} best.\textsc{nom.sg.m} have.\textsc{imprf.3.sg.pass}\\
\trans `Whoever of them had done something very violent and cruel, was considered both the best man and citizen.' (Caesar, \textit{de Bello Civili} 3.32.3) 
\label{indef.enclitic5}
\end{exe}


In these texts, the indefinite is found in true clause-internal position only after \emph{alius} `else/other' and \emph{ali}-, and here it must be pointed out that we generally find \emph{si quis alius} `if someone else' and \emph{ne quis alius} `not anyone else', not \emph{si alius quis} or \emph{ne alius quis}. In addition, in Cicero's speeches we always find \emph{quis} and \emph{quid} separated from the relativizer by one or two other words in relative clauses\is{relative clauses} (7--8 examples). Also, (\ref{indef.enclitic6}) is a striking example.

\begin{exe}
\ex
\gll nei \emph{quis} in ieis locis inve ieis porticibus quid inaedificatum immolitumve habeto\\
not someone.\textsc{nom.sg.m} in this.\textsc{abl.pl.m} place.\textsc{abl.pl.m} in-or this.\textsc{abl.pl.f} portico.\textsc{abl.pl.f} something.\textsc{acc.sg.n} build.\textsc{acc.sg.n} erect-or.\textsc{acc.sg.n} have.\textsc{imp.sg.fut}\\
\trans `no one shall have anything built or erected in these places or in these porticoes' (CIL I.206.70) 
\label{indef.enclitic6}
\end{exe}

The same is true of the related indefinite \isi{adverbs}, in particular \emph{quando} `when', and is also true for indefinites in general, as far as I can tell, in the other archaic and classical texts. Admittedly, it is sometimes necessary to emancipate oneself from modern editors in order to recognize this. \citeauthor{Goetz1876}, for example, quite happily inserts enclitic\is{enclitics} \emph{quid} `what' in the middle of a clause and at the same time verse-initially in Plautus, \textit{Mercator} 774\label{mercator} (see his edition, \citealp[92]{RitschlGoetz1884}, as well as \citealp[244]{Goetz1876}), although the manuscripts provide the correct \emph{si quid}! Of course it is possible to dig up isolated exceptions, but the \emph{quid} in (\ref{indef.exclamative}), for example, should probably be read as an exclamation, hence orthotonic.

\begin{exe}
\ex
\gll tum captivorum \emph{quid} ducunt secum\\
then captive.\textsc{gen.pl.m} what.\textsc{acc.sg.n} bring.\textsc{pres.3pl} with.themselves.\textsc{abl.pl}\\
\trans `then, they are bringing so many captives with them' (Plautus, \textit{Epidicus} 210) 
\label{indef.exclamative}
\end{exe}

In view of this rigidity of the positional rule, neither the \isi{anastrophe} in (\ref{indef.enclitic7}) (cf. \citealp{SeyffertMueller1876} on this example) nor the frequent \hyperlink{p415}{\emph{[p415]}} separation of the attributive indefinite from its noun -- reminiscent of the examples adduced above for \ili{Greek} on pp\pageref{greekseparation}ff. -- should be surprising, e.g. (\ref{indef.enclitic8}), (\ref{indef.enclitic9}), etc., etc. I should also mention, only in passing, that \ili{Oscan} and \ili{Umbrian} \emph{pis}, \emph{pid} and \emph{pis}, \emph{pir} `who/what' usually immediately follow \emph{svaì}, \emph{svae} and \emph{sve, so} `if' in the manuscripts.

\begin{exe}
\ex
\gll si \emph{quos} inter societas aut est aut fuit\\
if someone.\textsc{acc.pl.m} between partnership.\textsc{nom.sg.f} or be.\textsc{pres.3sg} or was.\textsc{prf.3sg}\\
\trans `if there is or was partnership between some people' (Cicero, \textit{Laelius de Amicitia} 83) 
\label{indef.enclitic7}
\end{exe}

\begin{exe}
\ex
\gll ne \emph{qua} oriatur pecuniae cupiditas\\
that.not any.\textsc{nom.sg.f} arise.\textsc{pres.3sg.subj.} money.\textsc{gen.sg.f} envy.\textsc{nom.sg.f.}\\
\trans `that no money envy may arise' (Caesar, \textit{de Bello Gallico} 6.22.3) 
\label{indef.enclitic8}
\end{exe}

\begin{exe}
\ex
\gll ne \emph{qua} aut largitionibus aut animi confirmatione aut falsis nuntiis commutatio fieret voluntatis\\
that.not any.\textsc{nom.sg.f} or bribery.\textsc{abl.pl.f} or courage.\textsc{gen.sg.m} affirmation.\textsc{abl.sg.f} or false.\textsc{abl.pl.m} message.\textsc{abl.pl.m} change.\textsc{nom.sg.f} make.\textsc{inf.pres.pass} will.\textsc{gen.sg.f}\\
\trans `that not any change of will may take place due to bribery or encouragement or false messages' (Caesar, \textit{de Bello Civili} 1.21.1) 
\label{indef.enclitic9}
\end{exe}%**Boosting courage may not be the best way to translate confirmatione animi. Maybe encouragement may serve as a better translation.**

It is well known that \emph{quisque} `each person/anyone', deriving from enclitic\is{enclitics} \emph{quis}, is an enclitic,\is{enclitics} and that, though it occurs clause-internally more often than \emph{quis}, it is generally only found after superlatives, ordinals, \emph{unus} `one/single/alone' and \emph{suus} `his/her/its/their own', and otherwise after the first word in the clause. In the inscriptions\is{inscriptions|(} of CIL I the positional rule is fully clear: \emph{quisque} after \emph{primus} `first' in 198.46, 198.64 and 198.67, after \emph{suus} in 206.92=102, otherwise word-internally only in (\ref{quisque1}); in all other examples it is in second position, often admittedly such that the relativizer is followed first by the noun to which it belongs as an attribute and only then by \emph{quisque}, e.g. (\ref{quisque2})--(\ref{quisque4}), and with a following \isi{genitive} e.g. in (\ref{quisque5}).

\begin{exe}
\ex
\gll quamque viam h{[}ac{]} l{[}ege{]} \emph{quemque} tueri oportebit\\
each.\textsc{acc.sg.f} road.\textsc{acc.sg.f} this.\textsc{abl.sg.f} law.\textsc{abl.sg.f} each.\textsc{acc.sg.m} protect.\textsc{inf.pres.pass} be.necessary.\textsc{fut.3sg}\\
\trans `with this law, it will be necessary for everyone to protect each road' (CIL I.206.I.22) 
\label{quisque1}
\end{exe}

\begin{exe}
\ex
\gll quo die \emph{quisque} triumphabit\\
whoever.\textsc{abl.sg.m} day.\textsc{abl.sg.m} each.\textsc{nom.sg.m} win.\textsc{fut.3sg}\\
\trans `on whichever day everyone will win' (CIL I.206.I.63) 
\label{quisque2}
\end{exe}

\begin{exe}
\ex
\gll quot annos \emph{quisque} eorum habet\\
how.many year.\textsc{acc.pl.m} each.\textsc{nom.sg.m} he.\textsc{gen.pl.m} have.\textsc{pres.3sg}\\
\trans `how many years every one of them has' (CIL I.206.I.147) 
\label{quisque3}
\end{exe}%**a less literal translation of the above would be: how old they are**

\begin{exe}
\ex
\gll qua in parte urbis \emph{quisque} eorum curet\\
whoever.\textsc{abl.sg.f} in part.\textsc{abl.sg.f} city.\textsc{gen.sg.f} each.\textsc{nom.sg.m} he.\textsc{gen.pl.m} take.care.of.\textsc{pres.3sg.subj}\\
\trans `in whichever part of the city every one of them should take care of [...]' (CIL I.206.I.26) 
\label{quisque4}
\end{exe}

\begin{exe}
\ex
\gll quantum agri loci \emph{quoiusque} in populi leiberi ... datus adsignatusve est\\
how.much land.\textsc{gen.sg.m} place.\textsc{gen.sg.m} each.\textsc{gen.sg.m} in people.\textsc{gen.sg.m} free.\textsc{gen.sg.m} ~ give.\textsc{ptcl.prf.pass} assign-or.\textsc{ptcl.prf.pass} be.\textsc{pres.3sg}\\
\trans `how much land and place has been given or assigned to any free people' (CIL I.200.I.71) 
\label{quisque5}
\end{exe}

But even in these examples the preposing of \emph{quisque} before the words with which it stands in an attributive relation makes sense only from the perspective of our positional law: \emph{quisque eorum} `whichever of them' in (\ref{quisque3})--(\ref{quisque4}) (and many other such cases), \emph{quoiusque in populi leiberi} `any of the free people' in (\ref{quisque5}). And examples in which \emph{quisque} splits an attributively linked constituent through its striving to be placed near the start of the clause are not at all rare: see (\ref{quisque6})--(\ref{quisque8}).

\begin{exe}
\ex
\gll quem \emph{quisque} eorum agrum posidebit\\
whoever.\textsc{acc.sg.m} each.\textsc{nom.sg.m} he.\textsc{gen.pl.m} land.\textsc{acc.sg.m} possess.\textsc{fut.3sg}\\
\trans `whichever land each of them will possess' (CIL I.199.39)
\label{quisque6}
\end{exe}

\begin{exe}
\ex
\gll quam in \emph{quisque} decuriam ... lectus erit\\
whoever.\textsc{acc.sg.f} in each.\textsc{nom.sg.m} detachment.\textsc{acc.sg.f} ~ select.\textsc{nom.sg.m} be.\textsc{fut.3sg}\\
\trans `into whichever detachment each will be selected' (CIL I.202.I.33, I.202.I.37, I.202.I.41, I.202.II.5) 
\label{quisque7}
\end{exe}

\begin{exe}
\ex
\gll qua in \emph{quisque} decuria est\\
whoever.\textsc{abl.sg.f} each.\textsc{nom.sg.m} detachment.\textsc{abl.sg.f} be.\textsc{pres.3sg}\\
\trans `in whichever detachment each is' (CIL I.202.II.27) 
\label{quisque8}
\end{exe}

The last two examples show that in word sequences like \emph{quam in decuriam} the preposition\is{prepositions} was perceived as belonging to the relativizer. Similarly, \emph{quisque} may disrupt the connection between governing noun and \isi{genitive}, for instance, as in \emph{quantum viae} `how much of the road' in (\ref{quisque9}), and (\ref{quisque10}).

\begin{exe}
\ex
\gll quantum \emph{quoiusque} ante aedificium viae ... erit\\
how.much each.\textsc{gen.sg.m} in.front building.\textsc{acc.sg.n} road.\textsc{gen.sg.f} ~ be.\textsc{fut.3sg}\\
\trans `how much of the road will be in front of each one's building' (CIL I.206.I.39) 
\label{quisque9}
\end{exe}

\begin{exe}
\ex
\gll quod \emph{quibusque} in rebus ... iouris ... fuit\\
who.\textsc{nom.sg.n} each.\textsc{abl.pl.f} in things.\textsc{abl.pl.f} ~ law.\textsc{gen.sg.n} ~ be.\textsc{prf.3sg}\\
\trans `what of the law has been applied in all situations' (CIL I.204.II.23) 
\label{quisque10}
\end{exe}%**This translation caused me a lot of difficulties. I'm not really sure whether it's correct.**

So much for the older \hyperlink{p416}{\emph{[p416]}} inscriptions.\is{inscriptions|)} The other older literature provides similar results, including the notable \isi{tmesis} in (\ref{quisque11}). However, \emph{quisque} has generally also become able to be used orthotonically and to take clause-initial position. This is even more true for \emph{uterque} `both/each of two', whose original enclitic\is{enclitics} nature is clear and can still be seen in examples like (\ref{uterque1}). On the other hand, \emph{ubique} remained true to its origins even longer: Cicero in his speeches and Caesar always use it in its actual meaning ``in each individual place'' (``everywhere'' is written by both as \emph{omnibus locis}), but it is also always attached to a relativizer (Caesar, \textit{de Bello Civili} 2.20.8 attaches it to interrogative\is{interrogatives} \emph{quid}).

\begin{exe}
\ex
\gll quod \emph{quoique} quomque inciderit in mentem\\
who.\textsc{acc.sg.n} each.\textsc{dat.sg.m} each.\textsc{acc.sg.m} fall.into\textsc{prf.3sg.subj.} in mind.\textsc{acc.sg.f}\\
\trans `anything that may come into anyone's mind' (Terence, \textit{Heauton Timorumenos} 484) 
\label{quisque11}
\end{exe}

\begin{exe}
\ex
\gll in eo \emph{uterque} proelio potabimus\\
in this.\textsc{abl.sg.m} each.of.two.\textsc{nom.sg.m} battle.\textsc{abl.sg.n} drink.\textsc{fut.1pl}\\
\trans `both of us will drink in this battle' (Plautus, \textit{Menaechmi} 186) 
\label{uterque1}
\end{exe}

That the other class of indefinites in Latin, those beginning with \emph{u}-, were subject to the same positional rules as those beginning with velar consonants, is shown by Festus 162b.22, quite apart from the unmistakable tendency of \emph{ullus} `any', \emph{unquam} `ever' and \emph{usquam} `anywhere' to occupy second position.\is{pronouns|)}\is{indefinites|)}


\section{Latin: particles and vocatives}\is{particles|(}

Among the particles of Latin one finds some that have always been bound to the second position (\emph{que} `and', \emph{autem} `but', \emph{ne} `\textsc{neg/q}'), some that either vacillate between first and second position from the very start or are pulled hither and thither through changing usage (the affirmative particles and also \emph{enim} `truly{\slash}because' and \emph{igitur} `therefore'), and finally some for which the vacillation and freedom is even greater, like \emph{tandem} `at last'. All these particles occasionally cause the sort of tmesis\is{tmesis|(} demonstrated for the \isi{pronouns}; for example, \emph{enim} separates \emph{cunque} in (\ref{tmesis1}), and \emph{igitur} and \emph{tandem} separate \emph{quomodo} and friends, and also \emph{jusjurandum}, in (\ref{tmesis2})--(\ref{tmesis5}).

\begin{exe}
\ex
\gll qualis \emph{enim} cunque est\\
of.what.kind.\textsc{nom.sg.m} truly ever be.\textsc{pres.3sg}\\
\trans `of what kind ever it truly is' (Ovid, \textit{ex Ponto} 4.13.6)
\label{tmesis1}
\end{exe}

\begin{exe}
\ex
\gll quonam \emph{igitur} haec modo gesta sunt\\
which.\textsc{abl.sg.m} then this.\textsc{nom.pl.n} way.\textsc{abl.sg.m} happen.\textsc{ptcl.prf.nom.pl.n} be.\textsc{pres.3pl}\\ 
\trans `Which way, then, did these things happen?' (Cicero, \textit{pro Cluentio} 66)
\label{tmesis2}
\end{exe}

\begin{exe}
\ex
\gll quocunque \emph{igitur} haec modo\\
whatever.\textsc{abl.sg.m} then this.\textsc{abl.sg.m} way.\textsc{abl.sg.m}\\ 
\trans `whatever way then these things' (Cicero, \textit{pro Scauro} 50)
\label{tmesis3}
\end{exe}

\begin{exe}
\ex
\gll jus \emph{igitur} jurandum\\
law.\textsc{nom.sg.n} then swear.\textsc{grnd.nom.sg.n}\\ 
\trans `an oath then' (Cicero, \textit{de Officiis} 3.104) 
\label{tmesis4}
\end{exe}

\begin{exe}
\ex
\gll quo \emph{tandem} modo\\
which.\textsc{abl.sg.m} eventually way.\textsc{abl.sg.m}\\
\trans `which way eventually' (Cicero, \textit{in Verrem} 3.80)
\label{tmesis5}
\end{exe}

A particularly tmesis-inducing word is \emph{que} `and', which has this effect not only in cases like those given above (e.g. (\ref{tmesis6})) but also separates \isi{prepositions} from verbs ((\ref{tmesis7})--(\ref{tmesis8})) \hyperlink{p417}{\emph{[p417]}} and \isi{prepositions} from case, the latter especially when it means `if': Old Latin\il{Latin, Old} (\ref{tmesis9}) (Plautus, \textit{Trinummus} 832 with the freer word order \emph{absque foret te}).

\begin{exe}
\ex
\gll juris\emph{que} jurandi\\
law-and\textsc{gen.sg.n} swear.\textsc{grnd.gen.sg.n}\\ 
\trans `and an oath' (Cicero, \textit{pro Caelio} 54)
\label{tmesis6}
\end{exe}

\begin{exe}
\ex
\gll trans\emph{que} dato, endo\emph{que} plorato\\
across-and give.\textsc{fut.imp} in-and cry.\textsc{fut.imp}\\ 
\trans `and he will surrender, and he will call/implore' (Festus 309a.30)
\label{tmesis7}
\end{exe}

\begin{exe}
\ex
\gll dis\emph{que} tulissent\\
apart-and carry.\textsc{prf.3pl.subj}\\ 
\trans `and they have carried away' (Plautus, \textit{Trinummus} 833)
\label{tmesis8}
\end{exe}

\begin{exe}
\ex
\gll abs\emph{que} me esset; abs\emph{que} te foret; abs\emph{que} una hac foret; abs\emph{que} eo esset\\
without-and I.\textsc{abl.sg.m} be.\textsc{imprf.3sg.subj} without-and you.\textsc{abl.sg.m} be.\textsc{imprf.3sg.subj} without-and one.\textsc{abl.sg.f} this.\textsc{abl.sg.f} be.\textsc{imprf.3sg.subj} without-and he.\textsc{abl.sg.m} be.\textsc{imprf.3sg.subj}\\ 
\trans `and if it wasn't for me; and if it wasn't for you; and if it wasn't for this one thing; and if it wasn't for him'
\label{tmesis9}
\end{exe}

It is to the demerit of Latinists that they have continued to view \emph{absque} `without/apart from' as a normal preposition\is{prepositions} long after \citet{Schoemann1871} and \citet{Brugman1877} discovered the truth.\footnote{\emph{Translator's note}: Wackernagel writes Brugmann here, but this must be a slip, possibly related to the better-known Karl Brugmann.\ia{Brugmann, Karl}} For, assuming that Cicero, \textit{ad Atticum} 1.19.1 should be read as in (\ref{tmesis10}), which I do not believe \citet{Woelfflin1882} to have proven, and assuming also that the meaning `without' does not derive from an error on the part of second-century archaists but rather was native to everyday language in Cicero's time, it is of course possible that in the time between Terence and Cicero the phrase \emph{absque me esset} could first lose the verb (so that simple \emph{absque me} was used as a hypothetical\is{hypotheticals} ``without me = if I had not been there''; cf. (\ref{tmesis11}) ``without you, i.e. if you had not been there'' and (\ref{tmesis12})) and, subsequent to the loss of the verb, the hypothetical\is{hypotheticals} meaning could have disappeared and \emph{absque me} could have taken on the meaning of ``without me'' in the sense of ``as I am not there''.\is{semantic change} Very similar developments can be demonstrated for the concessive particles. (On \emph{absque} in general, see \citealp{Praun1889}.)

\begin{exe}
\ex
\gll abs\emph{que} argumento ac sententia\\
without-and content.\textsc{abl.sg.n} and purpose.\textsc{abl.sg.f}\\ 
\trans `and without any content and purpose' (Cicero, \textit{ad Atticum} 1.19.1)
\label{tmesis10}
\end{exe}

\begin{exe}
\ex
\gll abs\emph{que} te uno forsitan lingua Graeca longe anteisset, sed tu\\
without-and you.\textsc{abl.sg.m} one.\textsc{abl.sg.m} maybe language.\textsc{nom.sg.f} Greek.\textsc{nom.sg.f} long precede.\textsc{pstprf.3sg.subj} but you.\textsc{nom.sg.m}\\ 
\trans `and if it wasn't for you alone, the Greek language would have preceded but you' (Gellius 2.21.20)
\label{tmesis11}
\end{exe}

\begin{exe}
\ex
\gll abs\emph{que} te, satis superque et aetatis et laboris\\
without-and you.\textsc{abl.sg.m} enough and.more and time.\textsc{gen.sg.f} and work.\textsc{gen.sg.m}\\
\trans `and if it wasn't for you, [I would have] more than enough of both time and work' (Fronto 85.24 N)
\label{tmesis12}
\end{exe}\is{tmesis|)}

The only particles that can count as fully sound support for our positional law are those that serve not to link clauses but to qualify the specific clause or constituent they belong to. First, \emph{quidem} `in fact/indeed', which is formally distinguished from \ili{Indo-Iranian} \emph{cid} only by the addition of -\emph{em}, and essentially functionally identical to it. Like \emph{cid}, it cannot follow unstressed words, and originally especially not the verb (cf. \citealp[73]{Bartholomae1888} on \emph{cid}), and like \emph{cid} it occupies a position either after the first word of the clause (see e.g. (\ref{quidem1})) or after the \hyperlink{p418}{\emph{[p418]}} stressed word whose meaning needs to be emphasized (for instance because of a contrast), depending on its function.

\begin{exe}
\ex
\gll Tiberium \emph{quidem} Gracchum\\
Tiberius.\textsc{acc.sg.m} indeed Gracchus.\textsc{acc.sg.m}\\
\trans `indeed, Tiberius Gracchus' (Cicero, \textit{Laelius de Amicitia} 37) 
\label{quidem1}
\end{exe}

This positional alternation is particularly clear in archaic texts when it cooccurs with the assertion particles, especially with \emph{hercle} `by Hercules'. There are innumerable examples of \emph{quidem hercle} `indeed by Hercules' etc. after the first word of the clause, but we also often find \emph{hercle -- quidem}. According to \citet[64f.]{Kellerhoff1891}, some examples of the latter order can be explained through metrical license, and others are inexplicable. But without exception they show \emph{quidem} after a stressed personal\is{personal pronouns} pronoun\is{pronouns}, demonstrative, \emph{si} `if' or \emph{nunc} `now': in all these cases, \emph{quidem} is attached to the orthotonic word following \emph{hercle} etc. (Also (\ref{quidem2}), an example not found in \citealp{Kellerhoff1891}.)

\begin{exe}
\ex
\gll tam pol id \emph{quidem}\\
so by.Pollux he.\textsc{nom.sg.n} indeed\\
\trans `indeed, it [is] so' (Plautus, \textit{Bacchides} 1194) 
\label{quidem2}
\end{exe}

As well as \emph{quidem} we have \emph{quŏque}, which I believe should be identified with \ili{Sanskrit} \emph{kva ca} and therefore assigned the original meaning `wherever, however'. A word with the meaning `however' was suitable for expressing the inclusion of a referent in a statement; this also explains the archaic connection between \emph{quoque} and \emph{etiam} `also'. It is because of the word's function that it, like \emph{ge} `in fact/at least/only' and to an extent \emph{quidem}, can occupy various positions in the clause despite being an enclitic\is{enclitics} -- specifically, wherever the word is whose referent is to be designated as added. But just as \emph{ge} occasionally detaches itself from its word and removes itself to the start of the clause, following the general tendency of \isi{enclitics} (see above p\pageref{greekge}), \emph{quoque} does the same: (\ref{quoque1}) with \emph{quoque quattuor} rather than \emph{quattuor quoque}, (\ref{quoque2}) with \emph{quoque ... Iuno} rather than \emph{Iuno quoque} (cf. \citealp{Spengel1886} on this example), (\ref{quoque3}) rather than \emph{ab eo ... quoque quibus}, (\ref{quoque4}) with \emph{quoque stipem} rather than \emph{stipem quoque}, (\ref{quoque5}) with \emph{quoque illa nomina} rather than \emph{illa nomina quoque}. Likewise (\ref{quoque6}) with \emph{quoque ... Varro} rather than \emph{Varro quoque} and (\ref{quoque7}) with \emph{quoque lascivi ... Catulli} rather than \emph{lascivi Catulli quoque}.

\begin{exe}
\ex
\gll ab hoc \emph{quoque} quattuor partes urbis tribus dictae\\
from this.\textsc{abl.sg.n} also four part.\textsc{nom.pl.f} city.\textsc{gen.sg.f} tribe.\textsc{nom.pl.f} name.\textsc{ptcl.acc.pl.f}\\
\trans `from this, four parts of the city were used as names for the tribes' (Varro, \textit{de lingua Latina} 5.56)
\label{quoque1}
\end{exe}

\begin{exe}
\ex
\gll quae ideo \emph{quoque} videtur ab Latinis Iuno Lucina dicta\\
who.\textsc{nom.sg.f} therefore also seem.\textsc{pres.3sg.pass.} from Latin.\textsc{abl.pl.m} Juno Lucina name.\textsc{ptcl.nom.sg.f}\\
\trans `therefore, she also seems to be called Iuno Lucina by the Latins' (Varro, \textit{de lingua Latina} 5.69)
\label{quoque2}
\end{exe}

\begin{exe}
\ex
\gll ab eo \emph{quoque}, quibus ..., tribuni aerarii dicti\\
from he.\textsc{abl.sg.m} also who.\textsc{abl.pl.m} ~ tribune.\textsc{nom.pl.m} treasury.\textsc{nom.pl.m} name.\textsc{nom.pl.m}\\
\trans `from this, these people ... were also called tribuni aerarii' (Varro, \textit{de lingua Latina} 5.181)
\label{quoque3}
\end{exe}

\begin{exe}
\ex
\gll aes \emph{quoque} stipem dicebant\\
copper.coin.\textsc{acc.sg.n} also gift.\textsc{acc.sg.f} name.\textsc{imprf.3pl.}\\
\trans `they also call a copper coin a gift' (Varro, \textit{de lingua Latina} 5.182)
\label{quoque4}
\end{exe}

\begin{exe}
\ex
\gll hinc \emph{quoque} illa nomina\\
hence also that.\textsc{nom.pl.n} name.\textsc{nom.pl.n}\\
\trans `hence also these names' (Varro, \textit{de lingua Latina} 8.84)
\label{quoque5}
\end{exe}

\begin{exe}
\ex
\gll haec \emph{quoque} perfecto ludebat Iasone Varro\\
this.\textsc{acc.pl.n} also finish.\textsc{abl.sg.m} play.\textsc{imprf.3sg} Jason.\textsc{abl.sg} Varro.\textsc{nom.sg}\\
\trans `having finished his work on Jason, Varro also played with these themes' (Propertius 2.34.85)
\label{quoque6}
\end{exe}

\begin{exe}
\ex
\gll haec \emph{quoque} lascivi cantarunt scripta Catulli\\
this.\textsc{nom.pl.n} also playful.\textsc{gen.sg.m} sing.\textsc{perf.3pl} write.\textsc{ptcl.nom.pl.n} Catullus.\textsc{gen.sg}\\
\trans `the writings of Catullus also sang of these themes' (Propertius 2.34.87)
\label{quoque7}
\end{exe}

The position of the question\is{interrogatives} particle \emph{ne} also seems significant. By virtue of its meaning, this particle has no more claim to stand close to the start of the clause than \isi{negation} in Latin \hyperlink{p419}{\emph{[p419]}} itself or in \ili{German} \emph{etwa} `for instance' or \emph{vielleicht} `perhaps'. Only enclisis\is{enclitics} explains the long-acknowledged rule that \emph{ne} belongs immediately after the first word of the clause, whatever the nature of that word. It is beyond the scope of this paper to go through all the evidence and discuss the real and apparent exceptions, using the material in \citet[75ff.]{Hand1845} and \citet[42--46]{Kaempf1886} (on the latter see the review by \citealp{Abraham1887}, who suggests punctuation after the pronoun\is{pronouns} in examples like (\ref{ne1}) and (\ref{ne2})). It is enough to point to the fact that the classical and later language still maintains this rule, and that the word \emph{utrumne}\label{utrumne} instead of \emph{utrum} `whether', attested since Catullus, is derived from this. Post-Homeric \ili{Greek} \emph{toigār} `so/therefore' attracted the particle \emph{toi} (still separate at the time of Homer) to itself because it had become customary to view it as the first word of a clause rather than an independent clause; \emph{utrum} attracted \emph{ne} for similar reasons.

\begin{exe}
\ex
\gll sed ego sum\emph{ne} infelix?\\
but I.\textsc{nom.sg} be.\textsc{pres.1sg}-\textsc{ne} unhappy.\textsc{nom.sg.m}\\
\trans `but am I unhappy?' (Plautus, \textit{Mostellaria} 362)
\label{ne1}
\end{exe}

\begin{exe}
\ex
\gll sed tu novisti\emph{n} fidicinam Acrobolistidem?\\
but you.\textsc{nom.sg} get.to.know.\textsc{prf.2sg}-\textsc{ne} female.lutist.\textsc{acc.sg.f} Acrobolistides.\textsc{acc.sg.f}\\
\trans `but do you know the female lutist Acrobolistides?' (Plautus, \textit{Epidicus} 503)
\label{ne2}
\end{exe}

A certain weakening of the old rule can be observed in that, if a sentence consisting of a protasis and an apodosis was to be marked as interrogative\is{interrogatives} by \emph{ne}, the classical language inserted \emph{ne} in the apodosis, while the earlier language attached -\emph{ne} directly to the conjunction in the protasis. This is related to the common habit of attaching \emph{ne} to the relativizer\is{relative pronouns} in a relative clause\is{relative clauses} and then using such a relative clause without the addition of a main clause to ask whether the statement given in the previous clause holds for the referent described in the relative clause.\is{relative clauses} Other subordinate\is{subordination} clauses were also used in this way. (On all of this see \citealp{BrixNiemeyer1888} on Trinummus 360 and \citealp{Lorenz1883,Lorenz1886} on Miles 965 and Mostellaria 738.)

From here we have, I think, a way to understand a particle that has so far been incorrectly explained. \citet[14f.]{Ribbeck1869}, influenced by \citet[526]{StolzSchmalz1890}, derives \emph{sin} `but if' from the collocation of \emph{si} `if' with the \isi{negation} \emph{ne}. The meaning `if not' that corresponds to this origin is, according to Ribbeck, still visible in \hyperlink{p420}{\emph{[p420]}} examples like (\ref{sin1}). It then became customary to add \emph{aliter} `otherwise', \emph{secus} `otherwise/differently' or \emph{minus} `less' to \emph{sin}, ``tautologically or transitionally'', and also, when the other case hinted at by \emph{sin} `if not' needed to be formulated more specifically, did this in the form of simple \isi{parataxis}. Thus, according to Ribbeck,\ia{Ribbeck, Otto} \emph{sin} eventually became a normal adversative conjunction.

\begin{exe}
\ex
\gll si pares aeque inter se, quiescendum; \emph{sin}, latius manabit, et quidem ad nos, deinde communiter\\
if appropriate.\textsc{nom.pl.m} equally between himself.\textsc{acc.pl.m} rest.\textsc{grnd.nom.sg.n} if.not more.broadly spread.\textsc{fut.3sg} and indeed to we.\textsc{acc.pl} then jointly\\
\trans `if they are evenly matched, one has to rest; but if it spreads, and indeed spreads to us, then [we must act] jointly' (Cicero, \textit{ad Attticum} 16.13b2)
\label{sin1}
\end{exe}

There are several objections that can be made to this account. I do not want to dispute the possibility that there could have been a \emph{sin} meaning `if not', since \emph{quin} shows that the \isi{negation} \emph{ne} could be enclitic\is{enclitics} and lose its vowel. (However, \emph{sine} does not belong here, but rather equates to Indo-European *sn̥nē, i.e. the old locative of \emph{senu}-, and crucially is cognate with \ili{Greek} \emph{āneu} `without', which is not related to \ili{Gothic} \emph{inu} and Old High German\il{German, Old High} \emph{āno} `without', since these correspond to \ili{Sanskrit} \emph{anu}, \emph{ānu} `after, along, alongside' = \ili{Proto-Indo-European} *enu, *ēnu. The \isi{semantic change} that needs to be assumed here, `along(side)' > `aside from' > `without', is completely natural.) But there is no evidence at all for the claim that \emph{sin} originally had this meaning `if not'. The examples that \citet{Ribbeck1869} deploys or intends to deploy in this sense are suspect from the start, because there is no explanation of how this negative\is{negation} meaning, which had already vanished by Plautus's time, could have returned to such common use by the time of Cicero. And looking at the examples oneself ((\ref{sin1}) above plus (\ref{sin2})--(\ref{sin6})) reveals that they do not show what they are supposed to show. Following a conjecture by \citet[347]{Vahlen1879}, (\ref{sin7}) (with \emph{sin} where the manuscripts have \emph{sed}) could be added to the list; but this reading is hardly likely to become generally accepted. (\citet{StolzSchmalz1890} also mention \hyperlink{p421}{\emph{[p421]}} examples in early Latin, but nowhere can I find evidence of these.) In all these examples we are simply dealing with aposiopesis,\footnote{\emph{Translator's note}: this word refers to the rhetorical device in which a sentence is deliberately broken off mid-flow, with the reader being left to infer what follows.} as is appropriate to Priapeian and epistolary style. It is particularly the first two examples, with their \emph{quod di omen avertant} ((\ref{sin2})) and \emph{sed nihil opus est reliqua scribere} ((\ref{sin3})), that eliminate all doubt.

\begin{exe}
\ex
\gll qui si conservatus erit, vicimus; \emph{sin} ..., quod di omen avertant, omnis omnium cursus est ad vos\\
who.\textsc{nom.sg.m} if rescue.\textsc{ptcp.prf.nom.sg.m} be.\textsc{fut.3sg} win.\textsc{prf.1pl} if.not ~ who.\textsc{acc.sg.n} god.\textsc{nom.pl.m} omen.\textsc{acc.sg.n} avert.\textsc{pres.3pl.subj} all.\textsc{nom.sg.m} all.{gen.pl.m} road.{nom.sg.m} be.\textsc{pres.3sg} to you.\textsc{acc.pl}\\
\trans `if he will be saved, we have won. If not ... this omen may the gods avert, the whole road leads for all to you' (Cicero, \textit{Epistulae} 12.6.2)
\label{sin2}
\end{exe}

\begin{exe}
\ex
\gll si perficitis quod agitis, me ad vos venire oportet; \emph{sin} autem ... Sed nihil opus est reliqua scribere.\\
if finish.\textsc{pres.2pl} who.\textsc{acc.sg.n} do.\textsc{pres.2pl} I.\textsc{acc.sg} to you.\textsc{acc.pl} come.\textsc{inf.pres} be.necessary.\textsc{pres.3sg} if.not on.the.other.hand ~ but not.at.all work.\textsc{nom.sg.n} be.\textsc{pres.3sg} leave.behind.\textsc{ptcp.nom.pl.n} write.\textsc{inf.pres}\\
\trans `if you finish what you are dealing with, I have to come to you; on the other hand, if not ... But it is not necessary to spell out the rest.' (Cicero, \textit{Epistulae} 14.3.5)
\label{sin3}
\end{exe}

\begin{exe}
\ex
\gll si vir esse volet, praeclara συνοδία. \emph{Sin} autem, erimus nos, qui solemus.\\
if man.\textsc{nom.sg.m} be.\textsc{inf.pres} want.\textsc{fut.3sg} great.\textsc{nom.sg.f} group.of.travellers.\textsc{nom.sg.f} if.not on.the.other.hand be.\textsc{fut.1pl} we.\textsc{nom.pl} who.\textsc{nom.pl.m} be.used.to.\textsc{pres.1pl}\\
\trans `if he wants to be a man, it will be a great group of travellers. On the other hand, if not, we will continue as we are used to.' (Cicero, \textit{ad Atticum} 10.7.2)
\label{sin4}
\end{exe}

\begin{exe}
\ex
\gll atque utinam tu quoque eodem die! \emph{sin} quod ..., multa enim utique postridie.\\
and hopefully you.\textsc{nom.sg} also the.same.\textsc{abl.sg.m} day.\textsc{abl.sg.m} if.not who.\textsc{nom.sg.n} ~ much.\textsc{acc.pl.n} because undoubtedly the.next.day\\
\trans `and hopefully you [can come over] on the same day as well! If not, because many things [may come up], then undoubtedly the next day.' (Cicero, \textit{ad Atticum} 13.22.4)
\label{sin5}
\end{exe}

\begin{exe}
\ex
\gll donec proterva nil mei manu carpes, licebit ipsa sis pudicior Vesta. \emph{Sin}, haec mei te ventris arma laxabunt.\\
as.long.as wanton.\textsc{abl.sg.f} nothing I.\textsc{gen.sg} hand.\textsc{abl.sg.f} harvest.\textsc{fut.2sg} be.allowed.\textsc{fut.3sg} himself.\textsc{abl.sg.f} be.\textsc{pres.2sg.subj} chaster.\textsc{nom.sg.m} Vesta.\textsc{nom.sg.f} if.not this.\textsc{nom.pl.n} I.\textsc{gen.sg} you.\textsc{acc.sg.m} belly.\textsc{gen.sg.m} weapon.\textsc{nom.sg.n} stretch.\textsc{fut.3pl}\\
\trans `as long as you will not steal anything from me with wanton hand, you may be chaster than Vesta herself. If not, these belly-weapons of mine will stretch you.' (\textit{Priapeia} 31)
\label{sin6}
\end{exe}

\begin{exe}
\ex
\gll \emph{sin}, ne te capiant, primo si forte negabit, taedia\\
if.not that.not you.\textsc{acc.sg.} capture.\textsc{pres.3pl.subj} at.first if maybe refuse.\textsc{fut.3sg} sadness.\textsc{nom.pl.n}\\
\trans `if not, let not sadness capture you if he will refuse you at first' (Tibullus 1.4.15)
\label{sin7}
\end{exe}

Once these examples fall away, Ribbeck's \citeyearpar{Ribbeck1869} hypothesis is robbed of the one feature that particularly recommended it: the connection to actual linguistic usage. Now, of course, the hypothesis that \emph{sin} initially meant `if not' in the time before our literary attestation, and later developed into the only attested meaning\is{semantic change} `but if', could nevertheless be correct. But this development is also not easy to construe. \citeauthor{Ribbeck1869} only discusses this point very briefly. If I understand him correctly, he thinks that a clause like (\ref{sin8}) was understood by inserting `if this is not the case' after \emph{sin} `if not', and let the more precise description of the opposite case follow from this: \emph{ipse animum pepulit} ``[in the case that] he himself had given direction to his inclinations'', and finally the apodosis \emph{vivit} `he lives'. But an asyndetic connection such as the one proposed here between \emph{sin} and what follows seems unthinkable to me: \emph{sed} (or repetition of \emph{si}) would surely be impermissible. There is probably an adversative asyndetic connection, but only insofar as the contrast is thus made perceptible by other means, through parallel structure of the two constituents or through preposing of the word that is the main carrier of the contrast in the second constituent.

\begin{exe}
\ex
\gll si animus hominem pepulit, actumst, animo servit, non sibi. \emph{sin} ipse animum pepulit, vivit\\
if heart.\textsc{nom.sg.m} human.\textsc{acc.sg.m} push.\textsc{prf.3sg} do.\textsc{prf.3sg.pass} heart.\textsc{dat.sg.m} obey.\textsc{pres.3sg} not himself.\textsc{dat.sg} if.not himself.\textsc{nom.sg.m} heart.\textsc{acc.sg.m} push.\textsc{prf.3sg} live.\textsc{pres.3sg}\\
\trans `if his heart brings forward the human part of him, it is done, he obeys his heart and not himself. If he himself does not bring forward his heart, he lives' (Plautus, \textit{Trinummus} 309)
\label{sin8}
\end{exe}

I believe that a much simpler solution suggests itself. Among his examples of \emph{ne} attached to the conjunction of the protasis, Brix gives the example of (\ref{ne1}) \citep{BrixNiemeyer1888}.

\begin{exe}
\ex
\gll Acanthio: At ego maledicentiorem quam te novi neminem. Charinus: \emph{Sin} saluti quod tibi esse censeo, id consuadeo? Acanthio: apage istiusmodi salutem, cum cruciatu quae advenit.\\
Acanthio.\textsc{nom.sg.m} but I.\textsc{nom.sg.} more.slanderous.\textsc{acc.sg.m} than you.\textsc{acc.sg} get.to.know.\textsc{prf.1sg} nobody Charinus.\textsc{nom.sg.m} if health.\textsc{dat.sg.f} what.\textsc{acc.sg.n} you.\textsc{dat.sg.m} be.\textsc{pres.inf} think.\textsc{pres.1sg} he.\textsc{acc.sg.n} recommend.\textsc{pres.1sg} Acanthio.\textsc{nom.sg.m} go.away of.that.kind health.\textsc{acc.sg.f} with torture.\textsc{abl.sg.m} which.\textsc{nom.sg.m} arrive.\textsc{pres.3sg}\\
\trans `Acanthio: But I don‘t know anyone who is more slanderous than you. Charinus: If I recommend you something which I think is healthy for you? Acanthio: Go away with such health which comes with torture.' (Plautus, \textit{Mercator} 142f) 
\label{ne1x}
\end{exe}

Brix reformulates Charinus's words as in (\ref{ne2x}). This is clearly in line with \hyperlink{p422}{\emph{[p422]}} conversational style in Plautus, in which interrogative\is{interrogatives} clauses marked as such by -\emph{ne} are extraordinarily frequently used for objections, e.g. (\ref{ne3x})--(\ref{ne6x}).

\begin{exe}
\ex
\gll tum\emph{ne} maledicentem me dicis si tibi id consuadeo\\
then slanderous.\textsc{acc.sg.m} I.\textsc{acc.sg.m} say.\textsc{pres.2sg} if you.\textsc{dat.sg} he.\textsc{acc.sg.n} recommend.\textsc{pres.1sg}\\
\trans `then you call me slanderous if I recommend it to you' (\citet{BrixNiemeyer1888}) 
\label{ne2x}
\end{exe}

\begin{exe}
\ex
\gll ego\emph{n} ubi filius corrumpatur meus, ibi potem?\\
I.\textsc{nom.sg}-\textsc{ne} where son.\textsc{nom.sg.m} corrupt.\textsc{pres.3sg.subj} my.\textsc{nom.sg.m} there drink.\textsc{pres.1sg.subj}\\
\trans `Am I supposed to drink there where my son was corrupted?' (Plautus, \textit{Bacchides} 1189) 
\label{ne3x}
\end{exe}

\begin{exe}
\ex
\gll ego\emph{n} quom haec cum illo accubet, inspectem?\\
I.\textsc{nom.sg}-\textsc{ne} when this.\textsc{nom.sg.f} with that.\textsc{abl.sg.m} lie.with.\textsc{pres.3sg.subj} look.at.\textsc{pres.1sg.subj}\\
\trans `Shall I look at it when she is lying with him?' (Plautus, \textit{Bacchides} 1192) 
\label{ne4x}
\end{exe}

\begin{exe}
\ex
\gll ego\emph{ne} indotatam te uxorem ut patiar?\\
I.\textsc{nom.sg}-\textsc{ne} not.provided.with.a.dowry.\textsc{acc.sg.f} you.\textsc{acc.sg} wife.\textsc{acc.sg.f} that tolerate.\textsc{pres.1sg.subj.pass}\\
\trans `Should I tolerate that you take a wife with no dowry?' (Plautus, \textit{Trinummus} 378) 
\label{ne5x}
\end{exe}

\begin{exe}
\ex
\gll at sci\emph{n} quam iracundus siem\\
but know.\textsc{pres.2sg}-\textsc{ne} how angry.\textsc{nom.sg.m} be.\textsc{pres.1sg.subj}\\
\trans `but do you know how angry I am?' (Plautus, \textit{Bacchides} 194) 
\label{ne6x}
\end{exe}

Clauses in which the interrogative\is{interrogatives} consists (elliptically)\is{ellipsis} only of a subordinate\is{subordination} clause with \emph{ne} -- exactly the type of \emph{ne}-clause to which the above example belongs -- are particularly frequently used in this way: (\ref{ne7x})--(\ref{ne11x}).

\begin{exe}
\ex
\gll Sosia: paulisper mane, dum edormiscat unum somnum. Amphitryon: quae\emph{ne} vigilans somniat?\\
Sosia.\textsc{nom.sg.f} for.a.brief.period.of.time stay.\textsc{imp.sg} until sleep.out.\textsc{pres.3sg.subj} one.\textsc{acc.sg.m} sleep.\textsc{acc.sg.m} Amphitryon.\textsc{nom.sg.m} who.\textsc{nom.sg.f}-\textsc{ne} watch.\textsc{ptcp.pres.nom.sg.m} sleep.\textsc{pres.3sg.subj}\\
\trans `Sosia: Stay for a moment until she has slept out one sleep. Amphitryon: But is she sleeping while she's watching?' (Plautus, \textit{Amphitryon} 297) 
\label{ne7x}
\end{exe}

\begin{exe}
\ex
\gll Cappadox: dum quidem hercle ita iudices, ne quisquam a me argentum auferat. Therapontigonus: quod\emph{ne} promisti?\\
Cappadoxius.\textsc{nom.sg.m} as.long.as indeed by.hercules so judge.\textsc{pres.2sg.subj} that anyone.\textsc{nom.sg.m} from I.\textsc{acc.sg} money.\textsc{acc.sg.n} take.away.\textsc{pres.3sg.subj} Therapontigonus.\textsc{nom.sg.m} what.\textsc{acc.sg.n}-\textsc{ne} promise.\textsc{perf.2sg}\\
\trans `Cappadox: By Hercules, as long as you judge in a way that nobody takes away money from me. Therapontigonus: [But it's the money] Which you promised?' (Plautus, \textit{Curculio} 704f) 
\label{ne8x}
\end{exe}

\begin{exe}
\ex
\gll quem\emph{ne} ego excepi in mari\\
who.\textsc{acc.sg.m}-\textsc{ne} I.\textsc{nom.sg} catch.\textsc{perf.1sg} in sea.\textsc{abl.sg.n}\\
\trans `but I caught him in the sea' (Plautus, \textit{Rudens} 1019) 
\label{ne9x}
\end{exe}

\begin{exe}
\ex
\gll quod\emph{ne} ego inveni in mari?\\
who.\textsc{acc.sg.n}-\textsc{ne} I.\textsc{nom.sg} find.\textsc{perf.1sg} in sea.\textsc{abl.sg.n}\\
\trans `but I found it in the sea?' (Plautus, \textit{Rudens} 1231) 
\label{ne10x}
\end{exe}

\begin{exe}
\ex
\gll Demipho: illud mihi argentum rursum iube rescribi Phormio. Phormio: quod\emph{ne} ego discripsi porro illis quibus debui?\\
Demipho.\textsc{nom.sg.m} that.\textsc{acc.sg.n} I.\textsc{dat.sg} money.\textsc{acc.sg.n} again command.\textsc{imp.pres} write.back.\textsc{inf.pres.pass} Phormio.\textsc{nom.sg.m} Phormio.\textsc{nom.sg.m} who.\textsc{acc.sg.n}-\textsc{ne} I.\textsc{nom.sg.m} distribute.\textsc{perf.1sg} further that.\textsc{abl.pl.m} who.\textsc{abl.pl.m} owe.\textsc{perf.1sg}\\\
\trans `Demipho: Command that the money will be returned to me, Phormio. Phormio: But I have transferred it further to the people I owed something to?' (Terence, \textit{Phormio} 923) 
\label{ne11x}
\end{exe}

A second example with a similar use of \emph{sin} is (\ref{sin9}).

\begin{exe}
\ex
\gll Paegnium: ne me attrecta subigitatrix. Sophoclidisca: \emph{sin} te amo? Paegnium: male operam locas.\\
Paegnium.\textsc{nom.sg.m} not I.\textsc{acc.sg} touch.\textsc{imp.sg.pres} lascivious.woman{voc.sg.f} Sophoclidisca.\textsc{nom.sg.f} if you.\textsc{acc.sg} love.\textsc{pres.1sg} Paegnium.\textsc{nom.sg.m} badly work.\textsc{acc.sg.f} put.\textsc{pres.2sg}\\
\trans `Paegnium: Don't touch me, you lascivious woman. Sophoclidisca: But if I love you? Paegnium: Your effort is worthless.' (Plautus, \textit{Persa} 227) 
\label{sin9}
\end{exe}

Most readers of Plautus would, of course, translate \emph{sin} in both examples as `but if', identifying it as the normal \emph{sin}. Far from wanting to criticize this, I in fact see it as evidence that the normal \emph{sin} is identical to that found in these examples from Plautus. We can make an objection in the form of an interrogative clause\is{interrogatives} not only to others, but also to ourselves. In this sense we find objecting \emph{quine}, \emph{quemne} in (\ref{ne12x}) ``but that one I have left'' and (\ref{ne13x}) ``but he is fleeing'' (see the above translation of \emph{quine} in the examples from Plautus and Terence). And it is possible to respond to a self-addressed objection oneself with the type of apodosis found in the two examples of \emph{sin} from Plautus, in which the first speaker objects and the second speaker responds to the objection using an asyndetically \hyperlink{p423}{\emph{[p423]}} added apodosis: \emph{apage istiusmodi salutem} ``then away with that sort of benefit'', and \emph{male operam locus} ``well, then you are wasting your time''.

\begin{exe}
\ex
\gll an patris auxilium sperem? \emph{quemne} ipsa reliqui ... ?\\
or father.\textsc{nom.sg.m} help.\textsc{acc.sg.n} hope.\textsc{pres.1sg.subj} who.\textsc{acc.sg.m}-\textsc{ne} herself.\textsc{nom.sg.f} leave.\textsc{perf.1sg}\\
\trans `or should I hope for my father's help? Who I myself left ... ?' (Catullus 64.180) 
\label{ne12x}
\end{exe}

\begin{exe}
\ex
\gll coniugis an fido consoler memet amore? \emph{quine} fugit lentos incurvans gurgite remos?\\
spouse.\textsc{gen.sg.f} or faithful.\textsc{abl.sg.m} console.\textsc{pres.1sg.subj} I.\textsc{acc.sg} love.\textsc{abl.sg.m} who.\textsc{nom.sg.m}-\textsc{ne} flee.\textsc{pres.3sg} slow.\textsc{acc.pl.m} bend.\textsc{ptcp.pres.nom.sg.m} eddy.\textsc{abl.sg.m} oar.\textsc{acc.pl.m}\\
\trans `Or am I supposed to console myself with the faithful love of my spouse? But he is fleeing while he is bending his slow oars in the eddy.' (Catullus 64.182f) 
\label{ne13x}
\end{exe}
%**The translation of the second part may be a bit awkward but I didn't really find something more satisfying.** 

Correspondingly, in the example from Plautus analysed above according to Ribbeck's \citeyearpar{Ribbeck1869} hypothesis, the original use of \emph{sin} is produced by the punctuation: \emph{sin ipse animum pepulit? vivit}. ``But how so, if he himself has given direction to his inclinations? Well, then he lives.'' It is an entirely natural development that over the course of time the clause type actually used for objections came to be used for an opposing case, and that in connection with this the \emph{sin}-interrogative\is{interrogatives} was perceived as protasis and the original answer as apodosis.

If \citet[210]{Mueller1872} is correct in reading \emph{sin} in (\ref{sin10}) (where the manuscripts have \emph{sint}, and the first printed edition has \emph{si}; cf. Nonius 290.4 in \citealp[456]{Mueller1888}), this adds a third instructive example to the two from Plautus, because here, too, \emph{sin} serves to introduce an objection, the difference being that this is announced by \emph{quid}, and that a \emph{ne}-clause follows which further specifies the question. According to \citet{Mueller1872}, this is an objection that one addresses to oneself. The same scholar's \emph{quodsin ulla} `but.if any.\textsc{nom.sg.f}' (Lucilius 4 Fragment 22 verse 38) with inexplicable \emph{sin} rather than \emph{quodsi nulla} `but.if not.any.\textsc{nom.sg.f}' becomes redundant if the following line is read correctly.

\begin{exe}
\ex
\gll ad non sunt similes neque dant. quid? \emph{sin} dare vellent? acciperesne? doce\\
but not be.\textsc{pres.3pl} similar.\textsc{nom.pl.m} and.not give.\textsc{pres.3pl} what.\textsc{nom.sg.n} if give.\textsc{pres.inf} want.\textsc{imperf.3pl.subj} accept.\textsc{imperf.2sg} teach.\textsc{imp.pres}\\
\trans `But they are not similar nor do they give. What? But if they want to give? Would you accept? Tell me.' (Lucilius 29, Fragment 87, verse 107) 
\label{sin10}
\end{exe}

Decisive evidence comes from the particles of affirmation and surprise \emph{hercle} `by Hercules', \emph{pol}, \emph{edepol} `by Pollux', \emph{ecastor} `by Castor' and \emph{eccere} `by Ceres', which have the property of being able to occupy either the first or the second position in the clause without being able to occur further back in the clause, unless they are blocked by other \isi{enclitics} such as \emph{quidem} `indeed, in fact', \emph{autem} `but' (Plautus, \textit{Aulularia} 560), \emph{obsecro} `I implore', \emph{quaeso} `I beg (for)', \emph{credo} `I believe', or \emph{ego} `\textsc{1sg.nom}', \emph{tu} (\textsc{2sg.nom}) or \emph{ille} `that, he, it' after \emph{ne} (\textsc{q}), or \emph{tu} after \emph{et} `and', \emph{at} `but, yet' or \emph{vel} `or', by virtue of their own claim to this position. Various facts show us how strong the pressure is for this word class too to occupy second position. For one thing, while the collocation \emph{pol ego} `by Pollux, I' is sometimes in initial position and sometimes preceded by another word (and hence \emph{ego} is just as happy to occupy third position as second position), the reverse order \emph{ego pol} `I, by Pollux' is only \hyperlink{p424}{\emph{[p424]}} found clause-initially \citep[62]{Kellerhoff1891}, showing that \emph{pol} avoids third position. For another thing, when affirmation particles relate to a whole sentence consisting of protasis and apodosis, they are attached to the first word of the protasis; \emph{si hercle} `if by.Hercules', \emph{si quidem hercle} `if indeed by.Hercules', \emph{ni hercle} `if.not by.Hercules', \emph{postquam hercle} `after by.Hercules', \emph{si ecastor} `if by.Castor', \emph{si pol} `if by.Pollux', and \emph{si quidem pol} `if indeed by.Pollux' are quite usual, while the placement of \emph{hercle} `by Hercules' in the apodosis is not unheard of (see Plautus, \textit{Miles Gloriosus} 309, \textit{Persa} 627), but rare. (Cf. \citealp{BrixNiemeyer1888} on \textit{Trinummus} 457, \citealp{Lorenz1883,Lorenz1886} on \textit{Miles Gloriosus} 156, 1239, on \textit{Mostellaria} 229, \citealp[72f.]{Kellerhoff1891}) We have seen exactly the same phenomenon with interrogative\is{interrogatives} -\emph{ne}. But while this positioning is limited to earlier stages of the language for -\emph{ne}, it is still very much alive in the classical language for \emph{hercle} (\emph{hercules}): see \citet[477, §78]{SeyffertMueller1876} on \textit{Laelius}, who refer to \citet[43, 239, 269]{Wichert1856}, \citet{Weissenborn1853} on Livius 5.4.10, etc. The classical language thus generally retains the traditional position of the particle \emph{hercle} `by Hercules', the only one that lives on in the classical language, but nevertheless such that the placement of this particle in absolute clause-initial position falls out of use. The Imperial Age\footnote{\emph{Translator's note}: this refers to Latin produced in the period from the reign of Tiberius (14 CE) onwards.}, of course, permits more variability: Quintilian 1.2.4, Tacitus, \textit{Dialogus} 1, \textit{Historiae} 1.84, Pliny, \textit{Epistulae} 6.19.6, Gellius 7.2.1, etc.

Furthermore, these particles, like the \isi{enclitics} discussed earlier, often cause \isi{tmesis}. Alongside (\ref{particle1}) (as opposed to (\ref{particle2})), (\ref{particle3}), and (\ref{particle4}) (as opposed to \emph{nescio} `\textsc{neg}.know.\textsc{pres.1sg}'), this includes the splitting of collocations with \emph{per}, as in (\ref{particle5})--(\ref{particle8}), and the splitting of \emph{quicumque `whoever/whatever'}, as in (\ref{particle9}).

\begin{exe}
\ex
\gll ne \emph{hercle} operae pretium quidem\\
not by.hercules work.\textsc{gen.sg.f} price.\textsc{acc.sg.n} indeed\\
\trans `by Hercules, it is not even worth the work' (Plautus, \textit{Miles Gloriosus} 31) 
\label{particle1}
\end{exe}

\begin{exe}
\ex
\gll ne unum quidem \emph{hercle}\\
not one.\textsc{acc.sg.m} indeed by.hercules\\
\trans `not even a single one' (Plautus, \textit{Bacchides} 1027) 
\label{particle2}
\end{exe}

\begin{exe}
\ex
\gll cis \emph{hercle} paucas tempestates\\
within by.hercules few.\textsc{acc.pl.f} time.period\textsc{acc.pl.f}\\
\trans `by Hercules, soon' (Plautus, \textit{Mostellaria} 18) 
\label{particle3}
\end{exe}

\begin{exe}
\ex
\gll non \emph{edepol} scio\\
not by.Pollux know.\textsc{pres.1sg}\\
\trans `by Pollux, I don't know' (Plautus, \textit{Mostellaria} 18)
\label{particle4}
\end{exe}

\begin{exe}
\ex
\gll per \emph{pol} saepe peccas\\
very by.Pollux often sin.\textsc{pres.2sg}\\
\trans `by Pollux, you sin very often' (Plautus, \textit{Casina} 370) 
\label{particle5}
\end{exe}

\begin{exe}
\ex
\gll per \emph{ecastor} scitus puer est natus Pamphilo\\
very by.Castor clever.\textsc{nom.sg.m} boy.\textsc{nom.sg.m} be.\textsc{pres.2sg} be.born.\textsc{ptcp.perf.m} Pamphilus.\textsc{dat.sg.m}\\
\trans `by Castor, a very clever son was born to Pamphilus' (Terence, \textit{Andria} 416) 
\label{particle6}
\end{exe}

\begin{exe}
\ex
\gll per \emph{pol} quam paucos\\
very by.Pollux very few.\textsc{acc.pl.m}\\
\trans `by Pollux, very few' (Terence, \textit{Hecyra} 1) 
\label{particle7}
\end{exe}

\begin{exe}
\ex
\gll per \emph{hercle} rem mirandam Aristoteles ... dicit\\
very by.Hercules thing.\textsc{acc.sg.f} astonish.\textsc{ptcp.acc.sg.f} Aristotle.\textsc{nom.sg.m} ~ name.\textsc{pres.3sg}\\
\trans `by Hercules, Aristotle names a very astonishing thing' (Gellius 2.6.1) 
\label{particle8}
\end{exe}

\begin{exe}
\ex
\gll quoi \emph{pol} quomque occasio est\\
who.\textsc{dat.sg.m} by.Pollux ever occasion.\textsc{nom.sg.f} is\\
\trans `to whomever there is a chance' (Plautus, \textit{Persa} 210) 
\label{particle9}
\end{exe}

\emph{hercle} `by Hercules' etc., therefore, occupy either the first or the second position in the clause; if they are not initial and heavily stressed, they are treated in the manner of \isi{enclitics}. Anyone who it occurs to that these particles are actually vocatives\is{vocative|(} (cf. (\ref{particle10})) will immediately recall that peculiar rule of the \ili{Sanskrit} \hyperlink{p425}{\emph{[p425]}} grammarians and transmitters of the accentuated Vedic texts, that the vocative, if clause-initial, is orthotonic, and if it is clause-internal it is enclitic.\is{enclitics} (Cf. the explanation given by \citealp[34ff.]{Delbrueck1888}). One can add that, at least in the classical languages, the actual vocative also has an unmistakable tendency to occupy second position in the clause.

\begin{exe}
\ex
\gll doctis \emph{Juppiter} et laboriosis\\
teach.\textsc{abl.pl.m} by.Jupiter and demanding.\textsc{abl.pl.m}\\
\trans `by Jupiter, taught and demanding' (Catullus 1.7) 
\label{particle10}
\end{exe}

Now it is of course awkward that what is a firm law for the vocative-like particles\is{particles|)} is visible only as a tendency with the actual vocative. It can hardly be assumed that such a tendency is a weakening of an older, stricter law. The reverse is more probable: that the tendency found with the category of vocatives represented by \emph{hercle} became a rule, and that the invocation of a god for the purpose of affirmation led to stronger conventionalization than in other invocations of gods or in addressing other people. (\ili{Greek} shows great flexibility in the positioning of the corresponding \emph{Hērakleis} and similar invocations, as far as can be judged from the usage of the comics and orators.) A consequence of this, if we may assume a connection between position and stress with the vocatives, is that \ili{Sanskrit} enclisis\is{enclitics} was originally only a tendency and not an unconditional law, and that vocatives which were not clause-initial or verse-initial could also be orthotonic, a property which was then lost in \ili{Sanskrit} by virtue of its drive to generalize.

It has not escaped me that the tendency for the vocative to occupy second position can also be explained without reference to earlier enclisis.\is{enclitics} It is thus even more valuable to me that \citet[557]{StolzSchmalz1890}, starting from a completely different descriptive standpoint, also claim weak stress for the Latin vocative in second position.\il{Latin|)}\is{vocative|)}


\section{Verb position in Germanic and Proto-Indo-European}\is{verb position|(}\is{verb-second|(}

Our Modern \ili{German} rule (cf. \citealp[181ff., esp. 195]{Erdmann1886}) that the verb occupies second position in main clauses and final position in subordinate\is{subordination} clauses (both with certain exceptions that hold under specific \hyperlink{p426}{\emph{[p426]}} conditions) was already valid for Old High German\il{German, Old High} \isi{prose} and poetry,\is{poetry} as is well known. (In addition to the evidence \citeauthor{Erdmann1886} provides, see also \citealp[54ff.]{Tomanetz1879}; \citeyear[381]{Tomanetz1890}.) In fact, given that this positional rule leaves clear traces not only in Old Saxon\il{Saxon, Old} but also in Old English\il{English, Old} and Old Norse,\il{Norse, Old} it can probably be assumed to be Common Germanic.\il{Germanic, Common} However, as far as I can tell, all researchers who have engaged in detail with this \ili{Germanic} positional law are agreed that the difference in position between the two clause types should be considered an innovation. \citet[139ff.]{Bergaigne1877}, \citet[284]{Behaghel1878} and \citet[88ff.]{Ries1880} all maintain that verb-final order, as found in subordinate\is{subordination} clauses, was originally a property of all clauses and was later replaced only gradually in main clauses by a more recent rule with a different effect. However, when it comes to the how and why of such a change, the researchers in question have either remained silent or adduced reasons which are far from convincing when subjected to careful thought. \citet{Ries1880}, for example, claims that the natural drive to express more important information before less important information must have led to the verb being placed near the start of the clause in main clauses and not in subordinate\is{subordination} clauses, because the verb is more important in main clauses than in subordinate\is{subordination} clauses!

The opposite point of view is represented by \citet[82ff.]{Tomanetz1879}. He believes that a general change caused the verb to shift to final position in subordinate\is{subordination} clauses; originally, he claims, it would have occupied second position in these, just as in main clauses. Although Tomanetz's explanation has the advantage over Ries's\ia{Ries, John} in simplicity and clarity, he still does not succeed in avoiding the assumption -- completely unjustifiable, in my view -- that a pressure to differentiate main and subordinate\is{subordination} clauses had taken effect.\is{verb-second|)}

\hyperlink{p427}{\emph{[p427]}} \ili{Sanskrit}, \ili{Latin} and \ili{Lithuanian} regularly place the verb at the end of the clause. It is believed that this reveals a custom in their ancestor language.\il{Proto-Indo-European} And certainly for subordinate\is{subordination} clauses the additional evidence from \ili{Germanic} confirms final placement of the verb as Indo-European. For main clauses this unanimity is lacking, and, when other considerations are not decisive, it is at least as conceivable that what held for subordinate\is{subordination} clauses was extended to main clauses in \ili{Sanskrit}, \ili{Latin} and \ili{Lithuanian}, rather than the alternative, that \ili{Germanic} subsequently introduced a distinction between the two clause types. However, it is unlikely that the protolanguage stressed its verbs differently in main and subordinate\is{subordination} clauses and yet placed them in the same position. Furthermore, based on what has been presented, we must expect that in the ancestor language\il{Proto-Indo-European} the verb in the main clause was placed immediately after the first word in the clause because, and insofar as, it was enclitic.\is{enclitics} In other words: the \ili{German} positional law already held in the ancestor language.\il{Proto-Indo-European} It must be borne in mind that all clauses, not only those that we now view as subordinate\is{subordination} clauses, were seen as hypotactic\is{hypotaxis} in \ili{Sanskrit} and therefore, we may assume, had a stressed verb in the ancestor language,\il{Proto-Indo-European} so that at any rate verb-final position must have been very common.

I do not wish to deny that the proposal put forward here could be made less general. For the law regarding the placement of \isi{enclitics} (disregarding e.g. vocatives)\is{vocative} we have only been able to adduce examples in which the enclitic\is{enclitics} is no larger than two syllables. It could therefore be said that the law was only valid for monosyllabic and disyllabic \isi{enclitics}, and that those of more than two syllables remained in the position that the constituent in question would otherwise receive -- or at least, to express the idea more carefully, that above a certain size threshold an enclitic\is{enclitics} was not bound by the positional law of the \isi{enclitics}. Applying this to the verb would lead to the assumption that monosyllabic and disyllabic verb forms, or shorter verbal \hyperlink{p428}{\emph{[p428]}} forms below a certain threshold, moved to second position\is{verb-second} in main clauses, and that the other verbal forms in main clauses kept to the position that was dominant in subordinate\is{subordination} clauses. It could then furthermore be assumed that \ili{Germanic} has generalized the rule from the shorter verb forms to all others. Moreover, what happened in the languages that place all verbs finally becomes even clearer.\is{verb-second}

It is too much to ask for me to deliver a final verdict on the justification of this more limited version of my proposal. On the other hand, it is probably to be expected that I should take a further look around and ask whether the verbal positional law of the ancestor language\il{Proto-Indo-European} has really left no traces outside \ili{Germanic}. The absence of any hints of such a law could easily cause one to doubt the correctness of the explanations presented here.

Now, here it must be said that, other than the verb-final languages already mentioned, not only \ili{Celtic} but also (much more significantly for this kind of investigation) \ili{Greek} behaves very differently to \ili{Germanic}. One should expect that \ili{Greek}, since it has retained main clause stress on the verb, would also retain main clause positioning. But it is well known that this is not the case. The position of the verb is, on the whole, very free.

Against such facts it is welcome that two of the languages that prefer verb-finality display \ili{Germanic} main clause positioning in a particular case. For \ili{Lithuanian}, \citet[§1637]{Kurschat1876} states that, when the predicate consists of a \isi{copula} and a noun, in contrast to the general rule, it is not the noun that precedes but rather the \isi{copula}, which immediately follows the subject. A similar situation can be found with the verb \emph{esse} `to be' in \ili{Latin}.\il{Latin|(} \citet[441]{SeyffertMueller1876} on Cicero's \textit{Laelius de Amicitia} 70 has shown that \emph{esse} has a preference for attaching to the first word of the clause, both when it is an interrogative\is{interrogatives} pronoun\is{pronouns} or an interrogative\is{interrogatives} functioning as a relative pronoun\is{pronouns}\is{relative pronouns} and when it is a demonstrative or belongs to another word class. There are, according to Seyffert,\ia{Seyffert, August Oskar} \hyperlink{p429}{\emph{[p429]}} `innumerable' examples. From \textit{Laelius} he adduces: §56 \emph{qui \emph{sint} in amicitia} `who are in friendship' (interrogative),\is{interrogatives} 17 \emph{quae \emph{est} in me facultas} `what skill is in me' (relative),\is{relative clauses} 2 \emph{quanta \emph{esset} hominum admiratio} `how much amazement there was among people', 53 \emph{quam \emph{fuerint} inopes amicorum} `how poor they were in friends', 83 \emph{eorum \emph{est} habendus} `of them is to be had', 5 \emph{tum \emph{est} Cato locutus} `at that time Cato was the speaker', 17 \emph{nihil \emph{est} enim} `because nothing is ...', 48 \emph{ferream \emph{esse} quandam} `to be something iron-like', and 102 \emph{omnis \emph{est} e vita sublata iucunditas} `everything joyful is removed from life'.

A further phenomenon fits with this observation: extremely often in Cicero, in a clause that contains both \emph{est}/\emph{sunt} (be.\textsc{3sg/pl}) and \emph{enim} `truly'/\emph{igitur} `therefore/then'/\emph{autem} `but', it is not these \isi{particles} that are attached to the first word in the clause, despite their recognized claim to this position in other cases, but rather \emph{est}/\emph{sunt} pushing \emph{enim}, \emph{igitur}, \emph{autem} into third position. The correct observation is made by \citet{Madvig1839} on Cicero, \textit{De finibus} 1.43: ``The explanation for this word order pattern (\textit{sapientia est enim})\footnote{\emph{Translator's note}: `wisdom is truly'.} is that by virtue of a heavy accent on the first word, which conveys the most important information, the enclitic\is{enclitics} word is shifted to the background. In the case of the alternative word order {[}\textit{sapientia enim est}{]} the accent on the first word is less strong. It is my opinion that this rule -- which goes against the teaching of Görenz\ia{Görenz, Johann August} and others, who, unaware of the nature of the enclitic\is{enclitics} word, thought that a certain emphasis is inherent to \textit{est} when placed in second position -- will become firmly established on the basis of the evidence of the best manuscripts, and of the correct interpretation.'' (Cf. \citealp[411]{SeyffertMueller1876}.)

For further confirmation, one could point to examples such as (\ref{encliticest1}), where the position of \emph{quid} `what' presupposes enclitic\is{enclitics} placement of \emph{est}. In particular, however, with \emph{esse} `to be' we find tmeses\is{tmesis} similar to those found with the \isi{enclitics} discussed earlier: such as of \emph{per}- in (\ref{encliticest2}) and (\ref{encliticest3}), in which the erroneous use of such \isi{tmesis} in the middle of the clause betrays archaizing style.

\begin{exe}
\ex
\gll etiamne \emph{est} quid porro\\
also be.\textsc{pres.3sg} something.\textsc{nom.sg.n} further\\ 
\trans `is there anything further?' (Plautus, \textit{Bacchides} 274)
\label{encliticest1}
\end{exe}

\begin{exe}
\ex
\gll tunc mihi ille dixit: quod classe tu velles decedere, per \emph{fore} accommodatum tibi, si ad illam maritimam partem provinciae navibus accessissem\\
then I.\textsc{dat.sg} that.\textsc{nom.sg.m} say.\textsc{prf.3sg} that fleet.\textsc{abl.sg.f} you.\textsc{nom.sg} want.\textsc{imprf.2sg.subj} go.away.\textsc{pres.inf} very be.\textsc{fut.inf} convenient.\textsc{nom.sg.n} you.\textsc{dat.sg} if to that.\textsc{acc.sg.f} maritime.\textsc{acc.sg.f} part.\textsc{acc.sg.f} province.\textsc{gen.sg.f} ship.\textsc{abl.pl.f} arrive.\textsc{prf.1sg.subj}\\  
\trans `then he told me that you would like to leave with a fleet, [and] it would be very convenient for you if I arrived at that close to the sea located part of the province by ship' (Cicero, \textit{Epistulae} 3.5.3; 51 BCE)
\label{encliticest2}
\end{exe}

\begin{exe}
\ex
\gll Phaedo Elidensis ex cohorte illa Socratica fuit Socratique et Platoni per \emph{fuit} familiaris\\
Phaedo.\textsc{nom.sg.m} of.Elis.\textsc{nom.sg.m} from entourage.\textsc{abl.sg.f} that.\textsc{abl.sg.f} Socratic.\textsc{abl.sg.f} be.\textsc{prf.3sg} Socrates-and.\textsc{gen.sg.m} and Plato.\textsc{gen.sg.m} very be.\textsc{prf.3sg} familiar\\ 
\trans `Phaedo of Elis was part of that Socratic entourage and he was very familiar with Socrates and Plato' (Gellius 2.18.1)
\label{encliticest3}
\end{exe}

Tmesis\is{tmesis} of \emph{qui ... cunque} `who/what ... ever': (\ref{encliticest4}) and (\ref{encliticest5}). Also with a form of \emph{fieri} `become/happen': (\ref{encliticest6}).

\begin{exe}
\ex
\gll cum quibus \emph{erat} quomque una, eis se dedere\\
with who.\textsc{abl.pl.m} be.\textsc{imprf.3sg} ever together he.\textsc{dat.pl.m} himself.\textsc{acc.sg.m} devoted.\textsc{pres.inf}\\ 
\trans `whomever he was together with, he devoted himself to them' (Terence, \textit{Andria} 63)
\label{encliticest4}
\end{exe}

\begin{exe}
\ex
\gll quod \emph{erit} cunque visum, ages\\
who.\textsc{nom.sg.n} be.\textsc{fut.3sg} ever see.\textsc{ptcp.prf.nom.sg.n} do.\textsc{fut.2sg}\\
\trans `whatever will be seen, you will do it' (Cicero, \textit{De finibus} 4.69)
\label{encliticest5}
\end{exe}

\begin{exe}
\ex
\gll istius hominis ubi \emph{fit} quomque mentio\\
that.\textsc{gen.sg.m} man.\textsc{gen.sg.m} where happen.\textsc{pres.3sg} ever mention.\textsc{nom.sg.m}\\
\trans `wherever that man is mentioned' (Plautus, \textit{Bacchides} 252)
\label{encliticest6}
\end{exe}

If in Latin\il{Latin|)} we find attachment to the first word of the clause only with one or two verbs \hyperlink{p430}{\emph{[p430]}} which have retained the tradition of original enclisis\is{enclitics} (and with these verbs then, of course, in all clause types), in \ili{Greek} we find a similar remnant of the old positional norm with quite a number of verbs, but only in a particular clause type. In Ancient Greek\il{Greek, Ancient|(} inscriptions\is{inscriptions} we often find clauses where the subject is followed immediately by the verb, despite the fact that an appositional\is{apposition} description belongs to it; in these cases the \isi{apposition} is strikingly separated by the verb from the word that it belongs to. It makes no difference that sometimes a clause-initial case form other than the subject \isi{nominative} is separated in such a way from its \isi{apposition}, and that sometimes a \emph{me} precedes the verb. \citet[41--42]{Boeckh1828} on CIG 25 was the first to recognize the archaic nature of this kind of word order, and \citet[1472]{Schulze1890} (pp26f. of the separate printing) in his review of \citet{Meister1889} emphasized its historical linguistic importance. It will be useful to present the examples here.

Most commonly this order is found in dedicatory and sculptors' inscriptions.\is{inscriptions|(} With \emph{anéthēke} `dedicate': (\ref{subjectverb1}).

\begin{exe}
\ex Ἀλκίβιοϲ ἀνέθηκεν κιθαρῳδὸϲ νηϲιώτηϲ\\
\gll Alkíbios \emph{anéthēken} kitharōidòs nēsiṓtēs\\
Alkibios.\textsc{nom} dedicate.\textsc{3sg.aor} citharist.\textsc{nom.sg} islander.\textsc{nom.sg}\\
\trans `Alkibios, a citharist of the island, dedicated (this).' (CIA 1.357) 
\label{subjectverb1}
\end{exe}

Also CIA 1.376 \textit{Epikhárînos} {[}\textit{\emph{ané}}{]}\textit{\emph{thēken} ho O...}, 1.388 \textit{Strónb}{[}\textit{ikhos \emph{anéthēke}}{]} \textit{Stronbí}{[}\textit{khou} (oder -\textit{khídou}) \textit{Euōnumeús}{]} (expansion almost certain!), 1.399 \textit{Mē\-kha\-níō}{[}\textit{n}{]} \textit{\emph{anéthēken} ho gramma}{[}\textit{teús}{]}, 1.400 {[}\textit{Pu}{]}\textit{thogén}{[}\textit{eia}{]} \textit{\emph{anéthēke}}{[}\textit{\emph{n} Ag}{]}\textit{ur\-ríou eg} {[}\textit{L}{]}\textit{akiadō}{[}\textit{n}{]}, 1.415 \textit{Aiskhúlos \emph{anéthē}}{[}\textit{\emph{ke}}{]} \textit{Puthéou Paianieú}{[}\textit{s}{]}, 4\textsuperscript{1}.373f. \textit{Sí\-mōn \emph{a}}{[}\textit{\emph{néthēke}}{]} \textit{ho knapheùs }{[}\textit{érgōn}{]} \textit{dekátēn}, 4\textsuperscript{2}.373.90 \textit{Onḗsimós m' \emph{anéthēken} aparkhḕn Athēnaíai ho Smikúthou uiós}, 4\textsuperscript{2}.373.198 {[}\textit{ē deîna \emph{anéthēken}}{]} \textit{Eumēlídou gunḕ Sphēttóthen}, 4\textsuperscript{2}.373.12 \textit{Xenokléēs \emph{anéthēken} Sōsíneō}, 4\textsuperscript{2}.373.223 \textit{Khnaïádēs \emph{anéthēken} ho Pal(l)ēneús}, 4\textsuperscript{2}.373.224 {[}\textit{S}{]}\textit{mîkros \emph{anéth}}{[}\textit{\emph{ēke} ...}{]} \textit{ho skulodeps}{[}\textit{ós}{]}, 4\textsuperscript{2}.373.226 {[}\textit{ho deîna \emph{anéthēke}}{]}\textit{\emph{n} Kēphisieús}, Acropolis inscription (\citealp{Kabbadias1886}, \citealp[135]{Studniczka1887}) \textit{Néarkhos \emph{an}}{[}\textit{\emph{éthēke} Neárkhou ui}{]}\textit{ùs érgōn aparkhḗn} (according to \citealp{Robert1887} \textit{Néarkhos \emph{an}}{[}\textit{\emph{éthēke} ho kerame}{]}\textit{ús ...}, CIA 2.1648 (reign of Augustus!) \textit{Metrótimos \emph{anéthēken} Oēthen}, IGA 48 \textit{Aristoménēs \emph{a}}{[}\textit{\emph{n}}{]}\textit{\emph{éth}}{[}\textit{\emph{ēk}}{]}\textit{\emph{e} Alexía taî Dámatri taî Khthoníai Ermioneús}, IGA 96 (Tegea) {[}\textit{ho deîna \emph{ané}}{]}\textit{\emph{thēke(n)} was\-tuókhō}, IGA 486 (Milet) {[}\textit{Er}{]}\textit{mēsiánax ḗmeas \emph{anéthēken}} {[}\textit{ho ...}{]} \textit{... ídeō tōpóllōni}, IGA 512\textsuperscript{a} (Gela) \textit{Pantárēs m'} \hyperlink{p431}{\emph{[p431]}} \textit{\emph{anéthēke} Menekrátios}, 543 (Achaean) \textit{Kunískos me \emph{anéthēke} ṓrtamos wérgōn dekátan}, Delphic inscription in western Greek alphabet \citep[445]{Haussoullier1882} \textit{toì Kharopínou paîdes \emph{anéthesan} toû Paríou}, Naxian inscription from Delos \citep[464f.]{Homolle1888} \textit{Ei(th)ukartídēs m' \emph{anéthēke} ho Náxios poiḗsas}, Naukratis inscriptions I no. 218 \textit{Phánēs me \emph{anéthēke} tōpóllōn}{[}\textit{i tôi Mi}{]}\textit{lēsíōi ho Glaúkou}, II no. 722 \textit{Musós m' \emph{anéthēken} Onomakrítou}, 767 {[}\textit{ho deîna \emph{anéthēken} Aphrod}{]}\textit{ítēi ho Ph}{[}\textit{ilá}{]}\textit{mm}{[}\textit{ōnos}{]}, 780 \textit{Phílis m' \emph{anéthēke} oupiká}{[}\textit{rte}{]}\textit{os têi Aphrodí}{[}\textit{tēi}{]}, 784 \textit{Ermophánēs \emph{anéth}}{[}\textit{\emph{ēken}}{]} \textit{ho Nausité}{[}\textit{leus}{]}, 819 {[}\textit{L}{]}\textit{ákri}{[}\textit{tó}{]}\textit{s m' \emph{ané}}{[}\textit{\emph{thē}}{]}\textit{\emph{ke} ourmo}{[}\textit{th}{]}\textit{ém}{[}\textit{ios}{]}\textit{ tēphrodí}{[}\textit{tēi}{]}, Boeotian inscription \citep[123ff.]{Kretschmer1891} \textit{Timasíphilos m' \emph{anéthēke} tōpóllōni toî Ptōieîi ho Praólleios}.\is{inscriptions|)}

Also in verse: CIA 1.398 \textit{Diogén}{[}\textit{ēs}{]}\textit{ \emph{anéthēken} Aiskhúl(l)ou uùs Keph}{[}\textit{a}{]}\textit{lēos}, IGA 95 \textit{Praxitélēs \emph{anéthēke} Surakósios tód' ágalma}, Naukratis inscription\is{inscriptions} II no. 876 \textit{Ermagórēs m' \emph{anéthēke} ho T}{[}\textit{ḗios}{]} \textit{tōpóllōni}, Pausanias 6.10.7 (5th century) \textit{Kleosthénēs m' \emph{anéthēken} ho Póntios ex Epidámnou}, Erythrae epigram (\citealp[312]{Kaibel1878} no. 769; 4th century) {[}\textit{...}{]}\textit{-thérsēs \emph{anéthēken} Athēnaíēi polioúkhōi paîs Zōílou}, Kalymnos epigram (\citealp[315]{Kaibel1878} no. 778; also 4th century?) \textit{Nikías me \emph{anéthēke} Apóllōni uiòs Thrasumḗdeos}. Cf. also CIA 1.403 {[}\textit{tónde Purēs}{]} \textit{\emph{anéthēke} Polumnḗstou phílo}{[}\textit{s uiós}{]}, IGA 98 (Arcadian) \textit{Téllōn tónd' \emph{anéthēke} Daḗmonos aglaòs uiós}.

With Lesbian \textit{káththēke} `lay.down/dedicate': (\ref{subjectverb2}). Also Naukratis II 789 and 790 {[}\textit{ho deîná me}{]} \textit{\emph{káththēke} o} {[}sic{]} \textit{Mut}{[}\textit{ilḗnaios}{]}. Cf. 807 {[}\textit{Aphrodí}{]}\textit{tai ho M...} and 814 {[}\textit{Aphrod}{]}\textit{ítai ho Ke...}.

\begin{exe}
\ex {[}ὁ δεῖνα κάθ{]}θηκε τᾷ Ἀφροδίτᾳ ὀ {[}sic{]} Μυτιλήναιοϲ\\
\gll ho deîna \emph{káththēke} tâi Aphrodítāi o Mutilḗnaios\\
the.\textsc{m.nom.sg} such lay.\textsc{3sg.aor} the.\textsc{f.dat.sg} Aphrodite.\textsc{dat} the.\textsc{m.nom.sg} Mytilenean.\textsc{m.nom.sg}\\
\trans `the Mytilenean dedicated (this) to Aphrodite' (Naukratis II, 788) 
\label{subjectverb2}
\end{exe}

With \textit{epoíēse}/\textit{epoíei} `make': (\ref{subjectverb3}).

\begin{exe}
\ex Πύρροϲ ἐποίηϲεν Ἀθηναῖοϲ\\
\gll Púrrhos \emph{epoíēsen} Athēnaîos\\
Pyrrhus.\textsc{nom} make.\textsc{3sg.aor} Athenian.\textsc{m.nom.sg}\\
\trans `Pyrrhus the Athenian made (this).' (CIA 1.335) 
\label{subjectverb3}
\end{exe}

Also CIA 1.362 (cf. \citealp[144]{Studniczka1887}) {[}\textit{E}{]}\textit{uphrónios }{[}\textit{\emph{epoíēsen} ho}{]} \textit{kerameús} (the expansion is probably certain!), CIA 1.483 \textit{Kallōnídēs \emph{epoíei} ho Deiníou}, CIA 4.477\textsuperscript{b} {[}\textit{ho deîna \emph{epoíēsen}} or \textit{\emph{epoíei} P}{]}\textit{ários}, CIA 4\textsuperscript{2}.373.81 \textit{Kálōn \emph{epoíēsen} Ai}{[}\textit{ginḗ\-tēs}{]}, CIA 4\textsuperscript{2}.373.95 {[}\textit{á}{]}\textit{rkhermos \emph{epoíēsen} ho Khî}{[}\textit{os}{]}, CIA 4\textsuperscript{2}.373.220 \textit{Leṓbios \emph{epoíē\-sen} Puretiádēs} (or \textit{Purrētiádēs}), IGA 42 (Argos) \textit{átōtos \emph{epoíwēe} Argeîos k'Argeiádas Ageláida t'Argeíou}, IGA 44 (Argos) \textit{Polúkleitos \emph{epoíei} Argeîos}, IGA 44\textsuperscript{a} (Argos) {[}\textit{\emph{e}}{]}\textit{\emph{po}}{[}\textit{\emph{í}}{]}\textit{\emph{wēe} Argeîos}, IGA 47 (Argos) \textit{Krēsílas \emph{epoíēse} Kudōniát}{[}\textit{as}{]}, IGA 165 \textit{Ypa\-tó\-dōros Arissto}{[}\textit{geítōn}{]}\textit{ \emph{epoēsátan} Thēbaíō}, IGA 348 \textit{Paiṓnios \emph{epoíēse} Mendaîos}, IGA 498 \textit{Míkōn \emph{epoíēsen} Athēnaíos}, \citet{Loewy1885} \hyperlink{p432}{\emph{[p432]}} 44\textsuperscript{a} \textit{-ōn \emph{epóēse} Thēbaîos}, 57 \textit{X}{[}\textit{e}{]}\textit{no-}{[}\textit{... \emph{epoíē}}{]}\textit{\emph{sen} Eleu}{[}\textit{theréus}{?]}, 58 \textit{-ou }{[}\textit{\emph{e}}{]}\textit{\emph{póēsen} }{[}\textit{Sik}{]}\textit{eliṓtēs}, 96 \textit{Kléōn \emph{epóēse} Sikuṓnios}, 103 {[}\textit{Daídalos \emph{ep}}{]}\textit{\emph{oíēse} Patroklé}{[}\textit{ous}{]}, 135\textsuperscript{d} \citep[388]{Loewy1885} {[}\textit{Sp}{]}\textit{oudías \emph{epoíēse} Athēnaîos}, 277 \textit{Timódamos T}{[}\textit{imodámou \emph{e}}{]}\textit{\emph{poíēse} Ampra}{[}\textit{kiṓtēs}{]}\label{Loewy277}, 297 (Apo\-theosis of Homer) \textit{Arkhélaos Apollōníou \emph{epoíēse} Priēneús}, 404 \textit{Níkandros e}{[}\textit{poíēsen}{]} \textit{ánd}{[}\textit{rios}{]}, \citet[72]{Klein1887} \textit{Eúkheiros \emph{epoíēsen} ourgotímou uiús} (twice), \citet[73]{Klein1887} \textit{Ergotélēs \emph{epoíēsen} ho Neárkhou}, \citet[202]{Klein1887} \textit{Xenóphantos \emph{epoíēsen} Athē\-n}{[}\textit{aîos}{]}, \citet[202]{Klein1887} (1 and 2) \textit{Teisías \emph{epoíēsen} Athēnaîos}, \citet[213]{Klein1887} \textit{Krítōn \emph{epoíēsen} Le(i)poûs ús}, i.e. \textit{uiús} according to the reading in \citet[144]{Studniczka1887}, Pausanias 6.9.1 \textit{tòn dè andriánta oi Ptolíkhos \emph{epoíēsen} Aiginḗtēs}, which allows one to infer an original inscription\is{inscriptions} \textit{Ptólikhos \emph{epoíēsen} Aiginḗtēs} (see \citealp[41--42]{Boeckh1828} on CIG 25).

Also in verse: CIA 4\textsuperscript{2}.373.105 \textit{Thēbádēs \emph{e}}{[}\textit{\emph{póēse} ...}{]}\textit{-nou paîs tód' ágalma}, Acropolis inscription\is{inscriptions} \citep[135ff.]{Studniczka1887} \textit{Antḗnōr \emph{ep}}{[}\textit{\emph{óēsen} '}{]}\textit{o Eumárous t}{[}\textit{ód' ágal\-ma}{]}, IGA 410 \textit{Alxḗnōr \emph{epoíēsen} ho Náxios, all' esídesthe}. Also IGA 349 \textit{Eúphrōn \emph{exepoíēs'} ouk adaḕs Pários}.

With \textit{égraphen}, \textit{égrapsen}, \textit{gráphei} `write': (\ref{subjectverb4}). Also \citet[29]{Klein1887} \textit{Timōní\-da}{[}\textit{s m'}{]} \textit{\emph{égrapse} Bía}, \citet[196.7]{Klein1887} \textit{Euthumídēs \emph{égrapsen} ho Pol(l)íou} (twice). \citet[194.2]{Klein1887} should be read the same way according to the illustration in \citet[Figure 188]{Gerhard1847}, as should \citet[195]{Klein1887}, both according to Dümmler.\ia{Dümmler, Ferdinand}\label{kleinx2} Cypriot\il{Greek, Cypriot} inscription\is{inscriptions} no. 147\textsuperscript{h} in \citet[148]{Meister1889}, \textit{-oikós me \emph{gráphei} Selamínios}.

\begin{exe}
\ex Τήλεφοϲ μ᾽ ἔγραφε ὁ Ἰαλύϲιοϲ\\
\gll Tḗlephos m' \emph{égraphe} ho Ialúsios\\
Telephos.\textsc{nom} me.\textsc{acc} write.\textsc{3sg.imp} the.\textsc{m.nom.sg} Ialysian.\textsc{m.nom.sg}\\
\trans `Telephos the Ialysian engraved me.' (IGA 482\textsuperscript{c}) 
\label{subjectverb4}
\end{exe}

Examples (\ref{subjectverb5})--(\ref{subjectverb7}) contain various synonyms of the above verbs.

\begin{exe}
\ex {[}Δ{]}ωρόθεοϲ ἐϝ{[}ε{]}ργάϲατο Ἀργεῖοϲ\\
\gll Dōrótheos \emph{ewergásato} Argeîos\\
Dorotheus.\textsc{nom} work.\textsc{3sg.aor.mid} Argive.\textsc{m.nom.sg}\\
\trans `Dorotheus the Argive wrought (this)' (IGA 48, Argos) 
\label{subjectverb5}
\end{exe}

\begin{exe}
\ex Πρίκων ἔ{[}π{]}α{[}ξα Κο{]}λώτα\\
\gll Príkōn \emph{épaxa} Kolṓta\\
Prikon.\textsc{nom} fix.\textsc{3sg.aor} Colotes.\textsc{gen}\\
\trans `Prikon, son of Colotes, built (this)' (IGA 555\textsuperscript{a}, Opus?) 
\label{subjectverb6}
\end{exe}

\begin{exe}
\ex Γιλίκα ἁμὲ κατέϲταϲε ὁ Σταϲικρέτεοϲ\\
\gll Gilíka hamè \emph{katéstase} ho Stasikréteos\\
Gilika.\textsc{nom} me.\textsc{acc} set.\textsc{3sg.aor} the.\textsc{m.nom.sg} Stasicrates.\textsc{gen}\\
\trans `Gilika, the son of Stasicrates, set (this) up' (Cypriot inscription\is{inscriptions} no. 73, \citealp{Deecke1884}) 
\label{subjectverb7}
\end{exe}

With \textit{eimí} `be': (\ref{subjectverb8}).

\begin{exe}
\ex {[}Π{]}όμπιόϲ εἰμι τοῦ Δημοκρίνεοϲ\\
\gll Pómpiós eimi toû Dēmokríneos\\
Pompeius.\textsc{nom} be.\textsc{1sg.prs} the.\textsc{m.nom.sg} Democrines.\textsc{gen}\\
\trans `I am Pompeius, son of Democrines' (IGA 387, Samos) 
\label{subjectverb8}
\end{exe}

Also IGA 492 (Sigeum), Ionic\il{Greek, Ionic} text: \textit{Phanodíkou \emph{eimì} tourmokráteos toû Prokonnēsíou}, Attic\il{Greek, Attic} text: \textit{Ph. \emph{eimì} toû Ermokrátous toû P.}, IGA 522 (Sicily) \textit{Longēnaîós \emph{eimi} dēmósios}, 528 (Cumae) \textit{Dēmokháridós \emph{eimi} toû ...}, 551 (Antipolis) \textit{érpōn \emph{eimì} theâs therápōn semnēs Aphrodítēs}, Rhodian inscription\is{inscriptions} in \citet[49]{Kirchhoff1887} \textit{Philtoûs \emph{ēmi} tâs kalâs a kúlix a poikíla}, Cypriot\il{Greek, Cypriot} inscription\is{inscriptions} 1 \citep{Deecke1884} \textit{Pra-}\hyperlink{p433}{\emph{[p433]}}\textit{totímō \emph{ēmì} tâs Paphías tō ierēwos}, 16 \textit{tâs theō \emph{ēmi} tâs Paphías} (likewise 65 and 66 in \citealp[46]{Hoffmann1891}), 23 \textit{Timokúpras \emph{ēmì} Timodámō}, \citet{Hoffmann1891} 78 \textit{Stasagórou \emph{ēmì} tō Stasándrō}, 79 \textit{Timándrō \emph{ēmì} tō Onasagórou}, 88 \textit{Pnutíllas \emph{ēmì} tâs Pnutagórau paidós}, and 121 \textit{Diweithémitós \emph{ēmi} tō basilēwos}.

To these can be added (\ref{subjectverb9}), where an adjective\is{adjectives} joined to \emph{eînai} represents the position of the verb, and also the examples in which an adjective\is{adjectives} without \emph{eînai} forms the predicate, e.g. (\ref{subjectverb10}).

\begin{exe}
\ex τᾶϲ Ἥραϲ ἱαρόϲ εἰμι τᾶϲ ἐν πεδίῳ\\
\gll tâs Hḗras \emph{hiarós} \emph{eimi} tâs en pedíōi\\
the.\textsc{f.gen.sg} Hera.\textsc{gen} holy.\textsc{m.nom.sg} be.\textsc{1sg.prs} the.\textsc{f.gen.sg} in plain.\textsc{dat.sg}\\
\trans `I am sacred to Hera of the plain.' (IGA 543) 
\label{subjectverb9}
\end{exe}

\begin{exe}
\ex Λέαγροϲ καλὸϲ ὁ παίϲ\\
\gll Léagros \emph{kalòs} ho país\\
Leagros.\textsc{nom} beautiful.\textsc{m.nom.sg} the.\textsc{m.nom.sg} child.\textsc{nom.sg}\\
\trans `The boy Leagros is beautiful.' \citep[44]{Klein1890} 
\label{subjectverb10}
\end{exe}

Also \citet[68]{Klein1890} \textit{Pantoxéna \emph{kalà} Korin(th)í}{[}\textit{a}{]},\label{Klein68} as the form \emph{KORINOI} given by \citeauthor{Klein1890} but not explained should probably be read; \citet[81]{Klein1890} \textit{Glaúkōn \emph{kalòs} Leágrou}; \citet[82]{Klein1890} \textit{Drómippos \emph{kalòs} Dromokleídou}, \textit{Díphilos \emph{kalòs} Melanṓpou}; \citep[83]{Klein1890} \textit{Líkhas \emph{kalòs} Sámios}, \textit{Alkim}{[}\textit{ḗ}{]}\textit{dēs \emph{kalòs} Αiskhulídou}; \citet[85]{Klein1890} \textit{Alkímakhos \emph{kalòs} Epikhárous}.

Outside the previously listed categories are (\ref{subjectverb11}), (\ref{subjectverb12}) and (\ref{subjectverb13}).

\begin{exe}
\ex Κλειϲθένηϲ ἐχορήγει Αὐτοκράτουϲ\\
\gll Kleisthénēs \emph{ekhorḗgei} Autokrátous\\
Cleisthenes.\textsc{nom} conduct.\textsc{3sg.imp} Autocrates.\textsc{gen}\\
\trans `Cleisthenes, son of Autocrates, endowed (this)' (CIA 4\textsuperscript{2}.377\textsuperscript{a}) 
\label{subjectverb11}
\end{exe}

\begin{exe}
\ex ἐν τἠπιάροι κ᾽ ἐνέχοιτο τοῖ ᾽νταῦτ᾽ ἐργα(μ)μένοι\\
\gll en tēpiároi \emph{k'} enékhoito toî 'ntaût' erga(m)ménoi\\
in the=sacrifice.\textsc{dat.irr} hold.\textsc{3sg.prs.opt} the.\textsc{n.dat.sg} here work.\textsc{ptcp.prf.pass.n.dat.sg}\\
\trans `He would be liable for a sacrifice performed here' (IGA 110.9, Elis) 
\label{subjectverb12}
\end{exe}

\begin{exe}
\ex Ἀκαμαντὶϲ ἐνίκα φυλὴ\\
\gll Akamantìs \emph{eníka} phulḕ\\
Acamantis.\textsc{nom} win.\textsc{3sg.imp} tribe.\textsc{nom.sg}\\
\trans `The tribe of Acamantis conquered.' (CIG 7806) 
\label{subjectverb13}
\end{exe}

Among the examples with \emph{anéthēke} and \emph{kaththēke} listed above, thirteen also contain a \isi{dative} in addition to subject, verb and \isi{apposition}; three (CIA 4\textsuperscript{1}.373f., IGA 95, IGA 543) also contain a substantivized \isi{accusative}, and CIA 4\textsuperscript{2}.373.90 contains both. While the \isi{accusative} alone always follows the \isi{apposition} (cf. also example (\ref{subjectverb14}) as well as the Antenor inscription),\is{inscriptions} the \isi{dative} is only found four times following the \isi{apposition} (IGA 486, Naukratis II.780, II.819, II.876) and eight times preceding it (Naukratis I.218, II.767, II.788, II.807, II.814, Hermes 26.123, Kaibel 769, Kaibel 778); finally, in IGA 48 the verb is followed by the \isi{genitive} of the father's name, then the \isi{dative} of the god's name with epithet, and only then the \isi{nominative} demonym that belongs to the subject.

\begin{exe}
\ex Θηβάδηϲ ἐ{[}πόηϲε ...{]}νου παῖϲ τόδ᾽ ἄγαλμα\\
\gll Thēbádēs \emph{epóēse} ...nou paîs tód' ágalma\\
Thebades.\textsc{nom} make.\textsc{3sg.aor} ...-\textsc{gen} child.\textsc{m.nom.sg} this.\textsc{n.acc.sg} statue.\textsc{acc.sg}\\
\trans `Thebades, son of ..., made this statue.' (CIA 4\textsuperscript{2}.373.105) 
\label{subjectverb14}
\end{exe}

In CIA 4\textsuperscript{2}.373.90 (=(\ref{manetheke}) above), \isi{accusative} and \isi{dative} are both inserted between the verb and the \isi{apposition}. This preposing of the case forms belonging to the verb over the \isi{apposition} is easy to understand: the verb attracts what it governs.

Using this type we can also explain the strange word order in CIA 4\textsuperscript{2}.373.82, expanded by \citet[143]{Studniczka1887} as in (\ref{strangewordorder}).

\begin{exe}
\ex Κρίτων Ἀθηναίᾳ ὁ Σκύθου ἀν{[}έθηκε καὶ ἐ{]}ποίη{[}ϲε{]} ({[}ἐ{]}ποίει?)\\
\gll Krítōn Athēnaíāi ho Skúthou \emph{anéthēke} kaì \emph{epoíēse} (\emph{epoíei}?)\\
Crito.\textsc{nom} Athenian.\textsc{f.dat.sg} the.\textsc{m.nom.sg} Scythes.\textsc{gen} dedicate.\textsc{3sg.aor} and make.\textsc{3sg.aor} (make.\textsc{3sg.imp}?)\\
\trans `Crito, the son of Scythes, made and dedicated (this) to an Athenian woman.' (CIA 4\textsuperscript{2}.373.82) 
\label{strangewordorder}
\end{exe}

The composer of the inscription\is{inscriptions} originally \hyperlink{p434}{\emph{[p434]}} envisaged the conventional word order \emph{Krítōn \emph{anéthēken} Athēnaíai ho Skúthou}, but then allowed the \isi{dative} \emph{Athēnaíai} to precede the \isi{apposition} when he was required by the addition of \emph{kaì epoíēse} to place \emph{anéthēke} after the \isi{apposition}.\il{Greek, Ancient|)}

\citet[xv]{Loewy1885} believes that he can show that this word order did not remain common after the first decades of the fourth century (cf. also CIA 2.1621--2.1648 and the sculptors' inscriptions\is{inscriptions} listed by \citet{Koehler1888} under No. 1621). The handful of later examples can reasonably be considered archaisms, especially as two of these (\citealp{Loewy1885} 277, 297, see above p\pageref{Loewy277}) deviate from the original norm by preposing the \isi{genitive} of the father's name before the verb. Even for the earlier period we cannot maintain that this positional norm was absolute \citep[324]{Hoffmann1891}, and in particular the Attic\il{Greek, Attic} dedicatory inscriptions\is{inscriptions} present us with numerous counterexamples. But the norm was very powerful, and in specific periods and specific areas it was decidedly dominant, justifying Schulze's \citeyearpar{Schulze1890} treatment of it as an Indo-European inheritance.

\ili{Sanskrit} provides striking parallels (\citealp[51ff.]{Delbrueck1878}, \citeyear[23f.]{Delbrueck1888}).\il{Sanskrit|(} In the language of the Brahmanas, we often find clauses that begin with \emph{sa} or \emph{sa ha} `precisely this one', followed immediately by the verb, mostly \emph{uvāca} (`speak/say'), and only then the more detailed description of the person announced by the pronoun,\is{pronouns} e.g. (\ref{Skt_ex4a}) and (\ref{Skt_ex4b}).

\begin{exe}
\ex \gll sa \emph{hovāca} gārgyaḥ\\
he.\textsc{masc.nom.sg} \textsc{ptc}-spoke descendant-of-Garga.\textsc{nom.sg.masc}\\
\trans `He, Gārgya, spoke'
(e.g. \textit{B\d{r}hadāra\d{n}yakopani\d{s}ad})
\label{Skt_ex4a}
\end{exe}

\begin{exe}
\ex \gll sa \emph{āikṣata} prajāpatiḥ\\
he.\textsc{masc.nom.sg} saw Brahmā\\
\trans `He, Brahmā, saw (...)'
(e.g. \textit{Śatapathabrāhmaṇa})
\label{Skt_ex4b}
\end{exe}

Similar is (\ref{SB_ex5}):

\begin{exe}
\ex \gll ta u hāita \emph{ūcur} devā ādityāḥ \\
\textsc{rel.masc.nom} \textsc{ptc} \textsc{ptc}-then spoke.\textsc{3.pl.perf} gods.\textsc{nom.pl} of-Aditi.\textsc{nom.pl} \\
\trans `The gods, sons of Aditi, then spoke'
(\textit{Śatapathabrāhmaṇa}, 3.1.3.4)\footnote{\emph{Translator's note}: The English translation here is based on \citet{Eggeling1885}.}
\label{SB_ex5}
\end{exe}

Sometimes the subject is also more heavily stressed; sometimes, under the influence of the tendency to end the clause with the verb, the \isi{apposition} is separated from the pronoun\is{pronouns} but still precedes the verb.

Furthermore, in the same Indic\il{Indo-Iranian} texts we find a striking placement of the verb in second position\is{verb-second} when the clause begins with \emph{íti ha}, \emph{tád u ha}, \emph{tád u sma}, or \emph{ápi ha}. These mostly involve the verbs \emph{uvāca} and \emph{āha} (`speak/say'); the name of the speaker then follows the verb -- in just the same way as in \ili{German} clauses with inversion.\il{Sanskrit|)}\is{verb position|)}

\hyperlink{p435}{\emph{[p435]}}


\section*{Addenda}
\addcontentsline{toc}{section}{Addenda}
\noindent These addenda add \hypertarget{addenda}{to} Section \ref{enclitic-archaic} pp\pageref{forAddenda1}--\pageref{forAddenda2} (concerning the inscriptions\is{inscriptions|(} with \textit{me} and \textit{emé}).

On p\pageref{forAddenda1} and p\pageref{forAddenda2}: Example (\ref{addendaEx1}) must be left out of consideration due to the state of the inscription; cf. \citet[155]{Roehl1882} on this example.

\begin{exe}
\ex {[}Π{]}εριφόνᾳ {[}ἀνέθη{]}κέ με (or -κ ἐμέ?) Ξενάγατοϲ\\
\gll Periphónāi anéthēké \emph{me}~(-k~\emph{emé}) Xenágatos\\
Periphone.\textsc{dat} dedicate.\textsc{3sg.aor} me.\textsc{acc} Xenagatos.\textsc{nom}\\
\trans `Xenagatos dedicated me to Periphone'
(IGA 538)\footnote{\emph{Translator's note}: Wackernagel writes 351, but this is a clear error based on the proximity of the page number 351.}
\label{addendaEx1}
\end{exe}

On p\pageref{forAddenda3}: Example (\ref{addendaEx2}); Metapontum inscription (Collitz 1643) \textit{Nikómakhós \emph{m'} epóei}; vase inscription no. 48 from \citet[65]{Klein1887} following \citet[195]{Six1888}\footnote{\emph{Translator's note}: Wackernagel refers to page 193 of \citet{Six1888}, but this is the first page of the article and does not contain the inscription in question.} \textit{Nikosthénēs em} (\citeauthor{Six1888}: \textit{\emph{m'} e-})\textit{poíēsen}.

\begin{exe}
\ex Οὑνπορίωνοϲ Φίλων με ἐποίηϲεν\\
\gll Hounporíōnos Phílōn \emph{me} epoíēsen\\
the=Emporion.\textsc{gen} Philo.\textsc{nom} me.\textsc{acc} make.\textsc{3sg.aor}\\
\trans `Philo, the son of Emporion, made me.' (CIA 4\textsuperscript{2}.373.103)
\label{addendaEx2}
\end{exe}

On p\pageref{forAddenda2}: \textit{emé} is also found twice in second position in the ancient vase inscription in \citet[168]{Pottier1888}: Example (\ref{addendaEx3}) and \textit{Oikōph(é)lēs \emph{em'} égrapsen} (written \textit{egraephsen}). See also \citet[180]{Pottier1888}: -\textit{polón emé}.

\begin{exe}
\ex ἐκεράμευϲεν ἐμεὶ Οἰκωφέληϲ\\
\gll ekerámeusen \emph{emeì} Oikōphélēs\\
throw.pots.\textsc{3sg.aor} me.\textsc{gen} Oikopheles.\textsc{nom}\\
\trans `Oikopheles made me.' \citep[168]{Pottier1888}
\label{addendaEx3}
\end{exe}\is{inscriptions|)}

\section*{List of critically discussed examples}\label{critical}
\addcontentsline{toc}{section}{List of critically discussed examples}
Homer 5.273 = 8.196 \dotfill p\pageref{Homer5273}\\
\phantom{Homer} 16.112 \dotfill p\pageref{moipron2}\\
\phantom{Homer} 13.321 \dotfill p\pageref{ke12}\\
Alcman, Fragment 52 \dotfill p\pageref{ask7}\\
Alcaeus, Fragment 68 \dotfill p\pageref{alcaeus}\\
\phantom{Alcaeus,} Fragment 83 \dotfill p\pageref{Alcaeus83}\\
Sappho, Fragment 2.7 \dotfill p\pageref{sappho1}\\
\phantom{Sappho, Fragment} 43 \dotfill p\pageref{sappho2}\\
\phantom{Sappho, Fragment} 66 \dotfill p\pageref{Sappho66}\\
\phantom{Sappho, Fragment} 97.4 Hiller (=100) \dotfill p\pageref{sappho3}\\
Pindar, \textit{Olympian Ode} 1.48 \dotfill p\pageref{Pind1.48}\\
Euripides, \textit{Medea} 1339 \dotfill p\pageref{relan27}\\
\phantom{Euripides,} Fragment 1029.4 \dotfill p\pageref{EurFragm1029}\\
Antiphon 5.38 \dotfill p\pageref{Antiph538}\\
Aristophanes, \textit{Acharnians} 779 \dotfill p\pageref{ask8}\\
\phantom{Aristophanes,} \textit{Frogs} 259 \dotfill p\pageref{AristophFrogs259}\\
\phantom{Aristophanes,} \textit{Ecclesiazusae} 916 \dotfill p\pageref{oposan8}\\
Demosthenes 18.43 \dotfill p\pageref{relan26}\\
\phantom{Demosthenes} 18.206 \dotfill p\pageref{relan22}\\
\phantom{Demosthenes} 24.64 \dotfill p\pageref{relan34}\\
\phantom{Demosthenes} \textit{Exordia} 1.3 \dotfill p\pageref{relan76}\\
\phantom{Demosthenes} \textit{Exordia} 3 \dotfill p\pageref{wordgroup42}\\
Callimachus, Fragment 114 \dotfill p\pageref{ask9}\\
Theocritus 2.159 \dotfill p\pageref{tiska22}\\
Pausanias 5.23.7 \dotfill p\pageref{paus4}\\
Palatine Anthology 6.140 \dotfill p\pageref{epi12}\\
\textit{Inscriptiones graecae antiquissimae} \citep{Roehl1882} 384 \dotfill p\pageref{IGA384}\\
\phantom{\textit{Inscriptiones graecae antiquissimae} \citet{Roehl1882}} 474 \dotfill p\pageref{IGA474}\\
Collitz, \textit{Collection of Greek dialect inscriptions} 26 \dotfill p\pageref{artprep2}\\
\phantom{Collitz, \textit{Collection of Greek dialect inscriptions}} 3184.8 \dotfill p\pageref{dodonian3}\\
\phantom{Collitz, \textit{Collection of Greek dialect inscriptions}} 3213.3 \dotfill p\pageref{tiska25}\\
\hyperlink{p436}{\emph{[p436]}}\\
\textit{Greek vases with masters' autographs} \citep{Klein1887} p51 \dotfill p\pageref{rulebreaker}\\
\phantom{\textit{Greek vases with masters' autographs} \citep{Klein1887}} p194.2 \dotfill p\pageref{kleinx2}\\
\phantom{\textit{Greek vases with masters' autographs} \citep{Klein1887}} p195.3 \dotfill p\pageref{kleinx2}\\
\textit{Greek vases with kalos inscriptions} \citep{Klein1890} p68 \dotfill p\pageref{Klein68}\\
\textit{Naukratis}, by Flinders Petrie, I, inscription 303 \dotfill p\pageref{naukratisx2}\\
\phantom{\textit{Naukratis}, by Flinders Petrie,} I, inscription 307 \dotfill p\pageref{naukratisx2}\\
\phantom{\textit{Naukratis}, by Flinders Petrie,} II, inscription 750 \dotfill p\pageref{Naukratis2750}\\
Plautus, \textit{Poenulus} 1258 \dotfill p\pageref{formulaeWithDi13}\\
\phantom{Plautus,} \textit{Mercator} 774\footnote{\textit{Translator's note:} Wackernagel here writes \textit{Bacchides} 1258 and \textit{Mercator} 784, but these must both be errors.} \dotfill p\pageref{mercator}

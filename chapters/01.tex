\chapter{Introduction}
\author{George Walkden\affiliation{University of Konstanz}\orcid{0000-0001-5950-9686}}

Not every linguist has a law named after them, but, even among those who do, Jacob Wackernagel is exceptional. First, his law is one of very few (especially from the nineteenth century) that are \emph{syntactic} in nature, having to do with the relative ordering of words. Secondly, it differs from the commonly recognized sound laws (e.g. those of Grimm,\ia{Grimm, Jacob} Verner,\ia{Verner, Karl} Grassmann\ia{Grassmann, Hermann} and Holtzmann;\ia{Holtzmann, Adolf} see \citealp{Collinge1985} for an overview) in that its scope is tremendous: far from being a single, punctual event as were the sound laws of history under the Neogrammarian\is{Neogrammarians} conception \citep{OsthoffBrugmann1878}, Wackernagel's law (he argues) left its traces in pretty much all of the Indo-European languages, even if its status as a synchronic principle of grammatical organization varies substantially. Thirdly, and relatedly, Wackernagel's law is still the subject of active research today among specialists in various languages, far beyond the Indo-European family which provided the context for the original law. That this is the case can be seen from the nearly 700 Google Scholar citations that Wackernagel's \citeyearpar{Wackernagel1892} hundred-page article has accrued by the date of writing. Wackernagel's law can safely be said to have entered the coveted realm of being `more cited than read'.

This introduction has three aims. In the following section we provide a brief biographical sketch, along with a quick summary of the article and a concise statement of the law itself. Section \ref{reception-implications} discusses the law's subsequent reception from publication until the present day, again without pretence of being exhaustive. Section \ref{notes-translation} outlines our rationale for, and the decisions we have made during, the translation process.

\section{Jacob Wackernagel and his law of Indo-European word order}\label{intro-biography}

\subsection{Jacob Wackernagel (1853--1938)}
Jacob Wackernagel was born in \isi{Basel}, Switzerland, in 1853, to a wealthy and academically-inclined family. Between 1872 and 1874 he studied at \isi{Göttingen} under the Indologist Theodor Benfey,\ia{Benfey, Theodor} arguably the figure with the most influence on Wackernagel's own views and scholarship. Like many of the philological luminaries of the time, his studies took him to Leipzig, where in 1874--75 he took classes with the prolific and powerful Georg Curtius\ia{Curtius, Georg} and the Neogrammarian\is{Neogrammarians} founder-figure August Leskien.\ia{Leskien, August} Shortly after this he returned to \isi{Basel}, where from 1876 he taught \ili{Greek} and \ili{Sanskrit}, and in 1879 he was appointed Professor of Greek, this chair having been vacated by the philosopher Friedrich Nietzsche.\ia{Nietszche, Friedrich} \isi{Basel} was where he would spend the rest of his academic career, with the exception of the years 1902--1915, when he occupied the Chair of Comparative Philology at \isi{Göttingen}.

Wackernagel's publications for the most part focused on ancient and historical Greek,\il{Greek, Ancient} especially in the first half of his career: these include two book-length works, \emph{Über einige antike Anredeformen} (`On some forms of address in antiquity'; \citeyear{Wackernagel1912}) and \emph{Sprachliche Untersuchungen zu Homer} (`Linguistic investigations of Homer'; \citeyear{Wackernagel1916}). \ili{Sanskrit} was by no means neglected, however: his \ili{Sanskrit} grammar (\emph{Altindische Grammatik}) was his magnum opus \citep[x]{Langslow2009}, though only the first volume \citeyearpar{Wackernagel1896} and the first part of the second \citeyearpar{Wackernagel1905} were published during his lifetime. In 1936 he retired, and two years later, in 1938, he died, at the age of eighty-four. More detailed biographical treatments of Wackernagel can be found in \citet{Schwyzer1938}, \citet{Schlerath1990}, \citet[viii--xviii]{Langslow2009}, and in particular \citet{Schmitt1990}.

\subsection{Wackernagel's scholarship}

On the whole, Wackernagel's attention was focused on concrete problems in the history or prehistory of specific Indo-European languages. He seldom wrote on general linguistic issues, with the most important exception being his two-volume \emph{Vorlesungen über Syntax}\is{Lectures on Syntax} (`Lectures on Syntax'; \citeyear{Wackernagel1920,Wackernagel1924}), recently translated into \ili{English} \citep{Langslow2009}. Despite its name, this work is more focused on the nature and properties of morphological categories than on syntax proper.\footnote{Delimiting the domain of syntax was a hot topic at the time Wackernagel was writing: \citet{Ries1894} in particular had opened up controversy. Wackernagel was fully aware of the limitations of his treatment of syntax and planned to address it in a third volume, which unfortunately never saw the light of day.}\is{Lectures on Syntax}
Nor did he devote much attention to comparative Indo-European linguistics \emph{per se}: only \ili{Sanskrit}, \ili{Greek}, \ili{Latin} and Iranian\il{Indo-Iranian} featured in the titles of his published works and the courses he taught \citep[xi]{Langslow2009}. The article featured in the present book \citep{Wackernagel1892} is thus quite exceptional in its scope and generality.

The article is heavily dominated by discussion of \ili{Greek} data: the first seven sections and 70 of 104 pages are devoted almost exclusively to \ili{Greek}. Wackernagel turns his attention to \ili{Indo-Iranian} in section VIII, closing with some suggestive remarks on \ili{Germanic} (modern \ili{German} and \ili{Gothic}). Section IX starts with some similarly tentative comments on \ili{Celtic}, but quickly moves on to \ili{Latin}, which also occupies sections X and XI. From a comparative or general linguistic perspective, however, section XII -- the final section, comprising the last ten pages -- is the most immediately rewarding. Here Wackernagel engages with the modern \ili{German} evidence in more detail, and discusses the scope of his theory and the diachronic development of the Indo-European daughter languages, especially as regards the position of finite verbs.\is{verb position}

\subsection{Wackernagel's law}

Wackernagel's law is given in (\ref{law}).\footnote{\citet[218]{Collinge1985} notes that Wackernagel himself did not claim credit for the law, instead crediting it in the first volume of his \textit{Lectures}\is{Lectures on Syntax} to \citet{Delbrueck1878} on \ili{Sanskrit} (see \citealp[57]{Langslow2009}). \citeauthor{Collinge1985} therefore suggests that the law should be called `the law of Delbrück and Wackernagel'. Since it was Wackernagel who established the wider validity of such a law outside \ili{Sanskrit} alone, we have retained the traditional attribution here.} For other overviews of the law, its scope and validity, see \citet[218--219]{Collinge1985}, \citet{Krisch1990}, and \citet{Goldstein2014}.

\begin{exe}
\ex\label{law} \textbf{Wackernagel's law}\\
Enclitics\is{enclitics} occupy second position.
\end{exe}

This simple statement immediately raises a number of related issues: i) Which languages or varieties does the law in (\ref{law}) apply to? ii) What elements count as enclitics\is{enclitics} in these varieties? iii) What does `second position' mean more precisely? iv) Why would such a law hold?

The article is devoted primarily to answering i) and ii). As regards i), Wackernagel is clear that the law's effects can be found in \ili{Greek} (particularly Homeric Greek,\il{Greek, Homeric} with traces of the law to be found at later stages too), \ili{Latin}, and \ili{Sanskrit}, and on this basis concludes that it must have held in the ancestor language, \ili{Proto-Indo-European}, as well. On \ili{Germanic} and \ili{Celtic} he is more tentative. The answer to ii) is extensional: a non-exhaustive list made up primarily of \isi{particles} and \isi{pronouns}, some of which, Wackernagel notes, are more prototypical than others.

iii) would be seen as crucial by most present-day linguists,\footnote{Cf. \citet[18--20]{Zwicky1977} and \citet[72--73]{Anderson1993}. \citet{Halpern1995}, for instance, makes the case that there exist both 2W systems, in which enclitics\is{enclitics} follow the first word, and 2D systems, in which enclitics\is{enclitics} follow the first constituent.} but Wackernagel is not particularly explicit on this point (cf. \citealp[11]{AzizHanna2015}). The obvious answer is that second position is counted in terms of words; however, though most of Wackernagel's examples can all be characterized in this way, not all of them can. Though not operating with anything like a modern constituency or dependency grammar, Wackernagel does employ the notion of \textit{Wortgruppe} `word group', and discusses relations between words. Yet `constituent' or `semantic unit' does not seem to be the appropriate way to understand the second-position requirement either. The waters are muddied still further by Wackernagel's discussion (at the end of section VIII) of examples from \ili{Gothic} in which \textit{word-internal} second position appears to be crucial, e.g. \ili{Gothic} \textit{ga-\textbf{u}-laubeis} `do you believe?', with the interrogative\is{interrogatives} morpheme -\textit{u}- occurring after the first morpheme of the verbal form. Finally, Wackernagel is also not very clear about the \emph{domain} over which the law holds: he most often uses the word \emph{Satz} (`clause'), but he is flexible as to where clause boundaries actually lie, and this is one area in which later linguists (e.g. \citealp{Fraenkel1932,Fraenkel1933,Fraenkel1965,Ruijgh1990}) have sought to improve on Wackernagel's formulation. To some extent, then, second position for Wackernagel is a flexible notion.

Despite this uncertainty, Wackernagel's precision and level of detail when discussing the examples themselves can hardly be called into question. Sometimes (e.g. \citealp[24]{HarrisCampbell1995}) Wackernagel's law is framed as a tendency. For Wackernagel himself, though, it was clearly not intended to be understood in this way. The close attention paid in every section to potential counterexamples -- and the effort expended in trying to explain them away -- is more reminiscent of the modern theoretical linguist's modus operandi than of the cataloguing and quantification usually associated with Wackernagel's contemporaries (e.g. \citet{Ries1880}, \citet{Behaghel}). Moreover, given the use of the word \textit{Gesetz} `law' in the article's title, and given that Wackernagel would have been well aware of how the term had been appropriated by the Neogrammarians\is{Neogrammarians} for exceptionless generalizations (e.g. \citealp{OsthoffBrugmann1878}), it would have been bizarre for Wackernagel to aim for anything else, even though he himself never identified as a Neogrammarian.\footnote{We thus fundamentally disagree with \citet[250--251]{AzizHanna2015}, who claims that Wackernagel never intended his law as a \textit{Regel} `rule'. The fact that Wackernagel attempts to explain away counterexamples where possible, and the fact that he himself uses the term \textit{Regel} `rule' at several points in the article, both militate against this interpretation.}\is{Neogrammarians} Clearly, though, Wackernagel is ready to concede that the law is not equally operative in all the diachronic stages of the languages in question, and this may be the reason why more recent linguists have attempted to water down his statement of the law.

Wackernagel also shares with the Neogrammarians\is{Neogrammarians} (and with probably the majority of modern linguists) an approach to linguistic generalizations that is mentalist\is{mentalism} at its core. We see this, for instance, in his use of the term \textit{Stellungsgefühl} `position-feeling', with its echoes of Wundtian psychologism (see recently \citealp{Fortis2019} on the notion of \textit{Formgefühl} `form-feeling' and its use by \citealp{Wundt1874}), even if this mentalism is rarely at the forefront of Wackernagel's article. We also see Wackernagel's mentalism, for instance, in his \textit{Lectures on Syntax},\is{Lectures on Syntax} where in the first volume he distinguishes three types of linguistic relatedness: the first is ``based on human nature, on general laws of the human psyche, fundamental relatedness'' \citep[11]{Langslow2009}, giving rise to syntactic features that are `best described precisely in terms of their universality'.\is{Lectures on Syntax}\is{mentalism}\is{universals}

As to iv) -- the `why'-question -- Wackernagel hints at an answer without really spelling it out: enclitics\is{enclitics} are unstressed, and it was this property that led them to occupy the second position.\footnote{\citet[294--295]{Hale2017} suggests that Wackernagel's reasoning is based on Optimality-Theory-style\is{Optimality Theory} competing motivations: there is a drive for enclitics\is{enclitics} to be initial, but they cannot occupy absolute initial position because that requires them to be stressed. Hence they occupy second position as a compromise.} Wackernagel reaches this conclusion in section XII,\is{verb position|(} where he aims to account for the disparity in modern \ili{German} between verbs in main and subordinate\is{subordination} clauses. The suggestion is that the basic position of the verb was final, and that at an earlier stage verbs in main clauses `moved' (\textit{rückten}) to second position\is{verb-second} in order to be unstressed.\footnote{This movement-based account foreshadows early transformational proposals for \ili{German} such as that of \citet{Bach1962} by seventy years.} Here Wackernagel also explores a more restrictive version of his law, in which only mono- or disyllabic verb forms were affected. (Later the rule became purely syntactic, and affected all verb forms in main clauses, whether stressed or not.)\is{verb position|)}

If Wackernagel's explanation for his law is fundamentally prosodic, then it differs in a crucial way from more recent proposals that have sought to build on Wackernagel's insights. It is to the legacy of his law that we now turn.

\section{Reception and implications}\label{reception-implications}

Wackernagel's law has been described by Calvert Watkins -- himself a key figure in the understanding of Indo-European syntax -- as ``[o]ne of the few generally accepted syntactic statements about I[ndo-]E[uropean]'' \citep[1036]{Watkins1964}. Writing in the early 1990s, Alice Harris \& Lyle Campbell likewise call it ``one of the firmest discoveries in the history of syntactic change'' \citep[29]{HarrisCampbell1995}, and \citet{Krisch1990} describes it as ``perhaps the only word order rule for Indo-European which has remained undisputed in its essentials since its discovery''.\footnote{``Das Wackernagelsche Gesetz ist die vielleicht einzige in ihren Grundzügen von ihrer Entdeckung bis heute immer unumstrittene Wortstellungsregel für das Indogermanische'' \citep[64--65]{Krisch1990}.} For more than a hundred years, Wackernagel's law was taken to be a robust generalization about the history of Indo-European syntax. Even more importantly, perhaps, the article triggered an outpouring of research into (en)clitics\is{enclitics} and the relation between syntax and prosody that has showed no signs of abating in recent years. An overview of the first century of this work can be found in the bibliography of \citet{NevisEtal1994}, supplemented by \citet{Janse1994}, and the papers in \citet{EichnerRix1990} and \citet{HalpernZwicky1996}. Particularly in the early 1980s, with the simultaneous flourishing of theoretical studies on the syntax-prosody interface\is{syntax-prosody interface} (e.g. \citealp{Klavans1982,Kaisse1985,Selkirk1984,Selkirk1986,NesporVogel1986})
and on cross-linguistic comparative syntax in the \isi{Principles and Parameters} mould (e.g. \citealp{Chomsky1981,Rizzi1982,Hale1983}), a cottage industry of clitic studies developed, which in the 21st century can safely be said to have lost its cottage status and developed into full-scale heavy industry. In this section we first detail the reception of Wackernagel's law within Indo-European studies, then discuss its more general relevance and implications during the latter part of the 20th century, before finishing with an examination of some more critical voices.

\subsection{Wackernagel's Law in Indo-European (1892--1990)}

The impact of Wackernagel's article within Indo-European studies and historical linguistics was tremendous from the beginning, and follow-up studies soon showed that other languages and varieties conformed to the same pattern that Wackernagel had identified. 

\citet{Nilsson1904} brings in \ili{Slavic} varieties such as Old Bulgarian\il{Bulgarian, Old} as well as varieties of modern \ili{Polish}, aiming to show that Wackernagel's law applies here too. 
\citet{Ivanov1958} argued that Wackernagel's law was relevant also to \ili{Lithuanian}, and to \ili{Hittite} and \ili{Tocharian}, which had not yet been discovered at the time Wackernagel was writing (see also \citealp{Carruba1969,Hoffner1973,Garrett1990,Luraghi1998} on \ili{Anatolian}). As regards \ili{Celtic} linguistics, the distinctive VSO\is{verb position} order found in the Insular Celtic\il{Celtic, Insular} languages is explained by \citet{Watkins1963}, building on \citet{Vendryes1912} and \citet{Dillon1947}, as closely linked to Wackernagel's law: certain enclitics\is{enclitics} had a close relationship with the verb, and drew it along to the beginning of the sentence as a host, resulting in verb-initial clauses.\is{verb position}

\citet{Thurneysen1892},\is{verb-second|(} who explicitly credits Wackernagel with the impetus to finish and publish his study, adduces word order evidence from Old French\il{French, Old} and connects its verb positioning\is{verb position|(} to Wackernagel's law; this paper has itself been extremely influential within historical \il{Romance} linguistics, spawning a substantial literature on clitic \isi{pronouns} (see e.g. \citealp{Wanner1987} and \citealp{Fontana1993} for historical perspectives) and verb position (recently for instance \citealp{Kaiser2002,Wolfe2018}). 

Within \ili{Germanic} linguistics in particular, the focus during this period was on something that Wackernagel himself had addressed only tentatively: verb-second and the position of the finite verb. \citet[315--318]{Ries1907} investigates word order in Beowulf and finds some support for Wackernagel's claims about the position of unstressed verbs, at least for auxiliaries and modals, but does not accept his diachronic reconstruction of asymmetric verb positioning for \ili{Proto-Germanic} or \ili{Proto-Indo-European}.\footnote{\citet[15--16]{Hopper1975} claims that \citet{Ries1907} and \citet{Delbrueck1907} both supported Wackernagel's view. In fact, neither of them did, at least as regards the specifics of the diachronic development.} \citet{Kuhn1933}\is{Kuhn's laws|(} built on Wackernagel through an empirical investigation of poetic\is{poetry} texts from Old English,\il{English, Old} Old Norse\il{Norse, Old} and Old Saxon.\il{Saxon, Old} He proposed two further laws: the \ili{Germanic} \emph{Satzpartikelgesetz} (clausal particle law) states that ``clausal \isi{particles} occur in the first dip in the clause, proclitic\is{proclitics} to either its first or second stressed word'' \citep[8]{Kuhn1933}, and the \ili{Germanic} \emph{Satzspitzengesetz} (clause-initial law), stating that ``there must be clausal \isi{particles} in an initial dip'' \citep[43]{Kuhn1933}. While Kuhn's second law is nowadays mostly considered to have been falsified \citep{Momma1997,Mines2002}, Kuhn's first law remains influential. \citet{Dewey2006}, for instance, posits a stage of `intonational verb-second' during which the placement of the finite verb in \ili{Germanic} was regulated primarily by prosodic considerations.\is{verb position|)}\is{verb-second|)}\is{Kuhn's laws|)}

Among the languages that were Wackernagel's main focus -- \ili{Greek}, and to a lesser extent \ili{Latin} and \ili{Sanskrit} -- research during this period primarily strove to make the law more precise and to test its predictions in different types of texts and grammatical contexts. Work in this vein includes \citet{Dover1960}, \citet{Marshall1987} and \citet{Ruijgh1990} for historical \ili{Greek}, \citet{Marouzeau1907,Marouzeau1953} and \citet{Fraenkel1932,Fraenkel1933,Fraenkel1965} for \ili{Latin}, and \citet{Hale1987PhD,Hale1987wackernagel} and \citet{Krisch1990} for \ili{Sanskrit}.

Not everyone was uniformly positive. \citet[81--83]{Delbrueck1900}, while accepting Wackernagel's findings on enclitic\is{enclitics} positioning in general, argued against Wackernagel's view that the verb\is{verb position} occupied second\is{verb-second} position in main clauses in \ili{Proto-Indo-European}, since, he argued, verbs in Indo-European were in general weakly stressed rather than entirely unstressed.\footnote{\citet[81]{Delbrueck1900} somewhat mischaracterizes \citet{Wackernagel1892} when he claims that the latter argued for a subject-verb word order: \citet{Wackernagel1892} is silent on the issue of what element occupies first position.} In his review of \citet{Ries1907}, he takes a similar but not identical position: in \ili{Proto-Indo-European}, verbs were unstressed in main clauses and stressed in subordinate\is{subordination} clauses, but their basic position was final in both cases; the development of asymmetric verb positioning as in modern \ili{German} belonged to \ili{Germanic} times \citep[75--76]{Delbrueck1907}.\is{verb position}

Work on Wackernagel's law in historically-attested Indo-European languages evidently did not stop with \citet{Watkins1964} or with the papers in \citet{EichnerRix1990}. However, the 1970s and 1980s gave the law a new lease of life by extending its linguistic range, and it is to this development that we now turn.

\subsection{Wackernagel, clitics, and the syntax-prosody interface (1977--present)}\is{syntax-prosody interface|(}

Although Wackernagel did have a concept of linguistic universals,\is{universals} it evidently did not occur to him to think of his law as universal, or as a reflex of universal pressures.\is{universals} This suggestion was first made much later, by \citet[613]{Kurylowicz1958}, in a commentary on \citet{Ivanov1958}, and was not really taken seriously at the time (cf. \citealp[1036]{Watkins1964}). It was not until the flowering of work on clitics and prosody in generative linguistics of the late 1970s and particularly the 1980s that this line of thinking came to be pursued more systematically.\footnote{\citet{Wackernagel1892} nowhere uses the simple term `clitic', referring only to enclitics\is{enclitics} (\textit{Enklitika}). The generalization of the term `clitic' to refer to both proclitics\is{proclitics} and enclitics\is{enclitics} in the modern sense seems to be due to \citet[155]{Nida1946} \citep{Haspelmath2015}.}

Important early work by \citet{Steele1975} on constituent order typology identified a category of languages in which modals consistently occupy clausal second position;\is{verb-second} \citeauthor{Steele1975} links this to Wackernagel's law. On the basis of \ili{Uto-Aztecan} data, \citet{Steele1977} suggests a diachronic relation between Wackernagel's Law and topicalization (cf. also \citealp{Hock1982}). In both cases, the forces at work must necessarily be active far beyond Indo-European. 

The decisive push towards more explicit theorizing of clitics came from \citet{Zwicky1977}. During the 1970s, with the rise of morphology as a separate domain in generative theorizing, clitics were occasionally alluded to as a challenge due to their apparently intermediate nature between bound and free forms (\citealp[166--169]{Matthews1974}, \citealp[3-4]{Aronoff1976}), on the borderline between the morphological and syntactic components of the grammar. \citet{Zwicky1977} draws a distinction between three types of clitic:

\begin{enumerate}
    \item \textbf{Special clitics}:\is{special clitics} clitics that show unusual syntactic behaviour and unusual phonological alternations as compared to their stressed free-form counterparts
    \item \textbf{Simple clitics}: clitics that behave syntactically like their stressed free-form counterparts and are related to them through a general phonological rule
    \item \textbf{Bound words}: clitics with no stressed free-form counterparts, which can be associated with words of various morphosyntactic categories
\end{enumerate}

\citet[9]{Zwicky1977} is also responsible for introducing crucial terminology in the study of clitics such as \textbf{host} (the word to which a clitic is attached)\footnote{\citet[note 5]{Zwicky1977} attributes the term to Hetzron (p.c.).} and \textbf{group} (the host plus all of its clitics). Second-position clitics and Wackernagel's law also receive discussion. In fact, virtually all of the theoretical issues that more recent research on clitics has addressed are raised -- if only briefly -- in \citeauthor{Zwicky1977}'s relatively short paper, including clitic positioning with respect to the host (pro-, en- or endoclitic\footnote{\citeauthor{Zwicky1977} uses the term `endoclitic'\is{endoclitics} to refer to clitics that are word-internal but placed at morpheme boundaries. In more recent research the usual term for this is `mesoclitic',\is{mesoclitics} with endoclitic\is{endoclitics} reserved for the much rarer phenomenon of clitics that disrupt the root of the host; see e.g. \citet{Smith2013}.}),\is{enclitics}\is{proclitics}\is{endoclitics} relative ordering of clitics within a group, the phonological relation of clitics to corresponding nonclitic forms, the phonological integration of clitics with their hosts, and more.

A few years later, \citet[283]{Zwicky1985} is able to speak of a ``recent flurry of work on clitics''. Important roughly contemporary contributions include \citet{Klavans1979,Klavans1982,Klavans1985},  \citet{Kaisse1982,Kaisse1985}, and \citet{ZwickyPullum1983}; the latter, for instance, provide a set of diagnostics for distinguishing clitics from inflectional affixes, while \citet{Zwicky1985} addresses the problem of distinguishing clitics from independent words. This flurry informed, and was informed by, more general proposals about prosody and the nature of the interface between syntax and phonology such as \citet{Selkirk1984,Selkirk1986} and \citet{NesporVogel1986}.\footnote{This is still a lively field today. To take just a few examples, \citet{Dehe2014} challenges prominent theories of the syntax-prosody interface using corpus data; \citet{Boegel2015} presents a full theory of the syntax-prosody interface within \isi{Lexical-Functional Grammar}; and \citet{Guenes2015} develops a derivational approach to prosody that is compatible with Minimalist\is{Minimalist Program} assumptions about syntactic structure-building and the interfaces.} \citet{Klavans1995} is a book-length treatment of clitics from the mid-1990s, contemporaneous with \citet{Halpern1995}, which deals with the placement of a set of second-position clitics through an operation of Prosodic Inversion at the syntax-prosody interface.

Another factor pushing Wackernagel's law back into the spotlight, during roughly the same period, was the expansion of cross-linguistic work in generative syntactic theory.\is{Principles and Parameters|(} \citet{Hale1973} on \ili{Warlpiri} and \citet{Kayne1975} on \ili{French} were two early works in this vein that engaged with the clitic question; however, with the advent of the Principles and Parameters research programme (\citealp{Chomsky1981,Chomsky1982,Borer1981,Rizzi1982}; see \citealp{Roberts1997} for an accessible introduction), comparative generative syntax expanded dramatically. In this approach, language can be characterized in terms of a set of universal,\is{universals} invariant cognitive principles alongside a set of discrete points of variation, the parameters. \citet{Hale1983} influentially proposed a Configurationality\is{configurationality} Parameter regulating the relation between syntax and the lexicon: one setting of this parameter allowed for `nonconfigurational' languages exhibiting relatively flexible orderings of constituents. Since Hale's theory was built upon \ili{Warlpiri}, a language with substantial constituent order flexibility and `Wackernagel' clitic auxiliaries, it is unsurprising that this kind of analysis has also been popular for early Indo-European languages (see \citealp{Ledgeway2012} for extensive discussion).\footnote{For Warlpiri, in the meantime, the idea of nonconfigurationality\is{configurationality} has been debunked \citep{Legate2002}, and at the current state of research it is not clear whether nonconfigurationality remains a useful notion in linguistic theory. See also \citet{Legate2008}, who shows, \textit{pace} \citeauthor{Hale1973}, that the notion of second position is not relevant to the Warlpiri clitic system, and that clitic placement is not conditioned by syllable structure, instead being best viewed as syntactic.} \citet{Borer1981}, \citet{Rivero1986} and the papers in \citet{Borer1986} present parametric approaches to cliticization in various languages.

Cross-pollination from Principles and Parameters can also be seen in contemporaneous theorizing about the typology of clitics. \citet{Klavans1985} develops a theory of clitic positioning based on three parameters: dominance (initial/final), precedence (before/after), and phonological liaison (proclitic/enclitic).\footnote{\citet{Klavans1979,Klavans1985} denies the existence of endoclisis\is{endoclitics} in the sense of \citet{Zwicky1977}. The present consensus seems to be that endoclisis is cross-linguistically rare but possible \citep{Harris2002,Smith2013}.}\is{enclitics}\is{proclitics}\is{endoclitics} This theory derives a version of Wackernagel's law \citep[117]{Klavans1985}.

Work in the 1990s and 2000s, by generative linguists and others, explored the morphology, phonology and syntax of clitics in a very wide range of languages (see e.g. \citealp{HalpernZwicky1996,BeukemaDenDikken2000,FranksKing2000,GerlachGrijzenhout2000,Boskovic2001,Anderson2005,Roberts2010,SpencerLuis2012,SalvesenHelland2013} for book-length treatments). Mention must be made of the now vast literature on clitics in \ili{Slavic} (particularly South Slavic)\il{Slavic, South} languages \citep{RadanovicKocic1988,RadanovicKocic1996,NevisJoseph1993,Schuetze1994,DimitrovaVulchanova1995,DimitrovaVulchanova1998,Progovac1996,Progovac2000,Tomic1996,Tomic2000,Franks1997,Franks2000,Franks2008,FranksBoskovic2001,FranksKing2000,Boskovic2000,Boskovic2001,Boskovic2002,Boskovic2016,Pancheva2005,Migdalski2010,Migdalski2012,Migdalski2016,DiesingZec2011,Harizanov2014,Despic2017} and in other languages of the Balkans (e.g. \citealp{Francu2009} and \citealp{AlboiuHill2012} on \ili{Romanian}).\footnote{\citet{Francu2009} proposes that Wackernagel's law was operative in historical \ili{Romanian}; \citet{AlboiuHill2012} make the case that it wasn't.}\is{Principles and Parameters|)}

The modern\is{verb position|(}\is{verb-second|(} understanding of Wackernagel and his insights has been shaped substantially by \citeauthor{Anderson1993}'s \citeyearpar{Anderson1993} influential paper `Wackernagel's revenge'. Here, \citeauthor{Anderson1993} picks up on the notion that there is a deep connection between clitic placement and verb-second constituent order. Since (he argues) clitic placement cannot be accounted for using syntactic approaches to verb-second, the picture ought to be reversed: verb-second should be accounted for using a technical apparatus developed for clitic phenomena. Following the \isi{morphological theory} developed in \citet{Anderson1992}, he proposes that (special)\is{special clitics} clitics are phrasal affixes, i.e. the reflex of word-formation rules applying to phrases. Verb-second is then derived using exactly such a rule, realizing the inflectional features of a clause in the position after its first constituent: movement of the verb is a byproduct of the need for these features to be spelled out affixally in second position (cf. recently \citealp{BayerFreitag2020}).\footnote{More recently the relation between second-position clitic systems and verb-second has also been explored in depth by \citet{Migdalski2010,Migdalski2016}. \citet{Boskovic2020} argues against a unification of verb-second and second-position clitics.} As \citeauthor{Anderson1993} acknowledges, his take on verb-second is substantially different from Wackernagel's in that he locates the explanatory action in morphology\is{morphological theory} rather than in prosody, and substantially different from the consensus among generative syntacticians in that he locates the explanatory action in morphology rather than in syntax.\is{verb position|)}\is{verb-second|)}

In \citet{Anderson2005} this perspective is further developed, along with a new typology of clitics, building on and replacing that of \citet{Zwicky1977}. For \citeauthor{Anderson2005}, the crucial distinction is between \textbf{simple} and \textbf{special} clitics:\is{special clitics} \citeauthor{Zwicky1977}'s category of bound words plays no role. Special clitics\is{special clitics} are those whose positioning is governed by a set of principles distinct from those regulating free forms. Crucially, for \citeauthor{Anderson2005} (unlike \citeauthor{Zwicky1977}), special clitics are purely morphosyntactically defined, and may or may not be phonological clitics. Simple clitics then are those phonological clitics that do \emph{not} display any aberrant morphosyntactic behaviour. This dichotomy has been adopted in a variety of subsequent work (see e.g. \citealp[95]{Boegel2015}).\footnote{Special clitics,\is{special clitics} although perhaps the most interesting type of clitics theoretically,\is{morphological theory} are not uncontroversial: see \citet{SpencerLuis2012} and particularly \citet{BermudezOteroPayne2011} for critical discussion.}\is{syntax-prosody interface|)}

Clitics and Wackernagel's findings also become relevant to general linguistics\is{grammaticalization|(} during the same period as part of grammaticalization theory. \citet{Givon1971}, in making the case that bound morphemes originate diachronically via cliticization of originally independent words, had effectively rediscovered the phenomenon of grammaticalization (\citealp{Meillet1912}; cf. also \citealp{Kurylowicz1965}). \citet{Lehmann2015}, first published in working-paper form in 1982 and in wider circulation from 1995 onwards, gave the programmatic impetus to researchers in this area. \citeauthor{Lehmann2015} describes the increase in bondedness that grammaticalizing items undergo as the first step of coalescence: ``the subordination\is{subordination} of the grammaticalized item under an adjacent accent, called cliticization'' \citep[157]{Lehmann2015}. Though the semantic,\is{semantic change} syntactic and pragmatic aspects of grammaticalization remain better studied than its phonological and morphological aspects, there are several works within grammaticalization theory on the cline \textsc{free word} > \textsc{clitic} > \textsc{affix}: \citet{Schiering2006,Schiering2010}, for instance, presents a cross-linguistic study of the process, showing that the overall phonological profile of the language significantly influences the ultimate fate of individual words and clitics.

The development from affix to clitic has also been taken as evidence for the existence of \isi{degrammaticalization}. \citet{Norde2001}, for instance, discusses the \ili{Swedish} possessive -\textit{s} in this connection. This -\textit{s} originated as a well-behaved morphological \isi{genitive} case ending, but in the Early Modern Swedish\il{Swedish, Early Modern} period appears to be a clitic marking possession, as it attaches at the end of a phrase, e.g. \textit{konungen i Danmarck\textbf{s} krigzfolck} `the king of Denmark's army'. In response, \citet{Boerjars2003} argues that the placement of an element must be distinguished from its attachment: \ili{Swedish} -\textit{s} is still an affix rather than a clitic, because it is attached as an affix, even though it is placed with respect to a phrase (cf. \citet{Anderson1993} on phrasal affixes, discussed above). \citeauthor{Boerjars2003} observes that true group genitives\is{genitive} in which the -\textbf{s} ending is found on an element other than a noun are few and far between, suggesting that the ending still has a strong preference to be attached to nouns. If -\textit{s} is not a clitic, then its development since Old Swedish\il{Swedish, Old} is not an instance of \isi{degrammaticalization}.\footnote{In response, \citet{Norde2010} downplays the importance of change in morphological status (`debonding'), arguing that other aspects also indicate that degrammaticalization has taken place.} This is not the only purported instance of the development clitic > affix, however: \citet{Kiparsky2012} lists many more, including the \ili{Setu} and \ili{Võru} (South \ili{Estonian}) abessive case suffix -\textit{lta}, which has become an abessive clitic. Debonding seems to exist, then, though the question remains why this direction of change appears to be rarer than the alternative. \citet{Kiparsky2012} suggests that such instances of degrammaticalization only occur under strong analogical\is{analogy} pressure (cf. \citealp{Plank1995}).\is{grammaticalization|)}

This section has shown that research on clitics and on the relationship between syntax, phonology and morphology has blossomed beyond anything that Wackernagel could have foreseen in 1892 -- both in terms of theoretical directions and in terms of languages investigated. Jacob Wackernagel undoubtedly deserves pride of place as progenitor of a large and fertile family of investigations. Closer to home, however, Wackernagel's law has been called into question for the very languages for which it was proposed, and this is the topic of the next subsection.

%Talk about Boskovic's theory of spellout?
%Compromise is interesting - see p399/67. Multiple placement of \emph{an} is a compromise between two pressures. Like multiple spellout!

\subsection{The clitics and the critics (1990--present)}
As we have seen, in summaries as late as the 1980s and 1990s Wackernagel's law is still presented as a robust generalization about early Indo-European languages (cf. also \citealp{Collinge1985}). However, writing in the early 2000s, \citet[168]{Clackson2007} observes that Wackernagel's Law ``now looks more problematic than it did forty years ago''.

The most robust challenge to Wackernagel's law is presented in a pair of works by \citet{Adams1994book,Adams1994pronouns}. Noting that Wackernagel's own treatment of the \ili{Latin} evidence was less than systematic, Adams starts by arguing, following \citet{Fraenkel1932,Fraenkel1933,Fraenkel1965}, that the proper domain for evaluation of Wackernagel's law is the `colon',\is{cola} not the clause, and that this allows a number of apparent exceptions to the law to be explained away.\footnote{The colon (plural \textit{cola}),\is{cola} a semantico-syntactico-phonologically independent unit, has never been particularly easy to define or to identify in historical texts. \citet{Scheppers2011} (on Ancient Greek)\il{Greek, Ancient} suggests that cola correspond to the intonation unit (IU) of discourse analysis. \citet[259--262]{Ledgeway2012} suggests that cola correspond to the phases of Minimalist\is{Minimalist Program} syntax: CP, \textit{v}P, PP and DP.} Even with this corrective, however, a striking number of exceptions are still found, leading Adams to propose that what has traditionally been viewed as Wackernagel's law (i.e. a second position requirement) in \ili{Latin} is in fact better viewed as an epiphenomenon of a different law requiring enclitics\is{enclitics} to be placed after a focalized or emphasized constituent, which itself may or may not be in first position. \citet{Adams1994book} explores this in relation to the \ili{Latin} enclitic\is{enclitics} \isi{copula} \textit{esse}, while \citet{Adams1994pronouns} presents a parallel study on unstressed personal\is{personal pronouns} \isi{pronouns}. Adams draws his material from classical \ili{Latin} \isi{prose} texts; \citet{Kruschwitz2004} shows that Adams's conclusions also hold for the corpus of \ili{Latin} inscriptions.\is{inscriptions}

For \ili{Indo-Iranian}, too, the empirical picture that has emerged is substantially more complex than section VIII of \citet{Wackernagel1892} suggests. \citet{Hale1987PhD,Hale1987wackernagel,Hale1996}, \citet{Krisch1990}, and \citet{Hock1996} do not (like Adams) aim to supplant Wackernagel's law entirely, but their work has nevertheless led to a picture in which the law must be relativized to particular syntactic positions or configurations. More recent contributions to the debate on clitics in \ili{Sanskrit} include \citet{Keydana2011}, \citet{Lowe2014} and \citet{Hale2017}, the latter stating that ``the empirical data for these languages is relatively poorly understood ... even in the specialist literature'' \citeyearpar[290]{Hale2017}. \citet{Keydana2011}, for instance, argues that Wackernagel clitics are not a homogeneous bunch, and can be split into three different classes:

\begin{enumerate}
    \item WL1: enclitics\is{enclitics} that follow a \textit{wh}-word\is{interrogatives} if one is present, but otherwise occupy second position in a sentence.
    \item WL2: clitics that always follow the first word of a sentence.
    \item WL3: clitics hosted by the element they take scope over.
\end{enumerate}

While WL1 clitics and WL2 clitics can in some sense be said to be `true' second-position clitics, WL3 clitics behave like the elements identified by \citet{Adams1994book,Adams1994pronouns} in that they are always enclitic\is{enclitics} to a particular constituent with a particular information-structural role, which does not have to be clause-initial. Moreover, following \citet{Hale1987wackernagel,Hale1987PhD}, most authors working on \ili{Sanskrit} clitics and second position have acknowledged that there is a discourse-functional syntactic position in the clausal left periphery that is somehow `outside' the clause proper and hence `does not count' for the positioning of certain enclitics\is{enclitics} (\citeauthor{Keydana2011}'s WL1 elements). The literature on Wackernagel's law in \ili{Indo-Iranian} is by now too large to be done justice to here, but it is worth noting that some of this work is explicitly concerned with the implications of these facts for the architecture of the grammar, and with finding the right division of labour between prosodic mechanisms, syntactic mechanisms, and brute-force stipulation, rather than simply describing the facts. Were Wackernagel alive today, it might well take him some time to see the connection between his simple law and the theoretically and empirically far more nuanced picture found in this recent work. In this sense, Wackernagel's law in its narrow sense can be said to have been falsified for \ili{Indo-Iranian} too.

Even in Ancient Greek,\il{Greek, Ancient} the variety most intensively investigated by Wackernagel, complexities arise that are not obviously captured in terms of a single second-position law. \citet{Taylor1990} argues that Wackernagel's law in its usual formulation does not account for Ancient Greek:\il{Greek, Ancient} unlike e.g. \citet{Dover1960} and \citet{Marshall1987}, it is necessary to take syntactic (constituent) structure into account in order to arrive at the correct statement of the generalizations. Moreover, once again, different clitics exhibit different behaviours. \citet[80--84]{Goldstein2016} shows, for instance, that the discourse \isi{particles} \textit{de} `but, and' and \textit{gar} `for', both described as `sentence-domain' clitics, do not occur in the usual position following the first \textit{prosodic} word, but instead show up after the first \textit{morphosyntactic} word, where other clitics such as the unstressed personal\is{personal pronouns} \isi{pronouns} behave more canonically. He also shows that there are instances in which \textit{de} and \textit{gar} appear to follow the first \textit{constituent}, rather than the first word. In one respect, though, \citet{Goldstein2016} actually maintains Wackernagel's law in a stronger form than Wackernagel himself: \textit{contra} e.g. \citet{Wackernagel1892} and \citet{Taylor1990}, Goldstein argues that the law was fully operative in the Classical Greek\il{Greek, Classical} period (6th--5th centuries BCE), and had not undergone a weakening since Homeric times.

The\is{verb position|(}\is{verb-second|(} Kuhn-Thurneysen-Wackernagel\ia{Kuhn, Hans}\ia{Thurneysen, Rudolf}\is{Kuhn's laws|(} hypothesis that \ili{Germanic} and Romance verb-second order has its origins in Wackernagel's law applied to finite verbs has also largely fallen out of favour in recent years. \citet[23--24, 315--318]{Ries1907} had already expressed scepticism, claiming that in the earliest texts there was no asymmetry between main and subordinate\is{subordination} clauses, and \citet{Fourquet1938} had been very critical about Kuhn's\ia{Kuhn, Hans} supposed laws. \citet[159]{Kiparsky1995} notes that finite verbs in second position in early \ili{Germanic} texts were (or at least could be) accented, thus rendering it unlikely that they were clitic elements. \citet[158]{Getty1997} goes further, arguing that ``the Wackernagel/Kuhn\ia{Kuhn, Hans} framework makes all the wrong predictions with respect to the behavior of finite verbs one can actually observe'', and that the crucial distinction instead seems to be between grammatical verbs (e.g. auxiliaries) and lexical verbs. Moreover, the question of how \ili{Germanic} moved from a 2W system, in Halpern's \citeyearpar{Halpern1995} terms -- in which the verb followed the first word -- to a 2D system in which it followed the first constituent is crucial, and has nowhere been addressed; there is no robust evidence for 2W verb-second anywhere in \ili{Germanic}.\is{Kuhn's laws|)} More recent accounts of the emergence of verb-second (e.g. \citet{HinterhoelzlPetrova2010,Walkden2012,Walkden2014book,Walkden2015,Walkden2017}) propose scenarios in which prosody plays no role, and in which the interplay between narrow syntax and information structure are central. As for \ili{Romance}, it has been debated whether the historically-attested languages are adequately characterized as verb-second at all. \citet{Kaiser2002} makes the case that they are not, while \citet{Wolfe2018} argues that they are. Neither author connects verb placement to prosody, however, and neither author argues for a strict linear second-position requirement.\is{verb position|)}\is{verb-second|)}

Strictly speaking, then, even given an appropriate definition of second position and the domain to which it applies, Wackernagel's law does not seem to hold at face value for \emph{any} of the Indo-European languages for which it was originally motivated. This hardly means that the proposal was a failure, though. On the contrary, \citet{Wackernagel1892} has been tremendously successful in stimulating research into clitics and second-position effects -- within and beyond the Indo-European languages -- even if an elegant, unified treatment is still lacking. At the very least, any theory of the prosody-syntax interface worth its salt will have to provide an account of the facts adduced by Jacob Wackernagel well over a century ago.
%In section 3, talk about Gothic -uh, Eythorsson and Ferraresi?


\section{Notes on the translation and edition}\label{notes-translation}

Our aim with this translation is to enable today's linguists to understand Wackernagel's argumentation without prior knowledge of any language other than \ili{English}. To that end, we've prioritized clarity over faithfulness, so that the translation is rather free. For instance, some of the \ili{English} linguistic terms used in the translation would not have been current in the \ili{English} of Wackernagel's time. Where possible we've tried to convey a sense of Wackernagel's rather idiosyncratic style, which jumps from stiffly legalistic to playful and back again within the space of a page. But this goal is secondary to conveying the linguistic point that he was trying to make. Those readers who are more interested in the history of language science or of philology should use this translation with care, and in conjunction with the \ili{German} original, which is also provided in Section \ref{original} of this book.

Wackernagel's original paper consisted of twelve numbered sections without names. For ease of navigation, we've added titles to these sections in the \ili{English} translation. We also indicate, both in the translation and in the original text, where the page boundaries were, and link between the two; in the translated version the positions of these markers are necessarily approximate given the free nature of the translation.

Referencing norms in Wackernagel's day were substantially looser than they are now, and Wackernagel in his paper took for granted the existence of a canon of texts in classical philology that all his intended readers would have been familiar with. A major part of preparing this translated edition consisted in tracking down these references, in the versions that Wackernagel himself would have had access to, and referencing them in the text according to modern norms (author, year, and -- where possible -- page). The availability of many nineteenth-century books and journals via the Internet Archive and Google Books greatly facilitated this task. Where it is ambiguous which edition of a given text Wackernagel was intending to reference, we have assumed the most recent pre-1892 edition. All references from both the original and this introduction are given in full in the bibliography at the end of the volume.

The edition of the \ili{German} text provided attempts to be as faithful to the original typesetting as possible. Where the original contains something ungrammatical or questionable, we have marked this with a following [sic].

I (George) initially started this translation as a solo project, but it quickly became clear that the translation of the \ili{German} on its own, without glosses and translations for Wackernagel's many examples, would be about as useful as a chocolate teapot. Christina came on board at this point, and later also Morgan, and the decision was made to gloss and translate all examples of four words or more, except in particularly repetitive sections. None of us have Wackernagel's compendious knowledge of the early Indo-European languages, and so substantial help was needed here. Morgan and Christina prepared the \ili{Greek} examples, of which there are well over a thousand. In translating the \ili{Greek} examples, we have made reference to the previous translations available through the Perseus site (\url{http://www.perseus.tufts.edu/hopper/}), and where necessary other sources such as \citet{LobelPage1968}; while we accept full responsibility for the translations presented here, in some cases it was not considered possible to improve upon the wording of an earlier translation. When Wackernagel's rendition of an example differs from that found in modern editions, this is mentioned in a footnote.

I'd like to thank Morgan and Christina for joining the team and putting in so much of their time and effort. We also offer profuse thanks to Moreno Mitrović\ia{Mitrović, Moreno} for help with the \ili{Sanskrit} examples, to Robin Meyer\ia{Meyer, Robin} for the Old Persian\il{Persian, Old} examples, and to Christoph Dreier and Thomas Konrad for the \ili{Latin} examples. Tina Bögel\ia{Bögel, Tina} provided valuable comments on this introduction. For help with tracking down rare books, we also gratefully acknowledge the help of Samuel Andersson,\ia{Andersson, Samuel} Lieven Danckaert,\ia{Danckaert, Lieven} Deepthi Gopal,\ia{Gopal, Deepthi} and Bettelou Los;\ia{Los, Bettelou} Lieven also helped out with a number of translations of \ili{Latin} quotations from secondary literature, and Laura Grestenberger provided useful feedback on part of the translation. The new edition of the original text was prepared and typeset by Anabel Roschmann. Thanks to everyone for the team effort!

This book is dedicated to my dad, Bob Walkden -- I've learned more about what it means to be a translator from him than from anyone else, and long before that he was helping me to learn how to be a person. Thanks, Dad!
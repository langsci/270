\chapter{Original text}\label{original}
\begin{otherlanguage*}{german}
\emph{\Large{Über ein Gesetz der indogermanischen Wortstellung}}\Footnote{1}{In den nachfolgenden Beispielsammlungen verdanke ich vieles den bekannten Hauptwerken über griechische Grammatik, sowie den Spezialwörterbüchern, ohne dass ich im einzelnen meine Gewährsmänner immer werde nennen können. Monros Grammar of the Homeric Dialect 2. Aufl., wo S. 335--338 über homerische Wortstellung Bemerkungen gegeben sind, die sich mit meinen Aufstellungen sehr nahe berühren, konnte ich nur flüchtig, Gehrings Index Homericus (Leipzig 1891) gar nicht mehr benützen.}

\section*{I.}
\addcontentsline{toc}{section}{I.}

\hypertarget{p333}{\emph{[S. 333]}}\label{p333} Albert Thumb hat vor vier Jahren in den Jahrbüchern für Philologie CXXXV 641--648 die Behauptung aufgestellt, die griechischen Pronominalakkusative μιν und νιν seien durch Verschmelzung von Partikeln mit dem alten Akkusativ des Pronominalstammes \emph{i} entstanden. Insbesondere das ionische μιν beruhe auf der Verbindung von \emph{im} mit einer Partikel \emph{ma}, älter \textit{sma}, die in thessalischem μα und altindischem \textit{sma} belegt sei. Den Hauptbeweis für diese Deutung entnimmt Thumb der angeblichen Thatsache, dass die Stellung von μιν bei Homer wesentlich dieselbe sei wie die Stellung von \emph{sma} im Rigveda. Es sei eben, auch nachdem der selbständige Gebrauch von \emph{sma} als Partikel geschwunden sei und μιν durchaus die Geltung einer einheitlichen Pronominalform erlangt habe, doch an μιν die für \emph{sma} gültig gewesene Stellungsregel haften geblieben, und es habe ein entsprechendes Stellungsgefühl dessen Anwendung begleitet. Und jedenfalls bei den Verfassern der homerischen Gedichte sei dieses Gefühl noch wirksam gewesen.

Nun beschränkt sich aber diese Ähnlichkeit der Stellung, wenn man das von Thumb beigebrachte Material nach den von ihm aufgestellten Gesichtspunkten unbefangen durchmustert, wesentlich darauf, dass μιν wie \emph{sma} im ganzen selten (genau genommen noch viel seltener als \emph{sma}) unmittelbar hinter Nomina und Adverbien nominalen Ursprungs steht. Und dieser allgemeinen farblosen Ähnlichkeit stehen wesentliche Abweichungen gegenüber. Zwar ist es ein seltsamer Irrtum Thumbs, wenn er zu dem zehnmaligen μή μιν Homers das \hypertarget{p334}{\emph{[S. 334]}}\label{p334} nach seiner Hypothese diesem im Altindischen entsprechende \emph{mā sma} daselbst nicht aufzutreiben weiss, da doch nicht nur Böhtlingk-Roth (s. v. \emph{mā}~9) zahlreiche Beispiele aufführen, darunter eines aus dem Rigveda (10, 272, 24 \emph{mā́ smāitādŕ̥g ápa gūhaḥ samaryḗ}), sondern es auch gerade über Bedeutung und Form der Präterita hinter \emph{mā sma} eine bekannte Regel der Sanskritgrammatik gibt (Panini~3, 3, 176. 6, 4, 74. Vgl. Benfey Vollst. Gramm. §~808~I Bem.~4). Aber in andern Fällen ist die Divergenz zwischen μιν und \emph{sma} thatsächlich. Nach
Thumb findet sich μιν bei Homer ca. 60 mal, in 10\% aller Belege, hinter subordinierenden Partikeln; \emph{sma} im Rigveda in solcher Weise nur selten und nur hinter \emph{yathā}. Und während \emph{sma} gern hinter Präpositionen steht, findet sich μιν nie hinter solchen.

Freilich will Thumb diese Abweichung daraus erklären,
dass die homerische Sprache es nicht liebe zwischen Präposition und Substantiv noch eine Partikel einzuschieben. Ja er wagt sogar die kühne Behauptung, dass in Rücksicht hierauf diese Abweichung seine Theorie geradezu stütze. Ich gestehe offen, dass ich diese Erklärung nicht verstehe. Wo \emph{sma} im Rigveda auf eine Präposition folgt, steht diese entweder als Verbalpräposition in tmesi (so wohl auch l, 51, 12 \emph{ā́ smā rátham -- tiṣ̌ṭhasi}, vgl. Grassmann Sp. 1598) oder, wenn überhaupt Fälle dieser zweiten Art belegt sind, in `Anastrophe'. Wenn also μιν die Stellungsgewohnheit von \emph{sma} teilt, so dürfen wir es nicht hinter den mit einem Kasus verbundenen Präpositionen suchen, und wenn es hier fehlt, dies nicht mit jener angeblichen homerischen Abneigung gegen Zwischenschiebung von Partikeln entschuldigen, sondern müssen es hinter selbständigen Präpositionen erwarten und in dem Umstand, dass es hier fehlt, eben einen Gegenbeweis gegen Thumbs Aufstellung erkennen.

Aber auch abgesehen von diesen und sonst etwa noch erwähnbaren Differenzen zwischen der Stellung des homerischen μιν und des vedischen \emph{sma}, war Thumb meines Erachtens verpflichtet zu untersuchen, ob sich die Stellung von μιν im homerischen Satz nicht auch noch von einem andern Gesichtspunkt aus, als dem der Qualität des vorausgehenden Wortes, bestimmen lasse, und ob ähnliche Stellungsgewohnheiten wie bei μιν sich nicht auch bei andern (etwa be\-deutungs-\hypertarget{p335}{\emph{[S. 335]}}\label{p335}verwandten oder formähnlichen) Wörtern finden, bei denen an Zusammenhang mit \emph{sma} nicht gedacht werden kann.

Und da scheint mir nun bemerkenswert, dass von den neun ‘vereinzelten’ Fällen, wo μιν auf ein nominales Adverb folgt, fünf (Ε~181. Ζ~173. Λ~479. Ο~160. δ~500) es an zweiter Stelle des Satzes haben, und dass ferner alle von Thumb aufgeführten Beispiele für μιν hinter dem Verb, dem Demonstrativum und den Negationen eben dasselbe zeigen. Von solcher Stellungsregel aus wird es nun auch verständlich, warum μιν so gern auf Partikeln und namentlich auch in Abweichung von \emph{sma} so gern auf subordinierende Partikeln folgt, und warum es ferner auf Pronomina wesentlich nur insofern unmittelbar folgt, als sie satzverknüpfend sind, also am Satzanfang stehen.

Oder um von anderm Standpunkt aus zu zählen, so bieten die Bücher Ν~Π~Ρ, die mit ihren 2465 Versen über die Sprache der ältern Teile der Ilias genügend Aufschluss geben können, μιν in folgenden Stellungen: 21 mal als zweites Wort des Satzes, 28 mal als drittes oder viertes, aber in der Weise, dass es vom ersten Wort nur durch ein Enklitikum oder eine den Enklitika gleichstehende Partikel, wie δέ, γάρ, getrennt ist. Dazu kommt εἰ καί μιν Ν~58 und τούνεκα καί μιν Ν~432, wo καί eng zum ersten Satzwort gehört; ἐπεὶ οὔ μιν Ρ~641, für welches die Neigung der Negationen im gleichen Satz stehende Enklitika auf sich folgen zu lassen in Betracht kommt (vgl. vorläufig οὔτιϲ, οὔπω, οὔ ποτε, auch οὐκ ἄν). Endlich Ρ~399 οὐδ᾽ εἰ μάλα μιν χόλοϲ ἵκοι. Wir haben also 49 Fälle, die unserer obigen Regel genau entsprechen; 3 Fälle, die besonderer Erklärung fähig sind, und nur 1 wirkliche Ausnahme. [Aus den andern Büchern verzeichnet Monro ² 337~f. bloss noch Γ~368 οὐδ᾽ ἔβαλόν \spation{μιν}. Φ~576 εἴ περ γὰρ φθάμενόϲ \spation{μιν} ἢ οὐτάϲῃ, wo er μιν streichen will. Κ~344 ἀλλ᾽ ἐῶμέν \spation{μιν} πρῶτα παρεξελθεῖν πεδίοιο.] Dies alles in Versen, also unter Bedingungen, die es erschweren an der gemeinüblichen Wortstellung festzuhalten. Besonders bemerkenswert ist die bekanntlich auch sonst häufige Phrase τῷ \spation{μιν} ἐειϲάμενοϲ προϲέφη oder προϲεφώνεε für τῷ ἐειϲάμενοϲ προϲέφη μιν, wo der Drang μιν an die zweite Stelle zu setzen deutlich genug wirksam ist. Ähnlich in der häufigen Wendung καί \spation{μιν} φωνήϲαϲ ἔπεα πτερόεντα προϲηύδα, wo μιν zu προϲηύδα gehört und nicht zu φωνήϲαϲ. Ferner beachte man Φ~347 χαίρει δέ μιν ὅϲτιϲ ἐθείρῃ \hypertarget{p336}{\emph{[S. 336]}}\label{p336} “es freut sich, wer es (das Feld) bearbeitet”. Hier ist das zum Nebensatz gehörige Pronomen in den Hauptsatz gezogen, ohne dass man doch von sogen. Prolepse sprechen kann, da das Verb des Hauptsatzes den Dativ verlangen würde. Einzig der Drang nach dem Satzanfang kann die Stellung des μιν erklären.

Für den nachhomerischen Gebrauch von μιν tritt Herodot als Hauptzeuge ein, bei dem mir ausser, auf alle Bücher sich erstreckender, sporadischer Leküre das siebente Buch das nötige Material geliefert hat. Und da kann ich wenigstens sagen, dass die Mehrzahl der Beispiele μιν an zweiter oder so gut wie zweiter Stelle zeigt, darunter so eigentümliche Fälle, wie die folgenden: (ich zitiere hier und später nach Steins Ausgabe mit deutschem Kommentar, deren Zeilenzahlen in der Regel annähernd für alle Ausgaben passen) 1,~204,~7 πολλά τε γάρ \spation{μιν} καὶ μεγάλα τὰ ἐπαείροντα καὶ ἐποτρύνοντα ἦν (μιν gehört zu den Partizipien). 1,~213,~3 ὥϲ \spation{μιν} ὅ τε οἶνοϲ ἀνῆκε καὶ ἔμαθε (μιν gehört blos [sic] zu ἀνῆκε). 2,~90,~7 ἀλλά \spation{μιν} οἱ ἱρέεϲ αὐτοὶ οἱ τοῦ Νείλου — θάπτουϲι. 5,~46,~11 οἱ γάρ \spation{μιν} Σελινούϲιοι ἐπαναϲτάντεϲ ἀπέκτειναν καταφυγόντα ἐπὶ Διὸϲ ἀγοραίου βωμόν. Vgl. Kallinos~1,~20 ὥϲπερ γάρ \spation{μιν} πύργον ἐν ὀφθαλμοῖϲιν ὁρῶϲιν, wobei ich hinzufügen möchte, dass die Elegiker bis auf Theognis und diesen eingerechnet μιν 12 mal an zweiter Stelle, nur einmal (Theognis~195) an dritter Stelle bieten.

Und dass nun dieses Drängen nach dem Satzanfang bei μιν nicht auf irgend welchen etymologischen Verhältnissen beruht, geht aus der ganz gleichartigen Behandlung des enklitischen Dativs οἱ ‘ihm’ hervor, der dem Akkusativ μιν ‘ihn’ in Bedeutung und Akzent ganz nahe steht, aber in der Lautform von ihm gänzlich abweicht. In den Büchern ΝΠΡ der Ilias findet sich jenes οἱ 92 mal. Und zwar 34 mal an zweiter Stelle, 53 mal an dritter oder vierter, aber so, dass es vom ersten Wort des Satzes durch ein Wort oder zwei Wörter getrennt ist, das bezw. die auf die zweite Stelle im Satz noch grössern Anspruch haben, wie δέ, τε, κε. Anders geartet sind nur fünf Stellen. Π~251 νηῶν μέν οἱ und Ρ~273 τῷ καί οἱ, wo μέν bezw. καί eng zum ersten Satzwort gehören; Ρ~153 νῦν δ᾽ οὔ οἱ und Ρ~410 δὴ τότε γ᾽ οὔ οἱ, die dem Gesetz unterliegen, dass bei Nachbarschaft von Negation und Enklitikum die Negation vorangehen muss. Daraus wäre auch Ρ~71 εἰ \hypertarget{p337}{\emph{[S. 337]}}\label{p337} μή οἱ ἀγάϲϲατο Φοῖβοϲ Ἀπόλλων zu erklären, wenn hier nicht die Untrennbarkeit von εἰ und μή schon einen genügenden Erklärungsgrund böte. Man darf also wohl sagen, dass die für μιν erschlossene Stellungsregel durchaus auch für οἱ gilt.

Diese Analogie zwischen μιν und οἱ setzt sich bei Herodot fort. Es findet sich bei ihm οἱ etwa doppelt so oft an zweiter oder so gut wie zweiter, als an anderweitiger Satzstelle. (Bei den ältern Elegikern scheint sich οἱ nur an zweiter Stelle zu finden.)

Besonders beachtenswert ist nun aber, dass diese Stellungsgewohnheit oft bei Homer und fast noch häufiger bei Herodot (vgl. Stein zu 1, 115, 8) dazu geführt hat, dem οἱ eine dem syntaktischen Zusammenhang widersprechende oder in andrer Hinsicht auffällige Stellung anzuweisen.

1)	Entschieden dativisches οἱ steht von dem regierenden Worte weit ab und drängt sich mitten in eine am Satzanfang stehende sonstige Wortgruppe ein. Ρ~232 τὸ δέ \spation{οἱ} κλέοϲ \spation{ἔϲϲεται} ὅϲϲον ἐμοί περ. γ~306 τῷ δέ οἱ ὀγδοάτῳ κακὸν \spation{ἤλυθε} δῖοϲ Ὀρέϲτηϲ. — Herodot~1, 75, 10 Θαλῆϲ \spation{οἱ} ὁ Μιλήϲιοϲ \spation{διεβίβαϲε}. 1, 199, 14 ἤ τίϲ \spation{οἱ} ξείνων ἀργύριον ἐμβαλὼν ἐϲ τὰ γούνατα \spation{μιχθῇ} (τίϲ geht dem οἱ voran, weil es selbst ein Enklitikum ist). 2, 108, 4 τούϲ τέ \spation{οἱ} λίθουϲ (folgen 14 Worte) οὗτοι ἦϲαν οἱ \spation{ἑλκύϲαντεϲ}. 4, 45, 19 ὅϲτιϲ \spation{οἱ} ἦν ὁ \spation{θέμενοϲ} (scil. τοὔνομα). 5, 92, β~8 ἐκ δέ \spation{οἱ} ταύτηϲ τῆϲ γυναίκοϲ οὐδ᾽ ἐξ ἄλληϲ παῖδεϲ \spation{ἐγίνοντο}. 6, 63, 2 ἐν δέ \spation{οἱ} χρόνῳ ἐλάϲϲονι ἡ γυνὴ \spation{τίκτει} τούτον. 7, 5, 14 οὗτοϲ μέν \spation{οἱ} ὁ λόγοϲ \spation{ἦν} \spation{τιμωρόϲ}.

2)	Genetivisches oder halbgenetivisches οἱ ist von seinem nachfolgenden Substantiv durch andre Worte getrennt: Δ~219 τά \spation{οἵ} ποτε \spation{πατρὶ} φίλα φρονέων πόρε Χείρων. Μ~333 ὅϲτιϲ \spation{οἱ} ἀρὴν \spation{ἑτάροιϲιν} ἀμύναι. Ρ~195 ἅ \spation{oἱ} θεοὶ οὐρανίωνεϲ \spation{πατρὶ} φίλῳ ἔπορον. δ~767 θεὰ δέ \spation{οἱ} ἔκλυεν \spation{ἀρῆϲ}. δ~771 ὅ \spation{οἱ} (Herwerden Revue de philologie II 195 ᾧ!) φόνοϲ \spation{υἷι} τέτυκται. Herodot~1, 34, 16 μή τί \spation{οἱ} κρεμάμενον τῷ \spation{παιδὶ} ἐμπέϲῃ.

3)	Genetivisches oder halbgenetivisches οἱ geht seinem Substantiv und dessen Attributen unmittelbar voraus, eine bei einem Enklitikum an und für sich unbegreifliche Stellung: Ι~244 μή \spation{οἱ ἀπειλὰϲ} ἐκτελέϲωϲι θεοί. Ρ~324 ὅϲ \spation{οἱ παρὰ πατρὶ γέροντι} κηρύϲϲων γήραϲκε. — Herodot~3, 14, 14 δεύτερά \spation{οἱ τὸν παῖδα} ἔπεμπε. 3, 15, 12 τήν \spation{οἱ ὁ πατὴρ} εἶχε ἀρχήν. \hypertarget{p338}{\emph{[S. 338]}}\label{p338} 3, 55, 10 καί \spation{οἱ} (καὶ οἷ?) \spation{τῷ πατρὶ} ἔφη Σάμιον τοὔνομα τεθῆναι, ὅτι \spation{οἱ ὁ πατὴρ} Ἀρχίηϲ ἐν Σάμῳ ἀριϲτεύϲαϲ ἐτελεύτηϲε. — Allerdings findet sich diese Wortfolge bei Herodot auch so, dass οἱ dabei nicht an zweiter Stelle steht, z. B. 1, 60, 8 εἰ βούλοιτό \spation{οἱ τὴν θυγατέρα} ἔχειν γυναῖκα. Aber ich glaube, die Sache liegt so: weil das an zweiter Stelle stehende οἱ so oft ein regierendes Substantiv hinter sich hatte, kam es auf, auch mitten im Satz οἱ dem regierenden Substantiv unmittelbar vorausgehen zu lassen.

4)	Genetivisches oder halb genetivisches οἱ steht zwischen dem ersten und zweiten Glied des regierenden Ausdrucks, auch dies eine für ein Enklitikum an sich auffällige Stellung. a) Zwischen Präposition nebst folgender Partikel und Artikel: Herodot~1, 108, 9 ἐκ γάρ \spation{οἱ} τῆϲ ὄψιοϲ οἱ τῶν μάγων ὀνειροπόλοι ἐϲήμαινον. b) Zwischen Artikel nebst folgender Partikel und Substantiv: Β~217 τὼ δέ \spation{οἱ} ὤμω κυρτώ. Ν~616 τὼ δέ \spation{οἱ} ὄϲϲε χαμαὶ πέϲον. Ρ~695 = Ψ~396 τὼ δέ \spation{οἱ} ὄϲϲε δακρυόφιν πλῆϲθεν. Ähnlich Ξ~438, Ο~607, Τ~365 und mehrfach in der Odyssee. Ψ~392 αἱ δέ \spation{οἱ} ἵπποι ἀμφίϲ ὁδοῦ δραμέτην. Ψ~500 αἱ δέ \spation{οἱ} ἵπποι ὑψόϲ᾽ ἀειρέϲθην. — Herodot~1, 1, 19 τὸ δέ \spation{οἱ} οὔνομα εἶναι — Ἰοῦν. 3, 3, 10 τῶν δέ \spation{οἱ} παίδων τὸν πρεϲβύτερον εἰπεῖν. 3, 48, 14 τόν τέ \spation{οἱ} παῖδα ἐκ τῶν ἀπολλυμένων ϲῴζειν. 3, 129, 5 ὁ γάρ \spation{οἱ} ἀϲτράγαλοϲ ἐξεχώρηϲε ἐκ τῶν ἄρθρων. 5, 95, 4 τὰ δέ \spation{οἱ} ὅπλα ἔχουϲι Ἀθηναῖοι. 6, 41, 7 τὴν δέ \spation{οἱ} πέμπτην τῶν νεῶν κατεῖλον διώκοντεϲ οἱ Φοίνικεϲ. — Ebenso die ionischen Dichter: Archilochus~29, 2~Bgk. ἡ δέ \spation{οἱ} κόμη ὤμουϲ κατεϲκίαζε καὶ μετάφρενα. 97, 1 ἡ δέ \spation{οἱ} ϲάθη — ἐπλήμμυρεν. c) Zwischen Artikel und Substantiv: Herodot~1, 82, 41 τῶν \spation{οἱ} ϲυλλοχιτέων διεφθαρμένων. 3, 153, 4 τῶν \spation{οἱ} ϲιτοφόρων ἡμιόνων μία ἔτεκε.

Parallelen hiezu liefern auch die nicht ionischen nachhomerischen Dichter, für die οἱ einen Bestandteil des traditionellen poetischen Sprachguts bildet. Ich bringe, was mir gerade vor die Augen gekommen ist. Zu 1) gehört Pindar Pyth.~2, 42 ἄνευ \spation{οἱ} Χαρίτων \spation{τέκεν} γόνον ὑπερφίαλον. Euphorion Anthol. Palat.~6, 278, 3 (=~Meineke Analecta Alexandrina S.~164) ἀντὶ δέ \spation{οἱ} πλοκαμῖδοϲ ἑκηβόλε καλὸϲ \spation{ἐπείη} ὡχαρνῆθεν ἀεὶ κιϲϲὸϲ ἀεξομένῳ. — Zu 2) Theokrit~2, 138 ἐγὼ δέ \spation{οἱ} ἁ ταχυπειθὴϲ \spation{χειρὸϲ} ἐφαψαμένα (vgl. Meineke zu 7, 88). — Zu 1) oder zu 2) Sophokles Aias 907 ἐν γάρ \spation{οἱ} χθονὶ \spation{πηκτὸν} \hypertarget{p339}{\emph{[S. 339]}}\label{p339} \spation{τόδ}᾽ \spation{ἔγχοϲ} περιπετέϲ κατηγορεῖ. — Zu 3) Europa 41 ἅτε \spation{οἱ} αἵματοϲ ἔϲκεν. — Zu 4) Sophokles Trachin. 650 ἁ δέ \spation{οἱ} φίλα δάμαρ τάλαιναν δυστάλαινα καρδίαν πάγκλαυτοϲ αἰὲν ὤλλυτο.

Die Inschriften der οἱ anwendenden Dialekte sind unergiebig. Für die Doris liefern nur die epidaurischen reichere Ausbeute, und diese gehören bekanntlich in eine verhältnismässig späte Zeit. Ich zähle in No.~3339 und 3340 Collitz vierzehn οἱ an zweiter, acht οἱ an anderweitiger Stelle. Die wenigen nicht-dorischen Beispiele, die ich zur Hand habe, fügen sich sämtlich der Regel. Tegea~1222, 33 Coll. μή \spation{οἱ} ἔϲτω ἴνδικον. Kypros~59, 3 Coll. ἀφ᾽ ὧ ϝοι τὰϲ εὐχωλὰϲ ἐπέτυχε oder ἐπέδυκε (vgl. Meister Griech. Dial.~II~148. Hoffmann~I 67 f.). id.~60, 29 Coll. ἀνοϲίϳα ϝοι γένοιτυ.

Nun könnte es aber jemand trotz alledem bemerkenswert finden, dass Thumb jene eigentümliche, angeblich an die Stellung von \emph{sma} im Veda erinnernde Stellungsgewohnheit bei μιν hat aufdecken können, und könnte geneigt sein, doch noch dahinter irgend etwas von Bedeutung zu vermuten. Um darüber Klarheit zu schaffen, scheint es am richtigsten, die von Thumb für μιν gegebene Statistik am Gebrauch von οἱ in ΝΠΡ zu messen. Thumb~1\textsuperscript{a}: “in 68\% sämtlicher Fälle steht μιν hinter einer Partikel”; οἱ in 66 von 92 Fällen, also in 72\% (33 mal hinter δέ, wie δέ auch vor μιν am häufigsten vorkommt; daneben in absteigender Häufigkeit hinter ἄρα, ῥα, καί, γάρ, οὐδέ, τε, ἔνθα, ἀλλά, ἤ, μέν, πωϲ, τάχα). — Thumb~1\textsuperscript{b}: “in 10\% steht μιν hinter einer subordinierenden Konjunktion”; οἱ viermal (hinter ὅ(τ)τι, ἐπεί, ὄφρα), also nur in 4\%, eine Differenz, die um so weniger ins Gewicht fällt, als Thumb für diese Kategorie eine Abweichung des μιν von \emph{sma} konstatieren muss, da \emph{sma} solche Stellung nicht liebt. — Thumb~2: “μιν niemals unmittelbar hinter Präpositionen (im Gegensatz zu \emph{sma}!)”; οἱ auch niemals. — Thumb~3: “οὔ μιν, μή μιν in 15 von 600 Beispielen”, also in 2½\%”; οὔ οἱ, μή οἱ in 3 von 92 Beispielen, also in 3¼\%. — Thumb~4: “μιν hinter Pronomina sehr häufig”, wie es scheint ca. 100 mal oder 16⅔\%; οἱ auch häufig, nämlich 17 mal, also in 18½\%. — Thumb~5 und 6: “μιν hinter Verbum und nominalen Wörtern in 3\%”; οἱ hinter αἰπύ Ν~317, αἵματι Ρ~51, also in 2\%.

Die Thumbschen Beobachtungen gelten also gerade so gut für οἱ wie für μιν. Οἱ findet sich hinter denselben Wör-\hypertarget{p340}{\emph{[S. 340]}}\label{p340}tern wie μιν und hinter diesen fast genau mit derselben Häufigkeit wie μιν. Wir haben es also bei dem, was Thumb für μιν nachweist, nicht mit irgend etwas für μιν Partikulärem zu thun, sondern mit einer, μιν und οἱ gemeinsamen Konsequenz des Stellungsgesetzes, das ihnen beiden die zweite Stellung im Satz anweist.

Wenn so der Herleitung des μιν aus \emph{sm(a)-im} der Hauptstützpunkt entzogen ist, so wird dieselbe geradezu widerlegt durch das Fehlen jeder Wirkung des angeblich ehemals vorhandenen Anlautes \emph{sm-}; man müsste doch bei Homer gelegentlich δέ μιν als Trochäus (oder Spondeus), ἀλλά μιν als Antibacchius (oder Molossus) erwarten; Thumb schweigt sich über diesen Punkt aus. Dazu kommt eine weitere Erwägung. Entweder ist die Zusammenrückung von \emph{sma} und \emph{im}, welche μιν ergeben haben soll, uralt. Dann ist das Vergessen der ursprünglichen Funktion von \emph{sma} in der Anwendung von μιν begreiflich, aber man müsste entsprechend altindischem \emph{*smēm} griechisch *(ϲ)μαιν erwarten. Oder die Zusammenrückung hat nicht lange vor Homer stattgefunden, in welchem Fall die Anwendung des spezifisch griechischen Elisionsgesetzes, also die Reihe μα ἰν — μ᾽ ἰν — μιν, begreiflich wird: dann versteht man nicht den völligen Untergang der Funktion von (ϲ)μα, die Behandlung von μιν ganz in Weise einer gewöhnlichen Pronominalform, zumal ja im Thessalischen in der Bedeutung ‘aber’ eine Partikel μα vorkommt, deren Gleichsetzung mit altind. \emph{sma} allerdings bestreitbar ist.

Noch weniger glücklich scheint mir Thumbs Erklärung des dorischen νιν aus \emph{nu-im}, da mir hier unüberwindliche lautliche Schwierigkeiten entgegenzustehen scheinen. Denn wenn er bemerkt: “dass auslautendes \emph{u}, wie im Altindischen (z.~B. \emph{kṓ nv átra}) vor Vokal unter gewissen Bedingungen ehemals als Konsonant \emph{(u̯)} gesprochen wurde, darf unbedenklich angenommen werden”: und sich hierfür auf Fälle wie πρόϲ aus \emph{proti̯}, εἰν aus \emph{eni̯}, ὑπείρ aus \emph{hyperi̯} (~=~altind. \emph{upary} neben \emph{upari}), lesb. πέρρ- aus \emph{peri̯}- beruft, in denen \emph{i̯} für \emph{i} in die Zeit der indogermanischen Urgemeinschaft hinaufreiche, so ist dabei übersehen, dass nicht alle auslautenden \emph{-i, -u} auf gleiche Linie gestellt werden dürfen. Im Rigveda findet sich Übergang von \emph{-i, -u} zu \emph{-y, -v} in etwelcher Häufigkeit gerade nur bei der Wortklasse, bei der das Griechische \hypertarget{p341}{\emph{[S. 341]}}\label{p341} Reflexe solches Übergangs zeigt, nämlich bei den zweisilbigen Präpositionen, wie \emph{abhi, prati, anu, pari, adhi}; sonst ausser dem jüngern X. Buch und den Vālakhilyas nur ganz sporadisch, bei Einsilblern nur in der Zusammensetzung \emph{avyuṣ̌ṭāḥ}~2, 28, 9, und dann in \emph{ny alipsata}~1, 191, 3, also in einem anerkannt späten Liede (Oldenberg Rigveda S.~I 438 Anm.). Und speziell \emph{nu} (ähnlich wie \emph{u}) entzieht sich solchem Sandhi durchaus, wird umgekehrt öfters lang und sogar mit Zerdehnung zweisilbig gemessen. Und selbst wenn wir auch trotz alle dem urgriechisches νϝιν, woraus dorisch νιν, hinter vokalischem Auslaut konstruieren könnten, so bliebe ein postkonsonantisches νιν doch unverständlich; eine Entwicklungsreihe ὅϲ νυ ἰν, ὅϲ νϝ ἰν, ὅϲ νιν lässt sich gar nicht denken.

Wenn übrigens Thumb S. 646 andeutet, dass die Stellung von νιν im Satz keine speziellen Analogieen mit derjenigen von altind. \emph{nu}, griech. νυ aufweise, und dies mit dem geringern Alter der νιν bietenden Sprachquellen (Pindars und der Tragiker) entschuldigt, so ist allerdings wahr, dass diese Autoren nicht bloss aus chronologischen Gründen, sondern auch wegen der grössern Künstlichkeit ihrer Wortstellung kein so reinliches Resultat für νιν liefern können, wie Homer und Herodot für μιν. Aber man wird doch fragen dürfen, ob nicht gewisse Tendenzen zu erkennen sind. Und da ist zu konstatieren, dass an 30 unter 47 äschyleischen Belegstellen νιν dem für μιν und οἱ eruierten Stellungsgesetz folgt, und zwar, was vielleicht beachtenswert ist, an 5 unter 7 in den Persern und den Septem, an 21 unter 32 in der Orestie, in 2 unter 5 im Prometheus. Etwas ungünstiger ist das Verhältnis bei Sophokles, wo von 81 Belegstellen 47 νιν an gesetzmässiger, 34 an ungesetzmässiger Stelle haben. Zu ersterer Klasse gehören die Fälle von Tmesis: Sophokles Antig.~432 ϲὺν δέ \spation{νιν} θηρώμεθα. 601 κατ᾽ αὖ \spation{νιν} φοινία θεῶν τῶν νερτέρων ἀμᾷ κοπίϲ. Übrigens ist eine Empfindung dafür, welches die eigentliche Stellung von νιν sei, auch sonst lebendig. Vgl. Aristoph. Acharn.~775, besonders aber Eurip. Medea~1258 ἀλλά \spation{νιν}, ὦ φάοϲ διογενέϲ, κατεῖργε. Helena~1519 τίϲ δέ \spation{νιν} ναυκληρία ἐκ τῆϲδ᾽ ἀπῆρε χθονόϲ. Iphig. Aul.~615 ὑμεῖϲ δὲ, νεάνιδέϲ, \spation{νιν} ἀγκάλαιϲ ἔπι δέξαϲθε. Bacch.~30 ὧν \spation{νιν} οὕνεκα κτανεῖν Ζῆν᾽ ἐξεκαυχῶντ(ο). — Dazu Theokrit.~2,~103 ἐγὼ δέ \spation{νιν} ὡϲ ἐνόηϲα. 6,~11~τὰ δέ \spation{νιν} καλὰ κύματα φαίνει. Höchst bemer-\hypertarget{p342}{\emph{[S. 342]}}\label{p342}kenswert ist endlich die kürzlich von Selivanov in den athen. Mitteil.~XVI~112~ff. herausgegebene alte rhodische Inschrift ϲᾶμα τόζ᾽ Ἰδαμενεὺϲ ποίηϲα ἵνα κλέοϲ εἴη· Ζεὺϲ δέ \spation{νιν} ὅϲτιϲ πημαίνοι, λειώλη θείη, wo das νιν syntaktisch zu πημαίνοι gehört, also mit dem oben S.~332~f. erwähnten μιν in Φ~347 χαίρει δέ μιν ὅϲτιϲ ἐθείρῃ aufs genaueste zusammenstimmt.

Diese wesentliche Übereinstimmung von νιν und μιν in der Stellung wirft Thumbs ganze Beweisführung nochmals um. Eines gebe ich ihm allerdings zu, dass μ-ιν, ν-ιν zu teilen und *ἰν der Akk. zu lat. \emph{is}, und das sowohl die Annahme zugrunde liegender Reduplikativbildungen *ἰμιμ, *ινιν, als die Annahme in μιν, νιν enthaltener Stämme \emph{mi-, ni-} verkehrt ist. Mir scheint es, bessere Belehrung vorbehalten, am einfachsten μ-, ν- aus dem Sandhi herzuleiten. Wenn es nebeneinander hiess αὐτίκα-μ-ιν (aus \emph{-km̥m im}) und αὐτίκα μάν, ἄρα-μ-ιν und ἄρα μάν, ῥα-μ-ιν und ῥα μάν (falls man für den Auslaut von ἄρα, ῥα labiale Nasalis sonans annehmen darf), so konnte wohl auch ἀλλά μιν neben ἀλλὰ μάν sich einstellen und μιν allmählich weiterwuchern; ἀλλά μιν~:~αὐτίκα μιν = μηκέτι~:~οὐκέτι. In ähnlicher Weise kann das ν- von νιν auf auslautender dentaler Nasalis sonans beruhen. Vgl. Kuhns Zeitschr.~XXVIII~119. 121. 125 über ἄττα aus ττα, οὕνεκα aus ἕνεκα und Verwandtes, sowie auch das prakritische Enklitikum \emph{m-iva, mmiva} für sanskr. \emph{iva}, dessen \emph{m} natürlich aus dem Auslaut der Akkusative und der Neutra stammt (Lassen Institut. S.~370). Weiteres Tobler Kuhns Zeitschr.~XXIII 423, G. Meyer Berliner philolog. Wochenschrift~1885 S.~943~f., Ziemer ibid. S.~1371, Schuchardt Litt. Blatt für rom. Philologie~1887 Sp.~181, Thielmann Archiv für lat. Lexikogr.~VI 167 Anm.

\section*{II.}
\addcontentsline{toc}{section}{II.}

Die Vorliebe von μιν, νιν, οἱ für die zweite Stelle im Satz gehört nun aber in einen grösseren Zusammenhang hinein. Bereits 1877 hat Bergaigne Mémoires de la Société de Linguistique~III~177.~178 darauf hingewiesen, dass die enklitischen Pronominalformen überhaupt “se placent de préférence après le premier mot de la proposition.” Er führt als Belege an Α~73 ὅ \spation{ϲφιν} εὔ φρονέων ἀγορήϲατο καὶ μετέειπειν. Α 120 ὅ \spation{μοι} γέραϲ ἔρχεται ἄλλῃ.

Diese Beobachtung bestätigt sich, sobald man anfängt \hypertarget{p343}{\emph{[S. 343]}}\label{p343} Beispiele zu sammeln. In den von mir zugrunde gelegten Büchern ΝΠΡ findet sich, um im Anschluss an μιν, νιν, οἱ mit dem Pronomen der dritten Person zu beginnen, ἑ viermal, allemal an zweiter oder möglichst nahe bei der zweiten Stelle (ich werde im folgenden diesen Unterschied nicht mehr berücksichtigen). ϲφι(ν) zwölfmal, und zwar elfmal regelmässig, regelwidrig nur Ρ~736 ἐπὶ δὲ πτόλεμοϲ τέτατό \spation{ϲφίν} [sic] (beachte auch Κ~559 τὸν δέ \spation{σφιν} ἄνακτ᾽ ἀγαθὸϲ Διομήδηϲ ἔκτανε, wo ϲφιν sich in die Gruppe τὸν δὲ ἄνακτα eingedrängt hat). ϲφιϲι(ν) sechsmal, immer regelmässig. ϲφεαϲ in Ρ~278 μάλα γάρ \spation{ϲφεαϲ} ὦκ᾽ ἐλέλιξεν. ϲφωε Ρ~531 εἰ μή \spation{ϲφω᾽} Αἴαντε διέκριναν μεμαῶτε. Aus dem sonstigen homerischen Gebrauch sei das hyperthetische καί \spation{ϲφεαϲ} φωνήϲαϲ ἔπεα πτερόεντα προϲηύδα angeführt.

Ebenso in der zweiten Person: \spation{ϲεο}, \spation{ϲευ} findet sich fünfmal, allemal an zweiter Stelle (weitere Beispiele s. unten); \spation{τοι} (bei dem ich aus naheliegenden Gründen die Fälle, wo es als Partikel gilt, mit einrechne, jedoch ohne ἤτοι, ἦτοι) findet sich 47 mal, und zwar 45 mal der Regel gemäss, nur zweimal anders: Ν~382 ἐπεὶ οὔ \spation{τοι} ἐεδνωταὶ κακοί εἰμεν, und Π~443 ἀτὰρ οὔ \spation{τοι} πάντεϲ ἐπαινέομεν θεοὶ ἄλλοι. An beiden Stellen hat die schon früher besprochene Tendenz der Negationen die Enklitika an sich anzulehnen die Hauptregel durchkreuzt. — \spation{ϲε} findet sich 21 mal, davon 19 mal nach der Regel, zweimal anders: Π~623 εἰ καὶ ἐγώ ϲε βάλοιμι, und Ρ~171 ἦ τ᾽ ἐφάμην ϲε.

Ebenso in der ersten Person: μευ findet sich Ν~626. Ρ~29, an beiden Stellen zunächst dem Satzanfang; \spation{μοι} findet sich mit Einrechnung von ὤμοι 32 mal, davon 27 mal der Regel gemäss, wozu als 28. Beleg wohl Ρ~97 ἀλλὰ τί ἦ \spation{μοι} ταῦτα φίλοϲ διελέξατο θυμόϲ gefügt werden darf. Abweichend sind Π~112 ἕϲπετε νῦν \spation{μοι} (ἕϲπετέ νύν μοι? bei welcher Schreibung diese Stelle zu den regelmässigen Beispielen gehören würde). Π~238 ἠδ᾽ ἔτι καὶ νῦν \spation{μοι} τόδ᾽ ἐπικρήηνον ἐέλδωρ. Π~523 ἀλλὰ ϲύ πέρ \spation{μοι} ἄναξ τόδε καρτερὸν ἕλκοϲ ἄκεϲϲαι. Π~55 αἰνὸν ἄχοϲ τό \spation{μοί} ἐϲτιν, Ausnahmen, die weder durch ihre Zahl noch durch ihre Beschaffenheit die Regel erschüttern können, während umgekehrt eine Stelle wie Τ~287 Πάτροκλέ \spation{μοι} δειλῇ πλεῖϲτον κεχαριϲμένε θυμῷ, wo der Anschluss von μοι an einen Vokativ schon den Alten auffiel, einen Beleg für die durchgreifende Gültigkeit der Regel liefert. Ähn-\hypertarget{p344}{\emph{[S. 344]}}\label{p344}lich auffällig ist μοι nach ἄλλ᾽ [sic] ἄγε: α~169. ἀλλ᾽ ἄγε \spation{μοι} τόδε εἰπέ — Endlich με findet sich 15 mal, immer nach der Regel. [Ausnahmen aus den andern Büchern bespricht Monro\textsuperscript{2} 336~ff., z.~T. mit Änderungsvorschlägen.]

Auch ausserhalb Homers lassen sich Spuren der alten Regel nachweisen. So bei den \spation{Elegikern} bis Theognis (mit Einschluss desselben), die με 42 mal an zweiter, 4 mal an späterer; μοι 36 mal an zweiter, 5 mal an späterer; ϲε 27 mal an zweiter, 6 mal an späterer Stelle zeigen. So ferner auch in den von Homer weniger als die Elegiker abhängigen dialektischen Denkmälern. Denn wenn die \spation{Arkader} ihr ϲφεῖϲ ziemlich frei gestellt zu haben scheinen, so stimmt um so besser der \spation{dorische} Akkusativ τυ: Fragm. lyr. adesp. 43~Α (poeta lyr. gr. ed. Bergk~3\textsuperscript{4}, S.~701) καί \spation{τυ} φίλιππον ἔθηκεν. Epicharm bei Athen.~4, 139~Β ἐκάλεϲε γάρ \spation{τύ} τιϲ; Sophron bei Apollonius de pron. 68~Β τί \spation{τυ} ἐγὼν ποιέω; Aristoph. Acharn.~730 ἐπόθουν \spation{τυ} ναί τὸν φίλιον ἇπερ ματέρα. Dazu der (von Ahrens II 255 nicht erwähnte) dorische Orakelspruch bei Stephanus Byz.~73, 14~M. (aus Ephorus) ποῖ \spation{τυ} λαβὼν <ἄξω> καὶ ποῖ \spation{τυ} καθίζω und die Mehrzahl der ungefähr dreissig theokriteischen Beispiele, darunter bemerkenswert 5, 74 μή \spation{τύ} τιϲ ἠρώτη (= att. μήτιϲ ϲε εἰρώτα), wo μήτιϲ durch τυ entzwei gesprengt ist, und 1, 82 ἁ δέ \spation{τυ} κώρα πάϲαϲ ἀνὰ κράναϲ, πάντ᾽ ἄλϲεα ποϲϲὶ φορεῖται ζατεῦϲ(α), wo das von Brunck aus dem best überlieferten aber unmetrischen τοι sicher hergestellte τυ als Akkusativ zu ζατεῦϲα gehört, aber weit davon abstehend ἁ und κώρα von einander trennt. (Die einzige Stelle des Kallimachus epigr. 47 (46), 9 οὐδ᾽ ὅϲον ἀττάραγόν \spation{τυ} δεδοίκαμεϲ, widerspricht der Regel.) Höchst beachtenswert ist endlich das einzige inschriftliche Beispiel, das ich zur Hand habe: Collitz 3339, 70 (Epidauros) αἴ \spation{τύ} κα ὑγιῆ ποιήϲω (= att. ἐάν ϲε ὑγιᾶ π.), wo τυ zwischen die sonst eng verbundenen Partikeln αἰ und κα getreten ist. Das einzige abweichende Beispiel der vor-alexandrinischen Zeit, Sophron bei Apollon. de pron. 75 Α οὐχ ὁδεῖν \spation{τυ} ἐπίκαζε, kann, solange die Lesung nicht sicher gestellt ist, nicht ins Gewicht fallen.

Ganz nahe zu Homer stellen sich ferner die \spation{äolischen Dichter}. Ich zähle in deren Fragmenten, die ich nach Bergks Poetae lyrici, 4. Aufl., zitiere, 38 (oder je nach der Schreibung von Sappho fragm. 2,~7 und fragm. 100 — siehe gleich \hypertarget{p345}{\emph{[S. 345]}}\label{p345} nachher — 36) Belege der enklitischen Formen des Personalpronomens. 30 folgen der homerischen Regel, darunter sämtliche sicheren (12) Beispiele von με und sämtliche 10 Beispiele von μοι. Abweichend ist τοι dreimal (Sappho~2, 2.~8.~70,~1) und ϲε einmal (Sappho 104,~2). Bleiben drei Stellen mit bestrittner Lesung, deren handschriftliche Überlieferung ich zunächst hersetze: Sappho 2,~7 ὡϲ γάρ ϲ᾽ ἴδω βροχεώϲ \spation{με} φωνὰϲ οὐδὲν ἔτ᾽ εἴκει, Sappho~43 ὄτα πάννυχοϲ \spation{ἄσφι} κατάργει, endlich Sappho 100 nach dem volleren Wortlaut bei Choirikios (Oeuvres de Charles Graux II 97) … ϲὲ τετίμηκεν ἐξόχωϲ ἡ Ἀφροδίτη. An der ersten wird nun die von Ahrens vorgeschlagene, von Vahlen in seiner Ausgabe der Schrift περὶ ὕψουϲ (Kap. 10, 2) gebilligte Lesung ὥϲ \spation{ϲε} γὰρ ϝίδω, βροχέωϲ \spation{με} φώναϲ κτἑ. nur um so wahrscheinlicher und Seidlers von Bergk und Hiller gebilligte Versetzung des cε hinter βροχέωϲ und Streichung des με nur um so unwahrscheinlicher. Für die zweite Stelle kann ich nun noch bestimmter die KZ. XXVIII 141 geforderte Lesung ὄτά \spation{ϲφι} πάννυχοϲ κατάγρειϲ [sic] als notwendig bezeichnen. Und an der dritten Stelle ergiebt sich nun Weils von Hiller (Antholog. lyr. fragm. 97) rezipierte Schreibung τετίμακ᾽ ἔξοχά ϲ᾽ Ἀφροδίτα als entschieden unwahrscheinlich.

So kommen wir durch Addition der 30 obigen Fälle, des ϲε und με bei Sappho~47 und des ϲφι für ἄϲφι bei Sappho~43 auf 33 regelrechte Beispiele gegenüber 4 regelwidrigen und einem (Sappho~100), wo die Überlieferung uns im Stich lässt und wir nicht einmal wissen, ob wir es mit einem Enklitikum zu thun haben. Ganz ausser Rechnung fällt Alc.~68, wo manche nach Bekker πάμπαν δὲ τυφὼϲ ἔκ ϲ᾽ ἕλετο φρέναϲ schreiben, aber hinter ἐκ vielmehr δ᾽ überliefert ist; vgl. was Bergk gegen Bekkers Schreibung bemerkt.

An mancher jener 33 Stellen werden obendrein durch das enklitische Pronomen Wortgruppen durchschnitten: Artikel und Substantiv Sappho~2, 13 ἀ δέ μ᾽ ἰδρὼϲ κακχέεται. 118, 3 Αἰθοπίᾳ \spation{με} κόρᾳ Λατοῦϲ ἀνέθηκεν Ἀρίϲτα. Attribut und Substantiv Sappho~34, 1 ϲμίκρα \spation{μοι} πάϊϲ ἔμμεν ἐφαίνεο κἄχαριϲ. Präposition und Verba Alcaeus~95 ἔκ μ᾽ ἔλαϲαϲ ἀλγέων. Vgl. auch Sappho~2, 5 τό \spation{μοι} μάν und 2, 7 ὥϲ \spation{ϲε} γάρ, wo μάν und γάρ auf die Stelle hinter τό, bezw. ὡϲ Anspruch gehabt hätten. Ebenfalls beachtenswert sind die Fälle, wo das Pronomen in sonst auffälliger Weise von don Wörtern abgetrennt \hypertarget{p346}{\emph{[S. 346]}}\label{p346} ist, zu denen es syntaktisch gehört: Sappho~1, 19 τίϲ ϲ᾽, ὦ Ψάπφ᾽ ἀδικήει. 104, 1, τίῳ ϲ᾽, ὦ φίλε γάμβρε, κάλωϲ ἐϊκάϲδω. 88 τί \spation{με} Πανδίονιϲ ὤραννα χελίδων. An einen satzeinleitendcn Vokativ ist μοι angelehnt Sappho~45 ἄγε δὴ, χέλυ δῖά, \spation{μοι} φωνάεϲϲα γένοιο. Endlich verweise ich auf Sappho~6 ἤ \spation{ϲε} Κύπροϲ ἢ Πάφοϲ ἢ Πάνορμοϲ.

Allgemein üblich ohne Unterschied der Dialekte ist es, das archaische (Klein Die griechischen Vasen mit Meistersignaturen\textsuperscript{2} S. 13) με in \spation{Weih-} und \spation{Künstlerinschriften} gleich hinter das erste Wort zu setzen. Es wird dienlich sein, die Beispiele vollständig zusammen zu stellen.

Ich beginne mit \spation{μ᾽ ἀνέθηκε}: \spation{Attika} Corpus inscript. att. 4\textsuperscript{2}, 373, 87 -ιτόϲ μ᾽ ἀνέθηκεν. 373, 90 Ὀνήϲιμός μ᾽ ἀνέθηκεν ἀπαρχὴν τἀθηναίᾳ ὁ Σμικύθου υἱόϲ. 373, 120 [ὁ δεῖνα] μ᾽ ἀνέθηκεν δεκάθην (sic!) Ἀθηναίᾳ. Inscript. graecae antiq. 1 (attisch oder euböisch) Σημωνίδηϲ μ᾽ ἀνέθηκεν. Vgl. 373, 100 [Στρόγ]γυλόϲ μ᾽ ἀνέθηκε, wo jedoch ein Dativ vorausgeht. Vielfach auch in Versen (obwohl hier natürlich Gegenbeispiele nicht fehlen: CIA. 1,~343.~374.~4\textsuperscript{2}, 373,~81 u. s. w.): CIA. 1, 349 -θάνηϲ μ᾽ ἀνέθηκεν Ἀθηναία[ι πολιούχψ]. 352 Ἰφιδίκη μ᾽ ἀνέθηκεν, 4\textsuperscript{2} 373, 85 Ἀλκίμαχόϲ μ᾽ ἀ[νέθηκε]. 373, 99 Τίμαρχόϲ μ᾽ ἀνέθηκε Διὸϲ κρατερόφρονι κούρῃ. 373, 215 (Vgl. Studnitzka Jahrbuch des archäol. Instituts~II (1887) 145) Νηϲιάδηϲ κεραμεύϲ με καὶ Ἀνδοκίδηϲ ἀνέθηκεν. 373, 216 Παλλάδι μ᾽ ἐγρεμάχᾳ Διονύϲιο[ϲ τό]δ᾽ ἄγαλμα ϲτῆϲε Κολοίου παῖϲ [εὐξά]μενοϲ δεκάτην. 373, 218 ἀνέθηκε δέ μ᾽ Εὐδίκου υἱόϲ. Inschrift von der Akropolis ed. Foucart Bull. de Corresp. hellén. 13, 160 [Ἑρμό?]δωρόϲ μ᾽ ἀνέθηκεν Ἀφροδίτῃ δῶρον ἀπαρξήν. — \spation{Böotien}: Inschrift nach Reinach behandelt von Kretschmer Hermes XXVI 123~ff. Τιμαϲίφιλόϲ μ᾽ ἀνέθεικε τὠπόλλωνι τοῖ Πτῳεῖι ὁ Πραόλλειοϲ. — \spation{Korinth} (von hier an scheide ich die poetischen und die prosaischen Inschriften nicht mehr): IGA. 20, 7 Σιμίων μ᾽ ἀνέθηκε Ποτειδάϝων[ι ϝάνακτι]. 20, 8 -ων μ’ ἀνέθηκε Ποτειδᾶνι ϝάν[ακτι]. 20, 9 (= 10 = 11) Φλέβων μ’ ἀνέθηκε Ποτειδᾶ[νι]. 20, 42 Δόρκων μ’ ἀνέθηκ[ε]. 20, 43 Ἴγρων μ᾽ ἀν[έθηκε]. 20, 47 Κυλοίδαϲ μ᾽ ἀνέθηκε. 20, 48 Εὐρυμήδηϲ μ᾽ ἀνέθηκε. 20, 49 Λυϲιάδαϲ μ᾽ [ἀνέθηκε]. 20, 83 — μ᾽ ἀνέθ[ηκε]. 20, 87 und 89 -ϲ μ᾽ ἀνέθηκε. 20, 87\textsuperscript{a} — με ἀνέθ(η)κε τῷ. 20, 94 — μ᾽ ἀνέθηκε. 20, 102 [Π]έριλόϲ μ᾽ —. — \spation{Korkyra}: IGA. 341 (= 3187 Collitz) Λόφιόϲ μ᾽ ἀνέθηκε. \hypertarget{p347}{\emph{[S. 347]}}\label{p347} — \spation{Hermione}: Kaibel 926 [Παν]τακλῆϲ μ᾽ ἀνέθηκεν. — \spation{Kyra} bei Aegina: Inschrift ed. Jamot Bull. Corr. \mbox{hellén}. 13, 186 οἱ φρουροί μ᾽ ἀ[νέθεϲαν?] — \spation{Lakonien}: IGA. 62\textsuperscript{a} (S. 174) Πλειϲτιάδαϲ μ᾽ ἀ[νέθηκε] Διοϲκώροιϲιν ἄ[γαλμα]. — \spation{Naxos}: IGA 407 Νικάνδρη μ᾽ ἀνέθηκεν ἑκηβόλῳ ἰοχεαίρῃ. 408 Δειναγόρηϲ μ᾽ ἀνέθηκεν ἑκηβόλῳ Ἀπόλλωνι. In Delos gefundene Inschrift ed. Homolle Bull. Corresp. hellén. 12, 464~f. Εὶ(θ)υκαρτίδηϲ [sic] μ᾽ ἀνέθηκε ὁ Νάξιοϲ ποιήϲαϲ. — \spation{Samos}: IGA. 384 Χηραμύηϲ μ᾽ ἀνέθ(η)κεν τἤρῃ ἄγαλμα. Röhl ergänzt am Anfang [Ἐνθάδε] und bemerkt: “Primam vocem versus hexametri utrum is qui inscripsit an is qui descripsit titulum omiserit, nunc in medio relinquo”. Sicher weder der eine noch der andere. Nicht der Urheber der Abschrift: Dümmler bemerkt mir, dass der von ihm gesehene Abklatsch keine Spur einer [sic] vor Χηραμύηϲ einst vorhandenen Wortes aufweise. Aber auch nicht der Steinmetz: weder der Sinn noch, wie man nun besser als vor zehn Jahren weiss, das Metrum verlangen eine Ergänzung; und die Stellung des με schliesst ein [sic] solche aus. —  \spation{Kalymna}: Kaibel 778 Νικίαϲ \spation{με} ἀνέθηκεν Ἀπόλλωνι υἱὸϲ Θραϲυμήδεοϲ. — \spation{Kypros}: Inschrift bei Hoffmann Die griech. Dialekte 1, 85 No. 163 [—] μ᾽ ἀ(νέ)θηκαν τῷ Ἀπόλ(λ)ωνι. Kaibel 794 (1.~Jahrhundert n.~Ch.) [Κεκρο]πíδηϲ μ᾽ ἀνεθηκε. — \spation{Achäisch} (Grossgriechenland): IGA. 543 Κυνίϲκοϲ \spation{με} ἀνέθηκεν ὥρταμοϲ ϝέργων δεκάταν. — \spation{Syrakus}: Inscriptiones Graecae Siciliae ed. Kaibel 5 Ἀλκιάδηϲ μ᾽ [ἀνέθηκεν]. — \spation{Naukratis}: Naukratis~I by Flinders Petrie (die Inschriften von Gardner S. 60—63) No. 5 Παρμένωνμ (sic!) \spation{με} ἀνέθηκε τὠππόλλωνι (sic!). 24 -ϲ \spation{με} ἀ[νέθηκε]. 80 -ϲ με ἀνέθηκεν τὠπολλων[ι]. 114 -ων μ[ε ἀνέθηκε]. 137 -ϲ μ᾽ ἀν[έθηκε]. 177 Πρώταρχόϲ \spation{με} [ἀνέθηκε τ]ὠπόλλωνι. 186 [Π]ρώταρχόϲ \spation{με} ἀνέθηκ[ε]. 202 [ὁ δεῖνα] \spation{με} ἀνέθηκε. 218 Φάνηϲ \spation{με} ἀνέθηκε τὠπόλλων[ι τῷ Μι]ληϲίῳ ὁ Γλαύκου. 220 Χαριδíων \spation{με} ἀνέθη[κε]. 223 [Πολύ]κεϲτόϲ μ᾽ ἀνέθηκε τ[ὠπόλλωνι]. 235 Σληύηϲ μ᾽ ἀνέθηκε τὠπόλλωνι. 237 [Χ]αρ(ό)φηϲ \spation{με} ἀνέθηκε τἀπό[λλωνι τῷ Μ]ιλαϲίῳ. 255 -ηϲ μ᾽ ἀνέθηκε. 259 -ϲ μ᾽ ἀ[νέθηκε]. 326 Να[ύπλι]όϲ \spation{με} [ἀνέθηκε [sic]. 327 -δηϲ μ᾽ ἀνέθηκε τὠπόλλωνι. 446 -ϲ \spation{με} ἀνέ[θηκεν]. id. vol.~II (by Gardner) S.~62—69: No.~701 Σώϲτρατόϲ μ᾽ ἀνέθηκεν τἠφροδίτῃ. 709 -οϲ μ᾽ ἀνέθηκε τῆ[ι Ἀφροδίτῃ] ἐπὶ τῆ —. 717 Καῖκόϲ μ᾽ [ἀνέ]θηκεν. 720 -οροϲ μ᾽ ἀν[έθηκεν]. 722 Μυϲόϲ μ᾽ ἀνέθηκεν Ὀνομακρίτου. 723 Ἄϲοϲ \hypertarget{p348}{\emph{[S. 348]}}\label{p348} μ᾽ ἀνέθηκεν. 734 -ναξ μ᾽ [ἀνέθηκεν]. 736 -ων με ἀν[έθηκεν]. 738 [ὁ δεῖνα] μ᾽ ἀνέθηκεν Ἀφροδίτῃ (?). 742 -ηιλόϲ μ᾽ ἀνέθηκεν. 748 Ἑρμηϲιφάνηϲ μ᾽ ἀνέθηκεν τἠφροδίτῃ. 770 -μηϲ με ἀν[έθηκε τ]ἠφροδίτη[ι]. 771 Χάρμ[η]ϲ \spation{με} [ἀνέθηκεν]. 775 [Κ]λεόδημοϲ \spation{με} ἀ[νέ]θηκε τῇ Ἀ[φροδίτῃ]. 776—777 Χάρμηϲ \spation{με} ἀνέθηκε τἠφροδίτῃ (bezw. τῇ Ἀ.) εὐχωλήν. 778 Ροῖκόϲ μ᾽ ἀνέθηκε τ[ῇ Ἀφρ]οδίτῃ. 780 Φιλίϲ μ᾽ ἀνέθηκε τ[ῇ Ἀφρ]οδί[τῃ]. 781 Θούτιμόϲ \spation{με} ἀνέθηκ[εν]. 785 [ὁ δεῖνα] μ᾽ ἀν[έθηκε τῇ Ἀφρ]οδίτῃ. 794 Πολύερμόϲ μ᾽ ἀν[έθηκε] τῇ Ἀφροδίτῃ. 799 Ὠχίλοϲ μ᾽ ἀνέθηκε. 817 [ὁ δεῖνα] καὶ Χ[ρυϲ]όδωρόϲ \spation{με} ἀνέθ[ηκαν]. 819 [Λ]άκρι[τό]ϲ μ᾽ ἀνέ[θη]κε οὑρμο[θ]έμ[ιοϲ] τἠφροδί[τῃ]. 876 Ἑρμαγόρηϲ μ᾽ ἀνέ\-θηκε ὁ Τ[ήιοϲ] τὠπόλλωνι (Vers!). 877 Πύρ(ρ)οϲ με ἀνέθηκεν. [\spation{Metapont}: 1643 Coll. ὄ [sic] τοι κεραμεύϲ μ᾽ ἀνέθηκε.]

Von der Norm weichen ab (ausser einigen poetischen Inschriften, siehe oben S.~343) bloss Naukratis 1, 303 [ὁ δεῖνα ἀνέθηκέ] με und 307 [ὁ δεῖνα ἀνέθηκ]έ με, beide Inschriften, wie sich nun ergiebt, falsch ergänzt, und die zweizeilige Inschrift Naukratis~2, 750, wo die obere Linie [τῇ Ἀφροδί]τῃ, die untere Ἑρμαγα\-θῖνόϲ μ᾽ ἀνέθ[ηκεν] bietet. Gardner liest danach τῇ Ἀ. Ἑ. μ᾽ ἀνέθηκεν. Aber Dümmler bemerkt mir, dass die obere Zeile, weil kürzer und den Raum nicht ausfüllend, nicht die erste Zeile sein könne, sondern offenbar den Schluss der untern längern Zeile bilde. Folglich muss, schon ganz abgesehen von unserer Stellungsregel, Ἑρμαγαθῖνόϲ μ᾽ ἀνέθ[ηκε] [τῇ Ἀφροδί]τῃ gelesen werden.

Ganz Analoges gilt für die mit Synonymis von ἀνέθηκε gebildeten Aufschriften: \spation{με κατέθηκε} \spation{Kypros}: Deecke 1 Κάϲ \spation{μι} κατέθηκε τᾷ Παφίᾳ Ἀφροδίτᾳ. 2 αὐτάρ \spation{μι} κατέ[θηκε] Ὀναϲίθεμιϲ. 3 αὐτάρ \spation{με} [κατέθηκε Ὀναϲί]θεμι[ϲ]. 15 αὐτάρ \spation{με} κατέθηκε [Ἀ]κεϲτόθεμιϲ. — \spation{Naukratis}~II No.~790 [ὁ δεῖνα μ]ε κάθθη[κε] ὀ [sic] Μυτιλήναιοϲ. 840 Νέαρχόϲ \spation{με} κά[θθηκε το]ῖϲ Δ[ιοϲκόροιϲι]. — \spation{μ᾽ ἐπέθηκε} \spation{Aegina}: IGA. 362 Διότιμόϲ μ᾽ ἐπέθηκε. — \spation{με (κατ)έϲταϲε} \spation{Kypros}: \mbox{Deecke}~71 κά \spation{μεν} ἔϲταϲαν [κα]ϲίγνη\-τοι (Vers!). Hoffmann I 46 No. 67 Γιλ(λ)ίκα \spation{με} κατέϲταϲε ὀ [sic] Σταϲικρέτεοϲ. — \spation{με ἔϝεξε} \spation{Kypros}: Hoffmann~I 46 No.~66 [αὐ]τάρ με ἔϝεξε [Ὀναϲί]θεμιϲ. — \spation{μ᾽ ἔδωκε} \spation{Sikyon}: IGA. 22 Ἐπαίνετόϲ μ᾽ ἔδωκεν Χαρόπῳ. Abweichend die böotische Inschrift IGA. 219 Χάρηϲ ἔδωκεν Εὐπλοίωνί με. Wozu Röhl: “Versu trimetro dedicationem includere studuit Chares, sed male ei cessit.” (Vgl. übrigens auch die Stel-\hypertarget{p349}{\emph{[S. 349]}}\label{p349}lung von ϲοι in der attischen Inschrift IGA. 2 τηνδί ϲοι Θούδημοϲ δίδωϲι.)

In poetischen Weihinschriften findet sich so gestelltes με bis in die Kaiserzeit: Kaibel 821 Βάκχῳ \spation{μ[ε]} Βάκχον καὶ προϲυμναίᾳ θεῷ \spation{ϲτάϲαντο}. 822, 9 Δᾳδοῦχοϲ \spation{με} Κόρηϲ, Βαϲιλᾶν, Διόϲ, ἱερὰ ϲηκῶν Ἥραϲ κλεῖθρα φέρων βωμὸν \spation{ἔθηκε} Ῥέῃ. 877\textsuperscript{b} (S.~XIX) \spation{ἄνθετο} μέν μ᾽ Ἐπίδαυροϲ. Vgl. 868 Ἀϲκληπιοῦ \spation{με} δμῶα πυρφόρο[ν θεοῦ oder ξένε] Πείϲωνα \spation{λεύϲϲειϲ}. (Mit andrer Stellung von με Kaibel 809, 813, 843.)

Ganz ebenso die Künsterinschriften [sic]: \spation{μ᾽ ἐποίηϲε}, \spation{μ᾽ ἐποίει}: CIA. 4\textsuperscript{2} 373, 206 [Ε]ὐθυκλῆϲ μ᾽ ἐποίηϲεν. IGA. 492 (attische Inschrift von Sigeum) καί μ᾽ ἐπο(ίη)ϲεν Αἵϲωποϲ καὶ ἁδελφοί. CIA. 1, 466 Ἀριϲτίων μ᾽ ἐπόηϲεν. 1, 469 (vgl. Löwy Inschriften griechischer Bildhauer S.~15) Ἄριϲτίων Πάρι[όϲ μ᾽ ἐπ]ό[ηϲ]ε (die Ergänzung sicher!). IGA. 378 (Thasos) Παρμένων \spation{με} ἐ[ποίηϲε]. IGA. 485 (Milet) Εὔδημόϲ \spation{με} ἐποίειν. IGA. 557 (Elis?) Κοῖόϲ μ᾽ ἀπόηϲεν. IGA. 22 (= Klein Griechische Vasen mit Meistersignaturen S.~40) Ἐξηκίαϲ μ’ ἐποίηϲε. Klein S.~41 Ἐξηκίαϲ μ᾽ ἐποίηϲεν εὖ. S.~31 Θεόζοτόϲ μ᾽ ἐπόηϲε. S.~34 Ἐργότιμόϲ μ’ ἐποίηϲεν. S.~43, 45 (bis!), 48 Ἄμαϲίϲ μ᾽ ἐποίηϲεν. S.~48 Χόλχοϲ μ᾽ ἐποίηϲεν. S.~66 -ϲ μ᾽ ἐποίηϲεν. S.~71 Νικοσθένηϲ μ᾽ ἐποίηϲεν. S.~75 Ἀνακλῆϲ \spation{με} ἐποίηϲεν. S.~75 Νικοσθένηϲ \spation{με} ἐποίηϲεν. S.~76 Ἀρχεκλῆϲ μ᾽ ἐποίηϲεν. S.~77 Γλαυκίτηϲ μ᾽ ἐποίηϲεν. S.~84 (bis!) Τληνπόλεμόϲ μ᾽ ἐποίηϲεν. S.~85 Γάγεοϲ μ᾽ ἐποίηϲεν. S.~90 Πανφαῖόϲ μ᾽ ἐποίηϲεν. S.~213 Λυϲίαϲ μ᾽ ἐποίηϲεν ἡμιχώνῃ. Dazu die metrische Aufschrift IGA. 536 [Γλαυκία]ι \spation{με} Κάλων γε[νεᾷ ϝ]αλεῖ[ο]ϲ ἐποίει. Dagegen kommt Löwy No.~411 [Ἀρτέ]μων \spation{με} ἐποίηϲε durch die Behandlung der Inschrift bei Köhler CIA. 2, 1181 in Wegfall. — Der Regel widerspricht Klein S.~51 Χαριταῖοϲ ἐποίηϲεν \spation{με}. Hier hat wohl <ἐ>μέ entweder ursprünglich dagestanden oder ist wenigstens beabsichtigt gewesen. (Vgl. über ἐμέ unten S.~351).

\spation{μ᾽ ἔγραψε}, \spation{μ᾽ ἔγραφε}: IGA. 20, 102 (Korinth) -ων μ᾽ [ἔγραψε] nach der Ergänzung von Blass No. 3119e Collitz. Kyprische Inschrift bei Hoffmann I 90 No.~189 -οικόϲ \spation{με} γράφει Σελαμίνιοϲ. Klein S.~29 Τιμωνίδαϲ μ᾽ ἔγραφε. S.~30 Χάρηϲ μ᾽ ἔγραψε. S.~38 Νέαρχόϲ μ᾽ ἔγραψεν καὶ <ἐποίηϲεν>. — Abweichend IGA. 474 (Kreta) -μων ἔγραφέ \spation{με}. Doch lässt sich diese Ausnahme leicht durch die Schreibung ἔγραφ᾽ ἐμέ beseitigen. Vergleiche die Inschrift bei Klein S.~40 κἀποίηϲ᾽ \hypertarget{p350}{\emph{[S. 350]}}\label{p350} ἐμέ mit eben solcher Elision, wo ἐμέ durch andere Aufzeichnungen derselben Inschrift mit ἐπόηϲε ἐμέ gesichert ist. [Vgl. in Betr. des inschriftlichen με noch die Nachträge.]

Zu den auf Steinen und Vasen überlieferten Inschriften mit με kommen einige z.~T. recht alte von Pausanias aus Olympia beigebrachte hinzu. 5, 25, 13 = 8, 42, 10 (aus Thasos) υἱόϲ μέν \spation{με} Μίκωνοϲ Ὀνάταϲ ἐξετέλεϲϲεν. 6, 10, 7 (5.~Jahrhundert) Κλεοϲθένηϲ μ᾽ ἀνέθηκεν ὁ Πόντιοϲ ἐξ Ἐπιδάμνου. 6, 19, 6 (altattisch) Ζηνί μ᾽ ἄγαλμ᾽ ἀνέθηκαν. In dem Epigramm bei Paus. 5, 23, 7 Zeile~3 καὶ μετρεῖτ᾽ Ἀρίϲτων ἠδὲ Τελέϲταϲ αὐτοκαϲίγνητοι καλὰ Λάκωνεϲ *ἔϲαν verbessert F. Dümmler nach freundlicher Mitteilung καί \spation{με} Κλειτορίοιϲ Ἀρίϲτων κτλ. — Hierher gehören auch die von Herodot 5, 59 und 5, 60 aus dem Ismenion beigebrachten Aufschriften Ἀμφιτρύων μ᾽ ἀνέθηκεν *ἐὼν ἀπὸ Τηλεβοάων und Σκαῖοϲ πυγμαχέων με ἑκηβόλῳ Ἀπόλλωνι νικήϲαϲ ἀνέθηκε, letztere die einzige regelwidrige in dieser Gruppe, zudem, weil metrisch, nicht schwer ins Gewicht fallend.

Auch die jüngern Epigrammatiker haben, wo sie das altertümliche με für ihre gedichteten Aufschriften anwandten, sich mit auffälliger Strenge an die Norm gehalten: Kallimachus Epigr.~23 (21 Wilamowitz), 1 ὅϲτιϲ ἐμὸν παρὰ σῆμα φέρειϲ πόδα, Καλλιμάχου \spation{με} ἴϲθι Κυρηναίου παῖδά τε καὶ γενέτην. 36 (34 W.), 1 τίν \spation{με}, λεοντάγχ᾽ ὦνα ϲυοκτόνε, φήγινον ὄζον θῆκε. 50 (49 W.), 1 τῆϲ Ἀγοράνακτοϲ \spation{με} λέγε, ξένε, κωμικὸν ὄντωϲ ἀγκεῖϲθαι νίκηϲ μάρτυρα τοῦ Ῥοδίου Πάμφιλον. 56 (55), 1 τῷ \spation{με} Κανωπίτῃ Καλλίϲτιον εἴκοϲι μύξαιϲ πλούϲιον ἡ Κριτίου λύχνον ἔθηκε θεῷ. Fragm. 95 (Laertius Diog.~1, 29) Θαλῆϲ \spation{με} τῷ μεδεῦντι Νείλεω δήμου δίδωϲι, τοῦτο δὶϲ λαβὼν ἀριϲτεῖον. — Anthol. Pal.~6, 49 (Athen.~6, 232~Β) καί μ᾽ ἐπὶ Πατρόκλῳ θῆκεν πόδαϲ ὠκὺϲ Ἀχιλλεύϲ. 6, 178, 1 δέξαι μ᾽ Ἡράκλειϲ Ἀρχεϲτράτου ἱερὸν ὅπλον. — Abweichend, doch nur unbedeutend abweichend 6, 209 1 Βιθυνὶϲ Κυθέρη \spation{με} τεῆϲ ἀνεθήκατο, Κύπρι, μορφῆϲ εἴδωλον λύγδινον εὐξαμένη. 6, 239, 1 ϲμήνεοϲ ἔκ \spation{με} ταμὼν γλυκερὸν θέροϲ ἀντὶ νομαίων γηραιὸϲ Κλείτων ϲπεῖϲε μελιϲϲοπόνοϲ. 6, 261, 1 χάλκεον ἀργυρέῳ \spation{με} πανείκελον, Ἰνδικὸν ἔργον, ὄλπην — — πέμπεν γηθομένῃ ϲὺν φρενὶ Κριναγόρηϲ. Dagegen wird für 6, 138, 1 πρὶν μὲν Καλλιτέληϲ μ᾽ ἱδρύϲατο die Überlieferung des Palatinus durch das auf einem Stein zum Vorschein gekommene Original \hypertarget{p351}{\emph{[S. 351]}}\label{p351} CIA. 1, 381 = Kaibel 758 widerlegt, das kein μ᾽ bietet. Hieraus ergiebt sich auch für 6, 140, 1 παιδὶ φιλοϲτεφάνῳ Σεμέλαϲ <μ᾽> ἀνέθηκε das von Hecker ergänzte μ᾽ als überflüssig.

Unsere Durchmusterung der Inschriften mit με ergiebt also, dass dasselbe bei poetischer Fassung mit Vorliebe, bei prosaischer so gut wie ausnahmslos an zweite Stelle gesetzt wurde. Denn wenn wir IGA. 474 ἔγραφ᾽ ἐμέ abteilen, Naukratis~1, 303 und 307, wo bloss ΜΕ bezw. ΕΜΕ überliefert ist, als ganz unsicher bei Seite lassen, endlich Naukratis~2, 750 die vom Schreiber der Inschrift wirklich gemeinte Wortfolge wiederherstellen, so bleiben nur IGA. 219 Χάρηϲ ἔδωκεν Εὐπλοίωνί με, was zwar nicht ein Vers ist, aber ein Vers sein will, und Klein S.~51 Χαριταῖοϲ ἐποίηϲέν με übrig. Letzteres ist also die einzige wirkliche Ausnahme; um so näher liegt die Vermutung eines Fehlers.

Andrerseits erhält unsre Regel noch weitere Bestätigung. Erstens dadurch, dass auch sonst in archaischen Inschriften, in welchen das Denkmal oder der durch das Denkmal Geehrte spricht, με die zweite Stelle hat: IGA. 473 (Rhodus) Κοϲμία ἠμί, ἆγε δέ \spation{με} Κλιτομίαϲ. 524 (Cumae) = Inscript. Siciliae ed. Kaibel 865 ὃϲ δ᾽ ἄν \spation{με} κλέψει, —. Zweitens (um dies einem spätern Abschnitt vorwegzunehmen) durch die analogen lateinischen Inschriften: \emph{Manios \spation{med} fefaked, Duenos \spation{med} feced, Novios Plautios \spation{med} Romai fecid}.

Besonders belehrend sind aber die paar Inschriften mit ἐμέ. Zweimal steht dieses ἐμέ auch an zweiter Stelle: IGA. 20, 8 (Korinth) Ἀπολλόδωροϲ \spation{ἐμὲ} ἀνέθ[ηκε] und Gazette archéol.~1888 S.~168 Μεναΐδαϲ \spation{ἐμ᾽} ἐποί(ϝ)ηϲε Χάροπ[ι]. Aber sechsmal steht ἐμέ anders: Klein S.~39 Ἐξεκίαϲ ἔγραψε κἀπόηϲε ἐμέ (Vers?) S.~40 Ἐξεκίαϲ ἔγραψε κἀ(ι)ποίηϲ᾽ \spation{ἐμέ} (Vers?). S.~51 Χαριταῖοϲ ἐποίηϲεν \spation{ἔμ᾽} εὖ. S.~82 Ἑρμογένηϲ ἐποίηϲεν \spation{ἐμέ}. S.~83 Ἑρμογένηϲ ἐποίηϲεν ἐνέ (liess \spation{ἐμέ}). S.~85 Σακω\-νίδηϲ ἔγραψεν ἐμέ. Diese Stellen zeigen, dass die regelmässige Stellung von με hinter dem ersten Wort nicht zufällig und dass sie durch seine enklitische Natur bedingt ist. [Vgl. noch die Nachträge.]

\section*{III.}
\addcontentsline{toc}{section}{III.}

Wichtiger für diese Frage (wie überhaupt für jede über etymologische Spielereien hinausreichende Sprachforschung) sind natürlich die umfangreichern Texte der ionischen und \hypertarget{p352}{\emph{[S. 352]}}\label{p352} der attischen Litteratur, vor allem wieder Herodot. So wenig allerdings, als bei μιν und οἱ, hat er bei den übrigen enklitischen Pronomina die alte Regel festgehalten.

Im siebenten Buche des Herodot findet sich ϲφεων 13 mal, davon 6 mal an zweiter Stelle; ϲφι 70 mal, davon 46 mal an zweiter Stelle; ϲφεαϲ 32 mal, davon 20 mal an zweiter Stelle; ϲφεα 1 mal, nicht an zweiter Stelle. Also von 116 Stellen, wo ϲφ-Formen vorliegen, folgen 72 der Regel, also ca. 62\%. Unvollständige Sammlungen aus den übrigen Büchern ergaben ein analoges Verhältnis.

Im Pronomen der zweiten Person haben wir in Herodot VII. ϲεο einmal, regelmässig; τοι (mit Ausschluss der Fälle, wo es deutlich Partikel ist) 45 mal, davon 18—20 mal an zweiter Stelle; ϲε 16 mal, davon 10 mal an zweiter Stelle. — Im Pronomen der ersten Person: μεο 3 mal, hiervon einmal regelmässig; μοι 37 mal, davon 24 mal an zweiter Stelle, wenn man 15, 6 ἔγνων δὲ ταῦτά \spation{μοι} ποιητέα ἐόντα. 47, 8 φέρε τοῦτό \spation{μοι} ἀτρεκέωϲ εἰπέ. 103, 3 ἄγε εἰπέ \spation{μοι} hierher stellen darf; με 6 mal, davon zweimal regelmässig. Also in der ersten und zweiten Person haben wir 58 mal regelmässige, 50 mal regelwidrige Stellung.

Es ergiebt sich aus dieser Statistik zwar mit völliger Klarheit, dass die alte Regel bei Herodot nicht mehr ohne weiters gilt, dass andere Stellungsregeln in Wirkung getreten sind. Aber zugleich auch, dass trotz und neben diesen neuern Regeln die alte Regel doch noch Kraft genug hat, um in mehr als der Hälfte der Fälle die Stellung des Pronomens zu bestimmen: freilich sind in dieser grössern Hälfte die Beispiele mit begriffen, wo für das Pronomen die zweite Stelle im Satz auch nach den jüngern Regeln das Natürliche war.

Bei den Attikern lassen Zählungen, die ich vorgenommen habe, auf ein noch weiteres Zurückgehen der alten Regel schliessen. Aber unverkennbare Spuren derselben finden sich in bestimmten Wendungen und Wortverbindungen auch noch bei ihnen, wie bei Herodot und überhaupt den nachhomerischen Autoren.

Jedem Leser der attischen Redner muss es auffallen, wie häufig der Aufforderungssatz, wodurch die Verlesung einer Urkunde oder das Herbeirufen von Zeugen veranlasst werden soll, mit καί μοι beginnt, ja man kann sagen, dass wenn er \hypertarget{p353}{\emph{[S. 353]}}\label{p353} überhaupt mit καί beginnt und μοι enthält, μοι sich ausnahmslos unmittelbar an καί anschliesst. Ich ordne die Beispiele nach der Chronologie der Redner, und die Wendungen nach der Zeit des ältesten Beispiels.

\spation{καί μοι} κάλει mit folgendem Objekt Andoc. 1, 14. 1, 28. 1, 112. Lys. 13, 79. 17, 2. 17, 3. 17, 9. 19, 59, 31, 16. Isocrates 17, 12. 17, 16. 18, 8. 18, 54. Isaeus 6, 37. 7, 10. 8, 42. 10, 7. Demosth. 29, 12. 29, 18. 41, 6. 57, 12. 57, 38. 57, 39. 57, 46. [Demosth.] 44, 14. 44, 44. 58, 32. 58, 33. 59, 25. 59, 28. 59, 32. 59, 34. 59, 40. Aeschines 1, 100. Oder mit andrer Stellung des Objekts \spation{καί μοι} μάρτυραϲ τούτων κάλει Antiphon 5, 56; \spation{καί μοι} ἁπάντων τούτων τοὺϲ μάρτυραϲ κάλει Andoc. 1, 127; \spation{καί μοι} τούτουϲ κάλει πρῶτον Isäus 5, 11.

\spation{καί μοι} λαβὲ καὶ ἀνάγνωθι mit folgendem Objekt Andoc. 1, 13. 1, 15.

\spation{καί μοι} ἀνάγνωθι mit folgendem Objekt Andoc. 1, 34. 1, 76. 1, 82. 1, 85. 1, 86. 1, 87. 1, 96. Lysias 10, 14. 10, 15. 13, 35. 13, 50. 14, 8. Isokrates 15, 29. 17, 52. Isaeus 5, 2 bis. 5, 4. 6, 7. 6, 8. [Demosth.] 34, 10. 34, 11. 34, 20. 34, 39. 43, 16. 46, 26. 47, 17. 47, 20. 47, 40. 47, 44. 48, 30. 59, 52. Aeschines 3, 24. Oder mit andrer Stellung des Objekts \spation{καί μοι} τὰϲ μαρτυρίαϲ ἀνάγνωθι ταύταϲ (ταυταϲί) Isaeus 2, 16. 2, 34; \spation{καί μοι} τούτων ἀνάγνωθι τὴν μαρτυρίαν [Demosth.] 50, 42; \spation{καί μοι} λαβὼν ἀνάγνωθι πρῶτον τὸν Σόλωνοϲ νόμον Demosth. 57, 31. Ohne Objekt [Demosth.] 47, 24.

\spation{καί μοι} ἀνάβητε μάρτυρεϲ (oder τούτων μάρτυρεϲ) Lysias 1, 29. 1, 42. 13, 64. 16, 14. 16, 17. 32, 27; contra Aeschinem Fr. 1 (Orat. att. ed. Sauppe 2, 172, 26) bei Athen. 13, 612 F. Isokrates 17, 37. 17, 41; \spation{καί μοι} τούτων ἀνάβητε μάρτυρεϲ Isokr. 17, 14; \spation{καί μοι} ἀνάβητε δεῦρο Lysias 20, 29; καί μοι ἀνάβηθι Lysias 16, 13. Isokr. 17, 32.

\spation{καί μοι} δεῦρ᾽ ἴτε μάρτυρεϲ Lysias 1, 10.

\spation{καί μοι} λαβέ mit folgendem Objekt Lysias 9, 8. Isokr. 18, 19. 19, 14. Isaeus 6, 16. 6, 48. 8, 17. 12, 11. Lykurg 125. Demosth. 18, 222. 30, 10. 30, 32. 30, 34. 31, 4. 36, 4. 41, 24. 41, 28. 55, 14. 55, 35. 57, 19. 57, 25. [Demosth.] 34, 7. 34, 17. 44, 14. 48, 3. 58, 51. 59, 87. 59, 104. Aeschines 2, 65; \spation{καί μοι} πάλιν λαβέ [Demosth.] 58, 49.

\spation{καί μοι} ἀπόκριναι Lysias 13, 32.

\hypertarget{p354}{\emph{[S. 354]}}\label{p354} καί μοι ἐπίλαβε τὸ ὕδωρ Lysias 23, 4. 23, 8. 23, 11. 23, 14. 23, 15.

\spation{καί μοι} ἀναγίγνωϲκε mit folgendem Objekt Demosth 27, 8. [Demosth.] 35, 27.

\spation{καί μοι} λέγε mit folgendem Objekt Demosth. 19, 130. 19, 154. 19, 276. 18, 53. 18, 83. 18, 105. 18, 163. 18, 218. 32, 13. 37, 17. 38, 3. 38, 14. [Demosth.] 34, 9. 56, 38. Aeschines 2, 91. 3, 27. 3, 32. 3, 39.

\spation{καί μοι} φέρε τὸ ψήφιϲμα τὸ τότε γενόμενον Demosth. 18, 179.

Abweichend ist blos [sic] Aeschines 1, 50 \spation{καὶ} τελευταίαν δέ \spation{μοι} λαβὲ τὴν αὐτοῦ Μιϲγόλα μαρτυρίαν. Hier haben wir aber nicht blosses καί, sondern καὶ — δέ. Und vor diesem δέ, also hinter καί, war ein stark betontes Wort erforderlich, somit μοι unmöglich.

Aber auch ausserhalb dieser rednerischen Wendung ist καί μοι am Anfang von Sätzen in der ganzen nachhomerischen Litteratur merkwürdig häufig (vgl. Blass zu Demosth. 18, 199). Hier ein paar Beispiele; jedes Schriftwerk bietet solche. Archilochus Fragm. 22 Bgk. \spation{καί μ᾽} οὔτ᾽ ἰάμβων οὔτε τερπωλέων μέλει. 45 \spation{καί μοι} ϲύμμαχοϲ γουνουμένῳ ἵλαοϲ γενεῦ. Sappho Fragm. 79 \spation{καί μοι} —. Solon bei Aristoteles Ἀθηναίων πολιτ. 14, 3 Kenyon. γιγνώϲκω, \spation{καί μοι} φρενὸϲ ἔνδοθεν ἄλγεα κεῖται, πρεϲβυτάτην ἐϲορῶν γαῖαν Ἰαονίαϲ. Theognis 258 \spation{καί μοι} τοῦτ᾽ ἀνιηρότατον. 1199 \spation{καί μοι} κραδίην ἐπάταξε μέλαιναν. Sophokles Elektra 116 \spation{καί μοι} τὸν ἐμὸν πέμψατ᾽ άδελφόν. id. Λαριϲϲαῖοι Fragm. 349 Nauck \spation{καί μοι} τρίτον ῥίπτοντι Δωτιεὺϲ ἀνὴρ ἀγχοῦ προϲῆψεν Ἔλατοϲ ἐν διϲκήματι. Herodot 7, 9\textsuperscript{a} 7 \spation{καί μοι} μέχρι Μακεδονίηϲ ἐλάϲαντι οὐδεὶϲ ἠντιώθη. 7, 152, 13 \spation{καί μοι} τοῦτο τὸ ἔποϲ ἐχέτω ἐϲ πάντα λόγον. Euripides Medea 1222 \spation{καί μοι} τὸ μὲν ϲὸν ἐκποδὼν ἔϲτω λόγου. Thucyd. 1, 137, 4 \spation{καί μοι} εὐεργεϲία ὀφείλεται. Aristoph. Ran. 755 \spation{καί μοι} φράϲον. Ekkles. 47 \spation{καί μοι} δοκεῖ κατὰ ϲχολὴν παρὰ τἀνδρὸϲ ἐξελθεῖν μόνη. Plato Apologie 21 D \spation{καί μοι} ταὐτὰ ταῦτα ἔδοξε. 25 Α (= Gorg. 462 Β) \spation{καί μοι} ἀπόκριναι. 31 E \spation{καί μοι} μὴ ἄχθεϲθε λέγοντι τἀληθῆ. Phaedo 60 C \spation{καί μοι} δοκεῖ (scil. Αἴϲωποϲ) — μῦθον ἂν ϲυνθεῖναι. 63 Α \spation{καί μοι} δοκεῖ Κέβηϲ εἰϲ ϲὲ τείνειν τὸν λόγον. (97 D \spation{καί μοι} φράϲειν.) 98 C \spation{καί μοι} ἔδοξεν (scil. Ἀναξαγόραϲ) ὁμοιότατον πεπονθέναι. Sympos. 173 Β \spation{καί μοι} ὡμολόγει. \hypertarget{p355}{\emph{[S. 355]}}\label{p355} 189 B \spation{καί μοι} ἔϲτω ἄρρητα τὰ εἰρημένα. 218 C \spation{καί μοι} φαίνῃ ὀκνεῖν. Gorgias 449 C \spation{καί μοι} ἐπίδειξιν αὐτοῦ τούτου ποίηϲαι. 482 Α \spation{καί μοί} ἐϲτιν τῶν ἑτέρων παιδικῶν πολὺ ἧττον ἔμπληκτοϲ. 485 Β \spation{καί μοι} δοκεῖ δουλοπρεπέϲ τι εἶναι. 492 D = 494 Β \spation{καί μοι} λέγε. 499 C \spation{καί μοι} ὥϲπερ παιδὶ χρῇ. Charmides 157 Β \spation{καί μοι} πάνυ ϲφόδρα ἐνετέλλετο. Sophistes 216 Β \spation{καί μοι} δοκεῖ θεὸϲ μὲν ἁνὴρ οὐδαμῶϲ εἶναι. 233 D \spation{καί μοι} πειρῶ προϲέχων τὸν νοῦν εὖ μάλα ἀποκρίναϲθαι, wo μοι vom regierenden Verbum durch πειρῶ getrennt ist. Leges 1, 642 C \spation{καί μοι} νῦν ἥ τε φωνὴ προϲφιλὴϲ ὑμῶν. Demosth. 18, 280 \spation{καί μοι} δοκεῖϲ προελέϲθαι. Philemon Fragm. 4, 4 Kock (2 S. 479) \spation{καί μοι} λέγειν τοῦτ᾽ ἔϲτιν ἁρμοϲτόν, Σόλων. Kallimachus Epigr. 41 (40 Wilamow.), 5 \spation{καί μοι} τέκν᾽ ἐγένοντο δύ᾽ ἄρσενα. (Recht selten ist μοι an ein satzeinleitendes καί \spation{nicht} angeschlossen: Plato Gorg. 485 C \spation{καὶ} πρέπειν \spation{μοι} δοκεῖ. 486 D \spation{καὶ} οὐδέν \spation{μοι} δεῖ ἄλληϲ βαϲάνου. Demosth. 18, 246 \spation{καὶ} ταῦτά \spation{μοι} πάντα πεποίηται.) [καί μοι auch Eurip. Hippol. 377. 1373.]

Speziell gehören zusammen als Beispiele sogenannter Prodiorthose (Blass zu Demosth. 18, 199) Plato Apol. 20 E \spation{καί μοι}, ὦ ἄνδρεϲ Ἀθηναῖοι, μὴ θορυβήϲητε. Vgl. die oben angeführte Stelle 31 E. Gorgias 486 A \spation{καί μοι} μηδὲν ἀχθεϲθῇϲ. Demosth. 5, 15 \spation{καί μοι} μὴ θορυβήϲῃ μηδείϲ. 20, 102 \spation{καί μοι} μηδὲν ὀργιϲθῇϲ. Und diesen Stellen sind wieder ganz ähnlich, nur dass wir den Genetiv des Pronomens haben, Demosth. 18, 199 \spation{καί μου} πρὸϲ Διὸϲ καὶ θεῶν μηδὲ εἷϲ τὴν ὑπερβολὴν θαυμάϲῃ. 18, 256 \spation{καί μου} πρὸϲ Διὸϲ μηδεμίαν ψυχρότητα καταγνῷ μηδείϲ.

Überhaupt ist die Neigung, das Pronomen an satzeinleitendes καί anzuschliessen, nicht auf μοι beschränkt. Gerade καί μου findet sich auch noch Theognis 1366 \spation{καί μου} παῦρ᾽ ἐπάκουϲον ἔπη. Aristoph. Ran. 1006 \spation{καί μου} τὰ ϲπλάγχν᾽ ἀγανακτεῖ. Plato Apol. 22 D \spation{καί μου} ταύτῃ ϲοφώτεροι ἦϲαν. Republ. 1, 327 Β \spation{καί μου} ὄπιϲθεν ὁ παῖϲ λαβόμενοϲ τοῦ ἱματίου. Parmen. 126 Α \spation{καί μου} λαβόμενοϲ τῆϲ χειρόϲ.

Für καί με erinnere ich an die schon vorher aufgeführten Weih- und Küstlerinschriften, die es enthalten: IGA. 492. Kyprisch Deecke 1, 71. Pausan. 5, 23, 7. Anthol. Pal. 6, 49. Vgl. Kaibel 806 \spation{καί μ᾽} ἔϲτεψε πατὴρ (ε)ἰϲαρίθμοιϲ ἔπεϲι. Jungkyprische Inschr. Deecke No. 30 \spation{καί με} χθὼν ἧδε καλύπτει. Dazu kommt \hypertarget{p356}{\emph{[S. 356]}}\label{p356} noch (Solon bei Aristot. Ἀθην. πολ. S. 30, 1 Kenyon. κἀδόκουν ἕκαϲτοϲ αὐτῶν ὄλβον εὑρήϲειν πολὺν \spation{καί με} κωτίλλοντα λείωϲ τραχὺν ἐκφανεῖν νόον.) Anakreon Fragm. 60 \spation{καί μ᾽} ἐπίβωτον κατὰ γείτοναϲ ποιήϲειϲ. Hipponax Fragm. 64 \spation{καί με} δεϲπότεω βεβροῦ λαχόντα λίϲϲομαι ϲε μὴ ῥαπίζεϲθαι. Theognis 503 \spation{καί με} βιᾶται οἶνοϲ. 786 \spation{καί μ᾽} ἐφίλευν προφρόνωϲ πάντεϲ ἐπερχόμενον. Sophokles Oed. Rex 72 \spation{καί μ’} ἦμαρ ἤδη ξυμμετρούμενον χρόνῳ λυπεῖ τί πράϲϲει. (Herodot 3, 35, 7 φάναι Πέρϲαϲ τε λέγειν ἀληθέα \spation{καί με} μὴ ϲωφρονέειν). Eurip. Alkestis 641 \spation{καί μ᾽} οὐ νομίζω παῖδα ϲὸν πεφυκέναι. Andromache 334 τέθνηκα τῇ ϲῇ θυγατρὶ \spation{καί μ᾽} ἀπώλεϲε. Med. 338 \spation{καί μ᾽} ἀπάλλαξον πόνων. Helena (278 πόϲιν ποθ᾽ ἥξειν \spation{καί μ᾽} ἀπαλλάξειν κακῶν.) 557 \spation{καί μ᾽} ἑλὼν θέλει δοῦναι τυράννοιϲ. Orestes 796 \spation{καί με} πρὸϲ τύμβον πόρευϲα πατρόϲ. 869 \spation{καί μ᾽} ἔφερβε ϲὸϲ δόμοϲ. Aristoph. [Eq. 862] Ran. (338 \spation{καί μ᾽} ἀϲφαλῶϲ πανήμερον παῖϲαί τε καὶ χορεῦϲαι.) [389 καί — με]. 916 \spation{καί με} τοῦτ᾽ ἔτερπεν. Plut. 353 \spation{καί μ᾽} οὐκ ἀρέϲκει. Demosth. 18, 59 \spation{καί με} μηδεὶϲ ἀπαρτᾶν νομίϲῃ τὸν λόγον τῆϲ γραφῆϲ.

Pronomen der II. Person: Theognis 241 \spation{καί ϲε} — νέοι ἄνδρεϲ — ᾄϲονται. 465 \spation{καί ϲοι} τὰ δίκαια φίλ᾽ ἔϲτω. 692 \spation{καί ϲε} Ποϲειδάων χάρμα φίλοιϲ ἀνάγοι. Herodot 7, 11, 4 \spation{καί τοι} ταύτην τὴν ἀτιμίην προϲτίθημι ἐόντι κακῷ καὶ ἀθύμῳ. Eurip. Medea 456 \spation{καί ϲ᾽} ἐβουλόμην μένειν. Helena 1280 \spation{καί ϲ᾽} οὐ κεναῖϲι χερϲὶ γῆϲ ἀποϲτελῶ. 1387 \spation{καί ϲε} προϲποιούμεθα (Nauck \spation{καὶ ϲέ}). Orestes 755 \spation{καί ϲ᾽} ἀναγκαῖον θανεῖν. 1047 \spation{καί ϲ᾽} ἀμείψαϲθαι θέλω φιλότητι χειρῶν. Bacch. 1172 ὁρῶ \spation{καί ϲε} δέξομαι ϲύγκωμον. Aristoph. Equites 300 \spation{καί ϲε} φαίνω τοῖϲ πρυτάνεϲιν. Pax 396 \spation{καί ϲε} θυϲίαιϲιν ἱεραῖϲι — ἀγαλοῦμεν. 403 \spation{καί ϲοι} φράϲω τι πρᾶγμα. 418 \spation{καί ϲοι} (al. καὶ ϲοὶ) τὰ μεγάλ᾽ ἡμεῖϲ Παναθήναι᾽ ἄξομεν. Plato Gorg. 482 D \spation{καί ϲου} κατεγέλα. 527 Α \spation{καί ϲε} ἴϲωϲ τυπτήϲει τιϲ. Anthol. Pal. 6, 157, 3 \spation{καί ϲοι} ἐπιρρέξει Γόργοϲ χιμάροιο νομαίηϲ αἷμα. Vgl. das oben S. 344 angeführte Fragm. lyr. adesp. 43 A \spation{καί τυ} φίλιππον ἔθηκεν.

Pronomen der III. Person: Archilochus Fragm. 27, 2 \spation{καί ϲφεαϲ} ὄλλυ᾽ ὥϲπερ ὀλλύειϲ. 74, 8 \spation{καί} ϲφιν θαλάϲϲηϲ ἠχέεντα κύματα φίλτερ᾽ ἠπείρου γένηται. Mimnerm. Fragm. 15 \spation{καί μιν} ἐπ᾽ ἀνθρώπουϲ βάξιϲ ἔχει χαλεπή. Theognis 405 \spation{καί οἱ} ἔθηκε δοκεῖν. 422 \spation{καί ϲφιν} πολλ᾽ ἀμέλητα μέλει. 732 \spation{καί σφιν} τοῦτο γένοιτο φίλον. 1347 \spation{καί μιν} ἔθηκεν δαίμονα. \hypertarget{p357}{\emph{[S. 357]}}\label{p357} Herodot 4, 119, 2 \spation{καί ϲφεων} ἐϲ\-χίϲ\-θη\-ϲαν αἱ γνῶμαι. Eurip. Or. 1200 \spation{καί νιν} δοκῶ. Bacch. 231 \spation{καί ϲφαϲ} ϲιδηραῖϲ ἁρμόϲαϲ ἐν ἄρκυϲι παύϲω — τῆϲδε βακχείαϲ. Kallimach. Epigr. 14 (12 Wilamow.), 3 \spation{καί ϲφιν} ἀνιηρὸν μὲν ἐρεῖϲ ἔποϲ, ἔμπα δὲ λέξειϲ.

Ein Beispiel für καί με und eines für καί ϲφεαϲ sei besonders herausgehoben: Plato Gorg. 506 B \spation{καί με} ἐὰν ἐξελέγχῃϲ, οὐκ ἀπεχθήϲομαί ϲοι. Herodot 6, 34, 12 \spation{καί ϲφεαϲ} ὡϲ οὐδεὶϲ ἐκάλεε, ἐκτράπονται ἐπ᾽ Ἀθηνέων. An beiden Stellen ist das Pronomen aus dem Nebensatz, in den es gehört, herausgenommen und an καί angehängt. — Übrigens findet sich καί mit folgendem enklitischem Pronomen auch bei Homer schon oft.

Auch noch andern regelmässig oder oft am Anfang des Satzes stehenden Partikeln ist diese Attraktionskraft eigen: so οὐ, μή, γάρ, εἰ, ἐάν. Auch ἀλλά ist hier zu nennen: Archiloch. 58, 3 \spation{ἀλλά μοί} [sic] ϲμικρόϲ τιϲ εἴη. 85 \spation{ἀλλά μ᾽} ὁ λυϲιμελήϲ, ὦταῖρε δάμναται πόθοϲ. Alcaeus 55, 2 θέλω τι ϝείπην, \spation{ἀλλά με} κωλύει αἴδωϲ. Theognis 941 \spation{ἀλλά μ᾽} ἑταῖροϲ ἐκλείπει. 1155 \spation{ἀλλά μοι} εἴη ζῆν ἀπὸ τῶν ὀλίγων. Eurip. Or. 1323 \spation{ἀλλά μοι} φόβοϲ τιϲ εἰϲελήλυθ(ε). Aristoph. Ran. 1338 (euripidisierend) \spation{ἀλλά μοι} ἀμφίπολοι λύχνον ἅψατε. Häufig ist ἀλλά μοι bei Plato (Apol. 39 Ε, 41 D, Phaedo 63 E, 72 D. Sympos. 207 C, 213 A. Gorgias 453 A, 476 B, 517 B u. s. w.). ἀλλά ϲε Theognis 1287, 1333. Eurip. Med. 759, 1389 u. s. w.

Ferner finden wir, wie bei Homer und Sappho, das enklitische Pronomen mehrmals sogar an einen Vokativ angelehnt, wenn ein solcher erstes Wort des Satzes ist oder auf das erste Wort des Satzes folgt: Hipponax Fragm. 85, 1 Μοῦϲά \spation{μοι} Εὐρυμεδοντιάδεα — ἐννεφ᾽ —. Vgl. Fragm. lyr. adesp. 30 A (Poetae lyr. ed. Bergk 3, 696) Μοῖϲά \spation{μοι} ἀμφὶ Σκάμανδρον ἐύρροον ἄρχομ᾽ ἀείδειν. Sophokles Antig. 544 μήτοι καϲιγνήτη μ᾽ ἀτιμάϲῃϲ. Eurip. Heraclid. 79 ὁδ᾽ ὦ ξένοι \spation{με}, ϲοὺϲ ἀτιμάζων θεούϲ, ἕλκει. Helena 670 ὁ Διόϲ, ὁ Διόϲ, ὦ πόϲι \spation{με} παῖϲ Ἑρμᾶϲ ἐπέλαϲεν Νείλῳ. Bacch. 1120 οἴκτιρε δ᾽ ὦ μῆτέρ \spation{με}. Andromeda Fragm. 118 Ν. ἔαϲον Ἀχοῖ \spation{με} ϲὺν φίλαιϲιν γόου κόρον λαβεῖν. Aristoph. Thesmoph. 1134 μέμνηϲο Περϲεῦ μ᾽ ὡϲ καταλείπειϲ. Theokrit. 2, 95 εἶ᾽ ἄγε Θεϲτυλί \spation{μοι} χαλεπᾶϲ νόϲω εὑρέ τι μᾶχοϲ.

Verwandt damit ist die Anlehnung an einen vorausge-\hypertarget{p358}{\emph{[S. 358]}}\label{p358}schickten imperativischen Ausdruck, wie im homerischen ἀλλ᾽ ἄγε μοι: Eurip. Bacch. 341 δεῦρό \spation{ϲου} ϲτέψω κάρα. Iphig. Aul. 1436 παῦϲαί \spation{με μὴ} κάκιζε, wo με zu κάκιζε gehört. Plato Gorg. 464 Β φέρε δή \spation{ϲοι}, ἐὰν δύνωμαι, ϲαφέϲτερον ἀποδείξω. 495 C ἴθι δή \spation{μοι}, ἐπειδὴ —, διελοῦ τάδε. Ion 535 Β ἔχε δή μοι τόδε εἰπέ. Ebenso die Anlehnung an βούλει, wenn eine 1. Sing. Konjunktivi folgt: Eurip. Kyklops 149 βούλει \spation{ϲε} γεύϲω. Plato Gorg. 516 C βούλει \spation{ϲοι} ὁμολογήϲω. 521 D βούλει \spation{ϲοι} εἴπω. Aeschines 3, 163 βούλει \spation{ϲε} θῶ φοβηθῆναι. — Im allgemeinen ähnlich sind Plato Euthydem. 297 C νεωϲτί, \spation{μοι} δοκεῖν, καταπεπλευκότι und Parmen. 137 Β τί οὖν, εἰπεῖν, \spation{μοι} ἀποκρινεῖται.

Öfters finden wir nun aber ein solches Pronomen der zweiten Stelle im Satz zu lieb von den Wörtern getrennt, zu denen es syntaktisch gehört. Theognis 559 λῷϲτά \spation{ϲε} μήτε λίην ἀφνεὸν κτεάτεϲϲι μήτε ϲέ γ᾽ ἐϲ πολλὴν χρημοϲύνην ἐλάϲαι. Wieder anders Eurip. Iphig. Taur. 1004 οὐδέ μ᾽ εἰ θανεῖν χρεών. Aristoph. Lysistr. 753 ἵνα μ᾽ εἰ καταλάβοι ὁ τόκοϲ ἔτ᾽ ἐν πόλει, τέκοιμι. Theokrit 2, 4 ὅϲ \spation{μοι} δωδεκαταῖοϲ ἀφ᾽ ὧ τάλαϲ οὐδέποθ᾽ ἵκει. Vgl. oben S. 357 über καί με, καί ϲφεαϲ. — Bei Partizipien: Sophokles Antig. 450 οὐ γάρ τί \spation{μοι} Ζεὺϲ ἦν ὁ \spation{κηρύξαϲ} τάδε. Eurip. Iphig. Aul. 1459 τίϲ μ᾽ εἶϲιν ἄξων. Plato Gorg. 521 D πονηρόϲ τίϲ μ᾽ ἔϲται ὁ \spation{εἰϲάγων}. [Demosth.] 59, 1 πολλά \spation{με} τὰ παρακαλοῦντα ἦν. (Vgl. auch Kock zu Aristoph. Av. 95). — Herodot 7, 235, 18 τάδε \spation{τοι} προϲδόκα \spation{ἔϲεϲθαι}. — Sophokles Antig. 546 μή \spation{μοι} θάνῃϲ ϲὺ \spation{κοινά}.

Leicht trennt das Pronomen vermöge derartiger Stellung eng zusammengehörige Wörter. So finden wir bei Alkman 26, 1 οὔ μ᾽ ἔτι, παρθενικαὶ μελιγάρυεϲ ἱμερόφωνοι, γυῖα φέρειν δύναται und fragm. lyr. adesp. 5 (Poetae lyr. ed. Bergk~3, 690) οὔ \spation{μοι} ἔτ᾽ εὐκελάδων ὕμνων μέλει durch με, μοι die Partikel οὐκέτι zerrissen. Ähnlich Eurip. Orest. 803 εἴ \spation{ϲε} μἢν [sic] δειναῖϲιν ὄντα ϲυμφοραῖϲ ἐπαρκέϲω. Plato Apol. 29 Ε ἐάν \spation{μοι} μὴ δοκῇ. Phaedrus 236Ε ἐάν \spation{μοι} μὴ εἴπῃϲ, obwohl es sonst stets εἰ μή, ἐὰν μή in enger Verbindung heisst. Plato Gorgias 448 A οὐδείϲ \spation{μέ} πω ἠρώτηκεν καινὸν οὐδέν. Auch Herodot 7, 153, 17 θωῦμά \spation{μοι} ὦν καὶ τοῦτο γέγονεν gehört hierher, da sonst ὦν unmittelbar hinter dem ersten Satzwort zu stehen pflegt.

Ein attributiver Genetiv ist vom regierenden Wort getrennt \hypertarget{p359}{\emph{[S. 359]}}\label{p359} bei Ion, wenn er zu Beginn seiner Τριαγμοί (bei Harpokration s. v. Ἴων) sagt: ἀρχὴ δέ \spation{μοι} τοῦ λόγου (Lobeck ἀρχὴ ἧδέ μοι). Ähnlich Eurip. Medea 281 τίνοϲ μ᾽ ἕκατι γῆϲ ἀποϲτέλλειϲ. Helena 674 ἁ Δίοϲ [sic] μ᾽ ἄλοχοϲ ὤλεϲεν. 670 ὁ Διόϲ, ὦ πόϲι, \spation{με} παῖϲ Ἑρμᾶϲ ἐπέλαϲεν Νείλῳ. Thucyd. 1, 128, 7 εἰ οὖν τί \spation{ϲε} τούτων ἀρέϲκει für τι τούτων ϲε. Andoc. 1, 47 ὅϲουϲ \spation{μοι} τῶν ϲυγγόνων ἀπώλλυεν. Theokrit. 18, 19 Ζηνόϲ \spation{τοι} θυγάτηρ ὑπὸ τὰν μίαν ἵκετο χλαῖαν. [Allerdings auch ἐμέ so: Eurip. Heraklid. 687 οὐδεὶϲ ἔμ᾽ ἐχθρῶν προϲβλέπων ἀνέξεται]

Ein attributives Adjektiv oder Pronomen oder eine Apposition ist durch ein enklitisches Pronomen von dem Satzteil, zu dem es oder sie gehört, abgetrennt: Herodot 3, 14, 34 δεϲπότηϲ \spation{ϲε} Καμβύϲηϲ, Ψαμμήνιτε, εἰρωτᾷ. 6, 111, 8 ἀπὸ ταύτηϲ \spation{ϲφι} τῆϲ μάχηϲ — κατεύχεται ὁ κῆρυξ Πλαταιεῦϲι (durch Πλαταιεῦϲι wird das weit abliegende ϲφι wieder aufgenommen). 7, 16\textsuperscript{a} 2 τά ϲε καὶ ἀμφότερα περιήκοντα ἀνθρώπων κακῶν ὁμιλίαι ϲφάλλουϲιν, wo τά mit ἀμφότερα, ϲε mit περιήκοντα zusammengehört. 9, 45, 16 ὀλίγων γάρ ϲφι ἡμερέων λείπεται ϲιτία. [Hippokrates] περὶ τέχνηϲ S.~52, 18 Gomp. ωὑτὸϲ δέ \spation{μοι} λόγοϲ καὶ ὑπὲρ τῶν ἄλλων. Eurip. Medea 1013 πολλή μ᾽ ἀνάγκη. Helena 94 Αἴαϲ μ᾽ ἀδελφὸϲ ὤλεϲ᾽ ἐν Τροίᾳ θανών. 593 τοὐκεῖ \spation{με} μέγεθοϲ τῶν πόνων πείθει. 1281 φήμαϲ δέ \spation{μοι} ἐϲθλὰϲ ἐνεγκών. 1643 διϲϲοὶ δέ \spation{ϲε} Διόϲκοροι καλοῦϲιν. Orestes 167 Ἑλένη ϲ᾽ ἀδελφὴ ταῖϲδε δωρεῖται χοαίϲ. 482 φίλου \spation{μοι} πατρόϲ ἐϲτιν ἔκγονοϲ. 1626 Φοιβόϲ μ᾽ ὁ Λητοῦϲ παῖϲ ὁδ᾽ ἐγγὺϲ ὢν καλῶ. Fragm. 911 χρύϲεαι δή \spation{μοι} πτέρυγεϲ περὶ νώτῳ. Rhesos 401 τίϲ γάρ \spation{ϲε} κήρυξ ἢ γερουϲία Φρυγῶν — οὐκ ἐπέϲκηψεν πόλει. Aristoph. Ran. 1332 (Euripides nachbildend) τίνα \spation{μοι} δύϲτανον ὄνειρον πέμπειϲ. Ekkles. 1113 αὐτή τέ \spation{μοι} δέϲποινα μακαριωτάτη. Plato Apol. 37 C πολλὴ μέντἄν [sic] \spation{με} φιλοψυχία ἔχοι. 40 C μέγα \spation{μοι} τεκμήριον τούτου γέγονεν. Phaedo 92 C οὗτοϲ οὖν \spation{ϲοι} ὁ λόγοϲ ἐκείνῳ πῶϲ ξυνᾴϲεται. Gorg. 456 Β μέγα δέ \spation{ϲοι} τεκμήριον ἐρῶ. 487 D ἱκανόν \spation{μοι} τεκμήριον ἐϲτιν. 488 Β τοῦτό \spation{μοι} αὐτὸ ϲαφῶϲ διόριϲον. 493 D φέρε δή, ἄλλην \spation{ϲοι} εἰκόνα λέγω. 513 C ὅντινά \spation{μοι} τρόπον δοκεῖϲ εὖ λέγειν. Phileb. 23 D τετάρτου \spation{μοι} γένουϲ αὖ προϲδεῖν φαίνεται. Xenophon Hellen. 3, 1, 11 ὁ ἀνήρ \spation{ϲοι} ὁ ἐμὸϲ καὶ τἆλλα φίλοϲ ἦν. Aeschin. 1, 116 δύο δέ \spation{μοι} τῆϲ κατηγορίαϲ εἴδη λέλειπται. Bion~9, 1 ἁ μεγάλα \spation{μοι} Κύ-\hypertarget{p360}{\emph{[S. 360]}}\label{p360}πριϲ ἔθ᾽ ὑπνώντι παρέϲτα. Leonidas Tarent. Anthol. Pal. 7, 660 Ξεῖνε, Συρηκόϲιόϲ \spation{τοι} ἀνὴρ τόδ᾽ ἐφίεται Ὄρθων. Die zahlreichen Stellen, wo auf so eingeschobenes Pronomen zunächst das Verbum folgt, wie Eurip. Heraclid. 236 τριϲϲαί μ᾽ ἀναγκάζουϲιν ϲυμφορᾶϲ ὁδοί. Plato Gorg. 463 B ταύτηϲ \spation{μοι} δοκεῖ πολλὰ — μόρια εἶναι. Kallimach. Epigr. 1, 3 δοῖόϲ \spation{με} καλεῖ γάμοϲ, will ich nicht alle aufführen, obwohl sie m. E. auch hierher gehören. In anderer Weise gehört hierher Plato Apol. 28 A ὅτι πολλή \spation{μοι} ἀπέχθεια γέγονεν καὶ πρὸϲ πολλούϲ u. dergl.

Oder das Pronomen schliesst sich an den Artikel an. Selten unmittelbar: The\-ognis 575=862 οἵ \spation{με} φίλοι προδιδοῦϲιν. 813 οἵ \spation{με} φίλοι προὔδωκαν. Theokrit 7, 43 τάν τοι, ἔφα, κορύναν δωρύττομαι. Meist folgt dem Artikel zunächst eine ‘postpositive’ Partikel: Herodot 1, 31, 10 οἱ δέ \spation{ϲφι} βόεϲ οὐ παρεγένοντο. 1, 115, 8 οἱ γάρ \spation{με} ἐκ τῆϲ κώμηϲ παῖδεϲ — ἐϲτήϲαντο βαϲιλέα. 1, 207, 6 τὰ δέ \spation{μοι} παθήματα τὰ ἐόντα ἀχάριτα μαθήματα γέγονε. 3, 63, 10 ὁ δέ \spation{μοι} μάγοϲ ταῦτα ἐνετείλατο. Aristoph. Ekkles. 913 ἡ γάρ \spation{μοι} μήτηρ βέβηκεν ἄλλῃ. Plato Phaedrus 236 D ὁ δέ \spation{μοι} λόγοϲ ὅρκοϲ ἔϲται. Sympos. 177 Α ἡ μέν \spation{μοι} ἀρχὴ τοῦ λόγου ἐϲτὶ κατὰ τὴν Εὐριπίδου Μελανίππην. Theokrit 5, 125 τὰ δέ \spation{τοι} ϲία καρπὸν ἐνείκαι. 1, 82 ἁ δέ \spation{τυ} κώρα πάϲαϲ ἀνὰ κράναϲ — φορείται φοιτεῦϲ(α). (Siehe oben S. 344).

Oder das Pronomen lehnt sich an eine Präposition und trennt sie dadurch von ihrem Kasus: Terpander Fragm. 2 ἀμφί \spation{μοι} αὖτε ἄναχθ᾽ ἑκαταβόλον ᾀδέτω ἁ φρήν. Hymn. auf Pan 1 ἀμφί \spation{μοι} Ἑρμείαο φίλον γόνον ἔννεπε Μοῦϲα. Rhesos 831 κατά \spation{με} γᾶϲ ζῶντα πόρευϲον. Auf die Präposition folgt zunächst noch eine Partikel Herodot 3, 69, 20 ἐν γάρ \spation{ϲε} τῇ νυκτὶ ταύτῃ ἀναιρέομαι. Kallimach. Hymn. 1, 10 ἐν δέ \spation{ϲε} Παρραϲίῃ Ῥείη τέκεν. Epigr. 2, 1 ἐϲ δέ \spation{με} δάκρυ ἤγαγεν.

Dazu der bekannte Fall, wo ein von wirklich gesetztem oder zu supplierendem Verbum des Bittens abhängiges ϲε zwischen πρόϲ und den davon ‘regierten’ Genetiv getreten ist: Eurip. Alc. 1098 μή, πρόϲ \spation{ϲε} τοῦ ϲπείραντοϲ ἄντομαι Διόϲ. Ähnlich Soph. Phil. 468. Oed. Col. 250. 1333. Eurip. Hiket. 277. (Dagegen Eurip. Med. 853 μή, πρὸϲ γονάτων \spation{ϲε} πάντωϲ πάντη ϲ᾽ ἱκετεύομεν). Das Verbum des Bittens ist zu ergänzen Soph. Trach. 436 μή, πρόϲ ϲε τοῦ κατ᾽ ἄκρον Οἰταῖον πάγον \hypertarget{p361}{\emph{[S. 361]}}\label{p361} Διὸϲ καταϲτράπτοντοϲ, ἐκκλέψῃϲ λόγον. Ebenso Eurip. Medea 324. Andromache 89. (Vgl. Iph. Taur. 1068.) In allen diesen Fällen nimmt ϲε die zweite Stelle hinter der nächst vorangehenden Interpunktion ein; Soph. Phil. 468 πρόϲ νύν \spation{ϲε} πατρόϲ, Oed. Col. 1333 πρόϲ νύν \spation{ϲε} κρηνῶν und Eurip. Helena 1237 πρόϲ νύν \spation{ϲε} γονάτων τῶνδ(ε), wo das enklitische νυν noch vorgeschoben ist, bilden natürlich keine Ausnahme. Aus den ausserattischen Dichtern kommt hinzu Alkman Fr. 52 πρὸϲ δέ \spation{τε} τῶν φίλων. Apollonius, dem wir dieses Fragment verdanken, scheint allerdings τε hier als orthotonisch zu betrachten, und ausschliesslich τυ als enklitische Akkusativform für das Dorische anzuerkennen. Aber enklitisches dorisches τε wird gesichert durch die Worte des Megarers Ar. Ach. 779 πάλιν τ᾽ ἀποιϲῶ ναὶ τὸν Ἑρμᾶν οἴκαδιϲ, wo man, weil man eben τὲ nicht anerkennen wollte, sich genötigt glaubte τυ mit unschönem Hiatus einzusetzen. Besonders aber ist Kallim. Fr. 114 = AP. 13, 10 zu vergleichen: ποτί τε Ζηνὸϲ (der Cod. Pal. ποτιτεζηνοϲ) ἱκνεῦμαι λιμενοϲκόπω; Bloomfield setzt unnötig das enklitische τυ. Immerhin fällt der von O. Schneider gegen ihn erhobene Vorwurf ‘foede erravit’ auf diesen selbst und die von ihm vorgezogene Vulgata-Schreibung ποτὶ τὲ Ζανὸϲ mit der sinnlosen Orthotonese und dem falschen Genetiv Ζανόϲ zurück.

Ohne Bezugnahme auf die zwei letztgenannten Stellen hat kürzlich Christ Philologische Kleinigkeiten München 1891 S. 4 f. für Pindar Olymp. 1, 48 ὕδατοϲ ὅτι \spation{τε} πυρὶ ζέοιϲαν εἰϲ ἀκμὰν μαχαίρᾳ τάμον κατὰ μέλη die Meinung geäussert, dass das als Partikel wenig ansprechende τε als Akkusativ des Pronomens zu nehmen sei, wie denn schon längst Bergk dafür hat ϲε einsetzen wollen. Die Stellung von τε empfiehlt diese Auffassung.

Aber auch gegenüber der Verbindung der Prä\-po\-si\-ti\-o\-nen mit dem Ver\-bum macht das alte Stellungsgesetz seinen Einfluss geltend (Krüger Dialektische Syntax 68, 48, 3). Man durchmustere die folgenden Beispiele nachhomerischer Tmesis: Alcäus Fr. 95 ἔκ μ᾽ ἔλαϲαϲ ἀλγέων. Anakreon 50, 1 ἀπό \spation{μοι} θανεῖν γένοιτ(ο). Hipponax Fr. 31 ἀπό ϲ᾽ ὀλέϲειεν Ἄρτεμιϲ, ϲὲ δὲ κὠπλλων. Sophokles El. 1067 κατά \spation{μοι} βόαϲον. Philoktet 817 ἀπό μ᾽ ὀλεῖϲ. Oed. Col. 1689 κατά \spation{με} φόνιοϲ Ἀίδαϲ ἕλοι. Eurip. Herakles 1053 διά μ᾽ ὀλεῖτε. Hiket. 45 ἀνά \hypertarget{p362}{\emph{[S. 362]}}\label{p362} \spation{μοι} τέκνα λῦϲαι. 829 κατά \spation{με} πέδον γᾶϲ ἕλοι. Hippolyt 1357 διά μ᾽ ἔφθειραϲ. Bacch. 579 ἀνά μ᾽ ἐκάλεϲεν. Aristoph. Acharn. 295 κατά \spation{ϲε} χώϲομεν. Plut. 65 ἀπό ϲ᾽ ὀλῶ κακὸν κακῶϲ. Plato Phaedr. 237 Α ξύμ \spation{μοι} λαβέϲθε τοῦ μύθου. Kallimach. Epigr. 1, 5 εἰ δ᾽ ἄγε, ϲύμ \spation{μοι} βούλευϲον. — Mit vorangehender Partikel u. dgl.: Sophokles Philoktet 1177 ἀπὸ νύν \spation{με} λείπετ᾽ ἤδη. Eurip. Or. 1047 ἔκ τοί \spation{με} τήξειϲ. Aristoph. Vesp. 437 ἔν τί \spation{ϲοι} παγήϲεται. 784 ἀνά τοί \spation{με} πείθειϲ. Vgl. oben S. 338 die ähnlichen Stellen mit νιν. Wenn vereinzelt (Alcäus Fr. 68 schrieb Bekker irrig τύφωϲ ἔκ ϲ᾽ ἕλετο φρέναϲ) das Pronomen durch solche Tmesis nicht an die zweite Stelle gekommen sein sollte, wird uns das nicht stören.

\section*{IV.}
\addcontentsline{toc}{section}{IV.}

Besondere Betrachtung verdienen μοι, τοι, (ϲφι), μεο — μευ — μου, ϲεο — ϲευ — ϲου, ϲφεων als attribute Genetive. Dass μοι, τοι, wie auch οἱ, die Genetivfunktion nicht erst nachträglich übernahmen, sondern entsprechend ihren indischen Korrelaten \emph{mē, tē, sē} von Haus aus besassen und mit dem Lokativ nichts zu thun haben (vgl. Delbrück Altind. Syntax S. 205), betrachte ich als sicher; dass die Genetivfunktion sich im Griechischen nicht bloss bei Homer (siehe Brugmann Grundriss II 819. Verf. Berliner philol. Woch. 1890 Sp. 39) und den Ioniern erhalten hat, ergibt sich zumal aus der Bemerkung von Wilamowitz zu Eurip. Herakles 626 (ϲύ τ᾽ ὦ γύναι μοι, ϲύλλογον ψυχῆϲ λαβέ): “Das Drama drückt in der Anrede das possessive Verhältnis bei Verwandtschaftswörtern durch den Dativ aus, θύγατέρ μοι, τέκνον μοι [Eurip. Ion 1399. Orestes 124. Iph. Aul. 613] γύναι μοι. Der Genetiv ist überhaupt nicht üblich; sein Eindringen, z. B. in der jüdisch-christlichen Litteratur, vielmehr ein Zeichen des Plebeiertums”.

Die natürlichste Stellung für diese Genetive schiene uns die hinter ihren Substantiven. Bekanntlich findet sich nun zwar diese recht oft, wie z. B. gerade bei den von Wilamowitz besprochenen vokativischen Verbindungen, aber daneben als völlig gleichberechtigt die Stellung vor dem Substantiv und dessen Attributen mit Einschluss des Artikels. Der Ursprung dieser seltsamen Stellung wird klar, wenn wir die ältesten Beispiele derselben prüfen. Schon Homer hat diese Stellung Α 273 καὶ μέν \spation{μευ} \spation{βουλέων} ξύνιεν. Ν 626 οἵ \spation{μευ} \hypertarget{p363}{\emph{[S. 363]}}\label{p363} \spation{κουριδίην} \spation{ἄλοχον} καὶ κτήματα πολλὰ μάψ᾽ οἴχεϲθ᾽ ἀνάγοντεϲ. Ε 311 καί \spation{μευ} \spation{κλέοϲ} ἦγον Ἀχαιοί. ι 20 καί \spation{μευ} \spation{κλέοϲ} οὐρανὸν ἵκει. (ι 405 ἦ μή τίϲ \spation{ϲευ μῆλα} βροτῶν ἀέκοντοϲ ἐλαύνει). μ 379 οἵ \spation{μευ βοῦϲ} ἔκτειναν. ο 467 οἵ \spation{μευ πατέρ᾽} ἀμφεπένοντο. κ 231 καί \spation{ϲευ φίλα γούναθ᾽} ἱκάνω. ω 381 τῷ κέ \spation{ϲφεων γούνατ᾽} ἔλυϲα hier überall so, dass sie durch unser Stellungsgesetz bewirkt ist. Die spätern haben sich dann gestattet diese Genetive weiter vom Satzanfang zu entfernen, aber die aus dem alten Stellungsgesetz folgende Voranstellung dann doch noch vielfach beibehalten. Nachwirkungen des ursprünglichen Zusammenhangs zwischen der Voranstellung und dem alten Stellungsgesetz zeigen sich aber mancherlei.

Erstens nehmen die vorangestellten Genetive eben doch häufig die zweite Stelle im Satz ein. Für μοι, τοι verweise ich auf Herodot 4, 29, 3 μαρτυρέει δέ \spation{μοι τῇ γνώμῃ} καὶ Ὁμήρου ἔποϲ. 7, 27, 8 ὅϲ \spation{τοι τὸν πατέρα} δωρήϲατο. Sophokles Trachin. 1233 ἥ \spation{μοι μητρὶ} μὲν θανεῖν μόνη μεταίτιοϲ. Für die eigentlichen Genetivformen auf folgende, die Zahl der Belege natürlich bei weitem nicht erschöpfende Beispiele: Hipponax Fragm. 76 λαιμᾷ δέ \spation{ϲευ τὸ χεῖλοϲ}. 83 λάβετέ \spation{μευ θαἰμάτια}. Herodot 4, 80, 11 ἔχειϲ δέ \spation{μευ τὸν ἀδελφεόν}. 7, 51, 3 ϲὺ δέ \spation{μευ ϲυμβουλίην} ἔνδεξαι. Eurip. Medea 1233 ὥϲ \spation{ϲου ϲυμφορὰϲ} οἰκτίρομεν. Helena 277 ἥ \spation{μου τὰϲ τύχαϲ} ὤχει μόνη. Hiket. 1162 ἔθιγέ \spation{μου φρενῶν}. Orestes 297 ϲύ \spation{μου τὸ δεινὸν καὶ διαφθαρὲν φρενῶν} ἴϲχναινε. Aristoph. Eq. 289 κυνοκοπήϲω ϲου τὸ νῶτον. 709 ἀπονυχιῶ \spation{ϲου τἀν πρυτανείῳ ϲιτία}. Pax 1212 ἀπώλεϲάϲ \spation{μου τὴν τέχνην} καὶ \spation{τὸν βίον}. Aves 139 καλῶϲ γέ \spation{μου τὸν υἱόν} ὦ Στιλβωνίδη οὐκ ἔκυϲαϲ. Lysistr. 409 ὀρχουμένηϲ \spation{μου τῆϲ γυναικὸϲ} ἑϲπέραϲ ἡ βάλανοϲ ἐκπέπτωκεν. Ranae 1006 καί \spation{μου τὰ ϲπλάγχν᾽} ἀγανακτεῖ. Plato Apol. 18 D διττούϲ \spation{μου τοὺϲ κατηγόρουϲ} γεγονέναι. 20 Α εἰ μέν \spation{ϲου τὼ υἱέε} πώλω ἢ μόϲχω ἐγενέϲθην. \mbox{Phaedo} 89 Β καταψήϲαϲ οὖν \spation{μου τὴν κεφαλὴν}. Alcaeus com. Fragm. 29 Kock ἐβίαϲέ \spation{μου τὴν} γυναῖκα. Aeschines 3, 16 ἀφομοιοῖ γάρ \spation{μου τὴν φύϲιν τοῖϲ} Σειρῆϲιν. Theokrit 2, 55 τί \spation{μευ μέλαν} ἐκ χροὸϲ \spation{αἷμα} — πέπωκαϲ. 2, 69 u. s. w. φράζεό \spation{μευ τὸν ἔρωθ᾽} ὅθεν ἵκετο. 5, 4 τόν \spation{μευ τὰν ϲύριγγα} πρόαν κλέψαντα Κομάταν. 5, 19 οὔ \spation{τευ τὰν ϲύριγγα} λαθὼν ἔκλεψε Κομάταϲ. 6, 36 καλὰ δέ \spation{μευ} ἁ μία \spation{κώρα}. 15, 31 τί \spation{μευ τὸ χιτώνιον} ἄρδειϲ. 15, 69 \hypertarget{p364}{\emph{[S. 364]}}\label{p364} δίχα \spation{μευ τὸ θερίϲτριον} ἤδη ἔϲχιϲται. 22, 10 οἱ δέ \spation{ϲφεων κατὰ πρύμναν} ἀείραντεϲ μέγα κῦμα.

Noch entschiedener ist der Einfluss unseres Stellungsgesetzes in den ohnehin auffälligen Beispielen anzuerkennen, wo der vorausgehende pronominale Genetiv vom regierenden Substantivum durch andre Worte getrennt ist. Dies zeigt sich an dem τοι Theokrits 7, 87 ὥϲ \spation{τοι} ἐγὼν ἐνόμευον ἀν᾽ ὤρεα τὰϲ καλὰϲ αἶγαϲ \spation{φωνᾶϲ} εἰϲαΐων, wo Meinekes Bemerkungen zu vergleichen sind. Ferner steht bei Homer an den in diese Klasse gehörigen Stellen der Genetiv regelmässig an zweiter Stelle: Ε 811 ἀλλά \spation{ϲευ} ἢ κάματοϲ πολυᾶϊξ \spation{γυῖα} δέδυκεν ἤ νύ ϲέ που δέοϲ ἴϲχει, wo die Stellung des Pronomens besonders bemerkenswert ist. Ι 355 μόγιϲ δέ \spation{μευ} ἔκφυγεν \spation{ὁρμήν}. Ζ 95 = Ρ 173 νῦν δέ \spation{ϲευ} ὠνοϲάμην πάγχυ \spation{φρέναϲ}. Τ 185 χαίρω \spation{ϲευ} Λαερτιάδη \spation{τὸν μῦθον} ἀκούϲαϲ. Κ 311 θεὰ δέ \spation{μευ} ἔκλυεν \spation{αὐδῆϲ}. Κ~485 οἵ \spation{μευ} φθινύθουϲι φίλον \spation{κῆρ}. (Nur π 92 ἦ μάλα \spation{μευ} καταδάπτετ᾽ ἀκούοντοϲ \spation{φίλον ἦτορ}, wo μευ erst an dritter Stelle steht, bildet eine, übrigens nicht sehr schwer wiegende Ausnahme.) — Und wenn nicht regelmässig, so doch überaus häufig nimmt auch bei den Spätern ein so von seinem Substantiv abgetrennter pronominaler Genetiv die zweite Stelle ein: Theognis 969 πρίν \spation{ϲου} κατὰ πάντα δαῆναι ἤθεα. Herodot 4, 119, 2 καί \spation{ϲφεων} ἐϲχίϲθηϲαν \spation{αἱ γνῶμαι}. Eurip. Helena 898 μή \spation{μου} κατείπῃϲ ϲῷ καϲιγνήτῳ \spation{πόϲιν}. Bacch. 341 δεῦρό \spation{ϲου} στέψω \spation{κάρα}. 615 οὐδέ \spation{ϲου} ϲυνῆψε \spation{χεῖρα}. Fragm. 687, 1 ἐμπλήϲθητί \spation{μου} πιὼν \spation{κελαινὸν αἷμα}. 930 οἴμοι, δράκων \spation{μου} γίγνεται \spation{τὸ ἥμιϲυ}. Aristoph. Eq. 708 ἐξαρπάϲομαί \spation{ϲου} τοῖϲ ὄνυξι \spation{τἄντερα}. Pax 1068 εἴθε \spation{ϲου} εἶναι ὤφελεν, ὦ λαζών, οὑτωϲὶ θερμὸϲ ὁ \spation{πλευμών}. Ran. 573 οἷϲ \spation{μου} κατέφαγεϲ τὰ \spation{φορτία}. Plato Phaedo 117 Β ἕωϲ ἄν \spation{ϲου} βάροϲ ἐν \spation{τοῖϲ ϲκέλεϲι} γένηται. Republ. 1, 327 B καί \spation{μου} ὄπιϲθεν λαβόμενοϲ ὁ παῖϲ \spation{τοῦ ἱματίου}. Parmen. 126 Α καί \spation{μου} λαβόμενοϲ \spation{τῆϲ χειρόϲ}. Demosth. 18, 199 καί \spation{μου} μηδὲ εἷϲ \spation{τὴν ὑπερβολὴν} θαυμάϲῃ. Theokrit 2, 82 ὥϲ \spation{μευ} περὶ \spation{θυμὸϲ} ἰάφθη. Bion 6, 1 εἴ \spation{μευ} καλὰ πέλει \spation{τὰ μελύδρια} [Menand. fr. 498].

Ganz Gleichartiges haben wir bei dem genetivischen οἱ getroffen (s. oben S.~337~f.). Und wie nun dieses auch mitten in der regierenden Wortgruppe, d. h. hinter deren erstem Wort, Stellung nehmen kann, so auch die von uns hier zu besprechenden Formen. Und zwar a) im Anschluss an eine Partikel \hypertarget{p365}{\emph{[S. 365]}}\label{p365} Hipponax Fr. 62 οἱ δέ \spation{μευ} πάντεϲ ὀδόντεϲ ἐντὸϲ ἐν γνάθοιϲ κεκινέαται. Anakreon fr. 81 αἱ δέ \spation{μευ} φρένεϲ ἐκκεκωφέαται. Herodot 3, 102, 19 αἱ γάρ \spation{ϲφι} κάμηλοι ἵππων οὐκ ἔϲϲονέϲ εἰϲιν. 4, 202, 3 τῶν δέ \spation{ϲφι} γυναικῶν τοὺϲ μαζοὺϲ ἀποταμοῦϲα. 9, 50, 7 οἵ τέ \spation{ϲφεων} ὀπέωνεϲ — ἀπεκεκληίατο. Aristoph. Eq. 787 τοῦτό γέ τοί \spation{ϲου} τοὖργον ἀληθῶϲ γενναῖον καὶ φιλόδημον. Theokrit 4, 1 ταὶ δέ \spation{μοι} αἶγεϲ βόϲκονται κατ᾽ ὄροϲ (Vgl. auch die bereits oben S. 359. 360 angeführten Stellen mit μοι Eurip. Or. 482, Aristoph. Ekkles. 913. 1113). b) unmittelbar hinter Artikel oder Präposition Herodot 7, 38, 12 ϲὺ δέ, ὦ βαϲιλεῦ, ἐμὲ ἐϲ τόδε ἡλικίηϲ ἥκοντα οἰκτίραϲ, τῶν \spation{μοι} παίδων παράλυϲον ἕνα τῆϲ ϲτρατιῆϲ. Ganz ebenso kyprisch (Deecke Nr. 26) ὄ \spation{μοι} πόϲιϲ Ὀναϲίτιμοϲ ‘mein Gatte ist Onasitimos’, was Hoffmann Die griechischen Dialekte I 323 als ‘sehr eigentümlich’ bezeichnet, während Meister Die griechischen Dialekte II 139. 140, sich sogar genötigt glaubt, ein neues Wort ὁμοίποϲιϲ ‘Mitgatte’ zu konstruieren\Footnote{2}{Auf Wunsch des Herrn Dr. Meister bemerke ich, dass er auf Grund von Wilamowitz’ Anmerkung zu Eurip. Herakles V. 626 (siehe oben S.~362) schon längst zur richtigen Auffassung dieser Worte gelangt war und vorgehabt hatte seine frühere Erklärung öffentlich zurückzunehmen.}). — Dazu aus den attischen Dichtern Eurip. Medea 144 διά \spation{μου} κεφαλᾶϲ φλὸξ οὐρανία βαίη. Hippolyt 1351 διά \spation{μου} κεφαλᾶϲ ᾄϲϲουϲ᾽ ὀδύναι. Heraclid. 799 εἷϲ \spation{μου} λόγοϲ ϲοι πάντα ϲημανεῖ τάδε. Aristoph. Lysistrate 416 ὦ ϲκυτοτόμε, τῆϲ \spation{μου} γυναικὸϲ τοὺϲ πόδαϲ. Vgl. Theokrit 5, 2 τό \spation{μευ} νάκοϲ ἐχθὲϲ ἔκλεψεν. Ausser am Satzanfang findet sich μου u. s. w. jedenfalls höchst selten so eingeschoben, und für die Stellen, wo es geschieht, wie z.~B. Aristoph. Ran. 485 δείϲαϲα γὰρ εἰϲ τὴν κάτω \spation{μου} κοιλίαν καθείρπυϲεν, dürfen wir voraussetzen, dass die am Satzanfang aufgekommene Einschiebung im Satzinnern nachgeahmt wurde.

Die Stellung der barytonetischen, also ursprünglich enklitischen Pluralformen ἥμων, ἧμιν u. s. w. will ich angesichts der Schwierigkeit sie an den einzelnen Stellen von den echtorthotonischen zu unterscheiden, hier nicht untersuchen (man beachte immerhin IGA. 486 (Milet) [Ἑρ]μηϲιάναξ ἥμεαϲ ἀνέθηκεν [ὁ…], ganz wie sonst μ᾽ ἀνέθηκεν und 482\textsuperscript{a} 5 (Elephan-\hypertarget{p366}{\emph{[S. 366]}}\label{p366}tine) ἔγραφε \spation{δ᾽ ἇμε} Ἄρχων Ἁμοιβίχου); wohl aber möchte ich daran erinnern, dass nach den Nachweisen Krügers, dessen ordnendem Scharfsinn wir ja überhaupt die feineren Gesetze für die Stellung dieser Genetive verdanken, αὐτοῦ, αὐτῆϲ, αὐτῶν in anaphorischer Bedeutung den gleichen Stellungsregeln wie μου unterliegt. Zwar gilt dies nicht für Homer, bei dem sich die anaphorische Bedeutung und die Tonlosigkeit von αὐτοῦ erst anzubahnen beginnt, und der es daher auch an Stellen, wo wir es mit \emph{eius} wiedergeben, weit vom Satzanfang stellt, wie z. B. B 347 ἄνυϲιϲ δ᾽ οὐκ ἔϲϲεται \spation{αὐτῶν}. Ρ~546 δὴ γὰρ νόοϲ ἐτράπετ᾽ \spation{αὐτοῦ}. (η~263 dagegen liegt in der gleichen Wendung ein Nachdruck auf αὐτῆϲ). μ~130 γόνοϲ δ᾽ οὐ γίγνεται \spation{αὐτῶν}, was einen sehr wertvollen indirekten Beweis für unsere Stellungsregel liefert. Wohl aber ist bei den Attikern αὐτοῦ, αὐτῆϲ, αὐτῶν gerade so gern dem regierenden Substantiv vorangestellt wie μου, und dann gerade wie μου häufig dem Satzanfang nahe, z. B. Thycyd. 1, 138, 1 ἐθαύμαϲέ τε \spation{αὐτοῦ} τὴν \spation{διάνοιαν}. 4, 109, 11 καὶ \spation{αὐτῶν τὴν χώραν} ἐμμείναϲ τῷ ϲτρατῷ ἐδῄου. Plato Gorg. 448 Ε ἐγκωμιάζειϲ μὲν \spation{αὐτοῦ τὴν τέχνην}. Und ebenso findet sich αὐτοῦ wie μου seinem Substantiv so vorangestellt, dass es durch ein oder mehrere Wörter davon getrennt ist, und auch da, wie μου, gern an zweiter Stelle z. B. Eurip. Heraclid. 12 ἐπεὶ γὰρ \spation{αὐτῶν} γῆϲ ἀπηλλάχθη \spation{πατήρ}. Wer endlich die von Stein zu 6, 30, 7 aufgeführten herodoteischen Stellen durchmustert, an denen αὐτοῦ zwischen Artikel und Substantiv steht, wird an diesen allen (und ebenso auch 1, 146, 10. 1, 177, 3. 2, 149, 19. 7, 129, 3) αὐτοῦ an zweiter Stelle finden, wobei ich 7, 156, 11 Μεγαρέαϲ τε τοὺϲ ἐν Σικελίῃ, ὡϲ — προϲεχώρηϲαν, τοὺϲ μὲν \spation{αὐτῶν} παχέαϲ — πολιήταϲ ἐποίηϲε mitrechne. Also ganz wie bei eingeschobnem μοι, μου. Die Attiker sind hier freier: Isokr. 18, 52 γνώϲεϲθε τὴν ἄλλην \spation{αὐτοῦ} πονηρίαν. Xenoph. Anab. 6, 2, 14 ὅπωϲ — αὐτοὶ καὶ οἱ \spation{αὐτῶν} ϲτρατιῶται ἐκπλεύϲειαν. Vielleicht kommt für das αὐτοῦ bei Isokrates wie für das μου Aristoph. Ran. 485 (oben S.~365) in Betracht, dass der Genetiv sich nicht an den Artikel, sondern an ein Attribut anlehnt.

\section*{V.}
\addcontentsline{toc}{section}{V.}

Bergaigne nimmt an, das in Abschnitt II—IV erörterte Stellungsgesetz der enklitischen Personalpronomina sei bei den \hypertarget{p367}{\emph{[S. 367]}}\label{p367} anaphorischen Pronomina entstanden; diese habe man gern dem vorausgehenden Satze möglichst nahe gerückt, um dadurch die Verbindung mit diesem besser zu markieren. Von den anaphorischen Pronomina sei dann die Stellungsregel auch auf die Pronomina der ersten und zweiten Person übergegangen, und durch diese ihre Stellung nach dem ersten Wort des Satzes und ihre Anlehnung an dasselbe seien die betr. Pronomina enklitisch geworden (Mémoires de la Société de Linguistique III 177. 178).

Diese Annahme hat wenig für sich. Denn gerade was bei οἱ, ϲφιν nach Bergaigne die Stellung nächst dem Satzanfang begünstigte, die Beziehung auf den vorausgehenden Satz, fehlt ja bei μοι, τοι. Dagegen wird die von Bergaigne verworfene Möglichkeit, dass “le langage s’est habitué à les construire après le premier mot, parce qu’ils étaient privés d’accent”, als Thatsache durch den Umstand erwiesen, dass auch ausserhalb des persönlichen Pronomens die Enklitika dieser Stellungsregel unterworfen werden. Schon Kühner Griechische Grammatik~I\textsuperscript{2} 268 Anm. 8 bemerkt, “bei der freien Wortstellung der griechischen Sprache darf man sich nicht wundern, wenn die Encliticae sich oftmals nicht an das Wort anschliessen, zu dem sie gehören, sondern an ein anderes, zu dem sie nicht gehören”. In welcher Richtung diese Abweichungen liegen, lässt Kühner unerörtert. Aber sämtliche Beispiele, die er a. a. O. folgen lässt, erledigen sich aus unserm Stellungsgesetz.

Unter den deklinabeln Enklitika kommt bloss noch das indefinite Pronomen in betracht. Sehr evident tritt bei diesem die Stellungsregel nicht zu Tage. Denn wenn man etwa darauf Gewicht legen wollte, dass die altertümlichen Formen του, τῳ auf den attischen Inschriften ausser CIA.~4, 61\textsuperscript{a} 15 — ἔχοντόϲ του, nur im unmittelbaren Anschluss an εἰ, ἐάν vorkommen (vgl. die Belege bei Meisterhans Grammatik der attischen Inschriften\textsuperscript{2} S.~123 Anm.~1106), so genügt es auf Thucydides zu verweisen, der diese Formen an ganz beliebigen Stellen des Satzes bietet. Doch ist bei Homer die Neigung τὶϲ an den Anfang zu rücken unverkennbar. Man beachte, ausser ὅϲτιϲ nebst Zubehör, εἴ τιϲ, μή τιϲ, besonders folgende Stellen: mit Losreissung zum gehörigen Nomen Ε~897 εἰ δέ \spation{τευ} ἐξ \spation{ἄλλου} γε θεῶν. Θ~515 ἵνα \spation{τιϲ} ϲτυγέῃϲι καὶ \spation{ἄλλοϲ}. Ν~464 \hypertarget{p368}{\emph{[S. 368]}}\label{p368} εἴ πέρ \spation{τί} ϲε \spation{κῆδοϲ} ἱκάνει (zugleich vor dem enklitischen ϲε!). Ψ~331 ἤ \spation{τευ} ϲῆμα \spation{βροτοῖο} πάλαι κατατεθνηῶτοϲ. γ~348 (— ὡϲ ὑμεῖϲ παρ᾽ ἐμεῖο θοὴν ἐπὶ νῆα κίοιτε) ὥϲ τέ \spation{τευ} ἢ παρὰ πάμπαν \spation{ἀνείμονοϲ} ἠὲ \spation{πενιχροῦ}. η~195 μηδέ τι μεϲϲηγύϲ γε \spation{κακὸν} καὶ πῆμα πάθῃϲιν. Mit Voranstellung von τιϲ vor ein sonst zur zweiten Stelle berechtigtes Wort (vgl. Ν~464) Π~37 καί \spation{τινά} τοι παρ [sic] Ζηνὸϲ ἐπέφραδε πότνια μήτηρ. λ~218 ὅτε \spation{τίϲ} κε θάνῃϲι (vgl. Hesiod Ἔργα~280 εἰ γάρ \spation{τίϲ} κ᾽ ἐθέλῃ. Peppmüller Berliner philolog. Wochenschrift 1890 Sp.~559). Hierher gehört das nicht seltene ὥϲ τίϲ τε statt ὥϲτε τιϲ  wie z.~B. Ρ~657 βῆ δ᾽ ἰέναι ὥϲ \spation{τίϲ} τε \spation{λέων} ἀπὸ μεϲϲαύλοιο.

Beispiele der ersten Kategorie lassen sich auch aus der Folgezeit beibringen (Kühner Gramm. II 572 Anm. 6): Theognis 833 οὐδέ \spation{τιϲ} ἡμῖν αἴτιοϲ \spation{ἀθανάτων}. 957 εἴ \spation{τι} παθὼν ἀπ᾽ ἐμεῦ \spation{ἀγαθὸν μέγα} μὴ χάριν οἶδαϲ. 1192 ἀλλά τί \spation{μοι} ζῶντι γένοιτ᾽ \spation{ἀγαθόν}. 1265 οὐδέ \spation{τιϲ} ἀντ᾽ ἀγαθῶν ἐϲτι \spation{χάριϲ} παρὰ ϲοί. Aeschyl. Fragm. 241 οὔπω \spation{τιϲ} Ἀκταίων᾽ \spation{ἄθηροϲ} ἡμέρα — ἔπεμψεν ἐϲ δόμουϲ. Herodot 2, 23, 3 οὐ γάρ \spation{τινα} ἔγωγε οἶδα \spation{ποταμὸν Ὠκεανὸν} ἐόντα. 7, 235, 9 αἰεί \spation{τι} προϲδοκῶν ἀπ᾽ αὐτῆϲ τοιοῦτο ἔϲεϲθαι. Eurip. Medea 283 μή μοί \spation{τι} δράϲῃϲ παῖδ᾽ \spation{ἀνήκεϲτον κακόν}. Elektra 26 μή \spation{τῳ} λαθραίωϲ τέκνα \spation{γενναίῳ} τέκοι. Helena 477 ἔϲτι γάρ \spation{τιϲ} ἐν δόμοιϲ \spation{τύχη}. Thucyd. 1, 10, 1 εἴ \spation{τι} τῶν τότε \spation{πόλιϲμα}. Aristoph. Pax 834 καί \spation{τίϲ} ἐϲτιν \spation{ἀϲτήρ}. Ran. 170 καὶ γάρ \spation{τιν᾽} ἐκφέρουϲι \spation{τουτονὶ νεκρόν}. Plato Phaedo 95~Β μή \spation{τιϲ} ἡμῖν \spation{βαϲκανία} περιτρέψῃ τὸν λόγον. 101 Α μή \spation{τίϲ} ϲοι \spation{ἐναντίοϲ λόγοϲ} ἀπαντήϲῃ. Sympos. 174 Ε καί \spation{τι} ἔφη αὐτόθι \spation{γελοῖον} παθεῖν. 218 Ε καί \spation{τίϲ} ἐϲτ᾽ ἐν ἐμοὶ \spation{δύναμιϲ}. Gorg. 493 Α ἤδη \spation{του} ἔγωγε καὶ ἤκουϲα \spation{τῶν ϲοφῶν}. Xenophon Hellen. 4, 1, 11 ὅταν \spation{τι} τοῖϲ φίλοιϲ \spation{ἀγαθὸν} εὑρίϲκω. 4, 8, 33 εἴ \spation{τί} που λαμβάνοι Ἀθηναίων \spation{πλοῖον}. Demosth. 18, 18 ἀλλά \spation{τιϲ} ἦν \spation{ἄκριτοϲ} καὶ παρὰ τούτοιϲ καὶ παρὰ τοῖϲ ἄλλοιϲ \spation{ἔριϲ}. 18, 65 ἦν ἄν \spation{τιϲ} κατὰ τῶν ἐναντιωθέντων οἷϲ ἔπραττεν ἐκεῖνοϲ, \spation{μέμψιϲ καὶ κατηγορία}. Menander Fragm. 572 Kock ὅταν \spation{τι} πράττῃϲ \spation{ὅϲιον}. Fragm. lyr. adesp. 58 Bgk. (3\textsuperscript{4}, 706) ἀλλά \spation{τιϲ} ἄμμι \spation{δαίμων}. Dazu Plato Leges 3, 683 Β εἰ γοῦν, ὦ ξένε, \spation{τιϲ} ἡμῖν ὑπόϲχοιτο \spation{θεόϲ}, wo zugleich auch noch die Anlehnung von τὶϲ an den Vokativ Beachtung verdient, vgl. das oben S.~343 über Πάτροκλέ μοι bemerkte. Aus Nachahmung derartiger Stellen ist dann die Wortfolge von Stellen wie Thucyd. 1, 106, 1 \hypertarget{p369}{\emph{[S. 369]}}\label{p369} καὶ αὐτῶν μέροϲ — ἐϲέπεϲεν ἔϲ \spation{του} χωρίον \spation{ἰδιώτου} zu erklären, wo mitten im Satze stehendes τὶϲ von dem später nachfolgenden Satzteil durch andere Wörter getrennt ist.

Und wie das homerische, drängt auch das nachhomerische τὶϲ andere Wörter von der ihnen zukommenden zweiten Stelle weg. Aus der attischen Litteratur gehört bloss etwa die Tmesis Aristoph. Vesp. 437 ἔν \spation{τί} ϲοι παγήϲεται und Stellen wie Plato Gorg. 520 E ὅντιν᾽ ἄν \spation{τιϲ} τρόπον ὡϲ βέλτιϲτοϲ εἴη hierher. Aber die Wortfolge τίϲ κε hinter dem Einleitungswort eines Konjunktivsatzes, welche die epische Sprache (abgesehen vom geimeinüblichen ὅϲτιϲ κε) nur in Einem homerischen und Einem hesiodischen Beispiel kennt, ist im Dorischen (natürlich mit κα statt κε) geradezu die Regel. (Vgl. Ahrens Dial. II 383). So im gortynischen Gesetz: 9, 43 αἴ τιϲ [sic] κα. 7, 13 αἴ τινά κα. 3, 29 (ebenso 6, 23. 6, 43. 9, 13) καἴ τί κ᾽. 8, 17 καί μέν τίϲ κ᾽. 3, 9 ὅτι δέ τίϲ κα. Abweichend 5, 13 = 17 = 22 αἰ δέ κα μή τιϲ und 4, 14 ᾧ δέ κα μή τιϲ ᾖ ϲτέγα, wo μή das Indefinitivum attrahiert hat, sowie ὁπῶ κά τιλ λῇ 10, 33. — Auf jüngern kretischen Inschriften CIG. 3048 (= Cauer\textsuperscript{2} 123), 33 εἰ δέ \spation{τινέϲ} κα τῶν ὁρμιωμένων (ebenso 3049, 9. 3058, 13). 3048, 38 εἴ \spation{τίϲ} κα ἄγῃ (ebenso 3049, 14. 3058, 16). — Auf den Tafeln von Heraklea 1, 105 καὶ αἴ \spation{τινί} κα ἄλλῳ. 1, 117 καὶ αἴ \spation{τινάϲ} κα ἄλλουϲ. 1, 119 αἰ δέ \spation{τινά} κα γήρᾳ — ἐκπέτωντι. 1, 127 καὶ εἴ \spation{τινέϲ} κα μὴ πεφυτεύκωντι. 1, 128 αἰ δέ \spation{τίϲ} κα ἐπιβῇ. 1, 151 αἰ δέ \spation{τιϲ} [sic] κα τῶν καρπιζομένων ἀποθάνει. 1, 173 αἴ \spation{τινά} κα γήρᾳ — ἐκπέτωντι. — Auf der Inschrift v. Orchomenos Dittenberger Syll. 178, 10 καὶ εἴ \spation{τίϲ} κα μὴ ἐμμένῃ. — Auf der Inschrift von Mykene Collitz 3316, 8 αἰ δέ \spation{τί} κα πένηται. — Auf den korkyräischen Inschriften Coll. 3206, 25 εἰ δέ \spation{τί} κ᾽ ἀδύνατον γένοιτο. 3206, 103 εἰ δέ \spation{τί} κα — μὴ ὀρθῶϲ ἀπολογίξωνται. 3206, 114 εἴ \spation{τινόϲ} κα ἄλλου δοκῆ. Dazu vielleicht Theokrit 2, 159 αἰ δέ \spation{τί} κά με — λυπῇ. (Siehe unten S.~372).

Angesichts so konstanten Gebrauchs, dem ich, abgesehen von den gortynischen Ausnahmen, wo teils μή im Spiele ist, teils nicht εἰ vorhergeht, nur Epi\-charm S.~217 Lor. (Athen.~6, 236 A) Z.~5 καἴ κά \spation{τιϲ} ἀντίον <τι> λῇ τήνῳ λέγειν und S.~281 Lor. (Athen.~2, 70 F) αἴ κά \spation{τιϲ} ἐκτρίψαϲ καλῶϲ παρατιθῇ νιν als Gegenbeispiele entgegenstellen kann, scheint es mir klar, dass auf der korkyräischen Inschrift 3213 Collitz (= CIG. \hypertarget{p370}{\emph{[S. 370]}}\label{p370} 1850), 3 das überlieferte αἴ κα πάϲχη nicht mit Boeckh in αἴ κά <τι> πάϲχη zu verbessern ist, sondern vielmehr in αἴ <τί> κα πάϲχη. Übrigens ist diese Stellungsgewohnheit nicht bloss dorisch: Tafel von Idalion, Z.~29 ὄπι ϲίϲ κε τὰϲ ϝρήταϲ τάϲδε λύϲη. — Vgl. ferner Sophron bei Athen. 3, 110 D ἄρτον γάρ \spation{τιϲ} τυρῶντα τοῖϲ παιδίοιϲ ἴαλε, mit Trennung von ἄρτον τυρῶντα.

Endlich kann man die Frage aufwerfen, ob nicht die von Herodot an den Prosaisten geläufige Zwischenschiebung von τὶϲ zwischen den Artikel nebst eventuellem Attribut und
das Substantiv des zugehörigen Genetivus partitivus (z.~B. τῶν τινα Λυδῶν, ἐϲ τῶν ἐκείνων τι χωρίων, τῶν ἄλλων τινὰϲ Ἑλλήνων) in Sätzen aufgekommen sei, wo τιϲ dadurch an zweite Stelle kam.

Die vom Indefinitum abgeleiteten Adverbia befolgen bei Homer unser Gesetz ziemlich streng. In ΝΠΡ findet sich που 14 mal, immer an zweiter Stelle, darunter beachtenswert Ν~293 μή \spation{πού} τιϲ ὑπερφιάλωϲ νεμεϲήϲῃ mit Trennung von μή und τιϲ und Ν~225 ἀλλά \spation{που}. — \spation{ποθι} zweimal, Ν~630 ἀλλά \spation{ποθι}, Ν~309 ἐπὶ οὔ \spation{ποθι} ἔλπομαι, wo noch οὐ vorhergeht. — \spation{πωϲ} neunmal, siebenmal an zweiter Stelle, dazu ἀλλ᾽ οὔ πωϲ Ν~729. Ρ~354 — \spation{ποτε} viermal, zweimal an zweiter Stelle, daneben Ν~776 ἄλλοτε δή \spation{ποτε} μᾶλλον ἐρωῆϲαι πολέμοιο μέλλω. Π~236 ἠμὲν δή \spation{ποτ᾽} ἐμὸν ἔποϲ ἔκλυεϲ εὐξαμένοιο. — \spation{πῇ} nur einmal (Π~110), korrekt. — \spation{πω} fünfmal korrekt, dazu Ρ~190 θέων δ᾽ ἐκίχανεν ἑταίρουϲ ὦκα μάλ᾽, οὔ \spation{πω} τῆλε, ποϲὶ κραιπνοῖϲι μεταϲπών. Ρ~377 δύο δ᾽ οὔ \spation{πω} φῶτε πεπύϲθην. [Ausnahmen aus den andern Büchern verzeichnet Monro\textsuperscript{2} S.~336~ff.]

Die nachhomerische Zeit verfährt bei diesen Partikeln recht frei. Reste des Alten liegen ausser in ἦπου, δήπου, vor in Stellen wie Theokrit 18, 1 ἔν \spation{ποκ᾽} ἄρα Σπάρτᾳ —. Antipater Anthol. Pal. 6, 219, 1 ἔκ \spation{ποτέ} τιϲ φρικτοῖο θεᾶϲ ϲεϲοβημένοϲ οἴϲτρῳ. (Nach solchen Mustern dann Pind. Pyth. 2, 33 ὅτι τε μεγαλοκευθέεϲιν ἔν \spation{ποτε} θαλάμοιϲ. Leonidas Anthol. Pal. 9, 9 Ἴξαλοϲ εὐπώγων αἰγὸϲ πόϲιϲ ἔν \spation{ποθ᾽} ἁλωῇ). Vgl. auch Plato Phaedo 73~D ἄλλη \spation{που} ἐπιϲτήμη ἀνθρώπου καὶ λύραϲ. 101~Β ὁ αὐτὸϲ γάρ \spation{που} φόβοϲ.

Viel ergebnisreicher ist die Betrachtung sonstiger enklitischer Partikeln. Zwar wenn τε und ῥα stets an zweiter Stelle stehen (Β~310 βωμοῦ ὑπαΐξαϲ πρόϲ \spation{ῥα} πλατάνιϲτον ὄρου-\hypertarget{p371}{\emph{[S. 371]}}\label{p371}ϲεν ist das Partizip einem Nebensatz gleichwertig), könnte man dies aus ihrer Funktion die Sätze zu verbinden erklären. Andererseits entzieht sich γε jeder durchgreifenden Stellungsregel, weil es an das Wort gebannt ist, auf dessen Begriff das Hauptgewicht der Bejahung fällt; höchstens könnte man darauf hinweisen, dass bei Thucydides mehrmals ein zu einem Partizip gehöriges γε nicht an dieses, sondern an ein früheres Wort angeschlossen ist (Stahl zu Thucyd. 2, 38, 1): 2, 38, 1 ἀγῶϲι μέν γε καὶ θυϲίαιϲ διετηϲίοιϲ νομίζοντεϲ. 4, 65, 4 οὕτω τῇ γε παρούϲῃ εὐτυχίᾳ χρώμενοι. 4, 86, 2 πίϲτειϲ γε διδοὺϲ τὰϲ μεγίϲταϲ. Vgl. Demosth. 18, 226 ὥϲ γ᾽ ἐμοὶ δοκεῖ statt ὡϲ ἔμοιγε δοκεῖ. — Ähnliches wie für γε, gilt für περ.

Aber Eine konstant enklitische Partikel kann doch genannt werden, die, obwohl durchaus nicht der Satzverbindung dienend, doch ganz unverkennbar Vorliebe für die zweite Stelle hat, nämlich \spation{κε} (κεν, κα). Schon G. Hermann De particula ἄν (Opuscula IV) S.~7 deutet dies mit den Worten an: “κεν, quae quod enclitica est ab incipienda oratione arcetur, etiam ante ea verba, ad quorum sententiam pertinet, poni potest, dummodo aliqua vox in eadem constructione verborum praecesserit”, und bringt als Beispiel Η~125 ἦ \spation{κε} μέγ᾽ οἰμώξειε γέρων ἱππηλάτα Πηλεύϲ. Doch denkt Hermann nicht daran, geradewegs der Partikel die zweite Stelle im Satz zu vindizieren. Und selbst der neueste Gesamtdarsteller des homerischen Gebrauchs von κε, Ε. Eberhard in Ebelings Lexikon, behandelt dessen Stellung zwar auf fast sieben eng gedruckten Spalten, aber ohne prinzipiell über Hermann hinauszukommen, so sehr das von ihm selbst zusammengebrachte Material ihn hätte auf die richtige Bahn bringen müssen. So wenn er im Anschluss an Schnorr hervorhebt, dass κε dem Verb nur dann folge, wenn dieses an der Spitze des Satzes stehe, und dem Partizip nur ψ~47 ἰδοῦϲά \spation{κε} θυμὸν ἰάνθηϲ, oder dass sich die und die Verbindung von κε mit einem vorausgehenden Wort nur “in introitu versus” finde.

Allgemein anerkannt ist vorerst, dass in allen griechischen Mundarten, die κε oder eine Nebenform desselben überhaupt besitzen, die Partikel dem einleitenden Pronomen oder Fügewort konjunktivischer Nebensätze ausnahmslos unmittelbar folgt, es sei denn, dass sich sonstige Enklitika oder Quasi-Enklitika, wie τε, δέ, γάρ, μέν, vereinzelt auch τὶϲ (siehe oben \hypertarget{p372}{\emph{[S. 372]}}\label{p372} S.~369), τὺ (siehe oben S.~344) und τοὶ (Theognis 633 ὅ \spation{τοί} κ᾽ ἐπὶ τὸν νόον ἔλθῃ) dazwischen drängen: ὅϲ κε, εἰϲ ὅ κε, εἴ κε, αἴ κε, ἐπείκε, ὅτε κε (dor. ὅκκα), ἕωϲ κε, ὄφρα κε, ὥϲ κε, ὅ(π)πωϲ κε oder ὃϲ δέ κε, εἰ δέ κε u. dergl. (Doch Epicharm S.~225 Lor. [Athen. 6, 236~A] Z.~10 \spation{αἴκα} δ᾽ ἐντύχω τοῖϲ περιπόλοιϲ und Theokrit 1, 5 \spation{αἴκα} δ᾽ αἶγα λάβῃ τῆνοϲ γέραϲ neben 1, 10 αἰ δέ κ᾽ ἀρέϲκῃ u. s. w.). Undenkbar scheint mir die von Ahrens für Theokrit 1, 159 vorgeschlagene, von Meineke und Hiller akzeptierte Schreibung αἰ δ᾽ ἔτι κά με — λυπῇ, so dass αἰ von κα durch ἔτι getrennt wäre. Der Zusammenhang hindert nicht das grammatisch einzig zulässige αἰ δέ τί κά με einzusetzen und diese Stelle den oben S.~369 aufgeführten mit τίϲ zwischen αἰ und κα einzureihen. (Gottfried Hermann εἰ δ᾽ ἔτι καί με — λυπεῖ, was weniger anspricht.)

Ganz Entsprechendes zeigen nun aber die andern Satzarten. Auch die Hauptsätze und interrogativen Nebensätze mit konjunktivischem Verb haben bei Homer κε ausnahmslos an zweiter Stelle, so in ΝΠΡ an folgenden Stellen: Π~129 ἐγὼ δέ \spation{κε} λαὸν ἀγείρω. Ν~742 (ἐπιφραϲϲαίμεθα βουλήν) ἤ \spation{κεν} ἐνὶ νήεϲϲι πολυκλήιϲι πέϲωμεν — ἤ \spation{κεν} ἔπειτα παρ [sic] νηῶν ἔλθωμεν. Ρ~506 ἤ κ᾽ αὐτὸϲ ἐνὶ πρώτοιϲιν ἁλώῃ. Ebenso die Futursätze: Ρ~241 ὥϲ \spation{κε} τάχα Τρώων κορέει κύναϲ ἠδ᾽ οἰωνούϲ. Ρ~557 εἴ κ᾽ Ἀχιλῆοϲ ἀγαυοῦ πιϲτὸν ἑταῖρον τείχει ὕπο Τρώων ταχέεϲ κύνεϲ ἑλκήϲουϲιν. Ρ~515 τὰ δέ \spation{κεν} Διὶ πάντα μελήϲει. (So auch sonst, und zwar auch auf die Gefahr hin Zusammengehöriges zu trennen: Γ~138 τῷ δέ \spation{κε} νικήϲαντι φίλη κεκλήϲῃ ἄκοιτιϲ). Nicht anders ist der Gebrauch beim Optativ und beim Präteritum. In ΝΠΡ haben wir κε 28 mal an zweiter oder so gut wie zweiter Stelle optativischer Sätze (mit Einschluss von Ν~127 ἃϲ οὔτ᾽ ἄν \spation{κεν} Ἄρηϲ ὀνόϲαιτο μετελθών οὔτε κ᾽ Ἀθηναίη und von Ρ~629 ὢ πόποι, ἤδη μέν \spation{κε} — γνοίη) und 7 mal an zweiter Stelle präteritaler Sätze. Diesen 35 Beispielen, worunter ἀλλά κεν Ν~290 [und dreimal in der Odyssee] und καί κεν Ν~377. Ρ~613 [und sonst noch oft, s. Ebeling~II 733] (vgl. καί μοι), ferner Ν~321 ἀνδρὶ δέ κ᾽ οὐκ εἴξειε μέγαϲ Τελαμώνιοϲ Αἴαϲ mit seiner Voranstellung von κε vor die Negation besonders bemerkenswert sind, steht nur Ein Gegenbeispiel gegenüber: Ρ~260 τῶν δ᾽ ἄλλων τίϲ \spation{κεν} ᾗϲι φρεϲὶν οὐνόματ᾽ εἴποι, wo die Entfernung des fragenden τίϲ von der ihm zukommenden Stelle am Satzanfang auch für κε, \hypertarget{p373}{\emph{[S. 373]}}\label{p373} das dem τίϲ nicht vorangehen durfte, eine Verschiebung nach sich gezogen hat.

Halten wir bei Homer weitere Umschau, so können wir namentlich konstatieren, dass die für die konjunktivischen Nebensätze anerkannte Regel, dass sich κε an das satzeinleitende Wort unmittelbar anschliessen soll, gerade so auch für die optativischen und indikativischen gilt, und ὅϲ κε, οἷοϲ κε, ὅθεν κε, ὅτε κε, εἰϲ ὅ κε, ἕωϲ κε, ὄφρα κε, ὥϲ κε, εἴ κε, αἵ κε bei ihnen gerade so eng zusammenhängen, wie bei den konjunktivischen. Der Ausnahmen für diese wie für die sonstigen κε-Sätze sind verschwindend wenige: Ψ~592 εἰ καί νύ \spation{κεν} οἴκοθεν ἄλλο μεῖζον ἐπαιτήϲειαϲ, wo eben εἰ καί eine ähnliche Einheit bildet wie εἴπερ; vgl. Ν~58 εἰ καί \spation{μιν}. Sodann, wiederum wie bei \spation{μιν}, mehrere Beispiele mit οὐ: Ξ~91 μῦθον ὃν οὔ \spation{κεν} ἀνήρ γε διὰ ϲτόμα πάμπαν ἄγοιτο. α~236 ἐπεὶ οὔ \spation{κε} θανόντι περ ὧδ᾽ ἀκαχοίμην. δ~64 ἐπεὶ οὔ \spation{κε} κακοὶ τοιούϲδε τέκοιεν. θ~280 τά γ᾽ οὔ \spation{κέ} τιϲ οὐδὲ ἴδοιτο, und vielleicht noch einige andere. Dann Α~256 ἄλλοι τε Τρῶεϲ μέγα \spation{κεν} κεχαροίατο θυμῷ. Eine viel seltsamere Ausnahme wäre, zumal da εἴ κε sonst immer zusammenbleibt, Ε~273 = Θ~196 εἰ τούτω \spation{κε} λάβοιμεν, ἀροίμεθά κεν κλέοϲ ἐϲθλόν. Aber schon zahlreiche Herausgeber, zuletzt auch Nauck, haben hier das sinngemässe γε eingesetzt. Um so auffälliger ist Naucks Schreibung γ~319 ὅθεν οὐκ ἔλποιτό \spation{κε} θυμῷ, ἐλθέμεν gegenüber dem γε aller Handschriften.

Auf den inschriftlichen Denkmälern der Dialekte, welche κε, κα anwenden, kommt diese Partikel ausserhalb der bereits besprochenen konjunktivischen Nebensätze nur selten vor, was durch den Inhalt der meisten derselben bedingt ist. Aeolisch haben wir ein paar mal ὤϲ κε c. optat, kyprisch das sehr bemerkenswerte τάϲ κε ζᾶϲ τάϲδε — ἔξο(ν)ϲι αἰϝεί, also κε an zweiter Stelle zwischen Artikel und Substantiv bei futurischem Verbum (Tafel von Idalion Z.~30; vgl. Hoffmann Griech. Dialekte I 70. 73, der gegenüber dem früher gelesenen γε das Richtige erkannt hat), argivisch (Collitz 3277, 8) ἇι κα δικάϲϲαιεν, korkyräisch (Collitz 3206, 84) ἀφ᾽ οὗ κ᾽ ἀρχ(ὰ) γένοιτο, epidaurisch in der grossen Heilungsinschrift (3339 Collitz) auf Z.~60 \spation{αἴ} \spation{κα} ὑγιῆ νιν ποιήϲαι, aber Z.~84 τοῦτον γὰρ οὐδέ \spation{κα} ὁ ἐν Ἐπιδαύρωι Ἀϲκλαπιὸϲ ὑγιῆ ποιῆϲαι δύναιτο, sowie bei Isyllos (3342 Collitz) neben (Z.~26) οὕτω τοί κ᾽ ἀμῶν περιφεί-\hypertarget{p374}{\emph{[S. 374]}}\label{p374}δοιτ᾽ εὐρύοπα Ζεύϲ im Vers, Z.~35~f. in Prosa ἢ λώιον οἷ \spation{κα} εἴη ἀγγράφοντι τὸν παιᾶνα. Ἐμάντευϲε λώιόν οἵ \spation{κα} εἶμεν ἀγγράφοντι.

Ein bischen [sic] reicher an Beispielen für κα sind bloss die dodonäischen und die eleischen Inschriften. Und nun beachte man, dass sämtliche mit τίνι θεῶν θύοντεϲ und Ähnlichem anfangenden und auf ein optativisches Verb ausgehenden Befragungen des dodonäischen Orakels, wenn sie κα haben, dieses unmittelbar hinter τίνι setzen und mit demselben also τίνι von dem nächst zugehörigen Genetiv trennen, ein deutlicher Beweis für den Drang von κα nach der zweiten Stelle: Collitz 1562, 1563, 1566, 1582\textsuperscript{a}, 1582\textsuperscript{b}, z.~B. (1563) τίνι \spation{κα} θεῶν [ἢ] ἡρώων θύοντεϲ καὶ εὐχ[ό](μ)ενο(ι) ὁμονοοῖεν ἐ[π]ὶ τὠγαθόν. — Ähnlich 1572\textsuperscript{a} τί \spation{κα} θύϲαϲ —.

Wenn Blass in der Inschrift 3184 Coll. (= 1564 Coll.) τίναϲ θεῶν ἱλαϲκόμενοϲ λώιον καὶ ἄμεινον πράϲϲοι, die Partikel κα, die allerdings hinter τίναϲ sicher nicht gestanden hat, an einem Zeilenende hinter λώιον einschieben will, weil sie unerlässlich sei, so übersieht er, dass die dodonäischen Inschriften den Optativ ohne κα mehrmals potenzial verwenden, z.~B. 1562~B τίνι θεῶν θύουϲα λώιον καὶ ἄμεινον πράϲϲοι καὶ τᾶϲ νόϲου παύϲαιτο. 1583, 2 ἦ μὴ ν[α](υ)κλαρῆ(ν) λώιογ καὶ ἄμεινομ [sic] πράϲϲοιμι. 1587\textsuperscript{a} τίνα θεῶν ἢ ἡρώων τιμᾶντι λώιον καὶ ἄμεινον εἴη. — Ausserhalb jener festen mit τίϲ beginnenden Formel ist allerdings auf diesen Inschriften die Stellung von κα eine freie: 1568, 1 ἦ τυγχάνοιμί \spation{κα}. 1573 — βέλτιομ μοί κ᾽ εἴη.

Bei den eleischen Inschriften müssen zunächst 1151, 12. 1154, 7. 1157, 4. 1158, 2 ausser Rechnung fallen, weil hier κα zwar überliefert, aber seine Stellung im Satz nicht erkennbar ist; ebenso alle Beispiele mit ergänztem κα, ausser 1151, 19, wo die Stelle des zu ergänzenden κα wenigstens negativ festgestellt werden kann. Es bleiben so 28 Beispiele: 21 bieten κα an zweiter oder so gut wie zweiter Stelle, wobei ich 1149, 9 ἐν τἠπιάροι κ᾽ ἐνέχοιτο und 1152, 7 ἐν ταῖ ζεκαμναίαι κ᾽ ἐνέχοιτο mit einrechne; diesen 21 stehen bloss 7 Gegenbeispiele gegenüber. Das Gewicht dieser Zahlen wird verstärkt durch die Beschaffenheit folgender Stellen: 1154, 1 τοὶ ζέ \spation{κα} θεοκόλοι. 1154, 3 πεντακατίαϲ \spation{κα} δαρχμάϲ. 1156, 2 ἀ δέ \spation{κα} ϝράτρα. 1156, 3 τῶν δέ \spation{κα} γραφέων. 1158, 1 ὀ δέ \spation{κα} ξένοϲ, \hypertarget{p375}{\emph{[S. 375]}}\label{p375} an welchen allen κα den Artikel oder ein Attribut von seinem Substantiv trennt. Dazu kommt 1157, 7 τῶν ζὲ προϲτιζίων οὐζέ \spation{κα} μί᾽ εἴη, wo κα zwar nicht an zweiter Stelle steht, aber die Tmesis doch ein Drängen der Partikel nach dem Satzanfang verrät.

Für die nachhomerischen Dichter darf man trotz der Spärlichkeit der Belege Geltung der Regel bis an den Schluss des sechsten Jahrhunderts behaupten. Die Fragmente der vorpindarischen Meliker, wie die der Elegiker vor Theognis bieten κε, κα nur an zweiter Stelle (siehe bes. auch Xenophanes 2, 10 ταῦτά χ᾽ ἅπαντα λάχοι). Sappho Fragm. 66 ὀ δ᾽ Ἄρευϲ φαῖϲί κεν Ἄφαιϲτον ἄγην ist schlecht überliefert, und Alcaeus 83 schreibt zwar Bergk: αἴ κ᾽ εἴπῃϲ, τὰ θέλειϲ, <αὐτὸϲ> ἀκούϲαιϲ <κε>, τά κ᾽ οὐ θέλοιϲ. Aber weder αὐτόϲ noch κε ist überliefert. Man wird jetzt andre Wege der Besserung versuchen müssen. Dann freilich die theognideische Spruchsammlung, Pindar und Epicharm gehn von der alten Norm ab: Theognis (neben Stellen wie 900 μέγα \spation{κεν} πῆμα βροτοῖϲιν ἐπῆν) 645, 653, 747, 765; Pindar öfters; Epicharm (gegenüber normalem Gebrauch S.~223, Busiris Fragm.~1; S.~264, Fragm.~33, 1 und S.~267 Vs.~12) S.~257, Fragm.~7, 1. S.~267, Vs.~9. S.~268, Vs.~16. S.~269, Vs.~11. S.~274, Fragm.~53; Vs.~167 Mullach: wobei man die Frage nach der Echtheit der einzelnen Stellen wohl auf sich beruhen lassen kann.

Von den noch übrigen enklitischen Partikeln θην, νυ, τοι steht \spation{θήν} [sic] bei Homer immer an zweiter Stelle (natürlich mit Einrechnung von Φ~568 καὶ γάρ \spation{θην} und Θ~448 οὐ μέν \spation{θην}); ebenso Aeschylus Prom. 928 ϲύ \spation{θην} ἃ χρῄζειϲ, ταῦτ᾽ ἐπιγλωϲϲᾷ Διόϲ; ebenso bei Theokrit in den ererbten Verbindungen τύ \spation{θην} 1, 97. 7, 83 (vgl. Aeschylus a.~a.~O.) und καὶ γάρ \spation{θην} 6, 34 (vgl. Φ~568), daneben noch in αἶνόϲ \spation{θην} 14, 43 und πείρᾳ \spation{θην} 15, 62. Zweimal (2, 114. 5, 111) hat Theokrit die Regel verletzt. Vor ihm schon Epicharm Ἐλπίϲ S.~226 Lor., Vs. 2 καίτοι νῦν γά \spation{θην} εὔωνον αἰνεῖ ϲῖτον.

\spation{νυ}, \spation{νυν} stehen bei Homer so gut wie immer an zweiter Stelle, zu schliessen aus der Bemerkung bei Ebeling s.~v.: “particula ut est enclitica, ita ad vocem gravissimam quamque se applicat.” Τ~95 καὶ γὰρ δή \spation{νύ} ποτε Ζεὺϲ ἄϲατο rechne ich nicht als Ausnahme. Umgekehrt fällt stark ins Gewicht, \hypertarget{p376}{\emph{[S. 376]}}\label{p376} erstens dass νυ andern Enklitika, wie μοι, τοι, οἱ, ϲε, τιϲ, τι, ποτε, που (doch Κ~105 ὅϲα \spation{πού νυν} ἐέλπεται), περ, κεν regelmässig vorangeht, und nur δέ vor sich hat; dazu νὺ γάρ Ν~257 neben γάρ νυ Ο~239. γὰρ δή νυ Τ~95. Zweitens trennt es öfters enge Verbindungen oder hilft solche trennen: Attribut und Substantiv Θ~104 ἠπεδανὸϲ δέ \spation{νύ} τοι θεράπων. Τ~169 θαρϲαλέον \spation{νύ} τοι ἦτορ ἐνὶ φρεϲίν. Ω~205 = 521 ϲιδήρειόν \spation{νύ} τοι ἦτορ. Artikel und Substantiv Α~382 οἱ δέ \spation{νυ} λαοὶ θνῆϲκον. Χ~405 ἡ δέ \spation{νυ} μήτηρ τίλλε κόμην. Präposition und Substantiv Ι~116 ἀντί \spation{νυ} πολλῶν λαῶν ἐϲτὶν ἀνήρ. Gegen die Regel verstösst, so viel ich sehe, nur α~217 ὡϲ δὴ ἔγωγ᾽ ὄφελον μάκαρόϲ \spation{νύ} τευ ἔμμεναι υἱὸϲ ἀνέροϲ.

Für den nachhomerischen Gebrauch verweise ich auf φέρε νυν, ἄγε νυν (Aristoph. Pax 1056), μή νυν, ferner auf das zumal bei Herodot so oft an zweiter Stelle zu lesende μέν νυν, sowie endlich auf Sophokles Philokt. 468 πρόϲ \spation{νύν} ϲε πατρὸϲ πρόϲ τε μητρόϲ — ἱκέτηϲ ἱκνοῦμαι. Oed. Col. 1333 πρόϲ \spation{νύν} ϲε κρηνῶν καὶ θεῶν ὁμογνίων αἰτῶ πιθέϲθαι. Eurip. Helena 137 πρόϲ \spation{νύν} ϲε γονάτων τῶνδ(ε). Ferner auf Sophokles Phil. 1177 ἀπό \spation{νύν} με λείπετ᾽ ἤδη. Eurip. Hiket. 56 μετά \spation{νυν} δόϲ. Vgl. auch Lobeck zum Aias Vs.~1332. — Im Kyprischen ist die Stellung von νυ freier: Tafel von Idal. 6 ἢ δυϝάνοι \spation{νυ}. 16 ἢ δώκοι \spation{νυ}. Ebenso im Böotischen: Collitz 488, 88 κὴ τὴ οὑπεραμερίη ἄκουρύ \spation{νυ} ἔνθω (= καὶ αἱ ὑπερημέριαι ἄκυροι ἔϲτων). — Ob übrigens in kypr. ὄνυ “hic”, τόνυ “hunc”, arkad. τάνυ “hanc” die Partikel νυ enthalten sei, scheint mir höchst zweifelhaft. Eher das υ von οὗτοϲ; vgl. ark. τωνί, ταννί.

Endlich noch ein Wort über τοι, soweit es reine Partikel geworden ist, für das die Stellung nach unserer Regel allgemein anerkannt ist; vgl. καίτοι, μέντοι. Darnach 1) Tmesis: Eurip. Herakles 1105 ἔκ \spation{τοι} πέπληγμαι. Orestes 1047 ἔκ \spation{τοί} με τήξειϲ. Aristoph. Vesp. 784 ἀνά \spation{τοί} με πείθειϲ. 2) Aristoph. Ekkles. 976 διά \spation{τοι} ϲὲ πόνουϲ ἔχω. Ferner mit γάρ τοι Theognis 287 ἐν γάρ \spation{τοι} πόλει ὧδε κακοψόγῳ ἁνδάνει οὐδέν. Plato Phaedo 60 C περὶ γάρ \spation{τοι} τῶν ποιημάτων. 108 D περὶ γάρ \spation{τοι} γῆϲ πολλὰ ἀκήκοα. 3) Sophokles Fragm. 855, 1 ὦ παῖδεϲ, ἥ \spation{τοι} Κύπριϲ οὐ Κύπριϲ μόνον. Eurip. Fragm. 222 Ν.\textsuperscript{2} τήν \spation{τοι} Δίκην λέγουϲι παῖδ᾽ εἶναι Χρόνου. Aristoph. Pax 511 οἵ \spation{τοι} γεωργοὶ τοὖργον ἐξέλκουϲι. Plato Sympos. \hypertarget{p377}{\emph{[S. 377]}}\label{p377} 219 A ἥ \spation{τοι} τῆϲ διανοίαϲ ὄψιϲ. Ferner mit γάρ τοι Eurip. Helena 93 τὸ γάρ \spation{τοι} πρᾶγμα ϲυμφορὰν ἔχει. Plato Apol. 29 Α τὸ γάρ \spation{τοι} θάνατον δεδιέναι. 4) Theognis 95 τοιοῦτόϲ \spation{τοι} ἑταῖροϲ (Bergk ἑταίρῳ) ἀνὴρ φίλοϲ. 605 πολλῷ \spation{τοι} πλέοναϲ λιμοῦ κόροϲ ὤλεϲεν ἤδη ἄνδραϲ. 837 διϲϲαί \spation{τοι} πόϲιοϲ κῆρεϲ δειλοῖϲι βροτοῖϲιν. 965 πολλοί \spation{τοι} κίβδηλοι — κρύπτουϲ(ι). 1027 ῥηιδίη \spation{τοι} πρῆξιϲ ἐν ἀνθρώποιϲ κακότητοϲ. 1030 δειλῶν \spation{τοι} κραδίη γίγνεται ὀξυτέρη. Aeschyl. Agam. 363 Δία \spation{τοι} ξένιον μέγαν αἰδοῦμαι. Eur. Or. 1167. Plato Sympos. 218 Ε ἀμήχανόν \spation{τοι} κάλλοϲ u. s. w.

Attisch τοιγάρτοι ist auch ein Zeichen für den Drang der Partikel nach vorn. Bei Homer kommt τοιγάρτοι noch nicht vor. Dafür haben wir noch mehrfach τοιγὰρ ἐγώ τοι — καταλέξω (oder ein anderes Futurum), wo eigentlich hinter τοιγάρ leicht zu interpungieren ist: “weil es so (τοί = Instrumental τώ + ι?) ist,~\mbox{—”.} Nachhomerisch wurde dann τοι — und ebenso οὖν — unmittelbar an τοιγάρ angeschlossen; τοιγάρτοι: τοιγάρ — τοι = latein. utrumne: utrum — ne (siehe unten).

\section*{VI.}
\addcontentsline{toc}{section}{VI.}

Dicht neben die Enklitika stellt sich eine Gruppe von Wörtern, die Krüger passend postpositive Partikeln nennt, weil sie gerade so wenig wie die Enklitika fähig sind an der Spitze eines Satzes zu stehen: ἄν, ἄρ, ἄρα, αὖ, γάρ, δέ, δῆτα, μέν, μήν, οὖν, τοίνυν. Woher diese Ähnlichkeit mit den Enklitika herrührt, habe ich hier nicht zu untersuchen. Doch scheinen verschiedene Momente in Betracht zu kommen: eine dieser Partikeln, nämlich αὖ, könnte ursprünglich wirklich enklitisch gewesen sein, da sie dem altindischen Enklitikum \emph{u} etymologisch entspricht, was ich gegenüber Kretschmer KZ. XXXI 364 festhalte. Sodann setzt sich τοίνυν aus zwei Enklitika τοι νυν zusammen. Das Ursprüngliche war jedenfalls z. B. αὐτόϲ τοί νυν. Seit wann man αὐτὸϲ τοίνυν sprach, lässt sich nicht mehr ermitteln. Bei andern lässt sich denken, dass sie erst allmählich postpositiv geworden seien, gerade wie im Lateinischen \emph{enim} und nach dessen Vorbild später \emph{namque} (\emph{itaque} nach \emph{igitur}). So wird man ἄν kaum von der lateinischen und gotischen Fragepartikel \emph{an} trennen können, und die ist in beiden Sprachen präpositiv. Man wird wohl sagen dürfen, dass im Griechischen die Partikel durch den Einfluss \hypertarget{p378}{\emph{[S.~378]}}\label{p378} von κε, mit dem sie bedeutungsgleich geworden war, von der ers\-ten Stelle im Satz weggelenkt und postpositiv geworden sei. Vor unsern Augen vollzieht sich eine derartige Wendung bei δή, das bei Homer und bei den seiner Sprache folgenden Dichtern den Satz einleiten kann, aber schon bei Homer entschieden postpositiv zu werden beginnt und dies in der Prosa ausschliesslich ist.

Nun liegt aber bei beiden Arten von postpositiven Partikeln, sowohl bei den von Haus aus enklitischen wie αὖ, als bei den unter den Einfluss eines Enklitikums getretenen wie ἄν, die Frage nahe, ob sie an der speziellen Stellungsregel der Enklitika, wie sie sich bei unserer Betrachtung herausgestellt hat, Anteil nehmen. Für diejenigen unter ihnen, die der Satzverknüpfung dienen, überhaupt für alle ausser ἄν, ist wohl anerkannt, dass sie dies thun, und bekannt, dass sie gerade so wie die eigentlichen Enklitika vermöge der Stellungsregel oft Tmesis und Ähnliches bewirken z. B. Sophokles Antig. 601 κατ᾽ \spation{αὖ} νιν φοινία θεῶν τῶν νερτέρων ἀμᾷ κοπίϲ. Eurip. Herakles 1085 ἀν᾽ \spation{αὖ} βακχεύϲει Καδμείων πόλιν. Häufig tritt οὖν zwischen Präposition und Kasus, zwischen Artikel und Substantiv. Ganz regelmässig thut dies δέ, bei dem überhaupt die Regel am schärfsten ist, da es vor allen Enklitika und Enklitoiden den Vortritt hat und nur äusserst selten an dritter Stelle steht. Bei den andern erleidet die Regel gewisse Einschränkungen: ἄρα folgt etwa einmal erst dem Verb z. B. Ε 748 Ἥρη δὲ μάϲτιγι θοῶϲ ἐπεμαίετ᾽ \spation{ἄρ᾽} ἵππουϲ. Herodot 4, 45, 21 πρότερον δὲ ἦν \spation{ἄρα} ἀνώνυμοϲ. Οὖν wird gern von der mit einem Verb verbundenen Präposition attrahiert und tritt dann zwischen sie und das Verbum: so überaus oft bei Herodot und Hippokrates; Hipponax (?) Fragm. 61 ἑϲπέρηϲ καθεύδοντα ἀπ᾽ \spation{οὖν} ἔδυϲε; Epicharm S.~225 Lor. (Athen.~6, 236~A) V.~76: τήνῳ κυδάζομαί τε κἀπ᾽ \spation{ὦν} ἠχθόμαν. Melanippides bei Ath. 10, 429~C τάχα δὴ τάχα τοὶ μὲν ἀπ᾽ \spation{ὦν} ὄλοντο. Sehr frei ist die Stellung von δή.

Eine Sonderstellung nimmt ἄν ein. Gottfried Hermann lehrt Opusc. 4, 7 “ἄν cum non sit enclitica et tamen initio poni nequeat, apertum est poni eam debere post eorum aliquod vocabulorum, ad quorum sententiam constituendam pertinet”, und stellt ἄν in scharfen Gegensatz zu κε. Schon bei Homer trete der Unterschied der Stellung an den beiden Beispielen \hypertarget{p379}{\emph{[S. 379]}}\label{p379} ἦ κε μέγ᾽ οἰμώξειε, wo κε unmittelbar auf ἦ folge, und ἦ ϲ᾽ ἂν τιϲαίμην, wo sich ἄν erst an das zweite Wort, ϲε, anschliesse, deutlich hervor. Dieser Unterschied zwischen ἄν und κεν muss uns überraschen. Wenn die Annahme richtig ist, dass ἄν durch den Einfluss von κε postpositiv geworden ist, so können wir für ἄν keine andre Stellung als die von κεν erwarten.

Ist aber der von Hermann behauptete Gegensatz wirklich vorhanden? Jedenfalls nicht in einer umfänglichen Kategorie von Sätzen, den Nebensätzen mit konjunktivischem Verbum. Denn hier ist unmittelbarer Anschluss an das satzeinleitende Wort bei ἄν ebenso unbedingte Regel wie bei κε(ν). Hierbei gilt ὅϲτιϲ als Worteinheit; ebenso ὁποῖόϲ τιϲ: Plato Phaedo 81 Ε ὁποῖ᾽ ἄττ᾽ \spation{ἂν} καὶ μεμελετηκυῖαι τύχωϲι. Xenophon Poroi 1, 1 \spation{ὁποῖοί τινεϲ ἂν} οἱ προϲτάται ὦϲι. Ferner gehen gewisse Partikeln, die selbst an den Satzanfang drängen, nämlich γάρ, γε, δέ, μέν, -περ, τε dem ἄν regelmässig voran, vereinzelt auch δή z. B. Plato Phaedo 114 B \spation{οἳ} δὲ δὴ \spation{ἂν} δόξωϲι διαφερόντωϲ προκεκρίϲθαι, μέντοι z.~Β. Xenophon Cyrop. 2, 1, 9 \spation{οἵ} γε μέντ᾽ \spation{ἂν} αὐτῶν φεύγωϲι, οὖν z.~Β. Aristoph. Ran. 1420 \spation{ὁπότεροϲ} οὖν \spation{ἂν} τῇ πόλει παραινέϲειν μέλλει τι χρηϲτόν, (wiewohl Herodot an einigen Stellen dem ἄν auch vor μέν und δέ den Vortritt lässt 1, 138, 5 ὃϲ \spation{ἂν} δὲ τῶν ἀϲτῶν λέπρην — ἔχῃ. 3, 72, 25 ὃϲ \spation{ἂν} μέν νυν τῶν πυλωρῶν ἑκὼν παρίῃ. 7, 8\textsuperscript{δ}3 ὃϲ \spation{ἂν} δὲ ἔχων ἥκῃ. 7, 8\textsuperscript{δ}3 ὃϲ \spation{ἂν} δὲ ἔχων ἥκῃ). [sic] Aber vor allen andern Wörtern hat ἄν den Vortritt. Die nicht entschuldbare Ausnahme Antiphon 5, 38 καθ᾽ ὧν μηνύῃ ἄν τιϲ hat Mätzner längst aus dem Oxoniensis, welcher καθ᾽ ὧν \spation{ἂν} μηνύῃ τιϲ schreibt, berichtigt. Um so unbegreiflicher ist noch in der zweiten Ausgabe der Fragm. Trag. von Nauck unter Euripides Fragm. 1029 den Versen zu begegnen ἀρετὴ δ᾽ ὅϲῳπερ μᾶλλον \spation{ἂν} χρῆϲθαι θέλῃϲ, τοϲῷδε μείζων γίγνεται καθ᾽ ἡμέραν. Für das fehlerhafte μᾶλλον ἂν vermutet Dümmler ἂν πλέον. Oder ist θέλῃϲ in θέλοιϲ zu ändern? — Sicherer scheint mir die Heilung einer dritten Stelle mit falsch gestelltem ἄν: Aristoph. Ran. 259 ὁπόϲον ἡ φάρυγξ \spation{ἂν} ἡμῶν χανδάνῃ. Es ist einfach umzustellen ἡ φάρυγξ \spation{ὁπόϲον} ἂν ἡμῶν, wodurch die Responsion mit Vers 264 οὐδέποτε· κεκράξομαι γάρ nicht schlechter wird. Ganz eng ist der Anschluss von ἄν an das Fügewort geworden in ion. ἤν, \hypertarget{p380}{\emph{[S. 380]}}\label{p380} att. ἄν, woraus durch nochmaligen Vortritt von εἰ das gewöhnliche ἐάν entstanden ist, in ὅταν, ἐπειδάν, ἐπάν = ion. ἐπήν, wo dann die Möglichkeit auch nur eine Partikel dem ἄν vorzuschieben wegfällt.

Aber auch in den andern Satzarten ist ursprünglich zwischen den Stellungsgewohnheiten von ἄν und denen von κε(ν) kein wesentlicher Unterschied zu bemerken. In Hauptsätzen wie in indikativischen und optativischen Nebensätzen finden wir bei Homer auf ἄν die Stellungsregel der Enklitika angewandt. Nur in wenigen Fällen entfernt sich ἄν etwas weiter von der Regel. Erstens hinter οὐ: Α~301 τῶν οὐκ \spation{ἄν} τι φέροιϲ. Β~488 πληθὺν δ᾽ οὐκ \spation{ἂν} ἐγὼ μυθήϲομαι οὐδ᾽ ὀνομήνω. Γ~66 ἑκὼν δ᾽ οὐκ \spation{ἄν} τιϲ ἕλοιτο. Ο~40 τὸ μὲν οὐκ \spation{ἂν} ἐγώ ποτε μὰψ ὀμόϲαιμι. Ρ~489 ἐπεὶ οὐκ \spation{ἂν} ἐφορμηθέντε γε νῶϊ τλαῖεν ἐναντίβιον ϲτάντεϲ μαχέϲαϲθαι Ἄρηι. Nun haben wir schon früher wiederholt beobachtet, dass die Negationen gern die Enklitika hinter sich nehmen. Und wenn bei κε diese Erscheinung weniger zu Tage tritt als bei ἄν, so darf an Ficks Bemerkung erinnert werden, dass das überhaupt im überlieferten Text auffallend häufige οὐκ ἄν mehrfach an die Stelle von οὔ κεν getreten scheine. (Doch siehe hiergegen Monro A Grammar of the Homeric Dialect~2. Ausg. S.~330). Dazu kommen noch drei weitere Stellen, eine mit καὶ ἄν: Ε~362 = 457 ὃϲ νῦν γε \spation{καὶ ἂν} Διὶ πατρὶ μάχοιτο, während Ξ~244~f. ἄλλον μέν κεν ἔγωγε θεῶν αἰειγενετάων ῥεῖα κατευνήϲαιμι \spation{καὶ ἂν} ποταμοῖο ῥέεθρα Ὠκεανοῦ das καὶ ἄν als neuer Satzanfang betrachtet werden kann. Eine mit τάχ᾽ ἄν: Α~205 ᾗϲ ὑπεροπλίῃϲι \spation{τάχ᾽ ἄν} ποτε θυμὸν ὀλέϲϲῃ. (Vgl. τάχ᾽ ἄν am Satzanfang β~76 \spation{τάχ᾽ ἄν} ποτε καὶ τίϲιϲ εἴη.) Endlich eine mit τότ᾽ ἄν (vgl. τότ᾽ ἄν am Satzanfang Σ~397, Ω~213, ι~211): Χ~108 ἐμοὶ δὲ \spation{τότ᾽ ἂν} πολὺ κέρδιον εἴη. Diese paar Stellen genügen doch gewiss nicht, um Hermanns scharfe Trennung von ἄν und κε(ν) zu rechtfertigen. Sein eigenes Beispiel ἦ ϲ᾽ \spation{ἂν} τιϲαίμην gegenüber ἦ κε μέγ᾽ οἰμώξειε besagt nichts, da ϲ(ε) enklitisch ist. Und aus εἴ περ ἄν gegenüber Η~387 αἴ κέ περ ὔμμι φίλον καὶ ἡδὺ γένοιτο lassen sich natürlich ebenfalls keine Folgerungen ziehen. Vergleiche überdies die freilich bestrittenen Verbindungen ὄφρ᾽ ἂν μέν κεν, οὔτ᾽ ἄν κεν.

Die nachhomerische Litteratur hat ἄν streng nach der alten Regel in den konjunktivischen Nebensätzen. Schwan-\hypertarget{p381}{\emph{[S. 381]}}\label{p381}kender ist der Gebrauch bei Nebensätzen mit anderm Modus. Doch haftet auch hier ἄν in gewissen Fällen fest am Einleitungswort. Besonders in betracht kommen die Verbindungen ὡϲ ἄν, ὅπωϲ ἄν, ὥϲπερ ἄν.

Am klarsten ist der Sachverhalt bei den mit ὡϲ und ὅπωϲ beginnenden, den Optativ oder Indikativ mit ἄν enthaltenden Final- und Konsekutivsätzen, dank den Sammlungen, die für die erstern Weber angelegt und publiziert hat (Weber Die Entwicklungsgeschichte der Absichtsätze [Beiträge zur historischen Syntax der griechischen Sprache herausgegeben von M. Schanz II] 1 und 2). In solchen Sätzen haben wir ὡϲ ἄν in unmittelbarer Folge nicht bloss bei Homer (z.~B. ρ~562 \spation{ὡϲ ἂν} πύρνα κατὰ μνηϲτῆραϲ ἀγείροι) sondern auch Archiloch. Fragm. 30 \spation{ὡϲ ἂν} καὶ γέρων ἠράϲϲατο und Fragm. 101 \spation{ὡϲ ἄν} ϲε θωϊὴ λάβοι. Pindar Olymp. 7, 42 \spation{ὡϲ ἂν} θεᾷ πρῶτοι κτίϲαιεν βωμόν. Sophokles bei Aristoph. Aves 1338 \spation{ὡϲ ἂν} ποταθείην. Herodot 1, 152, 4 \spation{ὡϲ ἂν} πυνθανόμενοι πλεῖϲτοι ϲυνέλθοιεν Σπαρτιητέων. Ebenso 5, 37, 9. 7, 176, 20. 8, 7, 2. 9, 22, 18. 9, 51, 14. [Andocides] 4, 23 \spation{ὡϲ ἂν} μάλιϲτα τὸν υἱὸν ἐχθρὸν ἑαυτῷ καὶ τῇ πόλει ποιήϲειε. Plato Phaedo 82~Ε \spation{ὡϲ ἂν} μάλιϲτα αὐτὸϲ ὁ δεδεμένοϲ ξυλλήπτωρ εἴη τοῦ δεδέϲθαι. Sympos. 187~D τοῖϲ μὲν κοϲμίοιϲ τῶν ἀνθρώπων, καὶ \spation{ὡϲ ἂν} κοϲμιώτεροι γίγνοιντο οἱ μή πω ὄντεϲ, δεῖ χαρίζεϲθαι. 190~C δοκῶ μοι — ἔχειν μηχανήν, \spation{ὡϲ ἂν} εἶεν ἄνθρωποι καὶ παύϲαιντο τῆϲ ἀκολαϲίαϲ. Demosth. 6, 37 \spation{ὡϲ δ᾽ ἂν} ἐξεταϲθείη μάλιϲτ᾽ ἀκριβῶϲ, μὴ γένοιτο, wo das ὡϲ ἄν doch wohl konsekutiv zu nehmen ist. Sehr häufig bei Xenophon, dem einzigen attischen Prosaisten, der häufig ὡϲ mit ἄν und dem Optativ in rein finalem Sinne verbindet. Von den siebzehn bei Weber S.~83~ff. aufgeführten Belegstellen haben vierzehn ἄν unmittelbar hinter ὡϲ, nur drei davon getrennt, final Cyrop. 5, 1, 18 \spation{ὡϲ} μηδενὸϲ \spation{ἂν} δέοιτο. 7, 5, 37 \spation{ὡϲ} ὅτι ἥκιϲτα \spation{ἂν} ἐπιφθόνοιϲ ϲπάνιοϲ τε καὶ ϲεμνὸϲ φανείη, konsekutiv Sympos. 9, 3 \spation{ὡϲ} πᾶϲ \spation{ἂν} ἔγνω, ὅτι ἀϲμένη ἤκουϲε: die ersten und einzigen Fälle, wo die den Zusammenschluss von ὡϲ und ἄν verlangende Tradition durchbrochen ist. Allerdings kommen nach der handschriftlichen Überlieferung noch zwei euripideische Verse hinzu: Iphig. Taur. 1024 \spation{ὡϲ} δὴ ϲκότοϲ λαβόντεϲ ἐκϲωθεῖμεν \spation{ἄν} und Iphig. Aul. 171 Ἀχαιῶν ϲτρατιὰν \spation{ὡϲ} ἴδοιμ᾽ \spation{ἄν}. Aber der erstere Vers ist seit Markland den Kritikern verdächtig, und im \hypertarget{p382}{\emph{[S. 382]}}\label{p382} zweiten schreibt man jetzt allgemein ὡϲ ἐϲιδοίμαν [Pl. Gorg. 453~C οὕτω προΐῃ, \spation{ὡϲ} μάλιϲτ᾽ \spation{ἂν} — ποιοίη ist ὡϲ relativ.]

Noch fester ist die Verbindung ὅπωϲ ἄν in solchen Sätzen: Aeschylus Agam. 362 \spation{ὅπωϲ ἄν} — μήτε πρὸ καιροῦ μήθ᾽ ὑπὲρ ἄϲτρων βέλοϲ ἠλίθιον ϲκήψειεν. Herodot 1, 75, 16 \spation{ὅκωϲ ἂν} τὸ ϲτρατόπεδον ἱδρυμένον κατὰ νώτου λάβοι. Ebenso 1, 91, 7. 1, 110, 16. 2, 126, 7. 3, 44, 5. 5, 98, 20. 8, 13, 9. — Thucydides 7, 65, 1 \spation{ὅπωϲ ἂν} ἀπολιϲθάνοι καὶ μὴ ἔχοι ἀντιλαβὴν ἡ χείρ. Aristoph. Ekkles. 881 \spation{ὅπωϲ ἂν} περιλάβοιμ᾽ αὐτῶν τινα. Plato Lysis 207~Ε \spation{ὅπωϲ ἂν} εὐδαιμονοίηϲ. Sehr häufig bei Xenophon, zwölfmal (ungerechnet ὅπωϲ “wie” nach Verben des Beratens und Überlegens) nach den Nachweisen von Weber 2, S.~83~ff., überall so, dass ἄν dem ὅπωϲ unmittelbar folgt; eigentümlich Sympos. 7, 2 ϲκοπῶ, \spation{ὅπωϲ ἂν} ὁ μὲν παῖϲ ὅδε ὁ ϲὸϲ καὶ ἡ παῖϲ ἧδε ὡϲ ῥᾷϲτα διάγοιεν, ἡμεῖϲ \spation{δ᾽ ἂν} μάλιϲτα (ἂν) εὐφραινοίμεθα. Corpus Inscr. Att. 2, 300, 20 (295/4 a.~Ch.) \spation{ὅπωϲ ἂν} ὁ δῆμο[ϲ ἀπαλλαγείη τ]οῦ πολέμου, wo der von Herwerden und Weber 2 S.~3 empfohlene Konjunktiv ἀπαλλαγῇ für die Lücke, deren Umfang durch die ϲτοιχηδὸν-Schreibung feststeht, zu kurz ist. — Nach allem dem kann kein Zweifel sein, dass Hermann und Velsen Aristoph. Ekkles. 916 mit Unrecht ὅπωϲ ϲαυτῆϲ <\spation{ἂν}> κατόναι(ο) schreiben wollen, und dass, wenn hier überhaupt ἄν einzusetzen ist, es seine Stelle unmittelbar hinter ὅπωϲ haben muss.

Den Finalsätzen mit ὡϲ, ὅπωϲ ganz nahe stehn die mit denselben Partikeln oder auch mit πῶϲ eingeleiteten indirekten Fragesätze mit Optativ und ἄν. a) ὡϲ ἄν ist unmittelbar verbunden Plato Republ. 5, 473~A ἐὰν οἷοί τε γενώμεθα εὑρεῖν, \spation{ὡϲ ἂν} ἐγγύτατα τῶν εἰρημένων πόλιϲ οἰκήϲειεν. Xenophon. Oeconom. 19, 18 διδάϲκει, \spation{ὡϲ ἂν} κάλλιϲτά τιϲ αὐτῇ χρῷτο. Demosth. 4, 13 τἆλλ᾽ \spation{ὡϲ ἂν} μοι βέλτιϲτα καὶ τάχιϲτα δοκεῖ παραϲκευαϲθῆναι, καὶ δὴ πειράϲομαι λέγειν. [20,87] Abweichend ist, so viel ich sehe, nur der zweite Teil des demosthenischen Beispiels 6, 3 \spation{ὡϲ} μὲν \spation{ἂν} εἴποιτε καὶ — ϲυνεῖτε, ἄμεινον Φιλίππου παρεϲκεύαϲθε, \spation{ὡϲ} δὲ κωλύϲαιτ᾽ \spation{ἂν} ἐκεῖνον —, παντελῶϲ ἀργῶϲ ἔχετε. [Demosth.] 10, 45 siehe unten, b) ὅπωϲ ἄν ist unmittelbar verbunden [Hippokrates] περὶ τέχνηϲ c. 2 pag. 42, 20 Gomp. οὐκ οἶδ᾽ \spation{ὅπωϲ ἄν} τιϲ αὐτὰ νομίϲειε μὴ ἐόντα. Auch häufig bei Xenophon: Anab.~2, 5, 7 τὸν γὰρ θεῶν πόλεμον οὐκ οἶδα —, \spation{ὅπωϲ ἂν} εἰϲ ἐχυρὸν χωρίον ἀποϲταίη. Ebenso Anab. \hypertarget{p383}{\emph{[S. 383]}}\label{p383} 3, 2, 27. 4, 3, 14. 5, 7, 20. Hellenika 2, 3, 13. 3, 2, 1. 7, 1, 27. 7, 1, 33. Cyropädie 1, 4, 13. 2, 1, 4. — Gegenbeispiele habe ich keine zur Hand. (Vgl. aber Eurip. Hel. 146~f. ὡϲ τύχω μαντευμάτων, \spation{ὅπῃ} νεὼϲ ϲτείλαιμ᾽ \spation{ἂν} οὔριον πτερόν.) c) πῶϲ ἄν unmittelbar verbunden z.~B. Xenophon Anab. 1, 7, 2 ϲυνεβουλεύετο, \spation{πῶϲ ἂν} τὴν μάχην ποιοῖτο. Demosth. 19, 14 εἰ — ἐϲκόπει —, \spation{πῶϲ ἂν} ἄριϲτ᾽ ἐναντιωθείη τῇ εἰρήνῃ. Auch hier habe ich keine Gegenbeispiele.

Aber auch das relativische ὡϲ, ὥϲπερ ‘wie’ zeigt die Eigentümlichkeit ἄν fest an sich zu fesseln; zwar haben wir, um mit ὡϲ zu beginnen, bei Sophokles Oed. Col. 1678 \spation{ὡϲ} μάλιϲτ᾽ \spation{ἂν} ἐν πόθῳ λάβοιϲ, bei Plato Phaedo 59~A \spation{ὡϲ} εἰκὸϲ δόξειεν \spation{ἂν} εἶναι παρόντι πένθει. 118~Β \spation{ὡϲ} ἡμεῖϲ φαῖμεν \spation{ἄν}. Sympos. 190~Α \spation{ὡϲ} ἀπὸ τούτων \spation{ἄν} τιϲ εἰκάϲειεν. Phileb. 15~C \spation{ὡϲ} γοῦν ἐγὼ φαίην \spation{ἄν}. Leges 4, 712~C \spation{ὥϲ} γ᾽ ἡμεῖϲ \spation{ἂν} οἰηθεῖμεν und öfters; bei Xenoph. Anab. 1, 5, 8 θᾶττον ἢ \spation{ὥϲ} τιϲ \spation{ἂν} ᾤετο, bei Pseudo-Demosth. 10, 45 \spation{ὡϲ} μὲν οὖν εἴποι τιϲ {ἄν}, — ταῦτ᾽ ἴϲωϲ ἐϲτίν· (der Rest des Satzes: \spation{ὡϲ} δὲ καὶ γένοιτ᾽ \spation{ἄν}, νόμῳ διορθώϲαϲθαι δεῖ, enthält fragendes ὡϲ). Aber diesen Beispielen gegenüber haben wir nicht bloss bei Plato Phaedrus 231~A ἑκόντεϲ, \spation{ὡϲ ἂν} ἄριϲτα περὶ τῶν οἰκείων βουλεύϲαιντο, πρὸϲ τὴν δύναμιν τὴν αὑτῶν εὖ ποιοῦϲιν, [Apol. 34 C]; bei Demosth. 27, 7 \spation{ὡϲ ἂν} ϲυντομώτατ᾽ εἴποι τιϲ. 39, 22 ϲτέρξαϲ \spation{ὡϲ ἂν} υἱόν τιϲ ϲτέρξαι. 45, 18 οὐδὲ μεμαρτύρηκεν ἁπλῶϲ, \spation{ὡϲ ἄν} τιϲ τἀληθῆ μαρτυρήϲειε. Proöm. 2, 3 (Ββ bei Blass) τὸ — μὴ πάνθ᾽ \spation{ὡϲ ἂν} ἡμεῖϲ βουλοίμεθ᾽ ἔχειν —, οὐδέν ἐϲτι θαυμαϲτόν, sondern vor allem kommt in betracht der elliptische Gebrauch von ὡϲ ἄν, der nur zu begreifen ist, wenn enge Verbindung von ὡϲ ἄν im Sprachbewusstsein festsass. Eigentlich ist bei solchem Gebrauch das Verb des Hauptsatzes in optativischer Form wiederholt zu denken, wie es an den angeführten Stellen Demosth. 39, 22 und 45, 18 wirklich wiederholt ist.

Es steht dieses ὡϲ ἄν a) vor εἰ Plato Protag. 344~B \spation{ὡϲ ὰν} εἰ λέγοι; vgl. das \spation{ὡϲανεί} der nachklassischen Gräzität; b) vor Partizipien; α) mit neuem Subjekt: Xenophon Cyrop. 1, 3, 8 καὶ τὸν Κῦρον ἐρέϲθαι προπετῶϲ, \spation{ὡϲ ἂν} παῖϲ μηδέπω ὑποπτήϲϲων. Memorab. 3, 8, 1 ἀπεκρίνατο, οὐχ ὥϲπερ οἱ φυλαττόμενοι —, ἀλλ᾽ \spation{ὡϲ ἂν} πεπειϲμένοι μάλιϲτα πράττειν τὰ δέοντα. Demosth. 4, 6 ἔχει τὰ μέν, \spation{ὡϲ ἂν} ἑλών τιϲ πολέμῳ. 24, 79 οὐδὲ ταῦθ᾽ ἁπλῶϲ — φανήϲεται γεγραφώϲ, ἀλλ᾽ \spation{ὡϲ} \hypertarget{p384}{\emph{[S. 384]}}\label{p384} \spation{ἂν} μάλιϲτά τιϲ ὑμᾶϲ ἐξαπατῆϲαι καὶ παρακρούϲαϲθαι βουλόμενοϲ. [Demosth.] 34, 22 ϲυγγραφὰϲ ἐποιήϲαντο —, \spation{ὡϲ ἂν} οἱ μάλιϲτα ἀπιϲτοῦντεϲ. Häufiger β) ohne ausdrückliche Nennung des eigentlich gedachten unbestimmten Subjekts (“wie einer thäte in der und der Verfassung”), wobei dann ὡϲ ἄν der Bedeutung von ἅτε sehr nahe kommt und das Partizip sich nach dem Kasus desjenigen Wortes im Hauptsatz richtet, dessen Begriff als Träger der partizipialen Bestimmung vorschwebt. So schon Solon Fragm. 36, 10 Bgk. (nun bestätigt durch Aristot. Ἀθην. πολιτεία S.~31, 10 Kenyon) γλῶϲϲαν οὐκέτ᾽ Ἀττικὴν ἱέτναϲ [sic] \spation{ὡϲ ἂν} πολλαχοῦ πλανωμένουϲ. Lysias 1, 12 ἡ γυνὴ οὐκ ἤθελεν ἀπιέναι, \spation{ὡϲ ἂν} ἀϲμένη με ἑορακυῖα. Xenophon Memorab. 3, 6, 4 διεϲιώπηϲεν, \spation{ὡϲ ἂν} τότε ϲκοπῶν, ὁπόθεν ἄρχοιτο. Demosth. 21, 14 κρότον τοιοῦτον \spation{ὡϲ ἂν} ἐπαινοῦντέϲ τε καὶ ϲυνηϲθέντεϲ ἐποιήϲατε. 19, 256 θρυλοῦντοϲ ἀεί, τὸ μὲν πρῶτον \spation{ὡϲ ἂν} εἰϲ κοινὴν γνώμην ἀποφαινομένου. 54, 7 διαλεχθείϲ τι πρὸϲ αὑτὸν οὕτωϲ \spation{ὡϲ ἂν} μεθύων. [Demosth.] 59, 24 ϲυνεδείπνει ἐναντίον πολλῶν Νέαιρα, \spation{ὡϲ ἂν} ἑταίρα οὖϲα. Aristot. Ἀθην. πολιτ. 19, 12 Keny. ϲημεῖον δ᾽ ἐ<πι>φέρουϲι τό τε ὄνομα τοῦ τέλουϲ, \spation{ὡϲ ἂν} ἀπὸ τοῦ πράγματοϲ κείμενον. Anthol. Palat. 6, 259, 6 ἔπτη δ᾽ \spation{ὡϲ ἂν} ἔχων τοὺϲ πόδαϲ ἡμετέρουϲ. c) Sonst: Aeschylus Suppl. 718 ἄγαν καλῶϲ κλύουϲά γ᾽ \spation{ὡϲ ἂν} οὐ φίλη. Thucyd. 1, 33, 1 \spation{ὡϲ ἂν} μάλιϲτα, μετὰ ἀειμνήϲτου μαρτυρίου τὴν χάριν καταθήϲεϲθε. 6, 57, 3 ἀπεριϲκέπτωϲ προϲπεϲόντεϲ καὶ \spation{ὡϲ ἂν} μάλιϲτα δι᾽ ὀργῆϲ. Xenophon. Cyrop. 5, 4, 29 δῶρα πολλὰ — φέρων καὶ ἄγων, \spation{ὡϲ ἂν} ἐξ οἴκου μεγάλου. Memorab. 2, 6, 38 εἴ ϲοι πείϲαιμι — (ἐπιτρέπειν) τὴν πόλιν ψευδόμενοϲ, \spation{ὡϲ ἂν} ϲτρατηγικῷ τε καὶ δικαϲτικῷ καὶ πολιτικῷ. Demosth. 1, 21 οὐδ᾽ \spation{ὡϲ ἂν} καλλιϲτ᾽ αὐτῷ τὰ παρόντ᾽ ἔχει. 18, 291 οὐχ \spation{ὡϲ ἂν} εὔνουϲ καὶ δίκαιοϲ πολίτηϲ ἔϲχε τὴν γνώμην. 23, 154 ἀφυλάκτων ὄντων, \spation{ὡϲ ἂν} πρὸϲ φίλον τῶν ἐν τῇ χώρᾳ. Corpus Inscr. Att. 2, 243 (vor 301 a.~Chr.), 34 ὑπὲρ τῶν ἱππέων τῶν αἰχμαλώτων \spation{ὡϲ ἂν} ὑπὲρ πολιτῶν.

Noch schlagender vielleicht ist der Gebrauch von ὥϲπερ. Zwar sagt Sophokles Fragm. 787 \spation{ὥϲπερ} ϲελήνηϲ ὄψιϲ εὐφρόναϲ δύο ϲτῆναι δύναιτ᾽ \spation{ἄν} und Demosthenes 4, 39 τὸν αὐτὸν τρόπον, \spation{ὥϲπερ} τῶν ϲτρατευμάτων ἀξιώϲειέ τιϲ \spation{ἂν} τὸν ϲτρατηγὸν ἡγεῖϲθαι. Aber dafür lesen wir bei Antiphon 6, 11 \spation{ὥϲπερ ἂν} ἥδιϲτα καὶ ἐπιτηδειότατα ἀμφοτέροιϲ ἐγίγνετο, ἐγὼ μὲν ἐκέλευον u. s. w., bei Plato Phaedo 87~Β δοκεῖ ὁμοίωϲ λέγεϲθαι \hypertarget{p385}{\emph{[S. 385]}}\label{p385} ταῦτα, \spation{ὥϲπερ ἄν} τιϲ περὶ ἀνθρώπου — λέγοι τοῦτον τὸν λόγον. Phaedrus 268~D ἀλλ᾽ \spation{ὥϲπερ ἂν} μουϲικὸϲ ἐντυχὼν ἀνδρὶ — οὐκ ἀγρίωϲ εἴποι \spation{ἄν} mit beachtenswertem doppeltem ἄν, bei Xenophon Hellen. 3, 1, 14 ἐκείνῳ δὲ πιϲτευούϲηϲ, \spation{ὥϲπερ ἂν} γυνὴ γαμβρὸν ἀϲπάζοιτο. Besonders aber, wenn dem Vergleichungssatz ein kondizionaler eingefügt ist, herrscht durchaus die Wortfolge ὥϲπερ ἂν εἰ —: Plato Apologie 17~D \spation{ὥϲπερ} οὖν \spation{ἄν}, εἴ [sic] τῷ ὄντι ξένοϲ ἐτύγχανον ὤν, ξυνεγιγνώϲκετε δήπου \spation{ἄν} μοι. Gorgias 447~D \spation{ὥϲπερ ἄν}, εἰ ἐτύγχανεν ὢν ὑποδημάτων δημιουργόϲ, ἀποκρίναιτο \spation{ἂν} δήπου ϲοι. 451~Α \spation{ὥϲπερ ἄν}, εἴ τίϲ με ἔροιτο —, εἴποιμ᾽ \spation{ἄν}. 453~C \spation{ὥϲπερ ἄν}, εἰ ἐτύγχανον —, ἆρ᾽ οὐκ \spation{ἂν} δικαίωϲ ϲε ἠρόμην; Protag. 311~Β \spation{ὥϲπερ ἄν}, εἰ ἐπενόειϲ — ἀργύριον τελεῖν —, εἴ τίϲ ϲε ἤρετο —, τί \spation{ἂν} ἀπεκρίνω. 318~Β \spation{ὥϲπερ ἄν}, εἰ — Ἱπποκράτηϲ ὅδε ἐπιθυμήϲειε — καὶ — ἀκούϲειεν —, εἰ αὐτὸν ἐπανέροιτο —, εἴποι \spation{ἂν} αὐτῷ. 327~Ε \spation{ὥϲπερ ἄν}, εἰ ζητοίηϲ, τίϲ διδάϲκαλοϲ τοῦ ἑλληνίζειν, οὐδ᾽ \spation{ἂν} εἷϲ φανείη, und öfters. Demosth. 20, 143 \spation{ὥϲπερ ἄν}, εἴ τιϲ — τάττοι, οὐκ \spation{ἂν} αὐτόϲ γ᾽ ἀδικεῖν παρεϲκευάϲθαι δόξαι.

Auch hier tritt der enge Anschluss von ἄν besonders daran zu Tage, dass ὥϲπερ ἄν überaus oft elliptisch ohne (optativisches oder präteritales) Verbum steht, entweder indem eine Form des Verbums εἰμί zu ergänzen ist, wie Demosth. 9, 30 \spation{ὥϲπερ ἄν}, εἰ υἱὸϲ — διῴκει τι μὴ καλῶϲ ἢ ὀρθῶϲ, αὐτὸ μὲν τοῦτ᾽ ἄξιον μέμψεωϲ, oder das Verbum des übergeordneten Satzes: Andoc. 1, 57 χρὴ ἀνθρωπίνωϲ περὶ τῶν πραγμάτων ἐκλογίζεϲθαι, \spation{ὥϲπερ ἂν} αὐτὸν ὄντα ἐν τῇ ϲυμφορᾷ (= ὥϲπερ ἄν τιϲ αὐτὸϲ ὢν — ἐκλογίζοιτο). Isäus 6, 64 τοῦτ᾽ αὐτὸ ἐπιδεικνύτω \spation{ὥϲπερ ἂν} ὑμῶν ἕκαϲτοϲ. Demosth. 18, 298 οὐδὲ — ὁμοίωϲ ὑμῖν, \spation{ὣϲπερ ἂν} τρυτάνη ῥέπων ἐπὶ τὸ λῆμμα ϲυμβεβούλευκα (V. C. ὥϲπερ ἂν εἰ, Blass bloss ὥϲπερ). 19, 226 \spation{ὥϲπερ ἂν} παρεϲτηκότοϲ αὐτοῦ. 21, 117 χρώμενοϲ \spation{ὥϲπερ ἂν} ἄλλοϲ τιϲ αὐτῷ τὰ πρὸ τούτου. 21, 225 δεῖ τοίνυν τούτοιϲ βοηθεῖν, \spation{ὥϲπερ ἂν} αὑτῷ τιϲ ἀδικουμένῳ. 29, 30 \spation{ὥϲπερ ἄν} τιϲ ϲυκοφαντεῖν ἐπιχειρῶν. (S. Blass nach A; die meisten ὥϲπερ ἂν εἴ τιϲ, mit welcher Lesart die Stelle unten einzufügen wäre.) 39, 10 πλὴν εἰ ϲημεῖον \spation{ὥϲπερ ἂν} ἄλλῳ τινί, τῷ χαλκίῳ προϲέϲται. 45, 35 \spation{ὥϲπερ ἂν} δοῦλοϲ δεϲπότῃ διδούϲ. 49, 27 \spation{ὥϲπερ ἂν} ἄλλοϲ τιϲ ἀποτυχών.

Zumal findet sich dieses bei folgendem εἰ c. optativo \hypertarget{p386}{\emph{[S. 386]}}\label{p386} oder praeterito: Isocrates 4, 69 \spation{ὥϲπερ ἂν} εἰ (“wie wenn”) πρὸϲ ἅπανταϲ ἀνθρώπουϲ ἐπολέμηϲαν. 18, 59 \spation{ὥϲπερ ἂν} εἴ τῳ Φρυνώνδαϲ πανουργίαν ὀνειδίϲειεν. Vgl. 10, 10. 15, 2. 15, 14. 15, 298. Ebenso Plato Protag. 341~C \spation{ὥϲπερ ἂν} εἰ ἤκουεν. Kratyl. 395~Ε \spation{ὤϲπερ ἂν} εἴ τιϲ ὀνομάϲειε καὶ εἴποι. Vgl. Krat. 430~A. Gorg. 479~A. Phaedo 98~C, 109~C, Sympos. 199~D, 204~E. Republik 7, 529~D u.s.w. Ebenso Xenophon Cyrop. 1, 3, 2 ἠϲπάζετο αὐτόν, \spation{ὥϲπερ ἂν} εἴ τιϲ — ἀϲπάζοιτο. Ebenso Demosthenes 6, 8 \spation{ὥϲπερ ἂν} εἰ πολεμοῦντεϲ τύχοιτε. 18, 194 \spation{ὥϲπερ ἂν} εἴ τιϲ ναύκληρον αἰτιῷτο (vgl. §~243) und andere Redner. [Demosth.] 35, 28 \spation{ὥϲπερ ἂν} εἴ τιϲ εἰϲ Αἴγιναν ἢ εἰϲ Μέγαρα ὁρμίϲαιτο. — Daran knüpft sich wieder ὥϲπερ ἂν εἰ (meist geschrieben ὡϲπερανεί) im Sinne von \emph{quasi} ‘wie’; vgl. ὡϲεί, ὡϲπερεί, ohne Verbum finitum gebraucht z.~B. Plato Gorgias 479~A ὡϲπερανεὶ παῖϲ. Isokrates 4, 148. Xenophon Sympos. 9, 4. Demosth. 18, 214. Über ὡϲπερανεί, καθαπερανεί bei Aristoteles belehrt der Bonitzsche Index S.~41.

Auch die Relativsätze geben zu Bemerkungen Anlass. Erstens folgt in der Verbindung οὐκ ἔϲτιν ὅϲτιϲ (oder auch in fragender Form ἔϲτιν ὅϲτιϲ….;), wo der Hauptsatz erst durch den Nebensatz seinen Inhalt erhält und also der Zusammenschluss beider Sätze ein besonders enger ist, das ἄν regelmässig unmittelbar auf das Relativum: Soph. Antig. 912 οὐκ ἔϲτ᾽ ἀδελφόϲ, \spation{ὅϲτιϲ ἂν} βλάϲτοι ποτέ. Eurip. El. 903 οὐκ ἔϲτιν οὐδεὶϲ \spation{ὅϲτιϲ ἂν} μέμψαιτό ϲε. [Heracl. 972]. Pl. Phaedo 78~A οὐκ ἔϲτιν εἰϲ \spation{ὅ τι ἂν} ἀναγκαιότερον ἀναλίϲκοιτε χρήματα. 89~D οὐκ ἔϲτιν, \spation{ὅτι ἄν} τιϲ μεῖζον — πάθοι. Phaedrus 243~Β τουτωνὶ οὐκ ἔϲτιν, \spation{ἅττ᾽ ἂν} ἐμοὶ εἶπεϲ ἡδίω. Demosth. 24, 138 οἶμαι γὰρ τοιοῦτον οὐδὲν εἶναι, \spation{ὅτου ἂν} ἀπέϲχετο. 24, 157 ἔϲτιν, \spation{ὅϲτιϲ ἂν} — ἐψήφιϲεν; 19, 309 ἔϲτιν, \spation{ὅϲτιϲ} ἂν — ὑπέμεινεν; 18, 43 οὐ γὰρ ἦν, \spation{ὅ τι ἂν} ἐποιεῖτε. 45, 33 ἔϲτιν οὖν, \spation{ὅϲτιϲ ἂν} τοῦ ξύλου καὶ τοῦ χωρίου — τοϲαύτην ὑπέμεινε φέρειν μίϲθωϲιν; ἔϲτι δ᾽ \spation{ὅϲτιϲ ἂν} — ἐπέτρεψεν; vgl. auch [Demosth.] 13, 22 οὐκ ἔϲτ᾽ οὐδείϲ, \spation{ὅϲτιϲ ἂν} εἴποι. Fast gleichwertig mit οὐκ ἔϲτιν ὅϲτιϲ sind solche Wendungen, wie die bei Sophokles Oed. Col. 252 vorliegende οὐ γὰρ ἴδοιϲ ἂν ἀθρῶν βροτῶν \spation{ὅϲτιϲ ἂν} εἰ θεὸϲ ἄγοι ἐκφυγεῖν δύναιτο oder die bei Plato Phaedo 107~Α οὐκ οἶδα εἰϲ \spation{ὅντιν᾽ ἄν} τιϲ ἄλλον καιρὸν ἀναβάλλοιτο und bei Xenophon Anab. 3, 1, 40 οὐκ οἶδα \spation{ὅ τι ἄν} τιϲ χρήϲαιτο αὐτῷ. Und ebenso eng wie in allen diesen \hypertarget{p387}{\emph{[S. 387]}}\label{p387} Beispielen ist der Zusammenschluss von Haupt- und Nebensatz, wenn ὅϲτιϲ durch οὕτω angekündigt ist: Isokrates 9, 35 οὐδεὶϲ γάρ ἐϲτιν οὕτω ῥᾴθυμοϲ \spation{ὅϲτιϲ ἂν} δέξαιτο.

Die Verbindung von ὅϲτιϲ und ἄν kann in solchen Sätzen allerdings unterbrochen werden, erstens durch ποτε, was ganz natürlich ist: Plato Phaedo 79~A τῶν δὲ κατὰ ταῦτα ἐχόντων οὐκ ἔϲτιν \spation{ὅτῳ} ποτ᾽ \spation{ἂν} ἄλλῳ ἐπιλάβοιο. Zweitens durch οὐκ: Isokr. 8, 52 ὧν οὐκ ἔϲτιν, \spation{ὅϲτιϲ} οὐκ \spation{ἄν} τιϲ καταφρονήϲειεν. Plato Gorgias 456~C οὐ γάρ ἐϲτιν, περὶ \spation{ὅτου} οὐκ \spation{ἂν} πιθανώτερον εἴποι ὁ ῥητορικόϲ. [491~Ε.] Symposion 179~Α οὐδεὶϲ οὕτω κακόϲ, \spation{ὅντινα} οὐκ \spation{ἂν} αυτὸϲ ὁ Ἔρωϲ ἔνθεον ποιήϲειεν. Xenophon Cyrop. 7, 5, 61 οὐδεὶϲ γάρ, \spation{ὅϲτιϲ} οὐκ \spation{ἂν} ἀξιώϲειεν. (Vgl. Lykurg 69 τίϲ ὅυτωϲ — φθονερόϲ ἐϲτιν —, \spation{ὃϲ} οὐκ \spation{ἂν} εὔξαιτο —;) Man beachte, dass von den Beispielen mit unmittelbar verbundenem ὅϲτιϲ ἄν keines im Relativsatze die Negation enthält, sodass also die Zwischenschiebung von οὐκ als Regel gelten kann. Sie ist auch gar nicht verwunderlich; man vergleiche, was oben S.~335, 336, 343 über die Voranstellung von οὐκ vor Enklitika und S.~380 über homerisches οὐκ ἄν zu bemerken war. Eigentümlich ist Demosth. 18, 206: Hier geben S und L, also die beste Textquelle: οὐκ ἔϲθ᾽ \spation{ὅϲτιϲ ἂν} οὐκ \spation{ἂν} εἰκότωϲ ἐπιτιμήϲειέ μοι. Wenn die Überlieferung richtig ist, so beruht die Ausdrucksweise auf einer Kontamination, auf dem Bedürfnis der üblichen Verbindung ὅϲτιϲ ἄν und der üblichen Verbindung (ὅϲτιϲ) οὐκ ἄν gleichmässig gerecht zu werden. In unmittelbarer Folge finden sich ἂν οὐκ ἄν auch Sophokles Oed. Rex 446. Elektra 439. Oed. Col. 1366. Fragm. inc. 673. Eurip. Heraklid. 74. Aristoph. Lysistr. 361 und ἂν οὐδ᾽ ἄν Sophokles Elektra 97 (noch öfter, und selbst bei Aristoteles noch, ἂν — οὐκ ἄν oder οὐδεὶϲ ἄν durch mehrere Wörter getrennt). Da immerhin dem vierten Jahrhundert ἂν οὐκ ἄν fremd und die Wiederholung von ἄν überhaupt nur nach längerem Zwischenraum eigen zu sein scheint, haben vielleicht die Herausgeber recht, die mit den übrigen Handschriften das erste der beiden ἄν streichen und einfach ὅϲτιϲ οὐκ ἄν schreiben.

Durch andere Wörter als ποτε oder οὐ werden ὅϲτιϲ und ἄν in solchen Sätzen bei den guten Attikern nicht getrennt. Freilich Xenophon hat Anabasis 2, 3, 23 οὔτ᾽ ἔϲτιν ὅτου ἕνεκα βουλοίμεθα ἂν τὴν βαϲιλέωϲ χώραν κακῶϲ ποιεῖν. 5, 77 ἔϲτιν \hypertarget{p388}{\emph{[S. 388]}}\label{p388} οὖν \spation{ὅϲτιϲ} τοῦτο \spation{ἂν} δύναιτο ὑμᾶϲ ἐξαπατῆϲαι. Ihm folgt auffälliger Weise Lykurg 39 τίϲ δ᾽ ἦν οὗτω ἢ μιϲόδημοϲ τότε ἢ μιϲαθήναιοϲ, \spation{ὅϲτιϲ} ἐδυνήθη \spation{ἄν}. Ist auch hierauf die Bemerkung von Blass, attische Beredsamkeit 3, 2, 103 anwendbar: “was (bei L.) als unklassisch oder sprachwidrig auffällt, muss auf Rechnung der anerkannt schlechten Überlieferung gesetzt werden?” Aber bei Demosthenes 18, 43 ist in dem Texte von Blass οὐ γὰρ ἦν \spation{ὅ τι} ἄλλ᾽ \spation{ἂν} ἐποιεῖτε das ἄλλο blosse Konjektur des Herausgebers. [Doch Eurip. Med. 1339 οὐκ ἔϲτιν, \spation{ἥτιϲ} τοῦτ᾽ \spation{ἂν} Ἑλληνὶϲ γυνὴ ἔτλη. Lies ἥτιϲ ἂν τόδ᾽?]

Weniger sicher war die Tradition in den Sätzen, wo eines der zu ὅϲτιϲ gehörigen relativen Adjektiva oder Adverbia in solchen Sätzen stand, oder wo zwar ὅϲτιϲ selbst sich an einen negativen Satz anschloss, aber zu dessen Ergänzung nicht unbedingt notwendig und daher nicht so eng mit ihm verbunden war. Zwar haben wir aus erster Kategorie Eurip. Kyklops 469 ἔϲτ᾽ οὖν \spation{ὅπωϲ ἂν} ὡϲπερεὶ ϲπονδῆϲ θεοῦ κἀγὼ λαβοίμην —; (nicht negativer Fragesatz!) Aristoph. Aves 627 οὐκ ἔϲτιν \spation{ὅπωϲ ἂν} ἐγώ ποθ᾽ ἑκὼν τῆϲ ϲῆϲ γνώμηϲ ἔτ᾽ ἀφείμην. Lysias 8, 7 οὐδὲν αὐτὸϲ ἐξηῦρον, \spation{ὁπόθεν ἂν} εἰκότωϲ ὑπερείδετε τὴν ἐμὴν ὁμιλίαν. Plato Sympos. 178~Ε οὐκ ἔϲτιν, \spation{ὅπωϲ ἂν} ἄμεινον οἰκήϲειαν τὴν ἑαυτῶν. 223~Α οὐκ ἔϲθ᾽ \spation{ὅπωϲ ἂν} ἐνθάδε μείναιμι. Xenophon Hellen. 6, 1, 9 οὐκ εἶναι ἔθνοϲ, \spation{ὁποίῳ ἂν} ἀξιώϲειαν ὑπήκοοι εἶναι Θετταλοί. Demosth. 24, 64 ἔϲτιν οὖν \spation{ὅπωϲ ἂν} ἐναντιώτερά τιϲ δύο θείη. (Obwohl der Revisor des Codex S oben an τιϲ ein zweites ἄν eingezeichnet hat, ist doch die von Weil und nach ihm von Blass vorgenommene Streichung des bloss im Augustanus fehlenden ἄν hinter ὅπωϲ und Versetzung desselben hinter ἐναντιώτερα unzulässig.) 18, 165 ἔϲτιν οὖν \spation{ὅπωϲ ἂν} μᾶλλον ἄνθρωποι πάνθ᾽ ὑπὲρ Φιλίππου πράττοντεϲ ἐξελεγχθεῖεν. (Vgl. auch οὐκ οἶδ᾽, \spation{ὅπωϲ ἄν} — oben S.~382.) Zu diesen Beispielen würde nicht in Widerspruch stehen Herodot 8, 119, 9 οὐκ ἔχω \spation{ὅκωϲ} οὐκ \spation{ἂν} ἴϲον πλῆθοϲ τοῖϲ Πέρϲῃϲι ἐξέβαλε, und wohl auch nicht Xenophon Anab. 5, 7, 7 τοῦτ᾽ οὖν ἐϲτιν \spation{ὅπωϲ} τιϲ \spation{ἂν} ὑμᾶϲ ἐξαπατήϲαι; aber wirklich in Widerspruch stehn Sophokles Antigone 1156 οὐκ ἔϲθ᾽ \spation{ὁποῖον} ϲτάντ᾽ \spation{ἂν} ἀνθρώπου βίον οὔτ᾽ αἰνέϲαιμ᾽ \spation{ἂν} οὔτε μεμψαίμην ποτέ. Aristoph. Nubes 1181 οὐ γὰρ ἔϲθ᾽ \spation{ὅπωϲ} μί᾽ ἡμέρα γένοιτ᾽ \spation{ἂν} ἡμέραι δύο. Vesp. 212 κοὐκ ἔϲθ᾽ \spation{ὅπωϲ} — \spation{ἂν} — — λάθοι. Pax 306 οὐ γὰρ ἔϲθ᾽ \spation{ὅπωϲ} \hypertarget{p389}{\emph{[S. 389]}}\label{p389} ἀπειπεῖν \spation{ἂν} δοκῶ μοι τήμερον. [Pl. Apol. 40~C.] Demosth. 15, 18 οὐ γὰρ ἔϲθ᾽ \spation{ὅπωϲ} — εὖνοι γένοιτ᾽ \spation{ἄν}. 19, 308 ἔϲτιν οὖν, \spation{ὅπωϲ} ταῦτ᾽ \spation{ἄν}, ἐκεῖνα προειρηκώϲ, — ἐτόλμηϲεν εἰπεῖν (geringere Handschriften: ὅπωϲ ἂν ταῦτ᾽). — Ähnlich lesen wir zwar Eurip. Alkestis 80 ἀλλ᾽ οὐδὲ φίλων πέλαϲ οὐδείϲ, \spation{ὅϲτιϲ ἂν} εἴποι. Plato Phaedo 57~Β οὔτε τιϲ ξένοϲ ἀφῖκται —, \spation{ὅϲτιϲ ἂν} ἡμῖν ϲαφέϲ τι ἀγγεῖλαι οἷόϲ τ᾽ ἦν περὶ τούτων, aber andrerseits Sophokles Oed. Rex 117 οὐδ᾽ ἄγγελόϲ τιϲ οὐδὲ ϲυμπράκτωρ ὁδοῦ κατεῖδ᾽ \spation{ὅτου} τιϲ ἐκμαθὼν ἐχρήϲατ᾽ \spation{ἄν}.

Eine zweite Gruppe hier in betracht kommender Relativsätze sind die mit ὅπερ eingeleiteten, bei denen ja das -περ begrifflich scharfe Unterordnung unter den Hauptsatz andeutet, also nach dem bei ὅϲτιϲ Beobachteten unmittelbaren Anschluss von ἄν an das Relativum fordern würde. Nun gilt zwar dieser Anschluss bei vollen ὅϲπερ-Sätzen nicht immer, sondern bloss in der Mehrzahl der Beispiele: Herodot 8, 136, 16 κατήλπιζε εὐπετέωϲ τῆϲ θαλάϲϲηϲ κρατήϲειν, \spation{τάπερ ἂν} καὶ ἦν. [Hippokrates] περὶ τέχνηϲ Kap.~5 S.~46, 12 Gomperz τοιαῦτα θεραπεύϲαντεϲ ἑωυτούϲ, \spation{ὁποῖά περ ἂν} ἐθεραπεύθηϲαν. Thucydides 2, 94, 1 ἐνόμιζον — ὅϲον οὐκ ἐϲπλεῖν αὐτούϲ· \spation{ὅπερ ἂν}, εἰ ἐβουλήθηϲαν μὴ κατοκνῆϲαι, ῥᾳδίωϲ ἂν ἐγένετο. Isokrates 8, 133 ἐὰν ϲυμβούλουϲ ποιώμεθα τοιούτουϲ —, \spation{οἵουϲ περ ἂν} περὶ τῶν ἰδίων ἡμῖν εἶναι βουληθεῖμεν. 15, 23 χρὴ τοιούτουϲ εἶναι κριτάϲ —, \spation{οἵων περ ἂν} αὐτοὶ τυγχάνειν ἀξιώϲειαν. 17, 21 ἀξιῶν τὴν αὐτὴν Παϲίωνι — γίγνεϲθαι ζημίαν, \spation{ἧϲπερ ἂν} αὐτὸϲ ἐτύγχανεν. Plato Kriton 52~D πράττειϲ \spation{ἅπερ ἂν} δοῦλοϲ φαυλότατοϲ πράξειεν. Sympos. 217~Β ᾤμην διαλέξεϲθαι αὐτόν μοι, \spation{ἅπερ ἂν} ἐραϲτὴϲ παιδικοῖϲ διαλεχθείη. Xenophon Anab. 5, 4, 34 ἐποίουν \spation{ἅπερ ἂν} ἄνθρωποι ἐν ἐρημίᾳ ποιήϲειαν. Aber mit Trennung des ἄν von ὅϲπερ Thucyd. 1, 33, 3 τὸν δὲ πόλεμον, δι᾽ \spation{ὅνπερ} χρήϲιμοι \spation{ἂν} εἶμεν, εἴ τιϲ ὑμῶν μὴ οἴεται ἔϲεϲθαι. Demosth. 6, 30 Φίλιπποϲ δ᾽ \spation{ἅπερ} εὔξαιϲθ᾽ \spation{ἂν} ὑμεῖϲ, — πράξει. 19, 328 ὑμεῖϲ δ᾽, \spation{ἅπερ} εὔξαιϲθ᾽ \spation{ἄν}, ἐλπίϲαντεϲ —.

Deutlich indessen tritt das Bewusstsein von der engen Zusammengehörigkeit von ἄν mit ὅϲπερ bei Ellipse des Verbums zu Tage, wobei die Ellipse des konjunktivischen Verbums z.~B. Eurip. Medea 1153 φίλουϲ νομίζουϲ᾽ \spation{οὕϲπερ ἂν} πόϲιϲ ϲέθεν. Isokrates 3, 60 φιλεῖν οἴεϲθε δεῖν καὶ τιμᾶν, \spation{οὕϲπερ ἂν} καὶ ὁ βαϲιλεύϲ. Demosth. 18, 280 τὸ τοὺϲ αὐτοὺϲ μιϲεῖν καὶ φιλεῖν, \spation{οὕϲπερ ἂν} ἡ πατρίϲ. CIA. 2, 589, 26 (um \hypertarget{p390}{\emph{[S. 390]}}\label{p390} 300 a.~Ch.) τελεῖν δὲ αὐτὸν τὰ αὐτὰ τέλη ἐν τῷ δήμῳ \spation{ἅπερ ἂγ} καὶ Πειραιεῖϲ verglichen werden kann. Als Beispiele mögen dienen Isokrates 4, 86 τοϲαύτην ποιηϲάμενοι ϲπουδὴν, \spation{ὅϲην περ ἂν} τῆϲ αὑτῶν χώραϲ πορθουμένηϲ. 5, 90 νικῆϲαι — τοϲοῦτον, \spation{ὅϲον περ ἂν} εἰ ταῖϲ γυναιξὶν αὐτῶν ϲυνέβαλον. 10, 49 τοϲοῦτον ἐφρόνηϲαν, \spation{ὅϲον περ ἄν}, εἰ πάντων ἡμῶν ἐκράτηϲαν. 14, 37 \spation{ἅπερ ἂν} εἰϲ τοὺϲ πολεμιωτάτουϲ, ἐξαμαρτεῖν ἐτόλμηϲαν. 15, 28 εἰϲ τὸν αὐτὸν καθέϲτηκα κίνδυνον, εἰϲ \spation{ὅνπερ ἄν}, εἰ πάνταϲ ἐτύγχανον ἠδικηκώϲ. Plato Republ. 2, 368~C δοκεῖ μοι — τοιαύτην ποιήϲαϲθαι ζήτηϲιν αὐτοῦ, \spation{οἵαν περ ἄν}, εἰ προϲέταξέ τιϲ. Xenophon Anab. 5, 4, 34 μόνοι τε ὄντεϲ ὅμοια ἔπραττον, \spation{ἅπερ ἂν} μετ᾽ ἄλλων ὄντεϲ. Demosth. 53, 12 ἀπεκρινάμην αὐτῷ, \spation{ἅπερ ἂν} νέοϲ ἄνθρωποϲ.

Unter den mit blossem ὅϲ eingeleiteten Relativsätzen sind die mit assimiliertem Pronomen am meisten als dem Hauptsatz eng verbunden gekennzeichnet. Dem entspricht, dass die meisten mir zur Hand liegenden Beispiele ἄν hinter ὅϲ haben: Plato Sympos. 218~A ἐγὼ δεδηγμένοϲ τὸ ἀλγεινότατον \spation{ὧν ἄν} τιϲ δηχθείη. Isäus 5, 31 ἐμμενεῖν \spation{οἷϲ ἂν} οὗτοι γνοῖεν. 5, 33 ἐμμενεῖν \spation{οἷϲ ἂν} αὐτοὶ γνοῖεν. Demosth. 18, 16 πρὸϲ ἅπαϲιν τοῖϲ ἄλλοιϲ, \spation{οἷϲ ἂν} εἰπεῖν τιϲ ὑπὲρ Κτηϲιφῶντοϲ ἔχοι. Doch ist die Zahl der Beispiele zu klein, um darauf eine Regel zu gründen, und Dem. 20, 136 μηδὲν \spation{ὧν} ἰδίᾳ φυλάξαιϲθ᾽ \spation{ἄν} widerspricht.

Ganz bunt und regellos scheint der Gebrauch bei den übrigen Relativsätzen. Doch glaube ich sagen zu können, dass die gewöhnlichen Relativsätze ἄν wohl beinahe eben so oft unmittelbar hinter dem Pronomen, als an einer spätern Stelle des Satzes haben. Eine natürliche Folge dieses Schwankens ist die nicht seltene Doppelsetzung von ἄν in Relativsätzen, z.~B. Thucyd. 2, 48, 3 ἀφ᾽ \spation{ὧν ἄν} τιϲ ϲκοπῶν, εἴ ποτε καὶ αὖθιϲ ἐπιπέϲοι, μάλιϲτ᾽ \spation{ἂν} ἔχοι τι προειδὼϲ μὴ ἀγνοεῖν. Demosth. 14, 27 \spation{ὅϲα γὰρ ἂν} νῦν πορίϲαιτ᾽ \spation{ἄν}. [Demosth.] 59, 70 \spation{οὓϲ ἄν} τιϲ δεόμενοϲ — εἴποι \spation{ἄν}. Vgl. das unten zu besprechende doppelte ἄν im Hauptsatz. Daher ist auch an einer Stelle, wie Demosth. Proöm. 1, 3 ἅ δεῖ καὶ δι᾽ \spation{ὧν} παυϲαίμεθ᾽ αἰϲχύνην ὀφλιϲκάνοντεϲ, wo sicher ein ἄν ausgefallen ist, von unserm Standpunkt der Betrachtung aus schlechterdings nicht auszumachen, ob \spation{δι᾽ ὧν} <\spation{ἂν}> παυϲαίμεθ᾽ oder \spation{δι᾽ ὧν} παυϲαίμεθ᾽ <\spation{ἄν}> (so die Herausgeber seit Bekker) zu \hypertarget{p391}{\emph{[S.~391]}}\label{p391} schreiben sei. Wo dagegen das Relativpronomen in der Weise des Latein an Stelle von οὗτοϲ bloss dazu dient eine zweite Hauptaussage an eine erste anzuknüpfen, wo wir also keinen Relativsatz, sondern einen Hauptsatz haben, steht ἄν nie hinter dem Pronomen; vgl. Andocides 1, 67 ἐν \spation{οἷϲ ἐγὼ} — δικαίωϲ \spation{ἂν} ὑπὸ πάντων ἐλεηθείην. Lysias 2, 34 \spation{ὃ τίϲ} ἰδὼν οὐκ \spation{ἂν} ἐφοβήθη; Demosth. 18, 49 ἐξ \spation{ὧν} ϲαφέϲτατ᾽ \spation{ἄν} τιϲ ἴδοι.

Dem entspricht, dass in allen übrigen Nebensätzen, die etwa ἄν c. optat. oder praeterito enthalten, das ἄν zumeist an einer spätern Stelle des Satzes steht, da ja in allen solchen Fällen der Nebensatz nicht als Nebensatz, sondern als Vertreter eines Hauptsatzes den betr. Modus hat. So bei ὡϲ ‘dass’ z.~B. Plato Sympos. 214~D \spation{ὡϲ} ἐγὼ οὐδ᾽ \spation{ἂν} ἕνα ἄλλον ἐπαινέϲαιμι (doch Thucyd. 5, 9, 3 \spation{ὡϲ ἂν} ἐπεξέλθοι τιϲ), ὥϲτε ‘so dass’ z.~B. Plato Sympos. 197~A \spation{ὥϲτε} καὶ οὗτοϲ Ἔρωτοϲ \spation{ἂν} εἴη μαθητήϲ, ὅτι ‘dass, weil’ z.~Β. Plato Phaedo 93~C δῆλον \spation{ὅτι} τοιαῦτ᾽ ἄττ᾽ \spation{ἂν} λέγοι. Sympos. 193~C \spation{ὅτι} οὕτωϲ \spation{ἂν} ἡμῶν τὸ γένοϲ εὔδαιμον γένοιτο. Demosth. 18, 79 \spation{ὅτι} τῶν ἀδικημάτων \spation{ἂν} ἐμέμνητο τῶν αὑτοῦ u. s. w. u. s. w. Ebenso bei ἐπεί ‘denn’ z.~B. Plato Kratyl. 410~A \spation{ἐπεὶ} ἔχοι γ᾽ \spation{ἄν} τιϲ εἰπεῖν περὶ αὐτῶν. Demosth. 18, 49 \spation{ἐπεὶ} διὰ γ᾽ ὑμᾶϲ πάλαι \spation{ἂν} ἀπωλώλειτε. Bei den Zeitpartikeln giebt die Überlieferung zu Zweifeln Anlass: ὅταν c. opt. ist überliefert Aeschyl. Pers. 450, ἕωϲ ἄν c. opt. Isokrat. 17, 15 und Plato Phaedo 101~D. (Sophokles Trach. 687 wird es seit Elmsley nicht mehr geschrieben). Sicher steht Demosth. 4, 31 \spation{ἡνίκ᾽ ἂν} ἡμεῖϲ μὴ δυναίμεθ᾽ ἐκεῖϲ᾽ ἀφικέϲθαι. — Xenophon Hellen. 2, 3, 48 \spation{πρὶν ἂν} μετέχοιεν. ibid. \spation{πρὶν ἂν} — καταϲτήϲειαν. 2, 4, 18 \spation{πρὶν ἂν} ἢ πέϲοι τιϲ ἢ τρωθείη wird ἄν gestrichen.

Von der Konjunktion ausnahmslos getrennt ist ἄν in optativischen εἰ-Sätzen: εἰ ‘ob’ z.~B. Plato Sympos. 210~B οὐκ οἶδ᾽ \spation{εἰ} οἷόϲ τ᾽ \spation{ἂν} εἴηϲ, εἰ ‘wenn’ z.~B. Eurip. Helena 825 \spation{εἰ} πῶϲ \spation{ἂν} ἀναπείϲαιμεν ἱκετεύοντέ νιν. Demosth. 4, 18 οὐδ᾽ \spation{εἰ} μὴ ποιήϲαιτ᾽ \spation{ἂν} ἤδη. 20, 62 οὐκοῦν αἰϲχρόν, \spation{εἰ} μέλλοντεϲ μὲν εὖ πάϲχειν ϲυκοφάντην \spation{ἂν} τὸν ταῦτα λέγονθ᾽ ἡγοῖϲθε, ἐπὶ τῷ δ᾽ ἀφελέϲθαι — ἀκούϲεϲθε. 19, 172 ἐξώληϲ ἀπολοίμην —, \spation{εἰ} προϲλαβών γ᾽ \spation{ἂν} ἀργύριον — ἐπρέϲβευϲα. Hier überall ist der durch ἄν angegebene hypothetische Charakter des Satzes nicht durch εἰ bedingt; vgl. die Erklärer zu den einzelnen Stellen.

\hypertarget{p392}{\emph{[S.~392]}}\label{p392} Besonders bezeichnend sind aber die Fälle, wo nach Ausdrücken des Befürchtens und Erwartens μή mit dem Optativ und ἄν steht: Sophokles Trachin. 631 δέδοικα γάρ, \spation{μὴ} πρῲ λέγοιϲ \spation{ἂν} τὸν πόθον. Thucyd. 2, 93, 3 οὔτε προϲδοκία οὐδεμία ἦν, \spation{μὴ ἄν} ποτε οἱ πολέμιοι ἐξαπιναίωϲ οὕτωϲ ἐπιπλεύϲειαν. Xenophon Anab. 6, 1, 28 ἐκεῖνο ἐννοῶ, \spation{μὴ} λίαν \spation{ἂν} ταχὺ ϲωφρονιϲθείην. Poroi 4, 41 φο\-βοῦν\-ται, \spation{μὴ} ματαία \spation{ἂν} γένοιτο αὕτη ἡ παραϲκευή. Hier ist es ausser allem Zweifel, dass der Optativ mit ἄν auf einer Beeinflussung des μή-Satzes durch den Hauptsatz beruht, und da hat unter vier Beispielen nur eines ἄν unmittelbar hinter μή.

Und hieraus wird es nun auch klar, warum die Stellung des ἄν in Konjunktivsätzen so ganz fest, in den andern Nebensätzen schwankend ist. In der klassischen Gräzität kommt ἄν cum conj. nur in Nebensätzen vor; was hätte also dieses ἄν aus seiner traditionellen Stellung bringen sollen? Dagegen ἄν c. indic. und c. opt. ist nicht bloss häufiger in den Haupt- als in den Nebensätzen, sondern auch in den letztern vielfach geradezu aus den Hauptsätzen übertragen. Notwendig mussten sich die Stellungsgewohnheiten, die ἄν im Hauptsatz hat, auf die betr. Nebensätze übertragen.

\section*{VII.}
\addcontentsline{toc}{section}{VII.}

Wie verhält es sich nun aber mit dieser freien Stellung von ἄν im Hauptsatz? Es ist unbestreitbar, dass in diesem das ἄν sehr weit vom Anfang entfernt stehen kann. Eine Grenze nach hinten bildet bloss das letzte im betr. Satz stehende und durch ἄν irgendwie qualifizierte Verbum finitum oder infinitum, wobei ich besonders darauf hinweise, dass Partizipien, die mit hypothetischen Nebensätzen gleichwertig sind, gern ἄν  hinter sich haben (vgl. z.~B. Aristoph. Ranae 96 γόνιμον δὲ ποιητὴν \spation{ἂν} οὐχ εὕροιϲ ἔτι ζητῶν \spation{ἄν}). Auf dieses Verbum darf ἄν nur in der Weise folgen, dass es sich ihm unmittelbar anschliesst. Doch finden sich Stellen, wo γ᾽ oder ein einsilbiges Enklitikon oder sonst ein Monosyllabon zwischen dem Verbum und ἄν steht: γ᾽: Plato Kratyl. 410~A ἐπεὶ \spation{ἔχοι γ᾽ ἄν} τιϲ εἰπεῖν περὶ αὐτῶν. — τιϲ: [Eur. Or. 694.] Demosth. 18, 282 τί δὲ μεῖζον \spation{ἔχοι τιϲ ἂν} εἰπεῖν. 18, 316 οὐ μὲν οὖν \spation{εἴποι τιϲ ἂν} ἡλίκαϲ. — ποτ᾽: Eurip. Helena 912~f. κεῖνοϲ δὲ πῶϲ τὰ ζῶντα τοῖϲ θανοῦϲιν \spation{ἀπο-}\hypertarget{p393}{\emph{[S.~393]}}\label{p393}\spation{δοίη} ποτ᾽ \spation{ἄν}. — οὐ: Sophokles Aias 1330 ἦ γὰρ \spation{εἴην} οὐκ \spation{ἂν} εὖ φρονῶν. — τάχ᾽: Oed. Rex 1115~f. τῇ δ᾽ ἐπιϲτήμῃ ϲύ μου \spation{προύχοιϲ} τάχ᾽ \spation{ἄν} που. — τάδ᾽: Eurip. Helena 97 τίϲ ϲωφρονῶν \spation{τλαίη} τάδ᾽ \spation{ἄν}. — ταῦτ᾽: Solon Fragm. 36, 1 \spation{ϲυμμαρτυροίη} ταῦτ᾽ \spation{ἂν} ἐν δίκῃ. — μεντ᾽: Aristoph. Ran. 743 \spation{ᾤμωξε} μέντ᾽ \spation{ἄν}. Plato Phaedo 76~Β \spation{βουλοίμην} μέντ᾽ \spation{ἄν}. Apol. 30~D. Doch lassen die drei letzten Stellen (Solon, Ar. Ran. 743, Pl. Phaedo 76~B) auch noch eine andere Erklärung zu. Wenn nämlich das Verbum am Anfang des Satzes steht, scheint jene obige Regel überhaupt nicht zu gelten: Sophokles Oed. Col. 125 \spation{προϲέβα} γὰρ οὐκ \spation{ἂν} ἀϲτιβὲϲ ἄλϲοϲ ἔϲ. Eurip. Hiketiden 944 \spation{ὄλοιντ᾽} ἰδοῦϲαι τοῦϲδ᾽ \spation{ἄν}. Demosth. 20, 61 \spation{μάθοιτε} δὲ τοῦτο μάλιϲτ᾽ \spation{ἄν}. Übrigens versteht es sich von selbst, dass wenn ein Satz mehrere ἄν enthält, die Regel für das letzte ἄν gilt. Sophokles Oed. Rex 1438 \spation{ἔδραϲ᾽} ἂν (εὖ τόδ᾽ ἴϲθ᾽) \spation{ἄν}. Elektra 697 \spation{δύναιτ᾽} ἂν ούδ᾽ \spation{ἂν} ἰϲχύων φυγεῖν. Aristoph. Nubes 977 \spation{ἠλείψατο} δ᾽ ἂν τοὐμφαλοῦ οὐδεὶϲ παῖϲ ὑπένερθεν τότ᾽ \spation{ἄν} ist die Entfernung des zweiten ἄν vom Verbum aus der Anfangsstellung des Verbums zu erklären. — Sonach haben die Herausgeber von Aristoph. Rittern Recht gehabt, wenn sie Vs. 707 das überlieferte ἐπὶ τῷ \spation{φάγοιϲ} ἥδιϲτ᾽ \spation{ἄν} in ἐπὶ τῷ φαγὼν ἥδοιτ᾽ (oder ἥδοι᾽) ἄν ändern; dagegen Aristophanes Ran. 949~f. οὐδὲν παρῆκ᾽ ἂν ἀργόν, ἀλλ᾽ \spation{ἔλεγεν} ἡ γυνή τέ μοι χὠ δοῦλοϲ οὐδὲν ἧττον χὠ δεϲπότηϲ χἠ παρθένοϲ χἠ γραῦϲ \spation{ἄν} bildet nur eine scheinbare Ausnahme, da bei jedem der aneinandergereihten Nominative ἔλεγεν hinzuzudenken ist. Vgl. Soph. Phil. 292 πρὸϲ τοῦτ᾽ ἄν. [Eurip. Or. 941 κοὐ φθάνοι θνήϲκων τιϲ ἄν.]

Aus dieser Regel lässt sich aber schon erkennen, was für Tendenzen dazu geführt haben, das ἄν des selbständigen Satzes in nachhomerischer Zeit von der Stelle wegzuziehen, die es in homerischer Zeit noch einnahm. Das Verb, dessen Modalität durch ἄν bestimmt wird, zog es an sich, daneben die Negationen, die Adverbia, besonders die superlativischen, überhaupt derjenige Satzteil, für den der durch ἄν angezeigte, hypothetische Charakter des Satzes am meisten in betracht kam, gerade wie die enklitischen Pronomina ihrer traditionellen Stellung dadurch verlustig gingen, dass das Bedürfnis immer stärker wurde, ihnen den Platz zu geben, den ihre Funktion im Satze zu fordern schien. Wie aber bei den en-\hypertarget{p394}{\emph{[S.~394]}}\label{p394}klitischen Pronomina, so hat auch bei ἄν die Tradition immer einen gewissen Einfluss bewahrt.

Erstens lässt sich auch bei ἄν die Neigung für Anlehnung an satzbeginnende Wörter nachweisen. So unbestreitbar an τίϲ und die zugehörigen Formen, besonders πῶϲ (Vgl. Jebb zu Sophokles Oed. Col. 1100, der auf Aeschyl. Agam. 1402 \spation{τίϲ ἂν} ἐν τάχει μὴ περιώδυνοϲ μὴ δεμνιοτήρηϲ μόλοι verweist. Vgl. Θ~77. Ω~367. θ~208. κ~573). Ferner ist hiefür die Beobachtung Werfers Acta philologorum Monacensium I 246~ff., zu verwerten, dass sich ἄν “paene innumeris locis” an γάρ anschliesse. Die Fülle der Beispiele verbietet eine Wiederholung und Ergänzung von Werfers Beispielsammlung an dieser Stelle. Ich will nur bemerken, erstens, dass zwar aus allen Litteraturgattungen Gegenbeispiele beigebracht werden können, aber doch γὰρ ἄν unendlich häufiger ist als γὰρ — ἄν, und zweitens, dass infolge der Setzung von ἄν gleich hinter γάρ sehr oft das Bedürfnis empfunden wird, in einem spätern Teil des Satzes ἄν nochmals einzufügen: Sophokles Oed. Rex 772 τῷ \spation{γὰρ ὰν} καὶ μείζονι λέξαιμ᾽ \spation{ἂν} ἢ ϲοί. 862 οὐδὲν \spation{γὰρ ἂν} πράξαιμ᾽ \spation{ἄν}. Fragm. 513 Nauck\textsuperscript{2}, 6 κἀμοὶ \spation{γὰρ ἂν} πατήρ γε δακρύων χάριν ἀνῆκτ᾽ \spation{ἂν} εἰϲ φῶϲ. Fragm. 833 ἀλλ᾽ οὐ \spation{γὰρ ἂν} τὰ θεῖα κρυπτόντων θεῶν μάθοιϲ \spation{ἄν}. Eurip. Hiket. 855 μόλιϲ \spation{γὰρ ἄν} τιϲ αὐτὰ τἀναγκαῖ᾽ ὁρᾶν δύναιτ᾽ \spation{ἂν} ἑϲτὼϲ πολεμίοιϲ ἐναντίοϲ. Helena 948 τὴν Τροίαν \spation{γὰρ ἂν} δειλοὶ γενόμενοι πλεῖϲτον αἰϲχύνοιμεν \spation{ἄν}. 1011 καὶ \spation{γὰρ ἂν} κεῖνοϲ βλέπων ἀπέδωκεν \spation{ἄν} ϲοι τῆνδ᾽ ἔχειν. 1298 εὐμενέϲτερον \spation{γὰρ ἂν} τῷ φιλτάτῳ μοι Μενέλεῳ τὰ πρόϲφορα δρῴηϲ \spation{ἄν}. Aristoph. Vesp. 927 οὐ \spation{γὰρ ἄν} ποτε τρέφειν δύναιτ᾽ \spation{ἂν} μία λόχμη κλέπτα δύο. Pax 321 οὐ \spation{γὰρ ἂν} χαίροντεϲ ἡμεῖϲ τήμερον παυϲαίμεθ᾽ \spation{ἄν}. Lysistr. 252 ἄλλωϲ \spation{γὰρ ἂν} ἄμαχοι γυναῖκεϲ καὶ μιαραὶ κεκλῄμεθ᾽ \spation{ἄν}. Thesmoph. 196 καὶ \spation{γὰρ ἂν} μαινοίμεθ᾽ \spation{ἄν}. Plato Apol. 35~D ϲαφῶϲ \spation{γὰρ ἄν}, εἰ πείθοιμι ὑμᾶϲ —, θεοὺϲ \spation{ἂν} διδάϲκοιμι. 40~D ἐγὼ \spation{γὰρ ἂν} οἶμαι, εἰ — δέοι —, οἶμαι \spation{ἂν} — τὸν μέγαν βαϲιλέα εὐαριθμήτουϲ \spation{ἂν} εὑρεῖν. (Vgl. Demosth. 14, 27 ὅϲα \spation{γὰρ ἂν} νῦν πορίϲαιτ᾽ \spation{ἄν}). Aristot. de caelo 227\textsuperscript{b} 24 οὔτε \spation{γὰρ ἂν} αἱ τῆϲ ϲελήνηϲ ἐκλείψειϲ τοιαύταϲ \spation{ἂν} εἶχον τὰϲ ἀποτομάϲ. De gener. et corr. 337\textsuperscript{b} 7 μέλλων \spation{γὰρ ἂν} βαδίζειν τιϲ οὐκ \spation{ἂν} βαδίϲειεν. De part. anim. 654\textsuperscript{a} 18 οὕτωϲ \spation{γὰρ ἂν} ἔχον χρηϲιμώτατον \spation{ἂν} εἴη. (vgl. Vahlen Zur Poetik 1460\textsuperscript{b} 7) u. s. w.

\hypertarget{p395}{\emph{[S.~395]}}\label{p395} Sodann ist darauf hinzuweisen, dass die Verbindungen κἄν aus καὶ ἄν ‘auch wohl’ und τάχ᾽ ἄν, in denen ἄν mit seinem Vorworte bis zur völligen Verblassung seiner eigenen Bedeutung verschmolzen ist, in der Mehrzahl der Fälle am Satzanfang stehen. Doch dürfen wir hierauf kein Gewicht legen, da gerade καὶ ἄν und τάχ᾽ ἄν sich schon bei Homer im Innern von Sätzen finden und überhaupt kein Grund vorhanden ist, den engen Anschluss von ἄν an καί und τάχα aus den Fällen herzuleiten, wo καί und τάχα den Satz beginnen. (καί ‘und’ hat ἄν unmittelbar hinter sich Herodot 4, 118, 21 \spation{καὶ ἂν} ἐδήλου).

Zweitens findet man ἂν [sic] vereinzelt wie die Enklitika hinter einem Vokativ: Aristoph. Pax 137 ἀλλ᾽ ὦ μέλ᾽ \spation{ἄν} μοι ϲιτίων διπλῶν ἔδει.

Drittens verdrängt es öfters οὖν, seltener τε, δέ von ihrem Platze: Herodot 7, 150, 8 οὕτω \spation{ἂν ὦν} εἶμεν. [Eur. Med. 504.] Ar. Lysistr. 191 τίϲ \spation{ἂν οὖν} γένοιτ᾽ \spation{ἂν} ὄρκοϲ. [Lysias] 20, 15 πῶϲ \spation{ἂν οὖν} οὐκ \spation{ἂν} δεινὰ πάϲχοιμεν. Plato Phaedo 64~Α πῶϲ \spation{ἂν οὖν} δὴ τοῦθ᾽ οὕτωϲ ἔχοι —, ἐγὼ πειράϲομαι φράϲαι. Sympos. 202~D πῶϲ \spation{ἂν οὖν} θεὸϲ εἴη ὅ γε τῶν καλῶν καὶ ἀγαθῶν ἄμοιροϲ, und öfters. Xen. Anab. 2, 5, 20 πῶϲ \spation{ἂν οὖν} ἔχοντεϲ τοϲούτοϲ πόρουϲ — ἔπειτα ἐκ τούτων πάντων τοῦτον \spation{ἂν} τὸν τρόπον ἐξελοίμεθα —; 5, 7, 8 πῶϲ \spation{ἂν οὖν} ἐγὼ ἤ [sic] βιαϲαίμην ὑμᾶϲ — ἢ ἐξαπατήϲαϲ ἄγοιμι. 5, 7, 9 πῶϲ \spation{ἂν οὖν} ἀνὴρ μᾶλλον δοίη δίκην. Respubl. Lacedaem. 5, 9 οὐκ \spation{ἂν οὖν} ῥᾳδίωϲ γέ τιϲ εὕροι Σπαρτιατῶν ὑγιεινοτέρουϲ. Demosth. 25, 33 τίϲ \spation{ἂν οὖν} εὖ φρονῶν αὑτὸν \spation{ἂν} ἢ τὰ τῆϲ πατρίδοϲ ϲυμφέροντα ταύτῃ ϲυνάψειε. [Demosth.] 46, 13 πῶϲ \spation{ἂν οὖν} μὴ εἰδὼϲ ὁ πατὴρ αὐτὸν Ἀθηναῖον ἐϲόμενον ἔδωκεν \spation{ἂν} τὴν ἑαυτοῦ γυναῖκα. Aeschines 1, 17 ἴϲωϲ \spation{ἂν οὖν} τιϲ θαυμάϲειεν. 3, 219 πῶϲ \spation{ἂν οὖν} ἐγὼ προεδεικνύμην Ἀλεξάνδρῳ. Dass in der Mehrzahl der Beispiele das dem οὖν vorausgeschickte ἄν sich an τίϲ oder πῶϲ anlehnt, passt zu dem oben S.~394 bemerkten. (Dass ἄν dem οὖν häufiger noch folgt, soll nicht geleugnet werden.) — Einem τε geht ἄν voraus Thucyd. 2, 63, 3 τάχιϲτ᾽ \spation{ἄν τε} πόλιν οἱ τοιοῦτοι ἀπολέϲειαν, einem δέ Thucyd. 6, 2, 4 τάχ᾽ \spation{ἂν δὲ} καὶ ἄλλωϲ ἐϲπλεύϲαντεϲ und vielleicht 6, 10, 4 ταχ᾽ \spation{ἂν δ᾽} ἴϲωϲ (die Mehrzahl der Handschr. und die Ausgaben τάχα δ᾽ ἂν ἴϲωϲ). Doch ist bei den beiden letzten Stellen der Zu-\hypertarget{p396}{\emph{[S.~396]}}\label{p396}sammenschluss mit τάχα für ἄν von wesentlicherer Bedeutung, als die Stellung an sich.

Viertens lässt sich ἄν gern durch einen Zwischensatz von den Hauptbestandteilen des Satzes, zu dem es gehört, trennen: Aristoph. Ran. 1222 οὐδ᾽ \spation{ἄν}, μὰ τὴν Δήμητρα, φροντίϲαιμί γε. Plato Phaedo 102~A ϲὺ δ᾽ — οἶμαι, \spation{ἄν}, ὡϲ ἐγὼ λέγω, ποιοίηϲ. Sympos. 202~D τί οὖν \spation{ἄν}, ἔφη, εἴη ὁ Ἔρωϲ. 202~Β καὶ πῶϲ \spation{ἄν}, ἔφη, ὦ Σώκρατεϲ, ὁμολογοῖτο. Republ. 1, 333~Α πρόϲ γε ὑποδημάτων \spation{ἄν}, οἶμαι, φαίηϲ κτῆϲιν. 4, 438~A ἴϲωϲ γὰρ \spation{ἄν}, ἔφη, δοκοίη τι λέγειν ὁ ταῦτα λέγων. Leges 2, 658~Α τί \spation{ἄν}, εἰ — (folgen sieben Zeilen), τί ποτ᾽ \spation{ἂν} ἡγούμεθα ἐκ ταύτηϲ τῆϲ προρρήϲεωϲ ξυμβαίνειν. Xenophon Hellen. 6, 1, 9 οἶμαι \spation{ἄν}, αὐτῶν εἰ καλῶϲ τιϲ ἐπιμελοῖτο, οὐκ εἶναι ἔθνοϲ. Cyrop. 2, 1, 5 ἐγὼ \spation{ἄν}, εἰ ἔχοιμι, ὡϲ τάχιϲτα ὅπλα ἐποιούμην τοῖϲ Πέρϲαιϲ. Demosth. 18, 195 τί \spation{ἄν}, εἴ που τῆϲ χώραϲ τοῦτο πάθοϲ ϲυνέβη, προϲδοκῆϲαι χρῆν.

Dass man dann gern nach dem Zwischensatz ἄν wiederholte, ist verständlich: Sophokles Antig. 69 οὔτ᾽ \spation{ἄν}, εἰ θέλοιϲ ἔτι πράϲϲειν, ἐμοῦ γ᾽ \spation{ἂν} ἡδέωϲ πράϲϲοιϲ μέτα. 466 ἀλλ᾽ \spation{ἄν}, εἰ τὸν ἐξ ἐμῆϲ μητρὸϲ θανόντ᾽ ἄθαπτον ἠνϲχόμην νέκυν, κείνοιϲ \spation{ἂν} ἤλγουν. Oed. Rex 1438 ἔδραϲ᾽ \spation{ἄν}, εὖ τόδ᾽ ἴϲθ᾽, \spation{ἄν}, εἰ μὴ — ἔχρῃζον. Elektra 333 ὥϲτ᾽ \spation{ἄν}, εἰ ϲθένοϲ λάβοιμι, δηλώϲαιμ᾽ \spation{ἄν}. 439 ἀρχὴν δ᾽ \spation{ἄν}, εἰ μὴ τλημονεϲτάτη γυνὴ παϲῶν ἔβλαϲτε, — χοὰϲ οὐκ \spation{ἄν} ποθ᾽ ὃν γ᾽ ἔκτεινε, τῷδ᾽ ἐπέϲτεφε. Thucyd. 1, 136, 5 ἐκεῖνον δ᾽ \spation{ἄν}, εἰ ἐκδοίη αὐτόν —, ϲωτηρίαϲ \spation{ἂν} τῆϲ ψυχῆϲ ἀποϲτερῆϲαι. Aristoph. Lysistr. 572 \spation{κἄν}, ὑμῖν εἴ τιϲ ἐνῆν νοῦϲ, ἐκ τῶν ἐρίων τῶν ἡμετέρων ἐπολιτεύεϲθ᾽ \spation{ἂν} ἅπαντα. Ranae 585 \spation{κἄν}, εἴ με τύπτοιϲ, οὐκ \spation{ἂν} ἀντείποιμί ϲοι. Plato Protag. 318~C \spation{κἄν}, εἰ Ὀρθαγόρᾳ τῷ Θηβαίῳ ϲυγ\-γε\-νό\-μενοϲ — ἐπανέροιτο αὐτόν —, εἴποι \spation{ἄν}. Leges 8, 841~C τάχα δ᾽ \spation{ἄν}, εἰ θεὸϲ ἐθέλοι, \spation{κἂν} δυοῖν θάτερα βιαϲαίμεθα περὶ ἐρωτικῶν. Demosth. 4, 1 ἐπιϲχὼν \spation{ἄν}, ἕωϲ —, εἰ —, ἡϲυχίαν \spation{ἂν} ἦγον. 21, 115 ἆρ᾽ \spation{ἄν}, εἴ γ᾽ εἶχε —, ταῦτ᾽ \spation{ἂν} εἴαϲεν. 37, 16 οὐδ᾽ \spation{ἄν}, εἴ τι γένοιτ᾽, ᾠήθην \spation{ἂν} δίκην μοι λαχεῖν ποτε τοῦτον. [Demosth.] 47, 66 καίτοι πῶϲ \spation{ἄν}, εἰ μὴ πεποριϲμένον τε ἦν —, εὐθὺϲ \spation{ἂν} ἀπέλαβον. Aeschines 1, 122 οἶμαι δ᾽ \spation{ἄν}, εἰ —, ταῖϲ ὑμετέραιϲ μαρτυρίαιϲ ῥᾳδίωϲ \spation{ἂν} ἀπολύϲαϲθαι τοὺϲ τοῦ κατηγόρου λόγουϲ. [Hen. [sic] Anabasis 7, 7, 38.]

Das Umgekehrte, wenn man will, aber doch etwas aus derselben Stellungsregel entspringendes liegt vor, wenn ein \hypertarget{p397}{\emph{[S.~397]}}\label{p397} syntaktisch zu einem Zwischensatz oder zu einem abhängigen Satz gehöriges ἄν hinter das erste Wort des übergeordneten Satzes gezogen wird: Plato Kriton 52~D ἄλλο τι οὖν, \spation{ἂν} φαῖεν, ἢ ξυνθήκαϲ τὰϲ πρὸϲ ἡμᾶϲ αὐτοὺϲ — παραβαίνειϲ. Phaedo 87~Α τί οὖν, \spation{ἂν} φαίη ὁ λόγοϲ, ἔτι ἀπιϲτεῖϲ. Hippias major 299~A μανθάνω, \spation{ἂν} ἴϲοϲ φαίη, καὶ ἐγώ. Demosth. 1, 14 τί οὖν, \spation{ἄν} τιϲ εἴποι, ταῦτα λέγειϲ. 1, 19 τί οὖν, \spation{ἄν} τιϲ εἴποι, ϲὺ γράφειϲ ταῦτ᾽ εἶναι ϲτρατιωτικά. Proöm. 35, 4 τί οὖν, \spation{ἄν} τιϲ εἴποι, ϲὺ παραινεῖϲ; [Demosth.] 45, 55 ὅτι νὴ Δί᾽, \spation{ἂν} εἴποι, τοῦτον εἰϲπεποίηκα υἱόν. — Vgl. auch Demosth. 11, 44 οὐκ \spation{ἂν} οἶδ᾽ ὅ τι πλέον εὕροι τούτου. Plato Timäus 26~Β ἐγὼ γάρ, ἃ μὲν χθὲϲ ἤκουϲα, οὐκ \spation{ἂν} οἶδ᾽ εἰ δυναίμην ἅπαντα ἐν μνήμῃ πάλιν λαβεῖν. Ähnliches οὐκ \spation{ἂν} οἶδ᾽ ὅ τι im Satzinnern Demosth. 45, 7. Auf dergleichen Wendungen basiert dann wohl wiederum das euripideische οὐκ (bezw. οὐ γὰρ) οἶδ᾽ \spation{ἂν} εἰ πείϲαιμι Medea 941. Alcestis 48. Eigentümlich Thucyd. 5, 9, 3 καὶ οὐκ \spation{ἂν} ἐλπίϲαντεϲ ὡϲ \spation{ἂν} ἐπεξέλθοι τιϲ, wo das erste ἄν nur als Antizipation aus dem Nebensatz erklärt werden kann.

Sechstens sprengt ἄν, gerade wie die Enklitika, öfters am Satzanfang stehende Wortgruppen auseinander. Dahin könnte man οὐδ᾽ ἂν εἷϲ stellen: Sophokles Oed. Rex 281 \spation{οὐδ᾽ ἂν εἷϲ} δύναιτ᾽ ἀνήρ. Oed. Col. 1656 \spation{οὐδ᾽ ἂν εἷϲ} θνητῶν φράϲειε. Plato Prot. 328~Α \spation{οὐδ᾽ ἂν εἷϲ} φανείη. Alcib. 122~D \spation{οὐδ᾽ ἂν εἷϲ} ἀμφιϲβητήϲειε. Demosth. 19, 312 \spation{οὐδ᾽ ἂν εἷϲ} εὖ οἶδ᾽ ὅτι φήϲειεν. 18, 69 \spation{οὐδ᾽ ἂν εἷϲ} ταῦτα φήϲειεν. 18, 94 \spation{οὐδ᾽ ἂν εἷϲ} εἰπεῖν ἔχοι. Aristot. Ἀθην. πολ. 21, 2~Κ. \spation{οὐδ᾽ ἂν εἷϲ} εἴποι. Doch findet sich diese Tmesis wenigstens ebenso häufig im Satzinnern: Lys. 19, 60. 24, 24. Isokr. 15, 223. 21, 20. Plato Sympos. 192~E, 214~D, 216~E. Gorg. 512~E. 519~C. Demosth. 14, 1. 20, 136. 18, 68. 18, 128. Lykurg 49. 57, und scheint somit wesentlich auf der Attraktionskraft des οὐδέ zu beruhen.

Einen bessern Beweis bildet das zweimalige γ᾽ ἂν οὖν statt γοῦν ἄν bei Thucydides: 1, 76, 4 ἄλλουϲ γ᾽ \spation{ἂν} οὖν οἰόμεθα τὰ ἡμέτερα λαβόντεϲ δεῖξαι \spation{ἄν} und 1, 77, 6 ὑμεῖϲ γ᾽ \spation{ἂν} οὖν, εἰ — ἄρξαιτε, τάχ᾽ \spation{ἂν} μεταβάλοιτε, sowie folgende Fälle, wo ἄν mitten in eine Wortgruppe eingedrungen ist: Solon fragm. 37, 4 πολλῶν \spation{ἂν} ἀνδρῶν ἧδ᾽ ἐχηρώθη πόλιϲ. Aeschyl. Pers. 632 μόνοϲ \spation{ἂν} θνητῶν πέραϲ εἴποι. 706 ἀνθρώπεια δ᾽ \spation{ἄν} τοι πήματ᾽ ἂν τύχοι βροτοῖϲ. Sophokles Aias 155 κατὰ δ᾽ \spation{ἄν} τιϲ \hypertarget{p398}{\emph{[S.~398]}}\label{p398} ἐμοῦ τοιαῦτα λέγων οὐκ \spation{ἂν} πείθοι. Oed. Rex 175 ἄλλον δ᾽ \spation{ἂν} ἄλλῳ προϲίδοιϲ. 502 ϲοφίᾳ δ᾽ \spation{ἂν} ϲοφίαν παραμείψειεν ἀνήρ. Elektra 1103 τίϲ οὖν \spation{ἂν} ὑμῶν τοῖϲ ἔϲω φράϲειεν \spation{ἄν}. Oed. Col. 1100 τίϲ \spation{ἂν} θεῶν ϲοι τόνδ᾽ ἄριϲτον ἄνδρ᾽ ἰδεῖν δοίη. Herodot 1, 56, 5 ἐφρόντιζε ἱϲτορέων, τοὺϲ \spation{ἂν} Ἑλλήνων δυνατωτάτουϲ ἐόνταϲ προϲκτήϲαιτο φίλουϲ. 1, 67, 7 ἐπειρώτεον, τίνα \spation{ἂν} θεῶν ἱλαϲάμενοι κατύπερθε τῷ πολέμῳ Τεγεητέων γενοίατο. 1, 196, 22 τὸ δὲ \spation{ἂν} χρυϲίον ἐγίνετο ἀπὸ τῶν εὐειδέων παρθένων. 7, 48, 8 ϲτρατοῦ \spation{ἂν} ἄλλου τιϲ τὴν ταχίϲτην ἄγερϲιν ποιέοιτο. 7, 135, 12 ἕκαϲτοϲ \spation{ἂν} ὑμῶν ἄρχοι γῆϲ Ἑλλάδοϲ. 7, 139, 9 κατά γε \spation{ἂν} τὴν ἤπειρον τοιάδε ἐγίνετο. [Hippokrates] περὶ τέχνηϲ c. 3 (s.~44, 8 Gomperz) ἐν ἄλλοιϲιν \spation{ἂν} λόγοιϲιν ϲαφέϲτερον διδαχθείη. (Vgl. auch c. 2, s.~42, 19~G. ἐπεὶ τῶν γε μὴ ἐόντων τίνα \spation{ἄν} τιϲ οὐϲίην θεηϲάμενοϲ ἀπαγγείλειεν ὡϲ ἔϲτιν). Thucyd. 1, 10 πολλὴν \spation{ἂν} οἶμαι ἀπιϲτίαν τῆϲ δυνάμεωϲ — τοῖϲ ἔπειτα πρὸϲ τὸ κλέοϲ αὐτῶν εἶναι. 1, 36, 3 βραχυτάτῳ δ᾽ \spation{ἂν} κεφαλαίῳ τῷδ᾽ \spation{ἂν} μὴ προέϲθαι ἡμᾶϲ μάθοιτε. 5, 22, 2 πρὸϲ γὰρ \spation{ἂν} τοὺϲ Ἀθηναίουϲ, εἰ ἐξῆν χωρεῖν. Aristoph. Thesmoph. 768 τίν᾽ οὖν \spation{ἂν} ἄγγελον πέμψαιμ᾽ ἐπ᾽ αὐτόν. Isokrates 5, 35 ϲκεπτέον, τί \spation{ἂν} ἀγαθὸν αὐτὰϲ ἐργαϲάμενοϲ φανείηϲ ἄξια — πεποιηκώϲ. Plato Apologie 25~B πολλή [sic] γὰρ \spation{ἄν} τιϲ εὐδαιμονία εἴη περὶ τοὺϲ νέουϲ. Phaedo 70~Α πολλὴ \spation{ἂν} ἐλπὶϲ εἴη καὶ καλὴ. 70~D 106~D ἄλλου {ἄν} του δέοι λόγου. 107~C οὐδεμία {ἂν} εἴη ἄλλη ἀποφυγή. Xenophon Anab. 3, 1, 6 ἐλθὼν δ᾽ ὁ Ξενοφῶν ἐπήρετο τὸν Ἀπόλλω, τίνι \spation{ἂν} θεῶν θύων καὶ εὐχόμενοϲ κάλλιϲτα καὶ ἄριϲτα ἔλθοι τὴν ὁδόν, ἣν ἐπινοεῖ, καὶ καλῶϲ πράξαϲ ϲωθείη, was sofort an das τίνι \spation{κα} θεῶν u.~s.~w. der dodonäischen Orakeltäfelchen (siehe oben S.~374) erinnert. Vgl. auch das Orakel bei [Demosth.] 43, 66 ἐπερωτᾷ ὁ δῆμοϲ, ὅ τι \spation{ἂν} δρῶϲιν — εἵη, und Herodot 1, 67, 7 oben. — Anab. 3, 2, 29 λαβόντεϲ δὲ τοὺϲ ἄρχονταϲ, ἀναρχίᾳ \spation{ἂν} καὶ ἀταξίᾳ ἐνόμιζον ἡμᾶϲ ἀπολέϲθαι. Poroi 3, 14 πολλὴ \spation{ἂν} καὶ ἀπὸ τούτων πρόϲοδοϲ γίγνοιτο. 4, 1 πάμπολλα \spation{ἂν} νομίζω χρήματα — προϲιέναι. Demosth. 1, 1 ἀντὶ πολλῶν \spation{ἄν}, ὦ ἄνδρεϲ Ἀθηναῖοι, χρημάτων ὑμᾶϲ ἑλέϲθαι νομίζω. 4, 12 πληϲίον μὲν ὄντεϲ, ἅπαϲιν \spation{ἂν} τοῖϲ πράγμαϲιν τεταραγμένοιϲ ἐπιϲτάντεϲ ὅπωϲ βούλεϲθε διοικήϲαιϲθε. 19, 48 τί \spation{ἂν} ποιῶν ὑμῖν χαρίϲαιτο. 18, 22 τί \spation{ἂν} εἰπών ϲέ τιϲ ὀρθῶϲ προϲείποι; (18, 81 ὅτι πολλὰ μὲν \spation{ἂν} χρήματα ἔδωκε Φιλιϲτίδηϲ). 18, 293 μείζων \spation{ἂν} δοθείη δωρειά. 29, 1 θαυμαϲίωϲ \spation{ἂν} ὡϲ εὐλαβούμην. 39, \hypertarget{p399}{\emph{[S.~399]}}\label{p399} 24 καίτοι, τίϲ \spation{ἂν} ὑμῶν οἴεται τὴν μητέρα πέμψαι; epist. 3, 37 τί \spation{ἂν} εἰπὼν μήθ᾽ ἁμαρτεῖν δοκοίην μήτε ψευϲαίμην. [Demosth.] 35, 26 τί \spation{ἄν} τιϲ ἄλλο ὄνομ᾽ ἔχοι θέϲθαι τῷ τοιούτῳ. — Dazu kommen die zahlreichen Stellen nach Art von Demosth. 21, 50 οὐκ \spation{ἂν} οἴεϲθε δημοϲίᾳ πάνταϲ ὑμᾶϲ προξένουϲ αὑτῶν ποιήϲαϲθαι. 

Unter diesen Beispielen, deren Zahl sich übrigens ohne grosse Mühe verdoppeln liesse, finden sich, wie unter den vorhergehenden Kategorien, mehrere, wo die spätere Hälfte des Satzes ein zweites ἄν enthält, mit dem das erste ἄν wieder aufgenommen wird. Ich füge einen besonders instruktiven Fall hinzu. Zu Demosth. 1, 1 (s.~oben) findet sich in proöm. 3 eine parallele Fassung, worin der zweite Teil des Satzes stark erweitert ist, statt χρημάτων ὑμᾶϲ ἑλέϲθαι νομίζω: χρημάτων τὸ μέλλον ϲυνοίϲειν περὶ ὧν νῦν τυγχάνετε ϲκοποῦντεϲ οἶμαι πάνταϲ ὑμᾶϲ ἑλέϲθαι, und hier ist nun der erweiterten Fassung des Satzes wegen hinter πάνταϲ das ἄν wiederholt. (Ganz irrig ist es, wenn Blass im Proöm deswegen das erste ἄν hinter πολλῶν gegen die bessere Überlieferung streicht). Ich glaube wir dürfen sagen, dass in allen Fällen, wo ἄν mehrfach gesetzt ist, dies einen Kompromiss darstellt zwischen dem traditionellen Drang ἄν nah beim Satzanfang zu haben und dem in der klassischen Sprache aufgekommenen Bedürfnis die Partikel dem Verb und andern Satzteilen (siehe oben S.~393) anzunähern: wodurch sich auch erklärt, warum doppeltes ἄν konjunktivischen Sätzen fremd ist. So sind für uns überhaupt alle Sätze mit mehrern ἄν, deren erstes die zweite Stelle inne hat, von Wert, nicht bloss die bereits angeführten. Ich lasse die mir unter die Hand gekommenen Beispiele folgen, natürlich mit Ausschluss von οὔτ᾽ ἄν — οὔτ᾽ ἄν, das nicht hierher gehört.

Aeschyl. Agam. 340 οὔ \spation{τἂν} ἑλόντεϲ αὖθιϲ ἀνθαλοῖεν \spation{ἄν}. 1048 ἐντὸϲ δ᾽ \spation{ἂν} οὖϲα μορϲίμων ἀγρευμάτων πείθοι᾽ \spation{ἄν}. Choeph. 349 λιπὼν \spation{ἂν} εὔκλειαν ἐν δόμοιϲιν — πολύχωϲτον \spation{ἂν} εἶχεϲ τάφον. Hiket. 227 πῶϲ δ᾽ \spation{ἂν} γαμῶν ἄκουϲαν ἄκοντοϲ πάρα ἁγνὸϲ γένοιτ᾽ \spation{ἄν}. Sophokles Aias 537 τί δῆτ᾽ \spation{ἂν} ὡϲ ἐκ τῶνδ᾽ \spation{ἂν} ὠφελοῖμί ϲε. 1058 ἡμεῖϲ μὲν \spation{ἂν} τήνδ᾽ ἣν ὅδ᾽ εἴληχεν τύχην θανόντεϲ \spation{ἂν} προὐκείμεθ᾽ αἰϲχίϲτῳ μόρῳ. 1078 ἀλλ᾽ ἄνδρα χρὴ — δοκεῖν, πεϲεῖν \spation{ἂν} \spation{κἂν} ἀπὸ ϲμικροῦ κακοῦ. Oed. Rex 139 τάχ᾽ \spation{ἂν} κἄμ᾽ \spation{ἂν} τοιαύτῃ χειρὶ τιμωρεῖν θέλοι. 446 \hypertarget{p400}{\emph{[S.~400]}}\label{p400} ϲυθείϲ τ᾽ \spation{ἂν} οὐκ \spation{ἂν} ἀλγύνοιϲ πλέον. 602 οὔτ᾽ \spation{ἂν} μετ᾽ ἄλλου δρῶντοϲ \spation{ἂν} τλαίην ποτέ. 1053 ἧδ᾽ \spation{ἂν} τάδ᾽ οὐχ᾽ ἥκιϲτ᾽ \spation{ἂν} Ἰοκάϲτη λέγοι. Elektra 697 δύναιτ᾽ \spation{ἂν} οὐδ᾽ \spation{ἂν} ἰϲχύων φυγεῖν. 1216 τίϲ οὖν \spation{ἂν} ἀξίαν γε ϲοῦ πεφηνότοϲ μεταβάλοιτ᾽ \spation{ἂν} ὧδε ϲιγὰν λόγων. Philoktet 222 ποίαϲ \spation{ἂν} ὑμᾶϲ πατρίδοϲ (oder πόλεοϲ) ἢ γένουϲ ποτὲ τύχοιμ᾽ \spation{ἂν} εἰπών; (so Dindorf und Heimreich für das handschriftliche ποίαϲ πάτραϲ ἂν ὑμᾶϲ ἢ γένουϲ ποτέ, wo der durch die Stellung von ὑμᾶϲ bewirkte metrische Fehler von andern weniger glücklich gebessert wird). Oed. Col. 391 τίϲ δ᾽ \spation{ἂν} τοιοῦδ᾽ ὑπ᾽ ἀνδρὸϲ εὖ πράξειεν \spation{ἄν}. 780 ἆρ᾽ \spation{ἂν} ματαίου τῆϲδ᾽ \spation{ἂν} ἡδονῆϲ τύχοιϲ. 977 πῶϲ \spation{ἂν} τό γ᾽ ἆκον πρᾶγμ᾽ \spation{ἂν} εἰκότωϲ ψέγοιϲ. 1366 ἦ \spation{τἂν} οὐκ \spation{ἂν} ἦ. Phaedra fr. 622, 1~Ν. οὐ γάρ ποτ᾽ \spation{ἂν} γένοιτ᾽ \spation{ἂν} ἀϲφαλὴϲ πόλιϲ. Fragm. inc. 673 πῶϲ \spation{ἂν} οὐκ \spation{ἂν} ἐν δίκῃ θάνοιμ᾽ \spation{ἄν} (mit drei ἄν!)

Herodot 2, 26, 9 ὁ ἥλιοϲ \spation{ἂν} ἀπελαυνόμενοϲ ἐκ μέϲου τοῦ οὐρανοῦ — ἤιε \spation{ἂν} τὰ ἄνω τῆϲ Εὐρώπηϲ. 2, 26, 11 διεξιόντα δ᾽ \spation{ἄν} μιν διὰ πάϲηϲ Εὐρώπηϲ ἔλπομαι ποιέειν \spation{ἂν} τὸν Ἴϲτρον. 3, 35, 17 οὐδ᾽ \spation{ἂν} αὐτὸν ἔγωγε δοκέω τὸν θεὸν οὕτω \spation{ἂν} κακῶϲ βαλεῖν. 7, 187, 5 οὐδ᾽ \spation{ἂν} τούτων ὑπὸ πλήθεοϲ οὐδεὶϲ \spation{ἂν} εἴποι πλῆθοϲ. Eurip. Alk. 72 πόλλ᾽ \spation{ἂν} ϲὺ λέξαϲ οὐδὲν \spation{ἂν} πλέον λάβοιϲ. id. 96 πῶϲ \spation{ἂν} ἔρημον τάφον Ἄδμητοϲ κεδνῆϲ \spation{ἂν} ἔπραξε γυναικόϲ. Androm. 934 οὐκ \spation{ἂν} ἔν γ᾽ ἐμοῖϲ δόμοιϲ βλέπουϲ᾽ \spation{ἂν} αὐγὰϲ τἄμ᾽ ἐκαρποῦτ᾽ \spation{ἂν} λέχη. Hekabe 742 ἄλγοϲ \spation{ἂν} προϲ\-θεί\-μεθ᾽ \spation{ἄν}. Helena 76 τῷδ᾽ \spation{ἂν} εὐϲτόχῳ πτερῷ ἀπόλαυϲιν εἰκοῦϲ ἔθανεϲ \spation{ἂν} Διὸϲ κόρηϲ. Heraclid. 721 φθάνοιϲ δ᾽ \spation{ἂν} οὐκ \spation{ἂν} τοῖϲδε ϲὸν κρύπτων δέμαϲ. (Vgl. hiezu Elmsley). Hiketiden 417 ἄλλοϲ τε πῶϲ \spation{ἂν} μὴ διορθεύων λόγουϲ ὀρθῶϲ δύναιτ᾽ \spation{ἂν} δῆμοϲ εὐθύνειν πόλιν. (606 τίν᾽ \spation{ἂν} λόγον, τάλαινα, τίν᾽ \spation{ἂν} τῶνδ᾽ αἰτία λάβοιμι). 853 οὐκ \spation{ἂν} δυναίμην οὔτ᾽ ἐρωτῆϲαι τάδε οὔτ᾽ \spation{ἂν} πιθέϲθαι. Hippolyt. 480 ἦ τἆρ᾽ \spation{ἂν} ὄψε γ᾽ ἄνδρεϲ ἐξεύροιεν \spation{ἄν}. Iphig. Taur. 1020 ἆρ᾽ \spation{ἂν} τύραννον διολέϲαι δυναίμεθ᾽ \spation{ἄν}. Medea 616 οὔτ᾽ \spation{ἂν} ξένοιϲι τοῖϲι ϲοῖϲ χρηϲαίμεθ᾽ \spation{ἄν}. Troades 456 οὐκέτ᾽ \spation{ἂν} φθάνοιϲ \spation{ἂν} αὔραν ἱϲτίοιϲ καραδοκῶν. 1244 ἀφανεῖϲ \spation{ἂν} ὄντεϲ οὐκ \spation{ἂν} ὑμνηθεῖμεν \spation{ἄν}. Meleagros fragm. 527 Nauck\textsuperscript{2} μόνον δ᾽ \spation{ἂν} (Nauck: malim ἕν) ἀντὶ χρημάτων οὐκ \spation{ἂν} λάβοιϲ.

Thucyd. 2, 41, 1 λέγω — καὶ κάθ᾽ ἕκαϲτον, δοκεῖν \spation{ἄν} μοι τὸν αὐτὸν ἄνδρα παρ᾽ ἡμῶν ἐπὶ πλεῖϲτ᾽ \spation{ἂν} εἴδη καὶ μετὰ χαρίτων μάλιϲτ᾽ εὐτραπέλωϲ τὸ ϲῶμα αὔταρκεϲ παρέχεϲθαι. (Vgl. \hypertarget{p401}{\emph{[S.~401]}}\label{p401} Stahl zu d. Stelle). 4, 114, 4 οὐδ᾽ \spation{ἂν} ϲφῶν πειραϲομένουϲ — αὐτοὺϲ δακεῖν ἧϲϲον, ἀλλὰ πολλῷ μᾶλλον — εὔνουϲ \spation{ἂν} ϲφίϲι γενέϲθαι. 6, 10, 4 τάχ᾽ \spation{ἂν} δ᾽ ἴϲωϲ, εἰ — λάβοιεν —, καὶ πάνυ \spation{ἂν} ξυνεπίθοιντο. 6, 11, 2 Σικελιῶται δ᾽ \spation{ἄν} μοι δοκοῦϲιν, ὥϲ γε νῦν ἔχουϲιν, καὶ ἔτι \spation{ἂν} ἧϲϲον δεινοὶ ἡμῖν γενέϲθαι. 6, 18, 2 βραχὺ \spation{ἄν} τι προϲκτώμενοι αὐτῇ περὶ αὐτῆϲ \spation{ἂν} ταύτηϲ μᾶλλον κινδυνεύοιμεν. 8, 46, 2 γενομένηϲ δ᾽ \spation{ἂν} — ἀρχῆϲ ἀπορεῖν \spation{ἂν} αὐτόν. Hippokrates περὶ ἀρχαίηϲ ἰητρικῆϲ 1, 572 Littré οὔτε \spation{ἂν} αὐτῷ τῷ λέγοντι οὔτε τοῖϲ ἀκούουϲι δῆλα \spation{ἂν} εἴη. Aristoph. Acharn. 218 οὐδ᾽ \spation{ἂν} ἐλαφρῶϲ \spation{ἂν} ἀπεπλίξατο. 308 πώϲ δέ γ᾽ \spation{ἂν} καλῶϲ λέγοιϲ \spation{ἄν}. Nubes 977 ἠλείψατο δ᾽ \spation{ἂν} τοὐμφαλοῦ οὐδεὶϲ παῖϲ ὑπένερθεν τότ᾽ \spation{ἄν}. 1383 μαμμᾶν δ᾽ \spation{ἂν} αἰτήϲαντοϲ ἧκόν ϲοι φέρων \spation{ἂν} ἄρτον. Pax 68 πῶϲ \spation{ἄν} ποτ᾽ ἀφικοίμην \spation{ἂν} εὐθὺ τοῦ Διόϲ. 646 ἡ δ᾽ Ἑλλὰϲ \spation{ἂν} ἐξερημωθεῖϲ᾽ \spation{ἂν} ὑμᾶϲ ἔλαθε. 1223 οὐκ \spation{ἂν} πριαίμην οὐδ᾽ \spation{ἂν} ἰϲχάδοϲ μιᾶϲ. Aves 829 καὶ πῶϲ \spation{ἂν} ἔτι γένοιτ᾽ \spation{ἂν} εὔτακτοϲ πόλιϲ. Lysistr. 113 ἐγὼ δέ \spation{τἂν} \spation{κἄν} (scil. ἐθέλοιμι), εἴ με χρείη — ἐκπιεῖν. 115 ἐγὼ δέ γ᾽ \spation{ἂν} \spation{κἂν} ὥϲπερ εἰ ψῆτταν δοκῶ δοῦναι \spation{ἂν} ἐμαυτῆϲ παρταμοῦϲα θἤμιϲυ. 147 μᾶλλον \spation{ἂν} διὰ τουτογὶ γένοιτ᾽ \spation{ἂν} εἰρήνη. 361 φωνὴν \spation{ἂν} οὐκ \spation{ἂν} εἶχον. Ranae 34 ἦ \spation{τἄν} ϲε κωκύειν \spation{ἂν} ἐκέλευον μακρά. 581 οὐκ \spation{ἂν} γενοίμην Ἡρακλῆϲ \spation{ἄν}. Ekkles. 118 οὐκ \spation{ἂν} φθάνοιϲ τὸ γένειον \spation{ἂν} περιδουμένη.

Plato Sympos. [Apol. 41 Α.] 176 C ἴϲωϲ \spation{ἂν} ἐγὼ περὶ τοῦ μεθύϲκεϲθαι — τἀληθῆ λέγων ἧττον \spation{ἂν} εἴην ἀηδήϲ. Phaedrus 232~C εἰκότωϲ \spation{ἂν} (Schanz konj. δή) τοὺϲ ἐρῶνταϲ μᾶλλον \spation{ἂν} φοβοῖο. 257~C τάχ᾽ οὖν \spation{ἂν} ὑπὸ φιλοτιμίαϲ ἐπίϲχοι ἡμῖν \spation{ἂν} τοῦ γράφειν. Republ. 7, 526~C οὐκ \spation{ἂν} ῥᾳδίωϲ οὐδὲ πολλὰ \spation{ἂν} εὕροιϲ ὡϲ τοῦτο. Menexenus 236~D \spation{κἂν} ὀλίγου, εἴ με κελεύοιϲ ἀποδύντα ὀρχήϲαϲθαι, χαριϲαίμην \spation{ἄν}. Sophist. 233~A πῶϲ οὖν \spation{ἄν} ποτέ τιϲ — δύναιτ᾽ \spation{ἂν} ὑγιέϲ τι λέγων ἀντειπεῖν. 233~Β ϲχολῇ ποτ᾽ \spation{ἂν} αὐτοῖϲ τιϲ χρήματα διδοὺϲ ἤθελεν \spation{ἂν} — μαθητὴϲ γίγνεϲθαι. [Legg. 5, 742~C]. Xen. Cyrop 1, 3, 11 ϲτὰϲ \spation{ἂν} ὥϲπερ οὗτοϲ ἐπὶ τῇ εἰϲόδῳ — λέγοιμ᾽ \spation{ἄν}. Xen. Anab. 1, 3, 6 ὑμῶν δ᾽ ἔρημοϲ ὤν, οὐκ \spation{ἂν} ἱκανὸϲ οἶμαι εἶναι οὔτ᾽ \spation{ἂν} φίλον ὠφελῆϲαι οὔτ᾽ \spation{ἂν} ἐχθρὸν ἀλέξαϲθαι. 4, 6, 13 δοκοῦμεν δ᾽ \spation{ἄν} μοι ταύτῃ προϲποιούμενοι προϲβαλεῖν ἐρημωτέρῳ \spation{ἂν} τῷ ὄρει χρῆϲθαι. 5, 6, 32 διαϲπαϲθέντεϲ δ᾽ \spation{ἂν} καὶ κατὰ μικρὰ γενομένηϲ τῆϲ δυνάμεωϲ οὔτ᾽ \spation{ἂν} τροφὴν δύναιϲθε λαμβάνειν οὔτε χαίροντεϲ \spation{ἂν} ἀπαλλάξαιτε. Oecon. 4, 5 ὦδ᾽ \spation{ἂν} — ἐπιϲκοποῦντεϲ — ἴϲωϲ \spation{ἂν}  \hypertarget{p402}{\emph{[S.~402]}}\label{p402} καταμάθοιμεν. II S.~283. Epikrates (fragm. com. ed. Kock) fr.~2/3, V.~17 εἶδεϲ δ᾽ \spation{ἂν} αὐτῆϲ Φαρνάβαζον θᾶττον \spation{ἄν}. (Demosth. 18, 240 τί \spation{ἂν} οἴεϲθ᾽ εἰ — ἀπῆλθον —, τί ποιεῖν \spation{ἂν} ἢ τί λέγειν τοὺϲ ἀϲεβεῖϲ ἀνθρώπουϲ τουτουϲί gehört, da die Wiederholung des ἄν durch die Wiederaufnahme des fragenden τί bewirkt ist, nicht hierher.) 27, 56 οὐκ \spation{ἂν} ἡγεῖϲθ᾽ αὐτὸν \spation{κἂν} ἐπιδραμεῖν. Aristot. poet. 25, 1460\textsuperscript{b} 7 ὧδ᾽ \spation{ἂν} θεωροῦϲιν γένοιτ᾽ \spation{ἂν} φανερόν und öfters; vgl. Vahlen zu d. Stelle und Wiener Sitzungsber. LVI 408. 438.

Wenn meine Beispielsammlung in ihrer Unvollständigkeit nicht gar zu ungleichmässig ist, ergibt sich starke Abnahme dieser Art von Doppelsetzung von ἄν im vierten Jahrhundert. Zumal die rednerische Prosa zeigt nur ganz spärliche Beispiele; bekanntlich hat Lysias ἄν gar nie doppelt gesetzt. Ich zweifle nicht, dass diese Abnahme auf fortschreitendes Erlöschen derjenigen Tradition zurückzuführen ist, welche ἄν an zweiter Stelle des Satzes forderte.

Nun findet sich Doppelsetzung des ἄν auch so, dass das erste ἄν nicht die zweite Stelle im Satz einnimmt, sondern eine spätere. Dies ist ganz natürlich, da ja die verschiedensten Satzteile ἄν gern hinter sich hatten, und folglich, sobald ein Satz breiter angelegt war, sich verschiedene mit einander kollidierende Ansprüche auf die Partikel geltend machen mussten. Die hieraus sich ergebenden Kombinationen zu betrachten und für eine jede die betr. Beispiele beizubringen, liegt ausserhalb unserer Aufgabe, die nur die Erforschung der Reste des alten Stellungsgesetzes in sich schliesst, so interessant und so wichtig für die Würdigung der jüngern Sprache es auch wäre, die in dieser herrschend gewordnen Tendenzen im Einzelnen klar zu legen.

\section*{VIII.}
\addcontentsline{toc}{section}{VIII.}

Das Stellungsgesetz, dessen Geltung im Griechischen auf den vorausgehenden Seiten besprochen worden ist, ist für einzelne der asiatischen Schwestersprachen längst anerkannt.

Für die \spation{Altindische Prosa} lehrt Delbrück Syntakt. Forschungen III 47: “Enklitische Wörter rücken möglichst nah an den Anfang des Satzes”. Wesentlich stimmt dazu die Bemerkung, die Bartholomae Ar. Forschungen II 3 für den \spation{Rigveda} giebt: “Auch bei oberflächlicher Betrachtung drängt \hypertarget{p403}{\emph{[S.~403]}}\label{p403} sich die Wahrnehmung auf, dass im RV. die enklitischen Formen der Personalpronomina, sowie gewisse Partikeln, in den meisten Fällen die zweite Stelle innerhalb des Verses oder des Vers-Abschnitts einnehmen”. Vgl. denselben Ar. Forschungen III 30 Anm. über \emph{sīm, smā̆}, sowie die harte Tmesis RV. 5, 2, 7 \emph{ṧunaṧ \spation{cic} chēpam niditaṃ sahasrād yūpād amun̑caḥ}.

Entsprechende Beobachtungen hat derselbe Gelehrte an den \spation{Gathas des Avesta} gemacht (Ar. Forschungen II 3—31). Er stellt dort S. 11 f. für diese die Regel auf: “Enklitische Pronomina und Partikeln lehnen sich an den ersten Hochton im Versglied an”, und ist dabei zur Anerkennung von Ausnahmen bloss bei \emph{cīṭ} genötigt, das eben oft einzelne Satzteile hervorzuheben hat und dann an die betr. Satzteile geheftet ist. Auch dies lässt sich zu der Delbrückschen Regel leicht in Beziehung setzen.

Ganz genau bewährt sich aber diese, wie es scheint, in der \spation{mittelindischen Prosa} (vgl. z. B. Jacobi Māhārāṣṭrī-Erzählungen S. 8 Z. 18 \emph{jena \spation{se} parikkhemi balavisesaṃ}, wo \emph{se} syntaktisch zu \emph{balavisesaṃ} gehört) und sicher im \spation{Altpersischen}, dessen Keilschriftdenkmäler sich durch ihre feierlich-korrekte Sprechweise und ihre genaue Unterscheidung der Enklitika in der Schrift für derartige Beobachtungen besonders eignen. Ich gebe das Material nach Spiegels zweiter Ausgabe vollständig, mit Ausnahme der Stellen, wo das Enklitikum ergänzt ist. Ausnahmslos an zweiter Stelle finden sich zunächst

\emph{\spation{maiy}}: hinter den geschlechtigen Nominativen \emph{Auramazdā} Bh. 1, 25. 55. 87. 94. 2, 24. 40. 60. 68. 3, 6, 17. 37. 44. 60. 65. 86. 4, 60. NR\textsuperscript{a} 50. \emph{dahyāuš} Bh. 4, 39 \emph{hauv} Bh. 2, 79. 3, 11; sodann hinter dem neutralen \emph{tya} (ausser Bh. 4, 65, über das der Lücke wegen nichts bestimmtes gesagt werden kann), Xerxes A 24. 30. C\textsuperscript{a} 13 (zweimal), C\textsuperscript{b} 22 (zweimal). D 19. E\textsuperscript{a} 19; endlich hinter \emph{utā} Bh. 4, 74. 78. Xerxes D 15 (dazu NR\textsuperscript{a} 52, Xerxes D 18. E\textsuperscript{a} 18. A 29, obwohl \emph{utā} an diesen Stellen nicht Sätze, sondern nur Satzglieder verbindet).

\emph{\spation{taiy}}: hinter den geschlechtigen Nominativen \emph{Auramazdā} Bh. 4, 58. 78, \emph{hauv} NR\textsuperscript{a} 57, [wo allerdings nach Thumbs Deutung KZ. XXXII 132 ff. \emph{taiy} an fünfter Stelle stände!] \hypertarget{p404}{\emph{[S.~404]}}\label{p404} hinter dem Neutrum \emph{ava} Bh. 4, 76. 79, hinter \emph{ada} NR\textsuperscript{a} 43. 45, hinter \emph{utā} Bh. 4, 58. 75. 79.

\emph{\spation{šaiy}} hinter \emph{hauv} Darius H 3. \emph{tyaiy} (Nom. Pl.) Bh. 1, 57. 2, 77. 3, 48. 51. 73. \emph{avaþā} 3, 14. \emph{utā} 2, 74. 89. 5, 11. \emph{pasāva} 2, 88.

Also \emph{maiy, taiy, šaiy} folgen der Regel an im ganzen 56 Stellen im Anschluss an die verschiedensten Wörter, und ohne dass eine einzige Stelle widerspricht. Besonderer Beachtung wert sind Bh. 1, 57 \emph{utā tyaišaiy fratamā martiyā anušiyā āhantā}, gegenüber dem \emph{uta martiyā tyai\spation{šaiy} fratamā} u. s. w. der übrigen Stellen mit \emph{tyaišaiy}, ferner Bh. 4, 74 = 4, 78, \emph{utāmaiy, yāvā taumā ahatiy, parikarāha-diš}, wo \emph{maiy} vor dem Zwischensatz, das Verbum erst dahinter kommt; vorzüglich aber Xerxes D 15 \emph{uta\spation{maiy} tya pitā akunauš} = καί μοι ἅττα ὁ πατὴρ ἐποίηϲεν, wo das in den Relativsatz gehörige \emph{maiy} dem Anschluss an \emph{utā} zu liebe vor das Relativpronomen gestellt ist.

Ganz ähnliche Resultate ergeben sich bei den übrigen personalen Enklitika: beim enklitischen \emph{\spation{mām}}, das an der einzigen Belegstelle (Bh. 1, 52) auf satzeinleitendes \emph{mātya} folgt; bei \emph{\spation{šim}}: hinter den Nominativen \emph{āpi} Bh. 1, 95. \emph{kāra} 1, 50. \emph{adam} 1, 52, sowie \emph{haruva} 2, 75. 90; hinter dein Akkusativ \emph{šatram} 1, 59; hinter den Partikeln \emph{avadā} 1, 59. 3, 79. 5, 14. \emph{nai} 4, 49. \emph{pasāva} 2, 90; bei \emph{\spation{šiš}} hinter \emph{avadā} 3, 52; bei \emph{\spation{šām}} hinter den Nominativen \emph{adam} NR\textsuperscript{a} 18; \emph{hya} Bh. 2, 13; dem Akkusativ \emph{avam} Bh. 2, 20. 83., dem Neutrum \emph{tya} Bh. 1, 19. NR\textsuperscript{a} 20. 36; hinter den Partikeln \emph{avathā} 2, 27. 37. 42. 62. 83. 98. 3, 8. 19. 40. 47. 56. 63. 68. 84, und \emph{utā} 3, 56.

Diesen 35 Stellen, die damit zu den obigen 56 hinzukommen, stehen allerdings 3 abweichende gegenüber: Bh. 1, 14 \emph{vašnā Auramazdāha adam\spation{šām} xšāyaþiya āham}; 4, 6 \emph{vašn[ā Aurama]zdāha adam\spation{šām} ajanam}; NR\textsuperscript{a} 35 \emph{vašnā Auramazdāha adam\spation{šim} gāþvā niyašādayam}; immerhin schliesst sich an allen drei das Enklitikon unmittelbar an das Subjekt \emph{adam} an. Und mehr als ausgeglichen werden diese Ausnahmen durch solche Stellen wie Bh. 2, 75 = 2, 90 \emph{haruva\spation{šim} kāra avaina} (“universus eum populus videbat”) wo das Pronomen zwischen Attribut und Substantiv getreten ist, oder wie Bh. 3, 56 \emph{utā\spation{sām} 1 martiyam maþištam akunauš}, wo \emph{šām} syn-\hypertarget{p405}{\emph{[S.~405]}}\label{p405}taktisch zu \emph{maþištam} gehört (“und er machte einen Menschen zum Obersten derselben”).

Sieht man von \emph{hacāma} ‘von mir’ und \emph{haca avadaša} ‘von da aus’ ab, so bleiben noch \emph{\spation{‑ciy}} (= altind. \emph{cit}) und \emph{\spation{dim}, \spation{diš}}. Letztere folgen der Regel hinter dem Nominativ \emph{drauga} Bh. 4, 34, dem neutralen \emph{tyā} Bh. 1, 65, der Partikel \emph{naiy} 4, 73. 78, \emph{pasāva} Bh. 4, 35. NR\textsuperscript{a} 33, der Verbalform \emph{visanāha} Bh. 4, 77. Kaum als Ausnahme kann 4, 74 gelten: \emph{utāmaiy, yāvā taumā ahatiy, parikarāhadiš} (Spiegel: “sondern sie mir, so lange deine Familie dauert, bewahrst”): denn wenn sich hier \emph{diš} auch nicht an das erste Wort des Satzes schlechthin anschliesst, so doch an das erste auf den Zwischensatz folgende Wort. So widerspricht nur NR\textsuperscript{a} 42 \emph{[yath]ā xšnās[āha\spation{diš}]} “damit du sie kennst”, und da mag man billig fragen, ob nicht die Ergänzung falsch sei.

Dagegen \emph{\spation{ciy}} emanzipiert sich von der Regel. Zwar steht es Bh. 1, 53 hinter \emph{kaš}, S. 23 hinter \emph{hauv} und Xerxes D 20. C\textsuperscript{a} 14.\textsuperscript{b} 24 an zweiter, aber Bh. 1, 46 hinter \emph{kaš}, 1, 53 hinter \emph{ciš}, 1, 63. 67. 69 hinter \emph{paruvam}, 4, 46 und Xerxes D 13 hinter \emph{aniyaš} an dritter Stelle oder noch weiter hinten im Satz. Es steht eben hinter dem Wort, das der Hervorhebung bedarf; vgl. die Stellung von \emph{cīṭ} im Avesta (oben S. 403).

So die indoiranischen Sprachen. Aber auch ausserhalb derselben bieten sich belehrende Parallelen dar. Dass vorerst den \spation{germanischen} Sprachen unser Stellungsgesetz nicht fremd ist, zeigt schon die Behandlung der schwachbetonten Personalpronomina im Neuhochdeutschen. Zumal, wenn \emph{sich} im Nebensatz und dann in weiter Entfernung vom Verbum steht, kommt uns das Gesetz zum Bewusstsein, freilich als eine unbequeme Fessel, deren wir uns in schriftlicher Darstellung gern dadurch entledigen, dass wir das Pronomen zum Verbum ziehen. Wir glauben hierdurch deutlicher zu sein, empfinden aber solche Stellung doch als unschön. Und oft entschlüpft uns in mündlicher Rede doppeltes \emph{sich}, eines am traditionellen Platze zu Anfang, und eines beim Verbum: ganz analog dem doppelten ἄv der Griechen. — Auch bei den andern persönlichen Pronomina kann man solche Tendenz beobachten.

Doch wage ich auf diesem Gebiet eingehendere Erörterungen nicht, und möch\-te nur noch an die von Kluge KZ. \hypertarget{p406}{\emph{[S.~406]}}\label{p406} XXVI 80 in ihrer Bedeutung hervorgehobenen gotischen Tmesen \emph{ga-u-laubeis, ga-u-ƕa-sēƕi, us-nu-gibiþ} und die Fälle erinnern, wo \emph{u(h)} und ähnliche Partikeln im Gotischen Präposition und Kasus trennen. Mit Recht erkennt Kluge in diesem Drang der Enklitika sich unmittelbar an das erste Wort anzuschliessen, einen alten Rest aus der Vorzeit. Das lehrreichste Beispiel ist unstreitig \emph{ga-u-ƕa-sēƕi} mit seinem Einschub des Indefinitums \emph{ƕa} = τὶ.

\section*{IX.}
\addcontentsline{toc}{section}{IX.}

Indem ich dahingestellt lasse, ob das Pronomen infixum des \spation{Keltischen} (Zeuss Grammatica celtica S. 327 ff.) nicht von hier aus Licht empfange, wende ich mich sogleich zum \spation{Latein}, und konstatiere hier zum voraus, dass die Latinisten alter Schule schon längst lehren, dass zumal in klassischer Prosa die Stelle unmittelbar hinter dem ersten Wort des Satzes mit Tonschwäche verbunden sei, und die dorthin gestellten Wörter entweder von Haus aus enklitisch seien oder es durch eben diese Stellung werden (Reisig Vorlesungen über latein. Sprachwissenschaft S. 818; Madvig zu Cic. de finibus I 43; Seyffert-Müller zu Cic. Laelius\textsuperscript{2} S. 49. 64; Schmalz Latein. Syntax\textsuperscript{2} S. 557 u. s. w.) Für die Einzeluntersuchung ist es nun allerdings unbequem, dass die Überlieferung anders als im Griechischen keine äussern Kennzeichen zur Unterscheidung orthotonischer und enklitischer Formen liefert. Trotzdem können wir ziemlich sicher gehen. Denn gesetzt z. B. es zeige ein \spation{Casus obliquus} eines \spation{persönlichen Pronomens}, auf dem nach Ausweis des Zusammenhangs keinerlei Nachdruck liegt, genau dieselben Stellungseigentümlichkeiten, die wir bei μοι und seinen Genossen gefunden haben, so muss in einem solchen Fall sowohl die enklitische Betonung des betr. Pronomens als die Gültigkeit des fürs Griechische aufgestellten Stellungsgesetzes auch fürs Latein m. E. als erwiesen gelten. Und solche Fälle finden sich genug.

Erstens eigentliche Tmesis zwischen Präposition und Verbum (vgl. fürs Griechische oben S. 361): \emph{sub vos placo, ob vos sacro} (Festus 190\textsuperscript{b} 2. 309\textsuperscript{a} 30). Zweitens Zertrennung anderer, sonst zur Einheit verwachsener Wortverbindungen durch ein der zweiten Stelle zustrebendes schwach betontes Pronomen: a) mit \emph{per} verbundener Adjektive: Cicero de orat. \hypertarget{p407}{\emph{[S.~407]}}\label{p407} (1, 214 \emph{in quo per \spation{mihi} mirum visum est}). 2, 271 \emph{nam sicut, quod apud Catonem ist} [sic] \emph{—, per \spation{mihi} scitum videtur —: sic profecto se res habet}. ad Quintum fr. 1, 7 (9), 2 \emph{\spation{per mihi} benigne respondit}. ad Att. 1, 4, 3 \emph{quod ad me de Hermathena scribis, per \spation{mihi} gratum est}. 1, 20, 7 \emph{per \spation{mihi}, per, inquam, gratum feceris}. Dass Lael. 16 \emph{pergratum \spation{mihi} feceris, spero item Scaevolae} steht und nicht \emph{per \spation{mihi} gratum}, wie Orelli verlangte, dient zur Bestätigung unserer Regel, da \emph{mihi} wegen des Gegensatzes zu \emph{Scaevolae} stark betont gewesen sein muss (Seyffert-Müller zu d. St. S. 95\textsuperscript{2}). Die weitern Fälle, in denen \emph{per} Tmesis erleidet, werden im Verlauf zur Erwähnung kommen, ausser de or. 1. 205 \emph{ista sunt per grata per\spation{que} iucunda} und ad. Att. 10, 1, 1 \emph{per \spation{enim} magni aestimo}, in welch beiden Beispielen übrigens eine, die zweite Stelle verlangende, Partikel die Trennung bewirkt hat.

b)	Des Pronomens \emph{qui-cunque} (Neue\textsuperscript{3} 2, 489), nebst Zubehör (dessen Tmesis in Fällen wie Cicero pro Sest. 68 \emph{quod iudicium cunque subierat}. De divin. 2, 7 \emph{qua re cunque}. Lucrez 4, 867 \emph{quae loca cunque}. 6, 85 \emph{qua de causa cunque}. 6, 867 \emph{quae semina cunque}. Horaz Oden 1, 6, 3 \emph{quam rem cunque} und in den von Neue aus Gellius und Appuleius angeführten Stellen; ferner in Cicero de legibus 2, 46 \emph{quod ad cunque legis genus} besondrer Art ist). Cicero de orat. 3, 60 \emph{quam \spation{se} cunque in partem dedisset}. Tuscul. 2, 15 \emph{quo ea \spation{me} cunque ducet}. De divin. 2, 149 \emph{quo \spation{te} cunque verteris}. Verg. Aen. 1, 610 \emph{quae \spation{me} cunque vocant terrae}. 8, 74 \emph{quo \spation{te} cunque lacus miserantem incommoda nostra fonte tenet}. 12, 61 \emph{qui \spation{te} cunque manent isto certamine casus}. Horaz Oden 1, 7, 25 \emph{quo \spation{nos} cunque feret melior Fortuna parente}. 1, 27, 14 \emph{quae te cunque domat Venus}. (Ovid. trist 2, 78 \emph{delicias legit qui \spation{tibi} cunque meas.}) Martial 2, 61, 6 \emph{nomen quod \spation{tibi} cunque datur}. Darnach Terenz Andria 263 \emph{quae \spation{meo} quomque animo lubitum est facere}. Ausser an diesen Stellen und den unten wegen andrer Enklitika anzuführenden kommt Tmesis von \emph{quicunque} nur Lucrez 6, 1002. Horaz 1, 9, 14. 1, 16, 2. Sat. 2, 5, 51 vor, wo ganz beliebige Wörter dazwischen getreten sind. (Vgl. Horaz Sat. 1, 9, 33 \emph{garrulus hunc quando consumet cunque.}) Wir dürfen ruhig hierin poetische Freiheiten erkennen.

c)	Des Adverbs \emph{quomodo}. Plautus Cistell 1, 1, 47 \emph{ne-}\hypertarget{p408}{\emph{[S.~408]}}\label{p408}\emph{cesse est, quo tu \spation{me} modo voles esse, ita esse mater}. Cicero pro Rosc. Am. 89 \emph{quo \spation{te} modo iactaris}. in Pisonem 89 \emph{quo \spation{te} modo ad tuam intemperantiam innovasti}. pro Scauro 50 \emph{quo \spation{te} nunc modo appellem}. Vgl. pro Rab. Post. 19 \emph{quonam \spation{se} modo defendet}. pro Scauro 50 \emph{quocunque igitur \spation{te} modo —}. Weiteres unten; Trennung durch volltonige Wörter scheint sich nicht zu finden. Denn Cicero de lege agr. 1, 25 \emph{quo uno modo} ist besondrer Art.

Drittens ist die Trennung von Präposition und regiertem Kasus in der bekannten Bittformel zu nennen: Plautus Bacch. 905 \emph{per \spation{te} ere obsecro deos immortales}. Menaechmi 990 \emph{per ego \spation{vobis} deos atque homines dico}. Terenz Andria 538 \emph{per \spation{te} deos oro et nostram amicitiam, Chremes}. 834 \emph{per ego \spation{te} deos oro}. Tibull 3, 11 (= 4, 5,) 7 \emph{per \spation{te} dulcissima furta perque tuos oculos per geniumque rogo}. Livius 23, 9, 2 \emph{per ego \spation{te}, inquit, fili, quaecunque iura iungunt liberos parentibus, precor quaesoque}. Curtius 5, 8, 16 \emph{per ego \spation{vos} decora maiorum — oro et obtestor}. Lucan 10, 370 \emph{per \spation{te} quod fecimus una perdidimusque nefas — ades} (das Verbum des Bittens ist hier, wie im folgenden Beispiel, weggelassen). Silius 1, 658 \emph{per \spation{vos} culta diu Rutulae primordia gentis —, conservate pios}. Das \emph{per}, woran sich das Pronomen \emph{te, vos, vobis} anhängt, steht also immer am Anfang des Satzes.

Viertens seien die paar Beispiele von Trennung minder enger Wortgruppen angeführt, die von den vorgenannten Latinisten als Belege für Ciceros Neigung das tonlose Pronomina hinter dem ersten Wort einzuschieben beigebracht werden: (de orat. 3, 209 \emph{his autem de rebus sol \spation{me} ille admonuit.}) Brutus 12 \emph{populus \spation{se} Romanus erexit}. orator 52 \emph{sentiebam, non \spation{te} id sciscitari}. de offic. 1, 151 \emph{in agros \spation{se} possessionesque contulit}. (Laelius 15 \emph{idque eo \spation{mihi} magis est cordi}. 87 \emph{ut aliquis \spation{nos} deus ex hac hominum frequentia tolleret.})

Fünftens sind einige Fälle zu nennen, wo ein zwei Gliedern des Satzes gemeinsames Pronomen ins erste eingeschoben wird (Müller zum Laelius XX 72). Cic. epist 4, 7, 2 \emph{sed idem etiam illa vidi, neque te consilium civilis belli ita gerendi nec copias Cn. Pompei — probare}. Laelius 37 \emph{nec \spation{se} comitem illius furoris, sed ducem praebuit}. Sallust or. Philippi 16 \emph{neque \spation{te} provinciae neque leges neque di penates} \hypertarget{p409}{\emph{[S.~409]}}\label{p409} \emph{civem patiuntur}. (Ebenso, aber ohne Einfluss des Stellungsgesetzes Caesar bell. civ. 1, 85, 11 \emph{quae omnia et \spation{se} tulisse patienter et esse laturum}, wozu jedoch Paul: “\emph{se} omittendum esse verborum consecutio docet”.)

Anderes geben die bisherigen Forschungen über die Stellung des Pronomens bei den Komikern an die Hand. (Vgl. Kämpf De pronominum personalium usu et conlocatione apud poëtas scenicos Romanorum: Berliner Studien für klass. Philologie u. Archäologie III 2. 1886). Aus Kämpf hebe ich namentlich die Beobachtung hervor (S. 31. 36), dass sich die Personalpronomina in der grossen Mehrzahl der Fälle an Fragewörter und an satzeinleitende Konjunktionen unmittelbar anschliessen; (vgl. z. B. bei Joseph Bach in Studemunds Studien auf d. Gebiete des archaischen Lateins II 243 die Zusammenstellung der Fälle mit \emph{quid tibi} und folgendem den Akkusativ regierenden Substantivum verbale auf \emph{-tio}), ebenso (S.~40) an die Affirmativpartikeln, wie \emph{hercle, pol, edepol} u. s. w., die, worauf später die Rede kommen wird, entweder die erste oder die zweite Stelle im Satz einnehmen. Sehr beachtenswert ist auch die an eine Beobachtung Kellerhoffs geknüpfte Bemerkung Kämpfs, dass in den überaus zahlreichen Fällen, wo die Negation an der Spitze des Verses steht, sich ein allfällig vorhandenes Pronomen personale daran anlehnt.

Am lehrreichsten ist aber der Nachweis, den Langen Rhein. Museum XII (1857) 426~ff. betreffend die Beteuerungs-, Wunsch- und Verwünschungsformeln mit \emph{di, di deaeque} oder einem einzelnen Gottesnamen als Subjekt und konjunktivischem (oder futurischem) Verbum als Prädikat gegeben hat. (Vgl. auch Kellerhoff in Studemunds Studien II 77~f.). Wo \emph{di, di deaeque}, oder der betr. Gottesname am Satzanfang steht, folgen die vom Verb regierten pronominalen Akkusative und Dative \emph{me, te, tibi}, ebenso die in diesen Wendungen seltener vorkommenden \emph{vos, vobis, (istum), istunc, istam, istunc, istaec, illum} dem Subjekt unmittelbar. Wo das Subjekt mehrgliedrig ist, findet sich das Pronomen zwar vereinzelt erst nach der ganzen Subjektgruppe: Plautus Casina 275 \emph{Hercules dique \spation{istam} perdant}. Vgl. Epidicus 192 \emph{di hercle omnes \spation{me} adiuvant, augent, amant}, wo Langen (und nach ihm Götz) \emph{di \spation{me} hercle omnes} ändert. Mostell. 192 \emph{di deaeque omnes \spation{me} pessumis exemplis interficiant}. (Ritschl  \hypertarget{p410}{\emph{[S.~410]}}\label{p410} \emph{me omnes}). Öfter ist das Pronomen nach dem ersten Gliede eingeschoben: Aulul. 658 \emph{Iuppiter \spation{te} dique perdant}. (Dasselbe Captivi 868. Curculio 317. Rudens 1112). Captivi 919 \emph{Diespiter \spation{te} dique, Ergasile, perdant}. Pseudolus 271 \emph{di \spation{te} deaeque ament}. Mostell. 463 \emph{di \spation{te} deaeque omnes faxint cum istoc omine}. 684 \emph{di \spation{te} deaeque omnes funditus perdant, senex}. Ebenso bei attributiver Gruppe: Menaechmi 596 \emph{di \spation{illum} omnes perdant}. Terenz Phormio 519 \emph{di \spation{tibi} omnes id quod es dignus duint}. Eine Mittelstellung nimmt Plautus Persa 292 ein: \emph{di deaeque \spation{me} omnes perdant}; ebenso Mostell. 192 nach Ritschls Schreibung, siehe oben.

Schon dies ist beachtenswert; von besondrer Bedeutung ist aber, dass wenn an der Spitze des Satzes ein \emph{ita, itaque, ut, utinam, hercle, qui, at} steht, darauf nicht etwa zuerst \emph{di} oder der Göttername und dann erst das Pronomen folgt, sondern in diesem Fall das Pronomen dem nominalen Subjekt vorangeht. Wo \emph{at} und \emph{ita} verbunden sind, steht das Pronomen dahinter Curculio 574 \emph{at ita \spation{me} machaera et clypeus bene iuvent}. Miles glor. 501 \emph{at ita \spation{me} di deaeque omnes ament}; dagegen zwischen beiden Partikeln Poenulus 1258 \emph{at \spation{me} ita dei servent}, wo ich dem Metrum lieber mit der Schreibung \emph{med}, als mit der von den Neuern vorgezogenen Umstellung \emph{at ita me} aufhelfen würde. Auch hinter andern Anfangswörtern, als den angeführten Partikeln, geht das Pronomen dem Subjekt \emph{di} voraus: Pseudolus 430 \emph{si \spation{te} di ament}. 936 \emph{tantum \spation{tibi} boni di immortales duint}. Mostell. 655 \emph{malum quod} (= κακόν τι) \emph{\spation{isti} di deaeque omnes duint} u. s. w. An der widerstrebenden Stelle Plautus Casina 609 \emph{quin hercle di \spation{te} perdant} will Langen, dem sich Kellerhoff a. a. O. und Schöll in seiner Ausgabe anschliessen, \emph{quin hercle \spation{te} di perdant} umstellen, während Seyffert mittelst der Interpunktion \emph{quin hercle “di \spation{te} perdant”} dem Schaden abzuhelfen sucht.

Die Beobachtung von Langen bewährt sich auch an der klassischen Latinität. Insofern wenigstens als die Beteuerungsformeln mit \emph{ita, sic} auch hier das \emph{me, te, mihi} fast ausnahmslos unmittelbar hinter \emph{ita, sic} haben. Mit \emph{\spation{ita}}: Cicero divinatio in Caec. 41 \emph{ita \spation{mihi} deos velim propitios}. Verrina 5, 35 \emph{ita \spation{mihi} meam voluntatem — vestra populique Romani existimatio comprobet}. 5, 37 \emph{ita \spation{mihi} omnis deos propitios velim}. Epistulae 5, 21, 1 \emph{nam tecum esse, ita \spation{mihi} com-}\hypertarget{p411}{\emph{[S.~411]}}\label{p411}\emph{moda omnia quae opto contingant, ut vehementer velim}. ad Atticum 1, 16, 1 \emph{saepe, ita \spation{me} di iuvent, te — desideravi}. 16, 15, 3 [Octavianus] \emph{iurat “ita \spation{sibi} parentis honores consequi liceat”}. Catull 61, 196 \emph{at marite, ita \spation{me} iuvent caelites, nihilo minus pulcer es}. 66, 18 \emph{non (ita \spation{me} divi) vera gemunt (iuerint)}. 97, 1 \emph{non, ita \spation{me} di ament, quicquam referre putavi}. Diese Stellung bleibt auch, wenn dem \emph{ita} noch eine Partikel vorgeschoben wird: Cicero in Catil. 4, 11 \emph{nam ita \spation{mihi} salva republica vobiscum perfrui liceat, ut —}. epist. 10, 12, 1 \emph{tamen ita \spation{te} victorem complectar —, ut —}. (Plancus ad Ciceronem epist. 10, 9, 2 \emph{ita ab imminentibus malis respublica me adiuvante liberetur} und Petron. 74 \emph{ita genium meum propitium habeam} kommen natürlich nicht in betracht.

Mit \emph{\spation{sic}}: Catull. 17, 5 \emph{sic \spation{tibi} bonus ex tua pons libidine fiat}. Virgil Ecl. 10, 4 \emph{sic \spation{tibi}, cum fluctus supterlabere Sicanos, Doris amara suam non intermisceat undam}. Horaz Oden 1, 3, 1 \emph{sic \spation{te} diva potens Cypri — regat}. Tibull 2, 5, 121 \emph{sic \spation{tibi} sint intonsi Phoebe capilli}. Properz 1, 18, 11 \emph{sic \spation{mihi} te referas levis}. 3, 6, 2 \emph{sic \spation{tibi} sint dominae Lygdame dempta iuga}. Ovid. Heroid. 4, 169 \emph{sic \spation{tibi} secretis agilis dea saltibus adsit}. 4, 173 \emph{sic \spation{tibi} dent nymphae}. Metamorph. 14, 763 \emph{sic \spation{tibi} nec vernum nascentia frigus adurat poma}. Corpus inscr. lat. 4, 2776 \emph{presta mi sinceru(m): sic \spation{te} amet que custodit ortu(m) Venus}. Vgl. Martial 7, 93, 8 \emph{perpetuo liceat sic \spation{tibi} ponte frui}, wo das Pronomen zwar nicht an zweiter Stelle, aber doch unmittelbar hinter \emph{sic} steht. Bei einem Ablativus absolutus (Horaz Oden 1, 28, 25 \emph{sic — Venusinae plectantur silvae \spation{te} sospite}) und beim Possessivum (Petron. 75 \emph{rogo, sic peculium \spation{tuum} fruniscaris}; doch Virgil Ecl. 9, 30 \emph{sic \spation{tua} Cyrneas fugiant examina taxos}) haben wir kein Recht Geltung der Regel zu erwarten. Auch Ovid Trist. 5, 2, 51~f. (\emph{sic habites terras et \spation{te} desideret aether) sic ad pacta \spation{tibi} sidera tardus eas} kann nicht als Verletzung der Regel gelten. Dagegen ist auffällig Tibull 1, 4, 1 \emph{sic umbrosa \spation{tibi} contingant tecta Priape}. Petron 61 \emph{sic felicem \spation{me} videas}.

Aus Ausdrücken wie die eben besprochnen sind \emph{mehercule, mediusfidius, mecastor} bekanntlich verkürzt. Daraus scheint sich mir auch ihre Stellung zu erklären. In der grossen Mehrzahl der Beispiele stehn sie an zweiter Stelle des \hypertarget{p412}{\emph{[S.~412]}}\label{p412} Satzes. So die beiden ersten ausnahmslos in Ciceros Reden. Vgl. für \emph{mehercule} auch Terenz Eunuch. 416. Cicero de or. 2, 7. Epist. 2, 11, 4. ad Atticum 10, 13, 1. 16, 15, 3. Caesar bei Cic. ad Att. 9, 7\textsuperscript{c} 1. Caelius bei Cic. epist. 8, 2, 1. Plancus ibid. 10, 11, 3. Plin. Epist. 6, 30; für \emph{mediusfidius} auch Cicero epist. 5, 21, 1. Tuscul. 1, 74 \emph{(ne ille \spation{mediusfidius} vir sapiens)}. Sallust Catil. 35, 2. Livius 5, 6, 1. 22, 59, 17. Seneca suas 6, 5. Plin. epist. 4, 3, 5. Besonders beweiskräftig ist die nicht seltene Einschiebung der zu einer ganzen Periode gehörigen Beteuerungspartikel hinter die einleitende Partikel des Vordersatzes: \emph{si mehercule} Cicero pro Caecina 64. Catil. 2, 16. pro Scauro fragm. 10 Müller. Sallust Catil. 52, 35. \emph{quanto mehercule} Sallust Histor. oratio Philippi 17. \emph{si mediusfidius} Cicero pro Sulla 83. pro Plancio 9. Livius 5, 6, 1. 22, 59, 17. Die Stellen wo eine dieser beiden Partikeln an einer spätern Stelle des Satzes steht, sind bedeutend weniger zahlreich (\emph{mehercule}: Terenz Eunuch. 67. Catull 38, 2. Phaedrus 3, 5, 4. Plin. epist. 4, 1, 1. — \emph{mediusfidius}: Cato bei Gellius 10, 14, 3. Cicero ad Atticum 8, 15 A 2. Quintil. 5, 12, 17). Bemerkenswert sind Cicero Att. 4, 4\textsuperscript{b} 2 \emph{\spation{mediusfidius}, ne tu emisti locum praeclarum}, und 5, 16, 3 \emph{\spation{mehercule} etiam adventu nostro reviviscunt —}, durch die ganz eigentümliche Voranstellung der Partikel. — Was das vorklassische \emph{mecastor} betrifft, so entsprechen Plautus Aulul. 67 \emph{noenum \spation{mecastor} quid ego ero dicam meo — queo comminisci} und auch Men. 734 \emph{ne istuc \spation{mecastor} iam patrem accersam meum} der Regel, Aulul. 172 \emph{novi hominem haud malum \spation{mecastor}} widerspricht ihr.

Von der Stellungsregel für das vokativische \emph{hercule} und dessen Genossen (sie\-he unten) unterscheidet sich die für \emph{mehercule} und Genossen darin, dass, von den isolierten Stellen Cicero Att. 4, 4\textsuperscript{b} 2. 5, 16, 3 abgesehen, die mit \emph{me-} gebildeten von der ersten Stelle im Satz ausgeschlossen sind. Hiernach wird man ihre Neigung für die zweite Stelle nicht mit der bei \emph{hercule} beobachtbaren zusammenstellen, sondern aus der enklitischen Natur des \emph{me} herleiten.

\section*{X.}
\addcontentsline{toc}{section}{X.}

Gehn wir zu andern Formen über! Wenn der Vokativ \emph{mī} wirklich dem μοι in griechischem τέκνον μοι u. dergl. (s. \hypertarget{p413}{\emph{[S.~413]}}\label{p413} oben S.~362) gleichzusetzen ist, wie Brugmann Grundriss II 819 annimmt, so ist jedenfalls dem Wort in dieser Verwendung die Enklisis schon in vorhistorischer Zeit abhanden gekommen, da es sich bereits bei Plautus im Satzanfang findet. Es wäre nicht undenkbar, dass die Voranstellung von \emph{mi} vor das Substantivum, zu dem es gehört, in solchen Sätzen aufgekommen wäre, wo der Vokativ nicht an erster Stelle stand, ihm also \emph{mi}, um an die ihm zukommende zweite Stelle im Satz zu gelangen, dem Vokativ vorangestellt werden musste.

Sicherer als dies ist, dass die obliquen Kasus von \emph{is}, gerade wie att. αὐτοῦ und das enklitische \emph{asmāi} des Altindischen, der Weise von \emph{me, te} folgen. Und so lesen wir z.~B. Cicero Lael. 10 \emph{quam \spation{id} recte fecerim}, wie Brutus 12 \emph{populus \spation{se} Romanus erexit} (s. oben S.~408). Ja auch bei den demonstrativeren Pronomina \emph{iste, ille} haben wir enklitische Stellung in den S.~409~ff. besprochenen Wunsch- und Verwünschungssätzen.

Weiterhin ist es vielleicht einem oder andern Leser aufgefallen, dass in den Beispielen wo ein \emph{me, te} seiner Stellung wegen eine Wortgruppe zerreisst, demselben mehrfach ein \emph{ego}, vorhergeht: Plautus Men. 990 \emph{per \spation{ego} vobis deos — dico}. Terenz Andr. 834 \emph{per \spation{ego} te deos oro}. Ähnlich Livius 23, 9, 2. Curtius 5, 8, 16. Ferner Plautus Cistell. 1, 1, 47 \emph{quo \spation{tu} me modo voles esse}. Auch der Nominativ von \emph{is, ea, id}: Cicero Tusc. 2, 15 \emph{quo \spation{ea} me cunque duxit}. Man wird nicht bestreiten können, dass in solchen Fällen \emph{ego, tu, ea} eben auch enklitisch sind, und wird sich an die Enklisis von deutschem \emph{er, sie, es} im Nebensatz, und bei Inversion und Frage, auch im Hauptsatz erinnern. Dann sind auch Stellen wie Cicero de orat. 2, 97 \emph{quantulum \spation{id} cunque est}; de nat. deorum 2, 76 \emph{quale \spation{id} cunque est}, weiterhin pro Cluent. 66 \emph{quonam igitur \spation{haec} modo gesta sunt}, Sallust Cat. 52, 10 \emph{cuius \spation{haec} cunque modi videntur}, Terenz Ad. 36 \emph{ne aut \spation{ille} alserit aut ceciderit}, pro Deiot. 15 \emph{quonam \spation{ille} modo cum regno distractus esset}, auf diese Weise zu erklären. Übrigens ist auch das aufs Verb unmittelbar folgende \emph{ego, tu}, wie im Griechischen ἐγώ in gleicher Stellung, gewiss als wesentlich enklitisch zu fassen.

Bei den \spation{Indefinita} hält das Latein noch strenger an der alten Regel fest als das Griechische und erkennt man \hypertarget{p414}{\emph{[S.~414]}}\label{p414} dieselbe auch schon längst an, allerdings nicht mit ganz richtiger Formulierung. Nehmen wir den Sprachgebrauch der alten Inschriften, der Kommentarien Caesars und der Reden Ciceros nach dem Index zu CIL. I und den Lexica von Meusel und Merguet zusammen, so ergiebt sich, dass sich \emph{\spation{quis}, \spation{quid}} in der unendlichen Mehrzahl der Belege an satzeinleitende Wörter wie \emph{ē-, nē} nebst \emph{dum nē, num}, das Relativum \emph{qui} nebst seinen Formen, \emph{quo, cum, quamvis, neque} anschliesst. Natürlich hat \emph{-ve} (in \emph{neve, sive} u. sonst) vor ihm den Vortritt, seltener — bei Caesar nur einmal — haben ihn pronominale Enklitika: CIL. I 206, 71 \emph{neve eorum \spation{quod} saeptum clausumve habeto}. ibid. 94 und 104 \emph{dum eorum \spation{quid} faciet}. Vgl. 205 II 15. 41 \emph{qui ita \spation{quid} confessus erit}. Cicero Verrina 5, 168 \emph{quod eum \spation{quis} ignoret}. Caesar bell. civ. 3, 32, 3 \emph{qui horum \spation{quid} acerbissime crudelissimeque fecerat, is et vir et civis optimus habebatur}. Im eigentlichen Satzinnern findet sich in den genannten Texten das Indefinitum im ganzen nur hinter \emph{alius} und \emph{ali-}, wobei zu beachten ist, dass es \emph{si quis alius, ne quis alius}, nicht \emph{si alius quis, ne alius quis} zu heissen pflegt. Daneben finden wir in Ciceros Reden \emph{quis, quid} in Relativsätzen vom Relativum stets (an 7—8 Stellen) durch ein oder zwei andre Wörter getrennt. Eine auffällige Ausnahme ausserdem bildet CIL. I 206, 70 \emph{nei quis in ieis locis inve ieis porticibus \spation{quid} inaedificatum immolitumve habeto}.

Ganz dasselbe gilt für die zugehörigen indefiniten Adverbia, besonders \emph{quando}, und gilt andrerseits für die Indefinita überhaupt, so viel ich sehe, in den sonstigen archaischen und klassischen Texten. Freilich muss man sich, um das zu erkennen, gelegentlich von den modernen Herausgebern emanzipieren. Hat doch z.~B. Götz in Plautus Mercator 774 ganz fröhlich das enklitische \emph{quid} mitten in einen Satz und zugleich an den Anfang des Verses gestellt (s. dessen Ausgabe sowie Acta societ. phil. Lips. VI 244), obgleich die Überlieferung das korrekte \emph{si quid} bietet! Vereinzelte Ausnahmen lassen sich natürlich auftreiben, doch ist z.~B. Plaut. Epid. 210 \emph{tum captivorum \spation{quid} ducunt secum} das \emph{quid} wohl exclamativ zu fassen, also orthotoniert.

Angesichts solcher Strenge der Stellungsregel kann weder die Anastrophe Cicero Lael. 83 \emph{si \spation{quos} inter societas aut est aut fuit} (vgl. Seyffert z. d. St.), noch die häufige, \hypertarget{p415}{\emph{[S.~415]}}\label{p415} an die oben S.~367, 368 zusammengestellten Beispiele des Griechischen erinnernde Abtrennung des attributiven Indefinitums von seinem Nomen befremden z.~B. Caesar bell. gall. 6, 22, 3 \emph{ne \spation{qua} oriatur pecuniae \spation{cupiditas}}. bell. civ. 1, 21, 1 \emph{ne \spation{qua} aut largitionibus aut animi confirmatione aut falsis nuntiis \spation{commutatio} fieret voluntatis} u. s. w. u. s. w. Daran, dass im Oskischen und Umbrischen \emph{pis, pid; pis, pir} meist in unmittelbarem Anschluss an \emph{svaì, svae; sve, so} ‘wenn’ überliefert sind, sei nur im Vorbeigehn erinnert.

Dass \emph{\spation{quisque}} als auf enklitischem \emph{quis} beruhend ein Enklitikum ist und dass es zwar häufiger als \emph{quis} im Satzinnern steht, aber in der Regel doch nur hinter Superlativen, Ordinalien, \emph{unus} und \emph{suus}, sonst hinter dem ersten Satzwort, ist bekannt. In den Inschriften von CIL. I zeigt sich die Stellungsregel in voller Deutlichkeit: \emph{quisque} hinter \emph{primus} 198, 46. 64. 67, hinter \emph{suus} 206, 92 = 102, sonst im Wortinnern nur 206, 22 \emph{quamque viam h. l. \spation{quemque} tueri oportebit}; in allen übrigen Beispielen an zweiter Stelle, öfters freilich so, dass auf das Relativum zuerst das Substantiv, zu dem dasselbe als Attribut gehört, und dann erst \emph{quisque} folgt, z.~B. 206, 63 \emph{quo die \spation{quisque} triumphabit}, id. 147 \emph{quot annos \spation{quisque} eorum habet}, id. 26 \emph{qua in parte urbis \spation{quisque} eorum curet}, ebenso bei folgendem Genetiv z.~B. 200, 71 \emph{quantum agri loci quoiusque in populi leiberi — datus adsignatusve est}. Aber auch in diesen Beispielen ist die Voranstellung von \emph{quisque} vor die Wörter, zu denen es selbst im Attributivverhältnis steht: \emph{quisque eorum} (so auch sonst noch öfter), \emph{\spation{quoiusque} in populi leiberi}, nur aus unserm Stellungsgesetz begreiflich. Und insbesondere sind die Beispiele gar nicht selten, wo \emph{quisque} der Anfangsstellung zu lieb eine attributiv verbundene Wortgruppe spaltet: 199, 39 \emph{quem \spation{quisque} eorum agrum posidebit}; 202 I 33. 37. 41. II 5 \emph{quam in \spation{quisque} decuriam — lectus erit}; 202 II 27 \emph{qua in \spation{quisque} decuria est}. Die beiden letzten Beispiele zeigen, dass in Wortfolgen nach der Art von \emph{quam in decuriam} die Präposition als zum Relativum gehörig empfunden wurde. Ähnlich zerreisst \emph{quis\-que} auch etwa die Verbindung zwischen regierendem Substantiv und Genetiv, so \emph{quantum viae} in 206, 39 \emph{quantum \spation{quoiusque} ante aedificium viae — erit}, 204, 2, 23 \emph{quod \spation{quibusque} in rebus — iouris — fuit}. So die alten In-\hypertarget{p415}{\emph{[S.~416]}}\label{p416}schriften. Die übrige ältere Litteratur gibt ähnliches, darunter die beachtenswerte Tmesis \emph{quod \spation{quoique} quomque inciderit in mentem} (Terenz Heaut. 484). Allerdings ist \emph{quisque} allmählich auch orthotonischer Verwendung und der Stellung am Satzanfang fähig geworden. Noch viel mehr ist dies bei \emph{uterque} der Fall, dessen ursprüngliche Enklisis selbstverständlich ist und auch in Stellen wie Plaut. Menaechmi 186 \emph{in eo \spation{uterque} proelio potabimus} noch hervortritt. Andrerseits ist \emph{ubique} um so länger dem Ursprünglichen treu geblieben; Cicero in seinen Reden und ebenso Caesar haben es nicht nur immer in seiner eigentlichen Bedeutung “an jedem einzelnen Ort” verwendet, (— “überall” wird von beiden mit \emph{omnibus locis} gegeben —), sondern es auch immer an ein Relativum (Caesar de bello civ. 2, 20, 8 an interrogatives \emph{quid}) angelehnt.

Dass der andere Indefinitstamm des Latein, der mit \emph{u-}beginnende, überhaupt denselben Stellungsregeln wie der gutturale unterlag, zeigt, abgesehen von der unverkennbaren Neigung, die \emph{ullus, unquam, usquam} für die zweite Stelle haben, Festus 162\textsuperscript{b} 22.

\section*{XI.}
\addcontentsline{toc}{section}{XI.}

Unter den \spation{Partikeln} des Latein finden sich einige von jeher und immer an die zweite Stelle gefesselte: \emph{que, autem, ne}; einige, die zwischen erster und zweiter Stelle teils von Anfang an schwanken teils durch den wechselnden Gebrauch hin und her geschoben werden, wie die Beteuerungspartikeln, wie ferner \emph{enim, igitur}; endlich einige, bei denen Schwanken und Freiheit noch grösser ist: so \emph{tandem}. Alle diese Partikeln bewirken gelegentlich die beim Pronomen nachgewiesenen Tmesen; so z.~B. \emph{enim} die von \emph{cunque}: Ovid ex Ponto 4, 13, 6 \emph{qualis \spation{enim} cunque est}; \emph{igitur} und \emph{tandem} die von \emph{quomodo} und Genossen, auch von \emph{jusjurandum}: Cicero pro Cluentio 66 \emph{quonam \spation{igitur} haec modo gesta sunt}. pro Scauro 50 \emph{quocunque \spation{igitur} te modo}, de officiis 3, 104 \emph{jus \spation{igitur} jurandum}. Verrina 3, 80 \emph{quo \spation{tandem} modo}. Besonders tmetisch ist \emph{que}, insofern es nicht bloss in Fällen wie die oben genannten in solcher Weise wirkt (z. B. Cicero pro Caelio 54 \emph{juris\spation{que} jurandi}), sondern auch Präposition und Verbum (Festus 309\textsuperscript{a} 30 \emph{trans\spation{que} dato, endo\spation{que} plo-}\hypertarget{p417}{\emph{[S.~417]}}\label{p417}\emph{rato}; Plautus Trinummus 833 \emph{dis\spation{que} tulissent}) und Präposition und Kasus trennt, letzteres zumal in der Bedeutung ‘wenn’: altlateinisch \emph{abs\spation{que} me esset, abs\spation{que} te foret, abs\spation{que} una hac foret, abs\spation{que} eo esset} (Trinummus 832 mit freierer Wortfolge \emph{abs\spation{que} foret te}). Es ist kein Ruhm für die Latinisten, dass sie, nachdem von Schömann und Brugmann längst das Richtige gesagt ist, noch immer \emph{absque} als gewöhnliche Präposition ansehen mögen. Denn gesetzt auch, dass bei Cicero ad Atticum 1, 19, 1 wirklich \emph{absque argumento ac sententia} “ohne — Inhalt” zu lesen sei, was mir Wölfflin nicht bewiesen zu haben scheint, gesetzt also, dass die Bedeutung ‘ohne’ nicht auf einem Irrtum der Archaisten des zweiten Jahrhunderts beruhe, sondern schon der Umgangssprache der ciceronischen Zeit eigen gewesen sei, so konnte ja in der Zeit zwischen Terenz und Cicero die Phrase \emph{absque me esset} zunächst das Verb verlieren, so dass blosses \emph{absque me} als hypothetisches “ohne mich = wenn ich nicht gewesen wäre” gebraucht wurde: vergleiche Gellius 2, 21, 20 \emph{\spation{absque} te uno forsitan lingua Graeca longe anteisset, sed tu —} “ohne dich d. h. wenn du nicht gewesen wärest”, und Fronto 85, 24 N. \emph{\spation{absque} te, satis superque et aetatis et laboris} und infolge der Weglassung des Verbums sich dann weiter die hypothetische Bedeutung verflüchtigen, \emph{absque me} die Bedeutung “ohne mich” im Sinne von “indem ich nicht (dabei) bin” annehmen. Ganz ähnliche Entwicklungen lassen sich bei den Konzessivpartikeln nachweisen. (Vgl. über \emph{absque} im allgemeinen Praun in Wölfflins Archiv für latein. Lexikogr. VI 197—212).

Als ganz sichere Stützen unseres Stellungsgesetzes können indessen nur die Partikeln gelten, die nicht der Satzverbindung, sondern bloss der Qualifizierung des Satzes oder Satztheiles dienen, zu dem sie speziell gehören. Erstens \emph{\spation{quidem}}, das sich von indoiran. \emph{cid} formell nur durch den Zusatz von \emph{-em}, in der Funktion nur unwesentlich unterscheidet. Wie dieses kann es nicht hinter unbetonten Wörtern, besonders ursprünglich nicht hinter dem Verbum stehen (vgl., was \emph{cid} betrifft, Bartholomae in Bezzenbergers Beitr. XIII 73), und nimmt wie \emph{cid} je nach seiner Funktion entweder hinter dem ersten Wort des Satzes (beachte z.~B. Cic. Lael. 37 \emph{Tiberium \spation{quidem} Gracchum}) oder aber hinter demjenigen be-\hypertarget{p418}{\emph{[S.~418]}}\label{p418}tonten Wort seine Stellung, dessen Begriff (etwa eines Gegensatzes wegen) hervorgehoben werden soll. Besonders klar zeigt sich dieser Wechsel der Stellung bei der archaischen Zusammenordnung mit den Beteuerungspartikeln, namentlich mit \emph{hercle}. Unzähligemal findet sich \emph{quidem hercle} u. s. w. hinter dem ersten Wort des Satzes, oft aber auch \emph{hercle — quidem}. Nach Kellerhoff in Studemunds Studien a. d. G. d. archaischen Lateins II 64~f. sind die Beispiele letzterer Stellung teils durch metrische Lizenz zu entschuldigen, teils unerklärbar. Aber ohne Ausnahme zeigen sie \emph{quidem} hinter einem betonten Personale, Demonstrativum, \emph{si} oder \emph{nunc}: in allen diesen Fällen ist \emph{quidem} durch das auf \emph{hercle} und dergl. folgende Orthotonumenon angezogen worden. (Auch Plaut. Bach. 1194 \emph{tam pol id quidem}, welche Stelle bei Kellerhoff fehlt.)

An \emph{quidem} sei \emph{\spation{quŏque}} angeschlossen, das ich gleich altind. \emph{kva ca} setzen und ihm also als ursprüngliche Bedeutung ‘\emph{jederorts, jedenfalls}’ geben zu müssen glaube. Ein Wort mit der Bedeutung \emph{jedenfalls} war geeignet das Miteingeschlossensein eines Begriffs in eine Aussage auszudrücken; die archaische Verbindung von \emph{quoque} mit \emph{etiam} wird so auch ganz verständlich. Es liegt in der Funktion des Wortes, dass es, wie γε und z.~T. \emph{quidem}, trotz seiner Enklise an beliebigen Stellen des Satzes stehen kann, wo eben das Wort steht, dessen Begriff als hinzugefügt zu bezeichnen ist. Aber wie γε gelegentlich etwa (s. oben S.~371) der allgemeinen Gewohnheit der Enklitika folgend sich von seinem Wort weg zum Satzanfang entfernt, so auch \emph{quoque}: Varro de lingua lat. 5, 56 \emph{ab hoc \spation{quoque} quattuor partes urbis tribus dictae} (statt \emph{quattuor quoque}). 5, 69 \emph{quae ideo \spation{quoque} videtur ab Latinis Iuno Lucina dicta} (st. \emph{Iuno quoque}) [vgl. A. Spengel zu der St.]. 5, 181 \emph{ab eo \spation{quoque}, quibus —, tribuni aerarii dicti} (st. \emph{ab eo [ii] quoque quibus —}). 5, 182 \emph{aes \spation{quoque} stipem dicebant} (st. \emph{stipem quoque}). 8, 84 \emph{hinc \spation{quoque} illa nomina —} (st. \emph{illa nomina quoque}). Ebenso Properz 2, 34, 85 \emph{haec \spation{quoque} perfecto ludebat Iasone Varro} (st. \emph{Varro quoque}). 2, 34, 87 \emph{haec \spation{quoque} lascivi cantarunt scripta Catulli} (st. \emph{lascivi Catulli quoque}).

Bedeutsam scheint ferner die Stellung der Fragepartikel \emph{ne}, die ihrer Bedeutung wegen doch nicht mehr Anspruch hatte dicht beim Satzanfang zu stehen, als im Latein selbst \hypertarget{p419}{\emph{[S.~419]}}\label{p419} die Negation oder als im Deutschen z.~B. \emph{etwa} oder \emph{vielleicht}. Nur die Enklisis erklärt die übrigens längst anerkannte Regel, das [sic] \emph{ne} unmittelbar hinter das erste Wort des Satzes gehöre, von welcher Natur immer dasselbe auch sei. Es ist nicht meine Aufgabe, im Anschluss an Hand Tursellinus 4, 75~ff. und Kämpf De pronominum personalium usu et collocatione S.~42—46 (vgl. zu letzterm die Rezension von Abraham Berliner philologische Wochenschrift 1886, 227, welcher für Sätze wie Plautus Mostell. 362 \emph{sed ego sumne infelix?} Epidicus 503 \emph{sed tu novistin fidicinam Acrobolistidem?} Interpunktion hinter dem Pronomen verlangt) das gesamte Material zu durchgehen und die wirklichen und scheinbaren Ausnahmen zu besprechen. Es genüge darauf hinzuweisen, dass noch die klassische und spätere Sprache diese Regel kennt und darauf das seit Catull zu belegende \emph{utrumne} statt \emph{utrum — ne} zurückzuführen ist. Wie im nachhomerischen Griechischen τοιγάρ, weil man sich gewöhnt hatte darin nicht mehr einen selbständigen Satz, sondern das erste Wort eines Satzes zu erblicken, das bei Homer noch davon getrennte τοι an sich zog (s. oben S.~377), so \emph{utrum} aus gleichartigem Grunde das \emph{‑ne}.

Eine gewisse Abschwächung der alten Regel ist nur darin zu erkennen, dass, wenn eine aus Vordersatz und Nachsatz bestehende Periode durch \emph{ne} als interrogativ zu bezeichnen war, die klassische Sprache \emph{ne} erst im Nachsatz anzubringen pflegt, während in solchem Fall die alte Sprache \emph{-ne} gleich an das Fügewort des Vordersatzes anknüpfte. Mit letzterm hängt der häufige Gebrauch zusammen, in einem Relativsatz \emph{ne} an das Relativum anzuhängen und dann mit solchem Relativsatz ohne Beifügung eines Hauptsatzes zu fragen, ob die im vorausgehenden Satz gegebene Aussage für den im Relativsatz beschriebenen Begriff gelte. Auch andere Nebensätze finden sich so verwendet. (Vgl. zu dem allem Brix zum Trinummus 360. Lorentz zum Miles 965, zur Mostellaria 738.)

Von da aus wird m. E. eine bisher falsch erklärte Partikel verständlich. Ribbeck Beiträge zur Lehre v. d. latein. Partikeln (1869) S.~14~f. leitet unter dem Beifall von Schmalz Lateinische Grammatik (Iwan Müllers Handbuch der klass. Altertumswiss. II) \textsuperscript{2} 526 \emph{sin} “wenn aber” aus einer Verbindung von \emph{si} mit der Negation \emph{ne} her. Die dieser Herkunft entsprechende Bedeutung “wenn nicht” zeige sich noch an \hypertarget{p420}{\emph{[S.~420]}}\label{p420} Stellen wie Cic. Att. 16, 13\textsuperscript{b} 2 \emph{si pares aeque inter se, quiescendum; sin, latius manabit, et quidem ad nos, deinde communiter}. Zu \emph{sin} habe man dann auch noch oft “tautologisch oder hinüberleitend” \emph{aliter, secus, minus} hinzugefügt; auch, wenn der durch solches \emph{sin} “wenn nicht” angedeutete andere Fall bestimmter zu formulieren war, dies in der Form einfacher Parataxis gethan. So sei \emph{sin} schliesslich eine gewöhnliche adversative Konjunktion geworden.

Gegen diese Erklärung können mehrere Einwendungen erhoben werden. Ich will die Möglichkeit, dass es ein \emph{sin} “wenn nicht” geben konnte, nicht bestreiten, da \emph{quin} zeigt, dass die Negation \emph{ne} enklitisch werden und ihren Vokal verlieren konnte. (Jedenfalls gehört \emph{sine} nicht hierher, sondern ist = indog. \emph{sn̥nḗ}, d. h. alter Lokativ von\emph{ senu-}, und der Hauptsache nach mit ἄνευ gleichzusetzen, mit welchem got. \emph{inu}, ahd. \emph{āno} nichts zu thun haben, da diese altindischen \emph{anu, ānu} = indog. \emph{enu, ēnu} entsprechen. Die hiefür anzunehmende Bedeutungsentwickelung “\emph{entlang, längs}” — “\emph{praeter}” — “\emph{ohne}” ist durchaus natürlich.) Aber dass \emph{sin} ursprünglich diese Bedeutung “wenn nicht” wirklich gehabt habe, dafür fehlt es völlig an Belegen. Denn diejenigen Beispiele, die Ribbeck teils beibringt, teils im Auge hat, in diesem Sinne zu verwenden, ist von vorn herein schon darum bedenklich, weil man nicht versteht wie die zu Plautus Zeit bereits verflüchtigte negative Bedeutung in ciceronischer Zeit wieder so lebendig sein konnte. Und sieht man die Beispiele selbst an, so ergiebt sich, dass sie das nicht beweisen, was sie beweisen sollen. Cicero Epist. 12, 6, 2 \emph{qui si conservatus erit, vicimus; \spation{sin} —, quod di omen avertant, omnis omnium cursus est ad vos}. 14, 3, 5 \emph{si perficitis quod agitis, me ad vos venire oportet; \spation{sin} autem —. Sed nihil opus est reliqua scribere}. ad Att. 10, 7, 2 \emph{si vir esse volet, praeclare} ϲυνοδία. \emph{Sin \spation{autem}, erimus nos, qui solemus}. 13, 22, 4 \emph{atque utinam tu quoque eodem die! sin \spation{quod} —, multa enim utique postridie}. 16, 13\textsuperscript{b} 2 s. oben. — Priap. 31 \emph{donec proterva nil mei manu carpes, licebit ipsa sis pudicior Vesta. \spation{Sin}, haec mei te ventris arma laxabunt}. Dazu käme nach einer Konjektur Vahlens Tibull 1, 4, 15 \emph{\spation{sin}} (Codd. \emph{sed}), \emph{ne te capiant, primo si forte negabit, taedia}; doch wird diese Schreibung wohl kaum allgemein rezipiert werden. (Schmalz spricht auch \hypertarget{p421}{\emph{[S.~421]}}\label{p421} von Belegen im alten Latein, doch finde ich nirgends solche nachgewiesen.) An allen diesen Stellen liegt einfach eine Aposiopese vor, wie solche dem Priapeen- und dem Briefstil ziemt. Besonders die beiden ersten Stellen mit ihrem \emph{quod di omen avertant} und \emph{sed nihil opus est reliqua scribere} schliessen jeden Zweifel aus.

Mit dem Wegfall dieser Stellen ist aber der Ribbeckschen Hypothese dasjenige entzogen, was sie besonders empfahl, die Anknüpfung an einen thatsächlichen Sprachgebrauch. Nun könnte die Hypothese freilich trotzdem richtig sein, \emph{sin} in der, hinter der litterarischen Überlieferung zurückliegenden Zeit zuerst “wenn nicht” bedeutet und sich dann zu der historisch allein bezeugten Bedeutung “wenn aber” entwickelt haben. Aber auch diese Entwicklung ist nicht so leicht konstruierbar. Ribbeck äusserst sich nur sehr kurz über diesen Punkt. Wenn ich ihn recht verstehe, so meint er, ein Satz wie z. B. Plautus Trin. 309 \emph{[si animus hominem pepulit, actumst, animo servit, non sibi.] sin ipse animum pepulit, vivit} sei ursprünglich so gemeint gewesen, dass man hinter \emph{sin} “wenn nicht” “wenn dies nicht der Fall ist” interpungiert hätte und darauf asyndetisch die genauere Bezeichnung des gegenteiligen Falles hätte folgen lassen: \emph{ipse animum pepulit} “[im Falle dass] er selbst seinen Neigungen die Richtung gegeben hat”, schliesslich die Apodosis \emph{vivit}. Mir schiene ein Asyndeton, wie das hier zwischen \emph{sin} und dem folgenden statuierte, undenkbar: \emph{sed} (oder eine Wiederholung des \emph{si}) wäre doch wohl unerlässlich. Wohl gibt es ein Asyndeton adversativum, aber nur in der Weise, dass der Gegensatz dabei auf andere Weise fühlbar gemacht wird, durch parallele Gestaltung der beiden Glieder oder durch Voranstellung des Wortes, das den Gegensatz hauptsächlich trägt im zweiten Gliede.

Ich glaube, es bietet sich ein viel einfacherer Weg. Brix giebt zum Trinummus 360 unter den Beispielen des an das Fügewort des Vordersatzes angeschlossenen \emph{ne} am Schluss folgende Stelle des Mercator 142~f.: Acanthio: \emph{At ego maledicentiorem quam te novi neminem}. Charinus: \emph{\spation{Sin} saluti quod tibi esse censeo, id consuadeo?} Acanthio: \emph{apage istiusmodi salutem, cum cruciatu quae advenit}. Brix umschreibt die Worte des Charinus mit \emph{tum\spation{ne} maledicentem me dicis, \spation{si} tibi id consuadeo}. Offenbar ganz gemäss der Weise plau-\hypertarget{p422}{\emph{[S.~422]}}\label{p422}tinischen Konversationsstils, wo Fragesätze, die als solche durch \emph{-ne} bezeichnet sind, ausserordentlich oft für Einwendungen dienen z.~B. Bacchides 1189 \emph{egon ubi filius corrumpatur meus, ibi potem?} 1192 \emph{egon quom haec cum illo accubet, inspectem?} Trin. 378 \emph{egone indotatam te uxorem ut patiar?} Bacch. 194 \emph{at scin quam iracundus siem?} Besonders häufig sind in dieser Weise die \emph{ne}-Sätze gebraucht, wo der Fragesatz elliptisch nur aus einem Nebensatz mit \emph{ne} besteht, also gerade die \emph{ne}-Sätze, zu denen obiges Beispiel gehört. Amphitr. 297 Sosia: \emph{paulisper mane, dum edormiscat unum somnum}. Amph.: \emph{quaene vigilans somniat?} “aber dann träumt sie ja mit offenen Augen.” Curculio 704~f. Cappadox: \emph{dum quidem hercle ita iudices, ne quisquam a me argentum auferat}. Therapontigonus: \emph{quodne promisti?} “aber du hast es ja versprochen”. Rudens 1019 \emph{quemne ego excepi in mari?} “aber ich habe ihn ja im Meere aufgefangen”. 1231 \emph{quodne ego inveni in mari?} “aber ich habe es ja im Meere gefunden.” Terenz Phormio 923 Demipho: \emph{illud mihi argentum rursum iube rescribi Phormio}. Phormio: \emph{quodne ego discripsi porro illis, quibus debui?} “aber ich habe es ja meinen Gläubigern gutgeschrieben.”

Ein zweite Stelle, wo \emph{sin} so steht, ist Persa 227: Paegnium: \emph{ne me attrecta subigitatrix}. Sophoclidisca: \emph{\spation{sin} te amo?} Paegnium: \emph{male operam locas}.

Die meisten Plautusleser werden freilich an beiden Stellen das \emph{sin} einfach mit “wenn aber” übersetzen und darin das gewöhnliche \emph{sin} erkennen. Weit entfernt dies tadeln zu wollen, erkenne ich darin gerade einen Beweis dafür, dass das gewöhnliche \emph{sin} mit dem \emph{sin} jener plautinischen Stellen identisch ist. Wir können nicht bloss andern, sondern auch uns selbst einen Einwurf in der Form eines Fragesatzes machen. In solcher Weise steht einwendendes \emph{quine, quemne} Catull 64, 180 \emph{an patris auxilium sperem? quemne ipsa reliqui—?} “aber den habe ich ja verlassen”. 182~f. \emph{coniugis an fido consoler memet amore? quine fugit lentos incurvans gurgite remos?} “aber der flieht ja” (s. oben die Übersetzung von \emph{quine} in den Beispielen aus Plautus und Terenz). Und wie an den beiden plautinischen \emph{sin}-Stellen auf die vom zweiten Sprecher als Einwendung gebrachte Möglichkeit der erste Sprecher zur Beseitigung der Einwendung als asyndetisch an-\hypertarget{p423}{\emph{[S.~423]}}\label{p423}gefügte Apodosis dasjenige giebt, was in dem betr. Fall eintreten würde: \emph{apage istiusmodi salutem} “\spation{dann} fort mit solchem Heil”, und \emph{male operam locas} “nun \spation{dann} verschwendest du deine Mühe” —, so kann man auch eine selbstgemachte Einwendung selbst mit derartiger Apodosis erledigen.

Demgemäss würde an der oben nach der Ribbeckschen Hypothese analysierten Plautusstelle der ursprüngliche Gebrauch von \emph{sin} hergestellt durch die Interpunktion: \emph{sin ipse animum pepulit? vivit}. “Wie aber, wenn er selbst seinen Neigungen die Richtung gegeben hat? Nun dann lebt er.” Dass im Verlauf die eigentlich für Einwendungen aufgekommene Satzform überhaupt für Setzung eines entgegengesetzten Falls verwendet, und dass im Zusammenhang damit der \emph{sin}-Fragesatz schlechtweg als Vordersatz, der ursprüngliche Antwortsatz schlechtweg als Nachsatz empfunden wurde, ist eine ganz natürliche Entwicklung.

Wenn Lucian Müller Lucil. 29, Fr. 87, V. 107 (vgl. zu Nonius 290, 4) richtig schreibt \emph{ad non sunt similes neque dant. quid? sin} (codd. \emph{sint}, ed. princ. Non. \emph{si}) \emph{dare vellent? acciperesne? doce}, so tritt hiermit zu den zwei loci didascalici des Plautus ein dritter. Denn auch hier dient \emph{sin} einem Einwand, mit dem Unterschied, dass derselbe durch \emph{quid} angekündigt ist, und dass ein die Frage näher präzisierender \emph{ne}-Satz folgt. Nach Lucian Müller ist es ein Einwand, den einer sich selbst macht. — Das \emph{quodsin ulla} (Lucil 4 Fr. 22 Vs. 38) desselben Gelehrten st. \emph{quodsi nulla} mit unerklärbarem \emph{-sin} wird durch richtige Schreibung der folgenden Zeile überflüssig.

Den Beschluss mögen die Beteuerungs- und Verwunderungspartikeln, \emph{hercle, pol, edepol, ecastor, eccere} bilden, die die Eigentümlichkeit haben, bald die erste bald die zweite Stelle im Satz einzunehmen, weiter hinten aber nicht stehen zu können, ausser wenn ihnen andre Enklitika, wie \emph{quidem, autem} (Aulul. 560), \emph{obsecro, quaeso, credo}, oder \emph{ego, tu, ille} hinter \emph{ne}, oder \emph{tu} hinter \emph{et, at, vel}, kraft eignen Anspruchs auf diese Stelle den Platz versperren. Wie stark der Drang nach der zweiten Stelle auch bei dieser Wortklasse ist, zeigt sich an manchem. So daran, dass während die Verbindung \emph{pol ego} bald am Satzanfang steht, bald ihr noch ein anderes Wort vorangeht und also \emph{ego} gleich gern an dritter wie an zweiter Stelle des Satzes steht, das umgekehrte \emph{ego pol} nur \hypertarget{p424}{\emph{[S.~424]}}\label{p424} am Satzanfang vorkommt (Kellerhoff in Studemunds Studien, a. d. G. d. arch. Latein II 62), \emph{pol} also die dritte Stelle scheut. So daran, dass die Beteuerungspartikeln, wenn sie sich auf eine ganze Periode beziehen, dem ersten Wort des Vordersatzes angefügt werden; \emph{si hercle, si quidem hercle, ni hercle, postquam hercle, si ecastor, si pol, si quidem pol} sind ganz gewöhnlich, während die Setzung von \emph{hercle} erst im Nachsatz zwar nicht unerhört (siehe Mil. Glor. 309, Persa 627), aber selten ist. (Vgl. Brix zum Trinumm. 457, Lorentz zum Miles 156. 1239, zur Mostell. 229, Kellerhoff Studien II 72~f.) Genau die gleiche Erscheinung haben wir beim fragenden \emph{-ne} getroffen. Aber während bei \emph{-ne} diese Stellung auf die alte Sprache beschränkt ist, lebt sie bei \emph{hercle, (hercules)} in der klassischen Sprache fort (Müller zum Laelius § 78\textsuperscript{2} S.~477, der auf Wichert Latein. Stilistik S.~43, 239, 269 verweist. Weissenborn zu Livius 5, 4, 10 u. s. w.), wie denn die klassische Sprache überhaupt die traditionelle Stellung der Partikel \emph{hercle}, der einzigen, die eben in die klassische Sprache fortlebt, festhält, immerhin so, dass die Setzung derselben an die Spitze des Satzes ausser Gebrauch kommt. Die Kaiserzeit gestattet sich dann freilich grössere Willkür: Quintil. 1, 2, 4. Tacitus Dial. 1. Histor. 1, 84. Plin. Epist. 6, 19, 6. Gell. 7, 2, 1 u. s. w.

Ferner veranlassen auch diese Partikeln, wie die früher besprochenen Enklitika, öfters Tmesis. Dahin gehört neben Miles Glor. 31 \emph{ne \spation{hercle} operae pretium quidem} (gegenüber Bacchides 1027 \emph{ne unum quidem \spation{hercle}}) und Mostell. 18 \emph{cis \spation{hercle} paucas tempestates} und \emph{non edepol scio} gegenüber \emph{nescio} besonders die Spaltung der Zusammensetzungen mit \emph{per}: Plautus Casina 370 \emph{per \spation{pol} saepe peccas}. Terenz Andria 416 \emph{per \spation{ecastor} scitus puer est natus Pamphilo}. Hecyra 1 \emph{per \spation{pol} quam paucos}. Gellius 2, 6, 1 \emph{per \spation{hercle} rem mirandam Aristoteles — dicit}, und die Spaltung von \emph{quicunque}: Plautus Persa 210 \emph{quoi \spation{pol} quomque occasio est}.

Also \emph{hercle} und Genossen haben entweder die erste oder die zweite Stelle im Satz inne; sie werden, wenn sie nicht stark betont am Anfang stehen, nach Art der Enklitika behandelt. Wer nun bedenkt, dass diese Partikeln eigentlich Vokative sind (vgl. Catull 1, 7 \emph{doctis \spation{Juppiter} et laboriosis}), wird sich sofort jener eigentümlichen Regel der Sanskritgram-\hypertarget{p425}{\emph{[S.~425]}}\label{p425}matiker und Überlieferer der akzentuierten Vedentexte erinnern, dass der Vokativ, wenn am Satzanfang stehend, orthotoniert, wenn im Satzinnern stehend, enklitisch sei. (Vgl. die Erklärung, die Delbrück Syntakt. Forsch. V~34~f. dafür gibt.) Es kommt hinzu, dass, wenigstens in den klassischen Sprachen, auch der wirkliche Vokativ unverkennbare Neigung für die zweite Stelle im Satz zeigt.

Nun macht freilich gerade der Umstand Schwierigkeit, dass was bei den vokativischen Partikeln Gesetz ist, sich beim wirklichen Vokativ nur als Neigung zeigt. Kaum darf man wohl annehmen, dass solche Neigung Abschwächung eines ältern strengern Gesetzes war. Viel wahrscheinlicher ist das Umgekehrte, dass bei der durch \emph{hercle} repräsentierten Kategorie von Vokativen die Neigung zur Regel geworden war, und dass sich die Anrufung eines Gottes zum Zweck der Beteuerung früh in strengerer Konventionalität bewegte, als sonstige Anrufungen von Göttern und gar als Anreden an Menschen. (Das Griechische verfährt in der Stellung des entsprechenden Ἡράκλειϲ und ähnlicher Anrufungen, soweit der Gebrauch der Komiker und der Redner ein Urteil gestattet, mit gros\-ser Freiheit.) Daraus folgt aber weiter, wenn wir anders bei den Vokativen innern Zusammenhang zwischen Stellung und Betonung annehmen dürfen, dass die altindische Enklisis von Hause aus nur Neigung, nicht unbedingtes Gesetz war, und dass gelegentlich auch der nicht am Satz- oder Versanfang stehende Vokativ orthotoniert sein konnte, was dann dem Altindischen vermöge seines Generalisierungstriebs verloren ging.

Es entgeht mir nicht, dass die Neigung des Vokativs für die zweite Stelle auch ohne Hinzunahme der alten Enklisis erklärt werden könnte. Um so wertvoller ist mir, dass von ganz anderm Standpunkt der Betrachtung aus Schmalz Lateinische Syntax\textsuperscript{2} S. 557 für den an zweiter Stelle stehenden Vokativ des Latein schwachen Ton behauptet.

\section*{XII.}
\addcontentsline{toc}{section}{XII.}

Unsere neuhochdeutsche Regel (vgl. Erdmann Grundzüge der deutschen Syntax S. 181 ff., besonders 195), dass dem Verbum im Hauptsatz die zweite, im Nebensatz die letzte Stelle zu geben sei (beides mit bestimmten, in besondern Verhält-\hypertarget{p426}{\emph{[S.~426]}}\label{p426}nissen begründeten Ausnahmen) hat bekanntlich der Hauptsache nach schon in der althochdeutschen Prosa und Poesie gegolten. (Vgl. ausser den Nachweisen Erdmanns besonders Tomanetz Die Relativsätze bei den ahd. Übersetzern des 8. und 9. Jahrhunderts, S. 54 ff., sowie denselben im Anzeiger für deutsches Altertum XVI (1890) 381.) Ja diese Stellungsregel kann in Rücksicht auf die deutlichen Spuren, die sich von ihr nicht bloss im Altsächsischen, sondern auch im Angelsächsischen, und weiterhin auch im Nordischen zeigen, wohl als gemein germanisch angesetzt werden. Trotzdem sind alle Forscher, die sich eingehender mit diesem germanischen Stellungsgesetz beschäftigt haben, so viel ich sehe, darin einig, die sich hier äussernde Scheidung der beiden Satzarten für unursprünglich zu erklären. Bergaigne (Mémoires Soc. de Linguistique III 139 f.), Behaghel (Germania XXIII 284) und Ries (Die Stellung von Subjekt und Prädikatsverbum im Heliand, Quellen und Forschungen XLI [1880] S. 88 ff.) behaupten, dass die Endstellung des Verbums, wie sie im Nebensatz vorliegt, ursprünglich allen Sätzen eigen gewesen und in den Hauptsätzen nur allmählich durch eine später aufgekommene entgegengesetzt wirkende Regel verdrängt worden sei. Über das Wie und die Möglichkeit einer solchen Verdrängung haben sich aber die genannten Forscher teils nicht ausgesprochen, teils haben sie dafür Gründe beigebracht, die mit Scharfsinn ausgedacht aber alles eher als überzeugend sind: wie wenn z. B. Ries behauptet, der natürliche Trieb, das Wichtigere vor dem weniger Wichtigen zum Ausdruck zu bringen, habe darum nur im Hauptsatz und nicht auch im Nebensatz zur Annäherung des Verbums an den Anfang führen müssen, weil das Verb für den Hauptsatz einen höhern Wert habe, als für den Nebensatz!

Den entgegengesetzten Standpunkt vertritt Tomanetz (a. a. O. S. 82 ff.): er glaubt, erst durch eine allmähliche Verschiebung sei das Verb im Nebensatz ans Ende gerückt; ursprünglich habe es auch hier wie im Hauptsatz die zweite Stelle inne gehabt. So sehr sich auch Tomanetz’ Ausführungen vor denen von Ries durch Einfachheit und Klarheit auszeichnen, vermag er doch nicht ohne die m.~E. völlig unzulässige Annahme durchzukommen, dass ein Streben Haupt- und Nebensatz zu differenzieren wirksam gewesen sei.

\hypertarget{p427}{\emph{[S.~427]}}\label{p427} Altindisch, Latein und Litauisch stellen das Verbum regelmässig ans Ende des Satzes. Man glaubt hierin eine Gewohnheit der Grundsprache erkennen zu können. Und gewiss wird für den Nebensatz durch das hier hinzukommende Zeugnis des Germanischen die Endstellung des Verbums als indogermanisch gesichert. Beim Hauptsatz fehlt diese Übereinstimmung und, wenn sonstige Erwägungen nicht den Entscheid geben, ist es zum mindesten ebenso gut denkbar, dass im Altindischen, Lateinischen und Litauischen etwas bloss für den Nebensatz Gültiges auf den Hauptsatz ausgedehnt worden sei, als dass das Germanische nachträglich eine Unterscheidung der beiden Satzarten eingeführt habe. Nun ist es aber ganz unwahrscheinlich, dass die Grundsprache das Verbum im Hauptsatz und im Nebensatz verschieden \spation{betont}, aber doch in beiden Satzarten gleich \spation{gestellt} hätte. Und weiterhin müssen wir auf Grund des früher Vorgetragenen erwarten, dass in der Grundsprache das Verbum des Hauptsatzes, weil und insofern es enklitisch war, unmittelbar hinter das erste Wort des Satzes gestellt worden sei. Mit andern Worten: das deutsche Stellungsgesetz hat schon in der Grundsprache gegolten. Dabei muss man sich gegenwärtig halten, dass nicht bloss die Sätze, die wir als Nebensätze ansehen, sondern alle als hypotaktisch empfundenen im Altindischen und somit, wie wir wohl annehmen dürfen, in der Grundsprache betontes Verbum hatten, also unter allen Umständen die Endstellung des Verbums sehr häufig vorkommen musste.

Ich will nicht verschweigen, dass die aufgestellte These einer Einschränkung fähig wäre. Für das Gesetz über die Stellung der Enklitika haben wir aus den verschiedenen Sprachen (etwa von den Vokativen abgesehen) nur solche Belege beibringen können, in denen das Enklitikum den Umfang von zwei Silben nicht überschritt.  Man könnte also sagen, dass das Gesetz nur für ein- und zweisilbige Enklitika galt, mehr als zweisilbige dagegen an der dem betr. Satzteil sonst zukommenden Stellung festhielten. [sic] oder wenigstens, wenn man sich vorsichtiger ausdrücken will, dass von irgend einem bestimmten Umfang an ein Enklitikum nicht an das Stellungsgesetz der Enklitika gebunden war. Dies auf das Verbum angewandt, würde zu der Annahme führen, dass die ein- und zweisilbigen Verbalformen, oder überhaupt die kürzern Verbal-\hypertarget{p428}{\emph{[S.~428]}}\label{p428}formen bis zu einem gewissen Umfang, im Hauptsatz an die zweite Stelle rückten, dass dagegen die andern Verbalformen auch im Hauptsatz die im Nebensatz herrschende Endstellung besassen. Es wäre dann weiter anzunehmen, dass das Germanische die für die kürzern Verbalformen gültige Regel generalisiert hätte. Und jedenfalls wäre dann die Praxis der das Verb überhaupt an das Ende stellenden Sprachen noch leichter verständlich.

Man wird nicht verlangen, dass ich über die Berechtigung dieser eventuellen Einschränkung meiner These ein abschliessendes Urteil abgebe. Wohl aber wird man erwarten, dass ich ein wenig weitere Umschau halte und frage, ob denn das verbale Stellungsgesetz der Grundsprache ausserhalb des Germanischen gar keine Spuren hinterlassen habe. Das Fehlen aller Anklänge an ein solches Gesetz könnte leicht Zweifel an der Richtigkeit der hier gegebenen Ausführungen rege machen.

Nun, da muss allerdings gesagt werden, dass ausser den bereits erwähnten, die Endstellung durchführenden Sprachen nicht bloss das Keltische, sondern, was bei einer derartigen Untersuchung weit schwerer ins Gewicht fällt, auch das Griechische der germanischen Weise fern steht. Man sollte erwarten, dass das Griechische, wie und weil es beim Verbum den Hauptsatz-\spation{Akzent} durchgeführt hat, so auch die Hauptsatz-\spation{Stellung} durchführen werde. Aber das ist bekanntlich nicht der Fall. Die Stellung des Verbums ist im Ganzen eine sehr freie.

Solchem Sachverhalt gegenüber ist es zunächst willkommen, dass gerade zwei die Endstellung bevorzugende Sprachen in einem bestimmten Fall die germanische Hauptsatzstellung aufweisen. Für das Litauische lehrt Kurschat Grammatik § 1637, dass, wenn das Prädikat aus Kopula und Nomen bestehe, gegen die allgemeine Regel nicht das Nomen vorausgehe, sondern die Kopula unmittelbar auf das Subjekt folge. Ganz ähnliches findet sich beim Verbum \emph{esse} im Latein. Seyffert zu Ciceros Laelius 70 (S. 441\textsuperscript{2}) hat ausgeführt, dass \emph{esse} sich gern an das erste Wort des Satzes anlehne, sowohl wenn dasselbe ein interrogativ oder relativ fungierenden [sic] Interrogativpronomen, als wenn es ein Demonstrativum sei oder sonst einer Wortklasse angehörte. Der Beispiele seien \hypertarget{p429}{\emph{[S.~429]}}\label{p429} ‘unzählig’ viele. Aus dem Laelius führt er unter anderm an: § 56 \emph{qui \spation{sint} in amicitia} (Interrog.). 17 \emph{quae \spation{est} in me facultas} (Relat.). 2 \emph{quanta \spation{esset} hominum admiratio}. 53 \emph{quam \spation{fuerint} inopes amicorum}. 83 \emph{eorum \spation{est} habendus}. 5 \emph{tum \spation{est} Cato locutus}. 17 \emph{nihil \spation{est} enim}. 48 \emph{ferream \spation{esse} quandam}. 102 \emph{omnis \spation{est} e vita sublata iucunditas}.

Zu dieser Beobachtung stimmt eine weitere Erscheinung: in einem Satz, der sowohl \emph{est, sunt} als \emph{enim, igitur, autem} enthält, werden namentlich bei Cicero überaus oft nicht diese Partikeln trotz ihres sonst anerkannten Anspruchs auf die zweite Stelle, sondern \emph{est, sunt} an das erste Wort des Satzes angelehnt und \emph{enim, igitur, autem} auf die dritte Stelle zurückgedrängt. Das Richtige darüber hat Madvig gesagt zu Cicero de finibus 1, 43: ea est huius positus \emph{(sapientia est enim)} ratio, ut elata voce in primo vocabulo, quo gravissima notio contineatur, obscuretur enclitica; in altero positu \emph{[sapientia enim est]} vox minus in primum vocabulum incidit. — Hanc regulam contrariam prorsus Goerenzii aliorumque praeceptis, qui naturam encliticae vocis ignorantes, adseverationem aliquam in \emph{est} secundo loco posito inesse putarunt adhibito optimorum codicum testimonio — et recta interpretatione stabilitum iri puto. (Vgl. Müller zum Laelius\textsuperscript{2} S. 411.)

Zur weitern Bestätigung könnte man auf Stellen wie Plaut. Bacch. 274 \emph{etiamne \spation{est} quid porro} verweisen, wo die Stellung von \emph{quid} enklitische Stellung von \emph{est} voraussetzt. Besonders finden sich aber bei \emph{esse} ähnliche Tmesen, wie bei den früher besprochnen Enklitika: solche von \emph{per-} bei Cicero epistul. 3, 5, 3 (51~a.~Ch.) \emph{tunc mihi ille dixit: quod classe tu velles decedere, per \spation{fore} accommodatum tibi, si ad illam maritimam partem provinciae navibus accessissem} und bei Gellius 2, 18, 1 \emph{Phaedo Elidensis ex cohorte illa Socratica fuit Socratique et Platoni \spation{per} fuit familiaris}, wo die fehlerhafte Anwendung solcher Tmesis mitten im Satzinnern den Archaisten verrät. Tmesis von \emph{qui — cunque}: Terenz Andria 63 \emph{cum quibus \spation{erat} quomque una, eis se dedere}. Cicero de finibus 4, 69 \emph{quod \spation{erit} cunque visum, ages}. Dazu bei einer Form von \emph{fieri}: Plautus Bacchides 252 \emph{istius hominis ubi \spation{fit} quomque mentio}.

Wenn das Latein nur bei ein, zwei Verben, wo sich die Tradition ursprünglicher Enklisis lebendig erhalten hatte, An-\hypertarget{p430}{\emph{[S.~430]}}\label{p430}lehnung an das erste Satzwort kennt (und bei diesem dann natürlich in allen Satzarten), so zeigt sich im Griechischen ein solcher Rest alter Stellungsgewohnheit bei einer ganzen Anzahl von Verben, aber nur in einer bestimmten Satzform. Auf altgriechischen Inschriften finden sich oft Sätze, wo auf das Subjekt, obwohl eine appositionelle Bestimmung dazu gehört, doch zuerst das Verbum und dann erst die appositionelle Bestimmung folgt, diese also in auffälliger Weise von dem Wort, zu dem sie gehört, durch das Verbum abgetrennt ist. Dass statt eines Subjektsnominativs auch etwa ein andrer Kasus, der an der Spitze des Satzes steht, in solcher Weise von seiner Apposition getrennt wird, und dass gelegentlich ein με dem Verbum noch vorgeschoben wird, macht keinen Unterschied. Boeckh zu CIG. 25 hat zuerst die Altertümlichkeit dieser Art von Wortstellung, Wilhelm Schulze in seiner Rezension von Meisters griech. Dialekten, Berliner philolog. Wochenschrift 1890, S.~1472 (S.~26~f. des Separatabdrucks) die sprachgeschichtliche Bedeutung derselben betont. Es wird nicht undienlich sein, hier die Beispiele zusammenzustellen.

Am häufigsten findet sich diese Stellung in Weih- und Künstlerinschriften. Mit \spation{ἀνέθηκε}: CIA. 1, 357 Ἀλκίβιοϲ ἀνέθηκεν κιθαρῳδὸϲ νηϲιώτηϲ. 1, 376 Ἐπιχαρῖνοϲ [ἀνέ]θηκεν ὁ Ὀ—. 1, 388 Στρόνβ[ιχοϲ ἀνέθηκε] Στρονβί[χου oder — χίδου Εὐ\-ω\-νυ\-μεύϲ] (fast sichere Ergänzung!). 1, 399 Μηχανίω[ν] ἀνέθηκεν ὁ γραμμα[τεύϲ]. 1, 400 [Πυ]θογέν[εια] ἀνέθηκε[ν Ἀγ]υρρίου ἐγ [Λ]ακιαδῶ[ν]. 1, 415 Αἰϲχύλοϲ ἀνέθη[κε] Πυθέου Παιανιεύ[ϲ]. 4\textsuperscript{1}, 373 f. Σίμων ἀ[νέθηκε] ὁ κναφεὺϲ [ἔργων] δεκάτην. 4\textsuperscript{2}, 373, 90 Ὀνήϲιμόϲ μ᾽ ἀνέθηκεν ἀπαρχὴν Ἀθηναίᾳ ὁ Σμικύθου υἱόϲ. 4\textsuperscript{2}, 373, 198 [ἡ δεῖνα ἀνέθηκεν] Εὐμηλίδου γυνὴ Σφηττόθεν. 4\textsuperscript{2}, 373, 12 Ξενοκλέηϲ ἀνέθηκεν Σωϲίνεω. 4\textsuperscript{2}, 373, 223 Χναϊάδηϲ ἀνέθηκεν ὁ Παλ(λ)ηνεύϲ. 4\textsuperscript{2}, 373, 224 [Σ]μῖκροϲ ἀνέθ[ηκε —] ὁ ϲκυλοδεψ[όϲ]. 4\textsuperscript{2}, 373, 226 [ὁ δεῖνα ἀνέθηκε]ν Κηφιϲιεύϲ. Inschrift von der Akropolis Νέαρχοϲ ἀν[έθηκε Νεάρχου υἱ]ὺϲ ἔργων ἀπαρχήν. So nach Kabbadias Studnitzka, Jahrbuch II (1887), S.~135~ff.; Robert: Νέαρχοϲ ἀν[έθηκε ὁ κεραμε]ύϲ —. CIA. 2, 1648 (augusteische Zeit!) Μετρότιμοϲ ἀνέθηκεν Ὀῆθεν. — Inscript. graecae antiq. 48 Ἀριϲτομένηϲ ἀ[ν]έθ[ηκ]ε Ἀλεξία τᾷ Δάματρι τᾷ Χθονίᾳ Ἑρμιονεύϲ. 96 (Tegea) [ὁ δεῖνα ἀνέ]θηκε(ν) ϝαϲτυόχω. 486 (Milet) [Ἑρ]μηϲιάναξ ἥμεαϲ ἀνέθηκεν [ὁ —] — ίδεω τὠπόλλωνι. 512\textsuperscript{a} (Gela) Παντάρηϲ μ᾽ \hypertarget{p431}{\emph{[S.~431]}}\label{p431} ἀνέθηκε Μενεκράτιοϲ. 543 (achäisch) Κυνίϲκοϲ με ἀνέθηκε ὥρταμοϲ ϝέργων δεκάταν. — Delphische Inschrift in westgriech. Alphabet, Bull. Corr. Hellén. 6, 445 τοὶ Χαροπίνου παῖδεϲ ἀνέθεϲαν τοῦ Παρίου. Naxische Inschrift von Delos ed. Homolle ibid. 12, 464~f., 12, 464~f. Εἰ(θ)υκαρτίδηϲ μ᾽ ἀνέθηκε ὁ Νάξιοϲ ποιήϲαϲ. — Inschriften von Naukratis I No.~218 Φάνηϲ με ἀνέθηκε τὠπόλλων[ι τῷ Μι]ληϲίῳ ὁ Γλαύκου. II No.~722 Μυϲόϲ μ’ ἀνέθηκεν Ὀνομακρίτου. 767 [ὁ δεῖνα ἀνέθηκεν Ἀφροδ]ίτῃ ὁ Φ[ιλά]μμ[ωνοϲ]. 780 Φίλιϲ μ᾽ ἀνέθηκε οὑπικά[ρτε]οϲ τῇ Ἀφροδί[τῃ]. 784 Ἑρμοφάνηϲ ἀνέθ[ηκεν] ὁ Ναυ\-ϲι\-τέ[\-λευϲ]. 819 [Λ]άκρι[τό]ϲ μ᾽ ἀνέ[θη]κε οὑρμο[θ]έμ[ιοϲ] τἠφροδί[τῃ]. — Böotische Inschrift ed. Kretschmer Hermes XXVI 123~ff. Τιμαϲίφιλοϲ μ’ ἀνέθηκε τὠπόλλωνι τοῖ Πτωιεῖι ὁ Πραόλλειοϲ.

Auch in Versen: CIA. 1, 398 Διογέν[ηϲ] ἀνέθηκεν Αἰϲχύλ(λ)ου ὑὺϲ Κεφ[α]λῆοϲ. IGA. 95 Πραξιτέληϲ ἀνέθηκε Συρακόϲιοϲ τόδ᾽ ἄγαλμα. Inschrift von Naukratis II No.~876 Ἑρμαγόρηϲ μ᾽ ἀνέθηκε ὁ Τ[ήιοϲ] τὠπόλλωνι. Pausanias 6, 10, 7 (5. Jahrhundert) Κλεοϲθένηϲ μ’ ἀνέθηκεν ὁ Πόντιοϲ ἐξ Ἐπιδάμνου. Epigramm von Erythrae Kaibel No.~769 (4. Jahrhundert) [—]-θέρϲηϲ ἀνέθηκεν Ἀθηναίῃ πολιούχῳ παῖϲ Ζωΐλου. Von Kalymna Kaibel No.~778 (id.?) Νικίαϲ με ἀνέθηκε Ἀπόλλωνι υἱὸϲ Θραϲυμήδεοϲ. Vgl. auch CIA. 1, 403 [τόνδε Πυρῆϲ] ἀνέθηκε Πολυμνήϲτου φίλο[ϲ υἱόϲ]. IGA. 98 (Arkadisch) Τέλλων τόνδ᾽ ἀνέθηκε Δαήμονοϲ ἀγλαὸϲ υἱόϲ.

Mit lesbischem \spation{κάθθηκε}: Inschriften von Naukratis II No. 788 [ὁ δεῖνα κάθ]θη\-κε τᾷ Ἀφροδίτᾳ ὀ Μυτιλήναιοϲ. 789 und 790 [ὁ δεῖνά με] κάθθηκε ὀ Μυ\-τ[ι\-λή\-ναι\-οϲ]. Vgl. 807 [Ἀφροδί]τᾳ ὁ Μ—. 814 [Ἀφροδ]ίτᾳ ὁ Κε—.

Mit \spation{ἐποίηϲε}, \spation{ἐποίει}: CΙΑ. 1, 335 Πύρροϲ ἐποίηϲεν Ἀθηναῖοϲ. 1, 362 (vgl. Studnitzka Jahrbuch II [1887], S.~144) [Ε]ὐφρόνιοϲ [ἐποίηϲεν ὁ] κεραμεύϲ (die Ergänzung wohl sicher!). 1, 483 Καλλωνίδηϲ ἐποίει ὁ Δεινίου, 4, 477\textsuperscript{b} [ὁ δεῖνα ἐποίηϲεν oder ἐποίει Π]άριοϲ. 4\textsuperscript{2}, 373, 81 Κάλων ἐποίηϲεν Αἱ[γινήτηϲ]. 4\textsuperscript{2}, 373, 95 [Ἄ]ρχερμοϲ ἐποίηϲεν ὁ Χῖ[οϲ]. 4\textsuperscript{2}, 373, 220 Λεώβιοϲ ἐποίηϲεν Πυρετιάδηϲ (oder Πυρρητιάδηϲ). IGA. 42 (Argos) Ἄτωτοϲ ἐποίϝηἑ Ἀργεῖοϲ κἈργειάδαϲ Ἁγελᾴδα τἈργείου. 44 (id.) Πολύκλειτοϲ ἐποίει Ἀργεῖοϲ. 44\textsuperscript{a} (id.) — [ἐ]πο[ί]ϝηἑ Ἀργεῖοϲ. 47 (id.) Κρηϲίλαϲ ἐποίηϲε Κυδωνιάτ[αϲ]. 165 Ὑπατόδωροϲ Ἀριϲϲτο[γείτων] ἐποηϲάταν Θηβαίω. 348 Παιώνιοϲ ἐποίηϲε Μενδαῖοϲ. 498 Μίκων ἐποίηϲεν Ἀθη\-ναί\-οϲ. Loewy Inschriften \hypertarget{p432}{\emph{[S.~432]}}\label{p432} griechischer Bildhauer No. 44\textsuperscript{a} -ων ἐπόηϲε Θηβαῖοϲ. 57 Ξ[ε]νο[— ἐποίη]ϲεν Ἐλευ[θερέυϲ?] No. 58. -ου [ἐ]πόηϲεν [Σικ]ε\-λι\-ώ\-τηϲ. 96 Κλέων ἐπόηϲε Σικυώνιοϲ. 103 [Δαίδαλοϲ ἐπ]οίηϲε Πατροκλέ[ουϲ]. 135\textsuperscript{d} (S.~388) [Σπ]ουδίαϲ ἐποίηϲε Ἀθηναῖοϲ. 277 Τιμόδαμοϲ Τ[ιμοδάμου ἐ]ποίηϲε Ἀμπρα[κιώτηϲ]. 297 (Apotheose Homers) Ἀρχέλαοϲ Ἀπολλωνίου ἐποίηϲε Πρι\-η\-νεύϲ. 404 Νίκανδροϲ ἐ[ποίηϲεν] Ἄνδ[ριοϲ]. Klein Griechische Vasen mit Meistersignaturen S.~72 Εὔχειροϲ ἐποίηϲεν οὑργοτίμου υἱὕϲ (zweimal). S.~73 Ἐργοτέληϲ ἐποίηϲεν ὁ Νεάρχου. S.~202 Ξενόφαντοϲ ἐποίηϲεν Ἀθην[αῖοϲ]. S.~202, 1 und 2 Τειϲίαϲ ἐποίηϲεν Ἁθηναῖοϲ. S.~213 Κρίτων ἐποίηϲεν Λε(ι)ποῦϲ ὕϲ d. i. υἱύϲ, nach der Lesung von Studnitzka Jahrbuch II 1887 S.~144. Pausanias 6, 9, 1 τὸν δὲ ἀνδριάντα οἱ Πτολίχοϲ ἐποίηϲεν Αἰγινήτηϲ, was auf eine Originalinschrift Πτόλιχοϲ ἐποίηϲεν Αἰγινήτηϲ schliessen lässt (vgl. Boeckh zu CIG. 25).

Auch in Versen: CIA. 4\textsuperscript{2}, 373, 105 Θηβάδηϲ ἐ[πόηϲε —]-νου παῖϲ τόδ᾽ ἄγαλμα. Inschrift von der Akropolis ed. Studnitzka Jahrbuch II 1887 S.~135~ff. Ἀντήνωρ ἐπ[όηϲεν ῾]o Εὐμάρουϲ τ[όδ᾽ ἄγαλμα] IGA. 410 Ἀλξήνωρ ἐποίηϲεν ὁ Νάξιοϲ, ἀλλ᾽ ἐϲίδεϲθε. Auch 349 Εὔφρων ἐξεποίηϲ᾽ οὐκ ἀδαὴϲ Πάριοϲ.

Mit \spation{ἔγραφεν}, \spation{ἔγραψεν}, \spation{γράφει} IGA. 482\textsuperscript{c} Τήλεφοϲ μ᾽ ἔγραφε ὁ Ἰαλύϲιοϲ. Klein Griechische Vasen mit Meistersignaturen. S.~29 Τιμωνίδα[ϲ μ᾽] ἔγραψε Βία. S.~196, 7 Εὐθυμίδηϲ ἔγραψεν ὁ Πολ(λ)ίου (zweimal). Ebenso ist 194, 2 (nach der Abbildung in Gerhards Vasenbildern 188) und ebenso 195 zu lesen, beides nach Dümmler. Kyprische Inschrift No. 147\textsuperscript{h} bei Meister Griechische Dialekte II 148 -οικόϲ με γράφει Σελαμίνιοϲ.

Mit verschiedenen Synonymis obiger Verba: IGA. 48 (Argos) [Δ]ωρόθεοϲ ἐ\-ϝ[ε]ρ\-γά\-ϲα\-το Ἀργεῖοϲ. 555\textsuperscript{a} (Opus?) Πρίκων ἔ[π]α[ξα Κο]λώτα. Kyprische Inschrift No. 73 Deecke Γιλίκα ἁμὲ \spation{κατέϲταϲε} ὁ Σταϲικρέτεοϲ.

Mit \spation{εἰμί}: IGA. 387 (Samos) [Π]όμπιόϲ εἰμι τοῦ Δημοκρίνεοϲ. 492 (Sigeum) ionischer Text: Φανοδίκου εἰμὶ τοὐρμοκράτεοϲ τοῦ Προκοννηϲίου; attischer Text: Φ. εἰμὶ τοῦ Ἑρμοκράτουϲ τοῦ Π. 522 (Sizilien) Λονγηναῖόϲ εἰμι δημόϲιοϲ. 528 (Cumae) Δημοχάριδόϲ εἰμι τοῦ —. 551 (Antipolis) Τέρπων εἰμὶ θεᾶϲ θεράπων ϲεμνῆϲ Ἀφροδίτηϲ. Rhodische Inschrift bei Kirchhoff Studien zur Gesch. des griech. Alph.\textsuperscript{4} S.~49 Φιλτοῦϲ ἠμι τᾶϲ καλᾶϲ ἁ κύλιξ ἁ ποικίλα. Kyprische Inschr. 1 Deecke Πρα-\hypertarget{p433}{\emph{[S.~433]}}\label{p433}τοτίμω ἠμὶ τᾶϲ Παφίαϲ τῶ ἰερῆϝοϲ. 16 D. tᾶϲ θεῶ ἠμι τᾶϲ Παφίαϲ (ebenso 65. 66 Hoffm.). 23 D. Τιμοκύπραϲ ἠμὶ Τιμοδάμω. 78 Η. Σταϲαγόρου ἠμὶ τῶ Σταϲάνδρω. 79 Η. Τιμάνδρω ἠμὶ τῶ Ὀναϲαγόρου. 88 Η. Πνυτίλλαϲ ἠμὶ τᾶϲ Πνυταγόραυ παιδόϲ. 121 Η. Διϝειθέμιτόϲ ἠμι τῶ βαϲιλῆϝοϲ.

Daran schliesst sich IGA. 543 τᾶϲ Ἥραϲ ἱαρόϲ εἰμι τᾶϲ ἐν πεδίῳ, wo ein Adjektiv verbunden mit εἶναι die Stelle des Verbums vertritt, und daran wieder die Beispiele, wo ein Adjektiv ohne εἶναι das Prädikat bildet: Klein Die griechischen Vasen mit Lieblingsinschriften S.~44 Λέαγροϲ καλὸϲ ὁ παίϲ. S.~68 Παντοξένα καλὰ Κοριν(θ)ί[α], wie das von Klein gegebene aber nicht erklärte ΚΟΡΙΝΟΙ wohl zu lesen ist. S.~81 Γλαύκων καλὸϲ Λεάγρου. S.~82 Δρόμιπποϲ καλὸϲ Δρομοκλείδου, Δίφιλοϲ καλὸϲ Μελανώπου. S.~83 Λίχαϲ καλὸϲ Σάμιοϲ, Ἀλκιμ[ή]δηϲ καλὸϲ Αἰϲχυλίδου. S.~85 Ἀλκίμαχοϲ καλὸϲ Ἐπιχάρουϲ.

Ausserhalb der bisher aufgeführten Kategorien liegen CIA. 4\textsuperscript{2}, 337\textsuperscript{a} Κλειϲθένηϲ ἐχορήγει Αὐτοκράτουϲ. IGA. 110, 9 (Elis) ἐν τἠπιάροι κ᾽ ἐνέχοιτο τοῖ ᾽νταῦτ᾽ ἐγρα(μ)μένοι. CIG. 7806 Ἀκαμαντὶϲ ἐνίκα φυλή.

Unter den aufgeführten Beispielen von ἀνέθηκε und κάθθηκε enthalten dreizehn ausser Subjekt, Verbum und Apposition auch noch einen Dativ, drei (CIA. 4\textsuperscript{1}, 373 f. IGA. 95. 543) einen substantivischen Akkusativ, 4\textsuperscript{2}, 373, 90 beides. Während nun der blosse Akkusativ überall auf die Apposition folgt (vgl. auch CIA. 4\textsuperscript{2}, 373, 105 Θηβάδηϲ ἐ[πόηϲε —]νου παῖϲ τόδ᾽ ἄγαλμα, sowie die Inschrift des Antenor), findet sich der Dativ nur viermal (IGA. 486. Naukratis II 780. 819. 876) hinter der Apposition, achtmal (Naukratis I 218. II 767. 788. 807. 814. Hermes 26, 123. Kaibel 769. 778) davor; endlich in IGA. 48 folgt auf das Verbum zunächst der Genetiv des Vaternamens, dann der Dativ des Götternamens samt Epitheton und dann erst das zum Subjekt gehörige nominativische Ethnikon. In CIA. 4\textsuperscript{2}, 373, 90 sind Akkusativ und Dativ zusammen zwischen Verbum und Apposition eingeschoben. — Diese Voranstellung der zum Verb gehörigen Kasus vor die Apposition ist leicht verständlich; das Verb attrahiert seine Bestimmungen.

Aus diesem Typus erklärt sich die seltsame Wortfolge in CIA. 4\textsuperscript{2}, 373, 82, ergänzt von Studnitzka Jahrbuch II 1887 S. 143: Κρίτων Ἀθηναίᾳ ὁ Σκύθου ἀν[έθηκε καὶ ἐ]ποίη[ϲε] oder [ἐ]ποίει. Der Verfasser der  Inschrift hatte zunächst die kon-\hypertarget{p434}{\emph{[S.~434]}}\label{p434}ventionelle Wortfolge Κρίτων ἀνέθηκεν Ἀθηναίᾳ ὁ Σκύθου vor Augen und liess hiernach, als er durch die Beifügung von καὶ ἐποίηϲε genötigt war, ἀνέθηκε hinter die Apposition zu rücken, doch den Dativ Ἀθηναίᾳ vor der Apposition stehen.

Loewy Inschriften griechischer Bildhauer S. XV glaubt erweisen zu können, dass diese Wortstellung über die ersten Jahrzehnte des vierten Jahrhunderts hinaus nicht üblich gewesen sei (vgl. auch CIA. 2, 1621—1648 und die von Köhler zu No. 1621 verzeichneten Künstlerinschriften). Die paar spätern Beispiele darf man füglich als Archaismen betrachten, zumal zwei derselben (Loewy 277. 297, s. oben S.~431) durch Voranstellung des Genetivs des Vaternamens vor das Verbum von der ursprünglichen Weise abgehen. Ausnahmslose Herrschaft dieser Stellungsgewohnheit kann man auch für frühere Zeit nicht behaupten (Hoffmann Griech. Dialekte I 324), und namentlich weisen die attischen Weihinschriften zahlreiche Gegenbeispiele auf. Aber sehr mächtig und zu gewissen Zeiten und in gewissen Gegenden entschieden vorherrschend war diese Gewohnheit doch, um so berechtigter ist Schulze’s Auffassung derselben als eines indogermanischen Erbteils.

Das Altindische liefert augenfällige Parallelen. (Delbrück Syntaktische Forschungen III 51~ff. V~23~f.). Häufig sind in der Brahmanasprache Sätze, die mit \emph{sa} oder \emph{sa ha} “dieser eben” beginnen, darauf gleich das Verbum, meist \emph{uvāca}, folgen lassen, und dann erst die nähere Bezeichnung der vorher mittelst des Pronomens angekündigten Person beifügen z.~B. \emph{sá hovāca gā́rgyaḥ, sá āikṣ̌ata prajāpatiḥ}. Ähnlich Cat. Br. 3, 1, 3, 4 \emph{tá u hāitá ūcur devā́ ādityā́ḥ}. Manchmal ist auch das Subjekt stärker belastet; manchmal, unter dem Einfluss der Gewohnheit den Satz mit dem Verbum zu schliessen, die Apposition zwar vom Pronomen getrennt, aber doch dem Verbum vorangeschickt.

Weiterhin findet sich nun auch in denselben indischen Texten auffälliges Setzen des Verbums an zweite Stelle, wenn der Satz mit \emph{íti ha, tád u ha, tád u sma, ápi ha} beginnt. Es handelt sich dabei meist um die Verba \emph{uvāca, āha}; der Name des Sprechers folgt dann erst nach dem Verbum. Also ganz die Weise deutscher Sätze mit Inversion.

\hfill \spation{Jacob Wackernagel.} \hspace{1ex}

\hypertarget{p435}{\emph{[S.~435]}}\label{p435}

\section*{Nachträge}
\addcontentsline{toc}{section}{Nachträge}

zu Abschnitt II S. 346—351 (betr. die Inschriften mit με, ἐμέ).

Zu S. 346, 351: IGA. 351 (lokrisch) [Π]εριφόνᾳ [ἀνέθη]κέ με (oder -κ ἐμέ?) Ξενάγατοϲ muss wegen des Zustandes der Inschrift ausser Betracht fallen; vgl. Röhl z. d. St.

Zu S. 349: CIA. 4\textsuperscript{2}, 373, 103 Οὑνπορίωνοϲ Φίλων με ἐποίηϲεν. — Inschrift von Metapont Collitz 1643 Νικόμαχόϲ μ᾽ ἐπόει. — Vaseninschrift Klein S. 65 No. 48 nach Six Gazette archéol. 1888, 193 Νικοϲθένηϲ εμ (Six: μ᾽ ἐ-)ποίηϲεν.

Zu S. 351: ἐμέ noch zweimal an zweiter Stelle in der alten Vaseninschrift bei Pottier Gazette archéol. 1888, 168: ἐκεράμευϲεν ἐμεὶ Οἰκωφέληϲ und Οἰκωφ(έ)ληϲ ἔμ᾽ ἔγραψεν (geschrieben εγραεφϲεν). Vgl. auch ibid. 1888, 180: -πόλον ἐμέ.

\section*{Verzeichnis der kritisch behandelten Stellen}
\addcontentsline{toc}{section}{Verzeichnis der kritisch behandelten Stellen}

Homer Ε 273 = Θ 196 \dotfill S. \hyperlink{p373}{373}\\
\phantom{a}~„~ Π 112 \dotfill „ \hyperlink{p343}{343}\\
\phantom{a}~„~ γ 319 \dotfill „ \hyperlink{p373}{373}\\
Alkman Fragm. 52 Bgk. \dotfill „ \hyperlink{p361}{361}\\
Alcaeus Fragm. 68 Bgk. \dotfill „ \hyperlink{p345}{345}\\
\phantom{a}~„~ Fragm. 83 Bgk. \dotfill „ \hyperlink{p375}{375}\\
Sappho Fragm. 2, 7 Bgk. \dotfill „ \hyperlink{p345}{345}\\
\phantom{a}~„~ ~„~ 43 Bgk. \dotfill „ \hyperlink{p345}{345}\\
\phantom{a}~„~ ~„~ 66 Bgk. \dotfill „ \hyperlink{p375}{375}\\
\phantom{a}~„~ ~„~ 97, 4 Hiller (=100 Bgk.) \dotfill „ \hyperlink{p345}{345}\\
Pindar Olymp. 1, 48 \dotfill „ \hyperlink{p361}{361}\\
Euripides Medea 1339 \dotfill „ \hyperlink{p388}{388}\\
\phantom{a}~„~ Fragm. 1029, 4 \dotfill „ \hyperlink{p379}{379}\\
Antiphon 5,38 \dotfill „ \hyperlink{p379}{379}\\
Aristophanes Acharn. 779 \dotfill „ \hyperlink{p361}{361}\\
\phantom{a}~„~ Ranae 259 \dotfill „ \hyperlink{p379}{379}\\
\phantom{a}~„~ Eccles. 916 \dotfill „ \hyperlink{p382}{382}\\
Demosthenes 18, 43 \dotfill „ \hyperlink{p388}{388}\\
\phantom{a}~„~ 18, 206 \dotfill „ \hyperlink{p387}{387}\\
\phantom{a}~„~ 24, 64 \dotfill „ \hyperlink{p388}{388}\\
\phantom{a}~„~ prooem. 1, 3 \dotfill S. \hyperlink{p390}{390} f.\\
\phantom{a}~„~ ~„~ 3 \dotfill S. \hyperlink{p399}{399}\\
Callimachus Fragm. 114 \dotfill „ \hyperlink{p361}{361}\\
Theokrit 2, 159 \dotfill „ \hyperlink{p372}{372}\\
Pausanias 5, 23, 7 \dotfill „ \hyperlink{p350}{350}\\
Anthol. Palat. 6, 140 \dotfill „ \hyperlink{p351}{351}\\
Inscriptiones graecae antiquissimae ed. Röhl 384 \dotfill „ \hyperlink{p347}{347}\\
\phantom{a}~„~ ~„~ ~„~ ~„~ ~„~ 474 \dotfill „ \hyperlink{p349}{349}\\
Sammlung der griech. Dialektinschr. v. Collitz 26 \dotfill „ \hyperlink{p365}{365}\\
\phantom{a}~„~ ~„~ ~„~ ~„~ ~„~ ~„~ 3184, 8 \dotfill „ \hyperlink{p374}{374}\\
\phantom{a}~„~ ~„~ ~„~ ~„~ ~„~ ~„~ 3213, 3 \dotfill S. \hyperlink{p369}{369} f.\\
\hypertarget{p436}{\emph{[S.~436]}}\label{p436} Die griech. Vasen mit Meistersignaturen v. W. Klein S. 51 \dotfill S. \hyperlink{p349}{349}\\
\phantom{a}~„~ ~„~ ~„~ ~„~ ~„~ ~„~ ~„~ ~„~ S. 194, 2 \dotfill „ \hyperlink{p432}{432}\\
\phantom{a}~„~ ~„~ ~„~ ~„~ ~„~ ~„~ ~„~ ~„~ S. 195, 3 \dotfill „ \hyperlink{p432}{432}\\
\phantom{a}~„~ ~„~ ~„~ ~„~ Lieblingsinschr. ~„~ ~„~ ~„~ S. 68 \dotfill „ \hyperlink{p433}{433}\\
Naukratis. By Flinders Petrie I Inschrift No. 303 \dotfill „ \hyperlink{p348}{348}\\
\phantom{Naukratis. By Flinders Petrie} I ~„~ ~„~ 307 \dotfill „ \hyperlink{p348}{348}\\
\phantom{Naukratis. By Flinders Petrie} II ~„~ ~„~ 750 \dotfill „ \hyperlink{p348}{348}\\
Plautus Bacchides 1258 \dotfill „ \hyperlink{p410}{410}\\
\phantom{a}~„~ Mercator 784 \dotfill „ \hyperlink{p414}{414}

\hfill J. W. \hspace{1ex}







\end{otherlanguage*}